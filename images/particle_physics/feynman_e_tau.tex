\documentclass{article}

\begin{document}

\feynmandiagram[medium, vertical=a to b] {
i1 [particle = \( \textup{e}^{-} \)]
-- [fermion, momentum' = \( p_{1} \)] a -- [fermion, momentum' = \( p_{3} \)]
i2 [particle=\( \textup{e}^{-} \)],
a -- [boson, edge label' = \( \textup{\gamma} \), momentum = \( q \)] b,
f1 [particle = \( \textup{\tau}^{-} \)]
-- [anti fermion, momentum' = \( p_{2} \)] b -- [anti fermion, momentum' = \( p_{4} \)]
f2 [particle = \(\textup{\tau}^{-} \)],
};

\end{document}
