%!TEX root = ../main.tex
\documentclass[../main.tex]{subfiles}

\begin{document}

And now, with the board set and the pieces moved, we come to it at last: the results.

The following results were trained on the full muon neutrino charged current dataset, and tested on the corresponding test set using the aforementioned TCN 2.0 network.
The network performs very well on low energy events, up to around \SI{50}{\giga\electronvolt}, whereafter it performs worse than Retro Reco, in the zenith case by a significant amount.

\begin{figure}
    \centering
    %% Creator: Matplotlib, PGF backend
%%
%% To include the figure in your LaTeX document, write
%%   \input{<filename>.pgf}
%%
%% Make sure the required packages are loaded in your preamble
%%   \usepackage{pgf}
%%
%% and, on pdftex
%%   \usepackage[utf8]{inputenc}\DeclareUnicodeCharacter{2212}{-}
%%
%% or, on luatex and xetex
%%   \usepackage{unicode-math}
%%
%% Figures using additional raster images can only be included by \input if
%% they are in the same directory as the main LaTeX file. For loading figures
%% from other directories you can use the `import` package
%%   \usepackage{import}
%%
%% and then include the figures with
%%   \import{<path to file>}{<filename>.pgf}
%%
%% Matplotlib used the following preamble
%%   \usepackage{siunitx} \usepackage{amsmath} \usepackage{bm}
%%   \usepackage{fontspec}
%%
\begingroup%
\makeatletter%
\begin{pgfpicture}%
\pgfpathrectangle{\pgfpointorigin}{\pgfqpoint{6.201200in}{6.500000in}}%
\pgfusepath{use as bounding box, clip}%
\begin{pgfscope}%
\pgfsetbuttcap%
\pgfsetmiterjoin%
\definecolor{currentfill}{rgb}{1.000000,1.000000,1.000000}%
\pgfsetfillcolor{currentfill}%
\pgfsetlinewidth{0.000000pt}%
\definecolor{currentstroke}{rgb}{1.000000,1.000000,1.000000}%
\pgfsetstrokecolor{currentstroke}%
\pgfsetdash{}{0pt}%
\pgfpathmoveto{\pgfqpoint{0.000000in}{0.000000in}}%
\pgfpathlineto{\pgfqpoint{6.201200in}{0.000000in}}%
\pgfpathlineto{\pgfqpoint{6.201200in}{6.500000in}}%
\pgfpathlineto{\pgfqpoint{0.000000in}{6.500000in}}%
\pgfpathclose%
\pgfusepath{fill}%
\end{pgfscope}%
\begin{pgfscope}%
\pgfsetbuttcap%
\pgfsetmiterjoin%
\definecolor{currentfill}{rgb}{1.000000,1.000000,1.000000}%
\pgfsetfillcolor{currentfill}%
\pgfsetlinewidth{0.000000pt}%
\definecolor{currentstroke}{rgb}{0.000000,0.000000,0.000000}%
\pgfsetstrokecolor{currentstroke}%
\pgfsetstrokeopacity{0.000000}%
\pgfsetdash{}{0pt}%
\pgfpathmoveto{\pgfqpoint{0.485140in}{4.444426in}}%
\pgfpathlineto{\pgfqpoint{3.025600in}{4.444426in}}%
\pgfpathlineto{\pgfqpoint{3.025600in}{5.862667in}}%
\pgfpathlineto{\pgfqpoint{0.485140in}{5.862667in}}%
\pgfpathclose%
\pgfusepath{fill}%
\end{pgfscope}%
\begin{pgfscope}%
\pgfsetbuttcap%
\pgfsetroundjoin%
\definecolor{currentfill}{rgb}{0.000000,0.000000,0.000000}%
\pgfsetfillcolor{currentfill}%
\pgfsetlinewidth{0.803000pt}%
\definecolor{currentstroke}{rgb}{0.000000,0.000000,0.000000}%
\pgfsetstrokecolor{currentstroke}%
\pgfsetdash{}{0pt}%
\pgfsys@defobject{currentmarker}{\pgfqpoint{0.000000in}{-0.048611in}}{\pgfqpoint{0.000000in}{0.000000in}}{%
\pgfpathmoveto{\pgfqpoint{0.000000in}{0.000000in}}%
\pgfpathlineto{\pgfqpoint{0.000000in}{-0.048611in}}%
\pgfusepath{stroke,fill}%
}%
\begin{pgfscope}%
\pgfsys@transformshift{0.916345in}{4.444426in}%
\pgfsys@useobject{currentmarker}{}%
\end{pgfscope}%
\end{pgfscope}%
\begin{pgfscope}%
\definecolor{textcolor}{rgb}{0.000000,0.000000,0.000000}%
\pgfsetstrokecolor{textcolor}%
\pgfsetfillcolor{textcolor}%
\pgftext[x=0.916345in,y=4.347204in,,top]{\color{textcolor}\rmfamily\fontsize{8.000000}{9.600000}\selectfont \(\displaystyle {-1}\)}%
\end{pgfscope}%
\begin{pgfscope}%
\pgfsetbuttcap%
\pgfsetroundjoin%
\definecolor{currentfill}{rgb}{0.000000,0.000000,0.000000}%
\pgfsetfillcolor{currentfill}%
\pgfsetlinewidth{0.803000pt}%
\definecolor{currentstroke}{rgb}{0.000000,0.000000,0.000000}%
\pgfsetstrokecolor{currentstroke}%
\pgfsetdash{}{0pt}%
\pgfsys@defobject{currentmarker}{\pgfqpoint{0.000000in}{-0.048611in}}{\pgfqpoint{0.000000in}{0.000000in}}{%
\pgfpathmoveto{\pgfqpoint{0.000000in}{0.000000in}}%
\pgfpathlineto{\pgfqpoint{0.000000in}{-0.048611in}}%
\pgfusepath{stroke,fill}%
}%
\begin{pgfscope}%
\pgfsys@transformshift{1.602799in}{4.444426in}%
\pgfsys@useobject{currentmarker}{}%
\end{pgfscope}%
\end{pgfscope}%
\begin{pgfscope}%
\definecolor{textcolor}{rgb}{0.000000,0.000000,0.000000}%
\pgfsetstrokecolor{textcolor}%
\pgfsetfillcolor{textcolor}%
\pgftext[x=1.602799in,y=4.347204in,,top]{\color{textcolor}\rmfamily\fontsize{8.000000}{9.600000}\selectfont \(\displaystyle {0}\)}%
\end{pgfscope}%
\begin{pgfscope}%
\pgfsetbuttcap%
\pgfsetroundjoin%
\definecolor{currentfill}{rgb}{0.000000,0.000000,0.000000}%
\pgfsetfillcolor{currentfill}%
\pgfsetlinewidth{0.803000pt}%
\definecolor{currentstroke}{rgb}{0.000000,0.000000,0.000000}%
\pgfsetstrokecolor{currentstroke}%
\pgfsetdash{}{0pt}%
\pgfsys@defobject{currentmarker}{\pgfqpoint{0.000000in}{-0.048611in}}{\pgfqpoint{0.000000in}{0.000000in}}{%
\pgfpathmoveto{\pgfqpoint{0.000000in}{0.000000in}}%
\pgfpathlineto{\pgfqpoint{0.000000in}{-0.048611in}}%
\pgfusepath{stroke,fill}%
}%
\begin{pgfscope}%
\pgfsys@transformshift{2.289253in}{4.444426in}%
\pgfsys@useobject{currentmarker}{}%
\end{pgfscope}%
\end{pgfscope}%
\begin{pgfscope}%
\definecolor{textcolor}{rgb}{0.000000,0.000000,0.000000}%
\pgfsetstrokecolor{textcolor}%
\pgfsetfillcolor{textcolor}%
\pgftext[x=2.289253in,y=4.347204in,,top]{\color{textcolor}\rmfamily\fontsize{8.000000}{9.600000}\selectfont \(\displaystyle {1}\)}%
\end{pgfscope}%
\begin{pgfscope}%
\pgfsetbuttcap%
\pgfsetroundjoin%
\definecolor{currentfill}{rgb}{0.000000,0.000000,0.000000}%
\pgfsetfillcolor{currentfill}%
\pgfsetlinewidth{0.803000pt}%
\definecolor{currentstroke}{rgb}{0.000000,0.000000,0.000000}%
\pgfsetstrokecolor{currentstroke}%
\pgfsetdash{}{0pt}%
\pgfsys@defobject{currentmarker}{\pgfqpoint{0.000000in}{-0.048611in}}{\pgfqpoint{0.000000in}{0.000000in}}{%
\pgfpathmoveto{\pgfqpoint{0.000000in}{0.000000in}}%
\pgfpathlineto{\pgfqpoint{0.000000in}{-0.048611in}}%
\pgfusepath{stroke,fill}%
}%
\begin{pgfscope}%
\pgfsys@transformshift{2.975706in}{4.444426in}%
\pgfsys@useobject{currentmarker}{}%
\end{pgfscope}%
\end{pgfscope}%
\begin{pgfscope}%
\definecolor{textcolor}{rgb}{0.000000,0.000000,0.000000}%
\pgfsetstrokecolor{textcolor}%
\pgfsetfillcolor{textcolor}%
\pgftext[x=2.975706in,y=4.347204in,,top]{\color{textcolor}\rmfamily\fontsize{8.000000}{9.600000}\selectfont \(\displaystyle {2}\)}%
\end{pgfscope}%
\begin{pgfscope}%
\pgfsetbuttcap%
\pgfsetroundjoin%
\definecolor{currentfill}{rgb}{0.000000,0.000000,0.000000}%
\pgfsetfillcolor{currentfill}%
\pgfsetlinewidth{0.803000pt}%
\definecolor{currentstroke}{rgb}{0.000000,0.000000,0.000000}%
\pgfsetstrokecolor{currentstroke}%
\pgfsetdash{}{0pt}%
\pgfsys@defobject{currentmarker}{\pgfqpoint{-0.048611in}{0.000000in}}{\pgfqpoint{-0.000000in}{0.000000in}}{%
\pgfpathmoveto{\pgfqpoint{-0.000000in}{0.000000in}}%
\pgfpathlineto{\pgfqpoint{-0.048611in}{0.000000in}}%
\pgfusepath{stroke,fill}%
}%
\begin{pgfscope}%
\pgfsys@transformshift{0.485140in}{4.444426in}%
\pgfsys@useobject{currentmarker}{}%
\end{pgfscope}%
\end{pgfscope}%
\begin{pgfscope}%
\definecolor{textcolor}{rgb}{0.000000,0.000000,0.000000}%
\pgfsetstrokecolor{textcolor}%
\pgfsetfillcolor{textcolor}%
\pgftext[x=0.328889in, y=4.405870in, left, base]{\color{textcolor}\rmfamily\fontsize{8.000000}{9.600000}\selectfont \(\displaystyle {0}\)}%
\end{pgfscope}%
\begin{pgfscope}%
\pgfsetbuttcap%
\pgfsetroundjoin%
\definecolor{currentfill}{rgb}{0.000000,0.000000,0.000000}%
\pgfsetfillcolor{currentfill}%
\pgfsetlinewidth{0.803000pt}%
\definecolor{currentstroke}{rgb}{0.000000,0.000000,0.000000}%
\pgfsetstrokecolor{currentstroke}%
\pgfsetdash{}{0pt}%
\pgfsys@defobject{currentmarker}{\pgfqpoint{-0.048611in}{0.000000in}}{\pgfqpoint{-0.000000in}{0.000000in}}{%
\pgfpathmoveto{\pgfqpoint{-0.000000in}{0.000000in}}%
\pgfpathlineto{\pgfqpoint{-0.048611in}{0.000000in}}%
\pgfusepath{stroke,fill}%
}%
\begin{pgfscope}%
\pgfsys@transformshift{0.485140in}{4.859229in}%
\pgfsys@useobject{currentmarker}{}%
\end{pgfscope}%
\end{pgfscope}%
\begin{pgfscope}%
\definecolor{textcolor}{rgb}{0.000000,0.000000,0.000000}%
\pgfsetstrokecolor{textcolor}%
\pgfsetfillcolor{textcolor}%
\pgftext[x=0.328889in, y=4.820673in, left, base]{\color{textcolor}\rmfamily\fontsize{8.000000}{9.600000}\selectfont \(\displaystyle {1}\)}%
\end{pgfscope}%
\begin{pgfscope}%
\pgfsetbuttcap%
\pgfsetroundjoin%
\definecolor{currentfill}{rgb}{0.000000,0.000000,0.000000}%
\pgfsetfillcolor{currentfill}%
\pgfsetlinewidth{0.803000pt}%
\definecolor{currentstroke}{rgb}{0.000000,0.000000,0.000000}%
\pgfsetstrokecolor{currentstroke}%
\pgfsetdash{}{0pt}%
\pgfsys@defobject{currentmarker}{\pgfqpoint{-0.048611in}{0.000000in}}{\pgfqpoint{-0.000000in}{0.000000in}}{%
\pgfpathmoveto{\pgfqpoint{-0.000000in}{0.000000in}}%
\pgfpathlineto{\pgfqpoint{-0.048611in}{0.000000in}}%
\pgfusepath{stroke,fill}%
}%
\begin{pgfscope}%
\pgfsys@transformshift{0.485140in}{5.274032in}%
\pgfsys@useobject{currentmarker}{}%
\end{pgfscope}%
\end{pgfscope}%
\begin{pgfscope}%
\definecolor{textcolor}{rgb}{0.000000,0.000000,0.000000}%
\pgfsetstrokecolor{textcolor}%
\pgfsetfillcolor{textcolor}%
\pgftext[x=0.328889in, y=5.235476in, left, base]{\color{textcolor}\rmfamily\fontsize{8.000000}{9.600000}\selectfont \(\displaystyle {2}\)}%
\end{pgfscope}%
\begin{pgfscope}%
\pgfsetbuttcap%
\pgfsetroundjoin%
\definecolor{currentfill}{rgb}{0.000000,0.000000,0.000000}%
\pgfsetfillcolor{currentfill}%
\pgfsetlinewidth{0.803000pt}%
\definecolor{currentstroke}{rgb}{0.000000,0.000000,0.000000}%
\pgfsetstrokecolor{currentstroke}%
\pgfsetdash{}{0pt}%
\pgfsys@defobject{currentmarker}{\pgfqpoint{-0.048611in}{0.000000in}}{\pgfqpoint{-0.000000in}{0.000000in}}{%
\pgfpathmoveto{\pgfqpoint{-0.000000in}{0.000000in}}%
\pgfpathlineto{\pgfqpoint{-0.048611in}{0.000000in}}%
\pgfusepath{stroke,fill}%
}%
\begin{pgfscope}%
\pgfsys@transformshift{0.485140in}{5.688835in}%
\pgfsys@useobject{currentmarker}{}%
\end{pgfscope}%
\end{pgfscope}%
\begin{pgfscope}%
\definecolor{textcolor}{rgb}{0.000000,0.000000,0.000000}%
\pgfsetstrokecolor{textcolor}%
\pgfsetfillcolor{textcolor}%
\pgftext[x=0.328889in, y=5.650279in, left, base]{\color{textcolor}\rmfamily\fontsize{8.000000}{9.600000}\selectfont \(\displaystyle {3}\)}%
\end{pgfscope}%
\begin{pgfscope}%
\definecolor{textcolor}{rgb}{0.000000,0.000000,0.000000}%
\pgfsetstrokecolor{textcolor}%
\pgfsetfillcolor{textcolor}%
\pgftext[x=0.273333in,y=5.153546in,,bottom,rotate=90.000000]{\color{textcolor}\rmfamily\fontsize{10.000000}{12.000000}\selectfont Density}%
\end{pgfscope}%
\begin{pgfscope}%
\pgfpathrectangle{\pgfqpoint{0.485140in}{4.444426in}}{\pgfqpoint{2.540460in}{1.418241in}}%
\pgfusepath{clip}%
\pgfsetbuttcap%
\pgfsetmiterjoin%
\pgfsetlinewidth{1.003750pt}%
\definecolor{currentstroke}{rgb}{0.313725,0.317647,0.309804}%
\pgfsetstrokecolor{currentstroke}%
\pgfsetdash{}{0pt}%
\pgfpathmoveto{\pgfqpoint{0.600615in}{4.444426in}}%
\pgfpathlineto{\pgfqpoint{0.600615in}{4.451572in}}%
\pgfpathlineto{\pgfqpoint{0.620902in}{4.451572in}}%
\pgfpathlineto{\pgfqpoint{0.620902in}{4.444426in}}%
\pgfpathlineto{\pgfqpoint{0.641189in}{4.444426in}}%
\pgfpathlineto{\pgfqpoint{0.641189in}{4.444426in}}%
\pgfpathlineto{\pgfqpoint{0.661475in}{4.444426in}}%
\pgfpathlineto{\pgfqpoint{0.661475in}{4.444426in}}%
\pgfpathlineto{\pgfqpoint{0.681762in}{4.444426in}}%
\pgfpathlineto{\pgfqpoint{0.681762in}{4.444426in}}%
\pgfpathlineto{\pgfqpoint{0.702049in}{4.444426in}}%
\pgfpathlineto{\pgfqpoint{0.702049in}{4.451572in}}%
\pgfpathlineto{\pgfqpoint{0.722336in}{4.451572in}}%
\pgfpathlineto{\pgfqpoint{0.722336in}{4.444426in}}%
\pgfpathlineto{\pgfqpoint{0.742623in}{4.444426in}}%
\pgfpathlineto{\pgfqpoint{0.742623in}{4.444426in}}%
\pgfpathlineto{\pgfqpoint{0.762909in}{4.444426in}}%
\pgfpathlineto{\pgfqpoint{0.762909in}{4.451572in}}%
\pgfpathlineto{\pgfqpoint{0.783196in}{4.451572in}}%
\pgfpathlineto{\pgfqpoint{0.783196in}{4.444426in}}%
\pgfpathlineto{\pgfqpoint{0.803483in}{4.444426in}}%
\pgfpathlineto{\pgfqpoint{0.803483in}{4.444426in}}%
\pgfpathlineto{\pgfqpoint{0.823770in}{4.444426in}}%
\pgfpathlineto{\pgfqpoint{0.823770in}{4.458719in}}%
\pgfpathlineto{\pgfqpoint{0.844056in}{4.458719in}}%
\pgfpathlineto{\pgfqpoint{0.844056in}{4.458719in}}%
\pgfpathlineto{\pgfqpoint{0.864343in}{4.458719in}}%
\pgfpathlineto{\pgfqpoint{0.864343in}{4.444426in}}%
\pgfpathlineto{\pgfqpoint{0.884630in}{4.444426in}}%
\pgfpathlineto{\pgfqpoint{0.884630in}{4.480159in}}%
\pgfpathlineto{\pgfqpoint{0.904917in}{4.480159in}}%
\pgfpathlineto{\pgfqpoint{0.904917in}{4.501599in}}%
\pgfpathlineto{\pgfqpoint{0.925203in}{4.501599in}}%
\pgfpathlineto{\pgfqpoint{0.925203in}{4.473012in}}%
\pgfpathlineto{\pgfqpoint{0.945490in}{4.473012in}}%
\pgfpathlineto{\pgfqpoint{0.945490in}{4.508745in}}%
\pgfpathlineto{\pgfqpoint{0.965777in}{4.508745in}}%
\pgfpathlineto{\pgfqpoint{0.965777in}{4.537332in}}%
\pgfpathlineto{\pgfqpoint{0.986064in}{4.537332in}}%
\pgfpathlineto{\pgfqpoint{0.986064in}{4.473012in}}%
\pgfpathlineto{\pgfqpoint{1.006351in}{4.473012in}}%
\pgfpathlineto{\pgfqpoint{1.006351in}{4.565918in}}%
\pgfpathlineto{\pgfqpoint{1.026637in}{4.565918in}}%
\pgfpathlineto{\pgfqpoint{1.026637in}{4.637384in}}%
\pgfpathlineto{\pgfqpoint{1.046924in}{4.637384in}}%
\pgfpathlineto{\pgfqpoint{1.046924in}{4.673117in}}%
\pgfpathlineto{\pgfqpoint{1.067211in}{4.673117in}}%
\pgfpathlineto{\pgfqpoint{1.067211in}{4.694556in}}%
\pgfpathlineto{\pgfqpoint{1.087498in}{4.694556in}}%
\pgfpathlineto{\pgfqpoint{1.087498in}{4.751729in}}%
\pgfpathlineto{\pgfqpoint{1.107784in}{4.751729in}}%
\pgfpathlineto{\pgfqpoint{1.107784in}{5.030446in}}%
\pgfpathlineto{\pgfqpoint{1.128071in}{5.030446in}}%
\pgfpathlineto{\pgfqpoint{1.128071in}{5.037593in}}%
\pgfpathlineto{\pgfqpoint{1.148358in}{5.037593in}}%
\pgfpathlineto{\pgfqpoint{1.148358in}{5.151938in}}%
\pgfpathlineto{\pgfqpoint{1.168645in}{5.151938in}}%
\pgfpathlineto{\pgfqpoint{1.168645in}{5.151938in}}%
\pgfpathlineto{\pgfqpoint{1.188931in}{5.151938in}}%
\pgfpathlineto{\pgfqpoint{1.188931in}{5.437802in}}%
\pgfpathlineto{\pgfqpoint{1.209218in}{5.437802in}}%
\pgfpathlineto{\pgfqpoint{1.209218in}{5.795131in}}%
\pgfpathlineto{\pgfqpoint{1.229505in}{5.795131in}}%
\pgfpathlineto{\pgfqpoint{1.229505in}{5.616467in}}%
\pgfpathlineto{\pgfqpoint{1.249792in}{5.616467in}}%
\pgfpathlineto{\pgfqpoint{1.249792in}{5.702226in}}%
\pgfpathlineto{\pgfqpoint{1.270079in}{5.702226in}}%
\pgfpathlineto{\pgfqpoint{1.270079in}{5.566441in}}%
\pgfpathlineto{\pgfqpoint{1.290365in}{5.566441in}}%
\pgfpathlineto{\pgfqpoint{1.290365in}{5.494975in}}%
\pgfpathlineto{\pgfqpoint{1.310652in}{5.494975in}}%
\pgfpathlineto{\pgfqpoint{1.310652in}{5.487828in}}%
\pgfpathlineto{\pgfqpoint{1.330939in}{5.487828in}}%
\pgfpathlineto{\pgfqpoint{1.330939in}{5.251991in}}%
\pgfpathlineto{\pgfqpoint{1.351226in}{5.251991in}}%
\pgfpathlineto{\pgfqpoint{1.351226in}{4.873221in}}%
\pgfpathlineto{\pgfqpoint{1.371512in}{4.873221in}}%
\pgfpathlineto{\pgfqpoint{1.371512in}{4.780316in}}%
\pgfpathlineto{\pgfqpoint{1.391799in}{4.780316in}}%
\pgfpathlineto{\pgfqpoint{1.391799in}{4.687410in}}%
\pgfpathlineto{\pgfqpoint{1.412086in}{4.687410in}}%
\pgfpathlineto{\pgfqpoint{1.412086in}{4.573064in}}%
\pgfpathlineto{\pgfqpoint{1.432373in}{4.573064in}}%
\pgfpathlineto{\pgfqpoint{1.432373in}{4.473012in}}%
\pgfpathlineto{\pgfqpoint{1.452659in}{4.473012in}}%
\pgfpathlineto{\pgfqpoint{1.452659in}{4.465866in}}%
\pgfpathlineto{\pgfqpoint{1.472946in}{4.465866in}}%
\pgfpathlineto{\pgfqpoint{1.472946in}{4.444426in}}%
\pgfusepath{stroke}%
\end{pgfscope}%
\begin{pgfscope}%
\pgfpathrectangle{\pgfqpoint{0.485140in}{4.444426in}}{\pgfqpoint{2.540460in}{1.418241in}}%
\pgfusepath{clip}%
\pgfsetbuttcap%
\pgfsetmiterjoin%
\pgfsetlinewidth{1.003750pt}%
\definecolor{currentstroke}{rgb}{0.949020,0.372549,0.360784}%
\pgfsetstrokecolor{currentstroke}%
\pgfsetdash{{1.000000pt}{1.650000pt}}{0.000000pt}%
\pgfpathmoveto{\pgfqpoint{0.600615in}{4.444426in}}%
\pgfpathlineto{\pgfqpoint{0.600615in}{4.467410in}}%
\pgfpathlineto{\pgfqpoint{0.620902in}{4.467410in}}%
\pgfpathlineto{\pgfqpoint{0.620902in}{4.467410in}}%
\pgfpathlineto{\pgfqpoint{0.641189in}{4.467410in}}%
\pgfpathlineto{\pgfqpoint{0.641189in}{4.459749in}}%
\pgfpathlineto{\pgfqpoint{0.661475in}{4.459749in}}%
\pgfpathlineto{\pgfqpoint{0.661475in}{4.482733in}}%
\pgfpathlineto{\pgfqpoint{0.681762in}{4.482733in}}%
\pgfpathlineto{\pgfqpoint{0.681762in}{4.459749in}}%
\pgfpathlineto{\pgfqpoint{0.702049in}{4.459749in}}%
\pgfpathlineto{\pgfqpoint{0.702049in}{4.528703in}}%
\pgfpathlineto{\pgfqpoint{0.722336in}{4.528703in}}%
\pgfpathlineto{\pgfqpoint{0.722336in}{4.551687in}}%
\pgfpathlineto{\pgfqpoint{0.742623in}{4.551687in}}%
\pgfpathlineto{\pgfqpoint{0.742623in}{4.475072in}}%
\pgfpathlineto{\pgfqpoint{0.762909in}{4.475072in}}%
\pgfpathlineto{\pgfqpoint{0.762909in}{4.475072in}}%
\pgfpathlineto{\pgfqpoint{0.783196in}{4.475072in}}%
\pgfpathlineto{\pgfqpoint{0.783196in}{4.490395in}}%
\pgfpathlineto{\pgfqpoint{0.803483in}{4.490395in}}%
\pgfpathlineto{\pgfqpoint{0.803483in}{4.513380in}}%
\pgfpathlineto{\pgfqpoint{0.823770in}{4.513380in}}%
\pgfpathlineto{\pgfqpoint{0.823770in}{4.528703in}}%
\pgfpathlineto{\pgfqpoint{0.844056in}{4.528703in}}%
\pgfpathlineto{\pgfqpoint{0.844056in}{4.559349in}}%
\pgfpathlineto{\pgfqpoint{0.864343in}{4.559349in}}%
\pgfpathlineto{\pgfqpoint{0.864343in}{4.589995in}}%
\pgfpathlineto{\pgfqpoint{0.884630in}{4.589995in}}%
\pgfpathlineto{\pgfqpoint{0.884630in}{4.628302in}}%
\pgfpathlineto{\pgfqpoint{0.904917in}{4.628302in}}%
\pgfpathlineto{\pgfqpoint{0.904917in}{4.658948in}}%
\pgfpathlineto{\pgfqpoint{0.925203in}{4.658948in}}%
\pgfpathlineto{\pgfqpoint{0.925203in}{4.597656in}}%
\pgfpathlineto{\pgfqpoint{0.945490in}{4.597656in}}%
\pgfpathlineto{\pgfqpoint{0.945490in}{4.635964in}}%
\pgfpathlineto{\pgfqpoint{0.965777in}{4.635964in}}%
\pgfpathlineto{\pgfqpoint{0.965777in}{4.704917in}}%
\pgfpathlineto{\pgfqpoint{0.986064in}{4.704917in}}%
\pgfpathlineto{\pgfqpoint{0.986064in}{4.735564in}}%
\pgfpathlineto{\pgfqpoint{1.006351in}{4.735564in}}%
\pgfpathlineto{\pgfqpoint{1.006351in}{4.743225in}}%
\pgfpathlineto{\pgfqpoint{1.026637in}{4.743225in}}%
\pgfpathlineto{\pgfqpoint{1.026637in}{4.789194in}}%
\pgfpathlineto{\pgfqpoint{1.046924in}{4.789194in}}%
\pgfpathlineto{\pgfqpoint{1.046924in}{4.888794in}}%
\pgfpathlineto{\pgfqpoint{1.067211in}{4.888794in}}%
\pgfpathlineto{\pgfqpoint{1.067211in}{4.858148in}}%
\pgfpathlineto{\pgfqpoint{1.087498in}{4.858148in}}%
\pgfpathlineto{\pgfqpoint{1.087498in}{4.973071in}}%
\pgfpathlineto{\pgfqpoint{1.107784in}{4.973071in}}%
\pgfpathlineto{\pgfqpoint{1.107784in}{4.881132in}}%
\pgfpathlineto{\pgfqpoint{1.128071in}{4.881132in}}%
\pgfpathlineto{\pgfqpoint{1.128071in}{4.996055in}}%
\pgfpathlineto{\pgfqpoint{1.148358in}{4.996055in}}%
\pgfpathlineto{\pgfqpoint{1.148358in}{5.095655in}}%
\pgfpathlineto{\pgfqpoint{1.168645in}{5.095655in}}%
\pgfpathlineto{\pgfqpoint{1.168645in}{4.988394in}}%
\pgfpathlineto{\pgfqpoint{1.188931in}{4.988394in}}%
\pgfpathlineto{\pgfqpoint{1.188931in}{5.225901in}}%
\pgfpathlineto{\pgfqpoint{1.209218in}{5.225901in}}%
\pgfpathlineto{\pgfqpoint{1.209218in}{5.087993in}}%
\pgfpathlineto{\pgfqpoint{1.229505in}{5.087993in}}%
\pgfpathlineto{\pgfqpoint{1.229505in}{5.034363in}}%
\pgfpathlineto{\pgfqpoint{1.249792in}{5.034363in}}%
\pgfpathlineto{\pgfqpoint{1.249792in}{5.179932in}}%
\pgfpathlineto{\pgfqpoint{1.270079in}{5.179932in}}%
\pgfpathlineto{\pgfqpoint{1.270079in}{5.087993in}}%
\pgfpathlineto{\pgfqpoint{1.290365in}{5.087993in}}%
\pgfpathlineto{\pgfqpoint{1.290365in}{5.218239in}}%
\pgfpathlineto{\pgfqpoint{1.310652in}{5.218239in}}%
\pgfpathlineto{\pgfqpoint{1.310652in}{5.026701in}}%
\pgfpathlineto{\pgfqpoint{1.330939in}{5.026701in}}%
\pgfpathlineto{\pgfqpoint{1.330939in}{5.019040in}}%
\pgfpathlineto{\pgfqpoint{1.351226in}{5.019040in}}%
\pgfpathlineto{\pgfqpoint{1.351226in}{5.133963in}}%
\pgfpathlineto{\pgfqpoint{1.371512in}{5.133963in}}%
\pgfpathlineto{\pgfqpoint{1.371512in}{4.973071in}}%
\pgfpathlineto{\pgfqpoint{1.391799in}{4.973071in}}%
\pgfpathlineto{\pgfqpoint{1.391799in}{4.827502in}}%
\pgfpathlineto{\pgfqpoint{1.412086in}{4.827502in}}%
\pgfpathlineto{\pgfqpoint{1.412086in}{4.735564in}}%
\pgfpathlineto{\pgfqpoint{1.432373in}{4.735564in}}%
\pgfpathlineto{\pgfqpoint{1.432373in}{4.727902in}}%
\pgfpathlineto{\pgfqpoint{1.452659in}{4.727902in}}%
\pgfpathlineto{\pgfqpoint{1.452659in}{4.643625in}}%
\pgfpathlineto{\pgfqpoint{1.472946in}{4.643625in}}%
\pgfpathlineto{\pgfqpoint{1.472946in}{4.444426in}}%
\pgfusepath{stroke}%
\end{pgfscope}%
\begin{pgfscope}%
\pgfsetrectcap%
\pgfsetmiterjoin%
\pgfsetlinewidth{0.803000pt}%
\definecolor{currentstroke}{rgb}{0.000000,0.000000,0.000000}%
\pgfsetstrokecolor{currentstroke}%
\pgfsetdash{}{0pt}%
\pgfpathmoveto{\pgfqpoint{0.485140in}{4.444426in}}%
\pgfpathlineto{\pgfqpoint{0.485140in}{5.862667in}}%
\pgfusepath{stroke}%
\end{pgfscope}%
\begin{pgfscope}%
\pgfsetrectcap%
\pgfsetmiterjoin%
\pgfsetlinewidth{0.803000pt}%
\definecolor{currentstroke}{rgb}{0.000000,0.000000,0.000000}%
\pgfsetstrokecolor{currentstroke}%
\pgfsetdash{}{0pt}%
\pgfpathmoveto{\pgfqpoint{3.025600in}{4.444426in}}%
\pgfpathlineto{\pgfqpoint{3.025600in}{5.862667in}}%
\pgfusepath{stroke}%
\end{pgfscope}%
\begin{pgfscope}%
\pgfsetrectcap%
\pgfsetmiterjoin%
\pgfsetlinewidth{0.803000pt}%
\definecolor{currentstroke}{rgb}{0.000000,0.000000,0.000000}%
\pgfsetstrokecolor{currentstroke}%
\pgfsetdash{}{0pt}%
\pgfpathmoveto{\pgfqpoint{0.485140in}{4.444426in}}%
\pgfpathlineto{\pgfqpoint{3.025600in}{4.444426in}}%
\pgfusepath{stroke}%
\end{pgfscope}%
\begin{pgfscope}%
\pgfsetrectcap%
\pgfsetmiterjoin%
\pgfsetlinewidth{0.803000pt}%
\definecolor{currentstroke}{rgb}{0.000000,0.000000,0.000000}%
\pgfsetstrokecolor{currentstroke}%
\pgfsetdash{}{0pt}%
\pgfpathmoveto{\pgfqpoint{0.485140in}{5.862667in}}%
\pgfpathlineto{\pgfqpoint{3.025600in}{5.862667in}}%
\pgfusepath{stroke}%
\end{pgfscope}%
\begin{pgfscope}%
\definecolor{textcolor}{rgb}{0.000000,0.000000,0.000000}%
\pgfsetstrokecolor{textcolor}%
\pgfsetfillcolor{textcolor}%
\pgftext[x=0.485140in,y=5.946000in,left,base]{\color{textcolor}\rmfamily\fontsize{10.000000}{12.000000}\selectfont Bin [0.0, 0.17), 1,964 events}%
\end{pgfscope}%
\begin{pgfscope}%
\pgfsetbuttcap%
\pgfsetmiterjoin%
\definecolor{currentfill}{rgb}{1.000000,1.000000,1.000000}%
\pgfsetfillcolor{currentfill}%
\pgfsetfillopacity{0.800000}%
\pgfsetlinewidth{1.003750pt}%
\definecolor{currentstroke}{rgb}{0.800000,0.800000,0.800000}%
\pgfsetstrokecolor{currentstroke}%
\pgfsetstrokeopacity{0.800000}%
\pgfsetdash{}{0pt}%
\pgfpathmoveto{\pgfqpoint{1.990822in}{5.463111in}}%
\pgfpathlineto{\pgfqpoint{2.947822in}{5.463111in}}%
\pgfpathquadraticcurveto{\pgfqpoint{2.970044in}{5.463111in}}{\pgfqpoint{2.970044in}{5.485333in}}%
\pgfpathlineto{\pgfqpoint{2.970044in}{5.784889in}}%
\pgfpathquadraticcurveto{\pgfqpoint{2.970044in}{5.807111in}}{\pgfqpoint{2.947822in}{5.807111in}}%
\pgfpathlineto{\pgfqpoint{1.990822in}{5.807111in}}%
\pgfpathquadraticcurveto{\pgfqpoint{1.968600in}{5.807111in}}{\pgfqpoint{1.968600in}{5.784889in}}%
\pgfpathlineto{\pgfqpoint{1.968600in}{5.485333in}}%
\pgfpathquadraticcurveto{\pgfqpoint{1.968600in}{5.463111in}}{\pgfqpoint{1.990822in}{5.463111in}}%
\pgfpathclose%
\pgfusepath{stroke,fill}%
\end{pgfscope}%
\begin{pgfscope}%
\pgfsetbuttcap%
\pgfsetmiterjoin%
\pgfsetlinewidth{1.003750pt}%
\definecolor{currentstroke}{rgb}{0.313725,0.317647,0.309804}%
\pgfsetstrokecolor{currentstroke}%
\pgfsetdash{}{0pt}%
\pgfpathmoveto{\pgfqpoint{2.013044in}{5.684444in}}%
\pgfpathlineto{\pgfqpoint{2.235267in}{5.684444in}}%
\pgfpathlineto{\pgfqpoint{2.235267in}{5.762222in}}%
\pgfpathlineto{\pgfqpoint{2.013044in}{5.762222in}}%
\pgfpathclose%
\pgfusepath{stroke}%
\end{pgfscope}%
\begin{pgfscope}%
\definecolor{textcolor}{rgb}{0.000000,0.000000,0.000000}%
\pgfsetstrokecolor{textcolor}%
\pgfsetfillcolor{textcolor}%
\pgftext[x=2.324156in,y=5.684444in,left,base]{\color{textcolor}\rmfamily\fontsize{8.000000}{9.600000}\selectfont IQR = 0.14}%
\end{pgfscope}%
\begin{pgfscope}%
\pgfsetbuttcap%
\pgfsetmiterjoin%
\pgfsetlinewidth{1.003750pt}%
\definecolor{currentstroke}{rgb}{0.949020,0.372549,0.360784}%
\pgfsetstrokecolor{currentstroke}%
\pgfsetdash{{1.000000pt}{1.650000pt}}{0.000000pt}%
\pgfpathmoveto{\pgfqpoint{2.013044in}{5.529111in}}%
\pgfpathlineto{\pgfqpoint{2.235267in}{5.529111in}}%
\pgfpathlineto{\pgfqpoint{2.235267in}{5.606889in}}%
\pgfpathlineto{\pgfqpoint{2.013044in}{5.606889in}}%
\pgfpathclose%
\pgfusepath{stroke}%
\end{pgfscope}%
\begin{pgfscope}%
\definecolor{textcolor}{rgb}{0.000000,0.000000,0.000000}%
\pgfsetstrokecolor{textcolor}%
\pgfsetfillcolor{textcolor}%
\pgftext[x=2.324156in,y=5.529111in,left,base]{\color{textcolor}\rmfamily\fontsize{8.000000}{9.600000}\selectfont IQR = 0.26}%
\end{pgfscope}%
\begin{pgfscope}%
\pgfsetbuttcap%
\pgfsetmiterjoin%
\definecolor{currentfill}{rgb}{1.000000,1.000000,1.000000}%
\pgfsetfillcolor{currentfill}%
\pgfsetlinewidth{0.000000pt}%
\definecolor{currentstroke}{rgb}{0.000000,0.000000,0.000000}%
\pgfsetstrokecolor{currentstroke}%
\pgfsetstrokeopacity{0.000000}%
\pgfsetdash{}{0pt}%
\pgfpathmoveto{\pgfqpoint{3.510740in}{4.444426in}}%
\pgfpathlineto{\pgfqpoint{6.051200in}{4.444426in}}%
\pgfpathlineto{\pgfqpoint{6.051200in}{5.862667in}}%
\pgfpathlineto{\pgfqpoint{3.510740in}{5.862667in}}%
\pgfpathclose%
\pgfusepath{fill}%
\end{pgfscope}%
\begin{pgfscope}%
\pgfsetbuttcap%
\pgfsetroundjoin%
\definecolor{currentfill}{rgb}{0.000000,0.000000,0.000000}%
\pgfsetfillcolor{currentfill}%
\pgfsetlinewidth{0.803000pt}%
\definecolor{currentstroke}{rgb}{0.000000,0.000000,0.000000}%
\pgfsetstrokecolor{currentstroke}%
\pgfsetdash{}{0pt}%
\pgfsys@defobject{currentmarker}{\pgfqpoint{0.000000in}{-0.048611in}}{\pgfqpoint{0.000000in}{0.000000in}}{%
\pgfpathmoveto{\pgfqpoint{0.000000in}{0.000000in}}%
\pgfpathlineto{\pgfqpoint{0.000000in}{-0.048611in}}%
\pgfusepath{stroke,fill}%
}%
\begin{pgfscope}%
\pgfsys@transformshift{3.941945in}{4.444426in}%
\pgfsys@useobject{currentmarker}{}%
\end{pgfscope}%
\end{pgfscope}%
\begin{pgfscope}%
\definecolor{textcolor}{rgb}{0.000000,0.000000,0.000000}%
\pgfsetstrokecolor{textcolor}%
\pgfsetfillcolor{textcolor}%
\pgftext[x=3.941945in,y=4.347204in,,top]{\color{textcolor}\rmfamily\fontsize{8.000000}{9.600000}\selectfont \(\displaystyle {-1}\)}%
\end{pgfscope}%
\begin{pgfscope}%
\pgfsetbuttcap%
\pgfsetroundjoin%
\definecolor{currentfill}{rgb}{0.000000,0.000000,0.000000}%
\pgfsetfillcolor{currentfill}%
\pgfsetlinewidth{0.803000pt}%
\definecolor{currentstroke}{rgb}{0.000000,0.000000,0.000000}%
\pgfsetstrokecolor{currentstroke}%
\pgfsetdash{}{0pt}%
\pgfsys@defobject{currentmarker}{\pgfqpoint{0.000000in}{-0.048611in}}{\pgfqpoint{0.000000in}{0.000000in}}{%
\pgfpathmoveto{\pgfqpoint{0.000000in}{0.000000in}}%
\pgfpathlineto{\pgfqpoint{0.000000in}{-0.048611in}}%
\pgfusepath{stroke,fill}%
}%
\begin{pgfscope}%
\pgfsys@transformshift{4.628399in}{4.444426in}%
\pgfsys@useobject{currentmarker}{}%
\end{pgfscope}%
\end{pgfscope}%
\begin{pgfscope}%
\definecolor{textcolor}{rgb}{0.000000,0.000000,0.000000}%
\pgfsetstrokecolor{textcolor}%
\pgfsetfillcolor{textcolor}%
\pgftext[x=4.628399in,y=4.347204in,,top]{\color{textcolor}\rmfamily\fontsize{8.000000}{9.600000}\selectfont \(\displaystyle {0}\)}%
\end{pgfscope}%
\begin{pgfscope}%
\pgfsetbuttcap%
\pgfsetroundjoin%
\definecolor{currentfill}{rgb}{0.000000,0.000000,0.000000}%
\pgfsetfillcolor{currentfill}%
\pgfsetlinewidth{0.803000pt}%
\definecolor{currentstroke}{rgb}{0.000000,0.000000,0.000000}%
\pgfsetstrokecolor{currentstroke}%
\pgfsetdash{}{0pt}%
\pgfsys@defobject{currentmarker}{\pgfqpoint{0.000000in}{-0.048611in}}{\pgfqpoint{0.000000in}{0.000000in}}{%
\pgfpathmoveto{\pgfqpoint{0.000000in}{0.000000in}}%
\pgfpathlineto{\pgfqpoint{0.000000in}{-0.048611in}}%
\pgfusepath{stroke,fill}%
}%
\begin{pgfscope}%
\pgfsys@transformshift{5.314853in}{4.444426in}%
\pgfsys@useobject{currentmarker}{}%
\end{pgfscope}%
\end{pgfscope}%
\begin{pgfscope}%
\definecolor{textcolor}{rgb}{0.000000,0.000000,0.000000}%
\pgfsetstrokecolor{textcolor}%
\pgfsetfillcolor{textcolor}%
\pgftext[x=5.314853in,y=4.347204in,,top]{\color{textcolor}\rmfamily\fontsize{8.000000}{9.600000}\selectfont \(\displaystyle {1}\)}%
\end{pgfscope}%
\begin{pgfscope}%
\pgfsetbuttcap%
\pgfsetroundjoin%
\definecolor{currentfill}{rgb}{0.000000,0.000000,0.000000}%
\pgfsetfillcolor{currentfill}%
\pgfsetlinewidth{0.803000pt}%
\definecolor{currentstroke}{rgb}{0.000000,0.000000,0.000000}%
\pgfsetstrokecolor{currentstroke}%
\pgfsetdash{}{0pt}%
\pgfsys@defobject{currentmarker}{\pgfqpoint{0.000000in}{-0.048611in}}{\pgfqpoint{0.000000in}{0.000000in}}{%
\pgfpathmoveto{\pgfqpoint{0.000000in}{0.000000in}}%
\pgfpathlineto{\pgfqpoint{0.000000in}{-0.048611in}}%
\pgfusepath{stroke,fill}%
}%
\begin{pgfscope}%
\pgfsys@transformshift{6.001306in}{4.444426in}%
\pgfsys@useobject{currentmarker}{}%
\end{pgfscope}%
\end{pgfscope}%
\begin{pgfscope}%
\definecolor{textcolor}{rgb}{0.000000,0.000000,0.000000}%
\pgfsetstrokecolor{textcolor}%
\pgfsetfillcolor{textcolor}%
\pgftext[x=6.001306in,y=4.347204in,,top]{\color{textcolor}\rmfamily\fontsize{8.000000}{9.600000}\selectfont \(\displaystyle {2}\)}%
\end{pgfscope}%
\begin{pgfscope}%
\pgfsetbuttcap%
\pgfsetroundjoin%
\definecolor{currentfill}{rgb}{0.000000,0.000000,0.000000}%
\pgfsetfillcolor{currentfill}%
\pgfsetlinewidth{0.803000pt}%
\definecolor{currentstroke}{rgb}{0.000000,0.000000,0.000000}%
\pgfsetstrokecolor{currentstroke}%
\pgfsetdash{}{0pt}%
\pgfsys@defobject{currentmarker}{\pgfqpoint{-0.048611in}{0.000000in}}{\pgfqpoint{-0.000000in}{0.000000in}}{%
\pgfpathmoveto{\pgfqpoint{-0.000000in}{0.000000in}}%
\pgfpathlineto{\pgfqpoint{-0.048611in}{0.000000in}}%
\pgfusepath{stroke,fill}%
}%
\begin{pgfscope}%
\pgfsys@transformshift{3.510740in}{4.444426in}%
\pgfsys@useobject{currentmarker}{}%
\end{pgfscope}%
\end{pgfscope}%
\begin{pgfscope}%
\definecolor{textcolor}{rgb}{0.000000,0.000000,0.000000}%
\pgfsetstrokecolor{textcolor}%
\pgfsetfillcolor{textcolor}%
\pgftext[x=3.354489in, y=4.405870in, left, base]{\color{textcolor}\rmfamily\fontsize{8.000000}{9.600000}\selectfont \(\displaystyle {0}\)}%
\end{pgfscope}%
\begin{pgfscope}%
\pgfsetbuttcap%
\pgfsetroundjoin%
\definecolor{currentfill}{rgb}{0.000000,0.000000,0.000000}%
\pgfsetfillcolor{currentfill}%
\pgfsetlinewidth{0.803000pt}%
\definecolor{currentstroke}{rgb}{0.000000,0.000000,0.000000}%
\pgfsetstrokecolor{currentstroke}%
\pgfsetdash{}{0pt}%
\pgfsys@defobject{currentmarker}{\pgfqpoint{-0.048611in}{0.000000in}}{\pgfqpoint{-0.000000in}{0.000000in}}{%
\pgfpathmoveto{\pgfqpoint{-0.000000in}{0.000000in}}%
\pgfpathlineto{\pgfqpoint{-0.048611in}{0.000000in}}%
\pgfusepath{stroke,fill}%
}%
\begin{pgfscope}%
\pgfsys@transformshift{3.510740in}{4.976654in}%
\pgfsys@useobject{currentmarker}{}%
\end{pgfscope}%
\end{pgfscope}%
\begin{pgfscope}%
\definecolor{textcolor}{rgb}{0.000000,0.000000,0.000000}%
\pgfsetstrokecolor{textcolor}%
\pgfsetfillcolor{textcolor}%
\pgftext[x=3.354489in, y=4.938099in, left, base]{\color{textcolor}\rmfamily\fontsize{8.000000}{9.600000}\selectfont \(\displaystyle {1}\)}%
\end{pgfscope}%
\begin{pgfscope}%
\pgfsetbuttcap%
\pgfsetroundjoin%
\definecolor{currentfill}{rgb}{0.000000,0.000000,0.000000}%
\pgfsetfillcolor{currentfill}%
\pgfsetlinewidth{0.803000pt}%
\definecolor{currentstroke}{rgb}{0.000000,0.000000,0.000000}%
\pgfsetstrokecolor{currentstroke}%
\pgfsetdash{}{0pt}%
\pgfsys@defobject{currentmarker}{\pgfqpoint{-0.048611in}{0.000000in}}{\pgfqpoint{-0.000000in}{0.000000in}}{%
\pgfpathmoveto{\pgfqpoint{-0.000000in}{0.000000in}}%
\pgfpathlineto{\pgfqpoint{-0.048611in}{0.000000in}}%
\pgfusepath{stroke,fill}%
}%
\begin{pgfscope}%
\pgfsys@transformshift{3.510740in}{5.508883in}%
\pgfsys@useobject{currentmarker}{}%
\end{pgfscope}%
\end{pgfscope}%
\begin{pgfscope}%
\definecolor{textcolor}{rgb}{0.000000,0.000000,0.000000}%
\pgfsetstrokecolor{textcolor}%
\pgfsetfillcolor{textcolor}%
\pgftext[x=3.354489in, y=5.470328in, left, base]{\color{textcolor}\rmfamily\fontsize{8.000000}{9.600000}\selectfont \(\displaystyle {2}\)}%
\end{pgfscope}%
\begin{pgfscope}%
\definecolor{textcolor}{rgb}{0.000000,0.000000,0.000000}%
\pgfsetstrokecolor{textcolor}%
\pgfsetfillcolor{textcolor}%
\pgftext[x=3.298933in,y=5.153546in,,bottom,rotate=90.000000]{\color{textcolor}\rmfamily\fontsize{10.000000}{12.000000}\selectfont Density}%
\end{pgfscope}%
\begin{pgfscope}%
\pgfpathrectangle{\pgfqpoint{3.510740in}{4.444426in}}{\pgfqpoint{2.540460in}{1.418241in}}%
\pgfusepath{clip}%
\pgfsetbuttcap%
\pgfsetmiterjoin%
\pgfsetlinewidth{1.003750pt}%
\definecolor{currentstroke}{rgb}{0.313725,0.317647,0.309804}%
\pgfsetstrokecolor{currentstroke}%
\pgfsetdash{}{0pt}%
\pgfpathmoveto{\pgfqpoint{3.777515in}{4.444426in}}%
\pgfpathlineto{\pgfqpoint{3.777515in}{4.445111in}}%
\pgfpathlineto{\pgfqpoint{3.945088in}{4.444426in}}%
\pgfpathlineto{\pgfqpoint{3.945088in}{4.446483in}}%
\pgfpathlineto{\pgfqpoint{3.952070in}{4.446483in}}%
\pgfpathlineto{\pgfqpoint{3.952070in}{4.444426in}}%
\pgfpathlineto{\pgfqpoint{3.959053in}{4.444426in}}%
\pgfpathlineto{\pgfqpoint{3.959053in}{4.446483in}}%
\pgfpathlineto{\pgfqpoint{3.966035in}{4.446483in}}%
\pgfpathlineto{\pgfqpoint{3.966035in}{4.444426in}}%
\pgfpathlineto{\pgfqpoint{3.979999in}{4.445111in}}%
\pgfpathlineto{\pgfqpoint{3.979999in}{4.446483in}}%
\pgfpathlineto{\pgfqpoint{4.000946in}{4.445797in}}%
\pgfpathlineto{\pgfqpoint{4.000946in}{4.448540in}}%
\pgfpathlineto{\pgfqpoint{4.007928in}{4.448540in}}%
\pgfpathlineto{\pgfqpoint{4.007928in}{4.446483in}}%
\pgfpathlineto{\pgfqpoint{4.014910in}{4.446483in}}%
\pgfpathlineto{\pgfqpoint{4.014910in}{4.448540in}}%
\pgfpathlineto{\pgfqpoint{4.021893in}{4.448540in}}%
\pgfpathlineto{\pgfqpoint{4.021893in}{4.446483in}}%
\pgfpathlineto{\pgfqpoint{4.028875in}{4.446483in}}%
\pgfpathlineto{\pgfqpoint{4.028875in}{4.448540in}}%
\pgfpathlineto{\pgfqpoint{4.035857in}{4.448540in}}%
\pgfpathlineto{\pgfqpoint{4.035857in}{4.447168in}}%
\pgfpathlineto{\pgfqpoint{4.049821in}{4.446483in}}%
\pgfpathlineto{\pgfqpoint{4.049821in}{4.451282in}}%
\pgfpathlineto{\pgfqpoint{4.056804in}{4.451282in}}%
\pgfpathlineto{\pgfqpoint{4.056804in}{4.447854in}}%
\pgfpathlineto{\pgfqpoint{4.084733in}{4.448540in}}%
\pgfpathlineto{\pgfqpoint{4.084733in}{4.449911in}}%
\pgfpathlineto{\pgfqpoint{4.091715in}{4.449911in}}%
\pgfpathlineto{\pgfqpoint{4.091715in}{4.451968in}}%
\pgfpathlineto{\pgfqpoint{4.112662in}{4.451282in}}%
\pgfpathlineto{\pgfqpoint{4.112662in}{4.454025in}}%
\pgfpathlineto{\pgfqpoint{4.119644in}{4.454025in}}%
\pgfpathlineto{\pgfqpoint{4.119644in}{4.452653in}}%
\pgfpathlineto{\pgfqpoint{4.133608in}{4.453339in}}%
\pgfpathlineto{\pgfqpoint{4.133608in}{4.456767in}}%
\pgfpathlineto{\pgfqpoint{4.140590in}{4.456767in}}%
\pgfpathlineto{\pgfqpoint{4.140590in}{4.454025in}}%
\pgfpathlineto{\pgfqpoint{4.147573in}{4.454025in}}%
\pgfpathlineto{\pgfqpoint{4.147573in}{4.455396in}}%
\pgfpathlineto{\pgfqpoint{4.161537in}{4.455396in}}%
\pgfpathlineto{\pgfqpoint{4.161537in}{4.462938in}}%
\pgfpathlineto{\pgfqpoint{4.168519in}{4.462938in}}%
\pgfpathlineto{\pgfqpoint{4.168519in}{4.461567in}}%
\pgfpathlineto{\pgfqpoint{4.175502in}{4.461567in}}%
\pgfpathlineto{\pgfqpoint{4.175502in}{4.462938in}}%
\pgfpathlineto{\pgfqpoint{4.189466in}{4.462938in}}%
\pgfpathlineto{\pgfqpoint{4.189466in}{4.478708in}}%
\pgfpathlineto{\pgfqpoint{4.196448in}{4.478708in}}%
\pgfpathlineto{\pgfqpoint{4.196448in}{4.470480in}}%
\pgfpathlineto{\pgfqpoint{4.203430in}{4.470480in}}%
\pgfpathlineto{\pgfqpoint{4.203430in}{4.473908in}}%
\pgfpathlineto{\pgfqpoint{4.210413in}{4.473908in}}%
\pgfpathlineto{\pgfqpoint{4.210413in}{4.476651in}}%
\pgfpathlineto{\pgfqpoint{4.217395in}{4.476651in}}%
\pgfpathlineto{\pgfqpoint{4.217395in}{4.483507in}}%
\pgfpathlineto{\pgfqpoint{4.224377in}{4.483507in}}%
\pgfpathlineto{\pgfqpoint{4.224377in}{4.479393in}}%
\pgfpathlineto{\pgfqpoint{4.231359in}{4.479393in}}%
\pgfpathlineto{\pgfqpoint{4.231359in}{4.483507in}}%
\pgfpathlineto{\pgfqpoint{4.238342in}{4.483507in}}%
\pgfpathlineto{\pgfqpoint{4.238342in}{4.493792in}}%
\pgfpathlineto{\pgfqpoint{4.245324in}{4.493792in}}%
\pgfpathlineto{\pgfqpoint{4.245324in}{4.505448in}}%
\pgfpathlineto{\pgfqpoint{4.252306in}{4.505448in}}%
\pgfpathlineto{\pgfqpoint{4.252306in}{4.513675in}}%
\pgfpathlineto{\pgfqpoint{4.259288in}{4.513675in}}%
\pgfpathlineto{\pgfqpoint{4.259288in}{4.524645in}}%
\pgfpathlineto{\pgfqpoint{4.266270in}{4.524645in}}%
\pgfpathlineto{\pgfqpoint{4.266270in}{4.530131in}}%
\pgfpathlineto{\pgfqpoint{4.273253in}{4.530131in}}%
\pgfpathlineto{\pgfqpoint{4.273253in}{4.532873in}}%
\pgfpathlineto{\pgfqpoint{4.280235in}{4.532873in}}%
\pgfpathlineto{\pgfqpoint{4.280235in}{4.538358in}}%
\pgfpathlineto{\pgfqpoint{4.287217in}{4.538358in}}%
\pgfpathlineto{\pgfqpoint{4.287217in}{4.549328in}}%
\pgfpathlineto{\pgfqpoint{4.294199in}{4.549328in}}%
\pgfpathlineto{\pgfqpoint{4.294199in}{4.569212in}}%
\pgfpathlineto{\pgfqpoint{4.301182in}{4.569212in}}%
\pgfpathlineto{\pgfqpoint{4.301182in}{4.576068in}}%
\pgfpathlineto{\pgfqpoint{4.308164in}{4.576068in}}%
\pgfpathlineto{\pgfqpoint{4.308164in}{4.584981in}}%
\pgfpathlineto{\pgfqpoint{4.315146in}{4.584981in}}%
\pgfpathlineto{\pgfqpoint{4.315146in}{4.611036in}}%
\pgfpathlineto{\pgfqpoint{4.322128in}{4.611036in}}%
\pgfpathlineto{\pgfqpoint{4.322128in}{4.628177in}}%
\pgfpathlineto{\pgfqpoint{4.329110in}{4.628177in}}%
\pgfpathlineto{\pgfqpoint{4.329110in}{4.633662in}}%
\pgfpathlineto{\pgfqpoint{4.336093in}{4.633662in}}%
\pgfpathlineto{\pgfqpoint{4.336093in}{4.652174in}}%
\pgfpathlineto{\pgfqpoint{4.343075in}{4.652174in}}%
\pgfpathlineto{\pgfqpoint{4.343075in}{4.669315in}}%
\pgfpathlineto{\pgfqpoint{4.350057in}{4.669315in}}%
\pgfpathlineto{\pgfqpoint{4.350057in}{4.714567in}}%
\pgfpathlineto{\pgfqpoint{4.357039in}{4.714567in}}%
\pgfpathlineto{\pgfqpoint{4.357039in}{4.744735in}}%
\pgfpathlineto{\pgfqpoint{4.364022in}{4.744735in}}%
\pgfpathlineto{\pgfqpoint{4.364022in}{4.763247in}}%
\pgfpathlineto{\pgfqpoint{4.371004in}{4.763247in}}%
\pgfpathlineto{\pgfqpoint{4.371004in}{4.796843in}}%
\pgfpathlineto{\pgfqpoint{4.377986in}{4.796843in}}%
\pgfpathlineto{\pgfqpoint{4.377986in}{4.809871in}}%
\pgfpathlineto{\pgfqpoint{4.384968in}{4.809871in}}%
\pgfpathlineto{\pgfqpoint{4.384968in}{4.838667in}}%
\pgfpathlineto{\pgfqpoint{4.391951in}{4.838667in}}%
\pgfpathlineto{\pgfqpoint{4.391951in}{4.903117in}}%
\pgfpathlineto{\pgfqpoint{4.398933in}{4.903117in}}%
\pgfpathlineto{\pgfqpoint{4.398933in}{4.950426in}}%
\pgfpathlineto{\pgfqpoint{4.405915in}{4.950426in}}%
\pgfpathlineto{\pgfqpoint{4.405915in}{4.984022in}}%
\pgfpathlineto{\pgfqpoint{4.412897in}{4.984022in}}%
\pgfpathlineto{\pgfqpoint{4.412897in}{5.012134in}}%
\pgfpathlineto{\pgfqpoint{4.419879in}{5.012134in}}%
\pgfpathlineto{\pgfqpoint{4.419879in}{5.062871in}}%
\pgfpathlineto{\pgfqpoint{4.426862in}{5.062871in}}%
\pgfpathlineto{\pgfqpoint{4.426862in}{5.145147in}}%
\pgfpathlineto{\pgfqpoint{4.433844in}{5.145147in}}%
\pgfpathlineto{\pgfqpoint{4.433844in}{5.194513in}}%
\pgfpathlineto{\pgfqpoint{4.440826in}{5.194513in}}%
\pgfpathlineto{\pgfqpoint{4.440826in}{5.234966in}}%
\pgfpathlineto{\pgfqpoint{4.447808in}{5.234966in}}%
\pgfpathlineto{\pgfqpoint{4.447808in}{5.239765in}}%
\pgfpathlineto{\pgfqpoint{4.454791in}{5.239765in}}%
\pgfpathlineto{\pgfqpoint{4.454791in}{5.287074in}}%
\pgfpathlineto{\pgfqpoint{4.461773in}{5.287074in}}%
\pgfpathlineto{\pgfqpoint{4.461773in}{5.362494in}}%
\pgfpathlineto{\pgfqpoint{4.468755in}{5.362494in}}%
\pgfpathlineto{\pgfqpoint{4.468755in}{5.374836in}}%
\pgfpathlineto{\pgfqpoint{4.475737in}{5.374836in}}%
\pgfpathlineto{\pgfqpoint{4.475737in}{5.435172in}}%
\pgfpathlineto{\pgfqpoint{4.482719in}{5.435172in}}%
\pgfpathlineto{\pgfqpoint{4.482719in}{5.511278in}}%
\pgfpathlineto{\pgfqpoint{4.489702in}{5.511278in}}%
\pgfpathlineto{\pgfqpoint{4.489702in}{5.521562in}}%
\pgfpathlineto{\pgfqpoint{4.496684in}{5.521562in}}%
\pgfpathlineto{\pgfqpoint{4.496684in}{5.568871in}}%
\pgfpathlineto{\pgfqpoint{4.503666in}{5.568871in}}%
\pgfpathlineto{\pgfqpoint{4.503666in}{5.620294in}}%
\pgfpathlineto{\pgfqpoint{4.510648in}{5.620294in}}%
\pgfpathlineto{\pgfqpoint{4.510648in}{5.627836in}}%
\pgfpathlineto{\pgfqpoint{4.517631in}{5.627836in}}%
\pgfpathlineto{\pgfqpoint{4.517631in}{5.647719in}}%
\pgfpathlineto{\pgfqpoint{4.524613in}{5.647719in}}%
\pgfpathlineto{\pgfqpoint{4.524613in}{5.701885in}}%
\pgfpathlineto{\pgfqpoint{4.531595in}{5.701885in}}%
\pgfpathlineto{\pgfqpoint{4.531595in}{5.727253in}}%
\pgfpathlineto{\pgfqpoint{4.538577in}{5.727253in}}%
\pgfpathlineto{\pgfqpoint{4.538577in}{5.753993in}}%
\pgfpathlineto{\pgfqpoint{4.545559in}{5.753993in}}%
\pgfpathlineto{\pgfqpoint{4.545559in}{5.780733in}}%
\pgfpathlineto{\pgfqpoint{4.552542in}{5.780733in}}%
\pgfpathlineto{\pgfqpoint{4.552542in}{5.795131in}}%
\pgfpathlineto{\pgfqpoint{4.559524in}{5.795131in}}%
\pgfpathlineto{\pgfqpoint{4.559524in}{5.743023in}}%
\pgfpathlineto{\pgfqpoint{4.566506in}{5.743023in}}%
\pgfpathlineto{\pgfqpoint{4.566506in}{5.757421in}}%
\pgfpathlineto{\pgfqpoint{4.573488in}{5.757421in}}%
\pgfpathlineto{\pgfqpoint{4.573488in}{5.736167in}}%
\pgfpathlineto{\pgfqpoint{4.580471in}{5.736167in}}%
\pgfpathlineto{\pgfqpoint{4.580471in}{5.760850in}}%
\pgfpathlineto{\pgfqpoint{4.587453in}{5.760850in}}%
\pgfpathlineto{\pgfqpoint{4.587453in}{5.764278in}}%
\pgfpathlineto{\pgfqpoint{4.594435in}{5.764278in}}%
\pgfpathlineto{\pgfqpoint{4.594435in}{5.699828in}}%
\pgfpathlineto{\pgfqpoint{4.601417in}{5.699828in}}%
\pgfpathlineto{\pgfqpoint{4.601417in}{5.660061in}}%
\pgfpathlineto{\pgfqpoint{4.608399in}{5.660061in}}%
\pgfpathlineto{\pgfqpoint{4.608399in}{5.636064in}}%
\pgfpathlineto{\pgfqpoint{4.615382in}{5.636064in}}%
\pgfpathlineto{\pgfqpoint{4.615382in}{5.633321in}}%
\pgfpathlineto{\pgfqpoint{4.622364in}{5.633321in}}%
\pgfpathlineto{\pgfqpoint{4.622364in}{5.610009in}}%
\pgfpathlineto{\pgfqpoint{4.629346in}{5.610009in}}%
\pgfpathlineto{\pgfqpoint{4.629346in}{5.536646in}}%
\pgfpathlineto{\pgfqpoint{4.636328in}{5.536646in}}%
\pgfpathlineto{\pgfqpoint{4.636328in}{5.470825in}}%
\pgfpathlineto{\pgfqpoint{4.643311in}{5.470825in}}%
\pgfpathlineto{\pgfqpoint{4.643311in}{5.391291in}}%
\pgfpathlineto{\pgfqpoint{4.650293in}{5.391291in}}%
\pgfpathlineto{\pgfqpoint{4.650293in}{5.347410in}}%
\pgfpathlineto{\pgfqpoint{4.657275in}{5.347410in}}%
\pgfpathlineto{\pgfqpoint{4.657275in}{5.264448in}}%
\pgfpathlineto{\pgfqpoint{4.664257in}{5.264448in}}%
\pgfpathlineto{\pgfqpoint{4.664257in}{5.276104in}}%
\pgfpathlineto{\pgfqpoint{4.671240in}{5.276104in}}%
\pgfpathlineto{\pgfqpoint{4.671240in}{5.226052in}}%
\pgfpathlineto{\pgfqpoint{4.678222in}{5.226052in}}%
\pgfpathlineto{\pgfqpoint{4.678222in}{5.154060in}}%
\pgfpathlineto{\pgfqpoint{4.685204in}{5.154060in}}%
\pgfpathlineto{\pgfqpoint{4.685204in}{5.106752in}}%
\pgfpathlineto{\pgfqpoint{4.692186in}{5.106752in}}%
\pgfpathlineto{\pgfqpoint{4.692186in}{5.018990in}}%
\pgfpathlineto{\pgfqpoint{4.699168in}{5.018990in}}%
\pgfpathlineto{\pgfqpoint{4.699168in}{4.973052in}}%
\pgfpathlineto{\pgfqpoint{4.706151in}{4.973052in}}%
\pgfpathlineto{\pgfqpoint{4.706151in}{4.935342in}}%
\pgfpathlineto{\pgfqpoint{4.713133in}{4.935342in}}%
\pgfpathlineto{\pgfqpoint{4.713133in}{4.901060in}}%
\pgfpathlineto{\pgfqpoint{4.720115in}{4.901060in}}%
\pgfpathlineto{\pgfqpoint{4.720115in}{4.824269in}}%
\pgfpathlineto{\pgfqpoint{4.727097in}{4.824269in}}%
\pgfpathlineto{\pgfqpoint{4.727097in}{4.805757in}}%
\pgfpathlineto{\pgfqpoint{4.734080in}{4.805757in}}%
\pgfpathlineto{\pgfqpoint{4.734080in}{4.737193in}}%
\pgfpathlineto{\pgfqpoint{4.741062in}{4.737193in}}%
\pgfpathlineto{\pgfqpoint{4.741062in}{4.731708in}}%
\pgfpathlineto{\pgfqpoint{4.748044in}{4.731708in}}%
\pgfpathlineto{\pgfqpoint{4.748044in}{4.666572in}}%
\pgfpathlineto{\pgfqpoint{4.755026in}{4.666572in}}%
\pgfpathlineto{\pgfqpoint{4.755026in}{4.647374in}}%
\pgfpathlineto{\pgfqpoint{4.762008in}{4.647374in}}%
\pgfpathlineto{\pgfqpoint{4.762008in}{4.600066in}}%
\pgfpathlineto{\pgfqpoint{4.768991in}{4.600066in}}%
\pgfpathlineto{\pgfqpoint{4.768991in}{4.587724in}}%
\pgfpathlineto{\pgfqpoint{4.775973in}{4.587724in}}%
\pgfpathlineto{\pgfqpoint{4.775973in}{4.568526in}}%
\pgfpathlineto{\pgfqpoint{4.782955in}{4.568526in}}%
\pgfpathlineto{\pgfqpoint{4.782955in}{4.555499in}}%
\pgfpathlineto{\pgfqpoint{4.789937in}{4.555499in}}%
\pgfpathlineto{\pgfqpoint{4.789937in}{4.527388in}}%
\pgfpathlineto{\pgfqpoint{4.796920in}{4.527388in}}%
\pgfpathlineto{\pgfqpoint{4.796920in}{4.508876in}}%
\pgfpathlineto{\pgfqpoint{4.803902in}{4.508876in}}%
\pgfpathlineto{\pgfqpoint{4.803902in}{4.504076in}}%
\pgfpathlineto{\pgfqpoint{4.810884in}{4.504076in}}%
\pgfpathlineto{\pgfqpoint{4.810884in}{4.501334in}}%
\pgfpathlineto{\pgfqpoint{4.817866in}{4.501334in}}%
\pgfpathlineto{\pgfqpoint{4.817866in}{4.473223in}}%
\pgfpathlineto{\pgfqpoint{4.824848in}{4.473223in}}%
\pgfpathlineto{\pgfqpoint{4.824848in}{4.471851in}}%
\pgfpathlineto{\pgfqpoint{4.831831in}{4.471851in}}%
\pgfpathlineto{\pgfqpoint{4.831831in}{4.464309in}}%
\pgfpathlineto{\pgfqpoint{4.838813in}{4.464309in}}%
\pgfpathlineto{\pgfqpoint{4.838813in}{4.465681in}}%
\pgfpathlineto{\pgfqpoint{4.845795in}{4.465681in}}%
\pgfpathlineto{\pgfqpoint{4.845795in}{4.460195in}}%
\pgfpathlineto{\pgfqpoint{4.852777in}{4.460195in}}%
\pgfpathlineto{\pgfqpoint{4.852777in}{4.454710in}}%
\pgfpathlineto{\pgfqpoint{4.859760in}{4.454710in}}%
\pgfpathlineto{\pgfqpoint{4.859760in}{4.451282in}}%
\pgfpathlineto{\pgfqpoint{4.866742in}{4.451282in}}%
\pgfpathlineto{\pgfqpoint{4.866742in}{4.454025in}}%
\pgfpathlineto{\pgfqpoint{4.873724in}{4.454025in}}%
\pgfpathlineto{\pgfqpoint{4.873724in}{4.446483in}}%
\pgfpathlineto{\pgfqpoint{4.880706in}{4.446483in}}%
\pgfpathlineto{\pgfqpoint{4.880706in}{4.448540in}}%
\pgfpathlineto{\pgfqpoint{4.887688in}{4.448540in}}%
\pgfpathlineto{\pgfqpoint{4.887688in}{4.445797in}}%
\pgfpathlineto{\pgfqpoint{4.901653in}{4.446483in}}%
\pgfpathlineto{\pgfqpoint{4.901653in}{4.447168in}}%
\pgfpathlineto{\pgfqpoint{4.908635in}{4.447168in}}%
\pgfpathlineto{\pgfqpoint{4.908635in}{4.445111in}}%
\pgfpathlineto{\pgfqpoint{4.922600in}{4.445111in}}%
\pgfpathlineto{\pgfqpoint{4.922600in}{4.447168in}}%
\pgfpathlineto{\pgfqpoint{4.929582in}{4.447168in}}%
\pgfpathlineto{\pgfqpoint{4.929582in}{4.444426in}}%
\pgfpathlineto{\pgfqpoint{5.020351in}{4.445111in}}%
\pgfpathlineto{\pgfqpoint{5.020351in}{4.444426in}}%
\pgfusepath{stroke}%
\end{pgfscope}%
\begin{pgfscope}%
\pgfpathrectangle{\pgfqpoint{3.510740in}{4.444426in}}{\pgfqpoint{2.540460in}{1.418241in}}%
\pgfusepath{clip}%
\pgfsetbuttcap%
\pgfsetmiterjoin%
\pgfsetlinewidth{1.003750pt}%
\definecolor{currentstroke}{rgb}{0.949020,0.372549,0.360784}%
\pgfsetstrokecolor{currentstroke}%
\pgfsetdash{{1.000000pt}{1.650000pt}}{0.000000pt}%
\pgfpathmoveto{\pgfqpoint{3.777515in}{4.444426in}}%
\pgfpathlineto{\pgfqpoint{3.777515in}{4.455451in}}%
\pgfpathlineto{\pgfqpoint{3.784497in}{4.455451in}}%
\pgfpathlineto{\pgfqpoint{3.784497in}{4.452695in}}%
\pgfpathlineto{\pgfqpoint{3.791479in}{4.452695in}}%
\pgfpathlineto{\pgfqpoint{3.791479in}{4.458208in}}%
\pgfpathlineto{\pgfqpoint{3.798461in}{4.458208in}}%
\pgfpathlineto{\pgfqpoint{3.798461in}{4.460964in}}%
\pgfpathlineto{\pgfqpoint{3.805444in}{4.460964in}}%
\pgfpathlineto{\pgfqpoint{3.805444in}{4.457518in}}%
\pgfpathlineto{\pgfqpoint{3.812426in}{4.457518in}}%
\pgfpathlineto{\pgfqpoint{3.812426in}{4.466477in}}%
\pgfpathlineto{\pgfqpoint{3.819408in}{4.466477in}}%
\pgfpathlineto{\pgfqpoint{3.819408in}{4.463720in}}%
\pgfpathlineto{\pgfqpoint{3.826390in}{4.463720in}}%
\pgfpathlineto{\pgfqpoint{3.826390in}{4.457518in}}%
\pgfpathlineto{\pgfqpoint{3.840355in}{4.456829in}}%
\pgfpathlineto{\pgfqpoint{3.840355in}{4.458897in}}%
\pgfpathlineto{\pgfqpoint{3.847337in}{4.458897in}}%
\pgfpathlineto{\pgfqpoint{3.847337in}{4.463031in}}%
\pgfpathlineto{\pgfqpoint{3.861301in}{4.463031in}}%
\pgfpathlineto{\pgfqpoint{3.861301in}{4.472678in}}%
\pgfpathlineto{\pgfqpoint{3.868284in}{4.472678in}}%
\pgfpathlineto{\pgfqpoint{3.868284in}{4.470611in}}%
\pgfpathlineto{\pgfqpoint{3.875266in}{4.470611in}}%
\pgfpathlineto{\pgfqpoint{3.875266in}{4.463031in}}%
\pgfpathlineto{\pgfqpoint{3.889230in}{4.463031in}}%
\pgfpathlineto{\pgfqpoint{3.889230in}{4.465098in}}%
\pgfpathlineto{\pgfqpoint{3.903195in}{4.465098in}}%
\pgfpathlineto{\pgfqpoint{3.903195in}{4.469922in}}%
\pgfpathlineto{\pgfqpoint{3.924141in}{4.470611in}}%
\pgfpathlineto{\pgfqpoint{3.924141in}{4.480258in}}%
\pgfpathlineto{\pgfqpoint{3.931124in}{4.480258in}}%
\pgfpathlineto{\pgfqpoint{3.931124in}{4.467166in}}%
\pgfpathlineto{\pgfqpoint{3.938106in}{4.467166in}}%
\pgfpathlineto{\pgfqpoint{3.938106in}{4.475435in}}%
\pgfpathlineto{\pgfqpoint{3.945088in}{4.475435in}}%
\pgfpathlineto{\pgfqpoint{3.945088in}{4.472678in}}%
\pgfpathlineto{\pgfqpoint{3.952070in}{4.472678in}}%
\pgfpathlineto{\pgfqpoint{3.952070in}{4.465788in}}%
\pgfpathlineto{\pgfqpoint{3.959053in}{4.465788in}}%
\pgfpathlineto{\pgfqpoint{3.959053in}{4.471300in}}%
\pgfpathlineto{\pgfqpoint{3.966035in}{4.471300in}}%
\pgfpathlineto{\pgfqpoint{3.966035in}{4.474746in}}%
\pgfpathlineto{\pgfqpoint{3.973017in}{4.474746in}}%
\pgfpathlineto{\pgfqpoint{3.973017in}{4.471300in}}%
\pgfpathlineto{\pgfqpoint{3.979999in}{4.471300in}}%
\pgfpathlineto{\pgfqpoint{3.979999in}{4.472678in}}%
\pgfpathlineto{\pgfqpoint{3.986981in}{4.472678in}}%
\pgfpathlineto{\pgfqpoint{3.986981in}{4.483015in}}%
\pgfpathlineto{\pgfqpoint{3.993964in}{4.483015in}}%
\pgfpathlineto{\pgfqpoint{3.993964in}{4.469922in}}%
\pgfpathlineto{\pgfqpoint{4.000946in}{4.469922in}}%
\pgfpathlineto{\pgfqpoint{4.000946in}{4.487149in}}%
\pgfpathlineto{\pgfqpoint{4.007928in}{4.487149in}}%
\pgfpathlineto{\pgfqpoint{4.007928in}{4.484393in}}%
\pgfpathlineto{\pgfqpoint{4.014910in}{4.484393in}}%
\pgfpathlineto{\pgfqpoint{4.014910in}{4.480947in}}%
\pgfpathlineto{\pgfqpoint{4.021893in}{4.480947in}}%
\pgfpathlineto{\pgfqpoint{4.021893in}{4.471300in}}%
\pgfpathlineto{\pgfqpoint{4.028875in}{4.471300in}}%
\pgfpathlineto{\pgfqpoint{4.028875in}{4.489216in}}%
\pgfpathlineto{\pgfqpoint{4.035857in}{4.489216in}}%
\pgfpathlineto{\pgfqpoint{4.035857in}{4.487838in}}%
\pgfpathlineto{\pgfqpoint{4.042839in}{4.487838in}}%
\pgfpathlineto{\pgfqpoint{4.042839in}{4.483015in}}%
\pgfpathlineto{\pgfqpoint{4.049821in}{4.483015in}}%
\pgfpathlineto{\pgfqpoint{4.049821in}{4.496107in}}%
\pgfpathlineto{\pgfqpoint{4.056804in}{4.496107in}}%
\pgfpathlineto{\pgfqpoint{4.056804in}{4.491284in}}%
\pgfpathlineto{\pgfqpoint{4.070768in}{4.491973in}}%
\pgfpathlineto{\pgfqpoint{4.070768in}{4.496796in}}%
\pgfpathlineto{\pgfqpoint{4.077750in}{4.496796in}}%
\pgfpathlineto{\pgfqpoint{4.077750in}{4.500242in}}%
\pgfpathlineto{\pgfqpoint{4.084733in}{4.500242in}}%
\pgfpathlineto{\pgfqpoint{4.084733in}{4.503687in}}%
\pgfpathlineto{\pgfqpoint{4.091715in}{4.503687in}}%
\pgfpathlineto{\pgfqpoint{4.091715in}{4.510578in}}%
\pgfpathlineto{\pgfqpoint{4.098697in}{4.510578in}}%
\pgfpathlineto{\pgfqpoint{4.098697in}{4.509200in}}%
\pgfpathlineto{\pgfqpoint{4.105679in}{4.509200in}}%
\pgfpathlineto{\pgfqpoint{4.105679in}{4.507822in}}%
\pgfpathlineto{\pgfqpoint{4.112662in}{4.507822in}}%
\pgfpathlineto{\pgfqpoint{4.112662in}{4.514713in}}%
\pgfpathlineto{\pgfqpoint{4.119644in}{4.514713in}}%
\pgfpathlineto{\pgfqpoint{4.119644in}{4.520914in}}%
\pgfpathlineto{\pgfqpoint{4.126626in}{4.520914in}}%
\pgfpathlineto{\pgfqpoint{4.126626in}{4.528494in}}%
\pgfpathlineto{\pgfqpoint{4.133608in}{4.528494in}}%
\pgfpathlineto{\pgfqpoint{4.133608in}{4.525738in}}%
\pgfpathlineto{\pgfqpoint{4.140590in}{4.525738in}}%
\pgfpathlineto{\pgfqpoint{4.140590in}{4.538831in}}%
\pgfpathlineto{\pgfqpoint{4.147573in}{4.538831in}}%
\pgfpathlineto{\pgfqpoint{4.147573in}{4.542276in}}%
\pgfpathlineto{\pgfqpoint{4.154555in}{4.542276in}}%
\pgfpathlineto{\pgfqpoint{4.154555in}{4.556058in}}%
\pgfpathlineto{\pgfqpoint{4.161537in}{4.556058in}}%
\pgfpathlineto{\pgfqpoint{4.161537in}{4.545722in}}%
\pgfpathlineto{\pgfqpoint{4.168519in}{4.545722in}}%
\pgfpathlineto{\pgfqpoint{4.168519in}{4.560881in}}%
\pgfpathlineto{\pgfqpoint{4.175502in}{4.560881in}}%
\pgfpathlineto{\pgfqpoint{4.175502in}{4.600159in}}%
\pgfpathlineto{\pgfqpoint{4.182484in}{4.600159in}}%
\pgfpathlineto{\pgfqpoint{4.182484in}{4.590512in}}%
\pgfpathlineto{\pgfqpoint{4.189466in}{4.590512in}}%
\pgfpathlineto{\pgfqpoint{4.189466in}{4.619454in}}%
\pgfpathlineto{\pgfqpoint{4.196448in}{4.619454in}}%
\pgfpathlineto{\pgfqpoint{4.196448in}{4.615319in}}%
\pgfpathlineto{\pgfqpoint{4.203430in}{4.615319in}}%
\pgfpathlineto{\pgfqpoint{4.203430in}{4.630479in}}%
\pgfpathlineto{\pgfqpoint{4.210413in}{4.630479in}}%
\pgfpathlineto{\pgfqpoint{4.210413in}{4.635303in}}%
\pgfpathlineto{\pgfqpoint{4.217395in}{4.635303in}}%
\pgfpathlineto{\pgfqpoint{4.217395in}{4.659421in}}%
\pgfpathlineto{\pgfqpoint{4.224377in}{4.659421in}}%
\pgfpathlineto{\pgfqpoint{4.224377in}{4.687673in}}%
\pgfpathlineto{\pgfqpoint{4.231359in}{4.687673in}}%
\pgfpathlineto{\pgfqpoint{4.231359in}{4.690430in}}%
\pgfpathlineto{\pgfqpoint{4.238342in}{4.690430in}}%
\pgfpathlineto{\pgfqpoint{4.238342in}{4.743489in}}%
\pgfpathlineto{\pgfqpoint{4.252306in}{4.744178in}}%
\pgfpathlineto{\pgfqpoint{4.252306in}{4.756582in}}%
\pgfpathlineto{\pgfqpoint{4.259288in}{4.756582in}}%
\pgfpathlineto{\pgfqpoint{4.259288in}{4.781389in}}%
\pgfpathlineto{\pgfqpoint{4.266270in}{4.781389in}}%
\pgfpathlineto{\pgfqpoint{4.266270in}{4.766918in}}%
\pgfpathlineto{\pgfqpoint{4.273253in}{4.766918in}}%
\pgfpathlineto{\pgfqpoint{4.273253in}{4.824113in}}%
\pgfpathlineto{\pgfqpoint{4.280235in}{4.824113in}}%
\pgfpathlineto{\pgfqpoint{4.280235in}{4.868214in}}%
\pgfpathlineto{\pgfqpoint{4.287217in}{4.868214in}}%
\pgfpathlineto{\pgfqpoint{4.287217in}{4.877861in}}%
\pgfpathlineto{\pgfqpoint{4.294199in}{4.877861in}}%
\pgfpathlineto{\pgfqpoint{4.294199in}{4.916450in}}%
\pgfpathlineto{\pgfqpoint{4.301182in}{4.916450in}}%
\pgfpathlineto{\pgfqpoint{4.301182in}{4.948837in}}%
\pgfpathlineto{\pgfqpoint{4.308164in}{4.948837in}}%
\pgfpathlineto{\pgfqpoint{4.308164in}{4.928165in}}%
\pgfpathlineto{\pgfqpoint{4.315146in}{4.928165in}}%
\pgfpathlineto{\pgfqpoint{4.315146in}{4.980535in}}%
\pgfpathlineto{\pgfqpoint{4.322128in}{4.980535in}}%
\pgfpathlineto{\pgfqpoint{4.322128in}{5.035662in}}%
\pgfpathlineto{\pgfqpoint{4.329110in}{5.035662in}}%
\pgfpathlineto{\pgfqpoint{4.329110in}{5.025326in}}%
\pgfpathlineto{\pgfqpoint{4.336093in}{5.025326in}}%
\pgfpathlineto{\pgfqpoint{4.336093in}{5.094924in}}%
\pgfpathlineto{\pgfqpoint{4.343075in}{5.094924in}}%
\pgfpathlineto{\pgfqpoint{4.343075in}{5.101125in}}%
\pgfpathlineto{\pgfqpoint{4.350057in}{5.101125in}}%
\pgfpathlineto{\pgfqpoint{4.350057in}{5.156941in}}%
\pgfpathlineto{\pgfqpoint{4.357039in}{5.156941in}}%
\pgfpathlineto{\pgfqpoint{4.357039in}{5.152807in}}%
\pgfpathlineto{\pgfqpoint{4.364022in}{5.152807in}}%
\pgfpathlineto{\pgfqpoint{4.364022in}{5.201043in}}%
\pgfpathlineto{\pgfqpoint{4.371004in}{5.201043in}}%
\pgfpathlineto{\pgfqpoint{4.371004in}{5.203110in}}%
\pgfpathlineto{\pgfqpoint{4.377986in}{5.203110in}}%
\pgfpathlineto{\pgfqpoint{4.377986in}{5.243077in}}%
\pgfpathlineto{\pgfqpoint{4.384968in}{5.243077in}}%
\pgfpathlineto{\pgfqpoint{4.384968in}{5.265128in}}%
\pgfpathlineto{\pgfqpoint{4.391951in}{5.265128in}}%
\pgfpathlineto{\pgfqpoint{4.391951in}{5.293380in}}%
\pgfpathlineto{\pgfqpoint{4.398933in}{5.293380in}}%
\pgfpathlineto{\pgfqpoint{4.398933in}{5.378827in}}%
\pgfpathlineto{\pgfqpoint{4.405915in}{5.378827in}}%
\pgfpathlineto{\pgfqpoint{4.405915in}{5.414660in}}%
\pgfpathlineto{\pgfqpoint{4.412897in}{5.414660in}}%
\pgfpathlineto{\pgfqpoint{4.412897in}{5.358844in}}%
\pgfpathlineto{\pgfqpoint{4.419879in}{5.358844in}}%
\pgfpathlineto{\pgfqpoint{4.419879in}{5.401567in}}%
\pgfpathlineto{\pgfqpoint{4.426862in}{5.401567in}}%
\pgfpathlineto{\pgfqpoint{4.426862in}{5.470476in}}%
\pgfpathlineto{\pgfqpoint{4.440826in}{5.471165in}}%
\pgfpathlineto{\pgfqpoint{4.440826in}{5.476677in}}%
\pgfpathlineto{\pgfqpoint{4.447808in}{5.476677in}}%
\pgfpathlineto{\pgfqpoint{4.447808in}{5.468408in}}%
\pgfpathlineto{\pgfqpoint{4.454791in}{5.468408in}}%
\pgfpathlineto{\pgfqpoint{4.454791in}{5.503552in}}%
\pgfpathlineto{\pgfqpoint{4.461773in}{5.503552in}}%
\pgfpathlineto{\pgfqpoint{4.461773in}{5.474610in}}%
\pgfpathlineto{\pgfqpoint{4.468755in}{5.474610in}}%
\pgfpathlineto{\pgfqpoint{4.468755in}{5.513199in}}%
\pgfpathlineto{\pgfqpoint{4.475737in}{5.513199in}}%
\pgfpathlineto{\pgfqpoint{4.475737in}{5.531804in}}%
\pgfpathlineto{\pgfqpoint{4.482719in}{5.531804in}}%
\pgfpathlineto{\pgfqpoint{4.482719in}{5.446358in}}%
\pgfpathlineto{\pgfqpoint{4.489702in}{5.446358in}}%
\pgfpathlineto{\pgfqpoint{4.489702in}{5.447736in}}%
\pgfpathlineto{\pgfqpoint{4.496684in}{5.447736in}}%
\pgfpathlineto{\pgfqpoint{4.496684in}{5.449803in}}%
\pgfpathlineto{\pgfqpoint{4.503666in}{5.449803in}}%
\pgfpathlineto{\pgfqpoint{4.503666in}{5.393298in}}%
\pgfpathlineto{\pgfqpoint{4.510648in}{5.393298in}}%
\pgfpathlineto{\pgfqpoint{4.510648in}{5.420172in}}%
\pgfpathlineto{\pgfqpoint{4.517631in}{5.420172in}}%
\pgfpathlineto{\pgfqpoint{4.517631in}{5.410525in}}%
\pgfpathlineto{\pgfqpoint{4.524613in}{5.410525in}}%
\pgfpathlineto{\pgfqpoint{4.524613in}{5.371936in}}%
\pgfpathlineto{\pgfqpoint{4.531595in}{5.371936in}}%
\pgfpathlineto{\pgfqpoint{4.531595in}{5.331969in}}%
\pgfpathlineto{\pgfqpoint{4.538577in}{5.331969in}}%
\pgfpathlineto{\pgfqpoint{4.538577in}{5.309229in}}%
\pgfpathlineto{\pgfqpoint{4.545559in}{5.309229in}}%
\pgfpathlineto{\pgfqpoint{4.545559in}{5.301649in}}%
\pgfpathlineto{\pgfqpoint{4.552542in}{5.301649in}}%
\pgfpathlineto{\pgfqpoint{4.552542in}{5.271330in}}%
\pgfpathlineto{\pgfqpoint{4.559524in}{5.271330in}}%
\pgfpathlineto{\pgfqpoint{4.559524in}{5.261682in}}%
\pgfpathlineto{\pgfqpoint{4.566506in}{5.261682in}}%
\pgfpathlineto{\pgfqpoint{4.566506in}{5.194841in}}%
\pgfpathlineto{\pgfqpoint{4.573488in}{5.194841in}}%
\pgfpathlineto{\pgfqpoint{4.573488in}{5.177614in}}%
\pgfpathlineto{\pgfqpoint{4.580471in}{5.177614in}}%
\pgfpathlineto{\pgfqpoint{4.580471in}{5.167278in}}%
\pgfpathlineto{\pgfqpoint{4.587453in}{5.167278in}}%
\pgfpathlineto{\pgfqpoint{4.587453in}{5.120420in}}%
\pgfpathlineto{\pgfqpoint{4.594435in}{5.120420in}}%
\pgfpathlineto{\pgfqpoint{4.594435in}{5.101125in}}%
\pgfpathlineto{\pgfqpoint{4.601417in}{5.101125in}}%
\pgfpathlineto{\pgfqpoint{4.601417in}{5.037040in}}%
\pgfpathlineto{\pgfqpoint{4.608399in}{5.037040in}}%
\pgfpathlineto{\pgfqpoint{4.608399in}{5.028771in}}%
\pgfpathlineto{\pgfqpoint{4.615382in}{5.028771in}}%
\pgfpathlineto{\pgfqpoint{4.615382in}{5.006720in}}%
\pgfpathlineto{\pgfqpoint{4.622364in}{5.006720in}}%
\pgfpathlineto{\pgfqpoint{4.622364in}{4.935745in}}%
\pgfpathlineto{\pgfqpoint{4.629346in}{4.935745in}}%
\pgfpathlineto{\pgfqpoint{4.629346in}{4.932299in}}%
\pgfpathlineto{\pgfqpoint{4.636328in}{4.932299in}}%
\pgfpathlineto{\pgfqpoint{4.636328in}{4.904047in}}%
\pgfpathlineto{\pgfqpoint{4.643311in}{4.904047in}}%
\pgfpathlineto{\pgfqpoint{4.643311in}{4.908181in}}%
\pgfpathlineto{\pgfqpoint{4.650293in}{4.908181in}}%
\pgfpathlineto{\pgfqpoint{4.650293in}{4.833071in}}%
\pgfpathlineto{\pgfqpoint{4.657275in}{4.833071in}}%
\pgfpathlineto{\pgfqpoint{4.657275in}{4.857189in}}%
\pgfpathlineto{\pgfqpoint{4.664257in}{4.857189in}}%
\pgfpathlineto{\pgfqpoint{4.664257in}{4.784145in}}%
\pgfpathlineto{\pgfqpoint{4.671240in}{4.784145in}}%
\pgfpathlineto{\pgfqpoint{4.671240in}{4.751758in}}%
\pgfpathlineto{\pgfqpoint{4.678222in}{4.751758in}}%
\pgfpathlineto{\pgfqpoint{4.678222in}{4.741422in}}%
\pgfpathlineto{\pgfqpoint{4.692186in}{4.741422in}}%
\pgfpathlineto{\pgfqpoint{4.692186in}{4.728329in}}%
\pgfpathlineto{\pgfqpoint{4.699168in}{4.728329in}}%
\pgfpathlineto{\pgfqpoint{4.699168in}{4.714548in}}%
\pgfpathlineto{\pgfqpoint{4.706151in}{4.714548in}}%
\pgfpathlineto{\pgfqpoint{4.706151in}{4.674581in}}%
\pgfpathlineto{\pgfqpoint{4.713133in}{4.674581in}}%
\pgfpathlineto{\pgfqpoint{4.713133in}{4.651152in}}%
\pgfpathlineto{\pgfqpoint{4.720115in}{4.651152in}}%
\pgfpathlineto{\pgfqpoint{4.720115in}{4.641505in}}%
\pgfpathlineto{\pgfqpoint{4.727097in}{4.641505in}}%
\pgfpathlineto{\pgfqpoint{4.727097in}{4.627723in}}%
\pgfpathlineto{\pgfqpoint{4.734080in}{4.627723in}}%
\pgfpathlineto{\pgfqpoint{4.734080in}{4.613252in}}%
\pgfpathlineto{\pgfqpoint{4.741062in}{4.613252in}}%
\pgfpathlineto{\pgfqpoint{4.741062in}{4.597403in}}%
\pgfpathlineto{\pgfqpoint{4.748044in}{4.597403in}}%
\pgfpathlineto{\pgfqpoint{4.748044in}{4.594647in}}%
\pgfpathlineto{\pgfqpoint{4.755026in}{4.594647in}}%
\pgfpathlineto{\pgfqpoint{4.755026in}{4.580176in}}%
\pgfpathlineto{\pgfqpoint{4.762008in}{4.580176in}}%
\pgfpathlineto{\pgfqpoint{4.762008in}{4.577420in}}%
\pgfpathlineto{\pgfqpoint{4.768991in}{4.577420in}}%
\pgfpathlineto{\pgfqpoint{4.768991in}{4.563638in}}%
\pgfpathlineto{\pgfqpoint{4.775973in}{4.563638in}}%
\pgfpathlineto{\pgfqpoint{4.775973in}{4.534696in}}%
\pgfpathlineto{\pgfqpoint{4.782955in}{4.534696in}}%
\pgfpathlineto{\pgfqpoint{4.782955in}{4.538831in}}%
\pgfpathlineto{\pgfqpoint{4.789937in}{4.538831in}}%
\pgfpathlineto{\pgfqpoint{4.789937in}{4.520225in}}%
\pgfpathlineto{\pgfqpoint{4.796920in}{4.520225in}}%
\pgfpathlineto{\pgfqpoint{4.796920in}{4.505755in}}%
\pgfpathlineto{\pgfqpoint{4.803902in}{4.505755in}}%
\pgfpathlineto{\pgfqpoint{4.803902in}{4.502309in}}%
\pgfpathlineto{\pgfqpoint{4.810884in}{4.502309in}}%
\pgfpathlineto{\pgfqpoint{4.810884in}{4.506444in}}%
\pgfpathlineto{\pgfqpoint{4.817866in}{4.506444in}}%
\pgfpathlineto{\pgfqpoint{4.817866in}{4.497485in}}%
\pgfpathlineto{\pgfqpoint{4.824848in}{4.497485in}}%
\pgfpathlineto{\pgfqpoint{4.824848in}{4.491284in}}%
\pgfpathlineto{\pgfqpoint{4.831831in}{4.491284in}}%
\pgfpathlineto{\pgfqpoint{4.831831in}{4.474746in}}%
\pgfpathlineto{\pgfqpoint{4.838813in}{4.474746in}}%
\pgfpathlineto{\pgfqpoint{4.838813in}{4.491973in}}%
\pgfpathlineto{\pgfqpoint{4.845795in}{4.491973in}}%
\pgfpathlineto{\pgfqpoint{4.845795in}{4.480947in}}%
\pgfpathlineto{\pgfqpoint{4.852777in}{4.480947in}}%
\pgfpathlineto{\pgfqpoint{4.852777in}{4.474746in}}%
\pgfpathlineto{\pgfqpoint{4.859760in}{4.474746in}}%
\pgfpathlineto{\pgfqpoint{4.859760in}{4.467855in}}%
\pgfpathlineto{\pgfqpoint{4.866742in}{4.467855in}}%
\pgfpathlineto{\pgfqpoint{4.866742in}{4.476124in}}%
\pgfpathlineto{\pgfqpoint{4.873724in}{4.476124in}}%
\pgfpathlineto{\pgfqpoint{4.873724in}{4.461653in}}%
\pgfpathlineto{\pgfqpoint{4.880706in}{4.461653in}}%
\pgfpathlineto{\pgfqpoint{4.880706in}{4.458208in}}%
\pgfpathlineto{\pgfqpoint{4.887688in}{4.458208in}}%
\pgfpathlineto{\pgfqpoint{4.887688in}{4.469922in}}%
\pgfpathlineto{\pgfqpoint{4.894671in}{4.469922in}}%
\pgfpathlineto{\pgfqpoint{4.894671in}{4.458897in}}%
\pgfpathlineto{\pgfqpoint{4.901653in}{4.458897in}}%
\pgfpathlineto{\pgfqpoint{4.901653in}{4.463031in}}%
\pgfpathlineto{\pgfqpoint{4.908635in}{4.463031in}}%
\pgfpathlineto{\pgfqpoint{4.908635in}{4.456140in}}%
\pgfpathlineto{\pgfqpoint{4.922600in}{4.455451in}}%
\pgfpathlineto{\pgfqpoint{4.922600in}{4.453384in}}%
\pgfpathlineto{\pgfqpoint{4.929582in}{4.453384in}}%
\pgfpathlineto{\pgfqpoint{4.929582in}{4.451317in}}%
\pgfpathlineto{\pgfqpoint{4.943546in}{4.451317in}}%
\pgfpathlineto{\pgfqpoint{4.943546in}{4.449249in}}%
\pgfpathlineto{\pgfqpoint{4.950529in}{4.449249in}}%
\pgfpathlineto{\pgfqpoint{4.950529in}{4.452006in}}%
\pgfpathlineto{\pgfqpoint{4.964493in}{4.451317in}}%
\pgfpathlineto{\pgfqpoint{4.964493in}{4.450628in}}%
\pgfpathlineto{\pgfqpoint{4.971475in}{4.450628in}}%
\pgfpathlineto{\pgfqpoint{4.971475in}{4.447871in}}%
\pgfpathlineto{\pgfqpoint{4.978457in}{4.447871in}}%
\pgfpathlineto{\pgfqpoint{4.978457in}{4.446493in}}%
\pgfpathlineto{\pgfqpoint{4.992422in}{4.447182in}}%
\pgfpathlineto{\pgfqpoint{4.992422in}{4.448560in}}%
\pgfpathlineto{\pgfqpoint{5.006386in}{4.448560in}}%
\pgfpathlineto{\pgfqpoint{5.006386in}{4.445804in}}%
\pgfpathlineto{\pgfqpoint{5.013369in}{4.445804in}}%
\pgfpathlineto{\pgfqpoint{5.013369in}{4.444426in}}%
\pgfpathlineto{\pgfqpoint{5.020351in}{4.444426in}}%
\pgfpathlineto{\pgfqpoint{5.020351in}{4.444426in}}%
\pgfusepath{stroke}%
\end{pgfscope}%
\begin{pgfscope}%
\pgfsetrectcap%
\pgfsetmiterjoin%
\pgfsetlinewidth{0.803000pt}%
\definecolor{currentstroke}{rgb}{0.000000,0.000000,0.000000}%
\pgfsetstrokecolor{currentstroke}%
\pgfsetdash{}{0pt}%
\pgfpathmoveto{\pgfqpoint{3.510740in}{4.444426in}}%
\pgfpathlineto{\pgfqpoint{3.510740in}{5.862667in}}%
\pgfusepath{stroke}%
\end{pgfscope}%
\begin{pgfscope}%
\pgfsetrectcap%
\pgfsetmiterjoin%
\pgfsetlinewidth{0.803000pt}%
\definecolor{currentstroke}{rgb}{0.000000,0.000000,0.000000}%
\pgfsetstrokecolor{currentstroke}%
\pgfsetdash{}{0pt}%
\pgfpathmoveto{\pgfqpoint{6.051200in}{4.444426in}}%
\pgfpathlineto{\pgfqpoint{6.051200in}{5.862667in}}%
\pgfusepath{stroke}%
\end{pgfscope}%
\begin{pgfscope}%
\pgfsetrectcap%
\pgfsetmiterjoin%
\pgfsetlinewidth{0.803000pt}%
\definecolor{currentstroke}{rgb}{0.000000,0.000000,0.000000}%
\pgfsetstrokecolor{currentstroke}%
\pgfsetdash{}{0pt}%
\pgfpathmoveto{\pgfqpoint{3.510740in}{4.444426in}}%
\pgfpathlineto{\pgfqpoint{6.051200in}{4.444426in}}%
\pgfusepath{stroke}%
\end{pgfscope}%
\begin{pgfscope}%
\pgfsetrectcap%
\pgfsetmiterjoin%
\pgfsetlinewidth{0.803000pt}%
\definecolor{currentstroke}{rgb}{0.000000,0.000000,0.000000}%
\pgfsetstrokecolor{currentstroke}%
\pgfsetdash{}{0pt}%
\pgfpathmoveto{\pgfqpoint{3.510740in}{5.862667in}}%
\pgfpathlineto{\pgfqpoint{6.051200in}{5.862667in}}%
\pgfusepath{stroke}%
\end{pgfscope}%
\begin{pgfscope}%
\definecolor{textcolor}{rgb}{0.000000,0.000000,0.000000}%
\pgfsetstrokecolor{textcolor}%
\pgfsetfillcolor{textcolor}%
\pgftext[x=3.510740in,y=5.946000in,left,base]{\color{textcolor}\rmfamily\fontsize{10.000000}{12.000000}\selectfont Bin [0.67, 0.83), 76,317 events}%
\end{pgfscope}%
\begin{pgfscope}%
\pgfsetbuttcap%
\pgfsetmiterjoin%
\definecolor{currentfill}{rgb}{1.000000,1.000000,1.000000}%
\pgfsetfillcolor{currentfill}%
\pgfsetfillopacity{0.800000}%
\pgfsetlinewidth{1.003750pt}%
\definecolor{currentstroke}{rgb}{0.800000,0.800000,0.800000}%
\pgfsetstrokecolor{currentstroke}%
\pgfsetstrokeopacity{0.800000}%
\pgfsetdash{}{0pt}%
\pgfpathmoveto{\pgfqpoint{5.016422in}{5.463111in}}%
\pgfpathlineto{\pgfqpoint{5.973422in}{5.463111in}}%
\pgfpathquadraticcurveto{\pgfqpoint{5.995644in}{5.463111in}}{\pgfqpoint{5.995644in}{5.485333in}}%
\pgfpathlineto{\pgfqpoint{5.995644in}{5.784889in}}%
\pgfpathquadraticcurveto{\pgfqpoint{5.995644in}{5.807111in}}{\pgfqpoint{5.973422in}{5.807111in}}%
\pgfpathlineto{\pgfqpoint{5.016422in}{5.807111in}}%
\pgfpathquadraticcurveto{\pgfqpoint{4.994200in}{5.807111in}}{\pgfqpoint{4.994200in}{5.784889in}}%
\pgfpathlineto{\pgfqpoint{4.994200in}{5.485333in}}%
\pgfpathquadraticcurveto{\pgfqpoint{4.994200in}{5.463111in}}{\pgfqpoint{5.016422in}{5.463111in}}%
\pgfpathclose%
\pgfusepath{stroke,fill}%
\end{pgfscope}%
\begin{pgfscope}%
\pgfsetbuttcap%
\pgfsetmiterjoin%
\pgfsetlinewidth{1.003750pt}%
\definecolor{currentstroke}{rgb}{0.313725,0.317647,0.309804}%
\pgfsetstrokecolor{currentstroke}%
\pgfsetdash{}{0pt}%
\pgfpathmoveto{\pgfqpoint{5.038644in}{5.684444in}}%
\pgfpathlineto{\pgfqpoint{5.260867in}{5.684444in}}%
\pgfpathlineto{\pgfqpoint{5.260867in}{5.762222in}}%
\pgfpathlineto{\pgfqpoint{5.038644in}{5.762222in}}%
\pgfpathclose%
\pgfusepath{stroke}%
\end{pgfscope}%
\begin{pgfscope}%
\definecolor{textcolor}{rgb}{0.000000,0.000000,0.000000}%
\pgfsetstrokecolor{textcolor}%
\pgfsetfillcolor{textcolor}%
\pgftext[x=5.349756in,y=5.684444in,left,base]{\color{textcolor}\rmfamily\fontsize{8.000000}{9.600000}\selectfont IQR = 0.16}%
\end{pgfscope}%
\begin{pgfscope}%
\pgfsetbuttcap%
\pgfsetmiterjoin%
\pgfsetlinewidth{1.003750pt}%
\definecolor{currentstroke}{rgb}{0.949020,0.372549,0.360784}%
\pgfsetstrokecolor{currentstroke}%
\pgfsetdash{{1.000000pt}{1.650000pt}}{0.000000pt}%
\pgfpathmoveto{\pgfqpoint{5.038644in}{5.529111in}}%
\pgfpathlineto{\pgfqpoint{5.260867in}{5.529111in}}%
\pgfpathlineto{\pgfqpoint{5.260867in}{5.606889in}}%
\pgfpathlineto{\pgfqpoint{5.038644in}{5.606889in}}%
\pgfpathclose%
\pgfusepath{stroke}%
\end{pgfscope}%
\begin{pgfscope}%
\definecolor{textcolor}{rgb}{0.000000,0.000000,0.000000}%
\pgfsetstrokecolor{textcolor}%
\pgfsetfillcolor{textcolor}%
\pgftext[x=5.349756in,y=5.529111in,left,base]{\color{textcolor}\rmfamily\fontsize{8.000000}{9.600000}\selectfont IQR = 0.21}%
\end{pgfscope}%
\begin{pgfscope}%
\pgfsetbuttcap%
\pgfsetmiterjoin%
\definecolor{currentfill}{rgb}{1.000000,1.000000,1.000000}%
\pgfsetfillcolor{currentfill}%
\pgfsetlinewidth{0.000000pt}%
\definecolor{currentstroke}{rgb}{0.000000,0.000000,0.000000}%
\pgfsetstrokecolor{currentstroke}%
\pgfsetstrokeopacity{0.000000}%
\pgfsetdash{}{0pt}%
\pgfpathmoveto{\pgfqpoint{0.485140in}{2.492796in}}%
\pgfpathlineto{\pgfqpoint{3.025600in}{2.492796in}}%
\pgfpathlineto{\pgfqpoint{3.025600in}{3.911037in}}%
\pgfpathlineto{\pgfqpoint{0.485140in}{3.911037in}}%
\pgfpathclose%
\pgfusepath{fill}%
\end{pgfscope}%
\begin{pgfscope}%
\pgfsetbuttcap%
\pgfsetroundjoin%
\definecolor{currentfill}{rgb}{0.000000,0.000000,0.000000}%
\pgfsetfillcolor{currentfill}%
\pgfsetlinewidth{0.803000pt}%
\definecolor{currentstroke}{rgb}{0.000000,0.000000,0.000000}%
\pgfsetstrokecolor{currentstroke}%
\pgfsetdash{}{0pt}%
\pgfsys@defobject{currentmarker}{\pgfqpoint{0.000000in}{-0.048611in}}{\pgfqpoint{0.000000in}{0.000000in}}{%
\pgfpathmoveto{\pgfqpoint{0.000000in}{0.000000in}}%
\pgfpathlineto{\pgfqpoint{0.000000in}{-0.048611in}}%
\pgfusepath{stroke,fill}%
}%
\begin{pgfscope}%
\pgfsys@transformshift{0.916345in}{2.492796in}%
\pgfsys@useobject{currentmarker}{}%
\end{pgfscope}%
\end{pgfscope}%
\begin{pgfscope}%
\definecolor{textcolor}{rgb}{0.000000,0.000000,0.000000}%
\pgfsetstrokecolor{textcolor}%
\pgfsetfillcolor{textcolor}%
\pgftext[x=0.916345in,y=2.395574in,,top]{\color{textcolor}\rmfamily\fontsize{8.000000}{9.600000}\selectfont \(\displaystyle {-1}\)}%
\end{pgfscope}%
\begin{pgfscope}%
\pgfsetbuttcap%
\pgfsetroundjoin%
\definecolor{currentfill}{rgb}{0.000000,0.000000,0.000000}%
\pgfsetfillcolor{currentfill}%
\pgfsetlinewidth{0.803000pt}%
\definecolor{currentstroke}{rgb}{0.000000,0.000000,0.000000}%
\pgfsetstrokecolor{currentstroke}%
\pgfsetdash{}{0pt}%
\pgfsys@defobject{currentmarker}{\pgfqpoint{0.000000in}{-0.048611in}}{\pgfqpoint{0.000000in}{0.000000in}}{%
\pgfpathmoveto{\pgfqpoint{0.000000in}{0.000000in}}%
\pgfpathlineto{\pgfqpoint{0.000000in}{-0.048611in}}%
\pgfusepath{stroke,fill}%
}%
\begin{pgfscope}%
\pgfsys@transformshift{1.602799in}{2.492796in}%
\pgfsys@useobject{currentmarker}{}%
\end{pgfscope}%
\end{pgfscope}%
\begin{pgfscope}%
\definecolor{textcolor}{rgb}{0.000000,0.000000,0.000000}%
\pgfsetstrokecolor{textcolor}%
\pgfsetfillcolor{textcolor}%
\pgftext[x=1.602799in,y=2.395574in,,top]{\color{textcolor}\rmfamily\fontsize{8.000000}{9.600000}\selectfont \(\displaystyle {0}\)}%
\end{pgfscope}%
\begin{pgfscope}%
\pgfsetbuttcap%
\pgfsetroundjoin%
\definecolor{currentfill}{rgb}{0.000000,0.000000,0.000000}%
\pgfsetfillcolor{currentfill}%
\pgfsetlinewidth{0.803000pt}%
\definecolor{currentstroke}{rgb}{0.000000,0.000000,0.000000}%
\pgfsetstrokecolor{currentstroke}%
\pgfsetdash{}{0pt}%
\pgfsys@defobject{currentmarker}{\pgfqpoint{0.000000in}{-0.048611in}}{\pgfqpoint{0.000000in}{0.000000in}}{%
\pgfpathmoveto{\pgfqpoint{0.000000in}{0.000000in}}%
\pgfpathlineto{\pgfqpoint{0.000000in}{-0.048611in}}%
\pgfusepath{stroke,fill}%
}%
\begin{pgfscope}%
\pgfsys@transformshift{2.289253in}{2.492796in}%
\pgfsys@useobject{currentmarker}{}%
\end{pgfscope}%
\end{pgfscope}%
\begin{pgfscope}%
\definecolor{textcolor}{rgb}{0.000000,0.000000,0.000000}%
\pgfsetstrokecolor{textcolor}%
\pgfsetfillcolor{textcolor}%
\pgftext[x=2.289253in,y=2.395574in,,top]{\color{textcolor}\rmfamily\fontsize{8.000000}{9.600000}\selectfont \(\displaystyle {1}\)}%
\end{pgfscope}%
\begin{pgfscope}%
\pgfsetbuttcap%
\pgfsetroundjoin%
\definecolor{currentfill}{rgb}{0.000000,0.000000,0.000000}%
\pgfsetfillcolor{currentfill}%
\pgfsetlinewidth{0.803000pt}%
\definecolor{currentstroke}{rgb}{0.000000,0.000000,0.000000}%
\pgfsetstrokecolor{currentstroke}%
\pgfsetdash{}{0pt}%
\pgfsys@defobject{currentmarker}{\pgfqpoint{0.000000in}{-0.048611in}}{\pgfqpoint{0.000000in}{0.000000in}}{%
\pgfpathmoveto{\pgfqpoint{0.000000in}{0.000000in}}%
\pgfpathlineto{\pgfqpoint{0.000000in}{-0.048611in}}%
\pgfusepath{stroke,fill}%
}%
\begin{pgfscope}%
\pgfsys@transformshift{2.975706in}{2.492796in}%
\pgfsys@useobject{currentmarker}{}%
\end{pgfscope}%
\end{pgfscope}%
\begin{pgfscope}%
\definecolor{textcolor}{rgb}{0.000000,0.000000,0.000000}%
\pgfsetstrokecolor{textcolor}%
\pgfsetfillcolor{textcolor}%
\pgftext[x=2.975706in,y=2.395574in,,top]{\color{textcolor}\rmfamily\fontsize{8.000000}{9.600000}\selectfont \(\displaystyle {2}\)}%
\end{pgfscope}%
\begin{pgfscope}%
\pgfsetbuttcap%
\pgfsetroundjoin%
\definecolor{currentfill}{rgb}{0.000000,0.000000,0.000000}%
\pgfsetfillcolor{currentfill}%
\pgfsetlinewidth{0.803000pt}%
\definecolor{currentstroke}{rgb}{0.000000,0.000000,0.000000}%
\pgfsetstrokecolor{currentstroke}%
\pgfsetdash{}{0pt}%
\pgfsys@defobject{currentmarker}{\pgfqpoint{-0.048611in}{0.000000in}}{\pgfqpoint{-0.000000in}{0.000000in}}{%
\pgfpathmoveto{\pgfqpoint{-0.000000in}{0.000000in}}%
\pgfpathlineto{\pgfqpoint{-0.048611in}{0.000000in}}%
\pgfusepath{stroke,fill}%
}%
\begin{pgfscope}%
\pgfsys@transformshift{0.485140in}{2.492796in}%
\pgfsys@useobject{currentmarker}{}%
\end{pgfscope}%
\end{pgfscope}%
\begin{pgfscope}%
\definecolor{textcolor}{rgb}{0.000000,0.000000,0.000000}%
\pgfsetstrokecolor{textcolor}%
\pgfsetfillcolor{textcolor}%
\pgftext[x=0.328889in, y=2.454241in, left, base]{\color{textcolor}\rmfamily\fontsize{8.000000}{9.600000}\selectfont \(\displaystyle {0}\)}%
\end{pgfscope}%
\begin{pgfscope}%
\pgfsetbuttcap%
\pgfsetroundjoin%
\definecolor{currentfill}{rgb}{0.000000,0.000000,0.000000}%
\pgfsetfillcolor{currentfill}%
\pgfsetlinewidth{0.803000pt}%
\definecolor{currentstroke}{rgb}{0.000000,0.000000,0.000000}%
\pgfsetstrokecolor{currentstroke}%
\pgfsetdash{}{0pt}%
\pgfsys@defobject{currentmarker}{\pgfqpoint{-0.048611in}{0.000000in}}{\pgfqpoint{-0.000000in}{0.000000in}}{%
\pgfpathmoveto{\pgfqpoint{-0.000000in}{0.000000in}}%
\pgfpathlineto{\pgfqpoint{-0.048611in}{0.000000in}}%
\pgfusepath{stroke,fill}%
}%
\begin{pgfscope}%
\pgfsys@transformshift{0.485140in}{3.032434in}%
\pgfsys@useobject{currentmarker}{}%
\end{pgfscope}%
\end{pgfscope}%
\begin{pgfscope}%
\definecolor{textcolor}{rgb}{0.000000,0.000000,0.000000}%
\pgfsetstrokecolor{textcolor}%
\pgfsetfillcolor{textcolor}%
\pgftext[x=0.328889in, y=2.993879in, left, base]{\color{textcolor}\rmfamily\fontsize{8.000000}{9.600000}\selectfont \(\displaystyle {1}\)}%
\end{pgfscope}%
\begin{pgfscope}%
\pgfsetbuttcap%
\pgfsetroundjoin%
\definecolor{currentfill}{rgb}{0.000000,0.000000,0.000000}%
\pgfsetfillcolor{currentfill}%
\pgfsetlinewidth{0.803000pt}%
\definecolor{currentstroke}{rgb}{0.000000,0.000000,0.000000}%
\pgfsetstrokecolor{currentstroke}%
\pgfsetdash{}{0pt}%
\pgfsys@defobject{currentmarker}{\pgfqpoint{-0.048611in}{0.000000in}}{\pgfqpoint{-0.000000in}{0.000000in}}{%
\pgfpathmoveto{\pgfqpoint{-0.000000in}{0.000000in}}%
\pgfpathlineto{\pgfqpoint{-0.048611in}{0.000000in}}%
\pgfusepath{stroke,fill}%
}%
\begin{pgfscope}%
\pgfsys@transformshift{0.485140in}{3.572073in}%
\pgfsys@useobject{currentmarker}{}%
\end{pgfscope}%
\end{pgfscope}%
\begin{pgfscope}%
\definecolor{textcolor}{rgb}{0.000000,0.000000,0.000000}%
\pgfsetstrokecolor{textcolor}%
\pgfsetfillcolor{textcolor}%
\pgftext[x=0.328889in, y=3.533517in, left, base]{\color{textcolor}\rmfamily\fontsize{8.000000}{9.600000}\selectfont \(\displaystyle {2}\)}%
\end{pgfscope}%
\begin{pgfscope}%
\definecolor{textcolor}{rgb}{0.000000,0.000000,0.000000}%
\pgfsetstrokecolor{textcolor}%
\pgfsetfillcolor{textcolor}%
\pgftext[x=0.273333in,y=3.201917in,,bottom,rotate=90.000000]{\color{textcolor}\rmfamily\fontsize{10.000000}{12.000000}\selectfont Density}%
\end{pgfscope}%
\begin{pgfscope}%
\pgfpathrectangle{\pgfqpoint{0.485140in}{2.492796in}}{\pgfqpoint{2.540460in}{1.418241in}}%
\pgfusepath{clip}%
\pgfsetbuttcap%
\pgfsetmiterjoin%
\pgfsetlinewidth{1.003750pt}%
\definecolor{currentstroke}{rgb}{0.313725,0.317647,0.309804}%
\pgfsetstrokecolor{currentstroke}%
\pgfsetdash{}{0pt}%
\pgfpathmoveto{\pgfqpoint{1.040080in}{2.492796in}}%
\pgfpathlineto{\pgfqpoint{1.040080in}{2.493270in}}%
\pgfpathlineto{\pgfqpoint{1.107463in}{2.494217in}}%
\pgfpathlineto{\pgfqpoint{1.107463in}{2.494691in}}%
\pgfpathlineto{\pgfqpoint{1.125840in}{2.494217in}}%
\pgfpathlineto{\pgfqpoint{1.125840in}{2.493270in}}%
\pgfpathlineto{\pgfqpoint{1.144217in}{2.494217in}}%
\pgfpathlineto{\pgfqpoint{1.144217in}{2.498953in}}%
\pgfpathlineto{\pgfqpoint{1.150342in}{2.498953in}}%
\pgfpathlineto{\pgfqpoint{1.150342in}{2.494691in}}%
\pgfpathlineto{\pgfqpoint{1.162594in}{2.495638in}}%
\pgfpathlineto{\pgfqpoint{1.162594in}{2.498953in}}%
\pgfpathlineto{\pgfqpoint{1.211599in}{2.499427in}}%
\pgfpathlineto{\pgfqpoint{1.211599in}{2.500374in}}%
\pgfpathlineto{\pgfqpoint{1.217725in}{2.500374in}}%
\pgfpathlineto{\pgfqpoint{1.217725in}{2.498953in}}%
\pgfpathlineto{\pgfqpoint{1.223850in}{2.498953in}}%
\pgfpathlineto{\pgfqpoint{1.223850in}{2.506057in}}%
\pgfpathlineto{\pgfqpoint{1.242227in}{2.506057in}}%
\pgfpathlineto{\pgfqpoint{1.242227in}{2.507951in}}%
\pgfpathlineto{\pgfqpoint{1.254478in}{2.508898in}}%
\pgfpathlineto{\pgfqpoint{1.254478in}{2.511266in}}%
\pgfpathlineto{\pgfqpoint{1.260604in}{2.511266in}}%
\pgfpathlineto{\pgfqpoint{1.260604in}{2.515055in}}%
\pgfpathlineto{\pgfqpoint{1.266730in}{2.515055in}}%
\pgfpathlineto{\pgfqpoint{1.266730in}{2.513161in}}%
\pgfpathlineto{\pgfqpoint{1.272855in}{2.513161in}}%
\pgfpathlineto{\pgfqpoint{1.272855in}{2.520738in}}%
\pgfpathlineto{\pgfqpoint{1.278981in}{2.520738in}}%
\pgfpathlineto{\pgfqpoint{1.278981in}{2.523106in}}%
\pgfpathlineto{\pgfqpoint{1.285107in}{2.523106in}}%
\pgfpathlineto{\pgfqpoint{1.285107in}{2.530684in}}%
\pgfpathlineto{\pgfqpoint{1.291232in}{2.530684in}}%
\pgfpathlineto{\pgfqpoint{1.291232in}{2.535420in}}%
\pgfpathlineto{\pgfqpoint{1.297358in}{2.535420in}}%
\pgfpathlineto{\pgfqpoint{1.297358in}{2.526895in}}%
\pgfpathlineto{\pgfqpoint{1.303484in}{2.526895in}}%
\pgfpathlineto{\pgfqpoint{1.303484in}{2.531631in}}%
\pgfpathlineto{\pgfqpoint{1.309609in}{2.531631in}}%
\pgfpathlineto{\pgfqpoint{1.309609in}{2.548207in}}%
\pgfpathlineto{\pgfqpoint{1.315735in}{2.548207in}}%
\pgfpathlineto{\pgfqpoint{1.315735in}{2.552470in}}%
\pgfpathlineto{\pgfqpoint{1.321861in}{2.552470in}}%
\pgfpathlineto{\pgfqpoint{1.321861in}{2.554364in}}%
\pgfpathlineto{\pgfqpoint{1.327986in}{2.554364in}}%
\pgfpathlineto{\pgfqpoint{1.327986in}{2.556258in}}%
\pgfpathlineto{\pgfqpoint{1.334112in}{2.556258in}}%
\pgfpathlineto{\pgfqpoint{1.334112in}{2.578044in}}%
\pgfpathlineto{\pgfqpoint{1.340238in}{2.578044in}}%
\pgfpathlineto{\pgfqpoint{1.340238in}{2.581833in}}%
\pgfpathlineto{\pgfqpoint{1.346363in}{2.581833in}}%
\pgfpathlineto{\pgfqpoint{1.346363in}{2.592252in}}%
\pgfpathlineto{\pgfqpoint{1.352489in}{2.592252in}}%
\pgfpathlineto{\pgfqpoint{1.352489in}{2.603145in}}%
\pgfpathlineto{\pgfqpoint{1.358615in}{2.603145in}}%
\pgfpathlineto{\pgfqpoint{1.358615in}{2.607881in}}%
\pgfpathlineto{\pgfqpoint{1.364740in}{2.607881in}}%
\pgfpathlineto{\pgfqpoint{1.364740in}{2.620194in}}%
\pgfpathlineto{\pgfqpoint{1.370866in}{2.620194in}}%
\pgfpathlineto{\pgfqpoint{1.370866in}{2.639612in}}%
\pgfpathlineto{\pgfqpoint{1.376992in}{2.639612in}}%
\pgfpathlineto{\pgfqpoint{1.376992in}{2.648610in}}%
\pgfpathlineto{\pgfqpoint{1.383117in}{2.648610in}}%
\pgfpathlineto{\pgfqpoint{1.383117in}{2.682236in}}%
\pgfpathlineto{\pgfqpoint{1.389243in}{2.682236in}}%
\pgfpathlineto{\pgfqpoint{1.389243in}{2.670396in}}%
\pgfpathlineto{\pgfqpoint{1.395369in}{2.670396in}}%
\pgfpathlineto{\pgfqpoint{1.395369in}{2.706389in}}%
\pgfpathlineto{\pgfqpoint{1.401494in}{2.706389in}}%
\pgfpathlineto{\pgfqpoint{1.401494in}{2.703548in}}%
\pgfpathlineto{\pgfqpoint{1.407620in}{2.703548in}}%
\pgfpathlineto{\pgfqpoint{1.407620in}{2.725333in}}%
\pgfpathlineto{\pgfqpoint{1.413746in}{2.725333in}}%
\pgfpathlineto{\pgfqpoint{1.413746in}{2.766537in}}%
\pgfpathlineto{\pgfqpoint{1.419871in}{2.766537in}}%
\pgfpathlineto{\pgfqpoint{1.419871in}{2.776009in}}%
\pgfpathlineto{\pgfqpoint{1.425997in}{2.776009in}}%
\pgfpathlineto{\pgfqpoint{1.425997in}{2.808687in}}%
\pgfpathlineto{\pgfqpoint{1.432123in}{2.808687in}}%
\pgfpathlineto{\pgfqpoint{1.432123in}{2.852258in}}%
\pgfpathlineto{\pgfqpoint{1.438248in}{2.852258in}}%
\pgfpathlineto{\pgfqpoint{1.438248in}{2.864098in}}%
\pgfpathlineto{\pgfqpoint{1.444374in}{2.864098in}}%
\pgfpathlineto{\pgfqpoint{1.444374in}{2.898671in}}%
\pgfpathlineto{\pgfqpoint{1.450500in}{2.898671in}}%
\pgfpathlineto{\pgfqpoint{1.450500in}{2.917141in}}%
\pgfpathlineto{\pgfqpoint{1.456625in}{2.917141in}}%
\pgfpathlineto{\pgfqpoint{1.456625in}{2.999547in}}%
\pgfpathlineto{\pgfqpoint{1.462751in}{2.999547in}}%
\pgfpathlineto{\pgfqpoint{1.462751in}{3.010440in}}%
\pgfpathlineto{\pgfqpoint{1.468877in}{3.010440in}}%
\pgfpathlineto{\pgfqpoint{1.468877in}{3.067272in}}%
\pgfpathlineto{\pgfqpoint{1.475002in}{3.067272in}}%
\pgfpathlineto{\pgfqpoint{1.475002in}{3.077691in}}%
\pgfpathlineto{\pgfqpoint{1.481128in}{3.077691in}}%
\pgfpathlineto{\pgfqpoint{1.481128in}{3.119842in}}%
\pgfpathlineto{\pgfqpoint{1.487254in}{3.119842in}}%
\pgfpathlineto{\pgfqpoint{1.487254in}{3.196565in}}%
\pgfpathlineto{\pgfqpoint{1.493379in}{3.196565in}}%
\pgfpathlineto{\pgfqpoint{1.493379in}{3.248187in}}%
\pgfpathlineto{\pgfqpoint{1.499505in}{3.248187in}}%
\pgfpathlineto{\pgfqpoint{1.499505in}{3.282286in}}%
\pgfpathlineto{\pgfqpoint{1.505631in}{3.282286in}}%
\pgfpathlineto{\pgfqpoint{1.505631in}{3.291758in}}%
\pgfpathlineto{\pgfqpoint{1.511756in}{3.291758in}}%
\pgfpathlineto{\pgfqpoint{1.511756in}{3.347169in}}%
\pgfpathlineto{\pgfqpoint{1.517882in}{3.347169in}}%
\pgfpathlineto{\pgfqpoint{1.517882in}{3.402107in}}%
\pgfpathlineto{\pgfqpoint{1.524007in}{3.402107in}}%
\pgfpathlineto{\pgfqpoint{1.524007in}{3.404001in}}%
\pgfpathlineto{\pgfqpoint{1.530133in}{3.404001in}}%
\pgfpathlineto{\pgfqpoint{1.530133in}{3.519560in}}%
\pgfpathlineto{\pgfqpoint{1.536259in}{3.519560in}}%
\pgfpathlineto{\pgfqpoint{1.536259in}{3.573550in}}%
\pgfpathlineto{\pgfqpoint{1.542384in}{3.573550in}}%
\pgfpathlineto{\pgfqpoint{1.542384in}{3.586811in}}%
\pgfpathlineto{\pgfqpoint{1.548510in}{3.586811in}}%
\pgfpathlineto{\pgfqpoint{1.548510in}{3.621857in}}%
\pgfpathlineto{\pgfqpoint{1.554636in}{3.621857in}}%
\pgfpathlineto{\pgfqpoint{1.554636in}{3.658324in}}%
\pgfpathlineto{\pgfqpoint{1.560761in}{3.658324in}}%
\pgfpathlineto{\pgfqpoint{1.560761in}{3.694318in}}%
\pgfpathlineto{\pgfqpoint{1.566887in}{3.694318in}}%
\pgfpathlineto{\pgfqpoint{1.566887in}{3.723207in}}%
\pgfpathlineto{\pgfqpoint{1.573013in}{3.723207in}}%
\pgfpathlineto{\pgfqpoint{1.573013in}{3.763463in}}%
\pgfpathlineto{\pgfqpoint{1.579138in}{3.763463in}}%
\pgfpathlineto{\pgfqpoint{1.579138in}{3.792827in}}%
\pgfpathlineto{\pgfqpoint{1.585264in}{3.792827in}}%
\pgfpathlineto{\pgfqpoint{1.585264in}{3.764884in}}%
\pgfpathlineto{\pgfqpoint{1.591390in}{3.764884in}}%
\pgfpathlineto{\pgfqpoint{1.591390in}{3.774830in}}%
\pgfpathlineto{\pgfqpoint{1.597515in}{3.774830in}}%
\pgfpathlineto{\pgfqpoint{1.597515in}{3.795668in}}%
\pgfpathlineto{\pgfqpoint{1.603641in}{3.795668in}}%
\pgfpathlineto{\pgfqpoint{1.603641in}{3.843502in}}%
\pgfpathlineto{\pgfqpoint{1.609767in}{3.843502in}}%
\pgfpathlineto{\pgfqpoint{1.609767in}{3.836398in}}%
\pgfpathlineto{\pgfqpoint{1.615892in}{3.836398in}}%
\pgfpathlineto{\pgfqpoint{1.615892in}{3.768673in}}%
\pgfpathlineto{\pgfqpoint{1.622018in}{3.768673in}}%
\pgfpathlineto{\pgfqpoint{1.622018in}{3.728891in}}%
\pgfpathlineto{\pgfqpoint{1.628144in}{3.728891in}}%
\pgfpathlineto{\pgfqpoint{1.628144in}{3.784775in}}%
\pgfpathlineto{\pgfqpoint{1.634269in}{3.784775in}}%
\pgfpathlineto{\pgfqpoint{1.634269in}{3.796142in}}%
\pgfpathlineto{\pgfqpoint{1.640395in}{3.796142in}}%
\pgfpathlineto{\pgfqpoint{1.640395in}{3.735521in}}%
\pgfpathlineto{\pgfqpoint{1.646521in}{3.735521in}}%
\pgfpathlineto{\pgfqpoint{1.646521in}{3.684846in}}%
\pgfpathlineto{\pgfqpoint{1.652646in}{3.684846in}}%
\pgfpathlineto{\pgfqpoint{1.652646in}{3.659271in}}%
\pgfpathlineto{\pgfqpoint{1.658772in}{3.659271in}}%
\pgfpathlineto{\pgfqpoint{1.658772in}{3.654062in}}%
\pgfpathlineto{\pgfqpoint{1.664898in}{3.654062in}}%
\pgfpathlineto{\pgfqpoint{1.664898in}{3.602913in}}%
\pgfpathlineto{\pgfqpoint{1.671023in}{3.602913in}}%
\pgfpathlineto{\pgfqpoint{1.671023in}{3.535662in}}%
\pgfpathlineto{\pgfqpoint{1.677149in}{3.535662in}}%
\pgfpathlineto{\pgfqpoint{1.677149in}{3.521454in}}%
\pgfpathlineto{\pgfqpoint{1.683275in}{3.521454in}}%
\pgfpathlineto{\pgfqpoint{1.683275in}{3.461307in}}%
\pgfpathlineto{\pgfqpoint{1.689400in}{3.461307in}}%
\pgfpathlineto{\pgfqpoint{1.689400in}{3.476462in}}%
\pgfpathlineto{\pgfqpoint{1.695526in}{3.476462in}}%
\pgfpathlineto{\pgfqpoint{1.695526in}{3.373217in}}%
\pgfpathlineto{\pgfqpoint{1.701652in}{3.373217in}}%
\pgfpathlineto{\pgfqpoint{1.701652in}{3.359483in}}%
\pgfpathlineto{\pgfqpoint{1.707777in}{3.359483in}}%
\pgfpathlineto{\pgfqpoint{1.707777in}{3.346696in}}%
\pgfpathlineto{\pgfqpoint{1.713903in}{3.346696in}}%
\pgfpathlineto{\pgfqpoint{1.713903in}{3.242030in}}%
\pgfpathlineto{\pgfqpoint{1.720029in}{3.242030in}}%
\pgfpathlineto{\pgfqpoint{1.720029in}{3.226875in}}%
\pgfpathlineto{\pgfqpoint{1.726154in}{3.226875in}}%
\pgfpathlineto{\pgfqpoint{1.726154in}{3.164360in}}%
\pgfpathlineto{\pgfqpoint{1.732280in}{3.164360in}}%
\pgfpathlineto{\pgfqpoint{1.732280in}{3.151099in}}%
\pgfpathlineto{\pgfqpoint{1.738406in}{3.151099in}}%
\pgfpathlineto{\pgfqpoint{1.738406in}{3.143522in}}%
\pgfpathlineto{\pgfqpoint{1.744531in}{3.143522in}}%
\pgfpathlineto{\pgfqpoint{1.744531in}{3.113211in}}%
\pgfpathlineto{\pgfqpoint{1.750657in}{3.113211in}}%
\pgfpathlineto{\pgfqpoint{1.750657in}{3.051643in}}%
\pgfpathlineto{\pgfqpoint{1.756783in}{3.051643in}}%
\pgfpathlineto{\pgfqpoint{1.756783in}{3.029384in}}%
\pgfpathlineto{\pgfqpoint{1.762908in}{3.029384in}}%
\pgfpathlineto{\pgfqpoint{1.762908in}{2.981077in}}%
\pgfpathlineto{\pgfqpoint{1.769034in}{2.981077in}}%
\pgfpathlineto{\pgfqpoint{1.769034in}{2.932296in}}%
\pgfpathlineto{\pgfqpoint{1.781285in}{2.933244in}}%
\pgfpathlineto{\pgfqpoint{1.781285in}{2.900092in}}%
\pgfpathlineto{\pgfqpoint{1.787411in}{2.900092in}}%
\pgfpathlineto{\pgfqpoint{1.787411in}{2.865992in}}%
\pgfpathlineto{\pgfqpoint{1.793536in}{2.865992in}}%
\pgfpathlineto{\pgfqpoint{1.793536in}{2.825737in}}%
\pgfpathlineto{\pgfqpoint{1.799662in}{2.825737in}}%
\pgfpathlineto{\pgfqpoint{1.799662in}{2.847522in}}%
\pgfpathlineto{\pgfqpoint{1.805788in}{2.847522in}}%
\pgfpathlineto{\pgfqpoint{1.805788in}{2.801583in}}%
\pgfpathlineto{\pgfqpoint{1.811913in}{2.801583in}}%
\pgfpathlineto{\pgfqpoint{1.811913in}{2.786901in}}%
\pgfpathlineto{\pgfqpoint{1.818039in}{2.786901in}}%
\pgfpathlineto{\pgfqpoint{1.818039in}{2.749961in}}%
\pgfpathlineto{\pgfqpoint{1.824165in}{2.749961in}}%
\pgfpathlineto{\pgfqpoint{1.824165in}{2.740015in}}%
\pgfpathlineto{\pgfqpoint{1.830290in}{2.740015in}}%
\pgfpathlineto{\pgfqpoint{1.830290in}{2.733858in}}%
\pgfpathlineto{\pgfqpoint{1.836416in}{2.733858in}}%
\pgfpathlineto{\pgfqpoint{1.836416in}{2.699285in}}%
\pgfpathlineto{\pgfqpoint{1.842542in}{2.699285in}}%
\pgfpathlineto{\pgfqpoint{1.842542in}{2.707810in}}%
\pgfpathlineto{\pgfqpoint{1.848667in}{2.707810in}}%
\pgfpathlineto{\pgfqpoint{1.848667in}{2.678921in}}%
\pgfpathlineto{\pgfqpoint{1.854793in}{2.678921in}}%
\pgfpathlineto{\pgfqpoint{1.854793in}{2.693602in}}%
\pgfpathlineto{\pgfqpoint{1.860919in}{2.693602in}}%
\pgfpathlineto{\pgfqpoint{1.860919in}{2.655714in}}%
\pgfpathlineto{\pgfqpoint{1.873170in}{2.655241in}}%
\pgfpathlineto{\pgfqpoint{1.873170in}{2.636297in}}%
\pgfpathlineto{\pgfqpoint{1.879296in}{2.636297in}}%
\pgfpathlineto{\pgfqpoint{1.879296in}{2.627772in}}%
\pgfpathlineto{\pgfqpoint{1.885421in}{2.627772in}}%
\pgfpathlineto{\pgfqpoint{1.885421in}{2.615458in}}%
\pgfpathlineto{\pgfqpoint{1.891547in}{2.615458in}}%
\pgfpathlineto{\pgfqpoint{1.891547in}{2.612143in}}%
\pgfpathlineto{\pgfqpoint{1.897673in}{2.612143in}}%
\pgfpathlineto{\pgfqpoint{1.897673in}{2.591305in}}%
\pgfpathlineto{\pgfqpoint{1.909924in}{2.590358in}}%
\pgfpathlineto{\pgfqpoint{1.909924in}{2.578044in}}%
\pgfpathlineto{\pgfqpoint{1.916050in}{2.578044in}}%
\pgfpathlineto{\pgfqpoint{1.916050in}{2.574255in}}%
\pgfpathlineto{\pgfqpoint{1.922175in}{2.574255in}}%
\pgfpathlineto{\pgfqpoint{1.922175in}{2.563836in}}%
\pgfpathlineto{\pgfqpoint{1.928301in}{2.563836in}}%
\pgfpathlineto{\pgfqpoint{1.928301in}{2.561468in}}%
\pgfpathlineto{\pgfqpoint{1.940552in}{2.561468in}}%
\pgfpathlineto{\pgfqpoint{1.940552in}{2.549154in}}%
\pgfpathlineto{\pgfqpoint{1.946678in}{2.549154in}}%
\pgfpathlineto{\pgfqpoint{1.946678in}{2.543471in}}%
\pgfpathlineto{\pgfqpoint{1.965055in}{2.543471in}}%
\pgfpathlineto{\pgfqpoint{1.965055in}{2.537788in}}%
\pgfpathlineto{\pgfqpoint{1.971181in}{2.537788in}}%
\pgfpathlineto{\pgfqpoint{1.971181in}{2.527369in}}%
\pgfpathlineto{\pgfqpoint{1.977306in}{2.527369in}}%
\pgfpathlineto{\pgfqpoint{1.977306in}{2.523580in}}%
\pgfpathlineto{\pgfqpoint{1.995683in}{2.524054in}}%
\pgfpathlineto{\pgfqpoint{1.995683in}{2.513161in}}%
\pgfpathlineto{\pgfqpoint{2.007935in}{2.512687in}}%
\pgfpathlineto{\pgfqpoint{2.007935in}{2.517897in}}%
\pgfpathlineto{\pgfqpoint{2.014060in}{2.517897in}}%
\pgfpathlineto{\pgfqpoint{2.014060in}{2.511266in}}%
\pgfpathlineto{\pgfqpoint{2.020186in}{2.511266in}}%
\pgfpathlineto{\pgfqpoint{2.020186in}{2.514108in}}%
\pgfpathlineto{\pgfqpoint{2.026312in}{2.514108in}}%
\pgfpathlineto{\pgfqpoint{2.026312in}{2.511266in}}%
\pgfpathlineto{\pgfqpoint{2.032437in}{2.511266in}}%
\pgfpathlineto{\pgfqpoint{2.032437in}{2.507951in}}%
\pgfpathlineto{\pgfqpoint{2.038563in}{2.507951in}}%
\pgfpathlineto{\pgfqpoint{2.038563in}{2.504636in}}%
\pgfpathlineto{\pgfqpoint{2.050814in}{2.505583in}}%
\pgfpathlineto{\pgfqpoint{2.050814in}{2.506530in}}%
\pgfpathlineto{\pgfqpoint{2.056940in}{2.506530in}}%
\pgfpathlineto{\pgfqpoint{2.056940in}{2.501795in}}%
\pgfpathlineto{\pgfqpoint{2.063065in}{2.501795in}}%
\pgfpathlineto{\pgfqpoint{2.063065in}{2.498479in}}%
\pgfpathlineto{\pgfqpoint{2.069191in}{2.498479in}}%
\pgfpathlineto{\pgfqpoint{2.069191in}{2.501795in}}%
\pgfpathlineto{\pgfqpoint{2.081442in}{2.500847in}}%
\pgfpathlineto{\pgfqpoint{2.081442in}{2.498006in}}%
\pgfpathlineto{\pgfqpoint{2.087568in}{2.498006in}}%
\pgfpathlineto{\pgfqpoint{2.087568in}{2.500374in}}%
\pgfpathlineto{\pgfqpoint{2.093694in}{2.500374in}}%
\pgfpathlineto{\pgfqpoint{2.093694in}{2.495638in}}%
\pgfpathlineto{\pgfqpoint{2.130448in}{2.494691in}}%
\pgfpathlineto{\pgfqpoint{2.130448in}{2.493743in}}%
\pgfpathlineto{\pgfqpoint{2.265212in}{2.493270in}}%
\pgfpathlineto{\pgfqpoint{2.265212in}{2.492796in}}%
\pgfusepath{stroke}%
\end{pgfscope}%
\begin{pgfscope}%
\pgfpathrectangle{\pgfqpoint{0.485140in}{2.492796in}}{\pgfqpoint{2.540460in}{1.418241in}}%
\pgfusepath{clip}%
\pgfsetbuttcap%
\pgfsetmiterjoin%
\pgfsetlinewidth{1.003750pt}%
\definecolor{currentstroke}{rgb}{0.949020,0.372549,0.360784}%
\pgfsetstrokecolor{currentstroke}%
\pgfsetdash{{1.000000pt}{1.650000pt}}{0.000000pt}%
\pgfpathmoveto{\pgfqpoint{1.040080in}{2.492796in}}%
\pgfpathlineto{\pgfqpoint{1.040080in}{2.515642in}}%
\pgfpathlineto{\pgfqpoint{1.052332in}{2.516118in}}%
\pgfpathlineto{\pgfqpoint{1.052332in}{2.512787in}}%
\pgfpathlineto{\pgfqpoint{1.058457in}{2.512787in}}%
\pgfpathlineto{\pgfqpoint{1.058457in}{2.521354in}}%
\pgfpathlineto{\pgfqpoint{1.064583in}{2.521354in}}%
\pgfpathlineto{\pgfqpoint{1.064583in}{2.517546in}}%
\pgfpathlineto{\pgfqpoint{1.070709in}{2.517546in}}%
\pgfpathlineto{\pgfqpoint{1.070709in}{2.515642in}}%
\pgfpathlineto{\pgfqpoint{1.076834in}{2.515642in}}%
\pgfpathlineto{\pgfqpoint{1.076834in}{2.525638in}}%
\pgfpathlineto{\pgfqpoint{1.082960in}{2.525638in}}%
\pgfpathlineto{\pgfqpoint{1.082960in}{2.528018in}}%
\pgfpathlineto{\pgfqpoint{1.089086in}{2.528018in}}%
\pgfpathlineto{\pgfqpoint{1.089086in}{2.518974in}}%
\pgfpathlineto{\pgfqpoint{1.095211in}{2.518974in}}%
\pgfpathlineto{\pgfqpoint{1.095211in}{2.537061in}}%
\pgfpathlineto{\pgfqpoint{1.101337in}{2.537061in}}%
\pgfpathlineto{\pgfqpoint{1.101337in}{2.532301in}}%
\pgfpathlineto{\pgfqpoint{1.113588in}{2.533253in}}%
\pgfpathlineto{\pgfqpoint{1.113588in}{2.530397in}}%
\pgfpathlineto{\pgfqpoint{1.119714in}{2.530397in}}%
\pgfpathlineto{\pgfqpoint{1.119714in}{2.539917in}}%
\pgfpathlineto{\pgfqpoint{1.131965in}{2.538965in}}%
\pgfpathlineto{\pgfqpoint{1.131965in}{2.537061in}}%
\pgfpathlineto{\pgfqpoint{1.138091in}{2.537061in}}%
\pgfpathlineto{\pgfqpoint{1.138091in}{2.540869in}}%
\pgfpathlineto{\pgfqpoint{1.144217in}{2.540869in}}%
\pgfpathlineto{\pgfqpoint{1.144217in}{2.543249in}}%
\pgfpathlineto{\pgfqpoint{1.150342in}{2.543249in}}%
\pgfpathlineto{\pgfqpoint{1.150342in}{2.556576in}}%
\pgfpathlineto{\pgfqpoint{1.156468in}{2.556576in}}%
\pgfpathlineto{\pgfqpoint{1.156468in}{2.545628in}}%
\pgfpathlineto{\pgfqpoint{1.162594in}{2.545628in}}%
\pgfpathlineto{\pgfqpoint{1.162594in}{2.570379in}}%
\pgfpathlineto{\pgfqpoint{1.168719in}{2.570379in}}%
\pgfpathlineto{\pgfqpoint{1.168719in}{2.557528in}}%
\pgfpathlineto{\pgfqpoint{1.174845in}{2.557528in}}%
\pgfpathlineto{\pgfqpoint{1.174845in}{2.562763in}}%
\pgfpathlineto{\pgfqpoint{1.180971in}{2.562763in}}%
\pgfpathlineto{\pgfqpoint{1.180971in}{2.573234in}}%
\pgfpathlineto{\pgfqpoint{1.187096in}{2.573234in}}%
\pgfpathlineto{\pgfqpoint{1.187096in}{2.576090in}}%
\pgfpathlineto{\pgfqpoint{1.193222in}{2.576090in}}%
\pgfpathlineto{\pgfqpoint{1.193222in}{2.562287in}}%
\pgfpathlineto{\pgfqpoint{1.199348in}{2.562287in}}%
\pgfpathlineto{\pgfqpoint{1.199348in}{2.580374in}}%
\pgfpathlineto{\pgfqpoint{1.205473in}{2.580374in}}%
\pgfpathlineto{\pgfqpoint{1.205473in}{2.585134in}}%
\pgfpathlineto{\pgfqpoint{1.211599in}{2.585134in}}%
\pgfpathlineto{\pgfqpoint{1.211599in}{2.589417in}}%
\pgfpathlineto{\pgfqpoint{1.217725in}{2.589417in}}%
\pgfpathlineto{\pgfqpoint{1.217725in}{2.586561in}}%
\pgfpathlineto{\pgfqpoint{1.223850in}{2.586561in}}%
\pgfpathlineto{\pgfqpoint{1.223850in}{2.607980in}}%
\pgfpathlineto{\pgfqpoint{1.236102in}{2.607980in}}%
\pgfpathlineto{\pgfqpoint{1.236102in}{2.613692in}}%
\pgfpathlineto{\pgfqpoint{1.242227in}{2.613692in}}%
\pgfpathlineto{\pgfqpoint{1.242227in}{2.625115in}}%
\pgfpathlineto{\pgfqpoint{1.254478in}{2.624639in}}%
\pgfpathlineto{\pgfqpoint{1.254478in}{2.643677in}}%
\pgfpathlineto{\pgfqpoint{1.260604in}{2.643677in}}%
\pgfpathlineto{\pgfqpoint{1.260604in}{2.662716in}}%
\pgfpathlineto{\pgfqpoint{1.266730in}{2.662716in}}%
\pgfpathlineto{\pgfqpoint{1.266730in}{2.659384in}}%
\pgfpathlineto{\pgfqpoint{1.272855in}{2.659384in}}%
\pgfpathlineto{\pgfqpoint{1.272855in}{2.669380in}}%
\pgfpathlineto{\pgfqpoint{1.278981in}{2.669380in}}%
\pgfpathlineto{\pgfqpoint{1.278981in}{2.675567in}}%
\pgfpathlineto{\pgfqpoint{1.285107in}{2.675567in}}%
\pgfpathlineto{\pgfqpoint{1.285107in}{2.680327in}}%
\pgfpathlineto{\pgfqpoint{1.291232in}{2.680327in}}%
\pgfpathlineto{\pgfqpoint{1.291232in}{2.714596in}}%
\pgfpathlineto{\pgfqpoint{1.297358in}{2.714596in}}%
\pgfpathlineto{\pgfqpoint{1.297358in}{2.726496in}}%
\pgfpathlineto{\pgfqpoint{1.309609in}{2.727448in}}%
\pgfpathlineto{\pgfqpoint{1.309609in}{2.764097in}}%
\pgfpathlineto{\pgfqpoint{1.315735in}{2.764097in}}%
\pgfpathlineto{\pgfqpoint{1.315735in}{2.768381in}}%
\pgfpathlineto{\pgfqpoint{1.321861in}{2.768381in}}%
\pgfpathlineto{\pgfqpoint{1.321861in}{2.811218in}}%
\pgfpathlineto{\pgfqpoint{1.327986in}{2.811218in}}%
\pgfpathlineto{\pgfqpoint{1.327986in}{2.792179in}}%
\pgfpathlineto{\pgfqpoint{1.334112in}{2.792179in}}%
\pgfpathlineto{\pgfqpoint{1.334112in}{2.830732in}}%
\pgfpathlineto{\pgfqpoint{1.340238in}{2.830732in}}%
\pgfpathlineto{\pgfqpoint{1.340238in}{2.855007in}}%
\pgfpathlineto{\pgfqpoint{1.346363in}{2.855007in}}%
\pgfpathlineto{\pgfqpoint{1.346363in}{2.861670in}}%
\pgfpathlineto{\pgfqpoint{1.352489in}{2.861670in}}%
\pgfpathlineto{\pgfqpoint{1.352489in}{2.912599in}}%
\pgfpathlineto{\pgfqpoint{1.358615in}{2.912599in}}%
\pgfpathlineto{\pgfqpoint{1.358615in}{2.933541in}}%
\pgfpathlineto{\pgfqpoint{1.364740in}{2.933541in}}%
\pgfpathlineto{\pgfqpoint{1.364740in}{2.927353in}}%
\pgfpathlineto{\pgfqpoint{1.370866in}{2.927353in}}%
\pgfpathlineto{\pgfqpoint{1.370866in}{3.010172in}}%
\pgfpathlineto{\pgfqpoint{1.376992in}{3.010172in}}%
\pgfpathlineto{\pgfqpoint{1.376992in}{3.024451in}}%
\pgfpathlineto{\pgfqpoint{1.383117in}{3.024451in}}%
\pgfpathlineto{\pgfqpoint{1.383117in}{3.065860in}}%
\pgfpathlineto{\pgfqpoint{1.389243in}{3.065860in}}%
\pgfpathlineto{\pgfqpoint{1.389243in}{3.086326in}}%
\pgfpathlineto{\pgfqpoint{1.395369in}{3.086326in}}%
\pgfpathlineto{\pgfqpoint{1.395369in}{3.146774in}}%
\pgfpathlineto{\pgfqpoint{1.401494in}{3.146774in}}%
\pgfpathlineto{\pgfqpoint{1.401494in}{3.249107in}}%
\pgfpathlineto{\pgfqpoint{1.407620in}{3.249107in}}%
\pgfpathlineto{\pgfqpoint{1.407620in}{3.245775in}}%
\pgfpathlineto{\pgfqpoint{1.413746in}{3.245775in}}%
\pgfpathlineto{\pgfqpoint{1.413746in}{3.299559in}}%
\pgfpathlineto{\pgfqpoint{1.419871in}{3.299559in}}%
\pgfpathlineto{\pgfqpoint{1.419871in}{3.353343in}}%
\pgfpathlineto{\pgfqpoint{1.425997in}{3.353343in}}%
\pgfpathlineto{\pgfqpoint{1.425997in}{3.376666in}}%
\pgfpathlineto{\pgfqpoint{1.432123in}{3.376666in}}%
\pgfpathlineto{\pgfqpoint{1.432123in}{3.443777in}}%
\pgfpathlineto{\pgfqpoint{1.438248in}{3.443777in}}%
\pgfpathlineto{\pgfqpoint{1.438248in}{3.473763in}}%
\pgfpathlineto{\pgfqpoint{1.444374in}{3.473763in}}%
\pgfpathlineto{\pgfqpoint{1.444374in}{3.544682in}}%
\pgfpathlineto{\pgfqpoint{1.450500in}{3.544682in}}%
\pgfpathlineto{\pgfqpoint{1.450500in}{3.576572in}}%
\pgfpathlineto{\pgfqpoint{1.456625in}{3.576572in}}%
\pgfpathlineto{\pgfqpoint{1.456625in}{3.641303in}}%
\pgfpathlineto{\pgfqpoint{1.462751in}{3.641303in}}%
\pgfpathlineto{\pgfqpoint{1.462751in}{3.662722in}}%
\pgfpathlineto{\pgfqpoint{1.468877in}{3.662722in}}%
\pgfpathlineto{\pgfqpoint{1.468877in}{3.665101in}}%
\pgfpathlineto{\pgfqpoint{1.475002in}{3.665101in}}%
\pgfpathlineto{\pgfqpoint{1.475002in}{3.730309in}}%
\pgfpathlineto{\pgfqpoint{1.481128in}{3.730309in}}%
\pgfpathlineto{\pgfqpoint{1.481128in}{3.717934in}}%
\pgfpathlineto{\pgfqpoint{1.487254in}{3.717934in}}%
\pgfpathlineto{\pgfqpoint{1.487254in}{3.763151in}}%
\pgfpathlineto{\pgfqpoint{1.493379in}{3.763151in}}%
\pgfpathlineto{\pgfqpoint{1.493379in}{3.746492in}}%
\pgfpathlineto{\pgfqpoint{1.499505in}{3.746492in}}%
\pgfpathlineto{\pgfqpoint{1.499505in}{3.776954in}}%
\pgfpathlineto{\pgfqpoint{1.505631in}{3.776954in}}%
\pgfpathlineto{\pgfqpoint{1.505631in}{3.799324in}}%
\pgfpathlineto{\pgfqpoint{1.511756in}{3.799324in}}%
\pgfpathlineto{\pgfqpoint{1.511756in}{3.770766in}}%
\pgfpathlineto{\pgfqpoint{1.517882in}{3.770766in}}%
\pgfpathlineto{\pgfqpoint{1.517882in}{3.761247in}}%
\pgfpathlineto{\pgfqpoint{1.524007in}{3.761247in}}%
\pgfpathlineto{\pgfqpoint{1.524007in}{3.742208in}}%
\pgfpathlineto{\pgfqpoint{1.530133in}{3.742208in}}%
\pgfpathlineto{\pgfqpoint{1.530133in}{3.719362in}}%
\pgfpathlineto{\pgfqpoint{1.536259in}{3.719362in}}%
\pgfpathlineto{\pgfqpoint{1.536259in}{3.690804in}}%
\pgfpathlineto{\pgfqpoint{1.542384in}{3.690804in}}%
\pgfpathlineto{\pgfqpoint{1.542384in}{3.656534in}}%
\pgfpathlineto{\pgfqpoint{1.548510in}{3.656534in}}%
\pgfpathlineto{\pgfqpoint{1.548510in}{3.629880in}}%
\pgfpathlineto{\pgfqpoint{1.554636in}{3.629880in}}%
\pgfpathlineto{\pgfqpoint{1.554636in}{3.612745in}}%
\pgfpathlineto{\pgfqpoint{1.566887in}{3.611793in}}%
\pgfpathlineto{\pgfqpoint{1.566887in}{3.566576in}}%
\pgfpathlineto{\pgfqpoint{1.573013in}{3.566576in}}%
\pgfpathlineto{\pgfqpoint{1.573013in}{3.504701in}}%
\pgfpathlineto{\pgfqpoint{1.579138in}{3.504701in}}%
\pgfpathlineto{\pgfqpoint{1.579138in}{3.469479in}}%
\pgfpathlineto{\pgfqpoint{1.585264in}{3.469479in}}%
\pgfpathlineto{\pgfqpoint{1.585264in}{3.450917in}}%
\pgfpathlineto{\pgfqpoint{1.591390in}{3.450917in}}%
\pgfpathlineto{\pgfqpoint{1.591390in}{3.409031in}}%
\pgfpathlineto{\pgfqpoint{1.597515in}{3.409031in}}%
\pgfpathlineto{\pgfqpoint{1.597515in}{3.384281in}}%
\pgfpathlineto{\pgfqpoint{1.603641in}{3.384281in}}%
\pgfpathlineto{\pgfqpoint{1.603641in}{3.329545in}}%
\pgfpathlineto{\pgfqpoint{1.609767in}{3.329545in}}%
\pgfpathlineto{\pgfqpoint{1.609767in}{3.293372in}}%
\pgfpathlineto{\pgfqpoint{1.615892in}{3.293372in}}%
\pgfpathlineto{\pgfqpoint{1.615892in}{3.302891in}}%
\pgfpathlineto{\pgfqpoint{1.622018in}{3.302891in}}%
\pgfpathlineto{\pgfqpoint{1.622018in}{3.229592in}}%
\pgfpathlineto{\pgfqpoint{1.628144in}{3.229592in}}%
\pgfpathlineto{\pgfqpoint{1.628144in}{3.216741in}}%
\pgfpathlineto{\pgfqpoint{1.634269in}{3.216741in}}%
\pgfpathlineto{\pgfqpoint{1.634269in}{3.164861in}}%
\pgfpathlineto{\pgfqpoint{1.640395in}{3.164861in}}%
\pgfpathlineto{\pgfqpoint{1.640395in}{3.103461in}}%
\pgfpathlineto{\pgfqpoint{1.646521in}{3.103461in}}%
\pgfpathlineto{\pgfqpoint{1.646521in}{3.075379in}}%
\pgfpathlineto{\pgfqpoint{1.652646in}{3.075379in}}%
\pgfpathlineto{\pgfqpoint{1.652646in}{3.037778in}}%
\pgfpathlineto{\pgfqpoint{1.658772in}{3.037778in}}%
\pgfpathlineto{\pgfqpoint{1.658772in}{3.006364in}}%
\pgfpathlineto{\pgfqpoint{1.664898in}{3.006364in}}%
\pgfpathlineto{\pgfqpoint{1.664898in}{3.012551in}}%
\pgfpathlineto{\pgfqpoint{1.671023in}{3.012551in}}%
\pgfpathlineto{\pgfqpoint{1.671023in}{2.962575in}}%
\pgfpathlineto{\pgfqpoint{1.677149in}{2.962575in}}%
\pgfpathlineto{\pgfqpoint{1.677149in}{2.944488in}}%
\pgfpathlineto{\pgfqpoint{1.683275in}{2.944488in}}%
\pgfpathlineto{\pgfqpoint{1.683275in}{2.906411in}}%
\pgfpathlineto{\pgfqpoint{1.689400in}{2.906411in}}%
\pgfpathlineto{\pgfqpoint{1.689400in}{2.879757in}}%
\pgfpathlineto{\pgfqpoint{1.695526in}{2.879757in}}%
\pgfpathlineto{\pgfqpoint{1.695526in}{2.841680in}}%
\pgfpathlineto{\pgfqpoint{1.701652in}{2.841680in}}%
\pgfpathlineto{\pgfqpoint{1.701652in}{2.835968in}}%
\pgfpathlineto{\pgfqpoint{1.707777in}{2.835968in}}%
\pgfpathlineto{\pgfqpoint{1.707777in}{2.827876in}}%
\pgfpathlineto{\pgfqpoint{1.713903in}{2.827876in}}%
\pgfpathlineto{\pgfqpoint{1.713903in}{2.787895in}}%
\pgfpathlineto{\pgfqpoint{1.720029in}{2.787895in}}%
\pgfpathlineto{\pgfqpoint{1.720029in}{2.786467in}}%
\pgfpathlineto{\pgfqpoint{1.726154in}{2.786467in}}%
\pgfpathlineto{\pgfqpoint{1.726154in}{2.735539in}}%
\pgfpathlineto{\pgfqpoint{1.738406in}{2.736491in}}%
\pgfpathlineto{\pgfqpoint{1.738406in}{2.716024in}}%
\pgfpathlineto{\pgfqpoint{1.744531in}{2.716024in}}%
\pgfpathlineto{\pgfqpoint{1.744531in}{2.706505in}}%
\pgfpathlineto{\pgfqpoint{1.750657in}{2.706505in}}%
\pgfpathlineto{\pgfqpoint{1.750657in}{2.687942in}}%
\pgfpathlineto{\pgfqpoint{1.756783in}{2.687942in}}%
\pgfpathlineto{\pgfqpoint{1.756783in}{2.667000in}}%
\pgfpathlineto{\pgfqpoint{1.762908in}{2.667000in}}%
\pgfpathlineto{\pgfqpoint{1.762908in}{2.671759in}}%
\pgfpathlineto{\pgfqpoint{1.769034in}{2.671759in}}%
\pgfpathlineto{\pgfqpoint{1.769034in}{2.653197in}}%
\pgfpathlineto{\pgfqpoint{1.775160in}{2.653197in}}%
\pgfpathlineto{\pgfqpoint{1.775160in}{2.647485in}}%
\pgfpathlineto{\pgfqpoint{1.781285in}{2.647485in}}%
\pgfpathlineto{\pgfqpoint{1.781285in}{2.634634in}}%
\pgfpathlineto{\pgfqpoint{1.787411in}{2.634634in}}%
\pgfpathlineto{\pgfqpoint{1.787411in}{2.626067in}}%
\pgfpathlineto{\pgfqpoint{1.793536in}{2.626067in}}%
\pgfpathlineto{\pgfqpoint{1.793536in}{2.618927in}}%
\pgfpathlineto{\pgfqpoint{1.799662in}{2.618927in}}%
\pgfpathlineto{\pgfqpoint{1.799662in}{2.608932in}}%
\pgfpathlineto{\pgfqpoint{1.805788in}{2.608932in}}%
\pgfpathlineto{\pgfqpoint{1.805788in}{2.600365in}}%
\pgfpathlineto{\pgfqpoint{1.811913in}{2.600365in}}%
\pgfpathlineto{\pgfqpoint{1.811913in}{2.602744in}}%
\pgfpathlineto{\pgfqpoint{1.818039in}{2.602744in}}%
\pgfpathlineto{\pgfqpoint{1.818039in}{2.579422in}}%
\pgfpathlineto{\pgfqpoint{1.824165in}{2.579422in}}%
\pgfpathlineto{\pgfqpoint{1.824165in}{2.575138in}}%
\pgfpathlineto{\pgfqpoint{1.830290in}{2.575138in}}%
\pgfpathlineto{\pgfqpoint{1.830290in}{2.566571in}}%
\pgfpathlineto{\pgfqpoint{1.836416in}{2.566571in}}%
\pgfpathlineto{\pgfqpoint{1.836416in}{2.562287in}}%
\pgfpathlineto{\pgfqpoint{1.842542in}{2.562287in}}%
\pgfpathlineto{\pgfqpoint{1.842542in}{2.549912in}}%
\pgfpathlineto{\pgfqpoint{1.848667in}{2.549912in}}%
\pgfpathlineto{\pgfqpoint{1.848667in}{2.551340in}}%
\pgfpathlineto{\pgfqpoint{1.854793in}{2.551340in}}%
\pgfpathlineto{\pgfqpoint{1.854793in}{2.549436in}}%
\pgfpathlineto{\pgfqpoint{1.860919in}{2.549436in}}%
\pgfpathlineto{\pgfqpoint{1.860919in}{2.550864in}}%
\pgfpathlineto{\pgfqpoint{1.867044in}{2.550864in}}%
\pgfpathlineto{\pgfqpoint{1.867044in}{2.541821in}}%
\pgfpathlineto{\pgfqpoint{1.873170in}{2.541821in}}%
\pgfpathlineto{\pgfqpoint{1.873170in}{2.543249in}}%
\pgfpathlineto{\pgfqpoint{1.879296in}{2.543249in}}%
\pgfpathlineto{\pgfqpoint{1.879296in}{2.528494in}}%
\pgfpathlineto{\pgfqpoint{1.885421in}{2.528494in}}%
\pgfpathlineto{\pgfqpoint{1.885421in}{2.529921in}}%
\pgfpathlineto{\pgfqpoint{1.891547in}{2.529921in}}%
\pgfpathlineto{\pgfqpoint{1.891547in}{2.527066in}}%
\pgfpathlineto{\pgfqpoint{1.897673in}{2.527066in}}%
\pgfpathlineto{\pgfqpoint{1.897673in}{2.517070in}}%
\pgfpathlineto{\pgfqpoint{1.903798in}{2.517070in}}%
\pgfpathlineto{\pgfqpoint{1.903798in}{2.529446in}}%
\pgfpathlineto{\pgfqpoint{1.909924in}{2.529446in}}%
\pgfpathlineto{\pgfqpoint{1.909924in}{2.525638in}}%
\pgfpathlineto{\pgfqpoint{1.916050in}{2.525638in}}%
\pgfpathlineto{\pgfqpoint{1.916050in}{2.519926in}}%
\pgfpathlineto{\pgfqpoint{1.928301in}{2.519926in}}%
\pgfpathlineto{\pgfqpoint{1.928301in}{2.512787in}}%
\pgfpathlineto{\pgfqpoint{1.934427in}{2.512787in}}%
\pgfpathlineto{\pgfqpoint{1.934427in}{2.517070in}}%
\pgfpathlineto{\pgfqpoint{1.940552in}{2.517070in}}%
\pgfpathlineto{\pgfqpoint{1.940552in}{2.509931in}}%
\pgfpathlineto{\pgfqpoint{1.946678in}{2.509931in}}%
\pgfpathlineto{\pgfqpoint{1.946678in}{2.507551in}}%
\pgfpathlineto{\pgfqpoint{1.952804in}{2.507551in}}%
\pgfpathlineto{\pgfqpoint{1.952804in}{2.505647in}}%
\pgfpathlineto{\pgfqpoint{1.965055in}{2.504695in}}%
\pgfpathlineto{\pgfqpoint{1.965055in}{2.507551in}}%
\pgfpathlineto{\pgfqpoint{1.971181in}{2.507551in}}%
\pgfpathlineto{\pgfqpoint{1.971181in}{2.505171in}}%
\pgfpathlineto{\pgfqpoint{1.977306in}{2.505171in}}%
\pgfpathlineto{\pgfqpoint{1.977306in}{2.501364in}}%
\pgfpathlineto{\pgfqpoint{1.983432in}{2.501364in}}%
\pgfpathlineto{\pgfqpoint{1.983432in}{2.502791in}}%
\pgfpathlineto{\pgfqpoint{1.989558in}{2.502791in}}%
\pgfpathlineto{\pgfqpoint{1.989558in}{2.499936in}}%
\pgfpathlineto{\pgfqpoint{1.995683in}{2.499936in}}%
\pgfpathlineto{\pgfqpoint{1.995683in}{2.501364in}}%
\pgfpathlineto{\pgfqpoint{2.001809in}{2.501364in}}%
\pgfpathlineto{\pgfqpoint{2.001809in}{2.498032in}}%
\pgfpathlineto{\pgfqpoint{2.007935in}{2.498032in}}%
\pgfpathlineto{\pgfqpoint{2.007935in}{2.499936in}}%
\pgfpathlineto{\pgfqpoint{2.020186in}{2.499936in}}%
\pgfpathlineto{\pgfqpoint{2.020186in}{2.501364in}}%
\pgfpathlineto{\pgfqpoint{2.026312in}{2.501364in}}%
\pgfpathlineto{\pgfqpoint{2.026312in}{2.498032in}}%
\pgfpathlineto{\pgfqpoint{2.044689in}{2.498032in}}%
\pgfpathlineto{\pgfqpoint{2.044689in}{2.495176in}}%
\pgfpathlineto{\pgfqpoint{2.050814in}{2.495176in}}%
\pgfpathlineto{\pgfqpoint{2.050814in}{2.496604in}}%
\pgfpathlineto{\pgfqpoint{2.081442in}{2.497556in}}%
\pgfpathlineto{\pgfqpoint{2.081442in}{2.498032in}}%
\pgfpathlineto{\pgfqpoint{2.093694in}{2.497080in}}%
\pgfpathlineto{\pgfqpoint{2.093694in}{2.494700in}}%
\pgfpathlineto{\pgfqpoint{2.105945in}{2.494700in}}%
\pgfpathlineto{\pgfqpoint{2.105945in}{2.497080in}}%
\pgfpathlineto{\pgfqpoint{2.112071in}{2.497080in}}%
\pgfpathlineto{\pgfqpoint{2.112071in}{2.493748in}}%
\pgfpathlineto{\pgfqpoint{2.124322in}{2.494700in}}%
\pgfpathlineto{\pgfqpoint{2.124322in}{2.495652in}}%
\pgfpathlineto{\pgfqpoint{2.136573in}{2.494700in}}%
\pgfpathlineto{\pgfqpoint{2.136573in}{2.494224in}}%
\pgfpathlineto{\pgfqpoint{2.142699in}{2.494224in}}%
\pgfpathlineto{\pgfqpoint{2.142699in}{2.496604in}}%
\pgfpathlineto{\pgfqpoint{2.154950in}{2.496128in}}%
\pgfpathlineto{\pgfqpoint{2.154950in}{2.493748in}}%
\pgfpathlineto{\pgfqpoint{2.161076in}{2.493748in}}%
\pgfpathlineto{\pgfqpoint{2.161076in}{2.495176in}}%
\pgfpathlineto{\pgfqpoint{2.167202in}{2.495176in}}%
\pgfpathlineto{\pgfqpoint{2.167202in}{2.493272in}}%
\pgfpathlineto{\pgfqpoint{2.185579in}{2.494224in}}%
\pgfpathlineto{\pgfqpoint{2.185579in}{2.494700in}}%
\pgfpathlineto{\pgfqpoint{2.197830in}{2.493748in}}%
\pgfpathlineto{\pgfqpoint{2.197830in}{2.493272in}}%
\pgfpathlineto{\pgfqpoint{2.265212in}{2.493272in}}%
\pgfpathlineto{\pgfqpoint{2.265212in}{2.492796in}}%
\pgfusepath{stroke}%
\end{pgfscope}%
\begin{pgfscope}%
\pgfsetrectcap%
\pgfsetmiterjoin%
\pgfsetlinewidth{0.803000pt}%
\definecolor{currentstroke}{rgb}{0.000000,0.000000,0.000000}%
\pgfsetstrokecolor{currentstroke}%
\pgfsetdash{}{0pt}%
\pgfpathmoveto{\pgfqpoint{0.485140in}{2.492796in}}%
\pgfpathlineto{\pgfqpoint{0.485140in}{3.911037in}}%
\pgfusepath{stroke}%
\end{pgfscope}%
\begin{pgfscope}%
\pgfsetrectcap%
\pgfsetmiterjoin%
\pgfsetlinewidth{0.803000pt}%
\definecolor{currentstroke}{rgb}{0.000000,0.000000,0.000000}%
\pgfsetstrokecolor{currentstroke}%
\pgfsetdash{}{0pt}%
\pgfpathmoveto{\pgfqpoint{3.025600in}{2.492796in}}%
\pgfpathlineto{\pgfqpoint{3.025600in}{3.911037in}}%
\pgfusepath{stroke}%
\end{pgfscope}%
\begin{pgfscope}%
\pgfsetrectcap%
\pgfsetmiterjoin%
\pgfsetlinewidth{0.803000pt}%
\definecolor{currentstroke}{rgb}{0.000000,0.000000,0.000000}%
\pgfsetstrokecolor{currentstroke}%
\pgfsetdash{}{0pt}%
\pgfpathmoveto{\pgfqpoint{0.485140in}{2.492796in}}%
\pgfpathlineto{\pgfqpoint{3.025600in}{2.492796in}}%
\pgfusepath{stroke}%
\end{pgfscope}%
\begin{pgfscope}%
\pgfsetrectcap%
\pgfsetmiterjoin%
\pgfsetlinewidth{0.803000pt}%
\definecolor{currentstroke}{rgb}{0.000000,0.000000,0.000000}%
\pgfsetstrokecolor{currentstroke}%
\pgfsetdash{}{0pt}%
\pgfpathmoveto{\pgfqpoint{0.485140in}{3.911037in}}%
\pgfpathlineto{\pgfqpoint{3.025600in}{3.911037in}}%
\pgfusepath{stroke}%
\end{pgfscope}%
\begin{pgfscope}%
\definecolor{textcolor}{rgb}{0.000000,0.000000,0.000000}%
\pgfsetstrokecolor{textcolor}%
\pgfsetfillcolor{textcolor}%
\pgftext[x=0.485140in,y=3.994370in,left,base]{\color{textcolor}\rmfamily\fontsize{10.000000}{12.000000}\selectfont Bin [1.33, 1.5), 127,688 events}%
\end{pgfscope}%
\begin{pgfscope}%
\pgfsetbuttcap%
\pgfsetmiterjoin%
\definecolor{currentfill}{rgb}{1.000000,1.000000,1.000000}%
\pgfsetfillcolor{currentfill}%
\pgfsetfillopacity{0.800000}%
\pgfsetlinewidth{1.003750pt}%
\definecolor{currentstroke}{rgb}{0.800000,0.800000,0.800000}%
\pgfsetstrokecolor{currentstroke}%
\pgfsetstrokeopacity{0.800000}%
\pgfsetdash{}{0pt}%
\pgfpathmoveto{\pgfqpoint{1.990822in}{3.511481in}}%
\pgfpathlineto{\pgfqpoint{2.947822in}{3.511481in}}%
\pgfpathquadraticcurveto{\pgfqpoint{2.970044in}{3.511481in}}{\pgfqpoint{2.970044in}{3.533704in}}%
\pgfpathlineto{\pgfqpoint{2.970044in}{3.833259in}}%
\pgfpathquadraticcurveto{\pgfqpoint{2.970044in}{3.855481in}}{\pgfqpoint{2.947822in}{3.855481in}}%
\pgfpathlineto{\pgfqpoint{1.990822in}{3.855481in}}%
\pgfpathquadraticcurveto{\pgfqpoint{1.968600in}{3.855481in}}{\pgfqpoint{1.968600in}{3.833259in}}%
\pgfpathlineto{\pgfqpoint{1.968600in}{3.533704in}}%
\pgfpathquadraticcurveto{\pgfqpoint{1.968600in}{3.511481in}}{\pgfqpoint{1.990822in}{3.511481in}}%
\pgfpathclose%
\pgfusepath{stroke,fill}%
\end{pgfscope}%
\begin{pgfscope}%
\pgfsetbuttcap%
\pgfsetmiterjoin%
\pgfsetlinewidth{1.003750pt}%
\definecolor{currentstroke}{rgb}{0.313725,0.317647,0.309804}%
\pgfsetstrokecolor{currentstroke}%
\pgfsetdash{}{0pt}%
\pgfpathmoveto{\pgfqpoint{2.013044in}{3.732815in}}%
\pgfpathlineto{\pgfqpoint{2.235267in}{3.732815in}}%
\pgfpathlineto{\pgfqpoint{2.235267in}{3.810593in}}%
\pgfpathlineto{\pgfqpoint{2.013044in}{3.810593in}}%
\pgfpathclose%
\pgfusepath{stroke}%
\end{pgfscope}%
\begin{pgfscope}%
\definecolor{textcolor}{rgb}{0.000000,0.000000,0.000000}%
\pgfsetstrokecolor{textcolor}%
\pgfsetfillcolor{textcolor}%
\pgftext[x=2.324156in,y=3.732815in,left,base]{\color{textcolor}\rmfamily\fontsize{8.000000}{9.600000}\selectfont IQR = 0.17}%
\end{pgfscope}%
\begin{pgfscope}%
\pgfsetbuttcap%
\pgfsetmiterjoin%
\pgfsetlinewidth{1.003750pt}%
\definecolor{currentstroke}{rgb}{0.949020,0.372549,0.360784}%
\pgfsetstrokecolor{currentstroke}%
\pgfsetdash{{1.000000pt}{1.650000pt}}{0.000000pt}%
\pgfpathmoveto{\pgfqpoint{2.013044in}{3.577481in}}%
\pgfpathlineto{\pgfqpoint{2.235267in}{3.577481in}}%
\pgfpathlineto{\pgfqpoint{2.235267in}{3.655259in}}%
\pgfpathlineto{\pgfqpoint{2.013044in}{3.655259in}}%
\pgfpathclose%
\pgfusepath{stroke}%
\end{pgfscope}%
\begin{pgfscope}%
\definecolor{textcolor}{rgb}{0.000000,0.000000,0.000000}%
\pgfsetstrokecolor{textcolor}%
\pgfsetfillcolor{textcolor}%
\pgftext[x=2.324156in,y=3.577481in,left,base]{\color{textcolor}\rmfamily\fontsize{8.000000}{9.600000}\selectfont IQR = 0.18}%
\end{pgfscope}%
\begin{pgfscope}%
\pgfsetbuttcap%
\pgfsetmiterjoin%
\definecolor{currentfill}{rgb}{1.000000,1.000000,1.000000}%
\pgfsetfillcolor{currentfill}%
\pgfsetlinewidth{0.000000pt}%
\definecolor{currentstroke}{rgb}{0.000000,0.000000,0.000000}%
\pgfsetstrokecolor{currentstroke}%
\pgfsetstrokeopacity{0.000000}%
\pgfsetdash{}{0pt}%
\pgfpathmoveto{\pgfqpoint{3.510740in}{2.492796in}}%
\pgfpathlineto{\pgfqpoint{6.051200in}{2.492796in}}%
\pgfpathlineto{\pgfqpoint{6.051200in}{3.911037in}}%
\pgfpathlineto{\pgfqpoint{3.510740in}{3.911037in}}%
\pgfpathclose%
\pgfusepath{fill}%
\end{pgfscope}%
\begin{pgfscope}%
\pgfsetbuttcap%
\pgfsetroundjoin%
\definecolor{currentfill}{rgb}{0.000000,0.000000,0.000000}%
\pgfsetfillcolor{currentfill}%
\pgfsetlinewidth{0.803000pt}%
\definecolor{currentstroke}{rgb}{0.000000,0.000000,0.000000}%
\pgfsetstrokecolor{currentstroke}%
\pgfsetdash{}{0pt}%
\pgfsys@defobject{currentmarker}{\pgfqpoint{0.000000in}{-0.048611in}}{\pgfqpoint{0.000000in}{0.000000in}}{%
\pgfpathmoveto{\pgfqpoint{0.000000in}{0.000000in}}%
\pgfpathlineto{\pgfqpoint{0.000000in}{-0.048611in}}%
\pgfusepath{stroke,fill}%
}%
\begin{pgfscope}%
\pgfsys@transformshift{3.941945in}{2.492796in}%
\pgfsys@useobject{currentmarker}{}%
\end{pgfscope}%
\end{pgfscope}%
\begin{pgfscope}%
\definecolor{textcolor}{rgb}{0.000000,0.000000,0.000000}%
\pgfsetstrokecolor{textcolor}%
\pgfsetfillcolor{textcolor}%
\pgftext[x=3.941945in,y=2.395574in,,top]{\color{textcolor}\rmfamily\fontsize{8.000000}{9.600000}\selectfont \(\displaystyle {-1}\)}%
\end{pgfscope}%
\begin{pgfscope}%
\pgfsetbuttcap%
\pgfsetroundjoin%
\definecolor{currentfill}{rgb}{0.000000,0.000000,0.000000}%
\pgfsetfillcolor{currentfill}%
\pgfsetlinewidth{0.803000pt}%
\definecolor{currentstroke}{rgb}{0.000000,0.000000,0.000000}%
\pgfsetstrokecolor{currentstroke}%
\pgfsetdash{}{0pt}%
\pgfsys@defobject{currentmarker}{\pgfqpoint{0.000000in}{-0.048611in}}{\pgfqpoint{0.000000in}{0.000000in}}{%
\pgfpathmoveto{\pgfqpoint{0.000000in}{0.000000in}}%
\pgfpathlineto{\pgfqpoint{0.000000in}{-0.048611in}}%
\pgfusepath{stroke,fill}%
}%
\begin{pgfscope}%
\pgfsys@transformshift{4.628399in}{2.492796in}%
\pgfsys@useobject{currentmarker}{}%
\end{pgfscope}%
\end{pgfscope}%
\begin{pgfscope}%
\definecolor{textcolor}{rgb}{0.000000,0.000000,0.000000}%
\pgfsetstrokecolor{textcolor}%
\pgfsetfillcolor{textcolor}%
\pgftext[x=4.628399in,y=2.395574in,,top]{\color{textcolor}\rmfamily\fontsize{8.000000}{9.600000}\selectfont \(\displaystyle {0}\)}%
\end{pgfscope}%
\begin{pgfscope}%
\pgfsetbuttcap%
\pgfsetroundjoin%
\definecolor{currentfill}{rgb}{0.000000,0.000000,0.000000}%
\pgfsetfillcolor{currentfill}%
\pgfsetlinewidth{0.803000pt}%
\definecolor{currentstroke}{rgb}{0.000000,0.000000,0.000000}%
\pgfsetstrokecolor{currentstroke}%
\pgfsetdash{}{0pt}%
\pgfsys@defobject{currentmarker}{\pgfqpoint{0.000000in}{-0.048611in}}{\pgfqpoint{0.000000in}{0.000000in}}{%
\pgfpathmoveto{\pgfqpoint{0.000000in}{0.000000in}}%
\pgfpathlineto{\pgfqpoint{0.000000in}{-0.048611in}}%
\pgfusepath{stroke,fill}%
}%
\begin{pgfscope}%
\pgfsys@transformshift{5.314853in}{2.492796in}%
\pgfsys@useobject{currentmarker}{}%
\end{pgfscope}%
\end{pgfscope}%
\begin{pgfscope}%
\definecolor{textcolor}{rgb}{0.000000,0.000000,0.000000}%
\pgfsetstrokecolor{textcolor}%
\pgfsetfillcolor{textcolor}%
\pgftext[x=5.314853in,y=2.395574in,,top]{\color{textcolor}\rmfamily\fontsize{8.000000}{9.600000}\selectfont \(\displaystyle {1}\)}%
\end{pgfscope}%
\begin{pgfscope}%
\pgfsetbuttcap%
\pgfsetroundjoin%
\definecolor{currentfill}{rgb}{0.000000,0.000000,0.000000}%
\pgfsetfillcolor{currentfill}%
\pgfsetlinewidth{0.803000pt}%
\definecolor{currentstroke}{rgb}{0.000000,0.000000,0.000000}%
\pgfsetstrokecolor{currentstroke}%
\pgfsetdash{}{0pt}%
\pgfsys@defobject{currentmarker}{\pgfqpoint{0.000000in}{-0.048611in}}{\pgfqpoint{0.000000in}{0.000000in}}{%
\pgfpathmoveto{\pgfqpoint{0.000000in}{0.000000in}}%
\pgfpathlineto{\pgfqpoint{0.000000in}{-0.048611in}}%
\pgfusepath{stroke,fill}%
}%
\begin{pgfscope}%
\pgfsys@transformshift{6.001306in}{2.492796in}%
\pgfsys@useobject{currentmarker}{}%
\end{pgfscope}%
\end{pgfscope}%
\begin{pgfscope}%
\definecolor{textcolor}{rgb}{0.000000,0.000000,0.000000}%
\pgfsetstrokecolor{textcolor}%
\pgfsetfillcolor{textcolor}%
\pgftext[x=6.001306in,y=2.395574in,,top]{\color{textcolor}\rmfamily\fontsize{8.000000}{9.600000}\selectfont \(\displaystyle {2}\)}%
\end{pgfscope}%
\begin{pgfscope}%
\pgfsetbuttcap%
\pgfsetroundjoin%
\definecolor{currentfill}{rgb}{0.000000,0.000000,0.000000}%
\pgfsetfillcolor{currentfill}%
\pgfsetlinewidth{0.803000pt}%
\definecolor{currentstroke}{rgb}{0.000000,0.000000,0.000000}%
\pgfsetstrokecolor{currentstroke}%
\pgfsetdash{}{0pt}%
\pgfsys@defobject{currentmarker}{\pgfqpoint{-0.048611in}{0.000000in}}{\pgfqpoint{-0.000000in}{0.000000in}}{%
\pgfpathmoveto{\pgfqpoint{-0.000000in}{0.000000in}}%
\pgfpathlineto{\pgfqpoint{-0.048611in}{0.000000in}}%
\pgfusepath{stroke,fill}%
}%
\begin{pgfscope}%
\pgfsys@transformshift{3.510740in}{2.492796in}%
\pgfsys@useobject{currentmarker}{}%
\end{pgfscope}%
\end{pgfscope}%
\begin{pgfscope}%
\definecolor{textcolor}{rgb}{0.000000,0.000000,0.000000}%
\pgfsetstrokecolor{textcolor}%
\pgfsetfillcolor{textcolor}%
\pgftext[x=3.354489in, y=2.454241in, left, base]{\color{textcolor}\rmfamily\fontsize{8.000000}{9.600000}\selectfont \(\displaystyle {0}\)}%
\end{pgfscope}%
\begin{pgfscope}%
\pgfsetbuttcap%
\pgfsetroundjoin%
\definecolor{currentfill}{rgb}{0.000000,0.000000,0.000000}%
\pgfsetfillcolor{currentfill}%
\pgfsetlinewidth{0.803000pt}%
\definecolor{currentstroke}{rgb}{0.000000,0.000000,0.000000}%
\pgfsetstrokecolor{currentstroke}%
\pgfsetdash{}{0pt}%
\pgfsys@defobject{currentmarker}{\pgfqpoint{-0.048611in}{0.000000in}}{\pgfqpoint{-0.000000in}{0.000000in}}{%
\pgfpathmoveto{\pgfqpoint{-0.000000in}{0.000000in}}%
\pgfpathlineto{\pgfqpoint{-0.048611in}{0.000000in}}%
\pgfusepath{stroke,fill}%
}%
\begin{pgfscope}%
\pgfsys@transformshift{3.510740in}{3.024623in}%
\pgfsys@useobject{currentmarker}{}%
\end{pgfscope}%
\end{pgfscope}%
\begin{pgfscope}%
\definecolor{textcolor}{rgb}{0.000000,0.000000,0.000000}%
\pgfsetstrokecolor{textcolor}%
\pgfsetfillcolor{textcolor}%
\pgftext[x=3.354489in, y=2.986068in, left, base]{\color{textcolor}\rmfamily\fontsize{8.000000}{9.600000}\selectfont \(\displaystyle {1}\)}%
\end{pgfscope}%
\begin{pgfscope}%
\pgfsetbuttcap%
\pgfsetroundjoin%
\definecolor{currentfill}{rgb}{0.000000,0.000000,0.000000}%
\pgfsetfillcolor{currentfill}%
\pgfsetlinewidth{0.803000pt}%
\definecolor{currentstroke}{rgb}{0.000000,0.000000,0.000000}%
\pgfsetstrokecolor{currentstroke}%
\pgfsetdash{}{0pt}%
\pgfsys@defobject{currentmarker}{\pgfqpoint{-0.048611in}{0.000000in}}{\pgfqpoint{-0.000000in}{0.000000in}}{%
\pgfpathmoveto{\pgfqpoint{-0.000000in}{0.000000in}}%
\pgfpathlineto{\pgfqpoint{-0.048611in}{0.000000in}}%
\pgfusepath{stroke,fill}%
}%
\begin{pgfscope}%
\pgfsys@transformshift{3.510740in}{3.556450in}%
\pgfsys@useobject{currentmarker}{}%
\end{pgfscope}%
\end{pgfscope}%
\begin{pgfscope}%
\definecolor{textcolor}{rgb}{0.000000,0.000000,0.000000}%
\pgfsetstrokecolor{textcolor}%
\pgfsetfillcolor{textcolor}%
\pgftext[x=3.354489in, y=3.517895in, left, base]{\color{textcolor}\rmfamily\fontsize{8.000000}{9.600000}\selectfont \(\displaystyle {2}\)}%
\end{pgfscope}%
\begin{pgfscope}%
\definecolor{textcolor}{rgb}{0.000000,0.000000,0.000000}%
\pgfsetstrokecolor{textcolor}%
\pgfsetfillcolor{textcolor}%
\pgftext[x=3.298933in,y=3.201917in,,bottom,rotate=90.000000]{\color{textcolor}\rmfamily\fontsize{10.000000}{12.000000}\selectfont Density}%
\end{pgfscope}%
\begin{pgfscope}%
\pgfpathrectangle{\pgfqpoint{3.510740in}{2.492796in}}{\pgfqpoint{2.540460in}{1.418241in}}%
\pgfusepath{clip}%
\pgfsetbuttcap%
\pgfsetmiterjoin%
\pgfsetlinewidth{1.003750pt}%
\definecolor{currentstroke}{rgb}{0.313725,0.317647,0.309804}%
\pgfsetstrokecolor{currentstroke}%
\pgfsetdash{}{0pt}%
\pgfpathmoveto{\pgfqpoint{4.176865in}{2.492796in}}%
\pgfpathlineto{\pgfqpoint{4.176865in}{2.493500in}}%
\pgfpathlineto{\pgfqpoint{4.216593in}{2.494203in}}%
\pgfpathlineto{\pgfqpoint{4.216593in}{2.495610in}}%
\pgfpathlineto{\pgfqpoint{4.224538in}{2.495610in}}%
\pgfpathlineto{\pgfqpoint{4.224538in}{2.492796in}}%
\pgfpathlineto{\pgfqpoint{4.248375in}{2.493500in}}%
\pgfpathlineto{\pgfqpoint{4.248375in}{2.497720in}}%
\pgfpathlineto{\pgfqpoint{4.256320in}{2.497720in}}%
\pgfpathlineto{\pgfqpoint{4.256320in}{2.494906in}}%
\pgfpathlineto{\pgfqpoint{4.264266in}{2.494906in}}%
\pgfpathlineto{\pgfqpoint{4.264266in}{2.497017in}}%
\pgfpathlineto{\pgfqpoint{4.280157in}{2.497720in}}%
\pgfpathlineto{\pgfqpoint{4.280157in}{2.499127in}}%
\pgfpathlineto{\pgfqpoint{4.296048in}{2.499127in}}%
\pgfpathlineto{\pgfqpoint{4.296048in}{2.500534in}}%
\pgfpathlineto{\pgfqpoint{4.303993in}{2.500534in}}%
\pgfpathlineto{\pgfqpoint{4.303993in}{2.503348in}}%
\pgfpathlineto{\pgfqpoint{4.319884in}{2.503348in}}%
\pgfpathlineto{\pgfqpoint{4.319884in}{2.511789in}}%
\pgfpathlineto{\pgfqpoint{4.327830in}{2.511789in}}%
\pgfpathlineto{\pgfqpoint{4.327830in}{2.513899in}}%
\pgfpathlineto{\pgfqpoint{4.335775in}{2.513899in}}%
\pgfpathlineto{\pgfqpoint{4.335775in}{2.522340in}}%
\pgfpathlineto{\pgfqpoint{4.343721in}{2.522340in}}%
\pgfpathlineto{\pgfqpoint{4.343721in}{2.518120in}}%
\pgfpathlineto{\pgfqpoint{4.351666in}{2.518120in}}%
\pgfpathlineto{\pgfqpoint{4.351666in}{2.520230in}}%
\pgfpathlineto{\pgfqpoint{4.359612in}{2.520230in}}%
\pgfpathlineto{\pgfqpoint{4.359612in}{2.525154in}}%
\pgfpathlineto{\pgfqpoint{4.367557in}{2.525154in}}%
\pgfpathlineto{\pgfqpoint{4.367557in}{2.539222in}}%
\pgfpathlineto{\pgfqpoint{4.375503in}{2.539222in}}%
\pgfpathlineto{\pgfqpoint{4.375503in}{2.544146in}}%
\pgfpathlineto{\pgfqpoint{4.383448in}{2.544146in}}%
\pgfpathlineto{\pgfqpoint{4.383448in}{2.560325in}}%
\pgfpathlineto{\pgfqpoint{4.391394in}{2.560325in}}%
\pgfpathlineto{\pgfqpoint{4.391394in}{2.562436in}}%
\pgfpathlineto{\pgfqpoint{4.399339in}{2.562436in}}%
\pgfpathlineto{\pgfqpoint{4.399339in}{2.565953in}}%
\pgfpathlineto{\pgfqpoint{4.407285in}{2.565953in}}%
\pgfpathlineto{\pgfqpoint{4.407285in}{2.591980in}}%
\pgfpathlineto{\pgfqpoint{4.415230in}{2.591980in}}%
\pgfpathlineto{\pgfqpoint{4.415230in}{2.613786in}}%
\pgfpathlineto{\pgfqpoint{4.423176in}{2.613786in}}%
\pgfpathlineto{\pgfqpoint{4.423176in}{2.641923in}}%
\pgfpathlineto{\pgfqpoint{4.439067in}{2.641923in}}%
\pgfpathlineto{\pgfqpoint{4.439067in}{2.679908in}}%
\pgfpathlineto{\pgfqpoint{4.447012in}{2.679908in}}%
\pgfpathlineto{\pgfqpoint{4.447012in}{2.695384in}}%
\pgfpathlineto{\pgfqpoint{4.454958in}{2.695384in}}%
\pgfpathlineto{\pgfqpoint{4.454958in}{2.730555in}}%
\pgfpathlineto{\pgfqpoint{4.462904in}{2.730555in}}%
\pgfpathlineto{\pgfqpoint{4.462904in}{2.729148in}}%
\pgfpathlineto{\pgfqpoint{4.470849in}{2.729148in}}%
\pgfpathlineto{\pgfqpoint{4.470849in}{2.757286in}}%
\pgfpathlineto{\pgfqpoint{4.478795in}{2.757286in}}%
\pgfpathlineto{\pgfqpoint{4.478795in}{2.826222in}}%
\pgfpathlineto{\pgfqpoint{4.486740in}{2.826222in}}%
\pgfpathlineto{\pgfqpoint{4.486740in}{2.876869in}}%
\pgfpathlineto{\pgfqpoint{4.494686in}{2.876869in}}%
\pgfpathlineto{\pgfqpoint{4.494686in}{2.864207in}}%
\pgfpathlineto{\pgfqpoint{4.502631in}{2.864207in}}%
\pgfpathlineto{\pgfqpoint{4.502631in}{2.931033in}}%
\pgfpathlineto{\pgfqpoint{4.510577in}{2.931033in}}%
\pgfpathlineto{\pgfqpoint{4.510577in}{2.998562in}}%
\pgfpathlineto{\pgfqpoint{4.518522in}{2.998562in}}%
\pgfpathlineto{\pgfqpoint{4.518522in}{3.053430in}}%
\pgfpathlineto{\pgfqpoint{4.526468in}{3.053430in}}%
\pgfpathlineto{\pgfqpoint{4.526468in}{3.113925in}}%
\pgfpathlineto{\pgfqpoint{4.534413in}{3.113925in}}%
\pgfpathlineto{\pgfqpoint{4.534413in}{3.196226in}}%
\pgfpathlineto{\pgfqpoint{4.542359in}{3.196226in}}%
\pgfpathlineto{\pgfqpoint{4.542359in}{3.201150in}}%
\pgfpathlineto{\pgfqpoint{4.550304in}{3.201150in}}%
\pgfpathlineto{\pgfqpoint{4.550304in}{3.296113in}}%
\pgfpathlineto{\pgfqpoint{4.558250in}{3.296113in}}%
\pgfpathlineto{\pgfqpoint{4.558250in}{3.383338in}}%
\pgfpathlineto{\pgfqpoint{4.566195in}{3.383338in}}%
\pgfpathlineto{\pgfqpoint{4.566195in}{3.420620in}}%
\pgfpathlineto{\pgfqpoint{4.574141in}{3.420620in}}%
\pgfpathlineto{\pgfqpoint{4.574141in}{3.465639in}}%
\pgfpathlineto{\pgfqpoint{4.582086in}{3.465639in}}%
\pgfpathlineto{\pgfqpoint{4.582086in}{3.520507in}}%
\pgfpathlineto{\pgfqpoint{4.590032in}{3.520507in}}%
\pgfpathlineto{\pgfqpoint{4.590032in}{3.571154in}}%
\pgfpathlineto{\pgfqpoint{4.597977in}{3.571154in}}%
\pgfpathlineto{\pgfqpoint{4.597977in}{3.595070in}}%
\pgfpathlineto{\pgfqpoint{4.605923in}{3.595070in}}%
\pgfpathlineto{\pgfqpoint{4.605923in}{3.668931in}}%
\pgfpathlineto{\pgfqpoint{4.613868in}{3.668931in}}%
\pgfpathlineto{\pgfqpoint{4.613868in}{3.666117in}}%
\pgfpathlineto{\pgfqpoint{4.621814in}{3.666117in}}%
\pgfpathlineto{\pgfqpoint{4.621814in}{3.729426in}}%
\pgfpathlineto{\pgfqpoint{4.629759in}{3.729426in}}%
\pgfpathlineto{\pgfqpoint{4.629759in}{3.747011in}}%
\pgfpathlineto{\pgfqpoint{4.637705in}{3.747011in}}%
\pgfpathlineto{\pgfqpoint{4.637705in}{3.716764in}}%
\pgfpathlineto{\pgfqpoint{4.645650in}{3.716764in}}%
\pgfpathlineto{\pgfqpoint{4.645650in}{3.790624in}}%
\pgfpathlineto{\pgfqpoint{4.653596in}{3.790624in}}%
\pgfpathlineto{\pgfqpoint{4.653596in}{3.768818in}}%
\pgfpathlineto{\pgfqpoint{4.661541in}{3.768818in}}%
\pgfpathlineto{\pgfqpoint{4.661541in}{3.771631in}}%
\pgfpathlineto{\pgfqpoint{4.669487in}{3.771631in}}%
\pgfpathlineto{\pgfqpoint{4.669487in}{3.761080in}}%
\pgfpathlineto{\pgfqpoint{4.677432in}{3.761080in}}%
\pgfpathlineto{\pgfqpoint{4.677432in}{3.754749in}}%
\pgfpathlineto{\pgfqpoint{4.685378in}{3.754749in}}%
\pgfpathlineto{\pgfqpoint{4.685378in}{3.694957in}}%
\pgfpathlineto{\pgfqpoint{4.693323in}{3.694957in}}%
\pgfpathlineto{\pgfqpoint{4.693323in}{3.643607in}}%
\pgfpathlineto{\pgfqpoint{4.701269in}{3.643607in}}%
\pgfpathlineto{\pgfqpoint{4.701269in}{3.578188in}}%
\pgfpathlineto{\pgfqpoint{4.709214in}{3.578188in}}%
\pgfpathlineto{\pgfqpoint{4.709214in}{3.538796in}}%
\pgfpathlineto{\pgfqpoint{4.717160in}{3.538796in}}%
\pgfpathlineto{\pgfqpoint{4.717160in}{3.480411in}}%
\pgfpathlineto{\pgfqpoint{4.725105in}{3.480411in}}%
\pgfpathlineto{\pgfqpoint{4.725105in}{3.453681in}}%
\pgfpathlineto{\pgfqpoint{4.733051in}{3.453681in}}%
\pgfpathlineto{\pgfqpoint{4.733051in}{3.417103in}}%
\pgfpathlineto{\pgfqpoint{4.740996in}{3.417103in}}%
\pgfpathlineto{\pgfqpoint{4.740996in}{3.343946in}}%
\pgfpathlineto{\pgfqpoint{4.748942in}{3.343946in}}%
\pgfpathlineto{\pgfqpoint{4.748942in}{3.306664in}}%
\pgfpathlineto{\pgfqpoint{4.756887in}{3.306664in}}%
\pgfpathlineto{\pgfqpoint{4.756887in}{3.225770in}}%
\pgfpathlineto{\pgfqpoint{4.764833in}{3.225770in}}%
\pgfpathlineto{\pgfqpoint{4.764833in}{3.217329in}}%
\pgfpathlineto{\pgfqpoint{4.772778in}{3.217329in}}%
\pgfpathlineto{\pgfqpoint{4.772778in}{3.174420in}}%
\pgfpathlineto{\pgfqpoint{4.780724in}{3.174420in}}%
\pgfpathlineto{\pgfqpoint{4.780724in}{3.190598in}}%
\pgfpathlineto{\pgfqpoint{4.788669in}{3.190598in}}%
\pgfpathlineto{\pgfqpoint{4.788669in}{3.088601in}}%
\pgfpathlineto{\pgfqpoint{4.796615in}{3.088601in}}%
\pgfpathlineto{\pgfqpoint{4.796615in}{3.034437in}}%
\pgfpathlineto{\pgfqpoint{4.804560in}{3.034437in}}%
\pgfpathlineto{\pgfqpoint{4.804560in}{3.026699in}}%
\pgfpathlineto{\pgfqpoint{4.812506in}{3.026699in}}%
\pgfpathlineto{\pgfqpoint{4.812506in}{2.953543in}}%
\pgfpathlineto{\pgfqpoint{4.820451in}{2.953543in}}%
\pgfpathlineto{\pgfqpoint{4.820451in}{2.942991in}}%
\pgfpathlineto{\pgfqpoint{4.828397in}{2.942991in}}%
\pgfpathlineto{\pgfqpoint{4.828397in}{2.928219in}}%
\pgfpathlineto{\pgfqpoint{4.836342in}{2.928219in}}%
\pgfpathlineto{\pgfqpoint{4.836342in}{2.915557in}}%
\pgfpathlineto{\pgfqpoint{4.844288in}{2.915557in}}%
\pgfpathlineto{\pgfqpoint{4.844288in}{2.843808in}}%
\pgfpathlineto{\pgfqpoint{4.852233in}{2.843808in}}%
\pgfpathlineto{\pgfqpoint{4.852233in}{2.854359in}}%
\pgfpathlineto{\pgfqpoint{4.860179in}{2.854359in}}%
\pgfpathlineto{\pgfqpoint{4.860179in}{2.781202in}}%
\pgfpathlineto{\pgfqpoint{4.868124in}{2.781202in}}%
\pgfpathlineto{\pgfqpoint{4.868124in}{2.779092in}}%
\pgfpathlineto{\pgfqpoint{4.876070in}{2.779092in}}%
\pgfpathlineto{\pgfqpoint{4.876070in}{2.768541in}}%
\pgfpathlineto{\pgfqpoint{4.884015in}{2.768541in}}%
\pgfpathlineto{\pgfqpoint{4.884015in}{2.728445in}}%
\pgfpathlineto{\pgfqpoint{4.891961in}{2.728445in}}%
\pgfpathlineto{\pgfqpoint{4.891961in}{2.718597in}}%
\pgfpathlineto{\pgfqpoint{4.899906in}{2.718597in}}%
\pgfpathlineto{\pgfqpoint{4.899906in}{2.700308in}}%
\pgfpathlineto{\pgfqpoint{4.907852in}{2.700308in}}%
\pgfpathlineto{\pgfqpoint{4.907852in}{2.675688in}}%
\pgfpathlineto{\pgfqpoint{4.915797in}{2.675688in}}%
\pgfpathlineto{\pgfqpoint{4.915797in}{2.662323in}}%
\pgfpathlineto{\pgfqpoint{4.931688in}{2.662323in}}%
\pgfpathlineto{\pgfqpoint{4.931688in}{2.634185in}}%
\pgfpathlineto{\pgfqpoint{4.939634in}{2.634185in}}%
\pgfpathlineto{\pgfqpoint{4.939634in}{2.629965in}}%
\pgfpathlineto{\pgfqpoint{4.947579in}{2.629965in}}%
\pgfpathlineto{\pgfqpoint{4.947579in}{2.620820in}}%
\pgfpathlineto{\pgfqpoint{4.955525in}{2.620820in}}%
\pgfpathlineto{\pgfqpoint{4.955525in}{2.597607in}}%
\pgfpathlineto{\pgfqpoint{4.963470in}{2.597607in}}%
\pgfpathlineto{\pgfqpoint{4.963470in}{2.608862in}}%
\pgfpathlineto{\pgfqpoint{4.971416in}{2.608862in}}%
\pgfpathlineto{\pgfqpoint{4.971416in}{2.586352in}}%
\pgfpathlineto{\pgfqpoint{4.979361in}{2.586352in}}%
\pgfpathlineto{\pgfqpoint{4.979361in}{2.589166in}}%
\pgfpathlineto{\pgfqpoint{4.987307in}{2.589166in}}%
\pgfpathlineto{\pgfqpoint{4.987307in}{2.569470in}}%
\pgfpathlineto{\pgfqpoint{4.995252in}{2.569470in}}%
\pgfpathlineto{\pgfqpoint{4.995252in}{2.580725in}}%
\pgfpathlineto{\pgfqpoint{5.003198in}{2.580725in}}%
\pgfpathlineto{\pgfqpoint{5.003198in}{2.567360in}}%
\pgfpathlineto{\pgfqpoint{5.011143in}{2.567360in}}%
\pgfpathlineto{\pgfqpoint{5.011143in}{2.558919in}}%
\pgfpathlineto{\pgfqpoint{5.019089in}{2.558919in}}%
\pgfpathlineto{\pgfqpoint{5.019089in}{2.563843in}}%
\pgfpathlineto{\pgfqpoint{5.027034in}{2.563843in}}%
\pgfpathlineto{\pgfqpoint{5.027034in}{2.548367in}}%
\pgfpathlineto{\pgfqpoint{5.034980in}{2.548367in}}%
\pgfpathlineto{\pgfqpoint{5.034980in}{2.549774in}}%
\pgfpathlineto{\pgfqpoint{5.042926in}{2.549774in}}%
\pgfpathlineto{\pgfqpoint{5.042926in}{2.533595in}}%
\pgfpathlineto{\pgfqpoint{5.050871in}{2.533595in}}%
\pgfpathlineto{\pgfqpoint{5.050871in}{2.532188in}}%
\pgfpathlineto{\pgfqpoint{5.058817in}{2.532188in}}%
\pgfpathlineto{\pgfqpoint{5.058817in}{2.537112in}}%
\pgfpathlineto{\pgfqpoint{5.066762in}{2.537112in}}%
\pgfpathlineto{\pgfqpoint{5.066762in}{2.523747in}}%
\pgfpathlineto{\pgfqpoint{5.074708in}{2.523747in}}%
\pgfpathlineto{\pgfqpoint{5.074708in}{2.526561in}}%
\pgfpathlineto{\pgfqpoint{5.082653in}{2.526561in}}%
\pgfpathlineto{\pgfqpoint{5.082653in}{2.525154in}}%
\pgfpathlineto{\pgfqpoint{5.090599in}{2.525154in}}%
\pgfpathlineto{\pgfqpoint{5.090599in}{2.527264in}}%
\pgfpathlineto{\pgfqpoint{5.098544in}{2.527264in}}%
\pgfpathlineto{\pgfqpoint{5.098544in}{2.515306in}}%
\pgfpathlineto{\pgfqpoint{5.106490in}{2.515306in}}%
\pgfpathlineto{\pgfqpoint{5.106490in}{2.508975in}}%
\pgfpathlineto{\pgfqpoint{5.114435in}{2.508975in}}%
\pgfpathlineto{\pgfqpoint{5.114435in}{2.514602in}}%
\pgfpathlineto{\pgfqpoint{5.122381in}{2.514602in}}%
\pgfpathlineto{\pgfqpoint{5.122381in}{2.520230in}}%
\pgfpathlineto{\pgfqpoint{5.130326in}{2.520230in}}%
\pgfpathlineto{\pgfqpoint{5.130326in}{2.512492in}}%
\pgfpathlineto{\pgfqpoint{5.138272in}{2.512492in}}%
\pgfpathlineto{\pgfqpoint{5.138272in}{2.505458in}}%
\pgfpathlineto{\pgfqpoint{5.146217in}{2.505458in}}%
\pgfpathlineto{\pgfqpoint{5.146217in}{2.508975in}}%
\pgfpathlineto{\pgfqpoint{5.154163in}{2.508975in}}%
\pgfpathlineto{\pgfqpoint{5.154163in}{2.504754in}}%
\pgfpathlineto{\pgfqpoint{5.170054in}{2.505458in}}%
\pgfpathlineto{\pgfqpoint{5.170054in}{2.502644in}}%
\pgfpathlineto{\pgfqpoint{5.185945in}{2.501941in}}%
\pgfpathlineto{\pgfqpoint{5.185945in}{2.499830in}}%
\pgfpathlineto{\pgfqpoint{5.193890in}{2.499830in}}%
\pgfpathlineto{\pgfqpoint{5.193890in}{2.506865in}}%
\pgfpathlineto{\pgfqpoint{5.201836in}{2.506865in}}%
\pgfpathlineto{\pgfqpoint{5.201836in}{2.501237in}}%
\pgfpathlineto{\pgfqpoint{5.209781in}{2.501237in}}%
\pgfpathlineto{\pgfqpoint{5.209781in}{2.499830in}}%
\pgfpathlineto{\pgfqpoint{5.217727in}{2.499830in}}%
\pgfpathlineto{\pgfqpoint{5.217727in}{2.497720in}}%
\pgfpathlineto{\pgfqpoint{5.225672in}{2.497720in}}%
\pgfpathlineto{\pgfqpoint{5.225672in}{2.500534in}}%
\pgfpathlineto{\pgfqpoint{5.233618in}{2.500534in}}%
\pgfpathlineto{\pgfqpoint{5.233618in}{2.496313in}}%
\pgfpathlineto{\pgfqpoint{5.249509in}{2.496313in}}%
\pgfpathlineto{\pgfqpoint{5.249509in}{2.500534in}}%
\pgfpathlineto{\pgfqpoint{5.257454in}{2.500534in}}%
\pgfpathlineto{\pgfqpoint{5.257454in}{2.497017in}}%
\pgfpathlineto{\pgfqpoint{5.273345in}{2.497720in}}%
\pgfpathlineto{\pgfqpoint{5.273345in}{2.495610in}}%
\pgfpathlineto{\pgfqpoint{5.297182in}{2.494906in}}%
\pgfpathlineto{\pgfqpoint{5.297182in}{2.494203in}}%
\pgfpathlineto{\pgfqpoint{5.305127in}{2.494203in}}%
\pgfpathlineto{\pgfqpoint{5.305127in}{2.497017in}}%
\pgfpathlineto{\pgfqpoint{5.313073in}{2.497017in}}%
\pgfpathlineto{\pgfqpoint{5.313073in}{2.494203in}}%
\pgfpathlineto{\pgfqpoint{5.368691in}{2.493500in}}%
\pgfpathlineto{\pgfqpoint{5.368691in}{2.492796in}}%
\pgfpathlineto{\pgfqpoint{5.440201in}{2.493500in}}%
\pgfpathlineto{\pgfqpoint{5.440201in}{2.494203in}}%
\pgfpathlineto{\pgfqpoint{5.464037in}{2.494203in}}%
\pgfpathlineto{\pgfqpoint{5.464037in}{2.492796in}}%
\pgfpathlineto{\pgfqpoint{5.464037in}{2.492796in}}%
\pgfusepath{stroke}%
\end{pgfscope}%
\begin{pgfscope}%
\pgfpathrectangle{\pgfqpoint{3.510740in}{2.492796in}}{\pgfqpoint{2.540460in}{1.418241in}}%
\pgfusepath{clip}%
\pgfsetbuttcap%
\pgfsetmiterjoin%
\pgfsetlinewidth{1.003750pt}%
\definecolor{currentstroke}{rgb}{0.949020,0.372549,0.360784}%
\pgfsetstrokecolor{currentstroke}%
\pgfsetdash{{1.000000pt}{1.650000pt}}{0.000000pt}%
\pgfpathmoveto{\pgfqpoint{4.176865in}{2.492796in}}%
\pgfpathlineto{\pgfqpoint{4.176865in}{2.506940in}}%
\pgfpathlineto{\pgfqpoint{4.184811in}{2.506940in}}%
\pgfpathlineto{\pgfqpoint{4.184811in}{2.514719in}}%
\pgfpathlineto{\pgfqpoint{4.192756in}{2.514719in}}%
\pgfpathlineto{\pgfqpoint{4.192756in}{2.509061in}}%
\pgfpathlineto{\pgfqpoint{4.200702in}{2.509061in}}%
\pgfpathlineto{\pgfqpoint{4.200702in}{2.511890in}}%
\pgfpathlineto{\pgfqpoint{4.208647in}{2.511890in}}%
\pgfpathlineto{\pgfqpoint{4.208647in}{2.515426in}}%
\pgfpathlineto{\pgfqpoint{4.216593in}{2.515426in}}%
\pgfpathlineto{\pgfqpoint{4.216593in}{2.514011in}}%
\pgfpathlineto{\pgfqpoint{4.224538in}{2.514011in}}%
\pgfpathlineto{\pgfqpoint{4.224538in}{2.521083in}}%
\pgfpathlineto{\pgfqpoint{4.232484in}{2.521083in}}%
\pgfpathlineto{\pgfqpoint{4.232484in}{2.517547in}}%
\pgfpathlineto{\pgfqpoint{4.240429in}{2.517547in}}%
\pgfpathlineto{\pgfqpoint{4.240429in}{2.526033in}}%
\pgfpathlineto{\pgfqpoint{4.248375in}{2.526033in}}%
\pgfpathlineto{\pgfqpoint{4.248375in}{2.528155in}}%
\pgfpathlineto{\pgfqpoint{4.256320in}{2.528155in}}%
\pgfpathlineto{\pgfqpoint{4.256320in}{2.535934in}}%
\pgfpathlineto{\pgfqpoint{4.264266in}{2.535934in}}%
\pgfpathlineto{\pgfqpoint{4.264266in}{2.551492in}}%
\pgfpathlineto{\pgfqpoint{4.272211in}{2.551492in}}%
\pgfpathlineto{\pgfqpoint{4.272211in}{2.549370in}}%
\pgfpathlineto{\pgfqpoint{4.280157in}{2.549370in}}%
\pgfpathlineto{\pgfqpoint{4.280157in}{2.557856in}}%
\pgfpathlineto{\pgfqpoint{4.288102in}{2.557856in}}%
\pgfpathlineto{\pgfqpoint{4.288102in}{2.564221in}}%
\pgfpathlineto{\pgfqpoint{4.296048in}{2.564221in}}%
\pgfpathlineto{\pgfqpoint{4.296048in}{2.561392in}}%
\pgfpathlineto{\pgfqpoint{4.303993in}{2.561392in}}%
\pgfpathlineto{\pgfqpoint{4.303993in}{2.582607in}}%
\pgfpathlineto{\pgfqpoint{4.311939in}{2.582607in}}%
\pgfpathlineto{\pgfqpoint{4.311939in}{2.591801in}}%
\pgfpathlineto{\pgfqpoint{4.319884in}{2.591801in}}%
\pgfpathlineto{\pgfqpoint{4.319884in}{2.603116in}}%
\pgfpathlineto{\pgfqpoint{4.327830in}{2.603116in}}%
\pgfpathlineto{\pgfqpoint{4.327830in}{2.614430in}}%
\pgfpathlineto{\pgfqpoint{4.335775in}{2.614430in}}%
\pgfpathlineto{\pgfqpoint{4.335775in}{2.630695in}}%
\pgfpathlineto{\pgfqpoint{4.343721in}{2.630695in}}%
\pgfpathlineto{\pgfqpoint{4.343721in}{2.646960in}}%
\pgfpathlineto{\pgfqpoint{4.351666in}{2.646960in}}%
\pgfpathlineto{\pgfqpoint{4.351666in}{2.654739in}}%
\pgfpathlineto{\pgfqpoint{4.359612in}{2.654739in}}%
\pgfpathlineto{\pgfqpoint{4.359612in}{2.693634in}}%
\pgfpathlineto{\pgfqpoint{4.367557in}{2.693634in}}%
\pgfpathlineto{\pgfqpoint{4.367557in}{2.711313in}}%
\pgfpathlineto{\pgfqpoint{4.375503in}{2.711313in}}%
\pgfpathlineto{\pgfqpoint{4.375503in}{2.749501in}}%
\pgfpathlineto{\pgfqpoint{4.383448in}{2.749501in}}%
\pgfpathlineto{\pgfqpoint{4.383448in}{2.783445in}}%
\pgfpathlineto{\pgfqpoint{4.391394in}{2.783445in}}%
\pgfpathlineto{\pgfqpoint{4.391394in}{2.818804in}}%
\pgfpathlineto{\pgfqpoint{4.399339in}{2.818804in}}%
\pgfpathlineto{\pgfqpoint{4.399339in}{2.883864in}}%
\pgfpathlineto{\pgfqpoint{4.407285in}{2.883864in}}%
\pgfpathlineto{\pgfqpoint{4.407285in}{2.920637in}}%
\pgfpathlineto{\pgfqpoint{4.415230in}{2.920637in}}%
\pgfpathlineto{\pgfqpoint{4.415230in}{2.958825in}}%
\pgfpathlineto{\pgfqpoint{4.423176in}{2.958825in}}%
\pgfpathlineto{\pgfqpoint{4.423176in}{3.019642in}}%
\pgfpathlineto{\pgfqpoint{4.431121in}{3.019642in}}%
\pgfpathlineto{\pgfqpoint{4.431121in}{3.098138in}}%
\pgfpathlineto{\pgfqpoint{4.439067in}{3.098138in}}%
\pgfpathlineto{\pgfqpoint{4.439067in}{3.182292in}}%
\pgfpathlineto{\pgfqpoint{4.447012in}{3.182292in}}%
\pgfpathlineto{\pgfqpoint{4.447012in}{3.219066in}}%
\pgfpathlineto{\pgfqpoint{4.454958in}{3.219066in}}%
\pgfpathlineto{\pgfqpoint{4.454958in}{3.335042in}}%
\pgfpathlineto{\pgfqpoint{4.462904in}{3.335042in}}%
\pgfpathlineto{\pgfqpoint{4.462904in}{3.410710in}}%
\pgfpathlineto{\pgfqpoint{4.470849in}{3.410710in}}%
\pgfpathlineto{\pgfqpoint{4.470849in}{3.464456in}}%
\pgfpathlineto{\pgfqpoint{4.478795in}{3.464456in}}%
\pgfpathlineto{\pgfqpoint{4.478795in}{3.561339in}}%
\pgfpathlineto{\pgfqpoint{4.486740in}{3.561339in}}%
\pgfpathlineto{\pgfqpoint{4.486740in}{3.576896in}}%
\pgfpathlineto{\pgfqpoint{4.494686in}{3.576896in}}%
\pgfpathlineto{\pgfqpoint{4.494686in}{3.636299in}}%
\pgfpathlineto{\pgfqpoint{4.502631in}{3.636299in}}%
\pgfpathlineto{\pgfqpoint{4.502631in}{3.673072in}}%
\pgfpathlineto{\pgfqpoint{4.518522in}{3.672365in}}%
\pgfpathlineto{\pgfqpoint{4.518522in}{3.763591in}}%
\pgfpathlineto{\pgfqpoint{4.526468in}{3.763591in}}%
\pgfpathlineto{\pgfqpoint{4.526468in}{3.803193in}}%
\pgfpathlineto{\pgfqpoint{4.534413in}{3.803193in}}%
\pgfpathlineto{\pgfqpoint{4.534413in}{3.843502in}}%
\pgfpathlineto{\pgfqpoint{4.542359in}{3.843502in}}%
\pgfpathlineto{\pgfqpoint{4.542359in}{3.839966in}}%
\pgfpathlineto{\pgfqpoint{4.550304in}{3.839966in}}%
\pgfpathlineto{\pgfqpoint{4.550304in}{3.830773in}}%
\pgfpathlineto{\pgfqpoint{4.558250in}{3.830773in}}%
\pgfpathlineto{\pgfqpoint{4.558250in}{3.741668in}}%
\pgfpathlineto{\pgfqpoint{4.566195in}{3.741668in}}%
\pgfpathlineto{\pgfqpoint{4.566195in}{3.755812in}}%
\pgfpathlineto{\pgfqpoint{4.574141in}{3.755812in}}%
\pgfpathlineto{\pgfqpoint{4.574141in}{3.723989in}}%
\pgfpathlineto{\pgfqpoint{4.582086in}{3.723989in}}%
\pgfpathlineto{\pgfqpoint{4.582086in}{3.688630in}}%
\pgfpathlineto{\pgfqpoint{4.590032in}{3.688630in}}%
\pgfpathlineto{\pgfqpoint{4.590032in}{3.655393in}}%
\pgfpathlineto{\pgfqpoint{4.597977in}{3.655393in}}%
\pgfpathlineto{\pgfqpoint{4.597977in}{3.594576in}}%
\pgfpathlineto{\pgfqpoint{4.605923in}{3.594576in}}%
\pgfpathlineto{\pgfqpoint{4.605923in}{3.616498in}}%
\pgfpathlineto{\pgfqpoint{4.613868in}{3.616498in}}%
\pgfpathlineto{\pgfqpoint{4.613868in}{3.499814in}}%
\pgfpathlineto{\pgfqpoint{4.621814in}{3.499814in}}%
\pgfpathlineto{\pgfqpoint{4.621814in}{3.470820in}}%
\pgfpathlineto{\pgfqpoint{4.629759in}{3.470820in}}%
\pgfpathlineto{\pgfqpoint{4.629759in}{3.499814in}}%
\pgfpathlineto{\pgfqpoint{4.637705in}{3.499814in}}%
\pgfpathlineto{\pgfqpoint{4.637705in}{3.354843in}}%
\pgfpathlineto{\pgfqpoint{4.645650in}{3.354843in}}%
\pgfpathlineto{\pgfqpoint{4.645650in}{3.321606in}}%
\pgfpathlineto{\pgfqpoint{4.653596in}{3.321606in}}%
\pgfpathlineto{\pgfqpoint{4.653596in}{3.298269in}}%
\pgfpathlineto{\pgfqpoint{4.661541in}{3.298269in}}%
\pgfpathlineto{\pgfqpoint{4.661541in}{3.257960in}}%
\pgfpathlineto{\pgfqpoint{4.669487in}{3.257960in}}%
\pgfpathlineto{\pgfqpoint{4.669487in}{3.184414in}}%
\pgfpathlineto{\pgfqpoint{4.677432in}{3.184414in}}%
\pgfpathlineto{\pgfqpoint{4.677432in}{3.145519in}}%
\pgfpathlineto{\pgfqpoint{4.685378in}{3.145519in}}%
\pgfpathlineto{\pgfqpoint{4.685378in}{3.104503in}}%
\pgfpathlineto{\pgfqpoint{4.693323in}{3.104503in}}%
\pgfpathlineto{\pgfqpoint{4.693323in}{2.993476in}}%
\pgfpathlineto{\pgfqpoint{4.701269in}{2.993476in}}%
\pgfpathlineto{\pgfqpoint{4.701269in}{3.011156in}}%
\pgfpathlineto{\pgfqpoint{4.709214in}{3.011156in}}%
\pgfpathlineto{\pgfqpoint{4.709214in}{3.009742in}}%
\pgfpathlineto{\pgfqpoint{4.717160in}{3.009742in}}%
\pgfpathlineto{\pgfqpoint{4.717160in}{2.968725in}}%
\pgfpathlineto{\pgfqpoint{4.725105in}{2.968725in}}%
\pgfpathlineto{\pgfqpoint{4.725105in}{2.871842in}}%
\pgfpathlineto{\pgfqpoint{4.733051in}{2.871842in}}%
\pgfpathlineto{\pgfqpoint{4.733051in}{2.890229in}}%
\pgfpathlineto{\pgfqpoint{4.740996in}{2.890229in}}%
\pgfpathlineto{\pgfqpoint{4.740996in}{2.868306in}}%
\pgfpathlineto{\pgfqpoint{4.748942in}{2.868306in}}%
\pgfpathlineto{\pgfqpoint{4.748942in}{2.804661in}}%
\pgfpathlineto{\pgfqpoint{4.756887in}{2.804661in}}%
\pgfpathlineto{\pgfqpoint{4.756887in}{2.794053in}}%
\pgfpathlineto{\pgfqpoint{4.764833in}{2.794053in}}%
\pgfpathlineto{\pgfqpoint{4.764833in}{2.798296in}}%
\pgfpathlineto{\pgfqpoint{4.772778in}{2.798296in}}%
\pgfpathlineto{\pgfqpoint{4.772778in}{2.777081in}}%
\pgfpathlineto{\pgfqpoint{4.780724in}{2.777081in}}%
\pgfpathlineto{\pgfqpoint{4.780724in}{2.767887in}}%
\pgfpathlineto{\pgfqpoint{4.788669in}{2.767887in}}%
\pgfpathlineto{\pgfqpoint{4.788669in}{2.708485in}}%
\pgfpathlineto{\pgfqpoint{4.796615in}{2.708485in}}%
\pgfpathlineto{\pgfqpoint{4.796615in}{2.695048in}}%
\pgfpathlineto{\pgfqpoint{4.804560in}{2.695048in}}%
\pgfpathlineto{\pgfqpoint{4.804560in}{2.704949in}}%
\pgfpathlineto{\pgfqpoint{4.812506in}{2.704949in}}%
\pgfpathlineto{\pgfqpoint{4.812506in}{2.681612in}}%
\pgfpathlineto{\pgfqpoint{4.820451in}{2.681612in}}%
\pgfpathlineto{\pgfqpoint{4.820451in}{2.678076in}}%
\pgfpathlineto{\pgfqpoint{4.828397in}{2.678076in}}%
\pgfpathlineto{\pgfqpoint{4.828397in}{2.642717in}}%
\pgfpathlineto{\pgfqpoint{4.836342in}{2.642717in}}%
\pgfpathlineto{\pgfqpoint{4.836342in}{2.639889in}}%
\pgfpathlineto{\pgfqpoint{4.844288in}{2.639889in}}%
\pgfpathlineto{\pgfqpoint{4.844288in}{2.647668in}}%
\pgfpathlineto{\pgfqpoint{4.852233in}{2.647668in}}%
\pgfpathlineto{\pgfqpoint{4.852233in}{2.617966in}}%
\pgfpathlineto{\pgfqpoint{4.860179in}{2.617966in}}%
\pgfpathlineto{\pgfqpoint{4.860179in}{2.622916in}}%
\pgfpathlineto{\pgfqpoint{4.868124in}{2.622916in}}%
\pgfpathlineto{\pgfqpoint{4.868124in}{2.609480in}}%
\pgfpathlineto{\pgfqpoint{4.876070in}{2.609480in}}%
\pgfpathlineto{\pgfqpoint{4.876070in}{2.601701in}}%
\pgfpathlineto{\pgfqpoint{4.884015in}{2.601701in}}%
\pgfpathlineto{\pgfqpoint{4.884015in}{2.593215in}}%
\pgfpathlineto{\pgfqpoint{4.891961in}{2.593215in}}%
\pgfpathlineto{\pgfqpoint{4.891961in}{2.574828in}}%
\pgfpathlineto{\pgfqpoint{4.899906in}{2.574828in}}%
\pgfpathlineto{\pgfqpoint{4.899906in}{2.584729in}}%
\pgfpathlineto{\pgfqpoint{4.915797in}{2.584022in}}%
\pgfpathlineto{\pgfqpoint{4.915797in}{2.553613in}}%
\pgfpathlineto{\pgfqpoint{4.923743in}{2.553613in}}%
\pgfpathlineto{\pgfqpoint{4.923743in}{2.557149in}}%
\pgfpathlineto{\pgfqpoint{4.931688in}{2.557149in}}%
\pgfpathlineto{\pgfqpoint{4.931688in}{2.541591in}}%
\pgfpathlineto{\pgfqpoint{4.939634in}{2.541591in}}%
\pgfpathlineto{\pgfqpoint{4.939634in}{2.557856in}}%
\pgfpathlineto{\pgfqpoint{4.947579in}{2.557856in}}%
\pgfpathlineto{\pgfqpoint{4.947579in}{2.546541in}}%
\pgfpathlineto{\pgfqpoint{4.955525in}{2.546541in}}%
\pgfpathlineto{\pgfqpoint{4.955525in}{2.537348in}}%
\pgfpathlineto{\pgfqpoint{4.963470in}{2.537348in}}%
\pgfpathlineto{\pgfqpoint{4.963470in}{2.533812in}}%
\pgfpathlineto{\pgfqpoint{4.971416in}{2.533812in}}%
\pgfpathlineto{\pgfqpoint{4.971416in}{2.531691in}}%
\pgfpathlineto{\pgfqpoint{4.979361in}{2.531691in}}%
\pgfpathlineto{\pgfqpoint{4.979361in}{2.538763in}}%
\pgfpathlineto{\pgfqpoint{4.987307in}{2.538763in}}%
\pgfpathlineto{\pgfqpoint{4.987307in}{2.528862in}}%
\pgfpathlineto{\pgfqpoint{4.995252in}{2.528862in}}%
\pgfpathlineto{\pgfqpoint{4.995252in}{2.517547in}}%
\pgfpathlineto{\pgfqpoint{5.003198in}{2.517547in}}%
\pgfpathlineto{\pgfqpoint{5.003198in}{2.524619in}}%
\pgfpathlineto{\pgfqpoint{5.011143in}{2.524619in}}%
\pgfpathlineto{\pgfqpoint{5.011143in}{2.519669in}}%
\pgfpathlineto{\pgfqpoint{5.019089in}{2.519669in}}%
\pgfpathlineto{\pgfqpoint{5.019089in}{2.509061in}}%
\pgfpathlineto{\pgfqpoint{5.027034in}{2.509061in}}%
\pgfpathlineto{\pgfqpoint{5.027034in}{2.515426in}}%
\pgfpathlineto{\pgfqpoint{5.034980in}{2.515426in}}%
\pgfpathlineto{\pgfqpoint{5.034980in}{2.517547in}}%
\pgfpathlineto{\pgfqpoint{5.042926in}{2.517547in}}%
\pgfpathlineto{\pgfqpoint{5.042926in}{2.511890in}}%
\pgfpathlineto{\pgfqpoint{5.050871in}{2.511890in}}%
\pgfpathlineto{\pgfqpoint{5.050871in}{2.507647in}}%
\pgfpathlineto{\pgfqpoint{5.058817in}{2.507647in}}%
\pgfpathlineto{\pgfqpoint{5.058817in}{2.512597in}}%
\pgfpathlineto{\pgfqpoint{5.074708in}{2.511890in}}%
\pgfpathlineto{\pgfqpoint{5.074708in}{2.504111in}}%
\pgfpathlineto{\pgfqpoint{5.082653in}{2.504111in}}%
\pgfpathlineto{\pgfqpoint{5.082653in}{2.502697in}}%
\pgfpathlineto{\pgfqpoint{5.098544in}{2.503404in}}%
\pgfpathlineto{\pgfqpoint{5.098544in}{2.505525in}}%
\pgfpathlineto{\pgfqpoint{5.106490in}{2.505525in}}%
\pgfpathlineto{\pgfqpoint{5.106490in}{2.501989in}}%
\pgfpathlineto{\pgfqpoint{5.114435in}{2.501989in}}%
\pgfpathlineto{\pgfqpoint{5.114435in}{2.499868in}}%
\pgfpathlineto{\pgfqpoint{5.122381in}{2.499868in}}%
\pgfpathlineto{\pgfqpoint{5.122381in}{2.497746in}}%
\pgfpathlineto{\pgfqpoint{5.146217in}{2.498454in}}%
\pgfpathlineto{\pgfqpoint{5.146217in}{2.500575in}}%
\pgfpathlineto{\pgfqpoint{5.154163in}{2.500575in}}%
\pgfpathlineto{\pgfqpoint{5.154163in}{2.497039in}}%
\pgfpathlineto{\pgfqpoint{5.162108in}{2.497039in}}%
\pgfpathlineto{\pgfqpoint{5.162108in}{2.499868in}}%
\pgfpathlineto{\pgfqpoint{5.170054in}{2.499868in}}%
\pgfpathlineto{\pgfqpoint{5.170054in}{2.497746in}}%
\pgfpathlineto{\pgfqpoint{5.185945in}{2.497039in}}%
\pgfpathlineto{\pgfqpoint{5.185945in}{2.494918in}}%
\pgfpathlineto{\pgfqpoint{5.193890in}{2.494918in}}%
\pgfpathlineto{\pgfqpoint{5.193890in}{2.496332in}}%
\pgfpathlineto{\pgfqpoint{5.217727in}{2.495625in}}%
\pgfpathlineto{\pgfqpoint{5.217727in}{2.494210in}}%
\pgfpathlineto{\pgfqpoint{5.225672in}{2.494210in}}%
\pgfpathlineto{\pgfqpoint{5.225672in}{2.497039in}}%
\pgfpathlineto{\pgfqpoint{5.233618in}{2.497039in}}%
\pgfpathlineto{\pgfqpoint{5.233618in}{2.494918in}}%
\pgfpathlineto{\pgfqpoint{5.249509in}{2.494918in}}%
\pgfpathlineto{\pgfqpoint{5.249509in}{2.496332in}}%
\pgfpathlineto{\pgfqpoint{5.257454in}{2.496332in}}%
\pgfpathlineto{\pgfqpoint{5.257454in}{2.493503in}}%
\pgfpathlineto{\pgfqpoint{5.464037in}{2.492796in}}%
\pgfpathlineto{\pgfqpoint{5.464037in}{2.492796in}}%
\pgfusepath{stroke}%
\end{pgfscope}%
\begin{pgfscope}%
\pgfsetrectcap%
\pgfsetmiterjoin%
\pgfsetlinewidth{0.803000pt}%
\definecolor{currentstroke}{rgb}{0.000000,0.000000,0.000000}%
\pgfsetstrokecolor{currentstroke}%
\pgfsetdash{}{0pt}%
\pgfpathmoveto{\pgfqpoint{3.510740in}{2.492796in}}%
\pgfpathlineto{\pgfqpoint{3.510740in}{3.911037in}}%
\pgfusepath{stroke}%
\end{pgfscope}%
\begin{pgfscope}%
\pgfsetrectcap%
\pgfsetmiterjoin%
\pgfsetlinewidth{0.803000pt}%
\definecolor{currentstroke}{rgb}{0.000000,0.000000,0.000000}%
\pgfsetstrokecolor{currentstroke}%
\pgfsetdash{}{0pt}%
\pgfpathmoveto{\pgfqpoint{6.051200in}{2.492796in}}%
\pgfpathlineto{\pgfqpoint{6.051200in}{3.911037in}}%
\pgfusepath{stroke}%
\end{pgfscope}%
\begin{pgfscope}%
\pgfsetrectcap%
\pgfsetmiterjoin%
\pgfsetlinewidth{0.803000pt}%
\definecolor{currentstroke}{rgb}{0.000000,0.000000,0.000000}%
\pgfsetstrokecolor{currentstroke}%
\pgfsetdash{}{0pt}%
\pgfpathmoveto{\pgfqpoint{3.510740in}{2.492796in}}%
\pgfpathlineto{\pgfqpoint{6.051200in}{2.492796in}}%
\pgfusepath{stroke}%
\end{pgfscope}%
\begin{pgfscope}%
\pgfsetrectcap%
\pgfsetmiterjoin%
\pgfsetlinewidth{0.803000pt}%
\definecolor{currentstroke}{rgb}{0.000000,0.000000,0.000000}%
\pgfsetstrokecolor{currentstroke}%
\pgfsetdash{}{0pt}%
\pgfpathmoveto{\pgfqpoint{3.510740in}{3.911037in}}%
\pgfpathlineto{\pgfqpoint{6.051200in}{3.911037in}}%
\pgfusepath{stroke}%
\end{pgfscope}%
\begin{pgfscope}%
\definecolor{textcolor}{rgb}{0.000000,0.000000,0.000000}%
\pgfsetstrokecolor{textcolor}%
\pgfsetfillcolor{textcolor}%
\pgftext[x=3.510740in,y=3.994370in,left,base]{\color{textcolor}\rmfamily\fontsize{10.000000}{12.000000}\selectfont Bin [1.83, 2.0), 65,319 events}%
\end{pgfscope}%
\begin{pgfscope}%
\pgfsetbuttcap%
\pgfsetmiterjoin%
\definecolor{currentfill}{rgb}{1.000000,1.000000,1.000000}%
\pgfsetfillcolor{currentfill}%
\pgfsetfillopacity{0.800000}%
\pgfsetlinewidth{1.003750pt}%
\definecolor{currentstroke}{rgb}{0.800000,0.800000,0.800000}%
\pgfsetstrokecolor{currentstroke}%
\pgfsetstrokeopacity{0.800000}%
\pgfsetdash{}{0pt}%
\pgfpathmoveto{\pgfqpoint{5.016422in}{3.511481in}}%
\pgfpathlineto{\pgfqpoint{5.973422in}{3.511481in}}%
\pgfpathquadraticcurveto{\pgfqpoint{5.995644in}{3.511481in}}{\pgfqpoint{5.995644in}{3.533704in}}%
\pgfpathlineto{\pgfqpoint{5.995644in}{3.833259in}}%
\pgfpathquadraticcurveto{\pgfqpoint{5.995644in}{3.855481in}}{\pgfqpoint{5.973422in}{3.855481in}}%
\pgfpathlineto{\pgfqpoint{5.016422in}{3.855481in}}%
\pgfpathquadraticcurveto{\pgfqpoint{4.994200in}{3.855481in}}{\pgfqpoint{4.994200in}{3.833259in}}%
\pgfpathlineto{\pgfqpoint{4.994200in}{3.533704in}}%
\pgfpathquadraticcurveto{\pgfqpoint{4.994200in}{3.511481in}}{\pgfqpoint{5.016422in}{3.511481in}}%
\pgfpathclose%
\pgfusepath{stroke,fill}%
\end{pgfscope}%
\begin{pgfscope}%
\pgfsetbuttcap%
\pgfsetmiterjoin%
\pgfsetlinewidth{1.003750pt}%
\definecolor{currentstroke}{rgb}{0.313725,0.317647,0.309804}%
\pgfsetstrokecolor{currentstroke}%
\pgfsetdash{}{0pt}%
\pgfpathmoveto{\pgfqpoint{5.038644in}{3.732815in}}%
\pgfpathlineto{\pgfqpoint{5.260867in}{3.732815in}}%
\pgfpathlineto{\pgfqpoint{5.260867in}{3.810593in}}%
\pgfpathlineto{\pgfqpoint{5.038644in}{3.810593in}}%
\pgfpathclose%
\pgfusepath{stroke}%
\end{pgfscope}%
\begin{pgfscope}%
\definecolor{textcolor}{rgb}{0.000000,0.000000,0.000000}%
\pgfsetstrokecolor{textcolor}%
\pgfsetfillcolor{textcolor}%
\pgftext[x=5.349756in,y=3.732815in,left,base]{\color{textcolor}\rmfamily\fontsize{8.000000}{9.600000}\selectfont IQR = 0.17}%
\end{pgfscope}%
\begin{pgfscope}%
\pgfsetbuttcap%
\pgfsetmiterjoin%
\pgfsetlinewidth{1.003750pt}%
\definecolor{currentstroke}{rgb}{0.949020,0.372549,0.360784}%
\pgfsetstrokecolor{currentstroke}%
\pgfsetdash{{1.000000pt}{1.650000pt}}{0.000000pt}%
\pgfpathmoveto{\pgfqpoint{5.038644in}{3.577481in}}%
\pgfpathlineto{\pgfqpoint{5.260867in}{3.577481in}}%
\pgfpathlineto{\pgfqpoint{5.260867in}{3.655259in}}%
\pgfpathlineto{\pgfqpoint{5.038644in}{3.655259in}}%
\pgfpathclose%
\pgfusepath{stroke}%
\end{pgfscope}%
\begin{pgfscope}%
\definecolor{textcolor}{rgb}{0.000000,0.000000,0.000000}%
\pgfsetstrokecolor{textcolor}%
\pgfsetfillcolor{textcolor}%
\pgftext[x=5.349756in,y=3.577481in,left,base]{\color{textcolor}\rmfamily\fontsize{8.000000}{9.600000}\selectfont IQR = 0.17}%
\end{pgfscope}%
\begin{pgfscope}%
\pgfsetbuttcap%
\pgfsetmiterjoin%
\definecolor{currentfill}{rgb}{1.000000,1.000000,1.000000}%
\pgfsetfillcolor{currentfill}%
\pgfsetlinewidth{0.000000pt}%
\definecolor{currentstroke}{rgb}{0.000000,0.000000,0.000000}%
\pgfsetstrokecolor{currentstroke}%
\pgfsetstrokeopacity{0.000000}%
\pgfsetdash{}{0pt}%
\pgfpathmoveto{\pgfqpoint{0.485140in}{0.541166in}}%
\pgfpathlineto{\pgfqpoint{3.025600in}{0.541166in}}%
\pgfpathlineto{\pgfqpoint{3.025600in}{1.959407in}}%
\pgfpathlineto{\pgfqpoint{0.485140in}{1.959407in}}%
\pgfpathclose%
\pgfusepath{fill}%
\end{pgfscope}%
\begin{pgfscope}%
\pgfsetbuttcap%
\pgfsetroundjoin%
\definecolor{currentfill}{rgb}{0.000000,0.000000,0.000000}%
\pgfsetfillcolor{currentfill}%
\pgfsetlinewidth{0.803000pt}%
\definecolor{currentstroke}{rgb}{0.000000,0.000000,0.000000}%
\pgfsetstrokecolor{currentstroke}%
\pgfsetdash{}{0pt}%
\pgfsys@defobject{currentmarker}{\pgfqpoint{0.000000in}{-0.048611in}}{\pgfqpoint{0.000000in}{0.000000in}}{%
\pgfpathmoveto{\pgfqpoint{0.000000in}{0.000000in}}%
\pgfpathlineto{\pgfqpoint{0.000000in}{-0.048611in}}%
\pgfusepath{stroke,fill}%
}%
\begin{pgfscope}%
\pgfsys@transformshift{0.916345in}{0.541166in}%
\pgfsys@useobject{currentmarker}{}%
\end{pgfscope}%
\end{pgfscope}%
\begin{pgfscope}%
\definecolor{textcolor}{rgb}{0.000000,0.000000,0.000000}%
\pgfsetstrokecolor{textcolor}%
\pgfsetfillcolor{textcolor}%
\pgftext[x=0.916345in,y=0.443944in,,top]{\color{textcolor}\rmfamily\fontsize{8.000000}{9.600000}\selectfont \(\displaystyle {-1}\)}%
\end{pgfscope}%
\begin{pgfscope}%
\pgfsetbuttcap%
\pgfsetroundjoin%
\definecolor{currentfill}{rgb}{0.000000,0.000000,0.000000}%
\pgfsetfillcolor{currentfill}%
\pgfsetlinewidth{0.803000pt}%
\definecolor{currentstroke}{rgb}{0.000000,0.000000,0.000000}%
\pgfsetstrokecolor{currentstroke}%
\pgfsetdash{}{0pt}%
\pgfsys@defobject{currentmarker}{\pgfqpoint{0.000000in}{-0.048611in}}{\pgfqpoint{0.000000in}{0.000000in}}{%
\pgfpathmoveto{\pgfqpoint{0.000000in}{0.000000in}}%
\pgfpathlineto{\pgfqpoint{0.000000in}{-0.048611in}}%
\pgfusepath{stroke,fill}%
}%
\begin{pgfscope}%
\pgfsys@transformshift{1.602799in}{0.541166in}%
\pgfsys@useobject{currentmarker}{}%
\end{pgfscope}%
\end{pgfscope}%
\begin{pgfscope}%
\definecolor{textcolor}{rgb}{0.000000,0.000000,0.000000}%
\pgfsetstrokecolor{textcolor}%
\pgfsetfillcolor{textcolor}%
\pgftext[x=1.602799in,y=0.443944in,,top]{\color{textcolor}\rmfamily\fontsize{8.000000}{9.600000}\selectfont \(\displaystyle {0}\)}%
\end{pgfscope}%
\begin{pgfscope}%
\pgfsetbuttcap%
\pgfsetroundjoin%
\definecolor{currentfill}{rgb}{0.000000,0.000000,0.000000}%
\pgfsetfillcolor{currentfill}%
\pgfsetlinewidth{0.803000pt}%
\definecolor{currentstroke}{rgb}{0.000000,0.000000,0.000000}%
\pgfsetstrokecolor{currentstroke}%
\pgfsetdash{}{0pt}%
\pgfsys@defobject{currentmarker}{\pgfqpoint{0.000000in}{-0.048611in}}{\pgfqpoint{0.000000in}{0.000000in}}{%
\pgfpathmoveto{\pgfqpoint{0.000000in}{0.000000in}}%
\pgfpathlineto{\pgfqpoint{0.000000in}{-0.048611in}}%
\pgfusepath{stroke,fill}%
}%
\begin{pgfscope}%
\pgfsys@transformshift{2.289253in}{0.541166in}%
\pgfsys@useobject{currentmarker}{}%
\end{pgfscope}%
\end{pgfscope}%
\begin{pgfscope}%
\definecolor{textcolor}{rgb}{0.000000,0.000000,0.000000}%
\pgfsetstrokecolor{textcolor}%
\pgfsetfillcolor{textcolor}%
\pgftext[x=2.289253in,y=0.443944in,,top]{\color{textcolor}\rmfamily\fontsize{8.000000}{9.600000}\selectfont \(\displaystyle {1}\)}%
\end{pgfscope}%
\begin{pgfscope}%
\pgfsetbuttcap%
\pgfsetroundjoin%
\definecolor{currentfill}{rgb}{0.000000,0.000000,0.000000}%
\pgfsetfillcolor{currentfill}%
\pgfsetlinewidth{0.803000pt}%
\definecolor{currentstroke}{rgb}{0.000000,0.000000,0.000000}%
\pgfsetstrokecolor{currentstroke}%
\pgfsetdash{}{0pt}%
\pgfsys@defobject{currentmarker}{\pgfqpoint{0.000000in}{-0.048611in}}{\pgfqpoint{0.000000in}{0.000000in}}{%
\pgfpathmoveto{\pgfqpoint{0.000000in}{0.000000in}}%
\pgfpathlineto{\pgfqpoint{0.000000in}{-0.048611in}}%
\pgfusepath{stroke,fill}%
}%
\begin{pgfscope}%
\pgfsys@transformshift{2.975706in}{0.541166in}%
\pgfsys@useobject{currentmarker}{}%
\end{pgfscope}%
\end{pgfscope}%
\begin{pgfscope}%
\definecolor{textcolor}{rgb}{0.000000,0.000000,0.000000}%
\pgfsetstrokecolor{textcolor}%
\pgfsetfillcolor{textcolor}%
\pgftext[x=2.975706in,y=0.443944in,,top]{\color{textcolor}\rmfamily\fontsize{8.000000}{9.600000}\selectfont \(\displaystyle {2}\)}%
\end{pgfscope}%
\begin{pgfscope}%
\definecolor{textcolor}{rgb}{0.000000,0.000000,0.000000}%
\pgfsetstrokecolor{textcolor}%
\pgfsetfillcolor{textcolor}%
\pgftext[x=1.755370in,y=0.289722in,,top]{\color{textcolor}\rmfamily\fontsize{10.000000}{12.000000}\selectfont \(\displaystyle \log_{10}(E_{\textup{true}}) - \log_{10}(E_{\textup{reco}}) \, \left[ E / \textup{GeV} \right]\)}%
\end{pgfscope}%
\begin{pgfscope}%
\pgfsetbuttcap%
\pgfsetroundjoin%
\definecolor{currentfill}{rgb}{0.000000,0.000000,0.000000}%
\pgfsetfillcolor{currentfill}%
\pgfsetlinewidth{0.803000pt}%
\definecolor{currentstroke}{rgb}{0.000000,0.000000,0.000000}%
\pgfsetstrokecolor{currentstroke}%
\pgfsetdash{}{0pt}%
\pgfsys@defobject{currentmarker}{\pgfqpoint{-0.048611in}{0.000000in}}{\pgfqpoint{-0.000000in}{0.000000in}}{%
\pgfpathmoveto{\pgfqpoint{-0.000000in}{0.000000in}}%
\pgfpathlineto{\pgfqpoint{-0.048611in}{0.000000in}}%
\pgfusepath{stroke,fill}%
}%
\begin{pgfscope}%
\pgfsys@transformshift{0.485140in}{0.541166in}%
\pgfsys@useobject{currentmarker}{}%
\end{pgfscope}%
\end{pgfscope}%
\begin{pgfscope}%
\definecolor{textcolor}{rgb}{0.000000,0.000000,0.000000}%
\pgfsetstrokecolor{textcolor}%
\pgfsetfillcolor{textcolor}%
\pgftext[x=0.328889in, y=0.502611in, left, base]{\color{textcolor}\rmfamily\fontsize{8.000000}{9.600000}\selectfont \(\displaystyle {0}\)}%
\end{pgfscope}%
\begin{pgfscope}%
\pgfsetbuttcap%
\pgfsetroundjoin%
\definecolor{currentfill}{rgb}{0.000000,0.000000,0.000000}%
\pgfsetfillcolor{currentfill}%
\pgfsetlinewidth{0.803000pt}%
\definecolor{currentstroke}{rgb}{0.000000,0.000000,0.000000}%
\pgfsetstrokecolor{currentstroke}%
\pgfsetdash{}{0pt}%
\pgfsys@defobject{currentmarker}{\pgfqpoint{-0.048611in}{0.000000in}}{\pgfqpoint{-0.000000in}{0.000000in}}{%
\pgfpathmoveto{\pgfqpoint{-0.000000in}{0.000000in}}%
\pgfpathlineto{\pgfqpoint{-0.048611in}{0.000000in}}%
\pgfusepath{stroke,fill}%
}%
\begin{pgfscope}%
\pgfsys@transformshift{0.485140in}{1.060003in}%
\pgfsys@useobject{currentmarker}{}%
\end{pgfscope}%
\end{pgfscope}%
\begin{pgfscope}%
\definecolor{textcolor}{rgb}{0.000000,0.000000,0.000000}%
\pgfsetstrokecolor{textcolor}%
\pgfsetfillcolor{textcolor}%
\pgftext[x=0.328889in, y=1.021448in, left, base]{\color{textcolor}\rmfamily\fontsize{8.000000}{9.600000}\selectfont \(\displaystyle {1}\)}%
\end{pgfscope}%
\begin{pgfscope}%
\pgfsetbuttcap%
\pgfsetroundjoin%
\definecolor{currentfill}{rgb}{0.000000,0.000000,0.000000}%
\pgfsetfillcolor{currentfill}%
\pgfsetlinewidth{0.803000pt}%
\definecolor{currentstroke}{rgb}{0.000000,0.000000,0.000000}%
\pgfsetstrokecolor{currentstroke}%
\pgfsetdash{}{0pt}%
\pgfsys@defobject{currentmarker}{\pgfqpoint{-0.048611in}{0.000000in}}{\pgfqpoint{-0.000000in}{0.000000in}}{%
\pgfpathmoveto{\pgfqpoint{-0.000000in}{0.000000in}}%
\pgfpathlineto{\pgfqpoint{-0.048611in}{0.000000in}}%
\pgfusepath{stroke,fill}%
}%
\begin{pgfscope}%
\pgfsys@transformshift{0.485140in}{1.578840in}%
\pgfsys@useobject{currentmarker}{}%
\end{pgfscope}%
\end{pgfscope}%
\begin{pgfscope}%
\definecolor{textcolor}{rgb}{0.000000,0.000000,0.000000}%
\pgfsetstrokecolor{textcolor}%
\pgfsetfillcolor{textcolor}%
\pgftext[x=0.328889in, y=1.540285in, left, base]{\color{textcolor}\rmfamily\fontsize{8.000000}{9.600000}\selectfont \(\displaystyle {2}\)}%
\end{pgfscope}%
\begin{pgfscope}%
\definecolor{textcolor}{rgb}{0.000000,0.000000,0.000000}%
\pgfsetstrokecolor{textcolor}%
\pgfsetfillcolor{textcolor}%
\pgftext[x=0.273333in,y=1.250287in,,bottom,rotate=90.000000]{\color{textcolor}\rmfamily\fontsize{10.000000}{12.000000}\selectfont Density}%
\end{pgfscope}%
\begin{pgfscope}%
\pgfpathrectangle{\pgfqpoint{0.485140in}{0.541166in}}{\pgfqpoint{2.540460in}{1.418241in}}%
\pgfusepath{clip}%
\pgfsetbuttcap%
\pgfsetmiterjoin%
\pgfsetlinewidth{1.003750pt}%
\definecolor{currentstroke}{rgb}{0.313725,0.317647,0.309804}%
\pgfsetstrokecolor{currentstroke}%
\pgfsetdash{}{0pt}%
\pgfpathmoveto{\pgfqpoint{1.313774in}{0.541166in}}%
\pgfpathlineto{\pgfqpoint{1.313774in}{0.545707in}}%
\pgfpathlineto{\pgfqpoint{1.328767in}{0.545707in}}%
\pgfpathlineto{\pgfqpoint{1.328767in}{0.543437in}}%
\pgfpathlineto{\pgfqpoint{1.343759in}{0.543437in}}%
\pgfpathlineto{\pgfqpoint{1.343759in}{0.552518in}}%
\pgfpathlineto{\pgfqpoint{1.358751in}{0.552518in}}%
\pgfpathlineto{\pgfqpoint{1.358751in}{0.560464in}}%
\pgfpathlineto{\pgfqpoint{1.373743in}{0.560464in}}%
\pgfpathlineto{\pgfqpoint{1.373743in}{0.567275in}}%
\pgfpathlineto{\pgfqpoint{1.388736in}{0.567275in}}%
\pgfpathlineto{\pgfqpoint{1.388736in}{0.599058in}}%
\pgfpathlineto{\pgfqpoint{1.403728in}{0.599058in}}%
\pgfpathlineto{\pgfqpoint{1.403728in}{0.639923in}}%
\pgfpathlineto{\pgfqpoint{1.418720in}{0.639923in}}%
\pgfpathlineto{\pgfqpoint{1.418720in}{0.688734in}}%
\pgfpathlineto{\pgfqpoint{1.433712in}{0.688734in}}%
\pgfpathlineto{\pgfqpoint{1.433712in}{0.746626in}}%
\pgfpathlineto{\pgfqpoint{1.448705in}{0.746626in}}%
\pgfpathlineto{\pgfqpoint{1.448705in}{0.805653in}}%
\pgfpathlineto{\pgfqpoint{1.463697in}{0.805653in}}%
\pgfpathlineto{\pgfqpoint{1.463697in}{0.901004in}}%
\pgfpathlineto{\pgfqpoint{1.478689in}{0.901004in}}%
\pgfpathlineto{\pgfqpoint{1.478689in}{1.037221in}}%
\pgfpathlineto{\pgfqpoint{1.493681in}{1.037221in}}%
\pgfpathlineto{\pgfqpoint{1.493681in}{1.151869in}}%
\pgfpathlineto{\pgfqpoint{1.508674in}{1.151869in}}%
\pgfpathlineto{\pgfqpoint{1.508674in}{1.256302in}}%
\pgfpathlineto{\pgfqpoint{1.523666in}{1.256302in}}%
\pgfpathlineto{\pgfqpoint{1.523666in}{1.290356in}}%
\pgfpathlineto{\pgfqpoint{1.538658in}{1.290356in}}%
\pgfpathlineto{\pgfqpoint{1.538658in}{1.395924in}}%
\pgfpathlineto{\pgfqpoint{1.553650in}{1.395924in}}%
\pgfpathlineto{\pgfqpoint{1.553650in}{1.424302in}}%
\pgfpathlineto{\pgfqpoint{1.568643in}{1.424302in}}%
\pgfpathlineto{\pgfqpoint{1.568643in}{1.513978in}}%
\pgfpathlineto{\pgfqpoint{1.583635in}{1.513978in}}%
\pgfpathlineto{\pgfqpoint{1.583635in}{1.578681in}}%
\pgfpathlineto{\pgfqpoint{1.598627in}{1.578681in}}%
\pgfpathlineto{\pgfqpoint{1.598627in}{1.599113in}}%
\pgfpathlineto{\pgfqpoint{1.613619in}{1.599113in}}%
\pgfpathlineto{\pgfqpoint{1.613619in}{1.587762in}}%
\pgfpathlineto{\pgfqpoint{1.628612in}{1.587762in}}%
\pgfpathlineto{\pgfqpoint{1.628612in}{1.498086in}}%
\pgfpathlineto{\pgfqpoint{1.643604in}{1.498086in}}%
\pgfpathlineto{\pgfqpoint{1.643604in}{1.543491in}}%
\pgfpathlineto{\pgfqpoint{1.658596in}{1.543491in}}%
\pgfpathlineto{\pgfqpoint{1.658596in}{1.482194in}}%
\pgfpathlineto{\pgfqpoint{1.673588in}{1.482194in}}%
\pgfpathlineto{\pgfqpoint{1.673588in}{1.518518in}}%
\pgfpathlineto{\pgfqpoint{1.688581in}{1.518518in}}%
\pgfpathlineto{\pgfqpoint{1.688581in}{1.369816in}}%
\pgfpathlineto{\pgfqpoint{1.703573in}{1.369816in}}%
\pgfpathlineto{\pgfqpoint{1.703573in}{1.293761in}}%
\pgfpathlineto{\pgfqpoint{1.718565in}{1.293761in}}%
\pgfpathlineto{\pgfqpoint{1.718565in}{1.261978in}}%
\pgfpathlineto{\pgfqpoint{1.733557in}{1.261978in}}%
\pgfpathlineto{\pgfqpoint{1.733557in}{1.204086in}}%
\pgfpathlineto{\pgfqpoint{1.748550in}{1.204086in}}%
\pgfpathlineto{\pgfqpoint{1.748550in}{1.131437in}}%
\pgfpathlineto{\pgfqpoint{1.763542in}{1.131437in}}%
\pgfpathlineto{\pgfqpoint{1.763542in}{1.118951in}}%
\pgfpathlineto{\pgfqpoint{1.778534in}{1.118951in}}%
\pgfpathlineto{\pgfqpoint{1.778534in}{1.057653in}}%
\pgfpathlineto{\pgfqpoint{1.793526in}{1.057653in}}%
\pgfpathlineto{\pgfqpoint{1.793526in}{0.985004in}}%
\pgfpathlineto{\pgfqpoint{1.808519in}{0.985004in}}%
\pgfpathlineto{\pgfqpoint{1.808519in}{0.996356in}}%
\pgfpathlineto{\pgfqpoint{1.823511in}{0.996356in}}%
\pgfpathlineto{\pgfqpoint{1.823511in}{0.924842in}}%
\pgfpathlineto{\pgfqpoint{1.838503in}{0.924842in}}%
\pgfpathlineto{\pgfqpoint{1.838503in}{0.921437in}}%
\pgfpathlineto{\pgfqpoint{1.853495in}{0.921437in}}%
\pgfpathlineto{\pgfqpoint{1.853495in}{0.876031in}}%
\pgfpathlineto{\pgfqpoint{1.868488in}{0.876031in}}%
\pgfpathlineto{\pgfqpoint{1.868488in}{0.823815in}}%
\pgfpathlineto{\pgfqpoint{1.883480in}{0.823815in}}%
\pgfpathlineto{\pgfqpoint{1.883480in}{0.877167in}}%
\pgfpathlineto{\pgfqpoint{1.898472in}{0.877167in}}%
\pgfpathlineto{\pgfqpoint{1.898472in}{0.824950in}}%
\pgfpathlineto{\pgfqpoint{1.913464in}{0.824950in}}%
\pgfpathlineto{\pgfqpoint{1.913464in}{0.782950in}}%
\pgfpathlineto{\pgfqpoint{1.928457in}{0.782950in}}%
\pgfpathlineto{\pgfqpoint{1.928457in}{0.796572in}}%
\pgfpathlineto{\pgfqpoint{1.943449in}{0.796572in}}%
\pgfpathlineto{\pgfqpoint{1.943449in}{0.779545in}}%
\pgfpathlineto{\pgfqpoint{1.958441in}{0.779545in}}%
\pgfpathlineto{\pgfqpoint{1.958441in}{0.729599in}}%
\pgfpathlineto{\pgfqpoint{1.973433in}{0.729599in}}%
\pgfpathlineto{\pgfqpoint{1.973433in}{0.739815in}}%
\pgfpathlineto{\pgfqpoint{1.988426in}{0.739815in}}%
\pgfpathlineto{\pgfqpoint{1.988426in}{0.722788in}}%
\pgfpathlineto{\pgfqpoint{2.003418in}{0.722788in}}%
\pgfpathlineto{\pgfqpoint{2.003418in}{0.676248in}}%
\pgfpathlineto{\pgfqpoint{2.018410in}{0.676248in}}%
\pgfpathlineto{\pgfqpoint{2.018410in}{0.703491in}}%
\pgfpathlineto{\pgfqpoint{2.033402in}{0.703491in}}%
\pgfpathlineto{\pgfqpoint{2.033402in}{0.670572in}}%
\pgfpathlineto{\pgfqpoint{2.048395in}{0.670572in}}%
\pgfpathlineto{\pgfqpoint{2.048395in}{0.634248in}}%
\pgfpathlineto{\pgfqpoint{2.063387in}{0.634248in}}%
\pgfpathlineto{\pgfqpoint{2.063387in}{0.629707in}}%
\pgfpathlineto{\pgfqpoint{2.078379in}{0.629707in}}%
\pgfpathlineto{\pgfqpoint{2.078379in}{0.622896in}}%
\pgfpathlineto{\pgfqpoint{2.093371in}{0.622896in}}%
\pgfpathlineto{\pgfqpoint{2.093371in}{0.633112in}}%
\pgfpathlineto{\pgfqpoint{2.108364in}{0.633112in}}%
\pgfpathlineto{\pgfqpoint{2.108364in}{0.609275in}}%
\pgfpathlineto{\pgfqpoint{2.123356in}{0.609275in}}%
\pgfpathlineto{\pgfqpoint{2.123356in}{0.601329in}}%
\pgfpathlineto{\pgfqpoint{2.138348in}{0.601329in}}%
\pgfpathlineto{\pgfqpoint{2.138348in}{0.611545in}}%
\pgfpathlineto{\pgfqpoint{2.153340in}{0.611545in}}%
\pgfpathlineto{\pgfqpoint{2.153340in}{0.601329in}}%
\pgfpathlineto{\pgfqpoint{2.168333in}{0.601329in}}%
\pgfpathlineto{\pgfqpoint{2.168333in}{0.595653in}}%
\pgfpathlineto{\pgfqpoint{2.183325in}{0.595653in}}%
\pgfpathlineto{\pgfqpoint{2.183325in}{0.577491in}}%
\pgfpathlineto{\pgfqpoint{2.198317in}{0.577491in}}%
\pgfpathlineto{\pgfqpoint{2.198317in}{0.572950in}}%
\pgfpathlineto{\pgfqpoint{2.213309in}{0.572950in}}%
\pgfpathlineto{\pgfqpoint{2.213309in}{0.565004in}}%
\pgfpathlineto{\pgfqpoint{2.243294in}{0.565004in}}%
\pgfpathlineto{\pgfqpoint{2.243294in}{0.571815in}}%
\pgfpathlineto{\pgfqpoint{2.258286in}{0.571815in}}%
\pgfpathlineto{\pgfqpoint{2.258286in}{0.563869in}}%
\pgfpathlineto{\pgfqpoint{2.273278in}{0.563869in}}%
\pgfpathlineto{\pgfqpoint{2.273278in}{0.560464in}}%
\pgfpathlineto{\pgfqpoint{2.288271in}{0.560464in}}%
\pgfpathlineto{\pgfqpoint{2.288271in}{0.557058in}}%
\pgfpathlineto{\pgfqpoint{2.303263in}{0.557058in}}%
\pgfpathlineto{\pgfqpoint{2.303263in}{0.549112in}}%
\pgfpathlineto{\pgfqpoint{2.318255in}{0.549112in}}%
\pgfpathlineto{\pgfqpoint{2.318255in}{0.558193in}}%
\pgfpathlineto{\pgfqpoint{2.333247in}{0.558193in}}%
\pgfpathlineto{\pgfqpoint{2.333247in}{0.552518in}}%
\pgfpathlineto{\pgfqpoint{2.348240in}{0.552518in}}%
\pgfpathlineto{\pgfqpoint{2.348240in}{0.550247in}}%
\pgfpathlineto{\pgfqpoint{2.363232in}{0.550247in}}%
\pgfpathlineto{\pgfqpoint{2.363232in}{0.545707in}}%
\pgfpathlineto{\pgfqpoint{2.378224in}{0.545707in}}%
\pgfpathlineto{\pgfqpoint{2.378224in}{0.551383in}}%
\pgfpathlineto{\pgfqpoint{2.393216in}{0.551383in}}%
\pgfpathlineto{\pgfqpoint{2.393216in}{0.545707in}}%
\pgfpathlineto{\pgfqpoint{2.408209in}{0.545707in}}%
\pgfpathlineto{\pgfqpoint{2.408209in}{0.546842in}}%
\pgfpathlineto{\pgfqpoint{2.453185in}{0.546842in}}%
\pgfpathlineto{\pgfqpoint{2.453185in}{0.545707in}}%
\pgfpathlineto{\pgfqpoint{2.468178in}{0.545707in}}%
\pgfpathlineto{\pgfqpoint{2.468178in}{0.543437in}}%
\pgfpathlineto{\pgfqpoint{2.498162in}{0.543437in}}%
\pgfpathlineto{\pgfqpoint{2.498162in}{0.542302in}}%
\pgfpathlineto{\pgfqpoint{2.513154in}{0.542302in}}%
\pgfpathlineto{\pgfqpoint{2.513154in}{0.544572in}}%
\pgfpathlineto{\pgfqpoint{2.528147in}{0.544572in}}%
\pgfpathlineto{\pgfqpoint{2.528147in}{0.541166in}}%
\pgfpathlineto{\pgfqpoint{2.558131in}{0.541166in}}%
\pgfpathlineto{\pgfqpoint{2.558131in}{0.542302in}}%
\pgfpathlineto{\pgfqpoint{2.573123in}{0.542302in}}%
\pgfpathlineto{\pgfqpoint{2.573123in}{0.545707in}}%
\pgfpathlineto{\pgfqpoint{2.588116in}{0.545707in}}%
\pgfpathlineto{\pgfqpoint{2.588116in}{0.541166in}}%
\pgfpathlineto{\pgfqpoint{2.603108in}{0.541166in}}%
\pgfpathlineto{\pgfqpoint{2.603108in}{0.542302in}}%
\pgfpathlineto{\pgfqpoint{2.633092in}{0.542302in}}%
\pgfpathlineto{\pgfqpoint{2.633092in}{0.541166in}}%
\pgfpathlineto{\pgfqpoint{2.633092in}{0.541166in}}%
\pgfusepath{stroke}%
\end{pgfscope}%
\begin{pgfscope}%
\pgfpathrectangle{\pgfqpoint{0.485140in}{0.541166in}}{\pgfqpoint{2.540460in}{1.418241in}}%
\pgfusepath{clip}%
\pgfsetbuttcap%
\pgfsetmiterjoin%
\pgfsetlinewidth{1.003750pt}%
\definecolor{currentstroke}{rgb}{0.949020,0.372549,0.360784}%
\pgfsetstrokecolor{currentstroke}%
\pgfsetdash{{1.000000pt}{1.650000pt}}{0.000000pt}%
\pgfpathmoveto{\pgfqpoint{1.313774in}{0.541166in}}%
\pgfpathlineto{\pgfqpoint{1.313774in}{0.571004in}}%
\pgfpathlineto{\pgfqpoint{1.328767in}{0.571004in}}%
\pgfpathlineto{\pgfqpoint{1.328767in}{0.590512in}}%
\pgfpathlineto{\pgfqpoint{1.343759in}{0.590512in}}%
\pgfpathlineto{\pgfqpoint{1.343759in}{0.603136in}}%
\pgfpathlineto{\pgfqpoint{1.358751in}{0.603136in}}%
\pgfpathlineto{\pgfqpoint{1.358751in}{0.626088in}}%
\pgfpathlineto{\pgfqpoint{1.373743in}{0.626088in}}%
\pgfpathlineto{\pgfqpoint{1.373743in}{0.676581in}}%
\pgfpathlineto{\pgfqpoint{1.388736in}{0.676581in}}%
\pgfpathlineto{\pgfqpoint{1.388736in}{0.731665in}}%
\pgfpathlineto{\pgfqpoint{1.403728in}{0.731665in}}%
\pgfpathlineto{\pgfqpoint{1.403728in}{0.830357in}}%
\pgfpathlineto{\pgfqpoint{1.418720in}{0.830357in}}%
\pgfpathlineto{\pgfqpoint{1.418720in}{0.930197in}}%
\pgfpathlineto{\pgfqpoint{1.433712in}{0.930197in}}%
\pgfpathlineto{\pgfqpoint{1.433712in}{1.080531in}}%
\pgfpathlineto{\pgfqpoint{1.448705in}{1.080531in}}%
\pgfpathlineto{\pgfqpoint{1.448705in}{1.218241in}}%
\pgfpathlineto{\pgfqpoint{1.463697in}{1.218241in}}%
\pgfpathlineto{\pgfqpoint{1.463697in}{1.500546in}}%
\pgfpathlineto{\pgfqpoint{1.478689in}{1.500546in}}%
\pgfpathlineto{\pgfqpoint{1.478689in}{1.615304in}}%
\pgfpathlineto{\pgfqpoint{1.493681in}{1.615304in}}%
\pgfpathlineto{\pgfqpoint{1.493681in}{1.725472in}}%
\pgfpathlineto{\pgfqpoint{1.508674in}{1.725472in}}%
\pgfpathlineto{\pgfqpoint{1.508674in}{1.828755in}}%
\pgfpathlineto{\pgfqpoint{1.523666in}{1.828755in}}%
\pgfpathlineto{\pgfqpoint{1.523666in}{1.831050in}}%
\pgfpathlineto{\pgfqpoint{1.538658in}{1.831050in}}%
\pgfpathlineto{\pgfqpoint{1.538658in}{1.891872in}}%
\pgfpathlineto{\pgfqpoint{1.553650in}{1.891872in}}%
\pgfpathlineto{\pgfqpoint{1.553650in}{1.835640in}}%
\pgfpathlineto{\pgfqpoint{1.568643in}{1.835640in}}%
\pgfpathlineto{\pgfqpoint{1.568643in}{1.758752in}}%
\pgfpathlineto{\pgfqpoint{1.583635in}{1.758752in}}%
\pgfpathlineto{\pgfqpoint{1.583635in}{1.702521in}}%
\pgfpathlineto{\pgfqpoint{1.598627in}{1.702521in}}%
\pgfpathlineto{\pgfqpoint{1.598627in}{1.569401in}}%
\pgfpathlineto{\pgfqpoint{1.613619in}{1.569401in}}%
\pgfpathlineto{\pgfqpoint{1.613619in}{1.433986in}}%
\pgfpathlineto{\pgfqpoint{1.628612in}{1.433986in}}%
\pgfpathlineto{\pgfqpoint{1.628612in}{1.327261in}}%
\pgfpathlineto{\pgfqpoint{1.643604in}{1.327261in}}%
\pgfpathlineto{\pgfqpoint{1.643604in}{1.280210in}}%
\pgfpathlineto{\pgfqpoint{1.658596in}{1.280210in}}%
\pgfpathlineto{\pgfqpoint{1.658596in}{1.196436in}}%
\pgfpathlineto{\pgfqpoint{1.673588in}{1.196436in}}%
\pgfpathlineto{\pgfqpoint{1.673588in}{1.124139in}}%
\pgfpathlineto{\pgfqpoint{1.688581in}{1.124139in}}%
\pgfpathlineto{\pgfqpoint{1.688581in}{0.985281in}}%
\pgfpathlineto{\pgfqpoint{1.703573in}{0.985281in}}%
\pgfpathlineto{\pgfqpoint{1.703573in}{0.953149in}}%
\pgfpathlineto{\pgfqpoint{1.718565in}{0.953149in}}%
\pgfpathlineto{\pgfqpoint{1.718565in}{0.981838in}}%
\pgfpathlineto{\pgfqpoint{1.733557in}{0.981838in}}%
\pgfpathlineto{\pgfqpoint{1.733557in}{0.948558in}}%
\pgfpathlineto{\pgfqpoint{1.748550in}{0.948558in}}%
\pgfpathlineto{\pgfqpoint{1.748550in}{0.883146in}}%
\pgfpathlineto{\pgfqpoint{1.763542in}{0.883146in}}%
\pgfpathlineto{\pgfqpoint{1.763542in}{0.892327in}}%
\pgfpathlineto{\pgfqpoint{1.778534in}{0.892327in}}%
\pgfpathlineto{\pgfqpoint{1.778534in}{0.806258in}}%
\pgfpathlineto{\pgfqpoint{1.793526in}{0.806258in}}%
\pgfpathlineto{\pgfqpoint{1.793526in}{0.822324in}}%
\pgfpathlineto{\pgfqpoint{1.808519in}{0.822324in}}%
\pgfpathlineto{\pgfqpoint{1.808519in}{0.823472in}}%
\pgfpathlineto{\pgfqpoint{1.823511in}{0.823472in}}%
\pgfpathlineto{\pgfqpoint{1.823511in}{0.762650in}}%
\pgfpathlineto{\pgfqpoint{1.838503in}{0.762650in}}%
\pgfpathlineto{\pgfqpoint{1.838503in}{0.752322in}}%
\pgfpathlineto{\pgfqpoint{1.853495in}{0.752322in}}%
\pgfpathlineto{\pgfqpoint{1.853495in}{0.741993in}}%
\pgfpathlineto{\pgfqpoint{1.868488in}{0.741993in}}%
\pgfpathlineto{\pgfqpoint{1.868488in}{0.707566in}}%
\pgfpathlineto{\pgfqpoint{1.883480in}{0.707566in}}%
\pgfpathlineto{\pgfqpoint{1.883480in}{0.700680in}}%
\pgfpathlineto{\pgfqpoint{1.898472in}{0.700680in}}%
\pgfpathlineto{\pgfqpoint{1.898472in}{0.709861in}}%
\pgfpathlineto{\pgfqpoint{1.913464in}{0.709861in}}%
\pgfpathlineto{\pgfqpoint{1.913464in}{0.686909in}}%
\pgfpathlineto{\pgfqpoint{1.928457in}{0.686909in}}%
\pgfpathlineto{\pgfqpoint{1.928457in}{0.671991in}}%
\pgfpathlineto{\pgfqpoint{1.943449in}{0.671991in}}%
\pgfpathlineto{\pgfqpoint{1.943449in}{0.658220in}}%
\pgfpathlineto{\pgfqpoint{1.958441in}{0.658220in}}%
\pgfpathlineto{\pgfqpoint{1.958441in}{0.663958in}}%
\pgfpathlineto{\pgfqpoint{1.973433in}{0.663958in}}%
\pgfpathlineto{\pgfqpoint{1.973433in}{0.653630in}}%
\pgfpathlineto{\pgfqpoint{1.988426in}{0.653630in}}%
\pgfpathlineto{\pgfqpoint{1.988426in}{0.613464in}}%
\pgfpathlineto{\pgfqpoint{2.003418in}{0.613464in}}%
\pgfpathlineto{\pgfqpoint{2.003418in}{0.616907in}}%
\pgfpathlineto{\pgfqpoint{2.018410in}{0.616907in}}%
\pgfpathlineto{\pgfqpoint{2.018410in}{0.601988in}}%
\pgfpathlineto{\pgfqpoint{2.033402in}{0.601988in}}%
\pgfpathlineto{\pgfqpoint{2.033402in}{0.588217in}}%
\pgfpathlineto{\pgfqpoint{2.048395in}{0.588217in}}%
\pgfpathlineto{\pgfqpoint{2.048395in}{0.614612in}}%
\pgfpathlineto{\pgfqpoint{2.063387in}{0.614612in}}%
\pgfpathlineto{\pgfqpoint{2.063387in}{0.596250in}}%
\pgfpathlineto{\pgfqpoint{2.078379in}{0.596250in}}%
\pgfpathlineto{\pgfqpoint{2.078379in}{0.603136in}}%
\pgfpathlineto{\pgfqpoint{2.093371in}{0.603136in}}%
\pgfpathlineto{\pgfqpoint{2.093371in}{0.590512in}}%
\pgfpathlineto{\pgfqpoint{2.108364in}{0.590512in}}%
\pgfpathlineto{\pgfqpoint{2.108364in}{0.585922in}}%
\pgfpathlineto{\pgfqpoint{2.123356in}{0.585922in}}%
\pgfpathlineto{\pgfqpoint{2.123356in}{0.572151in}}%
\pgfpathlineto{\pgfqpoint{2.138348in}{0.572151in}}%
\pgfpathlineto{\pgfqpoint{2.138348in}{0.566413in}}%
\pgfpathlineto{\pgfqpoint{2.153340in}{0.566413in}}%
\pgfpathlineto{\pgfqpoint{2.153340in}{0.565266in}}%
\pgfpathlineto{\pgfqpoint{2.168333in}{0.565266in}}%
\pgfpathlineto{\pgfqpoint{2.168333in}{0.566413in}}%
\pgfpathlineto{\pgfqpoint{2.183325in}{0.566413in}}%
\pgfpathlineto{\pgfqpoint{2.183325in}{0.561823in}}%
\pgfpathlineto{\pgfqpoint{2.198317in}{0.561823in}}%
\pgfpathlineto{\pgfqpoint{2.198317in}{0.559528in}}%
\pgfpathlineto{\pgfqpoint{2.213309in}{0.559528in}}%
\pgfpathlineto{\pgfqpoint{2.213309in}{0.558380in}}%
\pgfpathlineto{\pgfqpoint{2.228302in}{0.558380in}}%
\pgfpathlineto{\pgfqpoint{2.228302in}{0.565266in}}%
\pgfpathlineto{\pgfqpoint{2.243294in}{0.565266in}}%
\pgfpathlineto{\pgfqpoint{2.243294in}{0.552642in}}%
\pgfpathlineto{\pgfqpoint{2.258286in}{0.552642in}}%
\pgfpathlineto{\pgfqpoint{2.258286in}{0.553790in}}%
\pgfpathlineto{\pgfqpoint{2.273278in}{0.553790in}}%
\pgfpathlineto{\pgfqpoint{2.273278in}{0.544609in}}%
\pgfpathlineto{\pgfqpoint{2.288271in}{0.544609in}}%
\pgfpathlineto{\pgfqpoint{2.288271in}{0.546904in}}%
\pgfpathlineto{\pgfqpoint{2.303263in}{0.546904in}}%
\pgfpathlineto{\pgfqpoint{2.303263in}{0.544609in}}%
\pgfpathlineto{\pgfqpoint{2.318255in}{0.544609in}}%
\pgfpathlineto{\pgfqpoint{2.318255in}{0.548052in}}%
\pgfpathlineto{\pgfqpoint{2.333247in}{0.548052in}}%
\pgfpathlineto{\pgfqpoint{2.333247in}{0.553790in}}%
\pgfpathlineto{\pgfqpoint{2.348240in}{0.553790in}}%
\pgfpathlineto{\pgfqpoint{2.348240in}{0.548052in}}%
\pgfpathlineto{\pgfqpoint{2.363232in}{0.548052in}}%
\pgfpathlineto{\pgfqpoint{2.363232in}{0.543462in}}%
\pgfpathlineto{\pgfqpoint{2.378224in}{0.543462in}}%
\pgfpathlineto{\pgfqpoint{2.378224in}{0.546904in}}%
\pgfpathlineto{\pgfqpoint{2.393216in}{0.546904in}}%
\pgfpathlineto{\pgfqpoint{2.393216in}{0.544609in}}%
\pgfpathlineto{\pgfqpoint{2.408209in}{0.544609in}}%
\pgfpathlineto{\pgfqpoint{2.408209in}{0.541166in}}%
\pgfpathlineto{\pgfqpoint{2.423201in}{0.541166in}}%
\pgfpathlineto{\pgfqpoint{2.423201in}{0.543462in}}%
\pgfpathlineto{\pgfqpoint{2.438193in}{0.543462in}}%
\pgfpathlineto{\pgfqpoint{2.438193in}{0.544609in}}%
\pgfpathlineto{\pgfqpoint{2.453185in}{0.544609in}}%
\pgfpathlineto{\pgfqpoint{2.453185in}{0.541166in}}%
\pgfpathlineto{\pgfqpoint{2.468178in}{0.541166in}}%
\pgfpathlineto{\pgfqpoint{2.468178in}{0.543462in}}%
\pgfpathlineto{\pgfqpoint{2.483170in}{0.543462in}}%
\pgfpathlineto{\pgfqpoint{2.483170in}{0.546904in}}%
\pgfpathlineto{\pgfqpoint{2.498162in}{0.546904in}}%
\pgfpathlineto{\pgfqpoint{2.498162in}{0.543462in}}%
\pgfpathlineto{\pgfqpoint{2.528147in}{0.543462in}}%
\pgfpathlineto{\pgfqpoint{2.528147in}{0.541166in}}%
\pgfpathlineto{\pgfqpoint{2.543139in}{0.541166in}}%
\pgfpathlineto{\pgfqpoint{2.543139in}{0.542314in}}%
\pgfpathlineto{\pgfqpoint{2.588116in}{0.542314in}}%
\pgfpathlineto{\pgfqpoint{2.588116in}{0.541166in}}%
\pgfpathlineto{\pgfqpoint{2.633092in}{0.541166in}}%
\pgfpathlineto{\pgfqpoint{2.633092in}{0.541166in}}%
\pgfusepath{stroke}%
\end{pgfscope}%
\begin{pgfscope}%
\pgfsetrectcap%
\pgfsetmiterjoin%
\pgfsetlinewidth{0.803000pt}%
\definecolor{currentstroke}{rgb}{0.000000,0.000000,0.000000}%
\pgfsetstrokecolor{currentstroke}%
\pgfsetdash{}{0pt}%
\pgfpathmoveto{\pgfqpoint{0.485140in}{0.541166in}}%
\pgfpathlineto{\pgfqpoint{0.485140in}{1.959407in}}%
\pgfusepath{stroke}%
\end{pgfscope}%
\begin{pgfscope}%
\pgfsetrectcap%
\pgfsetmiterjoin%
\pgfsetlinewidth{0.803000pt}%
\definecolor{currentstroke}{rgb}{0.000000,0.000000,0.000000}%
\pgfsetstrokecolor{currentstroke}%
\pgfsetdash{}{0pt}%
\pgfpathmoveto{\pgfqpoint{3.025600in}{0.541166in}}%
\pgfpathlineto{\pgfqpoint{3.025600in}{1.959407in}}%
\pgfusepath{stroke}%
\end{pgfscope}%
\begin{pgfscope}%
\pgfsetrectcap%
\pgfsetmiterjoin%
\pgfsetlinewidth{0.803000pt}%
\definecolor{currentstroke}{rgb}{0.000000,0.000000,0.000000}%
\pgfsetstrokecolor{currentstroke}%
\pgfsetdash{}{0pt}%
\pgfpathmoveto{\pgfqpoint{0.485140in}{0.541166in}}%
\pgfpathlineto{\pgfqpoint{3.025600in}{0.541166in}}%
\pgfusepath{stroke}%
\end{pgfscope}%
\begin{pgfscope}%
\pgfsetrectcap%
\pgfsetmiterjoin%
\pgfsetlinewidth{0.803000pt}%
\definecolor{currentstroke}{rgb}{0.000000,0.000000,0.000000}%
\pgfsetstrokecolor{currentstroke}%
\pgfsetdash{}{0pt}%
\pgfpathmoveto{\pgfqpoint{0.485140in}{1.959407in}}%
\pgfpathlineto{\pgfqpoint{3.025600in}{1.959407in}}%
\pgfusepath{stroke}%
\end{pgfscope}%
\begin{pgfscope}%
\definecolor{textcolor}{rgb}{0.000000,0.000000,0.000000}%
\pgfsetstrokecolor{textcolor}%
\pgfsetfillcolor{textcolor}%
\pgftext[x=0.485140in,y=2.042741in,left,base]{\color{textcolor}\rmfamily\fontsize{10.000000}{12.000000}\selectfont Bin [2.33, 2.5), 20,928 events}%
\end{pgfscope}%
\begin{pgfscope}%
\pgfsetbuttcap%
\pgfsetmiterjoin%
\definecolor{currentfill}{rgb}{1.000000,1.000000,1.000000}%
\pgfsetfillcolor{currentfill}%
\pgfsetfillopacity{0.800000}%
\pgfsetlinewidth{1.003750pt}%
\definecolor{currentstroke}{rgb}{0.800000,0.800000,0.800000}%
\pgfsetstrokecolor{currentstroke}%
\pgfsetstrokeopacity{0.800000}%
\pgfsetdash{}{0pt}%
\pgfpathmoveto{\pgfqpoint{1.990822in}{1.559852in}}%
\pgfpathlineto{\pgfqpoint{2.947822in}{1.559852in}}%
\pgfpathquadraticcurveto{\pgfqpoint{2.970044in}{1.559852in}}{\pgfqpoint{2.970044in}{1.582074in}}%
\pgfpathlineto{\pgfqpoint{2.970044in}{1.881629in}}%
\pgfpathquadraticcurveto{\pgfqpoint{2.970044in}{1.903852in}}{\pgfqpoint{2.947822in}{1.903852in}}%
\pgfpathlineto{\pgfqpoint{1.990822in}{1.903852in}}%
\pgfpathquadraticcurveto{\pgfqpoint{1.968600in}{1.903852in}}{\pgfqpoint{1.968600in}{1.881629in}}%
\pgfpathlineto{\pgfqpoint{1.968600in}{1.582074in}}%
\pgfpathquadraticcurveto{\pgfqpoint{1.968600in}{1.559852in}}{\pgfqpoint{1.990822in}{1.559852in}}%
\pgfpathclose%
\pgfusepath{stroke,fill}%
\end{pgfscope}%
\begin{pgfscope}%
\pgfsetbuttcap%
\pgfsetmiterjoin%
\pgfsetlinewidth{1.003750pt}%
\definecolor{currentstroke}{rgb}{0.313725,0.317647,0.309804}%
\pgfsetstrokecolor{currentstroke}%
\pgfsetdash{}{0pt}%
\pgfpathmoveto{\pgfqpoint{2.013044in}{1.781185in}}%
\pgfpathlineto{\pgfqpoint{2.235267in}{1.781185in}}%
\pgfpathlineto{\pgfqpoint{2.235267in}{1.858963in}}%
\pgfpathlineto{\pgfqpoint{2.013044in}{1.858963in}}%
\pgfpathclose%
\pgfusepath{stroke}%
\end{pgfscope}%
\begin{pgfscope}%
\definecolor{textcolor}{rgb}{0.000000,0.000000,0.000000}%
\pgfsetstrokecolor{textcolor}%
\pgfsetfillcolor{textcolor}%
\pgftext[x=2.324156in,y=1.781185in,left,base]{\color{textcolor}\rmfamily\fontsize{8.000000}{9.600000}\selectfont IQR = 0.23}%
\end{pgfscope}%
\begin{pgfscope}%
\pgfsetbuttcap%
\pgfsetmiterjoin%
\pgfsetlinewidth{1.003750pt}%
\definecolor{currentstroke}{rgb}{0.949020,0.372549,0.360784}%
\pgfsetstrokecolor{currentstroke}%
\pgfsetdash{{1.000000pt}{1.650000pt}}{0.000000pt}%
\pgfpathmoveto{\pgfqpoint{2.013044in}{1.625852in}}%
\pgfpathlineto{\pgfqpoint{2.235267in}{1.625852in}}%
\pgfpathlineto{\pgfqpoint{2.235267in}{1.703629in}}%
\pgfpathlineto{\pgfqpoint{2.013044in}{1.703629in}}%
\pgfpathclose%
\pgfusepath{stroke}%
\end{pgfscope}%
\begin{pgfscope}%
\definecolor{textcolor}{rgb}{0.000000,0.000000,0.000000}%
\pgfsetstrokecolor{textcolor}%
\pgfsetfillcolor{textcolor}%
\pgftext[x=2.324156in,y=1.625852in,left,base]{\color{textcolor}\rmfamily\fontsize{8.000000}{9.600000}\selectfont IQR = 0.19}%
\end{pgfscope}%
\begin{pgfscope}%
\pgfsetbuttcap%
\pgfsetmiterjoin%
\definecolor{currentfill}{rgb}{1.000000,1.000000,1.000000}%
\pgfsetfillcolor{currentfill}%
\pgfsetlinewidth{0.000000pt}%
\definecolor{currentstroke}{rgb}{0.000000,0.000000,0.000000}%
\pgfsetstrokecolor{currentstroke}%
\pgfsetstrokeopacity{0.000000}%
\pgfsetdash{}{0pt}%
\pgfpathmoveto{\pgfqpoint{3.510740in}{0.541166in}}%
\pgfpathlineto{\pgfqpoint{6.051200in}{0.541166in}}%
\pgfpathlineto{\pgfqpoint{6.051200in}{1.959407in}}%
\pgfpathlineto{\pgfqpoint{3.510740in}{1.959407in}}%
\pgfpathclose%
\pgfusepath{fill}%
\end{pgfscope}%
\begin{pgfscope}%
\pgfsetbuttcap%
\pgfsetroundjoin%
\definecolor{currentfill}{rgb}{0.000000,0.000000,0.000000}%
\pgfsetfillcolor{currentfill}%
\pgfsetlinewidth{0.803000pt}%
\definecolor{currentstroke}{rgb}{0.000000,0.000000,0.000000}%
\pgfsetstrokecolor{currentstroke}%
\pgfsetdash{}{0pt}%
\pgfsys@defobject{currentmarker}{\pgfqpoint{0.000000in}{-0.048611in}}{\pgfqpoint{0.000000in}{0.000000in}}{%
\pgfpathmoveto{\pgfqpoint{0.000000in}{0.000000in}}%
\pgfpathlineto{\pgfqpoint{0.000000in}{-0.048611in}}%
\pgfusepath{stroke,fill}%
}%
\begin{pgfscope}%
\pgfsys@transformshift{3.941945in}{0.541166in}%
\pgfsys@useobject{currentmarker}{}%
\end{pgfscope}%
\end{pgfscope}%
\begin{pgfscope}%
\definecolor{textcolor}{rgb}{0.000000,0.000000,0.000000}%
\pgfsetstrokecolor{textcolor}%
\pgfsetfillcolor{textcolor}%
\pgftext[x=3.941945in,y=0.443944in,,top]{\color{textcolor}\rmfamily\fontsize{8.000000}{9.600000}\selectfont \(\displaystyle {-1}\)}%
\end{pgfscope}%
\begin{pgfscope}%
\pgfsetbuttcap%
\pgfsetroundjoin%
\definecolor{currentfill}{rgb}{0.000000,0.000000,0.000000}%
\pgfsetfillcolor{currentfill}%
\pgfsetlinewidth{0.803000pt}%
\definecolor{currentstroke}{rgb}{0.000000,0.000000,0.000000}%
\pgfsetstrokecolor{currentstroke}%
\pgfsetdash{}{0pt}%
\pgfsys@defobject{currentmarker}{\pgfqpoint{0.000000in}{-0.048611in}}{\pgfqpoint{0.000000in}{0.000000in}}{%
\pgfpathmoveto{\pgfqpoint{0.000000in}{0.000000in}}%
\pgfpathlineto{\pgfqpoint{0.000000in}{-0.048611in}}%
\pgfusepath{stroke,fill}%
}%
\begin{pgfscope}%
\pgfsys@transformshift{4.628399in}{0.541166in}%
\pgfsys@useobject{currentmarker}{}%
\end{pgfscope}%
\end{pgfscope}%
\begin{pgfscope}%
\definecolor{textcolor}{rgb}{0.000000,0.000000,0.000000}%
\pgfsetstrokecolor{textcolor}%
\pgfsetfillcolor{textcolor}%
\pgftext[x=4.628399in,y=0.443944in,,top]{\color{textcolor}\rmfamily\fontsize{8.000000}{9.600000}\selectfont \(\displaystyle {0}\)}%
\end{pgfscope}%
\begin{pgfscope}%
\pgfsetbuttcap%
\pgfsetroundjoin%
\definecolor{currentfill}{rgb}{0.000000,0.000000,0.000000}%
\pgfsetfillcolor{currentfill}%
\pgfsetlinewidth{0.803000pt}%
\definecolor{currentstroke}{rgb}{0.000000,0.000000,0.000000}%
\pgfsetstrokecolor{currentstroke}%
\pgfsetdash{}{0pt}%
\pgfsys@defobject{currentmarker}{\pgfqpoint{0.000000in}{-0.048611in}}{\pgfqpoint{0.000000in}{0.000000in}}{%
\pgfpathmoveto{\pgfqpoint{0.000000in}{0.000000in}}%
\pgfpathlineto{\pgfqpoint{0.000000in}{-0.048611in}}%
\pgfusepath{stroke,fill}%
}%
\begin{pgfscope}%
\pgfsys@transformshift{5.314853in}{0.541166in}%
\pgfsys@useobject{currentmarker}{}%
\end{pgfscope}%
\end{pgfscope}%
\begin{pgfscope}%
\definecolor{textcolor}{rgb}{0.000000,0.000000,0.000000}%
\pgfsetstrokecolor{textcolor}%
\pgfsetfillcolor{textcolor}%
\pgftext[x=5.314853in,y=0.443944in,,top]{\color{textcolor}\rmfamily\fontsize{8.000000}{9.600000}\selectfont \(\displaystyle {1}\)}%
\end{pgfscope}%
\begin{pgfscope}%
\pgfsetbuttcap%
\pgfsetroundjoin%
\definecolor{currentfill}{rgb}{0.000000,0.000000,0.000000}%
\pgfsetfillcolor{currentfill}%
\pgfsetlinewidth{0.803000pt}%
\definecolor{currentstroke}{rgb}{0.000000,0.000000,0.000000}%
\pgfsetstrokecolor{currentstroke}%
\pgfsetdash{}{0pt}%
\pgfsys@defobject{currentmarker}{\pgfqpoint{0.000000in}{-0.048611in}}{\pgfqpoint{0.000000in}{0.000000in}}{%
\pgfpathmoveto{\pgfqpoint{0.000000in}{0.000000in}}%
\pgfpathlineto{\pgfqpoint{0.000000in}{-0.048611in}}%
\pgfusepath{stroke,fill}%
}%
\begin{pgfscope}%
\pgfsys@transformshift{6.001306in}{0.541166in}%
\pgfsys@useobject{currentmarker}{}%
\end{pgfscope}%
\end{pgfscope}%
\begin{pgfscope}%
\definecolor{textcolor}{rgb}{0.000000,0.000000,0.000000}%
\pgfsetstrokecolor{textcolor}%
\pgfsetfillcolor{textcolor}%
\pgftext[x=6.001306in,y=0.443944in,,top]{\color{textcolor}\rmfamily\fontsize{8.000000}{9.600000}\selectfont \(\displaystyle {2}\)}%
\end{pgfscope}%
\begin{pgfscope}%
\definecolor{textcolor}{rgb}{0.000000,0.000000,0.000000}%
\pgfsetstrokecolor{textcolor}%
\pgfsetfillcolor{textcolor}%
\pgftext[x=4.780970in,y=0.289722in,,top]{\color{textcolor}\rmfamily\fontsize{10.000000}{12.000000}\selectfont \(\displaystyle \log_{10}(E_{\textup{true}}) - \log_{10}(E_{\textup{reco}}) \, \left[ E / \textup{GeV} \right]\)}%
\end{pgfscope}%
\begin{pgfscope}%
\pgfsetbuttcap%
\pgfsetroundjoin%
\definecolor{currentfill}{rgb}{0.000000,0.000000,0.000000}%
\pgfsetfillcolor{currentfill}%
\pgfsetlinewidth{0.803000pt}%
\definecolor{currentstroke}{rgb}{0.000000,0.000000,0.000000}%
\pgfsetstrokecolor{currentstroke}%
\pgfsetdash{}{0pt}%
\pgfsys@defobject{currentmarker}{\pgfqpoint{-0.048611in}{0.000000in}}{\pgfqpoint{-0.000000in}{0.000000in}}{%
\pgfpathmoveto{\pgfqpoint{-0.000000in}{0.000000in}}%
\pgfpathlineto{\pgfqpoint{-0.048611in}{0.000000in}}%
\pgfusepath{stroke,fill}%
}%
\begin{pgfscope}%
\pgfsys@transformshift{3.510740in}{0.541166in}%
\pgfsys@useobject{currentmarker}{}%
\end{pgfscope}%
\end{pgfscope}%
\begin{pgfscope}%
\definecolor{textcolor}{rgb}{0.000000,0.000000,0.000000}%
\pgfsetstrokecolor{textcolor}%
\pgfsetfillcolor{textcolor}%
\pgftext[x=3.354489in, y=0.502611in, left, base]{\color{textcolor}\rmfamily\fontsize{8.000000}{9.600000}\selectfont \(\displaystyle {0}\)}%
\end{pgfscope}%
\begin{pgfscope}%
\pgfsetbuttcap%
\pgfsetroundjoin%
\definecolor{currentfill}{rgb}{0.000000,0.000000,0.000000}%
\pgfsetfillcolor{currentfill}%
\pgfsetlinewidth{0.803000pt}%
\definecolor{currentstroke}{rgb}{0.000000,0.000000,0.000000}%
\pgfsetstrokecolor{currentstroke}%
\pgfsetdash{}{0pt}%
\pgfsys@defobject{currentmarker}{\pgfqpoint{-0.048611in}{0.000000in}}{\pgfqpoint{-0.000000in}{0.000000in}}{%
\pgfpathmoveto{\pgfqpoint{-0.000000in}{0.000000in}}%
\pgfpathlineto{\pgfqpoint{-0.048611in}{0.000000in}}%
\pgfusepath{stroke,fill}%
}%
\begin{pgfscope}%
\pgfsys@transformshift{3.510740in}{1.216720in}%
\pgfsys@useobject{currentmarker}{}%
\end{pgfscope}%
\end{pgfscope}%
\begin{pgfscope}%
\definecolor{textcolor}{rgb}{0.000000,0.000000,0.000000}%
\pgfsetstrokecolor{textcolor}%
\pgfsetfillcolor{textcolor}%
\pgftext[x=3.354489in, y=1.178164in, left, base]{\color{textcolor}\rmfamily\fontsize{8.000000}{9.600000}\selectfont \(\displaystyle {1}\)}%
\end{pgfscope}%
\begin{pgfscope}%
\pgfsetbuttcap%
\pgfsetroundjoin%
\definecolor{currentfill}{rgb}{0.000000,0.000000,0.000000}%
\pgfsetfillcolor{currentfill}%
\pgfsetlinewidth{0.803000pt}%
\definecolor{currentstroke}{rgb}{0.000000,0.000000,0.000000}%
\pgfsetstrokecolor{currentstroke}%
\pgfsetdash{}{0pt}%
\pgfsys@defobject{currentmarker}{\pgfqpoint{-0.048611in}{0.000000in}}{\pgfqpoint{-0.000000in}{0.000000in}}{%
\pgfpathmoveto{\pgfqpoint{-0.000000in}{0.000000in}}%
\pgfpathlineto{\pgfqpoint{-0.048611in}{0.000000in}}%
\pgfusepath{stroke,fill}%
}%
\begin{pgfscope}%
\pgfsys@transformshift{3.510740in}{1.892273in}%
\pgfsys@useobject{currentmarker}{}%
\end{pgfscope}%
\end{pgfscope}%
\begin{pgfscope}%
\definecolor{textcolor}{rgb}{0.000000,0.000000,0.000000}%
\pgfsetstrokecolor{textcolor}%
\pgfsetfillcolor{textcolor}%
\pgftext[x=3.354489in, y=1.853718in, left, base]{\color{textcolor}\rmfamily\fontsize{8.000000}{9.600000}\selectfont \(\displaystyle {2}\)}%
\end{pgfscope}%
\begin{pgfscope}%
\definecolor{textcolor}{rgb}{0.000000,0.000000,0.000000}%
\pgfsetstrokecolor{textcolor}%
\pgfsetfillcolor{textcolor}%
\pgftext[x=3.298933in,y=1.250287in,,bottom,rotate=90.000000]{\color{textcolor}\rmfamily\fontsize{10.000000}{12.000000}\selectfont Density}%
\end{pgfscope}%
\begin{pgfscope}%
\pgfpathrectangle{\pgfqpoint{3.510740in}{0.541166in}}{\pgfqpoint{2.540460in}{1.418241in}}%
\pgfusepath{clip}%
\pgfsetbuttcap%
\pgfsetmiterjoin%
\pgfsetlinewidth{1.003750pt}%
\definecolor{currentstroke}{rgb}{0.313725,0.317647,0.309804}%
\pgfsetstrokecolor{currentstroke}%
\pgfsetdash{}{0pt}%
\pgfpathmoveto{\pgfqpoint{4.435012in}{0.541166in}}%
\pgfpathlineto{\pgfqpoint{4.435012in}{0.544569in}}%
\pgfpathlineto{\pgfqpoint{4.461340in}{0.544569in}}%
\pgfpathlineto{\pgfqpoint{4.461340in}{0.544569in}}%
\pgfpathlineto{\pgfqpoint{4.487668in}{0.544569in}}%
\pgfpathlineto{\pgfqpoint{4.487668in}{0.558178in}}%
\pgfpathlineto{\pgfqpoint{4.513997in}{0.558178in}}%
\pgfpathlineto{\pgfqpoint{4.513997in}{0.571787in}}%
\pgfpathlineto{\pgfqpoint{4.540325in}{0.571787in}}%
\pgfpathlineto{\pgfqpoint{4.540325in}{0.616017in}}%
\pgfpathlineto{\pgfqpoint{4.566653in}{0.616017in}}%
\pgfpathlineto{\pgfqpoint{4.566653in}{0.745303in}}%
\pgfpathlineto{\pgfqpoint{4.592981in}{0.745303in}}%
\pgfpathlineto{\pgfqpoint{4.592981in}{0.843969in}}%
\pgfpathlineto{\pgfqpoint{4.619310in}{0.843969in}}%
\pgfpathlineto{\pgfqpoint{4.619310in}{1.099140in}}%
\pgfpathlineto{\pgfqpoint{4.645638in}{1.099140in}}%
\pgfpathlineto{\pgfqpoint{4.645638in}{1.306680in}}%
\pgfpathlineto{\pgfqpoint{4.671966in}{1.306680in}}%
\pgfpathlineto{\pgfqpoint{4.671966in}{1.565253in}}%
\pgfpathlineto{\pgfqpoint{4.698295in}{1.565253in}}%
\pgfpathlineto{\pgfqpoint{4.698295in}{1.847642in}}%
\pgfpathlineto{\pgfqpoint{4.724623in}{1.847642in}}%
\pgfpathlineto{\pgfqpoint{4.724623in}{1.796608in}}%
\pgfpathlineto{\pgfqpoint{4.750951in}{1.796608in}}%
\pgfpathlineto{\pgfqpoint{4.750951in}{1.891872in}}%
\pgfpathlineto{\pgfqpoint{4.777280in}{1.891872in}}%
\pgfpathlineto{\pgfqpoint{4.777280in}{1.670724in}}%
\pgfpathlineto{\pgfqpoint{4.803608in}{1.670724in}}%
\pgfpathlineto{\pgfqpoint{4.803608in}{1.837435in}}%
\pgfpathlineto{\pgfqpoint{4.829936in}{1.837435in}}%
\pgfpathlineto{\pgfqpoint{4.829936in}{1.578862in}}%
\pgfpathlineto{\pgfqpoint{4.856264in}{1.578862in}}%
\pgfpathlineto{\pgfqpoint{4.856264in}{1.422357in}}%
\pgfpathlineto{\pgfqpoint{4.882593in}{1.422357in}}%
\pgfpathlineto{\pgfqpoint{4.882593in}{1.371323in}}%
\pgfpathlineto{\pgfqpoint{4.908921in}{1.371323in}}%
\pgfpathlineto{\pgfqpoint{4.908921in}{1.177393in}}%
\pgfpathlineto{\pgfqpoint{4.935249in}{1.177393in}}%
\pgfpathlineto{\pgfqpoint{4.935249in}{1.146772in}}%
\pgfpathlineto{\pgfqpoint{4.961578in}{1.146772in}}%
\pgfpathlineto{\pgfqpoint{4.961578in}{1.071922in}}%
\pgfpathlineto{\pgfqpoint{4.987906in}{1.071922in}}%
\pgfpathlineto{\pgfqpoint{4.987906in}{0.990267in}}%
\pgfpathlineto{\pgfqpoint{5.014234in}{0.990267in}}%
\pgfpathlineto{\pgfqpoint{5.014234in}{0.942636in}}%
\pgfpathlineto{\pgfqpoint{5.040563in}{0.942636in}}%
\pgfpathlineto{\pgfqpoint{5.040563in}{0.874590in}}%
\pgfpathlineto{\pgfqpoint{5.066891in}{0.874590in}}%
\pgfpathlineto{\pgfqpoint{5.066891in}{0.860981in}}%
\pgfpathlineto{\pgfqpoint{5.093219in}{0.860981in}}%
\pgfpathlineto{\pgfqpoint{5.093219in}{0.806544in}}%
\pgfpathlineto{\pgfqpoint{5.119547in}{0.806544in}}%
\pgfpathlineto{\pgfqpoint{5.119547in}{0.803142in}}%
\pgfpathlineto{\pgfqpoint{5.145876in}{0.803142in}}%
\pgfpathlineto{\pgfqpoint{5.145876in}{0.786131in}}%
\pgfpathlineto{\pgfqpoint{5.172204in}{0.786131in}}%
\pgfpathlineto{\pgfqpoint{5.172204in}{0.775924in}}%
\pgfpathlineto{\pgfqpoint{5.198532in}{0.775924in}}%
\pgfpathlineto{\pgfqpoint{5.198532in}{0.714683in}}%
\pgfpathlineto{\pgfqpoint{5.224861in}{0.714683in}}%
\pgfpathlineto{\pgfqpoint{5.224861in}{0.690867in}}%
\pgfpathlineto{\pgfqpoint{5.251189in}{0.690867in}}%
\pgfpathlineto{\pgfqpoint{5.251189in}{0.673855in}}%
\pgfpathlineto{\pgfqpoint{5.277517in}{0.673855in}}%
\pgfpathlineto{\pgfqpoint{5.277517in}{0.684062in}}%
\pgfpathlineto{\pgfqpoint{5.303845in}{0.684062in}}%
\pgfpathlineto{\pgfqpoint{5.303845in}{0.653442in}}%
\pgfpathlineto{\pgfqpoint{5.330174in}{0.653442in}}%
\pgfpathlineto{\pgfqpoint{5.330174in}{0.653442in}}%
\pgfpathlineto{\pgfqpoint{5.356502in}{0.653442in}}%
\pgfpathlineto{\pgfqpoint{5.356502in}{0.602407in}}%
\pgfpathlineto{\pgfqpoint{5.382830in}{0.602407in}}%
\pgfpathlineto{\pgfqpoint{5.382830in}{0.599005in}}%
\pgfpathlineto{\pgfqpoint{5.409159in}{0.599005in}}%
\pgfpathlineto{\pgfqpoint{5.409159in}{0.616017in}}%
\pgfpathlineto{\pgfqpoint{5.435487in}{0.616017in}}%
\pgfpathlineto{\pgfqpoint{5.435487in}{0.588798in}}%
\pgfpathlineto{\pgfqpoint{5.461815in}{0.588798in}}%
\pgfpathlineto{\pgfqpoint{5.461815in}{0.571787in}}%
\pgfpathlineto{\pgfqpoint{5.488144in}{0.571787in}}%
\pgfpathlineto{\pgfqpoint{5.488144in}{0.568385in}}%
\pgfpathlineto{\pgfqpoint{5.514472in}{0.568385in}}%
\pgfpathlineto{\pgfqpoint{5.514472in}{0.575189in}}%
\pgfpathlineto{\pgfqpoint{5.540800in}{0.575189in}}%
\pgfpathlineto{\pgfqpoint{5.540800in}{0.558178in}}%
\pgfpathlineto{\pgfqpoint{5.567128in}{0.558178in}}%
\pgfpathlineto{\pgfqpoint{5.567128in}{0.561580in}}%
\pgfpathlineto{\pgfqpoint{5.593457in}{0.561580in}}%
\pgfpathlineto{\pgfqpoint{5.593457in}{0.558178in}}%
\pgfpathlineto{\pgfqpoint{5.619785in}{0.558178in}}%
\pgfpathlineto{\pgfqpoint{5.619785in}{0.544569in}}%
\pgfpathlineto{\pgfqpoint{5.646113in}{0.544569in}}%
\pgfpathlineto{\pgfqpoint{5.646113in}{0.554776in}}%
\pgfpathlineto{\pgfqpoint{5.672442in}{0.554776in}}%
\pgfpathlineto{\pgfqpoint{5.672442in}{0.544569in}}%
\pgfpathlineto{\pgfqpoint{5.698770in}{0.544569in}}%
\pgfpathlineto{\pgfqpoint{5.698770in}{0.544569in}}%
\pgfpathlineto{\pgfqpoint{5.725098in}{0.544569in}}%
\pgfpathlineto{\pgfqpoint{5.725098in}{0.547971in}}%
\pgfpathlineto{\pgfqpoint{5.751426in}{0.547971in}}%
\pgfpathlineto{\pgfqpoint{5.751426in}{0.547971in}}%
\pgfpathlineto{\pgfqpoint{5.777755in}{0.547971in}}%
\pgfpathlineto{\pgfqpoint{5.777755in}{0.544569in}}%
\pgfpathlineto{\pgfqpoint{5.804083in}{0.544569in}}%
\pgfpathlineto{\pgfqpoint{5.804083in}{0.544569in}}%
\pgfpathlineto{\pgfqpoint{5.830411in}{0.544569in}}%
\pgfpathlineto{\pgfqpoint{5.830411in}{0.541166in}}%
\pgfpathlineto{\pgfqpoint{5.856740in}{0.541166in}}%
\pgfpathlineto{\pgfqpoint{5.856740in}{0.541166in}}%
\pgfpathlineto{\pgfqpoint{5.883068in}{0.541166in}}%
\pgfpathlineto{\pgfqpoint{5.883068in}{0.541166in}}%
\pgfpathlineto{\pgfqpoint{5.909396in}{0.541166in}}%
\pgfpathlineto{\pgfqpoint{5.909396in}{0.544569in}}%
\pgfpathlineto{\pgfqpoint{5.935725in}{0.544569in}}%
\pgfpathlineto{\pgfqpoint{5.935725in}{0.541166in}}%
\pgfusepath{stroke}%
\end{pgfscope}%
\begin{pgfscope}%
\pgfpathrectangle{\pgfqpoint{3.510740in}{0.541166in}}{\pgfqpoint{2.540460in}{1.418241in}}%
\pgfusepath{clip}%
\pgfsetbuttcap%
\pgfsetmiterjoin%
\pgfsetlinewidth{1.003750pt}%
\definecolor{currentstroke}{rgb}{0.949020,0.372549,0.360784}%
\pgfsetstrokecolor{currentstroke}%
\pgfsetdash{{1.000000pt}{1.650000pt}}{0.000000pt}%
\pgfpathmoveto{\pgfqpoint{4.435012in}{0.541166in}}%
\pgfpathlineto{\pgfqpoint{4.435012in}{0.606555in}}%
\pgfpathlineto{\pgfqpoint{4.461340in}{0.606555in}}%
\pgfpathlineto{\pgfqpoint{4.461340in}{0.647853in}}%
\pgfpathlineto{\pgfqpoint{4.487668in}{0.647853in}}%
\pgfpathlineto{\pgfqpoint{4.487668in}{0.692592in}}%
\pgfpathlineto{\pgfqpoint{4.513997in}{0.692592in}}%
\pgfpathlineto{\pgfqpoint{4.513997in}{0.909407in}}%
\pgfpathlineto{\pgfqpoint{4.540325in}{0.909407in}}%
\pgfpathlineto{\pgfqpoint{4.540325in}{1.122780in}}%
\pgfpathlineto{\pgfqpoint{4.566653in}{1.122780in}}%
\pgfpathlineto{\pgfqpoint{4.566653in}{1.336153in}}%
\pgfpathlineto{\pgfqpoint{4.592981in}{1.336153in}}%
\pgfpathlineto{\pgfqpoint{4.592981in}{1.601149in}}%
\pgfpathlineto{\pgfqpoint{4.619310in}{1.601149in}}%
\pgfpathlineto{\pgfqpoint{4.619310in}{1.570176in}}%
\pgfpathlineto{\pgfqpoint{4.645638in}{1.570176in}}%
\pgfpathlineto{\pgfqpoint{4.645638in}{1.549527in}}%
\pgfpathlineto{\pgfqpoint{4.671966in}{1.549527in}}%
\pgfpathlineto{\pgfqpoint{4.671966in}{1.587383in}}%
\pgfpathlineto{\pgfqpoint{4.698295in}{1.587383in}}%
\pgfpathlineto{\pgfqpoint{4.698295in}{1.566734in}}%
\pgfpathlineto{\pgfqpoint{4.724623in}{1.566734in}}%
\pgfpathlineto{\pgfqpoint{4.724623in}{1.652772in}}%
\pgfpathlineto{\pgfqpoint{4.750951in}{1.652772in}}%
\pgfpathlineto{\pgfqpoint{4.750951in}{1.611474in}}%
\pgfpathlineto{\pgfqpoint{4.777280in}{1.611474in}}%
\pgfpathlineto{\pgfqpoint{4.777280in}{1.497904in}}%
\pgfpathlineto{\pgfqpoint{4.803608in}{1.497904in}}%
\pgfpathlineto{\pgfqpoint{4.803608in}{1.432515in}}%
\pgfpathlineto{\pgfqpoint{4.829936in}{1.432515in}}%
\pgfpathlineto{\pgfqpoint{4.829936in}{1.377451in}}%
\pgfpathlineto{\pgfqpoint{4.856264in}{1.377451in}}%
\pgfpathlineto{\pgfqpoint{4.856264in}{1.384334in}}%
\pgfpathlineto{\pgfqpoint{4.882593in}{1.384334in}}%
\pgfpathlineto{\pgfqpoint{4.882593in}{1.239791in}}%
\pgfpathlineto{\pgfqpoint{4.908921in}{1.239791in}}%
\pgfpathlineto{\pgfqpoint{4.908921in}{1.084924in}}%
\pgfpathlineto{\pgfqpoint{4.935249in}{1.084924in}}%
\pgfpathlineto{\pgfqpoint{4.935249in}{1.043626in}}%
\pgfpathlineto{\pgfqpoint{4.961578in}{1.043626in}}%
\pgfpathlineto{\pgfqpoint{4.961578in}{1.022977in}}%
\pgfpathlineto{\pgfqpoint{4.987906in}{1.022977in}}%
\pgfpathlineto{\pgfqpoint{4.987906in}{0.857785in}}%
\pgfpathlineto{\pgfqpoint{5.014234in}{0.857785in}}%
\pgfpathlineto{\pgfqpoint{5.014234in}{0.847460in}}%
\pgfpathlineto{\pgfqpoint{5.040563in}{0.847460in}}%
\pgfpathlineto{\pgfqpoint{5.040563in}{0.813045in}}%
\pgfpathlineto{\pgfqpoint{5.066891in}{0.813045in}}%
\pgfpathlineto{\pgfqpoint{5.066891in}{0.737332in}}%
\pgfpathlineto{\pgfqpoint{5.093219in}{0.737332in}}%
\pgfpathlineto{\pgfqpoint{5.093219in}{0.744215in}}%
\pgfpathlineto{\pgfqpoint{5.119547in}{0.744215in}}%
\pgfpathlineto{\pgfqpoint{5.119547in}{0.706358in}}%
\pgfpathlineto{\pgfqpoint{5.145876in}{0.706358in}}%
\pgfpathlineto{\pgfqpoint{5.145876in}{0.685709in}}%
\pgfpathlineto{\pgfqpoint{5.172204in}{0.685709in}}%
\pgfpathlineto{\pgfqpoint{5.172204in}{0.647853in}}%
\pgfpathlineto{\pgfqpoint{5.198532in}{0.647853in}}%
\pgfpathlineto{\pgfqpoint{5.198532in}{0.630645in}}%
\pgfpathlineto{\pgfqpoint{5.224861in}{0.630645in}}%
\pgfpathlineto{\pgfqpoint{5.224861in}{0.640970in}}%
\pgfpathlineto{\pgfqpoint{5.251189in}{0.640970in}}%
\pgfpathlineto{\pgfqpoint{5.251189in}{0.665060in}}%
\pgfpathlineto{\pgfqpoint{5.277517in}{0.665060in}}%
\pgfpathlineto{\pgfqpoint{5.277517in}{0.603113in}}%
\pgfpathlineto{\pgfqpoint{5.303845in}{0.603113in}}%
\pgfpathlineto{\pgfqpoint{5.303845in}{0.596230in}}%
\pgfpathlineto{\pgfqpoint{5.330174in}{0.596230in}}%
\pgfpathlineto{\pgfqpoint{5.330174in}{0.585906in}}%
\pgfpathlineto{\pgfqpoint{5.356502in}{0.585906in}}%
\pgfpathlineto{\pgfqpoint{5.356502in}{0.582464in}}%
\pgfpathlineto{\pgfqpoint{5.382830in}{0.582464in}}%
\pgfpathlineto{\pgfqpoint{5.382830in}{0.579023in}}%
\pgfpathlineto{\pgfqpoint{5.409159in}{0.579023in}}%
\pgfpathlineto{\pgfqpoint{5.409159in}{0.572140in}}%
\pgfpathlineto{\pgfqpoint{5.435487in}{0.572140in}}%
\pgfpathlineto{\pgfqpoint{5.435487in}{0.568698in}}%
\pgfpathlineto{\pgfqpoint{5.461815in}{0.568698in}}%
\pgfpathlineto{\pgfqpoint{5.461815in}{0.561815in}}%
\pgfpathlineto{\pgfqpoint{5.488144in}{0.561815in}}%
\pgfpathlineto{\pgfqpoint{5.488144in}{0.554932in}}%
\pgfpathlineto{\pgfqpoint{5.514472in}{0.554932in}}%
\pgfpathlineto{\pgfqpoint{5.514472in}{0.561815in}}%
\pgfpathlineto{\pgfqpoint{5.540800in}{0.561815in}}%
\pgfpathlineto{\pgfqpoint{5.540800in}{0.551491in}}%
\pgfpathlineto{\pgfqpoint{5.567128in}{0.551491in}}%
\pgfpathlineto{\pgfqpoint{5.567128in}{0.551491in}}%
\pgfpathlineto{\pgfqpoint{5.593457in}{0.551491in}}%
\pgfpathlineto{\pgfqpoint{5.593457in}{0.551491in}}%
\pgfpathlineto{\pgfqpoint{5.619785in}{0.551491in}}%
\pgfpathlineto{\pgfqpoint{5.619785in}{0.548049in}}%
\pgfpathlineto{\pgfqpoint{5.646113in}{0.548049in}}%
\pgfpathlineto{\pgfqpoint{5.646113in}{0.544608in}}%
\pgfpathlineto{\pgfqpoint{5.672442in}{0.544608in}}%
\pgfpathlineto{\pgfqpoint{5.672442in}{0.544608in}}%
\pgfpathlineto{\pgfqpoint{5.698770in}{0.544608in}}%
\pgfpathlineto{\pgfqpoint{5.698770in}{0.548049in}}%
\pgfpathlineto{\pgfqpoint{5.725098in}{0.548049in}}%
\pgfpathlineto{\pgfqpoint{5.725098in}{0.548049in}}%
\pgfpathlineto{\pgfqpoint{5.751426in}{0.548049in}}%
\pgfpathlineto{\pgfqpoint{5.751426in}{0.541166in}}%
\pgfpathlineto{\pgfqpoint{5.777755in}{0.541166in}}%
\pgfpathlineto{\pgfqpoint{5.777755in}{0.541166in}}%
\pgfpathlineto{\pgfqpoint{5.804083in}{0.541166in}}%
\pgfpathlineto{\pgfqpoint{5.804083in}{0.541166in}}%
\pgfpathlineto{\pgfqpoint{5.830411in}{0.541166in}}%
\pgfpathlineto{\pgfqpoint{5.830411in}{0.544608in}}%
\pgfpathlineto{\pgfqpoint{5.856740in}{0.544608in}}%
\pgfpathlineto{\pgfqpoint{5.856740in}{0.541166in}}%
\pgfpathlineto{\pgfqpoint{5.883068in}{0.541166in}}%
\pgfpathlineto{\pgfqpoint{5.883068in}{0.541166in}}%
\pgfpathlineto{\pgfqpoint{5.909396in}{0.541166in}}%
\pgfpathlineto{\pgfqpoint{5.909396in}{0.541166in}}%
\pgfpathlineto{\pgfqpoint{5.935725in}{0.541166in}}%
\pgfpathlineto{\pgfqpoint{5.935725in}{0.541166in}}%
\pgfusepath{stroke}%
\end{pgfscope}%
\begin{pgfscope}%
\pgfsetrectcap%
\pgfsetmiterjoin%
\pgfsetlinewidth{0.803000pt}%
\definecolor{currentstroke}{rgb}{0.000000,0.000000,0.000000}%
\pgfsetstrokecolor{currentstroke}%
\pgfsetdash{}{0pt}%
\pgfpathmoveto{\pgfqpoint{3.510740in}{0.541166in}}%
\pgfpathlineto{\pgfqpoint{3.510740in}{1.959407in}}%
\pgfusepath{stroke}%
\end{pgfscope}%
\begin{pgfscope}%
\pgfsetrectcap%
\pgfsetmiterjoin%
\pgfsetlinewidth{0.803000pt}%
\definecolor{currentstroke}{rgb}{0.000000,0.000000,0.000000}%
\pgfsetstrokecolor{currentstroke}%
\pgfsetdash{}{0pt}%
\pgfpathmoveto{\pgfqpoint{6.051200in}{0.541166in}}%
\pgfpathlineto{\pgfqpoint{6.051200in}{1.959407in}}%
\pgfusepath{stroke}%
\end{pgfscope}%
\begin{pgfscope}%
\pgfsetrectcap%
\pgfsetmiterjoin%
\pgfsetlinewidth{0.803000pt}%
\definecolor{currentstroke}{rgb}{0.000000,0.000000,0.000000}%
\pgfsetstrokecolor{currentstroke}%
\pgfsetdash{}{0pt}%
\pgfpathmoveto{\pgfqpoint{3.510740in}{0.541166in}}%
\pgfpathlineto{\pgfqpoint{6.051200in}{0.541166in}}%
\pgfusepath{stroke}%
\end{pgfscope}%
\begin{pgfscope}%
\pgfsetrectcap%
\pgfsetmiterjoin%
\pgfsetlinewidth{0.803000pt}%
\definecolor{currentstroke}{rgb}{0.000000,0.000000,0.000000}%
\pgfsetstrokecolor{currentstroke}%
\pgfsetdash{}{0pt}%
\pgfpathmoveto{\pgfqpoint{3.510740in}{1.959407in}}%
\pgfpathlineto{\pgfqpoint{6.051200in}{1.959407in}}%
\pgfusepath{stroke}%
\end{pgfscope}%
\begin{pgfscope}%
\definecolor{textcolor}{rgb}{0.000000,0.000000,0.000000}%
\pgfsetstrokecolor{textcolor}%
\pgfsetfillcolor{textcolor}%
\pgftext[x=3.510740in,y=2.042741in,left,base]{\color{textcolor}\rmfamily\fontsize{10.000000}{12.000000}\selectfont Bin [2.83, 3.0], 5,177 events}%
\end{pgfscope}%
\begin{pgfscope}%
\pgfsetbuttcap%
\pgfsetmiterjoin%
\definecolor{currentfill}{rgb}{1.000000,1.000000,1.000000}%
\pgfsetfillcolor{currentfill}%
\pgfsetfillopacity{0.800000}%
\pgfsetlinewidth{1.003750pt}%
\definecolor{currentstroke}{rgb}{0.800000,0.800000,0.800000}%
\pgfsetstrokecolor{currentstroke}%
\pgfsetstrokeopacity{0.800000}%
\pgfsetdash{}{0pt}%
\pgfpathmoveto{\pgfqpoint{5.016422in}{1.559852in}}%
\pgfpathlineto{\pgfqpoint{5.973422in}{1.559852in}}%
\pgfpathquadraticcurveto{\pgfqpoint{5.995644in}{1.559852in}}{\pgfqpoint{5.995644in}{1.582074in}}%
\pgfpathlineto{\pgfqpoint{5.995644in}{1.881629in}}%
\pgfpathquadraticcurveto{\pgfqpoint{5.995644in}{1.903852in}}{\pgfqpoint{5.973422in}{1.903852in}}%
\pgfpathlineto{\pgfqpoint{5.016422in}{1.903852in}}%
\pgfpathquadraticcurveto{\pgfqpoint{4.994200in}{1.903852in}}{\pgfqpoint{4.994200in}{1.881629in}}%
\pgfpathlineto{\pgfqpoint{4.994200in}{1.582074in}}%
\pgfpathquadraticcurveto{\pgfqpoint{4.994200in}{1.559852in}}{\pgfqpoint{5.016422in}{1.559852in}}%
\pgfpathclose%
\pgfusepath{stroke,fill}%
\end{pgfscope}%
\begin{pgfscope}%
\pgfsetbuttcap%
\pgfsetmiterjoin%
\pgfsetlinewidth{1.003750pt}%
\definecolor{currentstroke}{rgb}{0.313725,0.317647,0.309804}%
\pgfsetstrokecolor{currentstroke}%
\pgfsetdash{}{0pt}%
\pgfpathmoveto{\pgfqpoint{5.038644in}{1.781185in}}%
\pgfpathlineto{\pgfqpoint{5.260867in}{1.781185in}}%
\pgfpathlineto{\pgfqpoint{5.260867in}{1.858963in}}%
\pgfpathlineto{\pgfqpoint{5.038644in}{1.858963in}}%
\pgfpathclose%
\pgfusepath{stroke}%
\end{pgfscope}%
\begin{pgfscope}%
\definecolor{textcolor}{rgb}{0.000000,0.000000,0.000000}%
\pgfsetstrokecolor{textcolor}%
\pgfsetfillcolor{textcolor}%
\pgftext[x=5.349756in,y=1.781185in,left,base]{\color{textcolor}\rmfamily\fontsize{8.000000}{9.600000}\selectfont IQR = 0.25}%
\end{pgfscope}%
\begin{pgfscope}%
\pgfsetbuttcap%
\pgfsetmiterjoin%
\pgfsetlinewidth{1.003750pt}%
\definecolor{currentstroke}{rgb}{0.949020,0.372549,0.360784}%
\pgfsetstrokecolor{currentstroke}%
\pgfsetdash{{1.000000pt}{1.650000pt}}{0.000000pt}%
\pgfpathmoveto{\pgfqpoint{5.038644in}{1.625852in}}%
\pgfpathlineto{\pgfqpoint{5.260867in}{1.625852in}}%
\pgfpathlineto{\pgfqpoint{5.260867in}{1.703629in}}%
\pgfpathlineto{\pgfqpoint{5.038644in}{1.703629in}}%
\pgfpathclose%
\pgfusepath{stroke}%
\end{pgfscope}%
\begin{pgfscope}%
\definecolor{textcolor}{rgb}{0.000000,0.000000,0.000000}%
\pgfsetstrokecolor{textcolor}%
\pgfsetfillcolor{textcolor}%
\pgftext[x=5.349756in,y=1.625852in,left,base]{\color{textcolor}\rmfamily\fontsize{8.000000}{9.600000}\selectfont IQR = 0.26}%
\end{pgfscope}%
\begin{pgfscope}%
\definecolor{textcolor}{rgb}{0.000000,0.000000,0.000000}%
\pgfsetstrokecolor{textcolor}%
\pgfsetfillcolor{textcolor}%
\pgftext[x=0.620120in,y=6.370000in,left,top]{\color{textcolor}\rmfamily\fontsize{12.000000}{14.400000}\selectfont Energy error distribution in selected bins}%
\end{pgfscope}%
\begin{pgfscope}%
\pgfsetbuttcap%
\pgfsetmiterjoin%
\definecolor{currentfill}{rgb}{1.000000,1.000000,1.000000}%
\pgfsetfillcolor{currentfill}%
\pgfsetfillopacity{0.800000}%
\pgfsetlinewidth{1.003750pt}%
\definecolor{currentstroke}{rgb}{0.800000,0.800000,0.800000}%
\pgfsetstrokecolor{currentstroke}%
\pgfsetstrokeopacity{0.800000}%
\pgfsetdash{}{0pt}%
\pgfpathmoveto{\pgfqpoint{5.187533in}{6.101333in}}%
\pgfpathlineto{\pgfqpoint{6.123422in}{6.101333in}}%
\pgfpathquadraticcurveto{\pgfqpoint{6.145644in}{6.101333in}}{\pgfqpoint{6.145644in}{6.123556in}}%
\pgfpathlineto{\pgfqpoint{6.145644in}{6.422222in}}%
\pgfpathquadraticcurveto{\pgfqpoint{6.145644in}{6.444444in}}{\pgfqpoint{6.123422in}{6.444444in}}%
\pgfpathlineto{\pgfqpoint{5.187533in}{6.444444in}}%
\pgfpathquadraticcurveto{\pgfqpoint{5.165311in}{6.444444in}}{\pgfqpoint{5.165311in}{6.422222in}}%
\pgfpathlineto{\pgfqpoint{5.165311in}{6.123556in}}%
\pgfpathquadraticcurveto{\pgfqpoint{5.165311in}{6.101333in}}{\pgfqpoint{5.187533in}{6.101333in}}%
\pgfpathclose%
\pgfusepath{stroke,fill}%
\end{pgfscope}%
\begin{pgfscope}%
\pgfsetbuttcap%
\pgfsetmiterjoin%
\pgfsetlinewidth{1.003750pt}%
\definecolor{currentstroke}{rgb}{0.313725,0.317647,0.309804}%
\pgfsetstrokecolor{currentstroke}%
\pgfsetdash{}{0pt}%
\pgfpathmoveto{\pgfqpoint{5.209756in}{6.322222in}}%
\pgfpathlineto{\pgfqpoint{5.431978in}{6.322222in}}%
\pgfpathlineto{\pgfqpoint{5.431978in}{6.400000in}}%
\pgfpathlineto{\pgfqpoint{5.209756in}{6.400000in}}%
\pgfpathclose%
\pgfusepath{stroke}%
\end{pgfscope}%
\begin{pgfscope}%
\definecolor{textcolor}{rgb}{0.000000,0.000000,0.000000}%
\pgfsetstrokecolor{textcolor}%
\pgfsetfillcolor{textcolor}%
\pgftext[x=5.520867in,y=6.322222in,left,base]{\color{textcolor}\rmfamily\fontsize{8.000000}{9.600000}\selectfont CubeFlow}%
\end{pgfscope}%
\begin{pgfscope}%
\pgfsetbuttcap%
\pgfsetmiterjoin%
\pgfsetlinewidth{1.003750pt}%
\definecolor{currentstroke}{rgb}{0.949020,0.372549,0.360784}%
\pgfsetstrokecolor{currentstroke}%
\pgfsetdash{{1.000000pt}{1.650000pt}}{0.000000pt}%
\pgfpathmoveto{\pgfqpoint{5.209756in}{6.167333in}}%
\pgfpathlineto{\pgfqpoint{5.431978in}{6.167333in}}%
\pgfpathlineto{\pgfqpoint{5.431978in}{6.245111in}}%
\pgfpathlineto{\pgfqpoint{5.209756in}{6.245111in}}%
\pgfpathclose%
\pgfusepath{stroke}%
\end{pgfscope}%
\begin{pgfscope}%
\definecolor{textcolor}{rgb}{0.000000,0.000000,0.000000}%
\pgfsetstrokecolor{textcolor}%
\pgfsetfillcolor{textcolor}%
\pgftext[x=5.520867in,y=6.167333in,left,base]{\color{textcolor}\rmfamily\fontsize{8.000000}{9.600000}\selectfont Retro Reco}%
\end{pgfscope}%
\end{pgfpicture}%
\makeatother%
\endgroup%

    \caption{Energy error distribution in certain selected bins, representative of the overall performance difference.
    At low energy CubeFlow performs better, seen by a narrower distribution than that of Retro Reco.
    It is evident that both algorithms have an inherent bias: at low energy they tend to overshoot the energy, while at higher energies they undershoot.
    In the case of the neural network, this is probably caused by the relative lack of training data in the tails.}\label{fig:energy_bins}
\end{figure}
\begin{figure}
    \centering
    %% Creator: Matplotlib, PGF backend
%%
%% To include the figure in your LaTeX document, write
%%   \input{<filename>.pgf}
%%
%% Make sure the required packages are loaded in your preamble
%%   \usepackage{pgf}
%%
%% and, on pdftex
%%   \usepackage[utf8]{inputenc}\DeclareUnicodeCharacter{2212}{-}
%%
%% or, on luatex and xetex
%%   \usepackage{unicode-math}
%%
%% Figures using additional raster images can only be included by \input if
%% they are in the same directory as the main LaTeX file. For loading figures
%% from other directories you can use the `import` package
%%   \usepackage{import}
%%
%% and then include the figures with
%%   \import{<path to file>}{<filename>.pgf}
%%
%% Matplotlib used the following preamble
%%   \usepackage{siunitx} \usepackage{amsmath} \usepackage{bm}
%%   \usepackage{fontspec}
%%
\begingroup%
\makeatletter%
\begin{pgfpicture}%
\pgfpathrectangle{\pgfpointorigin}{\pgfqpoint{6.201200in}{6.500000in}}%
\pgfusepath{use as bounding box, clip}%
\begin{pgfscope}%
\pgfsetbuttcap%
\pgfsetmiterjoin%
\definecolor{currentfill}{rgb}{1.000000,1.000000,1.000000}%
\pgfsetfillcolor{currentfill}%
\pgfsetlinewidth{0.000000pt}%
\definecolor{currentstroke}{rgb}{1.000000,1.000000,1.000000}%
\pgfsetstrokecolor{currentstroke}%
\pgfsetdash{}{0pt}%
\pgfpathmoveto{\pgfqpoint{0.000000in}{0.000000in}}%
\pgfpathlineto{\pgfqpoint{6.201200in}{0.000000in}}%
\pgfpathlineto{\pgfqpoint{6.201200in}{6.500000in}}%
\pgfpathlineto{\pgfqpoint{0.000000in}{6.500000in}}%
\pgfpathclose%
\pgfusepath{fill}%
\end{pgfscope}%
\begin{pgfscope}%
\pgfsetbuttcap%
\pgfsetmiterjoin%
\definecolor{currentfill}{rgb}{1.000000,1.000000,1.000000}%
\pgfsetfillcolor{currentfill}%
\pgfsetlinewidth{0.000000pt}%
\definecolor{currentstroke}{rgb}{0.000000,0.000000,0.000000}%
\pgfsetstrokecolor{currentstroke}%
\pgfsetstrokeopacity{0.000000}%
\pgfsetdash{}{0pt}%
\pgfpathmoveto{\pgfqpoint{0.754048in}{4.447028in}}%
\pgfpathlineto{\pgfqpoint{3.055114in}{4.447028in}}%
\pgfpathlineto{\pgfqpoint{3.055114in}{5.864500in}}%
\pgfpathlineto{\pgfqpoint{0.754048in}{5.864500in}}%
\pgfpathclose%
\pgfusepath{fill}%
\end{pgfscope}%
\begin{pgfscope}%
\pgfsetbuttcap%
\pgfsetroundjoin%
\definecolor{currentfill}{rgb}{0.000000,0.000000,0.000000}%
\pgfsetfillcolor{currentfill}%
\pgfsetlinewidth{0.803000pt}%
\definecolor{currentstroke}{rgb}{0.000000,0.000000,0.000000}%
\pgfsetstrokecolor{currentstroke}%
\pgfsetdash{}{0pt}%
\pgfsys@defobject{currentmarker}{\pgfqpoint{0.000000in}{-0.048611in}}{\pgfqpoint{0.000000in}{0.000000in}}{%
\pgfpathmoveto{\pgfqpoint{0.000000in}{0.000000in}}%
\pgfpathlineto{\pgfqpoint{0.000000in}{-0.048611in}}%
\pgfusepath{stroke,fill}%
}%
\begin{pgfscope}%
\pgfsys@transformshift{0.986179in}{4.447028in}%
\pgfsys@useobject{currentmarker}{}%
\end{pgfscope}%
\end{pgfscope}%
\begin{pgfscope}%
\definecolor{textcolor}{rgb}{0.000000,0.000000,0.000000}%
\pgfsetstrokecolor{textcolor}%
\pgfsetfillcolor{textcolor}%
\pgftext[x=0.986179in,y=4.349806in,,top]{\color{textcolor}\rmfamily\fontsize{8.000000}{9.600000}\selectfont \(\displaystyle {-120}\)}%
\end{pgfscope}%
\begin{pgfscope}%
\pgfsetbuttcap%
\pgfsetroundjoin%
\definecolor{currentfill}{rgb}{0.000000,0.000000,0.000000}%
\pgfsetfillcolor{currentfill}%
\pgfsetlinewidth{0.803000pt}%
\definecolor{currentstroke}{rgb}{0.000000,0.000000,0.000000}%
\pgfsetstrokecolor{currentstroke}%
\pgfsetdash{}{0pt}%
\pgfsys@defobject{currentmarker}{\pgfqpoint{0.000000in}{-0.048611in}}{\pgfqpoint{0.000000in}{0.000000in}}{%
\pgfpathmoveto{\pgfqpoint{0.000000in}{0.000000in}}%
\pgfpathlineto{\pgfqpoint{0.000000in}{-0.048611in}}%
\pgfusepath{stroke,fill}%
}%
\begin{pgfscope}%
\pgfsys@transformshift{1.291881in}{4.447028in}%
\pgfsys@useobject{currentmarker}{}%
\end{pgfscope}%
\end{pgfscope}%
\begin{pgfscope}%
\definecolor{textcolor}{rgb}{0.000000,0.000000,0.000000}%
\pgfsetstrokecolor{textcolor}%
\pgfsetfillcolor{textcolor}%
\pgftext[x=1.291881in,y=4.349806in,,top]{\color{textcolor}\rmfamily\fontsize{8.000000}{9.600000}\selectfont \(\displaystyle {-80}\)}%
\end{pgfscope}%
\begin{pgfscope}%
\pgfsetbuttcap%
\pgfsetroundjoin%
\definecolor{currentfill}{rgb}{0.000000,0.000000,0.000000}%
\pgfsetfillcolor{currentfill}%
\pgfsetlinewidth{0.803000pt}%
\definecolor{currentstroke}{rgb}{0.000000,0.000000,0.000000}%
\pgfsetstrokecolor{currentstroke}%
\pgfsetdash{}{0pt}%
\pgfsys@defobject{currentmarker}{\pgfqpoint{0.000000in}{-0.048611in}}{\pgfqpoint{0.000000in}{0.000000in}}{%
\pgfpathmoveto{\pgfqpoint{0.000000in}{0.000000in}}%
\pgfpathlineto{\pgfqpoint{0.000000in}{-0.048611in}}%
\pgfusepath{stroke,fill}%
}%
\begin{pgfscope}%
\pgfsys@transformshift{1.597582in}{4.447028in}%
\pgfsys@useobject{currentmarker}{}%
\end{pgfscope}%
\end{pgfscope}%
\begin{pgfscope}%
\definecolor{textcolor}{rgb}{0.000000,0.000000,0.000000}%
\pgfsetstrokecolor{textcolor}%
\pgfsetfillcolor{textcolor}%
\pgftext[x=1.597582in,y=4.349806in,,top]{\color{textcolor}\rmfamily\fontsize{8.000000}{9.600000}\selectfont \(\displaystyle {-40}\)}%
\end{pgfscope}%
\begin{pgfscope}%
\pgfsetbuttcap%
\pgfsetroundjoin%
\definecolor{currentfill}{rgb}{0.000000,0.000000,0.000000}%
\pgfsetfillcolor{currentfill}%
\pgfsetlinewidth{0.803000pt}%
\definecolor{currentstroke}{rgb}{0.000000,0.000000,0.000000}%
\pgfsetstrokecolor{currentstroke}%
\pgfsetdash{}{0pt}%
\pgfsys@defobject{currentmarker}{\pgfqpoint{0.000000in}{-0.048611in}}{\pgfqpoint{0.000000in}{0.000000in}}{%
\pgfpathmoveto{\pgfqpoint{0.000000in}{0.000000in}}%
\pgfpathlineto{\pgfqpoint{0.000000in}{-0.048611in}}%
\pgfusepath{stroke,fill}%
}%
\begin{pgfscope}%
\pgfsys@transformshift{1.903283in}{4.447028in}%
\pgfsys@useobject{currentmarker}{}%
\end{pgfscope}%
\end{pgfscope}%
\begin{pgfscope}%
\definecolor{textcolor}{rgb}{0.000000,0.000000,0.000000}%
\pgfsetstrokecolor{textcolor}%
\pgfsetfillcolor{textcolor}%
\pgftext[x=1.903283in,y=4.349806in,,top]{\color{textcolor}\rmfamily\fontsize{8.000000}{9.600000}\selectfont \(\displaystyle {0}\)}%
\end{pgfscope}%
\begin{pgfscope}%
\pgfsetbuttcap%
\pgfsetroundjoin%
\definecolor{currentfill}{rgb}{0.000000,0.000000,0.000000}%
\pgfsetfillcolor{currentfill}%
\pgfsetlinewidth{0.803000pt}%
\definecolor{currentstroke}{rgb}{0.000000,0.000000,0.000000}%
\pgfsetstrokecolor{currentstroke}%
\pgfsetdash{}{0pt}%
\pgfsys@defobject{currentmarker}{\pgfqpoint{0.000000in}{-0.048611in}}{\pgfqpoint{0.000000in}{0.000000in}}{%
\pgfpathmoveto{\pgfqpoint{0.000000in}{0.000000in}}%
\pgfpathlineto{\pgfqpoint{0.000000in}{-0.048611in}}%
\pgfusepath{stroke,fill}%
}%
\begin{pgfscope}%
\pgfsys@transformshift{2.208985in}{4.447028in}%
\pgfsys@useobject{currentmarker}{}%
\end{pgfscope}%
\end{pgfscope}%
\begin{pgfscope}%
\definecolor{textcolor}{rgb}{0.000000,0.000000,0.000000}%
\pgfsetstrokecolor{textcolor}%
\pgfsetfillcolor{textcolor}%
\pgftext[x=2.208985in,y=4.349806in,,top]{\color{textcolor}\rmfamily\fontsize{8.000000}{9.600000}\selectfont \(\displaystyle {40}\)}%
\end{pgfscope}%
\begin{pgfscope}%
\pgfsetbuttcap%
\pgfsetroundjoin%
\definecolor{currentfill}{rgb}{0.000000,0.000000,0.000000}%
\pgfsetfillcolor{currentfill}%
\pgfsetlinewidth{0.803000pt}%
\definecolor{currentstroke}{rgb}{0.000000,0.000000,0.000000}%
\pgfsetstrokecolor{currentstroke}%
\pgfsetdash{}{0pt}%
\pgfsys@defobject{currentmarker}{\pgfqpoint{0.000000in}{-0.048611in}}{\pgfqpoint{0.000000in}{0.000000in}}{%
\pgfpathmoveto{\pgfqpoint{0.000000in}{0.000000in}}%
\pgfpathlineto{\pgfqpoint{0.000000in}{-0.048611in}}%
\pgfusepath{stroke,fill}%
}%
\begin{pgfscope}%
\pgfsys@transformshift{2.514686in}{4.447028in}%
\pgfsys@useobject{currentmarker}{}%
\end{pgfscope}%
\end{pgfscope}%
\begin{pgfscope}%
\definecolor{textcolor}{rgb}{0.000000,0.000000,0.000000}%
\pgfsetstrokecolor{textcolor}%
\pgfsetfillcolor{textcolor}%
\pgftext[x=2.514686in,y=4.349806in,,top]{\color{textcolor}\rmfamily\fontsize{8.000000}{9.600000}\selectfont \(\displaystyle {80}\)}%
\end{pgfscope}%
\begin{pgfscope}%
\pgfsetbuttcap%
\pgfsetroundjoin%
\definecolor{currentfill}{rgb}{0.000000,0.000000,0.000000}%
\pgfsetfillcolor{currentfill}%
\pgfsetlinewidth{0.803000pt}%
\definecolor{currentstroke}{rgb}{0.000000,0.000000,0.000000}%
\pgfsetstrokecolor{currentstroke}%
\pgfsetdash{}{0pt}%
\pgfsys@defobject{currentmarker}{\pgfqpoint{0.000000in}{-0.048611in}}{\pgfqpoint{0.000000in}{0.000000in}}{%
\pgfpathmoveto{\pgfqpoint{0.000000in}{0.000000in}}%
\pgfpathlineto{\pgfqpoint{0.000000in}{-0.048611in}}%
\pgfusepath{stroke,fill}%
}%
\begin{pgfscope}%
\pgfsys@transformshift{2.820387in}{4.447028in}%
\pgfsys@useobject{currentmarker}{}%
\end{pgfscope}%
\end{pgfscope}%
\begin{pgfscope}%
\definecolor{textcolor}{rgb}{0.000000,0.000000,0.000000}%
\pgfsetstrokecolor{textcolor}%
\pgfsetfillcolor{textcolor}%
\pgftext[x=2.820387in,y=4.349806in,,top]{\color{textcolor}\rmfamily\fontsize{8.000000}{9.600000}\selectfont \(\displaystyle {120}\)}%
\end{pgfscope}%
\begin{pgfscope}%
\pgfsetbuttcap%
\pgfsetroundjoin%
\definecolor{currentfill}{rgb}{0.000000,0.000000,0.000000}%
\pgfsetfillcolor{currentfill}%
\pgfsetlinewidth{0.803000pt}%
\definecolor{currentstroke}{rgb}{0.000000,0.000000,0.000000}%
\pgfsetstrokecolor{currentstroke}%
\pgfsetdash{}{0pt}%
\pgfsys@defobject{currentmarker}{\pgfqpoint{-0.048611in}{0.000000in}}{\pgfqpoint{-0.000000in}{0.000000in}}{%
\pgfpathmoveto{\pgfqpoint{-0.000000in}{0.000000in}}%
\pgfpathlineto{\pgfqpoint{-0.048611in}{0.000000in}}%
\pgfusepath{stroke,fill}%
}%
\begin{pgfscope}%
\pgfsys@transformshift{0.754048in}{4.447028in}%
\pgfsys@useobject{currentmarker}{}%
\end{pgfscope}%
\end{pgfscope}%
\begin{pgfscope}%
\definecolor{textcolor}{rgb}{0.000000,0.000000,0.000000}%
\pgfsetstrokecolor{textcolor}%
\pgfsetfillcolor{textcolor}%
\pgftext[x=0.328889in, y=4.408472in, left, base]{\color{textcolor}\rmfamily\fontsize{8.000000}{9.600000}\selectfont \(\displaystyle {0.0000}\)}%
\end{pgfscope}%
\begin{pgfscope}%
\pgfsetbuttcap%
\pgfsetroundjoin%
\definecolor{currentfill}{rgb}{0.000000,0.000000,0.000000}%
\pgfsetfillcolor{currentfill}%
\pgfsetlinewidth{0.803000pt}%
\definecolor{currentstroke}{rgb}{0.000000,0.000000,0.000000}%
\pgfsetstrokecolor{currentstroke}%
\pgfsetdash{}{0pt}%
\pgfsys@defobject{currentmarker}{\pgfqpoint{-0.048611in}{0.000000in}}{\pgfqpoint{-0.000000in}{0.000000in}}{%
\pgfpathmoveto{\pgfqpoint{-0.000000in}{0.000000in}}%
\pgfpathlineto{\pgfqpoint{-0.048611in}{0.000000in}}%
\pgfusepath{stroke,fill}%
}%
\begin{pgfscope}%
\pgfsys@transformshift{0.754048in}{4.620408in}%
\pgfsys@useobject{currentmarker}{}%
\end{pgfscope}%
\end{pgfscope}%
\begin{pgfscope}%
\definecolor{textcolor}{rgb}{0.000000,0.000000,0.000000}%
\pgfsetstrokecolor{textcolor}%
\pgfsetfillcolor{textcolor}%
\pgftext[x=0.328889in, y=4.581852in, left, base]{\color{textcolor}\rmfamily\fontsize{8.000000}{9.600000}\selectfont \(\displaystyle {0.0025}\)}%
\end{pgfscope}%
\begin{pgfscope}%
\pgfsetbuttcap%
\pgfsetroundjoin%
\definecolor{currentfill}{rgb}{0.000000,0.000000,0.000000}%
\pgfsetfillcolor{currentfill}%
\pgfsetlinewidth{0.803000pt}%
\definecolor{currentstroke}{rgb}{0.000000,0.000000,0.000000}%
\pgfsetstrokecolor{currentstroke}%
\pgfsetdash{}{0pt}%
\pgfsys@defobject{currentmarker}{\pgfqpoint{-0.048611in}{0.000000in}}{\pgfqpoint{-0.000000in}{0.000000in}}{%
\pgfpathmoveto{\pgfqpoint{-0.000000in}{0.000000in}}%
\pgfpathlineto{\pgfqpoint{-0.048611in}{0.000000in}}%
\pgfusepath{stroke,fill}%
}%
\begin{pgfscope}%
\pgfsys@transformshift{0.754048in}{4.793788in}%
\pgfsys@useobject{currentmarker}{}%
\end{pgfscope}%
\end{pgfscope}%
\begin{pgfscope}%
\definecolor{textcolor}{rgb}{0.000000,0.000000,0.000000}%
\pgfsetstrokecolor{textcolor}%
\pgfsetfillcolor{textcolor}%
\pgftext[x=0.328889in, y=4.755232in, left, base]{\color{textcolor}\rmfamily\fontsize{8.000000}{9.600000}\selectfont \(\displaystyle {0.0050}\)}%
\end{pgfscope}%
\begin{pgfscope}%
\pgfsetbuttcap%
\pgfsetroundjoin%
\definecolor{currentfill}{rgb}{0.000000,0.000000,0.000000}%
\pgfsetfillcolor{currentfill}%
\pgfsetlinewidth{0.803000pt}%
\definecolor{currentstroke}{rgb}{0.000000,0.000000,0.000000}%
\pgfsetstrokecolor{currentstroke}%
\pgfsetdash{}{0pt}%
\pgfsys@defobject{currentmarker}{\pgfqpoint{-0.048611in}{0.000000in}}{\pgfqpoint{-0.000000in}{0.000000in}}{%
\pgfpathmoveto{\pgfqpoint{-0.000000in}{0.000000in}}%
\pgfpathlineto{\pgfqpoint{-0.048611in}{0.000000in}}%
\pgfusepath{stroke,fill}%
}%
\begin{pgfscope}%
\pgfsys@transformshift{0.754048in}{4.967168in}%
\pgfsys@useobject{currentmarker}{}%
\end{pgfscope}%
\end{pgfscope}%
\begin{pgfscope}%
\definecolor{textcolor}{rgb}{0.000000,0.000000,0.000000}%
\pgfsetstrokecolor{textcolor}%
\pgfsetfillcolor{textcolor}%
\pgftext[x=0.328889in, y=4.928612in, left, base]{\color{textcolor}\rmfamily\fontsize{8.000000}{9.600000}\selectfont \(\displaystyle {0.0075}\)}%
\end{pgfscope}%
\begin{pgfscope}%
\pgfsetbuttcap%
\pgfsetroundjoin%
\definecolor{currentfill}{rgb}{0.000000,0.000000,0.000000}%
\pgfsetfillcolor{currentfill}%
\pgfsetlinewidth{0.803000pt}%
\definecolor{currentstroke}{rgb}{0.000000,0.000000,0.000000}%
\pgfsetstrokecolor{currentstroke}%
\pgfsetdash{}{0pt}%
\pgfsys@defobject{currentmarker}{\pgfqpoint{-0.048611in}{0.000000in}}{\pgfqpoint{-0.000000in}{0.000000in}}{%
\pgfpathmoveto{\pgfqpoint{-0.000000in}{0.000000in}}%
\pgfpathlineto{\pgfqpoint{-0.048611in}{0.000000in}}%
\pgfusepath{stroke,fill}%
}%
\begin{pgfscope}%
\pgfsys@transformshift{0.754048in}{5.140548in}%
\pgfsys@useobject{currentmarker}{}%
\end{pgfscope}%
\end{pgfscope}%
\begin{pgfscope}%
\definecolor{textcolor}{rgb}{0.000000,0.000000,0.000000}%
\pgfsetstrokecolor{textcolor}%
\pgfsetfillcolor{textcolor}%
\pgftext[x=0.328889in, y=5.101992in, left, base]{\color{textcolor}\rmfamily\fontsize{8.000000}{9.600000}\selectfont \(\displaystyle {0.0100}\)}%
\end{pgfscope}%
\begin{pgfscope}%
\pgfsetbuttcap%
\pgfsetroundjoin%
\definecolor{currentfill}{rgb}{0.000000,0.000000,0.000000}%
\pgfsetfillcolor{currentfill}%
\pgfsetlinewidth{0.803000pt}%
\definecolor{currentstroke}{rgb}{0.000000,0.000000,0.000000}%
\pgfsetstrokecolor{currentstroke}%
\pgfsetdash{}{0pt}%
\pgfsys@defobject{currentmarker}{\pgfqpoint{-0.048611in}{0.000000in}}{\pgfqpoint{-0.000000in}{0.000000in}}{%
\pgfpathmoveto{\pgfqpoint{-0.000000in}{0.000000in}}%
\pgfpathlineto{\pgfqpoint{-0.048611in}{0.000000in}}%
\pgfusepath{stroke,fill}%
}%
\begin{pgfscope}%
\pgfsys@transformshift{0.754048in}{5.313928in}%
\pgfsys@useobject{currentmarker}{}%
\end{pgfscope}%
\end{pgfscope}%
\begin{pgfscope}%
\definecolor{textcolor}{rgb}{0.000000,0.000000,0.000000}%
\pgfsetstrokecolor{textcolor}%
\pgfsetfillcolor{textcolor}%
\pgftext[x=0.328889in, y=5.275372in, left, base]{\color{textcolor}\rmfamily\fontsize{8.000000}{9.600000}\selectfont \(\displaystyle {0.0125}\)}%
\end{pgfscope}%
\begin{pgfscope}%
\pgfsetbuttcap%
\pgfsetroundjoin%
\definecolor{currentfill}{rgb}{0.000000,0.000000,0.000000}%
\pgfsetfillcolor{currentfill}%
\pgfsetlinewidth{0.803000pt}%
\definecolor{currentstroke}{rgb}{0.000000,0.000000,0.000000}%
\pgfsetstrokecolor{currentstroke}%
\pgfsetdash{}{0pt}%
\pgfsys@defobject{currentmarker}{\pgfqpoint{-0.048611in}{0.000000in}}{\pgfqpoint{-0.000000in}{0.000000in}}{%
\pgfpathmoveto{\pgfqpoint{-0.000000in}{0.000000in}}%
\pgfpathlineto{\pgfqpoint{-0.048611in}{0.000000in}}%
\pgfusepath{stroke,fill}%
}%
\begin{pgfscope}%
\pgfsys@transformshift{0.754048in}{5.487308in}%
\pgfsys@useobject{currentmarker}{}%
\end{pgfscope}%
\end{pgfscope}%
\begin{pgfscope}%
\definecolor{textcolor}{rgb}{0.000000,0.000000,0.000000}%
\pgfsetstrokecolor{textcolor}%
\pgfsetfillcolor{textcolor}%
\pgftext[x=0.328889in, y=5.448752in, left, base]{\color{textcolor}\rmfamily\fontsize{8.000000}{9.600000}\selectfont \(\displaystyle {0.0150}\)}%
\end{pgfscope}%
\begin{pgfscope}%
\pgfsetbuttcap%
\pgfsetroundjoin%
\definecolor{currentfill}{rgb}{0.000000,0.000000,0.000000}%
\pgfsetfillcolor{currentfill}%
\pgfsetlinewidth{0.803000pt}%
\definecolor{currentstroke}{rgb}{0.000000,0.000000,0.000000}%
\pgfsetstrokecolor{currentstroke}%
\pgfsetdash{}{0pt}%
\pgfsys@defobject{currentmarker}{\pgfqpoint{-0.048611in}{0.000000in}}{\pgfqpoint{-0.000000in}{0.000000in}}{%
\pgfpathmoveto{\pgfqpoint{-0.000000in}{0.000000in}}%
\pgfpathlineto{\pgfqpoint{-0.048611in}{0.000000in}}%
\pgfusepath{stroke,fill}%
}%
\begin{pgfscope}%
\pgfsys@transformshift{0.754048in}{5.660688in}%
\pgfsys@useobject{currentmarker}{}%
\end{pgfscope}%
\end{pgfscope}%
\begin{pgfscope}%
\definecolor{textcolor}{rgb}{0.000000,0.000000,0.000000}%
\pgfsetstrokecolor{textcolor}%
\pgfsetfillcolor{textcolor}%
\pgftext[x=0.328889in, y=5.622132in, left, base]{\color{textcolor}\rmfamily\fontsize{8.000000}{9.600000}\selectfont \(\displaystyle {0.0175}\)}%
\end{pgfscope}%
\begin{pgfscope}%
\pgfsetbuttcap%
\pgfsetroundjoin%
\definecolor{currentfill}{rgb}{0.000000,0.000000,0.000000}%
\pgfsetfillcolor{currentfill}%
\pgfsetlinewidth{0.803000pt}%
\definecolor{currentstroke}{rgb}{0.000000,0.000000,0.000000}%
\pgfsetstrokecolor{currentstroke}%
\pgfsetdash{}{0pt}%
\pgfsys@defobject{currentmarker}{\pgfqpoint{-0.048611in}{0.000000in}}{\pgfqpoint{-0.000000in}{0.000000in}}{%
\pgfpathmoveto{\pgfqpoint{-0.000000in}{0.000000in}}%
\pgfpathlineto{\pgfqpoint{-0.048611in}{0.000000in}}%
\pgfusepath{stroke,fill}%
}%
\begin{pgfscope}%
\pgfsys@transformshift{0.754048in}{5.834068in}%
\pgfsys@useobject{currentmarker}{}%
\end{pgfscope}%
\end{pgfscope}%
\begin{pgfscope}%
\definecolor{textcolor}{rgb}{0.000000,0.000000,0.000000}%
\pgfsetstrokecolor{textcolor}%
\pgfsetfillcolor{textcolor}%
\pgftext[x=0.328889in, y=5.795512in, left, base]{\color{textcolor}\rmfamily\fontsize{8.000000}{9.600000}\selectfont \(\displaystyle {0.0200}\)}%
\end{pgfscope}%
\begin{pgfscope}%
\definecolor{textcolor}{rgb}{0.000000,0.000000,0.000000}%
\pgfsetstrokecolor{textcolor}%
\pgfsetfillcolor{textcolor}%
\pgftext[x=0.273333in,y=5.155764in,,bottom,rotate=90.000000]{\color{textcolor}\rmfamily\fontsize{10.000000}{12.000000}\selectfont Density}%
\end{pgfscope}%
\begin{pgfscope}%
\pgfpathrectangle{\pgfqpoint{0.754048in}{4.447028in}}{\pgfqpoint{2.301066in}{1.417472in}}%
\pgfusepath{clip}%
\pgfsetbuttcap%
\pgfsetmiterjoin%
\pgfsetlinewidth{1.003750pt}%
\definecolor{currentstroke}{rgb}{0.313725,0.317647,0.309804}%
\pgfsetstrokecolor{currentstroke}%
\pgfsetdash{}{0pt}%
\pgfpathmoveto{\pgfqpoint{1.274414in}{4.447028in}}%
\pgfpathlineto{\pgfqpoint{1.274414in}{4.460945in}}%
\pgfpathlineto{\pgfqpoint{1.313058in}{4.460945in}}%
\pgfpathlineto{\pgfqpoint{1.313058in}{4.453986in}}%
\pgfpathlineto{\pgfqpoint{1.351703in}{4.453986in}}%
\pgfpathlineto{\pgfqpoint{1.351703in}{4.460945in}}%
\pgfpathlineto{\pgfqpoint{1.390347in}{4.460945in}}%
\pgfpathlineto{\pgfqpoint{1.390347in}{4.488779in}}%
\pgfpathlineto{\pgfqpoint{1.428991in}{4.488779in}}%
\pgfpathlineto{\pgfqpoint{1.428991in}{4.516614in}}%
\pgfpathlineto{\pgfqpoint{1.467636in}{4.516614in}}%
\pgfpathlineto{\pgfqpoint{1.467636in}{4.586200in}}%
\pgfpathlineto{\pgfqpoint{1.506280in}{4.586200in}}%
\pgfpathlineto{\pgfqpoint{1.506280in}{4.558366in}}%
\pgfpathlineto{\pgfqpoint{1.544925in}{4.558366in}}%
\pgfpathlineto{\pgfqpoint{1.544925in}{4.614035in}}%
\pgfpathlineto{\pgfqpoint{1.583569in}{4.614035in}}%
\pgfpathlineto{\pgfqpoint{1.583569in}{4.676662in}}%
\pgfpathlineto{\pgfqpoint{1.622213in}{4.676662in}}%
\pgfpathlineto{\pgfqpoint{1.622213in}{4.732331in}}%
\pgfpathlineto{\pgfqpoint{1.660858in}{4.732331in}}%
\pgfpathlineto{\pgfqpoint{1.660858in}{4.955007in}}%
\pgfpathlineto{\pgfqpoint{1.699502in}{4.955007in}}%
\pgfpathlineto{\pgfqpoint{1.699502in}{5.115056in}}%
\pgfpathlineto{\pgfqpoint{1.738146in}{5.115056in}}%
\pgfpathlineto{\pgfqpoint{1.738146in}{5.177684in}}%
\pgfpathlineto{\pgfqpoint{1.776791in}{5.177684in}}%
\pgfpathlineto{\pgfqpoint{1.776791in}{5.344691in}}%
\pgfpathlineto{\pgfqpoint{1.815435in}{5.344691in}}%
\pgfpathlineto{\pgfqpoint{1.815435in}{5.476904in}}%
\pgfpathlineto{\pgfqpoint{1.854080in}{5.476904in}}%
\pgfpathlineto{\pgfqpoint{1.854080in}{5.797001in}}%
\pgfpathlineto{\pgfqpoint{1.892724in}{5.797001in}}%
\pgfpathlineto{\pgfqpoint{1.892724in}{5.629994in}}%
\pgfpathlineto{\pgfqpoint{1.931368in}{5.629994in}}%
\pgfpathlineto{\pgfqpoint{1.931368in}{5.602160in}}%
\pgfpathlineto{\pgfqpoint{1.970013in}{5.602160in}}%
\pgfpathlineto{\pgfqpoint{1.970013in}{5.428194in}}%
\pgfpathlineto{\pgfqpoint{2.008657in}{5.428194in}}%
\pgfpathlineto{\pgfqpoint{2.008657in}{5.337732in}}%
\pgfpathlineto{\pgfqpoint{2.047301in}{5.337732in}}%
\pgfpathlineto{\pgfqpoint{2.047301in}{5.337732in}}%
\pgfpathlineto{\pgfqpoint{2.085946in}{5.337732in}}%
\pgfpathlineto{\pgfqpoint{2.085946in}{5.017635in}}%
\pgfpathlineto{\pgfqpoint{2.124590in}{5.017635in}}%
\pgfpathlineto{\pgfqpoint{2.124590in}{5.128973in}}%
\pgfpathlineto{\pgfqpoint{2.163234in}{5.128973in}}%
\pgfpathlineto{\pgfqpoint{2.163234in}{4.739290in}}%
\pgfpathlineto{\pgfqpoint{2.201879in}{4.739290in}}%
\pgfpathlineto{\pgfqpoint{2.201879in}{4.711456in}}%
\pgfpathlineto{\pgfqpoint{2.240523in}{4.711456in}}%
\pgfpathlineto{\pgfqpoint{2.240523in}{4.627952in}}%
\pgfpathlineto{\pgfqpoint{2.279168in}{4.627952in}}%
\pgfpathlineto{\pgfqpoint{2.279168in}{4.600118in}}%
\pgfpathlineto{\pgfqpoint{2.317812in}{4.600118in}}%
\pgfpathlineto{\pgfqpoint{2.317812in}{4.537490in}}%
\pgfpathlineto{\pgfqpoint{2.356456in}{4.537490in}}%
\pgfpathlineto{\pgfqpoint{2.356456in}{4.481821in}}%
\pgfpathlineto{\pgfqpoint{2.395101in}{4.481821in}}%
\pgfpathlineto{\pgfqpoint{2.395101in}{4.453986in}}%
\pgfpathlineto{\pgfqpoint{2.433745in}{4.453986in}}%
\pgfpathlineto{\pgfqpoint{2.433745in}{4.453986in}}%
\pgfpathlineto{\pgfqpoint{2.472389in}{4.453986in}}%
\pgfpathlineto{\pgfqpoint{2.472389in}{4.467904in}}%
\pgfpathlineto{\pgfqpoint{2.511034in}{4.467904in}}%
\pgfpathlineto{\pgfqpoint{2.511034in}{4.467904in}}%
\pgfpathlineto{\pgfqpoint{2.549678in}{4.467904in}}%
\pgfpathlineto{\pgfqpoint{2.549678in}{4.453986in}}%
\pgfpathlineto{\pgfqpoint{2.588322in}{4.453986in}}%
\pgfpathlineto{\pgfqpoint{2.588322in}{4.453986in}}%
\pgfpathlineto{\pgfqpoint{2.626967in}{4.453986in}}%
\pgfpathlineto{\pgfqpoint{2.626967in}{4.453986in}}%
\pgfpathlineto{\pgfqpoint{2.665611in}{4.453986in}}%
\pgfpathlineto{\pgfqpoint{2.665611in}{4.447028in}}%
\pgfpathlineto{\pgfqpoint{2.704256in}{4.447028in}}%
\pgfpathlineto{\pgfqpoint{2.704256in}{4.453986in}}%
\pgfpathlineto{\pgfqpoint{2.742900in}{4.453986in}}%
\pgfpathlineto{\pgfqpoint{2.742900in}{4.447028in}}%
\pgfusepath{stroke}%
\end{pgfscope}%
\begin{pgfscope}%
\pgfpathrectangle{\pgfqpoint{0.754048in}{4.447028in}}{\pgfqpoint{2.301066in}{1.417472in}}%
\pgfusepath{clip}%
\pgfsetbuttcap%
\pgfsetmiterjoin%
\pgfsetlinewidth{1.003750pt}%
\definecolor{currentstroke}{rgb}{0.949020,0.372549,0.360784}%
\pgfsetstrokecolor{currentstroke}%
\pgfsetdash{{1.000000pt}{1.650000pt}}{0.000000pt}%
\pgfpathmoveto{\pgfqpoint{1.274414in}{4.447028in}}%
\pgfpathlineto{\pgfqpoint{1.274414in}{4.454022in}}%
\pgfpathlineto{\pgfqpoint{1.313058in}{4.454022in}}%
\pgfpathlineto{\pgfqpoint{1.313058in}{4.475004in}}%
\pgfpathlineto{\pgfqpoint{1.351703in}{4.475004in}}%
\pgfpathlineto{\pgfqpoint{1.351703in}{4.488992in}}%
\pgfpathlineto{\pgfqpoint{1.390347in}{4.488992in}}%
\pgfpathlineto{\pgfqpoint{1.390347in}{4.544945in}}%
\pgfpathlineto{\pgfqpoint{1.428991in}{4.544945in}}%
\pgfpathlineto{\pgfqpoint{1.428991in}{4.537951in}}%
\pgfpathlineto{\pgfqpoint{1.467636in}{4.537951in}}%
\pgfpathlineto{\pgfqpoint{1.467636in}{4.628875in}}%
\pgfpathlineto{\pgfqpoint{1.506280in}{4.628875in}}%
\pgfpathlineto{\pgfqpoint{1.506280in}{4.712804in}}%
\pgfpathlineto{\pgfqpoint{1.544925in}{4.712804in}}%
\pgfpathlineto{\pgfqpoint{1.544925in}{4.628875in}}%
\pgfpathlineto{\pgfqpoint{1.583569in}{4.628875in}}%
\pgfpathlineto{\pgfqpoint{1.583569in}{4.761763in}}%
\pgfpathlineto{\pgfqpoint{1.622213in}{4.761763in}}%
\pgfpathlineto{\pgfqpoint{1.622213in}{4.817716in}}%
\pgfpathlineto{\pgfqpoint{1.660858in}{4.817716in}}%
\pgfpathlineto{\pgfqpoint{1.660858in}{4.859680in}}%
\pgfpathlineto{\pgfqpoint{1.699502in}{4.859680in}}%
\pgfpathlineto{\pgfqpoint{1.699502in}{5.062510in}}%
\pgfpathlineto{\pgfqpoint{1.738146in}{5.062510in}}%
\pgfpathlineto{\pgfqpoint{1.738146in}{5.195398in}}%
\pgfpathlineto{\pgfqpoint{1.776791in}{5.195398in}}%
\pgfpathlineto{\pgfqpoint{1.776791in}{5.328286in}}%
\pgfpathlineto{\pgfqpoint{1.815435in}{5.328286in}}%
\pgfpathlineto{\pgfqpoint{1.815435in}{5.279327in}}%
\pgfpathlineto{\pgfqpoint{1.854080in}{5.279327in}}%
\pgfpathlineto{\pgfqpoint{1.854080in}{5.482156in}}%
\pgfpathlineto{\pgfqpoint{1.892724in}{5.482156in}}%
\pgfpathlineto{\pgfqpoint{1.892724in}{5.342274in}}%
\pgfpathlineto{\pgfqpoint{1.931368in}{5.342274in}}%
\pgfpathlineto{\pgfqpoint{1.931368in}{5.440192in}}%
\pgfpathlineto{\pgfqpoint{1.970013in}{5.440192in}}%
\pgfpathlineto{\pgfqpoint{1.970013in}{5.356262in}}%
\pgfpathlineto{\pgfqpoint{2.008657in}{5.356262in}}%
\pgfpathlineto{\pgfqpoint{2.008657in}{5.202392in}}%
\pgfpathlineto{\pgfqpoint{2.047301in}{5.202392in}}%
\pgfpathlineto{\pgfqpoint{2.047301in}{5.195398in}}%
\pgfpathlineto{\pgfqpoint{2.085946in}{5.195398in}}%
\pgfpathlineto{\pgfqpoint{2.085946in}{5.027539in}}%
\pgfpathlineto{\pgfqpoint{2.124590in}{5.027539in}}%
\pgfpathlineto{\pgfqpoint{2.124590in}{4.929621in}}%
\pgfpathlineto{\pgfqpoint{2.163234in}{4.929621in}}%
\pgfpathlineto{\pgfqpoint{2.163234in}{4.887657in}}%
\pgfpathlineto{\pgfqpoint{2.201879in}{4.887657in}}%
\pgfpathlineto{\pgfqpoint{2.201879in}{4.817716in}}%
\pgfpathlineto{\pgfqpoint{2.240523in}{4.817716in}}%
\pgfpathlineto{\pgfqpoint{2.240523in}{4.789739in}}%
\pgfpathlineto{\pgfqpoint{2.279168in}{4.789739in}}%
\pgfpathlineto{\pgfqpoint{2.279168in}{4.656851in}}%
\pgfpathlineto{\pgfqpoint{2.317812in}{4.656851in}}%
\pgfpathlineto{\pgfqpoint{2.317812in}{4.719798in}}%
\pgfpathlineto{\pgfqpoint{2.356456in}{4.719798in}}%
\pgfpathlineto{\pgfqpoint{2.356456in}{4.579916in}}%
\pgfpathlineto{\pgfqpoint{2.395101in}{4.579916in}}%
\pgfpathlineto{\pgfqpoint{2.395101in}{4.551939in}}%
\pgfpathlineto{\pgfqpoint{2.433745in}{4.551939in}}%
\pgfpathlineto{\pgfqpoint{2.433745in}{4.565928in}}%
\pgfpathlineto{\pgfqpoint{2.472389in}{4.565928in}}%
\pgfpathlineto{\pgfqpoint{2.472389in}{4.530957in}}%
\pgfpathlineto{\pgfqpoint{2.511034in}{4.530957in}}%
\pgfpathlineto{\pgfqpoint{2.511034in}{4.516969in}}%
\pgfpathlineto{\pgfqpoint{2.549678in}{4.516969in}}%
\pgfpathlineto{\pgfqpoint{2.549678in}{4.475004in}}%
\pgfpathlineto{\pgfqpoint{2.588322in}{4.475004in}}%
\pgfpathlineto{\pgfqpoint{2.588322in}{4.468010in}}%
\pgfpathlineto{\pgfqpoint{2.626967in}{4.468010in}}%
\pgfpathlineto{\pgfqpoint{2.626967in}{4.454022in}}%
\pgfpathlineto{\pgfqpoint{2.665611in}{4.454022in}}%
\pgfpathlineto{\pgfqpoint{2.665611in}{4.475004in}}%
\pgfpathlineto{\pgfqpoint{2.704256in}{4.475004in}}%
\pgfpathlineto{\pgfqpoint{2.704256in}{4.461016in}}%
\pgfpathlineto{\pgfqpoint{2.742900in}{4.461016in}}%
\pgfpathlineto{\pgfqpoint{2.742900in}{4.447028in}}%
\pgfusepath{stroke}%
\end{pgfscope}%
\begin{pgfscope}%
\pgfsetrectcap%
\pgfsetmiterjoin%
\pgfsetlinewidth{0.803000pt}%
\definecolor{currentstroke}{rgb}{0.000000,0.000000,0.000000}%
\pgfsetstrokecolor{currentstroke}%
\pgfsetdash{}{0pt}%
\pgfpathmoveto{\pgfqpoint{0.754048in}{4.447028in}}%
\pgfpathlineto{\pgfqpoint{0.754048in}{5.864500in}}%
\pgfusepath{stroke}%
\end{pgfscope}%
\begin{pgfscope}%
\pgfsetrectcap%
\pgfsetmiterjoin%
\pgfsetlinewidth{0.803000pt}%
\definecolor{currentstroke}{rgb}{0.000000,0.000000,0.000000}%
\pgfsetstrokecolor{currentstroke}%
\pgfsetdash{}{0pt}%
\pgfpathmoveto{\pgfqpoint{3.055114in}{4.447028in}}%
\pgfpathlineto{\pgfqpoint{3.055114in}{5.864500in}}%
\pgfusepath{stroke}%
\end{pgfscope}%
\begin{pgfscope}%
\pgfsetrectcap%
\pgfsetmiterjoin%
\pgfsetlinewidth{0.803000pt}%
\definecolor{currentstroke}{rgb}{0.000000,0.000000,0.000000}%
\pgfsetstrokecolor{currentstroke}%
\pgfsetdash{}{0pt}%
\pgfpathmoveto{\pgfqpoint{0.754048in}{4.447028in}}%
\pgfpathlineto{\pgfqpoint{3.055114in}{4.447028in}}%
\pgfusepath{stroke}%
\end{pgfscope}%
\begin{pgfscope}%
\pgfsetrectcap%
\pgfsetmiterjoin%
\pgfsetlinewidth{0.803000pt}%
\definecolor{currentstroke}{rgb}{0.000000,0.000000,0.000000}%
\pgfsetstrokecolor{currentstroke}%
\pgfsetdash{}{0pt}%
\pgfpathmoveto{\pgfqpoint{0.754048in}{5.864500in}}%
\pgfpathlineto{\pgfqpoint{3.055114in}{5.864500in}}%
\pgfusepath{stroke}%
\end{pgfscope}%
\begin{pgfscope}%
\definecolor{textcolor}{rgb}{0.000000,0.000000,0.000000}%
\pgfsetstrokecolor{textcolor}%
\pgfsetfillcolor{textcolor}%
\pgftext[x=0.754048in,y=5.947833in,left,base]{\color{textcolor}\rmfamily\fontsize{10.000000}{12.000000}\selectfont Bin [0.0, 0.17), 1,971 events}%
\end{pgfscope}%
\begin{pgfscope}%
\pgfsetbuttcap%
\pgfsetmiterjoin%
\definecolor{currentfill}{rgb}{1.000000,1.000000,1.000000}%
\pgfsetfillcolor{currentfill}%
\pgfsetfillopacity{0.800000}%
\pgfsetlinewidth{1.003750pt}%
\definecolor{currentstroke}{rgb}{0.800000,0.800000,0.800000}%
\pgfsetstrokecolor{currentstroke}%
\pgfsetstrokeopacity{0.800000}%
\pgfsetdash{}{0pt}%
\pgfpathmoveto{\pgfqpoint{2.020337in}{5.464944in}}%
\pgfpathlineto{\pgfqpoint{2.977337in}{5.464944in}}%
\pgfpathquadraticcurveto{\pgfqpoint{2.999559in}{5.464944in}}{\pgfqpoint{2.999559in}{5.487167in}}%
\pgfpathlineto{\pgfqpoint{2.999559in}{5.786722in}}%
\pgfpathquadraticcurveto{\pgfqpoint{2.999559in}{5.808944in}}{\pgfqpoint{2.977337in}{5.808944in}}%
\pgfpathlineto{\pgfqpoint{2.020337in}{5.808944in}}%
\pgfpathquadraticcurveto{\pgfqpoint{1.998114in}{5.808944in}}{\pgfqpoint{1.998114in}{5.786722in}}%
\pgfpathlineto{\pgfqpoint{1.998114in}{5.487167in}}%
\pgfpathquadraticcurveto{\pgfqpoint{1.998114in}{5.464944in}}{\pgfqpoint{2.020337in}{5.464944in}}%
\pgfpathclose%
\pgfusepath{stroke,fill}%
\end{pgfscope}%
\begin{pgfscope}%
\pgfsetbuttcap%
\pgfsetmiterjoin%
\pgfsetlinewidth{1.003750pt}%
\definecolor{currentstroke}{rgb}{0.313725,0.317647,0.309804}%
\pgfsetstrokecolor{currentstroke}%
\pgfsetdash{}{0pt}%
\pgfpathmoveto{\pgfqpoint{2.042559in}{5.686278in}}%
\pgfpathlineto{\pgfqpoint{2.264781in}{5.686278in}}%
\pgfpathlineto{\pgfqpoint{2.264781in}{5.764056in}}%
\pgfpathlineto{\pgfqpoint{2.042559in}{5.764056in}}%
\pgfpathclose%
\pgfusepath{stroke}%
\end{pgfscope}%
\begin{pgfscope}%
\definecolor{textcolor}{rgb}{0.000000,0.000000,0.000000}%
\pgfsetstrokecolor{textcolor}%
\pgfsetfillcolor{textcolor}%
\pgftext[x=2.353670in,y=5.686278in,left,base]{\color{textcolor}\rmfamily\fontsize{8.000000}{9.600000}\selectfont IQR = 23.6}%
\end{pgfscope}%
\begin{pgfscope}%
\pgfsetbuttcap%
\pgfsetmiterjoin%
\pgfsetlinewidth{1.003750pt}%
\definecolor{currentstroke}{rgb}{0.949020,0.372549,0.360784}%
\pgfsetstrokecolor{currentstroke}%
\pgfsetdash{{1.000000pt}{1.650000pt}}{0.000000pt}%
\pgfpathmoveto{\pgfqpoint{2.042559in}{5.530944in}}%
\pgfpathlineto{\pgfqpoint{2.264781in}{5.530944in}}%
\pgfpathlineto{\pgfqpoint{2.264781in}{5.608722in}}%
\pgfpathlineto{\pgfqpoint{2.042559in}{5.608722in}}%
\pgfpathclose%
\pgfusepath{stroke}%
\end{pgfscope}%
\begin{pgfscope}%
\definecolor{textcolor}{rgb}{0.000000,0.000000,0.000000}%
\pgfsetstrokecolor{textcolor}%
\pgfsetfillcolor{textcolor}%
\pgftext[x=2.353670in,y=5.530944in,left,base]{\color{textcolor}\rmfamily\fontsize{8.000000}{9.600000}\selectfont IQR = 29.2}%
\end{pgfscope}%
\begin{pgfscope}%
\pgfsetbuttcap%
\pgfsetmiterjoin%
\definecolor{currentfill}{rgb}{1.000000,1.000000,1.000000}%
\pgfsetfillcolor{currentfill}%
\pgfsetlinewidth{0.000000pt}%
\definecolor{currentstroke}{rgb}{0.000000,0.000000,0.000000}%
\pgfsetstrokecolor{currentstroke}%
\pgfsetstrokeopacity{0.000000}%
\pgfsetdash{}{0pt}%
\pgfpathmoveto{\pgfqpoint{3.750134in}{4.447028in}}%
\pgfpathlineto{\pgfqpoint{6.051200in}{4.447028in}}%
\pgfpathlineto{\pgfqpoint{6.051200in}{5.864500in}}%
\pgfpathlineto{\pgfqpoint{3.750134in}{5.864500in}}%
\pgfpathclose%
\pgfusepath{fill}%
\end{pgfscope}%
\begin{pgfscope}%
\pgfsetbuttcap%
\pgfsetroundjoin%
\definecolor{currentfill}{rgb}{0.000000,0.000000,0.000000}%
\pgfsetfillcolor{currentfill}%
\pgfsetlinewidth{0.803000pt}%
\definecolor{currentstroke}{rgb}{0.000000,0.000000,0.000000}%
\pgfsetstrokecolor{currentstroke}%
\pgfsetdash{}{0pt}%
\pgfsys@defobject{currentmarker}{\pgfqpoint{0.000000in}{-0.048611in}}{\pgfqpoint{0.000000in}{0.000000in}}{%
\pgfpathmoveto{\pgfqpoint{0.000000in}{0.000000in}}%
\pgfpathlineto{\pgfqpoint{0.000000in}{-0.048611in}}%
\pgfusepath{stroke,fill}%
}%
\begin{pgfscope}%
\pgfsys@transformshift{3.982265in}{4.447028in}%
\pgfsys@useobject{currentmarker}{}%
\end{pgfscope}%
\end{pgfscope}%
\begin{pgfscope}%
\definecolor{textcolor}{rgb}{0.000000,0.000000,0.000000}%
\pgfsetstrokecolor{textcolor}%
\pgfsetfillcolor{textcolor}%
\pgftext[x=3.982265in,y=4.349806in,,top]{\color{textcolor}\rmfamily\fontsize{8.000000}{9.600000}\selectfont \(\displaystyle {-120}\)}%
\end{pgfscope}%
\begin{pgfscope}%
\pgfsetbuttcap%
\pgfsetroundjoin%
\definecolor{currentfill}{rgb}{0.000000,0.000000,0.000000}%
\pgfsetfillcolor{currentfill}%
\pgfsetlinewidth{0.803000pt}%
\definecolor{currentstroke}{rgb}{0.000000,0.000000,0.000000}%
\pgfsetstrokecolor{currentstroke}%
\pgfsetdash{}{0pt}%
\pgfsys@defobject{currentmarker}{\pgfqpoint{0.000000in}{-0.048611in}}{\pgfqpoint{0.000000in}{0.000000in}}{%
\pgfpathmoveto{\pgfqpoint{0.000000in}{0.000000in}}%
\pgfpathlineto{\pgfqpoint{0.000000in}{-0.048611in}}%
\pgfusepath{stroke,fill}%
}%
\begin{pgfscope}%
\pgfsys@transformshift{4.287966in}{4.447028in}%
\pgfsys@useobject{currentmarker}{}%
\end{pgfscope}%
\end{pgfscope}%
\begin{pgfscope}%
\definecolor{textcolor}{rgb}{0.000000,0.000000,0.000000}%
\pgfsetstrokecolor{textcolor}%
\pgfsetfillcolor{textcolor}%
\pgftext[x=4.287966in,y=4.349806in,,top]{\color{textcolor}\rmfamily\fontsize{8.000000}{9.600000}\selectfont \(\displaystyle {-80}\)}%
\end{pgfscope}%
\begin{pgfscope}%
\pgfsetbuttcap%
\pgfsetroundjoin%
\definecolor{currentfill}{rgb}{0.000000,0.000000,0.000000}%
\pgfsetfillcolor{currentfill}%
\pgfsetlinewidth{0.803000pt}%
\definecolor{currentstroke}{rgb}{0.000000,0.000000,0.000000}%
\pgfsetstrokecolor{currentstroke}%
\pgfsetdash{}{0pt}%
\pgfsys@defobject{currentmarker}{\pgfqpoint{0.000000in}{-0.048611in}}{\pgfqpoint{0.000000in}{0.000000in}}{%
\pgfpathmoveto{\pgfqpoint{0.000000in}{0.000000in}}%
\pgfpathlineto{\pgfqpoint{0.000000in}{-0.048611in}}%
\pgfusepath{stroke,fill}%
}%
\begin{pgfscope}%
\pgfsys@transformshift{4.593668in}{4.447028in}%
\pgfsys@useobject{currentmarker}{}%
\end{pgfscope}%
\end{pgfscope}%
\begin{pgfscope}%
\definecolor{textcolor}{rgb}{0.000000,0.000000,0.000000}%
\pgfsetstrokecolor{textcolor}%
\pgfsetfillcolor{textcolor}%
\pgftext[x=4.593668in,y=4.349806in,,top]{\color{textcolor}\rmfamily\fontsize{8.000000}{9.600000}\selectfont \(\displaystyle {-40}\)}%
\end{pgfscope}%
\begin{pgfscope}%
\pgfsetbuttcap%
\pgfsetroundjoin%
\definecolor{currentfill}{rgb}{0.000000,0.000000,0.000000}%
\pgfsetfillcolor{currentfill}%
\pgfsetlinewidth{0.803000pt}%
\definecolor{currentstroke}{rgb}{0.000000,0.000000,0.000000}%
\pgfsetstrokecolor{currentstroke}%
\pgfsetdash{}{0pt}%
\pgfsys@defobject{currentmarker}{\pgfqpoint{0.000000in}{-0.048611in}}{\pgfqpoint{0.000000in}{0.000000in}}{%
\pgfpathmoveto{\pgfqpoint{0.000000in}{0.000000in}}%
\pgfpathlineto{\pgfqpoint{0.000000in}{-0.048611in}}%
\pgfusepath{stroke,fill}%
}%
\begin{pgfscope}%
\pgfsys@transformshift{4.899369in}{4.447028in}%
\pgfsys@useobject{currentmarker}{}%
\end{pgfscope}%
\end{pgfscope}%
\begin{pgfscope}%
\definecolor{textcolor}{rgb}{0.000000,0.000000,0.000000}%
\pgfsetstrokecolor{textcolor}%
\pgfsetfillcolor{textcolor}%
\pgftext[x=4.899369in,y=4.349806in,,top]{\color{textcolor}\rmfamily\fontsize{8.000000}{9.600000}\selectfont \(\displaystyle {0}\)}%
\end{pgfscope}%
\begin{pgfscope}%
\pgfsetbuttcap%
\pgfsetroundjoin%
\definecolor{currentfill}{rgb}{0.000000,0.000000,0.000000}%
\pgfsetfillcolor{currentfill}%
\pgfsetlinewidth{0.803000pt}%
\definecolor{currentstroke}{rgb}{0.000000,0.000000,0.000000}%
\pgfsetstrokecolor{currentstroke}%
\pgfsetdash{}{0pt}%
\pgfsys@defobject{currentmarker}{\pgfqpoint{0.000000in}{-0.048611in}}{\pgfqpoint{0.000000in}{0.000000in}}{%
\pgfpathmoveto{\pgfqpoint{0.000000in}{0.000000in}}%
\pgfpathlineto{\pgfqpoint{0.000000in}{-0.048611in}}%
\pgfusepath{stroke,fill}%
}%
\begin{pgfscope}%
\pgfsys@transformshift{5.205070in}{4.447028in}%
\pgfsys@useobject{currentmarker}{}%
\end{pgfscope}%
\end{pgfscope}%
\begin{pgfscope}%
\definecolor{textcolor}{rgb}{0.000000,0.000000,0.000000}%
\pgfsetstrokecolor{textcolor}%
\pgfsetfillcolor{textcolor}%
\pgftext[x=5.205070in,y=4.349806in,,top]{\color{textcolor}\rmfamily\fontsize{8.000000}{9.600000}\selectfont \(\displaystyle {40}\)}%
\end{pgfscope}%
\begin{pgfscope}%
\pgfsetbuttcap%
\pgfsetroundjoin%
\definecolor{currentfill}{rgb}{0.000000,0.000000,0.000000}%
\pgfsetfillcolor{currentfill}%
\pgfsetlinewidth{0.803000pt}%
\definecolor{currentstroke}{rgb}{0.000000,0.000000,0.000000}%
\pgfsetstrokecolor{currentstroke}%
\pgfsetdash{}{0pt}%
\pgfsys@defobject{currentmarker}{\pgfqpoint{0.000000in}{-0.048611in}}{\pgfqpoint{0.000000in}{0.000000in}}{%
\pgfpathmoveto{\pgfqpoint{0.000000in}{0.000000in}}%
\pgfpathlineto{\pgfqpoint{0.000000in}{-0.048611in}}%
\pgfusepath{stroke,fill}%
}%
\begin{pgfscope}%
\pgfsys@transformshift{5.510772in}{4.447028in}%
\pgfsys@useobject{currentmarker}{}%
\end{pgfscope}%
\end{pgfscope}%
\begin{pgfscope}%
\definecolor{textcolor}{rgb}{0.000000,0.000000,0.000000}%
\pgfsetstrokecolor{textcolor}%
\pgfsetfillcolor{textcolor}%
\pgftext[x=5.510772in,y=4.349806in,,top]{\color{textcolor}\rmfamily\fontsize{8.000000}{9.600000}\selectfont \(\displaystyle {80}\)}%
\end{pgfscope}%
\begin{pgfscope}%
\pgfsetbuttcap%
\pgfsetroundjoin%
\definecolor{currentfill}{rgb}{0.000000,0.000000,0.000000}%
\pgfsetfillcolor{currentfill}%
\pgfsetlinewidth{0.803000pt}%
\definecolor{currentstroke}{rgb}{0.000000,0.000000,0.000000}%
\pgfsetstrokecolor{currentstroke}%
\pgfsetdash{}{0pt}%
\pgfsys@defobject{currentmarker}{\pgfqpoint{0.000000in}{-0.048611in}}{\pgfqpoint{0.000000in}{0.000000in}}{%
\pgfpathmoveto{\pgfqpoint{0.000000in}{0.000000in}}%
\pgfpathlineto{\pgfqpoint{0.000000in}{-0.048611in}}%
\pgfusepath{stroke,fill}%
}%
\begin{pgfscope}%
\pgfsys@transformshift{5.816473in}{4.447028in}%
\pgfsys@useobject{currentmarker}{}%
\end{pgfscope}%
\end{pgfscope}%
\begin{pgfscope}%
\definecolor{textcolor}{rgb}{0.000000,0.000000,0.000000}%
\pgfsetstrokecolor{textcolor}%
\pgfsetfillcolor{textcolor}%
\pgftext[x=5.816473in,y=4.349806in,,top]{\color{textcolor}\rmfamily\fontsize{8.000000}{9.600000}\selectfont \(\displaystyle {120}\)}%
\end{pgfscope}%
\begin{pgfscope}%
\pgfsetbuttcap%
\pgfsetroundjoin%
\definecolor{currentfill}{rgb}{0.000000,0.000000,0.000000}%
\pgfsetfillcolor{currentfill}%
\pgfsetlinewidth{0.803000pt}%
\definecolor{currentstroke}{rgb}{0.000000,0.000000,0.000000}%
\pgfsetstrokecolor{currentstroke}%
\pgfsetdash{}{0pt}%
\pgfsys@defobject{currentmarker}{\pgfqpoint{-0.048611in}{0.000000in}}{\pgfqpoint{-0.000000in}{0.000000in}}{%
\pgfpathmoveto{\pgfqpoint{-0.000000in}{0.000000in}}%
\pgfpathlineto{\pgfqpoint{-0.048611in}{0.000000in}}%
\pgfusepath{stroke,fill}%
}%
\begin{pgfscope}%
\pgfsys@transformshift{3.750134in}{4.447028in}%
\pgfsys@useobject{currentmarker}{}%
\end{pgfscope}%
\end{pgfscope}%
\begin{pgfscope}%
\definecolor{textcolor}{rgb}{0.000000,0.000000,0.000000}%
\pgfsetstrokecolor{textcolor}%
\pgfsetfillcolor{textcolor}%
\pgftext[x=3.384003in, y=4.408472in, left, base]{\color{textcolor}\rmfamily\fontsize{8.000000}{9.600000}\selectfont \(\displaystyle {0.000}\)}%
\end{pgfscope}%
\begin{pgfscope}%
\pgfsetbuttcap%
\pgfsetroundjoin%
\definecolor{currentfill}{rgb}{0.000000,0.000000,0.000000}%
\pgfsetfillcolor{currentfill}%
\pgfsetlinewidth{0.803000pt}%
\definecolor{currentstroke}{rgb}{0.000000,0.000000,0.000000}%
\pgfsetstrokecolor{currentstroke}%
\pgfsetdash{}{0pt}%
\pgfsys@defobject{currentmarker}{\pgfqpoint{-0.048611in}{0.000000in}}{\pgfqpoint{-0.000000in}{0.000000in}}{%
\pgfpathmoveto{\pgfqpoint{-0.000000in}{0.000000in}}%
\pgfpathlineto{\pgfqpoint{-0.048611in}{0.000000in}}%
\pgfusepath{stroke,fill}%
}%
\begin{pgfscope}%
\pgfsys@transformshift{3.750134in}{4.592693in}%
\pgfsys@useobject{currentmarker}{}%
\end{pgfscope}%
\end{pgfscope}%
\begin{pgfscope}%
\definecolor{textcolor}{rgb}{0.000000,0.000000,0.000000}%
\pgfsetstrokecolor{textcolor}%
\pgfsetfillcolor{textcolor}%
\pgftext[x=3.384003in, y=4.554138in, left, base]{\color{textcolor}\rmfamily\fontsize{8.000000}{9.600000}\selectfont \(\displaystyle {0.002}\)}%
\end{pgfscope}%
\begin{pgfscope}%
\pgfsetbuttcap%
\pgfsetroundjoin%
\definecolor{currentfill}{rgb}{0.000000,0.000000,0.000000}%
\pgfsetfillcolor{currentfill}%
\pgfsetlinewidth{0.803000pt}%
\definecolor{currentstroke}{rgb}{0.000000,0.000000,0.000000}%
\pgfsetstrokecolor{currentstroke}%
\pgfsetdash{}{0pt}%
\pgfsys@defobject{currentmarker}{\pgfqpoint{-0.048611in}{0.000000in}}{\pgfqpoint{-0.000000in}{0.000000in}}{%
\pgfpathmoveto{\pgfqpoint{-0.000000in}{0.000000in}}%
\pgfpathlineto{\pgfqpoint{-0.048611in}{0.000000in}}%
\pgfusepath{stroke,fill}%
}%
\begin{pgfscope}%
\pgfsys@transformshift{3.750134in}{4.738359in}%
\pgfsys@useobject{currentmarker}{}%
\end{pgfscope}%
\end{pgfscope}%
\begin{pgfscope}%
\definecolor{textcolor}{rgb}{0.000000,0.000000,0.000000}%
\pgfsetstrokecolor{textcolor}%
\pgfsetfillcolor{textcolor}%
\pgftext[x=3.384003in, y=4.699804in, left, base]{\color{textcolor}\rmfamily\fontsize{8.000000}{9.600000}\selectfont \(\displaystyle {0.004}\)}%
\end{pgfscope}%
\begin{pgfscope}%
\pgfsetbuttcap%
\pgfsetroundjoin%
\definecolor{currentfill}{rgb}{0.000000,0.000000,0.000000}%
\pgfsetfillcolor{currentfill}%
\pgfsetlinewidth{0.803000pt}%
\definecolor{currentstroke}{rgb}{0.000000,0.000000,0.000000}%
\pgfsetstrokecolor{currentstroke}%
\pgfsetdash{}{0pt}%
\pgfsys@defobject{currentmarker}{\pgfqpoint{-0.048611in}{0.000000in}}{\pgfqpoint{-0.000000in}{0.000000in}}{%
\pgfpathmoveto{\pgfqpoint{-0.000000in}{0.000000in}}%
\pgfpathlineto{\pgfqpoint{-0.048611in}{0.000000in}}%
\pgfusepath{stroke,fill}%
}%
\begin{pgfscope}%
\pgfsys@transformshift{3.750134in}{4.884025in}%
\pgfsys@useobject{currentmarker}{}%
\end{pgfscope}%
\end{pgfscope}%
\begin{pgfscope}%
\definecolor{textcolor}{rgb}{0.000000,0.000000,0.000000}%
\pgfsetstrokecolor{textcolor}%
\pgfsetfillcolor{textcolor}%
\pgftext[x=3.384003in, y=4.845469in, left, base]{\color{textcolor}\rmfamily\fontsize{8.000000}{9.600000}\selectfont \(\displaystyle {0.006}\)}%
\end{pgfscope}%
\begin{pgfscope}%
\pgfsetbuttcap%
\pgfsetroundjoin%
\definecolor{currentfill}{rgb}{0.000000,0.000000,0.000000}%
\pgfsetfillcolor{currentfill}%
\pgfsetlinewidth{0.803000pt}%
\definecolor{currentstroke}{rgb}{0.000000,0.000000,0.000000}%
\pgfsetstrokecolor{currentstroke}%
\pgfsetdash{}{0pt}%
\pgfsys@defobject{currentmarker}{\pgfqpoint{-0.048611in}{0.000000in}}{\pgfqpoint{-0.000000in}{0.000000in}}{%
\pgfpathmoveto{\pgfqpoint{-0.000000in}{0.000000in}}%
\pgfpathlineto{\pgfqpoint{-0.048611in}{0.000000in}}%
\pgfusepath{stroke,fill}%
}%
\begin{pgfscope}%
\pgfsys@transformshift{3.750134in}{5.029691in}%
\pgfsys@useobject{currentmarker}{}%
\end{pgfscope}%
\end{pgfscope}%
\begin{pgfscope}%
\definecolor{textcolor}{rgb}{0.000000,0.000000,0.000000}%
\pgfsetstrokecolor{textcolor}%
\pgfsetfillcolor{textcolor}%
\pgftext[x=3.384003in, y=4.991135in, left, base]{\color{textcolor}\rmfamily\fontsize{8.000000}{9.600000}\selectfont \(\displaystyle {0.008}\)}%
\end{pgfscope}%
\begin{pgfscope}%
\pgfsetbuttcap%
\pgfsetroundjoin%
\definecolor{currentfill}{rgb}{0.000000,0.000000,0.000000}%
\pgfsetfillcolor{currentfill}%
\pgfsetlinewidth{0.803000pt}%
\definecolor{currentstroke}{rgb}{0.000000,0.000000,0.000000}%
\pgfsetstrokecolor{currentstroke}%
\pgfsetdash{}{0pt}%
\pgfsys@defobject{currentmarker}{\pgfqpoint{-0.048611in}{0.000000in}}{\pgfqpoint{-0.000000in}{0.000000in}}{%
\pgfpathmoveto{\pgfqpoint{-0.000000in}{0.000000in}}%
\pgfpathlineto{\pgfqpoint{-0.048611in}{0.000000in}}%
\pgfusepath{stroke,fill}%
}%
\begin{pgfscope}%
\pgfsys@transformshift{3.750134in}{5.175357in}%
\pgfsys@useobject{currentmarker}{}%
\end{pgfscope}%
\end{pgfscope}%
\begin{pgfscope}%
\definecolor{textcolor}{rgb}{0.000000,0.000000,0.000000}%
\pgfsetstrokecolor{textcolor}%
\pgfsetfillcolor{textcolor}%
\pgftext[x=3.384003in, y=5.136801in, left, base]{\color{textcolor}\rmfamily\fontsize{8.000000}{9.600000}\selectfont \(\displaystyle {0.010}\)}%
\end{pgfscope}%
\begin{pgfscope}%
\pgfsetbuttcap%
\pgfsetroundjoin%
\definecolor{currentfill}{rgb}{0.000000,0.000000,0.000000}%
\pgfsetfillcolor{currentfill}%
\pgfsetlinewidth{0.803000pt}%
\definecolor{currentstroke}{rgb}{0.000000,0.000000,0.000000}%
\pgfsetstrokecolor{currentstroke}%
\pgfsetdash{}{0pt}%
\pgfsys@defobject{currentmarker}{\pgfqpoint{-0.048611in}{0.000000in}}{\pgfqpoint{-0.000000in}{0.000000in}}{%
\pgfpathmoveto{\pgfqpoint{-0.000000in}{0.000000in}}%
\pgfpathlineto{\pgfqpoint{-0.048611in}{0.000000in}}%
\pgfusepath{stroke,fill}%
}%
\begin{pgfscope}%
\pgfsys@transformshift{3.750134in}{5.321022in}%
\pgfsys@useobject{currentmarker}{}%
\end{pgfscope}%
\end{pgfscope}%
\begin{pgfscope}%
\definecolor{textcolor}{rgb}{0.000000,0.000000,0.000000}%
\pgfsetstrokecolor{textcolor}%
\pgfsetfillcolor{textcolor}%
\pgftext[x=3.384003in, y=5.282467in, left, base]{\color{textcolor}\rmfamily\fontsize{8.000000}{9.600000}\selectfont \(\displaystyle {0.012}\)}%
\end{pgfscope}%
\begin{pgfscope}%
\pgfsetbuttcap%
\pgfsetroundjoin%
\definecolor{currentfill}{rgb}{0.000000,0.000000,0.000000}%
\pgfsetfillcolor{currentfill}%
\pgfsetlinewidth{0.803000pt}%
\definecolor{currentstroke}{rgb}{0.000000,0.000000,0.000000}%
\pgfsetstrokecolor{currentstroke}%
\pgfsetdash{}{0pt}%
\pgfsys@defobject{currentmarker}{\pgfqpoint{-0.048611in}{0.000000in}}{\pgfqpoint{-0.000000in}{0.000000in}}{%
\pgfpathmoveto{\pgfqpoint{-0.000000in}{0.000000in}}%
\pgfpathlineto{\pgfqpoint{-0.048611in}{0.000000in}}%
\pgfusepath{stroke,fill}%
}%
\begin{pgfscope}%
\pgfsys@transformshift{3.750134in}{5.466688in}%
\pgfsys@useobject{currentmarker}{}%
\end{pgfscope}%
\end{pgfscope}%
\begin{pgfscope}%
\definecolor{textcolor}{rgb}{0.000000,0.000000,0.000000}%
\pgfsetstrokecolor{textcolor}%
\pgfsetfillcolor{textcolor}%
\pgftext[x=3.384003in, y=5.428132in, left, base]{\color{textcolor}\rmfamily\fontsize{8.000000}{9.600000}\selectfont \(\displaystyle {0.014}\)}%
\end{pgfscope}%
\begin{pgfscope}%
\pgfsetbuttcap%
\pgfsetroundjoin%
\definecolor{currentfill}{rgb}{0.000000,0.000000,0.000000}%
\pgfsetfillcolor{currentfill}%
\pgfsetlinewidth{0.803000pt}%
\definecolor{currentstroke}{rgb}{0.000000,0.000000,0.000000}%
\pgfsetstrokecolor{currentstroke}%
\pgfsetdash{}{0pt}%
\pgfsys@defobject{currentmarker}{\pgfqpoint{-0.048611in}{0.000000in}}{\pgfqpoint{-0.000000in}{0.000000in}}{%
\pgfpathmoveto{\pgfqpoint{-0.000000in}{0.000000in}}%
\pgfpathlineto{\pgfqpoint{-0.048611in}{0.000000in}}%
\pgfusepath{stroke,fill}%
}%
\begin{pgfscope}%
\pgfsys@transformshift{3.750134in}{5.612354in}%
\pgfsys@useobject{currentmarker}{}%
\end{pgfscope}%
\end{pgfscope}%
\begin{pgfscope}%
\definecolor{textcolor}{rgb}{0.000000,0.000000,0.000000}%
\pgfsetstrokecolor{textcolor}%
\pgfsetfillcolor{textcolor}%
\pgftext[x=3.384003in, y=5.573798in, left, base]{\color{textcolor}\rmfamily\fontsize{8.000000}{9.600000}\selectfont \(\displaystyle {0.016}\)}%
\end{pgfscope}%
\begin{pgfscope}%
\pgfsetbuttcap%
\pgfsetroundjoin%
\definecolor{currentfill}{rgb}{0.000000,0.000000,0.000000}%
\pgfsetfillcolor{currentfill}%
\pgfsetlinewidth{0.803000pt}%
\definecolor{currentstroke}{rgb}{0.000000,0.000000,0.000000}%
\pgfsetstrokecolor{currentstroke}%
\pgfsetdash{}{0pt}%
\pgfsys@defobject{currentmarker}{\pgfqpoint{-0.048611in}{0.000000in}}{\pgfqpoint{-0.000000in}{0.000000in}}{%
\pgfpathmoveto{\pgfqpoint{-0.000000in}{0.000000in}}%
\pgfpathlineto{\pgfqpoint{-0.048611in}{0.000000in}}%
\pgfusepath{stroke,fill}%
}%
\begin{pgfscope}%
\pgfsys@transformshift{3.750134in}{5.758020in}%
\pgfsys@useobject{currentmarker}{}%
\end{pgfscope}%
\end{pgfscope}%
\begin{pgfscope}%
\definecolor{textcolor}{rgb}{0.000000,0.000000,0.000000}%
\pgfsetstrokecolor{textcolor}%
\pgfsetfillcolor{textcolor}%
\pgftext[x=3.384003in, y=5.719464in, left, base]{\color{textcolor}\rmfamily\fontsize{8.000000}{9.600000}\selectfont \(\displaystyle {0.018}\)}%
\end{pgfscope}%
\begin{pgfscope}%
\definecolor{textcolor}{rgb}{0.000000,0.000000,0.000000}%
\pgfsetstrokecolor{textcolor}%
\pgfsetfillcolor{textcolor}%
\pgftext[x=3.328448in,y=5.155764in,,bottom,rotate=90.000000]{\color{textcolor}\rmfamily\fontsize{10.000000}{12.000000}\selectfont Density}%
\end{pgfscope}%
\begin{pgfscope}%
\pgfpathrectangle{\pgfqpoint{3.750134in}{4.447028in}}{\pgfqpoint{2.301066in}{1.417472in}}%
\pgfusepath{clip}%
\pgfsetbuttcap%
\pgfsetmiterjoin%
\pgfsetlinewidth{1.003750pt}%
\definecolor{currentstroke}{rgb}{0.313725,0.317647,0.309804}%
\pgfsetstrokecolor{currentstroke}%
\pgfsetdash{}{0pt}%
\pgfpathmoveto{\pgfqpoint{3.878236in}{4.447028in}}%
\pgfpathlineto{\pgfqpoint{3.878236in}{4.447703in}}%
\pgfpathlineto{\pgfqpoint{4.093691in}{4.448379in}}%
\pgfpathlineto{\pgfqpoint{4.093691in}{4.449055in}}%
\pgfpathlineto{\pgfqpoint{4.115237in}{4.449055in}}%
\pgfpathlineto{\pgfqpoint{4.115237in}{4.447703in}}%
\pgfpathlineto{\pgfqpoint{4.147555in}{4.448379in}}%
\pgfpathlineto{\pgfqpoint{4.147555in}{4.450406in}}%
\pgfpathlineto{\pgfqpoint{4.190646in}{4.451082in}}%
\pgfpathlineto{\pgfqpoint{4.190646in}{4.453109in}}%
\pgfpathlineto{\pgfqpoint{4.233737in}{4.453109in}}%
\pgfpathlineto{\pgfqpoint{4.233737in}{4.458514in}}%
\pgfpathlineto{\pgfqpoint{4.244510in}{4.458514in}}%
\pgfpathlineto{\pgfqpoint{4.244510in}{4.455811in}}%
\pgfpathlineto{\pgfqpoint{4.255283in}{4.455811in}}%
\pgfpathlineto{\pgfqpoint{4.255283in}{4.457838in}}%
\pgfpathlineto{\pgfqpoint{4.276828in}{4.458514in}}%
\pgfpathlineto{\pgfqpoint{4.276828in}{4.461892in}}%
\pgfpathlineto{\pgfqpoint{4.298374in}{4.461217in}}%
\pgfpathlineto{\pgfqpoint{4.298374in}{4.463919in}}%
\pgfpathlineto{\pgfqpoint{4.309147in}{4.463919in}}%
\pgfpathlineto{\pgfqpoint{4.309147in}{4.467973in}}%
\pgfpathlineto{\pgfqpoint{4.341465in}{4.467298in}}%
\pgfpathlineto{\pgfqpoint{4.341465in}{4.470676in}}%
\pgfpathlineto{\pgfqpoint{4.352238in}{4.470676in}}%
\pgfpathlineto{\pgfqpoint{4.352238in}{4.482838in}}%
\pgfpathlineto{\pgfqpoint{4.363010in}{4.482838in}}%
\pgfpathlineto{\pgfqpoint{4.363010in}{4.478784in}}%
\pgfpathlineto{\pgfqpoint{4.373783in}{4.478784in}}%
\pgfpathlineto{\pgfqpoint{4.373783in}{4.476081in}}%
\pgfpathlineto{\pgfqpoint{4.384556in}{4.476081in}}%
\pgfpathlineto{\pgfqpoint{4.384556in}{4.483514in}}%
\pgfpathlineto{\pgfqpoint{4.395329in}{4.483514in}}%
\pgfpathlineto{\pgfqpoint{4.395329in}{4.493648in}}%
\pgfpathlineto{\pgfqpoint{4.406101in}{4.493648in}}%
\pgfpathlineto{\pgfqpoint{4.406101in}{4.497027in}}%
\pgfpathlineto{\pgfqpoint{4.416874in}{4.497027in}}%
\pgfpathlineto{\pgfqpoint{4.416874in}{4.502432in}}%
\pgfpathlineto{\pgfqpoint{4.449193in}{4.502432in}}%
\pgfpathlineto{\pgfqpoint{4.449193in}{4.517972in}}%
\pgfpathlineto{\pgfqpoint{4.459965in}{4.517972in}}%
\pgfpathlineto{\pgfqpoint{4.459965in}{4.515270in}}%
\pgfpathlineto{\pgfqpoint{4.470738in}{4.515270in}}%
\pgfpathlineto{\pgfqpoint{4.470738in}{4.528107in}}%
\pgfpathlineto{\pgfqpoint{4.481511in}{4.528107in}}%
\pgfpathlineto{\pgfqpoint{4.481511in}{4.555809in}}%
\pgfpathlineto{\pgfqpoint{4.492284in}{4.555809in}}%
\pgfpathlineto{\pgfqpoint{4.492284in}{4.547026in}}%
\pgfpathlineto{\pgfqpoint{4.503056in}{4.547026in}}%
\pgfpathlineto{\pgfqpoint{4.503056in}{4.563242in}}%
\pgfpathlineto{\pgfqpoint{4.513829in}{4.563242in}}%
\pgfpathlineto{\pgfqpoint{4.513829in}{4.581485in}}%
\pgfpathlineto{\pgfqpoint{4.524602in}{4.581485in}}%
\pgfpathlineto{\pgfqpoint{4.524602in}{4.584187in}}%
\pgfpathlineto{\pgfqpoint{4.546147in}{4.583512in}}%
\pgfpathlineto{\pgfqpoint{4.546147in}{4.611214in}}%
\pgfpathlineto{\pgfqpoint{4.556920in}{4.611214in}}%
\pgfpathlineto{\pgfqpoint{4.556920in}{4.655807in}}%
\pgfpathlineto{\pgfqpoint{4.567693in}{4.655807in}}%
\pgfpathlineto{\pgfqpoint{4.567693in}{4.642294in}}%
\pgfpathlineto{\pgfqpoint{4.578466in}{4.642294in}}%
\pgfpathlineto{\pgfqpoint{4.578466in}{4.666618in}}%
\pgfpathlineto{\pgfqpoint{4.589238in}{4.666618in}}%
\pgfpathlineto{\pgfqpoint{4.589238in}{4.686888in}}%
\pgfpathlineto{\pgfqpoint{4.600011in}{4.686888in}}%
\pgfpathlineto{\pgfqpoint{4.600011in}{4.715941in}}%
\pgfpathlineto{\pgfqpoint{4.610784in}{4.715941in}}%
\pgfpathlineto{\pgfqpoint{4.610784in}{4.750400in}}%
\pgfpathlineto{\pgfqpoint{4.621557in}{4.750400in}}%
\pgfpathlineto{\pgfqpoint{4.621557in}{4.803778in}}%
\pgfpathlineto{\pgfqpoint{4.632329in}{4.803778in}}%
\pgfpathlineto{\pgfqpoint{4.632329in}{4.815939in}}%
\pgfpathlineto{\pgfqpoint{4.643102in}{4.815939in}}%
\pgfpathlineto{\pgfqpoint{4.643102in}{4.846344in}}%
\pgfpathlineto{\pgfqpoint{4.653875in}{4.846344in}}%
\pgfpathlineto{\pgfqpoint{4.653875in}{4.849723in}}%
\pgfpathlineto{\pgfqpoint{4.664648in}{4.849723in}}%
\pgfpathlineto{\pgfqpoint{4.664648in}{4.931478in}}%
\pgfpathlineto{\pgfqpoint{4.675420in}{4.931478in}}%
\pgfpathlineto{\pgfqpoint{4.675420in}{4.945667in}}%
\pgfpathlineto{\pgfqpoint{4.686193in}{4.945667in}}%
\pgfpathlineto{\pgfqpoint{4.686193in}{4.982828in}}%
\pgfpathlineto{\pgfqpoint{4.696966in}{4.982828in}}%
\pgfpathlineto{\pgfqpoint{4.696966in}{5.027422in}}%
\pgfpathlineto{\pgfqpoint{4.707739in}{5.027422in}}%
\pgfpathlineto{\pgfqpoint{4.707739in}{5.035530in}}%
\pgfpathlineto{\pgfqpoint{4.718512in}{5.035530in}}%
\pgfpathlineto{\pgfqpoint{4.718512in}{5.142960in}}%
\pgfpathlineto{\pgfqpoint{4.729284in}{5.142960in}}%
\pgfpathlineto{\pgfqpoint{4.729284in}{5.125393in}}%
\pgfpathlineto{\pgfqpoint{4.740057in}{5.125393in}}%
\pgfpathlineto{\pgfqpoint{4.740057in}{5.194986in}}%
\pgfpathlineto{\pgfqpoint{4.750830in}{5.194986in}}%
\pgfpathlineto{\pgfqpoint{4.750830in}{5.311876in}}%
\pgfpathlineto{\pgfqpoint{4.761603in}{5.311876in}}%
\pgfpathlineto{\pgfqpoint{4.761603in}{5.346334in}}%
\pgfpathlineto{\pgfqpoint{4.772375in}{5.346334in}}%
\pgfpathlineto{\pgfqpoint{4.772375in}{5.382820in}}%
\pgfpathlineto{\pgfqpoint{4.783148in}{5.382820in}}%
\pgfpathlineto{\pgfqpoint{4.783148in}{5.426738in}}%
\pgfpathlineto{\pgfqpoint{4.793921in}{5.426738in}}%
\pgfpathlineto{\pgfqpoint{4.793921in}{5.538223in}}%
\pgfpathlineto{\pgfqpoint{4.804694in}{5.538223in}}%
\pgfpathlineto{\pgfqpoint{4.804694in}{5.551060in}}%
\pgfpathlineto{\pgfqpoint{4.815466in}{5.551060in}}%
\pgfpathlineto{\pgfqpoint{4.815466in}{5.555114in}}%
\pgfpathlineto{\pgfqpoint{4.826239in}{5.555114in}}%
\pgfpathlineto{\pgfqpoint{4.826239in}{5.626059in}}%
\pgfpathlineto{\pgfqpoint{4.837012in}{5.626059in}}%
\pgfpathlineto{\pgfqpoint{4.837012in}{5.651058in}}%
\pgfpathlineto{\pgfqpoint{4.847785in}{5.651058in}}%
\pgfpathlineto{\pgfqpoint{4.847785in}{5.767948in}}%
\pgfpathlineto{\pgfqpoint{4.880103in}{5.767272in}}%
\pgfpathlineto{\pgfqpoint{4.880103in}{5.797001in}}%
\pgfpathlineto{\pgfqpoint{4.890876in}{5.797001in}}%
\pgfpathlineto{\pgfqpoint{4.890876in}{5.794974in}}%
\pgfpathlineto{\pgfqpoint{4.901648in}{5.794974in}}%
\pgfpathlineto{\pgfqpoint{4.901648in}{5.791596in}}%
\pgfpathlineto{\pgfqpoint{4.912421in}{5.791596in}}%
\pgfpathlineto{\pgfqpoint{4.912421in}{5.757813in}}%
\pgfpathlineto{\pgfqpoint{4.923194in}{5.757813in}}%
\pgfpathlineto{\pgfqpoint{4.923194in}{5.724705in}}%
\pgfpathlineto{\pgfqpoint{4.933967in}{5.724705in}}%
\pgfpathlineto{\pgfqpoint{4.933967in}{5.740246in}}%
\pgfpathlineto{\pgfqpoint{4.944739in}{5.740246in}}%
\pgfpathlineto{\pgfqpoint{4.944739in}{5.676058in}}%
\pgfpathlineto{\pgfqpoint{4.955512in}{5.676058in}}%
\pgfpathlineto{\pgfqpoint{4.955512in}{5.678760in}}%
\pgfpathlineto{\pgfqpoint{4.966285in}{5.678760in}}%
\pgfpathlineto{\pgfqpoint{4.966285in}{5.666598in}}%
\pgfpathlineto{\pgfqpoint{4.977058in}{5.666598in}}%
\pgfpathlineto{\pgfqpoint{4.977058in}{5.585519in}}%
\pgfpathlineto{\pgfqpoint{4.987830in}{5.585519in}}%
\pgfpathlineto{\pgfqpoint{4.987830in}{5.540250in}}%
\pgfpathlineto{\pgfqpoint{4.998603in}{5.540250in}}%
\pgfpathlineto{\pgfqpoint{4.998603in}{5.538223in}}%
\pgfpathlineto{\pgfqpoint{5.009376in}{5.538223in}}%
\pgfpathlineto{\pgfqpoint{5.009376in}{5.463224in}}%
\pgfpathlineto{\pgfqpoint{5.020149in}{5.463224in}}%
\pgfpathlineto{\pgfqpoint{5.020149in}{5.368631in}}%
\pgfpathlineto{\pgfqpoint{5.041694in}{5.369307in}}%
\pgfpathlineto{\pgfqpoint{5.041694in}{5.304443in}}%
\pgfpathlineto{\pgfqpoint{5.052467in}{5.304443in}}%
\pgfpathlineto{\pgfqpoint{5.052467in}{5.270660in}}%
\pgfpathlineto{\pgfqpoint{5.063240in}{5.270660in}}%
\pgfpathlineto{\pgfqpoint{5.063240in}{5.206472in}}%
\pgfpathlineto{\pgfqpoint{5.074013in}{5.206472in}}%
\pgfpathlineto{\pgfqpoint{5.074013in}{5.172689in}}%
\pgfpathlineto{\pgfqpoint{5.084785in}{5.172689in}}%
\pgfpathlineto{\pgfqpoint{5.084785in}{5.109853in}}%
\pgfpathlineto{\pgfqpoint{5.095558in}{5.109853in}}%
\pgfpathlineto{\pgfqpoint{5.095558in}{5.093637in}}%
\pgfpathlineto{\pgfqpoint{5.106331in}{5.093637in}}%
\pgfpathlineto{\pgfqpoint{5.106331in}{4.992287in}}%
\pgfpathlineto{\pgfqpoint{5.117104in}{4.992287in}}%
\pgfpathlineto{\pgfqpoint{5.117104in}{4.999044in}}%
\pgfpathlineto{\pgfqpoint{5.127876in}{4.999044in}}%
\pgfpathlineto{\pgfqpoint{5.127876in}{4.955802in}}%
\pgfpathlineto{\pgfqpoint{5.138649in}{4.955802in}}%
\pgfpathlineto{\pgfqpoint{5.138649in}{4.899046in}}%
\pgfpathlineto{\pgfqpoint{5.149422in}{4.899046in}}%
\pgfpathlineto{\pgfqpoint{5.149422in}{4.884857in}}%
\pgfpathlineto{\pgfqpoint{5.160195in}{4.884857in}}%
\pgfpathlineto{\pgfqpoint{5.160195in}{4.843642in}}%
\pgfpathlineto{\pgfqpoint{5.170967in}{4.843642in}}%
\pgfpathlineto{\pgfqpoint{5.170967in}{4.803102in}}%
\pgfpathlineto{\pgfqpoint{5.181740in}{4.803102in}}%
\pgfpathlineto{\pgfqpoint{5.181740in}{4.767292in}}%
\pgfpathlineto{\pgfqpoint{5.192513in}{4.767292in}}%
\pgfpathlineto{\pgfqpoint{5.192513in}{4.753778in}}%
\pgfpathlineto{\pgfqpoint{5.203286in}{4.753778in}}%
\pgfpathlineto{\pgfqpoint{5.203286in}{4.715941in}}%
\pgfpathlineto{\pgfqpoint{5.214058in}{4.715941in}}%
\pgfpathlineto{\pgfqpoint{5.214058in}{4.701077in}}%
\pgfpathlineto{\pgfqpoint{5.224831in}{4.701077in}}%
\pgfpathlineto{\pgfqpoint{5.224831in}{4.664591in}}%
\pgfpathlineto{\pgfqpoint{5.235604in}{4.664591in}}%
\pgfpathlineto{\pgfqpoint{5.235604in}{4.641619in}}%
\pgfpathlineto{\pgfqpoint{5.246377in}{4.641619in}}%
\pgfpathlineto{\pgfqpoint{5.246377in}{4.647024in}}%
\pgfpathlineto{\pgfqpoint{5.257149in}{4.647024in}}%
\pgfpathlineto{\pgfqpoint{5.257149in}{4.609187in}}%
\pgfpathlineto{\pgfqpoint{5.267922in}{4.609187in}}%
\pgfpathlineto{\pgfqpoint{5.267922in}{4.584863in}}%
\pgfpathlineto{\pgfqpoint{5.289468in}{4.584187in}}%
\pgfpathlineto{\pgfqpoint{5.289468in}{4.573377in}}%
\pgfpathlineto{\pgfqpoint{5.300241in}{4.573377in}}%
\pgfpathlineto{\pgfqpoint{5.300241in}{4.562566in}}%
\pgfpathlineto{\pgfqpoint{5.311013in}{4.562566in}}%
\pgfpathlineto{\pgfqpoint{5.311013in}{4.557161in}}%
\pgfpathlineto{\pgfqpoint{5.321786in}{4.557161in}}%
\pgfpathlineto{\pgfqpoint{5.321786in}{4.540269in}}%
\pgfpathlineto{\pgfqpoint{5.332559in}{4.540269in}}%
\pgfpathlineto{\pgfqpoint{5.332559in}{4.524729in}}%
\pgfpathlineto{\pgfqpoint{5.343332in}{4.524729in}}%
\pgfpathlineto{\pgfqpoint{5.343332in}{4.513918in}}%
\pgfpathlineto{\pgfqpoint{5.354104in}{4.513918in}}%
\pgfpathlineto{\pgfqpoint{5.354104in}{4.507837in}}%
\pgfpathlineto{\pgfqpoint{5.364877in}{4.507837in}}%
\pgfpathlineto{\pgfqpoint{5.364877in}{4.498378in}}%
\pgfpathlineto{\pgfqpoint{5.375650in}{4.498378in}}%
\pgfpathlineto{\pgfqpoint{5.375650in}{4.493648in}}%
\pgfpathlineto{\pgfqpoint{5.386423in}{4.493648in}}%
\pgfpathlineto{\pgfqpoint{5.386423in}{4.495000in}}%
\pgfpathlineto{\pgfqpoint{5.397195in}{4.495000in}}%
\pgfpathlineto{\pgfqpoint{5.397195in}{4.486892in}}%
\pgfpathlineto{\pgfqpoint{5.407968in}{4.486892in}}%
\pgfpathlineto{\pgfqpoint{5.407968in}{4.482838in}}%
\pgfpathlineto{\pgfqpoint{5.418741in}{4.482838in}}%
\pgfpathlineto{\pgfqpoint{5.418741in}{4.474730in}}%
\pgfpathlineto{\pgfqpoint{5.440286in}{4.475406in}}%
\pgfpathlineto{\pgfqpoint{5.440286in}{4.470000in}}%
\pgfpathlineto{\pgfqpoint{5.451059in}{4.470000in}}%
\pgfpathlineto{\pgfqpoint{5.451059in}{4.467298in}}%
\pgfpathlineto{\pgfqpoint{5.461832in}{4.467298in}}%
\pgfpathlineto{\pgfqpoint{5.461832in}{4.470000in}}%
\pgfpathlineto{\pgfqpoint{5.472605in}{4.470000in}}%
\pgfpathlineto{\pgfqpoint{5.472605in}{4.457838in}}%
\pgfpathlineto{\pgfqpoint{5.483377in}{4.457838in}}%
\pgfpathlineto{\pgfqpoint{5.483377in}{4.467298in}}%
\pgfpathlineto{\pgfqpoint{5.494150in}{4.467298in}}%
\pgfpathlineto{\pgfqpoint{5.494150in}{4.459190in}}%
\pgfpathlineto{\pgfqpoint{5.515696in}{4.459865in}}%
\pgfpathlineto{\pgfqpoint{5.515696in}{4.454460in}}%
\pgfpathlineto{\pgfqpoint{5.548014in}{4.454460in}}%
\pgfpathlineto{\pgfqpoint{5.548014in}{4.452433in}}%
\pgfpathlineto{\pgfqpoint{5.558787in}{4.452433in}}%
\pgfpathlineto{\pgfqpoint{5.558787in}{4.448379in}}%
\pgfpathlineto{\pgfqpoint{5.569559in}{4.448379in}}%
\pgfpathlineto{\pgfqpoint{5.569559in}{4.451757in}}%
\pgfpathlineto{\pgfqpoint{5.580332in}{4.451757in}}%
\pgfpathlineto{\pgfqpoint{5.580332in}{4.455136in}}%
\pgfpathlineto{\pgfqpoint{5.591105in}{4.455136in}}%
\pgfpathlineto{\pgfqpoint{5.591105in}{4.449730in}}%
\pgfpathlineto{\pgfqpoint{5.601878in}{4.449730in}}%
\pgfpathlineto{\pgfqpoint{5.601878in}{4.448379in}}%
\pgfpathlineto{\pgfqpoint{5.612651in}{4.448379in}}%
\pgfpathlineto{\pgfqpoint{5.612651in}{4.450406in}}%
\pgfpathlineto{\pgfqpoint{5.634196in}{4.450406in}}%
\pgfpathlineto{\pgfqpoint{5.634196in}{4.448379in}}%
\pgfpathlineto{\pgfqpoint{5.709605in}{4.447703in}}%
\pgfpathlineto{\pgfqpoint{5.709605in}{4.447028in}}%
\pgfpathlineto{\pgfqpoint{5.731151in}{4.447703in}}%
\pgfpathlineto{\pgfqpoint{5.731151in}{4.449055in}}%
\pgfpathlineto{\pgfqpoint{5.741924in}{4.449055in}}%
\pgfpathlineto{\pgfqpoint{5.741924in}{4.447703in}}%
\pgfpathlineto{\pgfqpoint{5.946606in}{4.447703in}}%
\pgfpathlineto{\pgfqpoint{5.946606in}{4.447028in}}%
\pgfusepath{stroke}%
\end{pgfscope}%
\begin{pgfscope}%
\pgfpathrectangle{\pgfqpoint{3.750134in}{4.447028in}}{\pgfqpoint{2.301066in}{1.417472in}}%
\pgfusepath{clip}%
\pgfsetbuttcap%
\pgfsetmiterjoin%
\pgfsetlinewidth{1.003750pt}%
\definecolor{currentstroke}{rgb}{0.949020,0.372549,0.360784}%
\pgfsetstrokecolor{currentstroke}%
\pgfsetdash{{1.000000pt}{1.650000pt}}{0.000000pt}%
\pgfpathmoveto{\pgfqpoint{3.878236in}{4.447028in}}%
\pgfpathlineto{\pgfqpoint{3.878236in}{4.448380in}}%
\pgfpathlineto{\pgfqpoint{3.899782in}{4.447704in}}%
\pgfpathlineto{\pgfqpoint{3.899782in}{4.449732in}}%
\pgfpathlineto{\pgfqpoint{3.910555in}{4.449732in}}%
\pgfpathlineto{\pgfqpoint{3.910555in}{4.448380in}}%
\pgfpathlineto{\pgfqpoint{3.921327in}{4.448380in}}%
\pgfpathlineto{\pgfqpoint{3.921327in}{4.451759in}}%
\pgfpathlineto{\pgfqpoint{3.932100in}{4.451759in}}%
\pgfpathlineto{\pgfqpoint{3.932100in}{4.448380in}}%
\pgfpathlineto{\pgfqpoint{3.953646in}{4.448380in}}%
\pgfpathlineto{\pgfqpoint{3.953646in}{4.449732in}}%
\pgfpathlineto{\pgfqpoint{3.964418in}{4.449732in}}%
\pgfpathlineto{\pgfqpoint{3.964418in}{4.452435in}}%
\pgfpathlineto{\pgfqpoint{3.975191in}{4.452435in}}%
\pgfpathlineto{\pgfqpoint{3.975191in}{4.449056in}}%
\pgfpathlineto{\pgfqpoint{3.985964in}{4.449056in}}%
\pgfpathlineto{\pgfqpoint{3.985964in}{4.451083in}}%
\pgfpathlineto{\pgfqpoint{3.996737in}{4.451083in}}%
\pgfpathlineto{\pgfqpoint{3.996737in}{4.449732in}}%
\pgfpathlineto{\pgfqpoint{4.018282in}{4.449732in}}%
\pgfpathlineto{\pgfqpoint{4.018282in}{4.451083in}}%
\pgfpathlineto{\pgfqpoint{4.039828in}{4.451083in}}%
\pgfpathlineto{\pgfqpoint{4.039828in}{4.457167in}}%
\pgfpathlineto{\pgfqpoint{4.050600in}{4.457167in}}%
\pgfpathlineto{\pgfqpoint{4.050600in}{4.449732in}}%
\pgfpathlineto{\pgfqpoint{4.061373in}{4.449732in}}%
\pgfpathlineto{\pgfqpoint{4.061373in}{4.452435in}}%
\pgfpathlineto{\pgfqpoint{4.072146in}{4.452435in}}%
\pgfpathlineto{\pgfqpoint{4.072146in}{4.455139in}}%
\pgfpathlineto{\pgfqpoint{4.104464in}{4.455815in}}%
\pgfpathlineto{\pgfqpoint{4.104464in}{4.457843in}}%
\pgfpathlineto{\pgfqpoint{4.115237in}{4.457843in}}%
\pgfpathlineto{\pgfqpoint{4.115237in}{4.455139in}}%
\pgfpathlineto{\pgfqpoint{4.126010in}{4.455139in}}%
\pgfpathlineto{\pgfqpoint{4.126010in}{4.453787in}}%
\pgfpathlineto{\pgfqpoint{4.136782in}{4.453787in}}%
\pgfpathlineto{\pgfqpoint{4.136782in}{4.459195in}}%
\pgfpathlineto{\pgfqpoint{4.147555in}{4.459195in}}%
\pgfpathlineto{\pgfqpoint{4.147555in}{4.461223in}}%
\pgfpathlineto{\pgfqpoint{4.158328in}{4.461223in}}%
\pgfpathlineto{\pgfqpoint{4.158328in}{4.463251in}}%
\pgfpathlineto{\pgfqpoint{4.169101in}{4.463251in}}%
\pgfpathlineto{\pgfqpoint{4.169101in}{4.460547in}}%
\pgfpathlineto{\pgfqpoint{4.179874in}{4.460547in}}%
\pgfpathlineto{\pgfqpoint{4.179874in}{4.465278in}}%
\pgfpathlineto{\pgfqpoint{4.190646in}{4.465278in}}%
\pgfpathlineto{\pgfqpoint{4.190646in}{4.457167in}}%
\pgfpathlineto{\pgfqpoint{4.201419in}{4.457167in}}%
\pgfpathlineto{\pgfqpoint{4.201419in}{4.464603in}}%
\pgfpathlineto{\pgfqpoint{4.212192in}{4.464603in}}%
\pgfpathlineto{\pgfqpoint{4.212192in}{4.465954in}}%
\pgfpathlineto{\pgfqpoint{4.233737in}{4.466630in}}%
\pgfpathlineto{\pgfqpoint{4.233737in}{4.476094in}}%
\pgfpathlineto{\pgfqpoint{4.244510in}{4.476094in}}%
\pgfpathlineto{\pgfqpoint{4.244510in}{4.477446in}}%
\pgfpathlineto{\pgfqpoint{4.255283in}{4.477446in}}%
\pgfpathlineto{\pgfqpoint{4.255283in}{4.476094in}}%
\pgfpathlineto{\pgfqpoint{4.266056in}{4.476094in}}%
\pgfpathlineto{\pgfqpoint{4.266056in}{4.480149in}}%
\pgfpathlineto{\pgfqpoint{4.276828in}{4.480149in}}%
\pgfpathlineto{\pgfqpoint{4.276828in}{4.478122in}}%
\pgfpathlineto{\pgfqpoint{4.287601in}{4.478122in}}%
\pgfpathlineto{\pgfqpoint{4.287601in}{4.480149in}}%
\pgfpathlineto{\pgfqpoint{4.298374in}{4.480149in}}%
\pgfpathlineto{\pgfqpoint{4.298374in}{4.488261in}}%
\pgfpathlineto{\pgfqpoint{4.309147in}{4.488261in}}%
\pgfpathlineto{\pgfqpoint{4.309147in}{4.474742in}}%
\pgfpathlineto{\pgfqpoint{4.319919in}{4.474742in}}%
\pgfpathlineto{\pgfqpoint{4.319919in}{4.485557in}}%
\pgfpathlineto{\pgfqpoint{4.330692in}{4.485557in}}%
\pgfpathlineto{\pgfqpoint{4.330692in}{4.498400in}}%
\pgfpathlineto{\pgfqpoint{4.341465in}{4.498400in}}%
\pgfpathlineto{\pgfqpoint{4.341465in}{4.499752in}}%
\pgfpathlineto{\pgfqpoint{4.352238in}{4.499752in}}%
\pgfpathlineto{\pgfqpoint{4.352238in}{4.495696in}}%
\pgfpathlineto{\pgfqpoint{4.363010in}{4.495696in}}%
\pgfpathlineto{\pgfqpoint{4.363010in}{4.492317in}}%
\pgfpathlineto{\pgfqpoint{4.373783in}{4.492317in}}%
\pgfpathlineto{\pgfqpoint{4.373783in}{4.499752in}}%
\pgfpathlineto{\pgfqpoint{4.384556in}{4.499752in}}%
\pgfpathlineto{\pgfqpoint{4.384556in}{4.519355in}}%
\pgfpathlineto{\pgfqpoint{4.395329in}{4.519355in}}%
\pgfpathlineto{\pgfqpoint{4.395329in}{4.520707in}}%
\pgfpathlineto{\pgfqpoint{4.406101in}{4.520707in}}%
\pgfpathlineto{\pgfqpoint{4.406101in}{4.519355in}}%
\pgfpathlineto{\pgfqpoint{4.416874in}{4.519355in}}%
\pgfpathlineto{\pgfqpoint{4.416874in}{4.531522in}}%
\pgfpathlineto{\pgfqpoint{4.427647in}{4.531522in}}%
\pgfpathlineto{\pgfqpoint{4.427647in}{4.532874in}}%
\pgfpathlineto{\pgfqpoint{4.438420in}{4.532874in}}%
\pgfpathlineto{\pgfqpoint{4.438420in}{4.553153in}}%
\pgfpathlineto{\pgfqpoint{4.449193in}{4.553153in}}%
\pgfpathlineto{\pgfqpoint{4.449193in}{4.535578in}}%
\pgfpathlineto{\pgfqpoint{4.459965in}{4.535578in}}%
\pgfpathlineto{\pgfqpoint{4.459965in}{4.573431in}}%
\pgfpathlineto{\pgfqpoint{4.470738in}{4.573431in}}%
\pgfpathlineto{\pgfqpoint{4.470738in}{4.563292in}}%
\pgfpathlineto{\pgfqpoint{4.481511in}{4.563292in}}%
\pgfpathlineto{\pgfqpoint{4.481511in}{4.592358in}}%
\pgfpathlineto{\pgfqpoint{4.492284in}{4.592358in}}%
\pgfpathlineto{\pgfqpoint{4.492284in}{4.587626in}}%
\pgfpathlineto{\pgfqpoint{4.503056in}{4.587626in}}%
\pgfpathlineto{\pgfqpoint{4.503056in}{4.612636in}}%
\pgfpathlineto{\pgfqpoint{4.513829in}{4.612636in}}%
\pgfpathlineto{\pgfqpoint{4.513829in}{4.598441in}}%
\pgfpathlineto{\pgfqpoint{4.524602in}{4.598441in}}%
\pgfpathlineto{\pgfqpoint{4.524602in}{4.622776in}}%
\pgfpathlineto{\pgfqpoint{4.535375in}{4.622776in}}%
\pgfpathlineto{\pgfqpoint{4.535375in}{4.621424in}}%
\pgfpathlineto{\pgfqpoint{4.546147in}{4.621424in}}%
\pgfpathlineto{\pgfqpoint{4.546147in}{4.659953in}}%
\pgfpathlineto{\pgfqpoint{4.556920in}{4.659953in}}%
\pgfpathlineto{\pgfqpoint{4.556920in}{4.671445in}}%
\pgfpathlineto{\pgfqpoint{4.567693in}{4.671445in}}%
\pgfpathlineto{\pgfqpoint{4.567693in}{4.676852in}}%
\pgfpathlineto{\pgfqpoint{4.578466in}{4.676852in}}%
\pgfpathlineto{\pgfqpoint{4.578466in}{4.727549in}}%
\pgfpathlineto{\pgfqpoint{4.589238in}{4.727549in}}%
\pgfpathlineto{\pgfqpoint{4.589238in}{4.737012in}}%
\pgfpathlineto{\pgfqpoint{4.600011in}{4.737012in}}%
\pgfpathlineto{\pgfqpoint{4.600011in}{4.741068in}}%
\pgfpathlineto{\pgfqpoint{4.610784in}{4.741068in}}%
\pgfpathlineto{\pgfqpoint{4.610784in}{4.750531in}}%
\pgfpathlineto{\pgfqpoint{4.621557in}{4.750531in}}%
\pgfpathlineto{\pgfqpoint{4.621557in}{4.785005in}}%
\pgfpathlineto{\pgfqpoint{4.632329in}{4.785005in}}%
\pgfpathlineto{\pgfqpoint{4.632329in}{4.794468in}}%
\pgfpathlineto{\pgfqpoint{4.643102in}{4.794468in}}%
\pgfpathlineto{\pgfqpoint{4.643102in}{4.801904in}}%
\pgfpathlineto{\pgfqpoint{4.653875in}{4.801904in}}%
\pgfpathlineto{\pgfqpoint{4.653875in}{4.870175in}}%
\pgfpathlineto{\pgfqpoint{4.664648in}{4.870175in}}%
\pgfpathlineto{\pgfqpoint{4.664648in}{4.884370in}}%
\pgfpathlineto{\pgfqpoint{4.675420in}{4.884370in}}%
\pgfpathlineto{\pgfqpoint{4.675420in}{4.912084in}}%
\pgfpathlineto{\pgfqpoint{4.686193in}{4.912084in}}%
\pgfpathlineto{\pgfqpoint{4.686193in}{4.931687in}}%
\pgfpathlineto{\pgfqpoint{4.696966in}{4.931687in}}%
\pgfpathlineto{\pgfqpoint{4.696966in}{4.943178in}}%
\pgfpathlineto{\pgfqpoint{4.707739in}{4.943178in}}%
\pgfpathlineto{\pgfqpoint{4.707739in}{4.995902in}}%
\pgfpathlineto{\pgfqpoint{4.718512in}{4.995902in}}%
\pgfpathlineto{\pgfqpoint{4.718512in}{5.011449in}}%
\pgfpathlineto{\pgfqpoint{4.729284in}{5.011449in}}%
\pgfpathlineto{\pgfqpoint{4.729284in}{5.058090in}}%
\pgfpathlineto{\pgfqpoint{4.740057in}{5.058090in}}%
\pgfpathlineto{\pgfqpoint{4.740057in}{5.106759in}}%
\pgfpathlineto{\pgfqpoint{4.750830in}{5.106759in}}%
\pgfpathlineto{\pgfqpoint{4.750830in}{5.152724in}}%
\pgfpathlineto{\pgfqpoint{4.761603in}{5.152724in}}%
\pgfpathlineto{\pgfqpoint{4.761603in}{5.168947in}}%
\pgfpathlineto{\pgfqpoint{4.772375in}{5.168947in}}%
\pgfpathlineto{\pgfqpoint{4.772375in}{5.191929in}}%
\pgfpathlineto{\pgfqpoint{4.783148in}{5.191929in}}%
\pgfpathlineto{\pgfqpoint{4.783148in}{5.214236in}}%
\pgfpathlineto{\pgfqpoint{4.793921in}{5.214236in}}%
\pgfpathlineto{\pgfqpoint{4.793921in}{5.316305in}}%
\pgfpathlineto{\pgfqpoint{4.804694in}{5.316305in}}%
\pgfpathlineto{\pgfqpoint{4.804694in}{5.327796in}}%
\pgfpathlineto{\pgfqpoint{4.815466in}{5.327796in}}%
\pgfpathlineto{\pgfqpoint{4.815466in}{5.383224in}}%
\pgfpathlineto{\pgfqpoint{4.826239in}{5.383224in}}%
\pgfpathlineto{\pgfqpoint{4.826239in}{5.423781in}}%
\pgfpathlineto{\pgfqpoint{4.837012in}{5.423781in}}%
\pgfpathlineto{\pgfqpoint{4.837012in}{5.440004in}}%
\pgfpathlineto{\pgfqpoint{4.847785in}{5.440004in}}%
\pgfpathlineto{\pgfqpoint{4.847785in}{5.476506in}}%
\pgfpathlineto{\pgfqpoint{4.858557in}{5.476506in}}%
\pgfpathlineto{\pgfqpoint{4.858557in}{5.518415in}}%
\pgfpathlineto{\pgfqpoint{4.869330in}{5.518415in}}%
\pgfpathlineto{\pgfqpoint{4.869330in}{5.531934in}}%
\pgfpathlineto{\pgfqpoint{4.880103in}{5.531934in}}%
\pgfpathlineto{\pgfqpoint{4.880103in}{5.619132in}}%
\pgfpathlineto{\pgfqpoint{4.890876in}{5.619132in}}%
\pgfpathlineto{\pgfqpoint{4.890876in}{5.594798in}}%
\pgfpathlineto{\pgfqpoint{4.901648in}{5.594798in}}%
\pgfpathlineto{\pgfqpoint{4.901648in}{5.621160in}}%
\pgfpathlineto{\pgfqpoint{4.912421in}{5.621160in}}%
\pgfpathlineto{\pgfqpoint{4.912421in}{5.663745in}}%
\pgfpathlineto{\pgfqpoint{4.923194in}{5.663745in}}%
\pgfpathlineto{\pgfqpoint{4.923194in}{5.615076in}}%
\pgfpathlineto{\pgfqpoint{4.933967in}{5.615076in}}%
\pgfpathlineto{\pgfqpoint{4.933967in}{5.629947in}}%
\pgfpathlineto{\pgfqpoint{4.944739in}{5.629947in}}%
\pgfpathlineto{\pgfqpoint{4.944739in}{5.596150in}}%
\pgfpathlineto{\pgfqpoint{4.955512in}{5.596150in}}%
\pgfpathlineto{\pgfqpoint{4.955512in}{5.614401in}}%
\pgfpathlineto{\pgfqpoint{4.966285in}{5.614401in}}%
\pgfpathlineto{\pgfqpoint{4.966285in}{5.510980in}}%
\pgfpathlineto{\pgfqpoint{4.977058in}{5.510980in}}%
\pgfpathlineto{\pgfqpoint{4.977058in}{5.503544in}}%
\pgfpathlineto{\pgfqpoint{4.987830in}{5.503544in}}%
\pgfpathlineto{\pgfqpoint{4.987830in}{5.462311in}}%
\pgfpathlineto{\pgfqpoint{4.998603in}{5.462311in}}%
\pgfpathlineto{\pgfqpoint{4.998603in}{5.425133in}}%
\pgfpathlineto{\pgfqpoint{5.009376in}{5.425133in}}%
\pgfpathlineto{\pgfqpoint{5.009376in}{5.436625in}}%
\pgfpathlineto{\pgfqpoint{5.020149in}{5.436625in}}%
\pgfpathlineto{\pgfqpoint{5.020149in}{5.358214in}}%
\pgfpathlineto{\pgfqpoint{5.030922in}{5.358214in}}%
\pgfpathlineto{\pgfqpoint{5.030922in}{5.338611in}}%
\pgfpathlineto{\pgfqpoint{5.041694in}{5.338611in}}%
\pgfpathlineto{\pgfqpoint{5.041694in}{5.243978in}}%
\pgfpathlineto{\pgfqpoint{5.052467in}{5.243978in}}%
\pgfpathlineto{\pgfqpoint{5.052467in}{5.257497in}}%
\pgfpathlineto{\pgfqpoint{5.063240in}{5.257497in}}%
\pgfpathlineto{\pgfqpoint{5.063240in}{5.198689in}}%
\pgfpathlineto{\pgfqpoint{5.074013in}{5.198689in}}%
\pgfpathlineto{\pgfqpoint{5.074013in}{5.175030in}}%
\pgfpathlineto{\pgfqpoint{5.084785in}{5.175030in}}%
\pgfpathlineto{\pgfqpoint{5.084785in}{5.125686in}}%
\pgfpathlineto{\pgfqpoint{5.095558in}{5.125686in}}%
\pgfpathlineto{\pgfqpoint{5.095558in}{5.108787in}}%
\pgfpathlineto{\pgfqpoint{5.106331in}{5.108787in}}%
\pgfpathlineto{\pgfqpoint{5.106331in}{5.058090in}}%
\pgfpathlineto{\pgfqpoint{5.117104in}{5.058090in}}%
\pgfpathlineto{\pgfqpoint{5.117104in}{5.022265in}}%
\pgfpathlineto{\pgfqpoint{5.127876in}{5.022265in}}%
\pgfpathlineto{\pgfqpoint{5.127876in}{5.003338in}}%
\pgfpathlineto{\pgfqpoint{5.138649in}{5.003338in}}%
\pgfpathlineto{\pgfqpoint{5.138649in}{5.000634in}}%
\pgfpathlineto{\pgfqpoint{5.149422in}{5.000634in}}%
\pgfpathlineto{\pgfqpoint{5.149422in}{4.950614in}}%
\pgfpathlineto{\pgfqpoint{5.160195in}{4.950614in}}%
\pgfpathlineto{\pgfqpoint{5.160195in}{4.885722in}}%
\pgfpathlineto{\pgfqpoint{5.170967in}{4.885722in}}%
\pgfpathlineto{\pgfqpoint{5.170967in}{4.877611in}}%
\pgfpathlineto{\pgfqpoint{5.181740in}{4.877611in}}%
\pgfpathlineto{\pgfqpoint{5.181740in}{4.834349in}}%
\pgfpathlineto{\pgfqpoint{5.192513in}{4.834349in}}%
\pgfpathlineto{\pgfqpoint{5.192513in}{4.818803in}}%
\pgfpathlineto{\pgfqpoint{5.203286in}{4.818803in}}%
\pgfpathlineto{\pgfqpoint{5.203286in}{4.831646in}}%
\pgfpathlineto{\pgfqpoint{5.214058in}{4.831646in}}%
\pgfpathlineto{\pgfqpoint{5.214058in}{4.744448in}}%
\pgfpathlineto{\pgfqpoint{5.224831in}{4.744448in}}%
\pgfpathlineto{\pgfqpoint{5.224831in}{4.749855in}}%
\pgfpathlineto{\pgfqpoint{5.235604in}{4.749855in}}%
\pgfpathlineto{\pgfqpoint{5.235604in}{4.744448in}}%
\pgfpathlineto{\pgfqpoint{5.257149in}{4.745124in}}%
\pgfpathlineto{\pgfqpoint{5.257149in}{4.718761in}}%
\pgfpathlineto{\pgfqpoint{5.267922in}{4.718761in}}%
\pgfpathlineto{\pgfqpoint{5.267922in}{4.701186in}}%
\pgfpathlineto{\pgfqpoint{5.278695in}{4.701186in}}%
\pgfpathlineto{\pgfqpoint{5.278695in}{4.685640in}}%
\pgfpathlineto{\pgfqpoint{5.289468in}{4.685640in}}%
\pgfpathlineto{\pgfqpoint{5.289468in}{4.664685in}}%
\pgfpathlineto{\pgfqpoint{5.300241in}{4.664685in}}%
\pgfpathlineto{\pgfqpoint{5.300241in}{4.649138in}}%
\pgfpathlineto{\pgfqpoint{5.311013in}{4.649138in}}%
\pgfpathlineto{\pgfqpoint{5.311013in}{4.645758in}}%
\pgfpathlineto{\pgfqpoint{5.321786in}{4.645758in}}%
\pgfpathlineto{\pgfqpoint{5.321786in}{4.627507in}}%
\pgfpathlineto{\pgfqpoint{5.343332in}{4.627507in}}%
\pgfpathlineto{\pgfqpoint{5.343332in}{4.597766in}}%
\pgfpathlineto{\pgfqpoint{5.364877in}{4.598441in}}%
\pgfpathlineto{\pgfqpoint{5.364877in}{4.574107in}}%
\pgfpathlineto{\pgfqpoint{5.375650in}{4.574107in}}%
\pgfpathlineto{\pgfqpoint{5.375650in}{4.583570in}}%
\pgfpathlineto{\pgfqpoint{5.397195in}{4.584246in}}%
\pgfpathlineto{\pgfqpoint{5.397195in}{4.553153in}}%
\pgfpathlineto{\pgfqpoint{5.407968in}{4.553153in}}%
\pgfpathlineto{\pgfqpoint{5.407968in}{4.545717in}}%
\pgfpathlineto{\pgfqpoint{5.418741in}{4.545717in}}%
\pgfpathlineto{\pgfqpoint{5.418741in}{4.543689in}}%
\pgfpathlineto{\pgfqpoint{5.429514in}{4.543689in}}%
\pgfpathlineto{\pgfqpoint{5.429514in}{4.547745in}}%
\pgfpathlineto{\pgfqpoint{5.440286in}{4.547745in}}%
\pgfpathlineto{\pgfqpoint{5.440286in}{4.515975in}}%
\pgfpathlineto{\pgfqpoint{5.451059in}{4.515975in}}%
\pgfpathlineto{\pgfqpoint{5.451059in}{4.529494in}}%
\pgfpathlineto{\pgfqpoint{5.461832in}{4.529494in}}%
\pgfpathlineto{\pgfqpoint{5.461832in}{4.519355in}}%
\pgfpathlineto{\pgfqpoint{5.472605in}{4.519355in}}%
\pgfpathlineto{\pgfqpoint{5.472605in}{4.513271in}}%
\pgfpathlineto{\pgfqpoint{5.483377in}{4.513271in}}%
\pgfpathlineto{\pgfqpoint{5.483377in}{4.507864in}}%
\pgfpathlineto{\pgfqpoint{5.494150in}{4.507864in}}%
\pgfpathlineto{\pgfqpoint{5.494150in}{4.513947in}}%
\pgfpathlineto{\pgfqpoint{5.504923in}{4.513947in}}%
\pgfpathlineto{\pgfqpoint{5.504923in}{4.491641in}}%
\pgfpathlineto{\pgfqpoint{5.515696in}{4.491641in}}%
\pgfpathlineto{\pgfqpoint{5.515696in}{4.509891in}}%
\pgfpathlineto{\pgfqpoint{5.526468in}{4.509891in}}%
\pgfpathlineto{\pgfqpoint{5.526468in}{4.492317in}}%
\pgfpathlineto{\pgfqpoint{5.537241in}{4.492317in}}%
\pgfpathlineto{\pgfqpoint{5.537241in}{4.485557in}}%
\pgfpathlineto{\pgfqpoint{5.548014in}{4.485557in}}%
\pgfpathlineto{\pgfqpoint{5.548014in}{4.490289in}}%
\pgfpathlineto{\pgfqpoint{5.558787in}{4.490289in}}%
\pgfpathlineto{\pgfqpoint{5.558787in}{4.485557in}}%
\pgfpathlineto{\pgfqpoint{5.569559in}{4.485557in}}%
\pgfpathlineto{\pgfqpoint{5.569559in}{4.495020in}}%
\pgfpathlineto{\pgfqpoint{5.580332in}{4.495020in}}%
\pgfpathlineto{\pgfqpoint{5.580332in}{4.485557in}}%
\pgfpathlineto{\pgfqpoint{5.601878in}{4.486233in}}%
\pgfpathlineto{\pgfqpoint{5.601878in}{4.477446in}}%
\pgfpathlineto{\pgfqpoint{5.612651in}{4.477446in}}%
\pgfpathlineto{\pgfqpoint{5.612651in}{4.472714in}}%
\pgfpathlineto{\pgfqpoint{5.623423in}{4.472714in}}%
\pgfpathlineto{\pgfqpoint{5.623423in}{4.467982in}}%
\pgfpathlineto{\pgfqpoint{5.634196in}{4.467982in}}%
\pgfpathlineto{\pgfqpoint{5.634196in}{4.465954in}}%
\pgfpathlineto{\pgfqpoint{5.655742in}{4.465278in}}%
\pgfpathlineto{\pgfqpoint{5.655742in}{4.467982in}}%
\pgfpathlineto{\pgfqpoint{5.666514in}{4.467982in}}%
\pgfpathlineto{\pgfqpoint{5.666514in}{4.473390in}}%
\pgfpathlineto{\pgfqpoint{5.677287in}{4.473390in}}%
\pgfpathlineto{\pgfqpoint{5.677287in}{4.465954in}}%
\pgfpathlineto{\pgfqpoint{5.688060in}{4.465954in}}%
\pgfpathlineto{\pgfqpoint{5.688060in}{4.461899in}}%
\pgfpathlineto{\pgfqpoint{5.698833in}{4.461899in}}%
\pgfpathlineto{\pgfqpoint{5.698833in}{4.463251in}}%
\pgfpathlineto{\pgfqpoint{5.720378in}{4.462575in}}%
\pgfpathlineto{\pgfqpoint{5.720378in}{4.464603in}}%
\pgfpathlineto{\pgfqpoint{5.731151in}{4.464603in}}%
\pgfpathlineto{\pgfqpoint{5.731151in}{4.458519in}}%
\pgfpathlineto{\pgfqpoint{5.752696in}{4.457843in}}%
\pgfpathlineto{\pgfqpoint{5.752696in}{4.461223in}}%
\pgfpathlineto{\pgfqpoint{5.763469in}{4.461223in}}%
\pgfpathlineto{\pgfqpoint{5.763469in}{4.457843in}}%
\pgfpathlineto{\pgfqpoint{5.774242in}{4.457843in}}%
\pgfpathlineto{\pgfqpoint{5.774242in}{4.453111in}}%
\pgfpathlineto{\pgfqpoint{5.817333in}{4.453787in}}%
\pgfpathlineto{\pgfqpoint{5.817333in}{4.451083in}}%
\pgfpathlineto{\pgfqpoint{5.838878in}{4.450408in}}%
\pgfpathlineto{\pgfqpoint{5.838878in}{4.453787in}}%
\pgfpathlineto{\pgfqpoint{5.849651in}{4.453787in}}%
\pgfpathlineto{\pgfqpoint{5.849651in}{4.449056in}}%
\pgfpathlineto{\pgfqpoint{5.860424in}{4.449056in}}%
\pgfpathlineto{\pgfqpoint{5.860424in}{4.454463in}}%
\pgfpathlineto{\pgfqpoint{5.871197in}{4.454463in}}%
\pgfpathlineto{\pgfqpoint{5.871197in}{4.450408in}}%
\pgfpathlineto{\pgfqpoint{5.881970in}{4.450408in}}%
\pgfpathlineto{\pgfqpoint{5.881970in}{4.449056in}}%
\pgfpathlineto{\pgfqpoint{5.892742in}{4.449056in}}%
\pgfpathlineto{\pgfqpoint{5.892742in}{4.450408in}}%
\pgfpathlineto{\pgfqpoint{5.903515in}{4.450408in}}%
\pgfpathlineto{\pgfqpoint{5.903515in}{4.448380in}}%
\pgfpathlineto{\pgfqpoint{5.914288in}{4.448380in}}%
\pgfpathlineto{\pgfqpoint{5.914288in}{4.452435in}}%
\pgfpathlineto{\pgfqpoint{5.925061in}{4.452435in}}%
\pgfpathlineto{\pgfqpoint{5.925061in}{4.449732in}}%
\pgfpathlineto{\pgfqpoint{5.935833in}{4.449732in}}%
\pgfpathlineto{\pgfqpoint{5.935833in}{4.452435in}}%
\pgfpathlineto{\pgfqpoint{5.946606in}{4.452435in}}%
\pgfpathlineto{\pgfqpoint{5.946606in}{4.447028in}}%
\pgfpathlineto{\pgfqpoint{5.946606in}{4.447028in}}%
\pgfusepath{stroke}%
\end{pgfscope}%
\begin{pgfscope}%
\pgfsetrectcap%
\pgfsetmiterjoin%
\pgfsetlinewidth{0.803000pt}%
\definecolor{currentstroke}{rgb}{0.000000,0.000000,0.000000}%
\pgfsetstrokecolor{currentstroke}%
\pgfsetdash{}{0pt}%
\pgfpathmoveto{\pgfqpoint{3.750134in}{4.447028in}}%
\pgfpathlineto{\pgfqpoint{3.750134in}{5.864500in}}%
\pgfusepath{stroke}%
\end{pgfscope}%
\begin{pgfscope}%
\pgfsetrectcap%
\pgfsetmiterjoin%
\pgfsetlinewidth{0.803000pt}%
\definecolor{currentstroke}{rgb}{0.000000,0.000000,0.000000}%
\pgfsetstrokecolor{currentstroke}%
\pgfsetdash{}{0pt}%
\pgfpathmoveto{\pgfqpoint{6.051200in}{4.447028in}}%
\pgfpathlineto{\pgfqpoint{6.051200in}{5.864500in}}%
\pgfusepath{stroke}%
\end{pgfscope}%
\begin{pgfscope}%
\pgfsetrectcap%
\pgfsetmiterjoin%
\pgfsetlinewidth{0.803000pt}%
\definecolor{currentstroke}{rgb}{0.000000,0.000000,0.000000}%
\pgfsetstrokecolor{currentstroke}%
\pgfsetdash{}{0pt}%
\pgfpathmoveto{\pgfqpoint{3.750134in}{4.447028in}}%
\pgfpathlineto{\pgfqpoint{6.051200in}{4.447028in}}%
\pgfusepath{stroke}%
\end{pgfscope}%
\begin{pgfscope}%
\pgfsetrectcap%
\pgfsetmiterjoin%
\pgfsetlinewidth{0.803000pt}%
\definecolor{currentstroke}{rgb}{0.000000,0.000000,0.000000}%
\pgfsetstrokecolor{currentstroke}%
\pgfsetdash{}{0pt}%
\pgfpathmoveto{\pgfqpoint{3.750134in}{5.864500in}}%
\pgfpathlineto{\pgfqpoint{6.051200in}{5.864500in}}%
\pgfusepath{stroke}%
\end{pgfscope}%
\begin{pgfscope}%
\definecolor{textcolor}{rgb}{0.000000,0.000000,0.000000}%
\pgfsetstrokecolor{textcolor}%
\pgfsetfillcolor{textcolor}%
\pgftext[x=3.750134in,y=5.947833in,left,base]{\color{textcolor}\rmfamily\fontsize{10.000000}{12.000000}\selectfont Bin [0.67, 0.83), 76,473 events}%
\end{pgfscope}%
\begin{pgfscope}%
\pgfsetbuttcap%
\pgfsetmiterjoin%
\definecolor{currentfill}{rgb}{1.000000,1.000000,1.000000}%
\pgfsetfillcolor{currentfill}%
\pgfsetfillopacity{0.800000}%
\pgfsetlinewidth{1.003750pt}%
\definecolor{currentstroke}{rgb}{0.800000,0.800000,0.800000}%
\pgfsetstrokecolor{currentstroke}%
\pgfsetstrokeopacity{0.800000}%
\pgfsetdash{}{0pt}%
\pgfpathmoveto{\pgfqpoint{4.957422in}{5.464944in}}%
\pgfpathlineto{\pgfqpoint{5.973422in}{5.464944in}}%
\pgfpathquadraticcurveto{\pgfqpoint{5.995644in}{5.464944in}}{\pgfqpoint{5.995644in}{5.487167in}}%
\pgfpathlineto{\pgfqpoint{5.995644in}{5.786722in}}%
\pgfpathquadraticcurveto{\pgfqpoint{5.995644in}{5.808944in}}{\pgfqpoint{5.973422in}{5.808944in}}%
\pgfpathlineto{\pgfqpoint{4.957422in}{5.808944in}}%
\pgfpathquadraticcurveto{\pgfqpoint{4.935200in}{5.808944in}}{\pgfqpoint{4.935200in}{5.786722in}}%
\pgfpathlineto{\pgfqpoint{4.935200in}{5.487167in}}%
\pgfpathquadraticcurveto{\pgfqpoint{4.935200in}{5.464944in}}{\pgfqpoint{4.957422in}{5.464944in}}%
\pgfpathclose%
\pgfusepath{stroke,fill}%
\end{pgfscope}%
\begin{pgfscope}%
\pgfsetbuttcap%
\pgfsetmiterjoin%
\pgfsetlinewidth{1.003750pt}%
\definecolor{currentstroke}{rgb}{0.313725,0.317647,0.309804}%
\pgfsetstrokecolor{currentstroke}%
\pgfsetdash{}{0pt}%
\pgfpathmoveto{\pgfqpoint{4.979644in}{5.686278in}}%
\pgfpathlineto{\pgfqpoint{5.201867in}{5.686278in}}%
\pgfpathlineto{\pgfqpoint{5.201867in}{5.764056in}}%
\pgfpathlineto{\pgfqpoint{4.979644in}{5.764056in}}%
\pgfpathclose%
\pgfusepath{stroke}%
\end{pgfscope}%
\begin{pgfscope}%
\definecolor{textcolor}{rgb}{0.000000,0.000000,0.000000}%
\pgfsetstrokecolor{textcolor}%
\pgfsetfillcolor{textcolor}%
\pgftext[x=5.290756in,y=5.686278in,left,base]{\color{textcolor}\rmfamily\fontsize{8.000000}{9.600000}\selectfont IQR = 22.18}%
\end{pgfscope}%
\begin{pgfscope}%
\pgfsetbuttcap%
\pgfsetmiterjoin%
\pgfsetlinewidth{1.003750pt}%
\definecolor{currentstroke}{rgb}{0.949020,0.372549,0.360784}%
\pgfsetstrokecolor{currentstroke}%
\pgfsetdash{{1.000000pt}{1.650000pt}}{0.000000pt}%
\pgfpathmoveto{\pgfqpoint{4.979644in}{5.530944in}}%
\pgfpathlineto{\pgfqpoint{5.201867in}{5.530944in}}%
\pgfpathlineto{\pgfqpoint{5.201867in}{5.608722in}}%
\pgfpathlineto{\pgfqpoint{4.979644in}{5.608722in}}%
\pgfpathclose%
\pgfusepath{stroke}%
\end{pgfscope}%
\begin{pgfscope}%
\definecolor{textcolor}{rgb}{0.000000,0.000000,0.000000}%
\pgfsetstrokecolor{textcolor}%
\pgfsetfillcolor{textcolor}%
\pgftext[x=5.290756in,y=5.530944in,left,base]{\color{textcolor}\rmfamily\fontsize{8.000000}{9.600000}\selectfont IQR = 26.23}%
\end{pgfscope}%
\begin{pgfscope}%
\pgfsetbuttcap%
\pgfsetmiterjoin%
\definecolor{currentfill}{rgb}{1.000000,1.000000,1.000000}%
\pgfsetfillcolor{currentfill}%
\pgfsetlinewidth{0.000000pt}%
\definecolor{currentstroke}{rgb}{0.000000,0.000000,0.000000}%
\pgfsetstrokecolor{currentstroke}%
\pgfsetstrokeopacity{0.000000}%
\pgfsetdash{}{0pt}%
\pgfpathmoveto{\pgfqpoint{0.754048in}{2.496167in}}%
\pgfpathlineto{\pgfqpoint{3.055114in}{2.496167in}}%
\pgfpathlineto{\pgfqpoint{3.055114in}{3.913639in}}%
\pgfpathlineto{\pgfqpoint{0.754048in}{3.913639in}}%
\pgfpathclose%
\pgfusepath{fill}%
\end{pgfscope}%
\begin{pgfscope}%
\pgfsetbuttcap%
\pgfsetroundjoin%
\definecolor{currentfill}{rgb}{0.000000,0.000000,0.000000}%
\pgfsetfillcolor{currentfill}%
\pgfsetlinewidth{0.803000pt}%
\definecolor{currentstroke}{rgb}{0.000000,0.000000,0.000000}%
\pgfsetstrokecolor{currentstroke}%
\pgfsetdash{}{0pt}%
\pgfsys@defobject{currentmarker}{\pgfqpoint{0.000000in}{-0.048611in}}{\pgfqpoint{0.000000in}{0.000000in}}{%
\pgfpathmoveto{\pgfqpoint{0.000000in}{0.000000in}}%
\pgfpathlineto{\pgfqpoint{0.000000in}{-0.048611in}}%
\pgfusepath{stroke,fill}%
}%
\begin{pgfscope}%
\pgfsys@transformshift{0.986179in}{2.496167in}%
\pgfsys@useobject{currentmarker}{}%
\end{pgfscope}%
\end{pgfscope}%
\begin{pgfscope}%
\definecolor{textcolor}{rgb}{0.000000,0.000000,0.000000}%
\pgfsetstrokecolor{textcolor}%
\pgfsetfillcolor{textcolor}%
\pgftext[x=0.986179in,y=2.398944in,,top]{\color{textcolor}\rmfamily\fontsize{8.000000}{9.600000}\selectfont \(\displaystyle {-120}\)}%
\end{pgfscope}%
\begin{pgfscope}%
\pgfsetbuttcap%
\pgfsetroundjoin%
\definecolor{currentfill}{rgb}{0.000000,0.000000,0.000000}%
\pgfsetfillcolor{currentfill}%
\pgfsetlinewidth{0.803000pt}%
\definecolor{currentstroke}{rgb}{0.000000,0.000000,0.000000}%
\pgfsetstrokecolor{currentstroke}%
\pgfsetdash{}{0pt}%
\pgfsys@defobject{currentmarker}{\pgfqpoint{0.000000in}{-0.048611in}}{\pgfqpoint{0.000000in}{0.000000in}}{%
\pgfpathmoveto{\pgfqpoint{0.000000in}{0.000000in}}%
\pgfpathlineto{\pgfqpoint{0.000000in}{-0.048611in}}%
\pgfusepath{stroke,fill}%
}%
\begin{pgfscope}%
\pgfsys@transformshift{1.291881in}{2.496167in}%
\pgfsys@useobject{currentmarker}{}%
\end{pgfscope}%
\end{pgfscope}%
\begin{pgfscope}%
\definecolor{textcolor}{rgb}{0.000000,0.000000,0.000000}%
\pgfsetstrokecolor{textcolor}%
\pgfsetfillcolor{textcolor}%
\pgftext[x=1.291881in,y=2.398944in,,top]{\color{textcolor}\rmfamily\fontsize{8.000000}{9.600000}\selectfont \(\displaystyle {-80}\)}%
\end{pgfscope}%
\begin{pgfscope}%
\pgfsetbuttcap%
\pgfsetroundjoin%
\definecolor{currentfill}{rgb}{0.000000,0.000000,0.000000}%
\pgfsetfillcolor{currentfill}%
\pgfsetlinewidth{0.803000pt}%
\definecolor{currentstroke}{rgb}{0.000000,0.000000,0.000000}%
\pgfsetstrokecolor{currentstroke}%
\pgfsetdash{}{0pt}%
\pgfsys@defobject{currentmarker}{\pgfqpoint{0.000000in}{-0.048611in}}{\pgfqpoint{0.000000in}{0.000000in}}{%
\pgfpathmoveto{\pgfqpoint{0.000000in}{0.000000in}}%
\pgfpathlineto{\pgfqpoint{0.000000in}{-0.048611in}}%
\pgfusepath{stroke,fill}%
}%
\begin{pgfscope}%
\pgfsys@transformshift{1.597582in}{2.496167in}%
\pgfsys@useobject{currentmarker}{}%
\end{pgfscope}%
\end{pgfscope}%
\begin{pgfscope}%
\definecolor{textcolor}{rgb}{0.000000,0.000000,0.000000}%
\pgfsetstrokecolor{textcolor}%
\pgfsetfillcolor{textcolor}%
\pgftext[x=1.597582in,y=2.398944in,,top]{\color{textcolor}\rmfamily\fontsize{8.000000}{9.600000}\selectfont \(\displaystyle {-40}\)}%
\end{pgfscope}%
\begin{pgfscope}%
\pgfsetbuttcap%
\pgfsetroundjoin%
\definecolor{currentfill}{rgb}{0.000000,0.000000,0.000000}%
\pgfsetfillcolor{currentfill}%
\pgfsetlinewidth{0.803000pt}%
\definecolor{currentstroke}{rgb}{0.000000,0.000000,0.000000}%
\pgfsetstrokecolor{currentstroke}%
\pgfsetdash{}{0pt}%
\pgfsys@defobject{currentmarker}{\pgfqpoint{0.000000in}{-0.048611in}}{\pgfqpoint{0.000000in}{0.000000in}}{%
\pgfpathmoveto{\pgfqpoint{0.000000in}{0.000000in}}%
\pgfpathlineto{\pgfqpoint{0.000000in}{-0.048611in}}%
\pgfusepath{stroke,fill}%
}%
\begin{pgfscope}%
\pgfsys@transformshift{1.903283in}{2.496167in}%
\pgfsys@useobject{currentmarker}{}%
\end{pgfscope}%
\end{pgfscope}%
\begin{pgfscope}%
\definecolor{textcolor}{rgb}{0.000000,0.000000,0.000000}%
\pgfsetstrokecolor{textcolor}%
\pgfsetfillcolor{textcolor}%
\pgftext[x=1.903283in,y=2.398944in,,top]{\color{textcolor}\rmfamily\fontsize{8.000000}{9.600000}\selectfont \(\displaystyle {0}\)}%
\end{pgfscope}%
\begin{pgfscope}%
\pgfsetbuttcap%
\pgfsetroundjoin%
\definecolor{currentfill}{rgb}{0.000000,0.000000,0.000000}%
\pgfsetfillcolor{currentfill}%
\pgfsetlinewidth{0.803000pt}%
\definecolor{currentstroke}{rgb}{0.000000,0.000000,0.000000}%
\pgfsetstrokecolor{currentstroke}%
\pgfsetdash{}{0pt}%
\pgfsys@defobject{currentmarker}{\pgfqpoint{0.000000in}{-0.048611in}}{\pgfqpoint{0.000000in}{0.000000in}}{%
\pgfpathmoveto{\pgfqpoint{0.000000in}{0.000000in}}%
\pgfpathlineto{\pgfqpoint{0.000000in}{-0.048611in}}%
\pgfusepath{stroke,fill}%
}%
\begin{pgfscope}%
\pgfsys@transformshift{2.208985in}{2.496167in}%
\pgfsys@useobject{currentmarker}{}%
\end{pgfscope}%
\end{pgfscope}%
\begin{pgfscope}%
\definecolor{textcolor}{rgb}{0.000000,0.000000,0.000000}%
\pgfsetstrokecolor{textcolor}%
\pgfsetfillcolor{textcolor}%
\pgftext[x=2.208985in,y=2.398944in,,top]{\color{textcolor}\rmfamily\fontsize{8.000000}{9.600000}\selectfont \(\displaystyle {40}\)}%
\end{pgfscope}%
\begin{pgfscope}%
\pgfsetbuttcap%
\pgfsetroundjoin%
\definecolor{currentfill}{rgb}{0.000000,0.000000,0.000000}%
\pgfsetfillcolor{currentfill}%
\pgfsetlinewidth{0.803000pt}%
\definecolor{currentstroke}{rgb}{0.000000,0.000000,0.000000}%
\pgfsetstrokecolor{currentstroke}%
\pgfsetdash{}{0pt}%
\pgfsys@defobject{currentmarker}{\pgfqpoint{0.000000in}{-0.048611in}}{\pgfqpoint{0.000000in}{0.000000in}}{%
\pgfpathmoveto{\pgfqpoint{0.000000in}{0.000000in}}%
\pgfpathlineto{\pgfqpoint{0.000000in}{-0.048611in}}%
\pgfusepath{stroke,fill}%
}%
\begin{pgfscope}%
\pgfsys@transformshift{2.514686in}{2.496167in}%
\pgfsys@useobject{currentmarker}{}%
\end{pgfscope}%
\end{pgfscope}%
\begin{pgfscope}%
\definecolor{textcolor}{rgb}{0.000000,0.000000,0.000000}%
\pgfsetstrokecolor{textcolor}%
\pgfsetfillcolor{textcolor}%
\pgftext[x=2.514686in,y=2.398944in,,top]{\color{textcolor}\rmfamily\fontsize{8.000000}{9.600000}\selectfont \(\displaystyle {80}\)}%
\end{pgfscope}%
\begin{pgfscope}%
\pgfsetbuttcap%
\pgfsetroundjoin%
\definecolor{currentfill}{rgb}{0.000000,0.000000,0.000000}%
\pgfsetfillcolor{currentfill}%
\pgfsetlinewidth{0.803000pt}%
\definecolor{currentstroke}{rgb}{0.000000,0.000000,0.000000}%
\pgfsetstrokecolor{currentstroke}%
\pgfsetdash{}{0pt}%
\pgfsys@defobject{currentmarker}{\pgfqpoint{0.000000in}{-0.048611in}}{\pgfqpoint{0.000000in}{0.000000in}}{%
\pgfpathmoveto{\pgfqpoint{0.000000in}{0.000000in}}%
\pgfpathlineto{\pgfqpoint{0.000000in}{-0.048611in}}%
\pgfusepath{stroke,fill}%
}%
\begin{pgfscope}%
\pgfsys@transformshift{2.820387in}{2.496167in}%
\pgfsys@useobject{currentmarker}{}%
\end{pgfscope}%
\end{pgfscope}%
\begin{pgfscope}%
\definecolor{textcolor}{rgb}{0.000000,0.000000,0.000000}%
\pgfsetstrokecolor{textcolor}%
\pgfsetfillcolor{textcolor}%
\pgftext[x=2.820387in,y=2.398944in,,top]{\color{textcolor}\rmfamily\fontsize{8.000000}{9.600000}\selectfont \(\displaystyle {120}\)}%
\end{pgfscope}%
\begin{pgfscope}%
\pgfsetbuttcap%
\pgfsetroundjoin%
\definecolor{currentfill}{rgb}{0.000000,0.000000,0.000000}%
\pgfsetfillcolor{currentfill}%
\pgfsetlinewidth{0.803000pt}%
\definecolor{currentstroke}{rgb}{0.000000,0.000000,0.000000}%
\pgfsetstrokecolor{currentstroke}%
\pgfsetdash{}{0pt}%
\pgfsys@defobject{currentmarker}{\pgfqpoint{-0.048611in}{0.000000in}}{\pgfqpoint{-0.000000in}{0.000000in}}{%
\pgfpathmoveto{\pgfqpoint{-0.000000in}{0.000000in}}%
\pgfpathlineto{\pgfqpoint{-0.048611in}{0.000000in}}%
\pgfusepath{stroke,fill}%
}%
\begin{pgfscope}%
\pgfsys@transformshift{0.754048in}{2.496167in}%
\pgfsys@useobject{currentmarker}{}%
\end{pgfscope}%
\end{pgfscope}%
\begin{pgfscope}%
\definecolor{textcolor}{rgb}{0.000000,0.000000,0.000000}%
\pgfsetstrokecolor{textcolor}%
\pgfsetfillcolor{textcolor}%
\pgftext[x=0.387917in, y=2.457611in, left, base]{\color{textcolor}\rmfamily\fontsize{8.000000}{9.600000}\selectfont \(\displaystyle {0.000}\)}%
\end{pgfscope}%
\begin{pgfscope}%
\pgfsetbuttcap%
\pgfsetroundjoin%
\definecolor{currentfill}{rgb}{0.000000,0.000000,0.000000}%
\pgfsetfillcolor{currentfill}%
\pgfsetlinewidth{0.803000pt}%
\definecolor{currentstroke}{rgb}{0.000000,0.000000,0.000000}%
\pgfsetstrokecolor{currentstroke}%
\pgfsetdash{}{0pt}%
\pgfsys@defobject{currentmarker}{\pgfqpoint{-0.048611in}{0.000000in}}{\pgfqpoint{-0.000000in}{0.000000in}}{%
\pgfpathmoveto{\pgfqpoint{-0.000000in}{0.000000in}}%
\pgfpathlineto{\pgfqpoint{-0.048611in}{0.000000in}}%
\pgfusepath{stroke,fill}%
}%
\begin{pgfscope}%
\pgfsys@transformshift{0.754048in}{2.645145in}%
\pgfsys@useobject{currentmarker}{}%
\end{pgfscope}%
\end{pgfscope}%
\begin{pgfscope}%
\definecolor{textcolor}{rgb}{0.000000,0.000000,0.000000}%
\pgfsetstrokecolor{textcolor}%
\pgfsetfillcolor{textcolor}%
\pgftext[x=0.387917in, y=2.606590in, left, base]{\color{textcolor}\rmfamily\fontsize{8.000000}{9.600000}\selectfont \(\displaystyle {0.003}\)}%
\end{pgfscope}%
\begin{pgfscope}%
\pgfsetbuttcap%
\pgfsetroundjoin%
\definecolor{currentfill}{rgb}{0.000000,0.000000,0.000000}%
\pgfsetfillcolor{currentfill}%
\pgfsetlinewidth{0.803000pt}%
\definecolor{currentstroke}{rgb}{0.000000,0.000000,0.000000}%
\pgfsetstrokecolor{currentstroke}%
\pgfsetdash{}{0pt}%
\pgfsys@defobject{currentmarker}{\pgfqpoint{-0.048611in}{0.000000in}}{\pgfqpoint{-0.000000in}{0.000000in}}{%
\pgfpathmoveto{\pgfqpoint{-0.000000in}{0.000000in}}%
\pgfpathlineto{\pgfqpoint{-0.048611in}{0.000000in}}%
\pgfusepath{stroke,fill}%
}%
\begin{pgfscope}%
\pgfsys@transformshift{0.754048in}{2.794124in}%
\pgfsys@useobject{currentmarker}{}%
\end{pgfscope}%
\end{pgfscope}%
\begin{pgfscope}%
\definecolor{textcolor}{rgb}{0.000000,0.000000,0.000000}%
\pgfsetstrokecolor{textcolor}%
\pgfsetfillcolor{textcolor}%
\pgftext[x=0.387917in, y=2.755569in, left, base]{\color{textcolor}\rmfamily\fontsize{8.000000}{9.600000}\selectfont \(\displaystyle {0.006}\)}%
\end{pgfscope}%
\begin{pgfscope}%
\pgfsetbuttcap%
\pgfsetroundjoin%
\definecolor{currentfill}{rgb}{0.000000,0.000000,0.000000}%
\pgfsetfillcolor{currentfill}%
\pgfsetlinewidth{0.803000pt}%
\definecolor{currentstroke}{rgb}{0.000000,0.000000,0.000000}%
\pgfsetstrokecolor{currentstroke}%
\pgfsetdash{}{0pt}%
\pgfsys@defobject{currentmarker}{\pgfqpoint{-0.048611in}{0.000000in}}{\pgfqpoint{-0.000000in}{0.000000in}}{%
\pgfpathmoveto{\pgfqpoint{-0.000000in}{0.000000in}}%
\pgfpathlineto{\pgfqpoint{-0.048611in}{0.000000in}}%
\pgfusepath{stroke,fill}%
}%
\begin{pgfscope}%
\pgfsys@transformshift{0.754048in}{2.943103in}%
\pgfsys@useobject{currentmarker}{}%
\end{pgfscope}%
\end{pgfscope}%
\begin{pgfscope}%
\definecolor{textcolor}{rgb}{0.000000,0.000000,0.000000}%
\pgfsetstrokecolor{textcolor}%
\pgfsetfillcolor{textcolor}%
\pgftext[x=0.387917in, y=2.904548in, left, base]{\color{textcolor}\rmfamily\fontsize{8.000000}{9.600000}\selectfont \(\displaystyle {0.009}\)}%
\end{pgfscope}%
\begin{pgfscope}%
\pgfsetbuttcap%
\pgfsetroundjoin%
\definecolor{currentfill}{rgb}{0.000000,0.000000,0.000000}%
\pgfsetfillcolor{currentfill}%
\pgfsetlinewidth{0.803000pt}%
\definecolor{currentstroke}{rgb}{0.000000,0.000000,0.000000}%
\pgfsetstrokecolor{currentstroke}%
\pgfsetdash{}{0pt}%
\pgfsys@defobject{currentmarker}{\pgfqpoint{-0.048611in}{0.000000in}}{\pgfqpoint{-0.000000in}{0.000000in}}{%
\pgfpathmoveto{\pgfqpoint{-0.000000in}{0.000000in}}%
\pgfpathlineto{\pgfqpoint{-0.048611in}{0.000000in}}%
\pgfusepath{stroke,fill}%
}%
\begin{pgfscope}%
\pgfsys@transformshift{0.754048in}{3.092082in}%
\pgfsys@useobject{currentmarker}{}%
\end{pgfscope}%
\end{pgfscope}%
\begin{pgfscope}%
\definecolor{textcolor}{rgb}{0.000000,0.000000,0.000000}%
\pgfsetstrokecolor{textcolor}%
\pgfsetfillcolor{textcolor}%
\pgftext[x=0.387917in, y=3.053527in, left, base]{\color{textcolor}\rmfamily\fontsize{8.000000}{9.600000}\selectfont \(\displaystyle {0.012}\)}%
\end{pgfscope}%
\begin{pgfscope}%
\pgfsetbuttcap%
\pgfsetroundjoin%
\definecolor{currentfill}{rgb}{0.000000,0.000000,0.000000}%
\pgfsetfillcolor{currentfill}%
\pgfsetlinewidth{0.803000pt}%
\definecolor{currentstroke}{rgb}{0.000000,0.000000,0.000000}%
\pgfsetstrokecolor{currentstroke}%
\pgfsetdash{}{0pt}%
\pgfsys@defobject{currentmarker}{\pgfqpoint{-0.048611in}{0.000000in}}{\pgfqpoint{-0.000000in}{0.000000in}}{%
\pgfpathmoveto{\pgfqpoint{-0.000000in}{0.000000in}}%
\pgfpathlineto{\pgfqpoint{-0.048611in}{0.000000in}}%
\pgfusepath{stroke,fill}%
}%
\begin{pgfscope}%
\pgfsys@transformshift{0.754048in}{3.241061in}%
\pgfsys@useobject{currentmarker}{}%
\end{pgfscope}%
\end{pgfscope}%
\begin{pgfscope}%
\definecolor{textcolor}{rgb}{0.000000,0.000000,0.000000}%
\pgfsetstrokecolor{textcolor}%
\pgfsetfillcolor{textcolor}%
\pgftext[x=0.387917in, y=3.202506in, left, base]{\color{textcolor}\rmfamily\fontsize{8.000000}{9.600000}\selectfont \(\displaystyle {0.015}\)}%
\end{pgfscope}%
\begin{pgfscope}%
\pgfsetbuttcap%
\pgfsetroundjoin%
\definecolor{currentfill}{rgb}{0.000000,0.000000,0.000000}%
\pgfsetfillcolor{currentfill}%
\pgfsetlinewidth{0.803000pt}%
\definecolor{currentstroke}{rgb}{0.000000,0.000000,0.000000}%
\pgfsetstrokecolor{currentstroke}%
\pgfsetdash{}{0pt}%
\pgfsys@defobject{currentmarker}{\pgfqpoint{-0.048611in}{0.000000in}}{\pgfqpoint{-0.000000in}{0.000000in}}{%
\pgfpathmoveto{\pgfqpoint{-0.000000in}{0.000000in}}%
\pgfpathlineto{\pgfqpoint{-0.048611in}{0.000000in}}%
\pgfusepath{stroke,fill}%
}%
\begin{pgfscope}%
\pgfsys@transformshift{0.754048in}{3.390040in}%
\pgfsys@useobject{currentmarker}{}%
\end{pgfscope}%
\end{pgfscope}%
\begin{pgfscope}%
\definecolor{textcolor}{rgb}{0.000000,0.000000,0.000000}%
\pgfsetstrokecolor{textcolor}%
\pgfsetfillcolor{textcolor}%
\pgftext[x=0.387917in, y=3.351485in, left, base]{\color{textcolor}\rmfamily\fontsize{8.000000}{9.600000}\selectfont \(\displaystyle {0.018}\)}%
\end{pgfscope}%
\begin{pgfscope}%
\pgfsetbuttcap%
\pgfsetroundjoin%
\definecolor{currentfill}{rgb}{0.000000,0.000000,0.000000}%
\pgfsetfillcolor{currentfill}%
\pgfsetlinewidth{0.803000pt}%
\definecolor{currentstroke}{rgb}{0.000000,0.000000,0.000000}%
\pgfsetstrokecolor{currentstroke}%
\pgfsetdash{}{0pt}%
\pgfsys@defobject{currentmarker}{\pgfqpoint{-0.048611in}{0.000000in}}{\pgfqpoint{-0.000000in}{0.000000in}}{%
\pgfpathmoveto{\pgfqpoint{-0.000000in}{0.000000in}}%
\pgfpathlineto{\pgfqpoint{-0.048611in}{0.000000in}}%
\pgfusepath{stroke,fill}%
}%
\begin{pgfscope}%
\pgfsys@transformshift{0.754048in}{3.539019in}%
\pgfsys@useobject{currentmarker}{}%
\end{pgfscope}%
\end{pgfscope}%
\begin{pgfscope}%
\definecolor{textcolor}{rgb}{0.000000,0.000000,0.000000}%
\pgfsetstrokecolor{textcolor}%
\pgfsetfillcolor{textcolor}%
\pgftext[x=0.387917in, y=3.500463in, left, base]{\color{textcolor}\rmfamily\fontsize{8.000000}{9.600000}\selectfont \(\displaystyle {0.021}\)}%
\end{pgfscope}%
\begin{pgfscope}%
\pgfsetbuttcap%
\pgfsetroundjoin%
\definecolor{currentfill}{rgb}{0.000000,0.000000,0.000000}%
\pgfsetfillcolor{currentfill}%
\pgfsetlinewidth{0.803000pt}%
\definecolor{currentstroke}{rgb}{0.000000,0.000000,0.000000}%
\pgfsetstrokecolor{currentstroke}%
\pgfsetdash{}{0pt}%
\pgfsys@defobject{currentmarker}{\pgfqpoint{-0.048611in}{0.000000in}}{\pgfqpoint{-0.000000in}{0.000000in}}{%
\pgfpathmoveto{\pgfqpoint{-0.000000in}{0.000000in}}%
\pgfpathlineto{\pgfqpoint{-0.048611in}{0.000000in}}%
\pgfusepath{stroke,fill}%
}%
\begin{pgfscope}%
\pgfsys@transformshift{0.754048in}{3.687998in}%
\pgfsys@useobject{currentmarker}{}%
\end{pgfscope}%
\end{pgfscope}%
\begin{pgfscope}%
\definecolor{textcolor}{rgb}{0.000000,0.000000,0.000000}%
\pgfsetstrokecolor{textcolor}%
\pgfsetfillcolor{textcolor}%
\pgftext[x=0.387917in, y=3.649442in, left, base]{\color{textcolor}\rmfamily\fontsize{8.000000}{9.600000}\selectfont \(\displaystyle {0.024}\)}%
\end{pgfscope}%
\begin{pgfscope}%
\pgfsetbuttcap%
\pgfsetroundjoin%
\definecolor{currentfill}{rgb}{0.000000,0.000000,0.000000}%
\pgfsetfillcolor{currentfill}%
\pgfsetlinewidth{0.803000pt}%
\definecolor{currentstroke}{rgb}{0.000000,0.000000,0.000000}%
\pgfsetstrokecolor{currentstroke}%
\pgfsetdash{}{0pt}%
\pgfsys@defobject{currentmarker}{\pgfqpoint{-0.048611in}{0.000000in}}{\pgfqpoint{-0.000000in}{0.000000in}}{%
\pgfpathmoveto{\pgfqpoint{-0.000000in}{0.000000in}}%
\pgfpathlineto{\pgfqpoint{-0.048611in}{0.000000in}}%
\pgfusepath{stroke,fill}%
}%
\begin{pgfscope}%
\pgfsys@transformshift{0.754048in}{3.836977in}%
\pgfsys@useobject{currentmarker}{}%
\end{pgfscope}%
\end{pgfscope}%
\begin{pgfscope}%
\definecolor{textcolor}{rgb}{0.000000,0.000000,0.000000}%
\pgfsetstrokecolor{textcolor}%
\pgfsetfillcolor{textcolor}%
\pgftext[x=0.387917in, y=3.798421in, left, base]{\color{textcolor}\rmfamily\fontsize{8.000000}{9.600000}\selectfont \(\displaystyle {0.027}\)}%
\end{pgfscope}%
\begin{pgfscope}%
\definecolor{textcolor}{rgb}{0.000000,0.000000,0.000000}%
\pgfsetstrokecolor{textcolor}%
\pgfsetfillcolor{textcolor}%
\pgftext[x=0.332362in,y=3.204903in,,bottom,rotate=90.000000]{\color{textcolor}\rmfamily\fontsize{10.000000}{12.000000}\selectfont Density}%
\end{pgfscope}%
\begin{pgfscope}%
\pgfpathrectangle{\pgfqpoint{0.754048in}{2.496167in}}{\pgfqpoint{2.301066in}{1.417472in}}%
\pgfusepath{clip}%
\pgfsetbuttcap%
\pgfsetmiterjoin%
\pgfsetlinewidth{1.003750pt}%
\definecolor{currentstroke}{rgb}{0.313725,0.317647,0.309804}%
\pgfsetstrokecolor{currentstroke}%
\pgfsetdash{}{0pt}%
\pgfpathmoveto{\pgfqpoint{0.897278in}{2.496167in}}%
\pgfpathlineto{\pgfqpoint{0.897278in}{2.496605in}}%
\pgfpathlineto{\pgfqpoint{1.086982in}{2.497043in}}%
\pgfpathlineto{\pgfqpoint{1.086982in}{2.497919in}}%
\pgfpathlineto{\pgfqpoint{1.100533in}{2.497043in}}%
\pgfpathlineto{\pgfqpoint{1.100533in}{2.496605in}}%
\pgfpathlineto{\pgfqpoint{1.127633in}{2.497043in}}%
\pgfpathlineto{\pgfqpoint{1.127633in}{2.498357in}}%
\pgfpathlineto{\pgfqpoint{1.161509in}{2.497919in}}%
\pgfpathlineto{\pgfqpoint{1.161509in}{2.497043in}}%
\pgfpathlineto{\pgfqpoint{1.168284in}{2.497043in}}%
\pgfpathlineto{\pgfqpoint{1.168284in}{2.498795in}}%
\pgfpathlineto{\pgfqpoint{1.195385in}{2.498357in}}%
\pgfpathlineto{\pgfqpoint{1.195385in}{2.500109in}}%
\pgfpathlineto{\pgfqpoint{1.222485in}{2.500985in}}%
\pgfpathlineto{\pgfqpoint{1.222485in}{2.502299in}}%
\pgfpathlineto{\pgfqpoint{1.229260in}{2.502299in}}%
\pgfpathlineto{\pgfqpoint{1.229260in}{2.499233in}}%
\pgfpathlineto{\pgfqpoint{1.236035in}{2.499233in}}%
\pgfpathlineto{\pgfqpoint{1.236035in}{2.502299in}}%
\pgfpathlineto{\pgfqpoint{1.242811in}{2.502299in}}%
\pgfpathlineto{\pgfqpoint{1.242811in}{2.504489in}}%
\pgfpathlineto{\pgfqpoint{1.256361in}{2.504051in}}%
\pgfpathlineto{\pgfqpoint{1.256361in}{2.501861in}}%
\pgfpathlineto{\pgfqpoint{1.263136in}{2.501861in}}%
\pgfpathlineto{\pgfqpoint{1.263136in}{2.503175in}}%
\pgfpathlineto{\pgfqpoint{1.283461in}{2.502737in}}%
\pgfpathlineto{\pgfqpoint{1.283461in}{2.504489in}}%
\pgfpathlineto{\pgfqpoint{1.290237in}{2.504489in}}%
\pgfpathlineto{\pgfqpoint{1.290237in}{2.506679in}}%
\pgfpathlineto{\pgfqpoint{1.297012in}{2.506679in}}%
\pgfpathlineto{\pgfqpoint{1.297012in}{2.504051in}}%
\pgfpathlineto{\pgfqpoint{1.303787in}{2.504051in}}%
\pgfpathlineto{\pgfqpoint{1.303787in}{2.508869in}}%
\pgfpathlineto{\pgfqpoint{1.310562in}{2.508869in}}%
\pgfpathlineto{\pgfqpoint{1.310562in}{2.507117in}}%
\pgfpathlineto{\pgfqpoint{1.324112in}{2.507993in}}%
\pgfpathlineto{\pgfqpoint{1.324112in}{2.504927in}}%
\pgfpathlineto{\pgfqpoint{1.330887in}{2.504927in}}%
\pgfpathlineto{\pgfqpoint{1.330887in}{2.514125in}}%
\pgfpathlineto{\pgfqpoint{1.344438in}{2.514125in}}%
\pgfpathlineto{\pgfqpoint{1.344438in}{2.516315in}}%
\pgfpathlineto{\pgfqpoint{1.364763in}{2.517191in}}%
\pgfpathlineto{\pgfqpoint{1.364763in}{2.518068in}}%
\pgfpathlineto{\pgfqpoint{1.385089in}{2.518506in}}%
\pgfpathlineto{\pgfqpoint{1.385089in}{2.520696in}}%
\pgfpathlineto{\pgfqpoint{1.398639in}{2.521572in}}%
\pgfpathlineto{\pgfqpoint{1.398639in}{2.522010in}}%
\pgfpathlineto{\pgfqpoint{1.412189in}{2.522448in}}%
\pgfpathlineto{\pgfqpoint{1.412189in}{2.526828in}}%
\pgfpathlineto{\pgfqpoint{1.418964in}{2.526828in}}%
\pgfpathlineto{\pgfqpoint{1.418964in}{2.523762in}}%
\pgfpathlineto{\pgfqpoint{1.425739in}{2.523762in}}%
\pgfpathlineto{\pgfqpoint{1.425739in}{2.532522in}}%
\pgfpathlineto{\pgfqpoint{1.432515in}{2.532522in}}%
\pgfpathlineto{\pgfqpoint{1.432515in}{2.528142in}}%
\pgfpathlineto{\pgfqpoint{1.439290in}{2.528142in}}%
\pgfpathlineto{\pgfqpoint{1.439290in}{2.531208in}}%
\pgfpathlineto{\pgfqpoint{1.446065in}{2.531208in}}%
\pgfpathlineto{\pgfqpoint{1.446065in}{2.532960in}}%
\pgfpathlineto{\pgfqpoint{1.452840in}{2.532960in}}%
\pgfpathlineto{\pgfqpoint{1.452840in}{2.531646in}}%
\pgfpathlineto{\pgfqpoint{1.459615in}{2.531646in}}%
\pgfpathlineto{\pgfqpoint{1.459615in}{2.550043in}}%
\pgfpathlineto{\pgfqpoint{1.473165in}{2.550919in}}%
\pgfpathlineto{\pgfqpoint{1.473165in}{2.540406in}}%
\pgfpathlineto{\pgfqpoint{1.479941in}{2.540406in}}%
\pgfpathlineto{\pgfqpoint{1.479941in}{2.550919in}}%
\pgfpathlineto{\pgfqpoint{1.486716in}{2.550919in}}%
\pgfpathlineto{\pgfqpoint{1.486716in}{2.564936in}}%
\pgfpathlineto{\pgfqpoint{1.493491in}{2.564936in}}%
\pgfpathlineto{\pgfqpoint{1.493491in}{2.553985in}}%
\pgfpathlineto{\pgfqpoint{1.500266in}{2.553985in}}%
\pgfpathlineto{\pgfqpoint{1.500266in}{2.560993in}}%
\pgfpathlineto{\pgfqpoint{1.507041in}{2.560993in}}%
\pgfpathlineto{\pgfqpoint{1.507041in}{2.551795in}}%
\pgfpathlineto{\pgfqpoint{1.513816in}{2.551795in}}%
\pgfpathlineto{\pgfqpoint{1.513816in}{2.569754in}}%
\pgfpathlineto{\pgfqpoint{1.520591in}{2.569754in}}%
\pgfpathlineto{\pgfqpoint{1.520591in}{2.568002in}}%
\pgfpathlineto{\pgfqpoint{1.527367in}{2.568002in}}%
\pgfpathlineto{\pgfqpoint{1.527367in}{2.576762in}}%
\pgfpathlineto{\pgfqpoint{1.534142in}{2.576762in}}%
\pgfpathlineto{\pgfqpoint{1.534142in}{2.583332in}}%
\pgfpathlineto{\pgfqpoint{1.540917in}{2.583332in}}%
\pgfpathlineto{\pgfqpoint{1.540917in}{2.587712in}}%
\pgfpathlineto{\pgfqpoint{1.547692in}{2.587712in}}%
\pgfpathlineto{\pgfqpoint{1.547692in}{2.592531in}}%
\pgfpathlineto{\pgfqpoint{1.554467in}{2.592531in}}%
\pgfpathlineto{\pgfqpoint{1.554467in}{2.606109in}}%
\pgfpathlineto{\pgfqpoint{1.561242in}{2.606109in}}%
\pgfpathlineto{\pgfqpoint{1.561242in}{2.600415in}}%
\pgfpathlineto{\pgfqpoint{1.568017in}{2.600415in}}%
\pgfpathlineto{\pgfqpoint{1.568017in}{2.606985in}}%
\pgfpathlineto{\pgfqpoint{1.574793in}{2.606985in}}%
\pgfpathlineto{\pgfqpoint{1.574793in}{2.609175in}}%
\pgfpathlineto{\pgfqpoint{1.588343in}{2.609613in}}%
\pgfpathlineto{\pgfqpoint{1.588343in}{2.626696in}}%
\pgfpathlineto{\pgfqpoint{1.595118in}{2.626696in}}%
\pgfpathlineto{\pgfqpoint{1.595118in}{2.640713in}}%
\pgfpathlineto{\pgfqpoint{1.601893in}{2.640713in}}%
\pgfpathlineto{\pgfqpoint{1.601893in}{2.637209in}}%
\pgfpathlineto{\pgfqpoint{1.608668in}{2.637209in}}%
\pgfpathlineto{\pgfqpoint{1.608668in}{2.645093in}}%
\pgfpathlineto{\pgfqpoint{1.622219in}{2.644217in}}%
\pgfpathlineto{\pgfqpoint{1.622219in}{2.656481in}}%
\pgfpathlineto{\pgfqpoint{1.628994in}{2.656481in}}%
\pgfpathlineto{\pgfqpoint{1.628994in}{2.678382in}}%
\pgfpathlineto{\pgfqpoint{1.635769in}{2.678382in}}%
\pgfpathlineto{\pgfqpoint{1.635769in}{2.688457in}}%
\pgfpathlineto{\pgfqpoint{1.642544in}{2.688457in}}%
\pgfpathlineto{\pgfqpoint{1.642544in}{2.702473in}}%
\pgfpathlineto{\pgfqpoint{1.649319in}{2.702473in}}%
\pgfpathlineto{\pgfqpoint{1.649319in}{2.722622in}}%
\pgfpathlineto{\pgfqpoint{1.656094in}{2.722622in}}%
\pgfpathlineto{\pgfqpoint{1.656094in}{2.708606in}}%
\pgfpathlineto{\pgfqpoint{1.662870in}{2.708606in}}%
\pgfpathlineto{\pgfqpoint{1.662870in}{2.736201in}}%
\pgfpathlineto{\pgfqpoint{1.669645in}{2.736201in}}%
\pgfpathlineto{\pgfqpoint{1.669645in}{2.751969in}}%
\pgfpathlineto{\pgfqpoint{1.676420in}{2.751969in}}%
\pgfpathlineto{\pgfqpoint{1.676420in}{2.747589in}}%
\pgfpathlineto{\pgfqpoint{1.683195in}{2.747589in}}%
\pgfpathlineto{\pgfqpoint{1.683195in}{2.769928in}}%
\pgfpathlineto{\pgfqpoint{1.689970in}{2.769928in}}%
\pgfpathlineto{\pgfqpoint{1.689970in}{2.764672in}}%
\pgfpathlineto{\pgfqpoint{1.696745in}{2.764672in}}%
\pgfpathlineto{\pgfqpoint{1.696745in}{2.814168in}}%
\pgfpathlineto{\pgfqpoint{1.703520in}{2.814168in}}%
\pgfpathlineto{\pgfqpoint{1.703520in}{2.829499in}}%
\pgfpathlineto{\pgfqpoint{1.710296in}{2.829499in}}%
\pgfpathlineto{\pgfqpoint{1.710296in}{2.857094in}}%
\pgfpathlineto{\pgfqpoint{1.717071in}{2.857094in}}%
\pgfpathlineto{\pgfqpoint{1.717071in}{2.836069in}}%
\pgfpathlineto{\pgfqpoint{1.723846in}{2.836069in}}%
\pgfpathlineto{\pgfqpoint{1.723846in}{2.867168in}}%
\pgfpathlineto{\pgfqpoint{1.730621in}{2.867168in}}%
\pgfpathlineto{\pgfqpoint{1.730621in}{2.927177in}}%
\pgfpathlineto{\pgfqpoint{1.737396in}{2.927177in}}%
\pgfpathlineto{\pgfqpoint{1.737396in}{2.924111in}}%
\pgfpathlineto{\pgfqpoint{1.744171in}{2.924111in}}%
\pgfpathlineto{\pgfqpoint{1.744171in}{2.991566in}}%
\pgfpathlineto{\pgfqpoint{1.750946in}{2.991566in}}%
\pgfpathlineto{\pgfqpoint{1.750946in}{2.978425in}}%
\pgfpathlineto{\pgfqpoint{1.757722in}{2.978425in}}%
\pgfpathlineto{\pgfqpoint{1.757722in}{2.994632in}}%
\pgfpathlineto{\pgfqpoint{1.764497in}{2.994632in}}%
\pgfpathlineto{\pgfqpoint{1.764497in}{3.019599in}}%
\pgfpathlineto{\pgfqpoint{1.771272in}{3.019599in}}%
\pgfpathlineto{\pgfqpoint{1.771272in}{3.101508in}}%
\pgfpathlineto{\pgfqpoint{1.778047in}{3.101508in}}%
\pgfpathlineto{\pgfqpoint{1.778047in}{3.096252in}}%
\pgfpathlineto{\pgfqpoint{1.784822in}{3.096252in}}%
\pgfpathlineto{\pgfqpoint{1.784822in}{3.141368in}}%
\pgfpathlineto{\pgfqpoint{1.791597in}{3.141368in}}%
\pgfpathlineto{\pgfqpoint{1.791597in}{3.182104in}}%
\pgfpathlineto{\pgfqpoint{1.798372in}{3.182104in}}%
\pgfpathlineto{\pgfqpoint{1.798372in}{3.223716in}}%
\pgfpathlineto{\pgfqpoint{1.805148in}{3.223716in}}%
\pgfpathlineto{\pgfqpoint{1.805148in}{3.261385in}}%
\pgfpathlineto{\pgfqpoint{1.811923in}{3.261385in}}%
\pgfpathlineto{\pgfqpoint{1.811923in}{3.301245in}}%
\pgfpathlineto{\pgfqpoint{1.818698in}{3.301245in}}%
\pgfpathlineto{\pgfqpoint{1.818698in}{3.323584in}}%
\pgfpathlineto{\pgfqpoint{1.825473in}{3.323584in}}%
\pgfpathlineto{\pgfqpoint{1.825473in}{3.406807in}}%
\pgfpathlineto{\pgfqpoint{1.832248in}{3.406807in}}%
\pgfpathlineto{\pgfqpoint{1.832248in}{3.446229in}}%
\pgfpathlineto{\pgfqpoint{1.839023in}{3.446229in}}%
\pgfpathlineto{\pgfqpoint{1.839023in}{3.527701in}}%
\pgfpathlineto{\pgfqpoint{1.845798in}{3.527701in}}%
\pgfpathlineto{\pgfqpoint{1.845798in}{3.544345in}}%
\pgfpathlineto{\pgfqpoint{1.852574in}{3.544345in}}%
\pgfpathlineto{\pgfqpoint{1.852574in}{3.634577in}}%
\pgfpathlineto{\pgfqpoint{1.859349in}{3.634577in}}%
\pgfpathlineto{\pgfqpoint{1.859349in}{3.665238in}}%
\pgfpathlineto{\pgfqpoint{1.866124in}{3.665238in}}%
\pgfpathlineto{\pgfqpoint{1.866124in}{3.731379in}}%
\pgfpathlineto{\pgfqpoint{1.872899in}{3.731379in}}%
\pgfpathlineto{\pgfqpoint{1.872899in}{3.736197in}}%
\pgfpathlineto{\pgfqpoint{1.879674in}{3.736197in}}%
\pgfpathlineto{\pgfqpoint{1.879674in}{3.781313in}}%
\pgfpathlineto{\pgfqpoint{1.886449in}{3.781313in}}%
\pgfpathlineto{\pgfqpoint{1.886449in}{3.814165in}}%
\pgfpathlineto{\pgfqpoint{1.893224in}{3.814165in}}%
\pgfpathlineto{\pgfqpoint{1.893224in}{3.805404in}}%
\pgfpathlineto{\pgfqpoint{1.900000in}{3.805404in}}%
\pgfpathlineto{\pgfqpoint{1.900000in}{3.846140in}}%
\pgfpathlineto{\pgfqpoint{1.906775in}{3.846140in}}%
\pgfpathlineto{\pgfqpoint{1.906775in}{3.787884in}}%
\pgfpathlineto{\pgfqpoint{1.913550in}{3.787884in}}%
\pgfpathlineto{\pgfqpoint{1.913550in}{3.826867in}}%
\pgfpathlineto{\pgfqpoint{1.920325in}{3.826867in}}%
\pgfpathlineto{\pgfqpoint{1.920325in}{3.788322in}}%
\pgfpathlineto{\pgfqpoint{1.927100in}{3.788322in}}%
\pgfpathlineto{\pgfqpoint{1.927100in}{3.726999in}}%
\pgfpathlineto{\pgfqpoint{1.933875in}{3.726999in}}%
\pgfpathlineto{\pgfqpoint{1.933875in}{3.687139in}}%
\pgfpathlineto{\pgfqpoint{1.940650in}{3.687139in}}%
\pgfpathlineto{\pgfqpoint{1.940650in}{3.616180in}}%
\pgfpathlineto{\pgfqpoint{1.947426in}{3.616180in}}%
\pgfpathlineto{\pgfqpoint{1.947426in}{3.539089in}}%
\pgfpathlineto{\pgfqpoint{1.954201in}{3.539089in}}%
\pgfpathlineto{\pgfqpoint{1.954201in}{3.513684in}}%
\pgfpathlineto{\pgfqpoint{1.960976in}{3.513684in}}%
\pgfpathlineto{\pgfqpoint{1.960976in}{3.465064in}}%
\pgfpathlineto{\pgfqpoint{1.967751in}{3.465064in}}%
\pgfpathlineto{\pgfqpoint{1.967751in}{3.396733in}}%
\pgfpathlineto{\pgfqpoint{1.974526in}{3.396733in}}%
\pgfpathlineto{\pgfqpoint{1.974526in}{3.324022in}}%
\pgfpathlineto{\pgfqpoint{1.981301in}{3.324022in}}%
\pgfpathlineto{\pgfqpoint{1.981301in}{3.288104in}}%
\pgfpathlineto{\pgfqpoint{1.988076in}{3.288104in}}%
\pgfpathlineto{\pgfqpoint{1.988076in}{3.245617in}}%
\pgfpathlineto{\pgfqpoint{1.994852in}{3.245617in}}%
\pgfpathlineto{\pgfqpoint{1.994852in}{3.225468in}}%
\pgfpathlineto{\pgfqpoint{2.001627in}{3.225468in}}%
\pgfpathlineto{\pgfqpoint{2.001627in}{3.167649in}}%
\pgfpathlineto{\pgfqpoint{2.008402in}{3.167649in}}%
\pgfpathlineto{\pgfqpoint{2.008402in}{3.116401in}}%
\pgfpathlineto{\pgfqpoint{2.015177in}{3.116401in}}%
\pgfpathlineto{\pgfqpoint{2.015177in}{3.093624in}}%
\pgfpathlineto{\pgfqpoint{2.021952in}{3.093624in}}%
\pgfpathlineto{\pgfqpoint{2.021952in}{3.041938in}}%
\pgfpathlineto{\pgfqpoint{2.028727in}{3.041938in}}%
\pgfpathlineto{\pgfqpoint{2.028727in}{3.008648in}}%
\pgfpathlineto{\pgfqpoint{2.035502in}{3.008648in}}%
\pgfpathlineto{\pgfqpoint{2.035502in}{2.974045in}}%
\pgfpathlineto{\pgfqpoint{2.042278in}{2.974045in}}%
\pgfpathlineto{\pgfqpoint{2.042278in}{2.955648in}}%
\pgfpathlineto{\pgfqpoint{2.049053in}{2.955648in}}%
\pgfpathlineto{\pgfqpoint{2.049053in}{2.931995in}}%
\pgfpathlineto{\pgfqpoint{2.055828in}{2.931995in}}%
\pgfpathlineto{\pgfqpoint{2.055828in}{2.907028in}}%
\pgfpathlineto{\pgfqpoint{2.062603in}{2.907028in}}%
\pgfpathlineto{\pgfqpoint{2.062603in}{2.889069in}}%
\pgfpathlineto{\pgfqpoint{2.069378in}{2.889069in}}%
\pgfpathlineto{\pgfqpoint{2.069378in}{2.882499in}}%
\pgfpathlineto{\pgfqpoint{2.076153in}{2.882499in}}%
\pgfpathlineto{\pgfqpoint{2.076153in}{2.845267in}}%
\pgfpathlineto{\pgfqpoint{2.082928in}{2.845267in}}%
\pgfpathlineto{\pgfqpoint{2.082928in}{2.826433in}}%
\pgfpathlineto{\pgfqpoint{2.089704in}{2.826433in}}%
\pgfpathlineto{\pgfqpoint{2.089704in}{2.798837in}}%
\pgfpathlineto{\pgfqpoint{2.096479in}{2.798837in}}%
\pgfpathlineto{\pgfqpoint{2.096479in}{2.790515in}}%
\pgfpathlineto{\pgfqpoint{2.103254in}{2.790515in}}%
\pgfpathlineto{\pgfqpoint{2.103254in}{2.756350in}}%
\pgfpathlineto{\pgfqpoint{2.110029in}{2.756350in}}%
\pgfpathlineto{\pgfqpoint{2.110029in}{2.785259in}}%
\pgfpathlineto{\pgfqpoint{2.116804in}{2.785259in}}%
\pgfpathlineto{\pgfqpoint{2.116804in}{2.743647in}}%
\pgfpathlineto{\pgfqpoint{2.123579in}{2.743647in}}%
\pgfpathlineto{\pgfqpoint{2.123579in}{2.771680in}}%
\pgfpathlineto{\pgfqpoint{2.130354in}{2.771680in}}%
\pgfpathlineto{\pgfqpoint{2.130354in}{2.724374in}}%
\pgfpathlineto{\pgfqpoint{2.137130in}{2.724374in}}%
\pgfpathlineto{\pgfqpoint{2.137130in}{2.709920in}}%
\pgfpathlineto{\pgfqpoint{2.143905in}{2.709920in}}%
\pgfpathlineto{\pgfqpoint{2.143905in}{2.691085in}}%
\pgfpathlineto{\pgfqpoint{2.150680in}{2.691085in}}%
\pgfpathlineto{\pgfqpoint{2.150680in}{2.703787in}}%
\pgfpathlineto{\pgfqpoint{2.157455in}{2.703787in}}%
\pgfpathlineto{\pgfqpoint{2.157455in}{2.664804in}}%
\pgfpathlineto{\pgfqpoint{2.164230in}{2.664804in}}%
\pgfpathlineto{\pgfqpoint{2.164230in}{2.677068in}}%
\pgfpathlineto{\pgfqpoint{2.171005in}{2.677068in}}%
\pgfpathlineto{\pgfqpoint{2.171005in}{2.659548in}}%
\pgfpathlineto{\pgfqpoint{2.177780in}{2.659548in}}%
\pgfpathlineto{\pgfqpoint{2.177780in}{2.661738in}}%
\pgfpathlineto{\pgfqpoint{2.184556in}{2.661738in}}%
\pgfpathlineto{\pgfqpoint{2.184556in}{2.638523in}}%
\pgfpathlineto{\pgfqpoint{2.191331in}{2.638523in}}%
\pgfpathlineto{\pgfqpoint{2.191331in}{2.636771in}}%
\pgfpathlineto{\pgfqpoint{2.198106in}{2.636771in}}%
\pgfpathlineto{\pgfqpoint{2.198106in}{2.624068in}}%
\pgfpathlineto{\pgfqpoint{2.211656in}{2.623630in}}%
\pgfpathlineto{\pgfqpoint{2.211656in}{2.609175in}}%
\pgfpathlineto{\pgfqpoint{2.218431in}{2.609175in}}%
\pgfpathlineto{\pgfqpoint{2.218431in}{2.600853in}}%
\pgfpathlineto{\pgfqpoint{2.225206in}{2.600853in}}%
\pgfpathlineto{\pgfqpoint{2.225206in}{2.596473in}}%
\pgfpathlineto{\pgfqpoint{2.231982in}{2.596473in}}%
\pgfpathlineto{\pgfqpoint{2.231982in}{2.599977in}}%
\pgfpathlineto{\pgfqpoint{2.238757in}{2.599977in}}%
\pgfpathlineto{\pgfqpoint{2.238757in}{2.590341in}}%
\pgfpathlineto{\pgfqpoint{2.245532in}{2.590341in}}%
\pgfpathlineto{\pgfqpoint{2.245532in}{2.581142in}}%
\pgfpathlineto{\pgfqpoint{2.252307in}{2.581142in}}%
\pgfpathlineto{\pgfqpoint{2.252307in}{2.572382in}}%
\pgfpathlineto{\pgfqpoint{2.259082in}{2.572382in}}%
\pgfpathlineto{\pgfqpoint{2.259082in}{2.577638in}}%
\pgfpathlineto{\pgfqpoint{2.265857in}{2.577638in}}%
\pgfpathlineto{\pgfqpoint{2.265857in}{2.569754in}}%
\pgfpathlineto{\pgfqpoint{2.272632in}{2.569754in}}%
\pgfpathlineto{\pgfqpoint{2.272632in}{2.579390in}}%
\pgfpathlineto{\pgfqpoint{2.279408in}{2.579390in}}%
\pgfpathlineto{\pgfqpoint{2.279408in}{2.566250in}}%
\pgfpathlineto{\pgfqpoint{2.286183in}{2.566250in}}%
\pgfpathlineto{\pgfqpoint{2.286183in}{2.562307in}}%
\pgfpathlineto{\pgfqpoint{2.299733in}{2.561431in}}%
\pgfpathlineto{\pgfqpoint{2.299733in}{2.555299in}}%
\pgfpathlineto{\pgfqpoint{2.306508in}{2.555299in}}%
\pgfpathlineto{\pgfqpoint{2.306508in}{2.549167in}}%
\pgfpathlineto{\pgfqpoint{2.313283in}{2.549167in}}%
\pgfpathlineto{\pgfqpoint{2.313283in}{2.552233in}}%
\pgfpathlineto{\pgfqpoint{2.320058in}{2.552233in}}%
\pgfpathlineto{\pgfqpoint{2.320058in}{2.544349in}}%
\pgfpathlineto{\pgfqpoint{2.326834in}{2.544349in}}%
\pgfpathlineto{\pgfqpoint{2.326834in}{2.536026in}}%
\pgfpathlineto{\pgfqpoint{2.340384in}{2.535150in}}%
\pgfpathlineto{\pgfqpoint{2.340384in}{2.534274in}}%
\pgfpathlineto{\pgfqpoint{2.347159in}{2.534274in}}%
\pgfpathlineto{\pgfqpoint{2.347159in}{2.529018in}}%
\pgfpathlineto{\pgfqpoint{2.360709in}{2.529894in}}%
\pgfpathlineto{\pgfqpoint{2.360709in}{2.530332in}}%
\pgfpathlineto{\pgfqpoint{2.367484in}{2.530332in}}%
\pgfpathlineto{\pgfqpoint{2.367484in}{2.525076in}}%
\pgfpathlineto{\pgfqpoint{2.374260in}{2.525076in}}%
\pgfpathlineto{\pgfqpoint{2.374260in}{2.528142in}}%
\pgfpathlineto{\pgfqpoint{2.381035in}{2.528142in}}%
\pgfpathlineto{\pgfqpoint{2.381035in}{2.519382in}}%
\pgfpathlineto{\pgfqpoint{2.387810in}{2.519382in}}%
\pgfpathlineto{\pgfqpoint{2.387810in}{2.523324in}}%
\pgfpathlineto{\pgfqpoint{2.394585in}{2.523324in}}%
\pgfpathlineto{\pgfqpoint{2.394585in}{2.517629in}}%
\pgfpathlineto{\pgfqpoint{2.401360in}{2.517629in}}%
\pgfpathlineto{\pgfqpoint{2.401360in}{2.516315in}}%
\pgfpathlineto{\pgfqpoint{2.414910in}{2.517191in}}%
\pgfpathlineto{\pgfqpoint{2.414910in}{2.522010in}}%
\pgfpathlineto{\pgfqpoint{2.421686in}{2.522010in}}%
\pgfpathlineto{\pgfqpoint{2.421686in}{2.514125in}}%
\pgfpathlineto{\pgfqpoint{2.435236in}{2.515001in}}%
\pgfpathlineto{\pgfqpoint{2.435236in}{2.512811in}}%
\pgfpathlineto{\pgfqpoint{2.442011in}{2.512811in}}%
\pgfpathlineto{\pgfqpoint{2.442011in}{2.511497in}}%
\pgfpathlineto{\pgfqpoint{2.455561in}{2.511497in}}%
\pgfpathlineto{\pgfqpoint{2.455561in}{2.505365in}}%
\pgfpathlineto{\pgfqpoint{2.462336in}{2.505365in}}%
\pgfpathlineto{\pgfqpoint{2.462336in}{2.507993in}}%
\pgfpathlineto{\pgfqpoint{2.482662in}{2.507993in}}%
\pgfpathlineto{\pgfqpoint{2.482662in}{2.503613in}}%
\pgfpathlineto{\pgfqpoint{2.489437in}{2.503613in}}%
\pgfpathlineto{\pgfqpoint{2.489437in}{2.508869in}}%
\pgfpathlineto{\pgfqpoint{2.496212in}{2.508869in}}%
\pgfpathlineto{\pgfqpoint{2.496212in}{2.502737in}}%
\pgfpathlineto{\pgfqpoint{2.502987in}{2.502737in}}%
\pgfpathlineto{\pgfqpoint{2.502987in}{2.504489in}}%
\pgfpathlineto{\pgfqpoint{2.509762in}{2.504489in}}%
\pgfpathlineto{\pgfqpoint{2.509762in}{2.502299in}}%
\pgfpathlineto{\pgfqpoint{2.516538in}{2.502299in}}%
\pgfpathlineto{\pgfqpoint{2.516538in}{2.503613in}}%
\pgfpathlineto{\pgfqpoint{2.523313in}{2.503613in}}%
\pgfpathlineto{\pgfqpoint{2.523313in}{2.500985in}}%
\pgfpathlineto{\pgfqpoint{2.530088in}{2.500985in}}%
\pgfpathlineto{\pgfqpoint{2.530088in}{2.504489in}}%
\pgfpathlineto{\pgfqpoint{2.543638in}{2.503613in}}%
\pgfpathlineto{\pgfqpoint{2.543638in}{2.500109in}}%
\pgfpathlineto{\pgfqpoint{2.550413in}{2.500109in}}%
\pgfpathlineto{\pgfqpoint{2.550413in}{2.501423in}}%
\pgfpathlineto{\pgfqpoint{2.563964in}{2.500547in}}%
\pgfpathlineto{\pgfqpoint{2.563964in}{2.499233in}}%
\pgfpathlineto{\pgfqpoint{2.577514in}{2.500109in}}%
\pgfpathlineto{\pgfqpoint{2.577514in}{2.501423in}}%
\pgfpathlineto{\pgfqpoint{2.591064in}{2.500547in}}%
\pgfpathlineto{\pgfqpoint{2.591064in}{2.498795in}}%
\pgfpathlineto{\pgfqpoint{2.611390in}{2.498357in}}%
\pgfpathlineto{\pgfqpoint{2.611390in}{2.496605in}}%
\pgfpathlineto{\pgfqpoint{2.618165in}{2.496605in}}%
\pgfpathlineto{\pgfqpoint{2.618165in}{2.498357in}}%
\pgfpathlineto{\pgfqpoint{2.624940in}{2.498357in}}%
\pgfpathlineto{\pgfqpoint{2.624940in}{2.497043in}}%
\pgfpathlineto{\pgfqpoint{2.652040in}{2.497919in}}%
\pgfpathlineto{\pgfqpoint{2.652040in}{2.499233in}}%
\pgfpathlineto{\pgfqpoint{2.672366in}{2.498357in}}%
\pgfpathlineto{\pgfqpoint{2.672366in}{2.497043in}}%
\pgfpathlineto{\pgfqpoint{2.909496in}{2.496605in}}%
\pgfpathlineto{\pgfqpoint{2.909496in}{2.496167in}}%
\pgfusepath{stroke}%
\end{pgfscope}%
\begin{pgfscope}%
\pgfpathrectangle{\pgfqpoint{0.754048in}{2.496167in}}{\pgfqpoint{2.301066in}{1.417472in}}%
\pgfusepath{clip}%
\pgfsetbuttcap%
\pgfsetmiterjoin%
\pgfsetlinewidth{1.003750pt}%
\definecolor{currentstroke}{rgb}{0.949020,0.372549,0.360784}%
\pgfsetstrokecolor{currentstroke}%
\pgfsetdash{{1.000000pt}{1.650000pt}}{0.000000pt}%
\pgfpathmoveto{\pgfqpoint{0.897278in}{2.496167in}}%
\pgfpathlineto{\pgfqpoint{0.897278in}{2.497043in}}%
\pgfpathlineto{\pgfqpoint{0.924379in}{2.496605in}}%
\pgfpathlineto{\pgfqpoint{0.924379in}{2.498796in}}%
\pgfpathlineto{\pgfqpoint{0.944704in}{2.497920in}}%
\pgfpathlineto{\pgfqpoint{0.944704in}{2.497043in}}%
\pgfpathlineto{\pgfqpoint{0.958255in}{2.497043in}}%
\pgfpathlineto{\pgfqpoint{0.958255in}{2.498796in}}%
\pgfpathlineto{\pgfqpoint{0.965030in}{2.498796in}}%
\pgfpathlineto{\pgfqpoint{0.965030in}{2.497043in}}%
\pgfpathlineto{\pgfqpoint{0.971805in}{2.497043in}}%
\pgfpathlineto{\pgfqpoint{0.971805in}{2.498796in}}%
\pgfpathlineto{\pgfqpoint{1.005681in}{2.498358in}}%
\pgfpathlineto{\pgfqpoint{1.005681in}{2.500111in}}%
\pgfpathlineto{\pgfqpoint{1.012456in}{2.500111in}}%
\pgfpathlineto{\pgfqpoint{1.012456in}{2.498358in}}%
\pgfpathlineto{\pgfqpoint{1.026006in}{2.497920in}}%
\pgfpathlineto{\pgfqpoint{1.026006in}{2.499673in}}%
\pgfpathlineto{\pgfqpoint{1.039556in}{2.498796in}}%
\pgfpathlineto{\pgfqpoint{1.039556in}{2.497043in}}%
\pgfpathlineto{\pgfqpoint{1.046331in}{2.497043in}}%
\pgfpathlineto{\pgfqpoint{1.046331in}{2.500111in}}%
\pgfpathlineto{\pgfqpoint{1.053107in}{2.500111in}}%
\pgfpathlineto{\pgfqpoint{1.053107in}{2.498796in}}%
\pgfpathlineto{\pgfqpoint{1.066657in}{2.499235in}}%
\pgfpathlineto{\pgfqpoint{1.066657in}{2.500549in}}%
\pgfpathlineto{\pgfqpoint{1.086982in}{2.500111in}}%
\pgfpathlineto{\pgfqpoint{1.086982in}{2.499235in}}%
\pgfpathlineto{\pgfqpoint{1.093757in}{2.499235in}}%
\pgfpathlineto{\pgfqpoint{1.093757in}{2.501426in}}%
\pgfpathlineto{\pgfqpoint{1.100533in}{2.501426in}}%
\pgfpathlineto{\pgfqpoint{1.100533in}{2.499673in}}%
\pgfpathlineto{\pgfqpoint{1.107308in}{2.499673in}}%
\pgfpathlineto{\pgfqpoint{1.107308in}{2.502741in}}%
\pgfpathlineto{\pgfqpoint{1.114083in}{2.502741in}}%
\pgfpathlineto{\pgfqpoint{1.114083in}{2.500988in}}%
\pgfpathlineto{\pgfqpoint{1.127633in}{2.501426in}}%
\pgfpathlineto{\pgfqpoint{1.127633in}{2.503617in}}%
\pgfpathlineto{\pgfqpoint{1.134408in}{2.503617in}}%
\pgfpathlineto{\pgfqpoint{1.134408in}{2.505809in}}%
\pgfpathlineto{\pgfqpoint{1.147959in}{2.504932in}}%
\pgfpathlineto{\pgfqpoint{1.147959in}{2.504494in}}%
\pgfpathlineto{\pgfqpoint{1.181834in}{2.503617in}}%
\pgfpathlineto{\pgfqpoint{1.181834in}{2.506247in}}%
\pgfpathlineto{\pgfqpoint{1.188609in}{2.506247in}}%
\pgfpathlineto{\pgfqpoint{1.188609in}{2.503617in}}%
\pgfpathlineto{\pgfqpoint{1.202160in}{2.504494in}}%
\pgfpathlineto{\pgfqpoint{1.202160in}{2.510192in}}%
\pgfpathlineto{\pgfqpoint{1.208935in}{2.510192in}}%
\pgfpathlineto{\pgfqpoint{1.208935in}{2.505809in}}%
\pgfpathlineto{\pgfqpoint{1.229260in}{2.506685in}}%
\pgfpathlineto{\pgfqpoint{1.229260in}{2.510192in}}%
\pgfpathlineto{\pgfqpoint{1.242811in}{2.510630in}}%
\pgfpathlineto{\pgfqpoint{1.242811in}{2.513698in}}%
\pgfpathlineto{\pgfqpoint{1.256361in}{2.513260in}}%
\pgfpathlineto{\pgfqpoint{1.256361in}{2.509315in}}%
\pgfpathlineto{\pgfqpoint{1.263136in}{2.509315in}}%
\pgfpathlineto{\pgfqpoint{1.263136in}{2.519396in}}%
\pgfpathlineto{\pgfqpoint{1.269911in}{2.519396in}}%
\pgfpathlineto{\pgfqpoint{1.269911in}{2.512383in}}%
\pgfpathlineto{\pgfqpoint{1.276686in}{2.512383in}}%
\pgfpathlineto{\pgfqpoint{1.276686in}{2.517204in}}%
\pgfpathlineto{\pgfqpoint{1.290237in}{2.516328in}}%
\pgfpathlineto{\pgfqpoint{1.290237in}{2.524217in}}%
\pgfpathlineto{\pgfqpoint{1.303787in}{2.523778in}}%
\pgfpathlineto{\pgfqpoint{1.303787in}{2.522025in}}%
\pgfpathlineto{\pgfqpoint{1.310562in}{2.522025in}}%
\pgfpathlineto{\pgfqpoint{1.310562in}{2.515013in}}%
\pgfpathlineto{\pgfqpoint{1.317337in}{2.515013in}}%
\pgfpathlineto{\pgfqpoint{1.317337in}{2.516766in}}%
\pgfpathlineto{\pgfqpoint{1.324112in}{2.516766in}}%
\pgfpathlineto{\pgfqpoint{1.324112in}{2.523340in}}%
\pgfpathlineto{\pgfqpoint{1.330887in}{2.523340in}}%
\pgfpathlineto{\pgfqpoint{1.330887in}{2.521149in}}%
\pgfpathlineto{\pgfqpoint{1.337663in}{2.521149in}}%
\pgfpathlineto{\pgfqpoint{1.337663in}{2.528161in}}%
\pgfpathlineto{\pgfqpoint{1.344438in}{2.528161in}}%
\pgfpathlineto{\pgfqpoint{1.344438in}{2.526408in}}%
\pgfpathlineto{\pgfqpoint{1.351213in}{2.526408in}}%
\pgfpathlineto{\pgfqpoint{1.351213in}{2.528599in}}%
\pgfpathlineto{\pgfqpoint{1.357988in}{2.528599in}}%
\pgfpathlineto{\pgfqpoint{1.357988in}{2.531229in}}%
\pgfpathlineto{\pgfqpoint{1.364763in}{2.531229in}}%
\pgfpathlineto{\pgfqpoint{1.364763in}{2.539118in}}%
\pgfpathlineto{\pgfqpoint{1.371538in}{2.539118in}}%
\pgfpathlineto{\pgfqpoint{1.371538in}{2.529914in}}%
\pgfpathlineto{\pgfqpoint{1.378313in}{2.529914in}}%
\pgfpathlineto{\pgfqpoint{1.378313in}{2.535612in}}%
\pgfpathlineto{\pgfqpoint{1.385089in}{2.535612in}}%
\pgfpathlineto{\pgfqpoint{1.385089in}{2.529038in}}%
\pgfpathlineto{\pgfqpoint{1.391864in}{2.529038in}}%
\pgfpathlineto{\pgfqpoint{1.391864in}{2.534297in}}%
\pgfpathlineto{\pgfqpoint{1.405414in}{2.533859in}}%
\pgfpathlineto{\pgfqpoint{1.405414in}{2.541310in}}%
\pgfpathlineto{\pgfqpoint{1.412189in}{2.541310in}}%
\pgfpathlineto{\pgfqpoint{1.412189in}{2.535174in}}%
\pgfpathlineto{\pgfqpoint{1.418964in}{2.535174in}}%
\pgfpathlineto{\pgfqpoint{1.418964in}{2.549199in}}%
\pgfpathlineto{\pgfqpoint{1.425739in}{2.549199in}}%
\pgfpathlineto{\pgfqpoint{1.425739in}{2.553582in}}%
\pgfpathlineto{\pgfqpoint{1.432515in}{2.553582in}}%
\pgfpathlineto{\pgfqpoint{1.432515in}{2.544378in}}%
\pgfpathlineto{\pgfqpoint{1.439290in}{2.544378in}}%
\pgfpathlineto{\pgfqpoint{1.439290in}{2.550075in}}%
\pgfpathlineto{\pgfqpoint{1.452840in}{2.550075in}}%
\pgfpathlineto{\pgfqpoint{1.452840in}{2.556211in}}%
\pgfpathlineto{\pgfqpoint{1.459615in}{2.556211in}}%
\pgfpathlineto{\pgfqpoint{1.459615in}{2.558841in}}%
\pgfpathlineto{\pgfqpoint{1.466390in}{2.558841in}}%
\pgfpathlineto{\pgfqpoint{1.466390in}{2.572866in}}%
\pgfpathlineto{\pgfqpoint{1.473165in}{2.572866in}}%
\pgfpathlineto{\pgfqpoint{1.473165in}{2.565415in}}%
\pgfpathlineto{\pgfqpoint{1.479941in}{2.565415in}}%
\pgfpathlineto{\pgfqpoint{1.479941in}{2.571113in}}%
\pgfpathlineto{\pgfqpoint{1.486716in}{2.571113in}}%
\pgfpathlineto{\pgfqpoint{1.486716in}{2.582508in}}%
\pgfpathlineto{\pgfqpoint{1.500266in}{2.581632in}}%
\pgfpathlineto{\pgfqpoint{1.500266in}{2.590397in}}%
\pgfpathlineto{\pgfqpoint{1.507041in}{2.590397in}}%
\pgfpathlineto{\pgfqpoint{1.507041in}{2.582508in}}%
\pgfpathlineto{\pgfqpoint{1.513816in}{2.582508in}}%
\pgfpathlineto{\pgfqpoint{1.513816in}{2.587329in}}%
\pgfpathlineto{\pgfqpoint{1.520591in}{2.587329in}}%
\pgfpathlineto{\pgfqpoint{1.520591in}{2.591274in}}%
\pgfpathlineto{\pgfqpoint{1.527367in}{2.591274in}}%
\pgfpathlineto{\pgfqpoint{1.527367in}{2.586453in}}%
\pgfpathlineto{\pgfqpoint{1.534142in}{2.586453in}}%
\pgfpathlineto{\pgfqpoint{1.534142in}{2.598286in}}%
\pgfpathlineto{\pgfqpoint{1.540917in}{2.598286in}}%
\pgfpathlineto{\pgfqpoint{1.540917in}{2.600916in}}%
\pgfpathlineto{\pgfqpoint{1.547692in}{2.600916in}}%
\pgfpathlineto{\pgfqpoint{1.547692in}{2.596533in}}%
\pgfpathlineto{\pgfqpoint{1.554467in}{2.596533in}}%
\pgfpathlineto{\pgfqpoint{1.554467in}{2.613188in}}%
\pgfpathlineto{\pgfqpoint{1.561242in}{2.613188in}}%
\pgfpathlineto{\pgfqpoint{1.561242in}{2.617571in}}%
\pgfpathlineto{\pgfqpoint{1.568017in}{2.617571in}}%
\pgfpathlineto{\pgfqpoint{1.568017in}{2.620201in}}%
\pgfpathlineto{\pgfqpoint{1.574793in}{2.620201in}}%
\pgfpathlineto{\pgfqpoint{1.574793in}{2.622392in}}%
\pgfpathlineto{\pgfqpoint{1.581568in}{2.622392in}}%
\pgfpathlineto{\pgfqpoint{1.581568in}{2.618009in}}%
\pgfpathlineto{\pgfqpoint{1.588343in}{2.618009in}}%
\pgfpathlineto{\pgfqpoint{1.588343in}{2.634226in}}%
\pgfpathlineto{\pgfqpoint{1.595118in}{2.634226in}}%
\pgfpathlineto{\pgfqpoint{1.595118in}{2.648689in}}%
\pgfpathlineto{\pgfqpoint{1.608668in}{2.649127in}}%
\pgfpathlineto{\pgfqpoint{1.608668in}{2.656578in}}%
\pgfpathlineto{\pgfqpoint{1.615443in}{2.656578in}}%
\pgfpathlineto{\pgfqpoint{1.615443in}{2.662714in}}%
\pgfpathlineto{\pgfqpoint{1.622219in}{2.662714in}}%
\pgfpathlineto{\pgfqpoint{1.622219in}{2.660084in}}%
\pgfpathlineto{\pgfqpoint{1.628994in}{2.660084in}}%
\pgfpathlineto{\pgfqpoint{1.628994in}{2.680684in}}%
\pgfpathlineto{\pgfqpoint{1.635769in}{2.680684in}}%
\pgfpathlineto{\pgfqpoint{1.635769in}{2.677177in}}%
\pgfpathlineto{\pgfqpoint{1.642544in}{2.677177in}}%
\pgfpathlineto{\pgfqpoint{1.642544in}{2.685066in}}%
\pgfpathlineto{\pgfqpoint{1.649319in}{2.685066in}}%
\pgfpathlineto{\pgfqpoint{1.649319in}{2.693394in}}%
\pgfpathlineto{\pgfqpoint{1.656094in}{2.693394in}}%
\pgfpathlineto{\pgfqpoint{1.656094in}{2.706104in}}%
\pgfpathlineto{\pgfqpoint{1.662870in}{2.706104in}}%
\pgfpathlineto{\pgfqpoint{1.662870in}{2.718814in}}%
\pgfpathlineto{\pgfqpoint{1.669645in}{2.718814in}}%
\pgfpathlineto{\pgfqpoint{1.669645in}{2.730648in}}%
\pgfpathlineto{\pgfqpoint{1.676420in}{2.730648in}}%
\pgfpathlineto{\pgfqpoint{1.676420in}{2.726265in}}%
\pgfpathlineto{\pgfqpoint{1.683195in}{2.726265in}}%
\pgfpathlineto{\pgfqpoint{1.683195in}{2.756068in}}%
\pgfpathlineto{\pgfqpoint{1.689970in}{2.756068in}}%
\pgfpathlineto{\pgfqpoint{1.689970in}{2.753877in}}%
\pgfpathlineto{\pgfqpoint{1.696745in}{2.753877in}}%
\pgfpathlineto{\pgfqpoint{1.696745in}{2.769655in}}%
\pgfpathlineto{\pgfqpoint{1.703520in}{2.769655in}}%
\pgfpathlineto{\pgfqpoint{1.703520in}{2.783242in}}%
\pgfpathlineto{\pgfqpoint{1.710296in}{2.783242in}}%
\pgfpathlineto{\pgfqpoint{1.710296in}{2.779735in}}%
\pgfpathlineto{\pgfqpoint{1.717071in}{2.779735in}}%
\pgfpathlineto{\pgfqpoint{1.717071in}{2.826632in}}%
\pgfpathlineto{\pgfqpoint{1.723846in}{2.826632in}}%
\pgfpathlineto{\pgfqpoint{1.723846in}{2.792446in}}%
\pgfpathlineto{\pgfqpoint{1.730621in}{2.792446in}}%
\pgfpathlineto{\pgfqpoint{1.730621in}{2.823125in}}%
\pgfpathlineto{\pgfqpoint{1.737396in}{2.823125in}}%
\pgfpathlineto{\pgfqpoint{1.737396in}{2.850299in}}%
\pgfpathlineto{\pgfqpoint{1.744171in}{2.850299in}}%
\pgfpathlineto{\pgfqpoint{1.744171in}{2.856873in}}%
\pgfpathlineto{\pgfqpoint{1.750946in}{2.856873in}}%
\pgfpathlineto{\pgfqpoint{1.750946in}{2.879664in}}%
\pgfpathlineto{\pgfqpoint{1.757722in}{2.879664in}}%
\pgfpathlineto{\pgfqpoint{1.757722in}{2.909905in}}%
\pgfpathlineto{\pgfqpoint{1.764497in}{2.909905in}}%
\pgfpathlineto{\pgfqpoint{1.764497in}{2.932258in}}%
\pgfpathlineto{\pgfqpoint{1.771272in}{2.932258in}}%
\pgfpathlineto{\pgfqpoint{1.771272in}{2.965129in}}%
\pgfpathlineto{\pgfqpoint{1.778047in}{2.965129in}}%
\pgfpathlineto{\pgfqpoint{1.778047in}{2.958993in}}%
\pgfpathlineto{\pgfqpoint{1.784822in}{2.958993in}}%
\pgfpathlineto{\pgfqpoint{1.784822in}{3.031748in}}%
\pgfpathlineto{\pgfqpoint{1.791597in}{3.031748in}}%
\pgfpathlineto{\pgfqpoint{1.791597in}{3.026488in}}%
\pgfpathlineto{\pgfqpoint{1.798372in}{3.026488in}}%
\pgfpathlineto{\pgfqpoint{1.798372in}{3.087410in}}%
\pgfpathlineto{\pgfqpoint{1.805148in}{3.087410in}}%
\pgfpathlineto{\pgfqpoint{1.805148in}{3.104064in}}%
\pgfpathlineto{\pgfqpoint{1.811923in}{3.104064in}}%
\pgfpathlineto{\pgfqpoint{1.811923in}{3.150961in}}%
\pgfpathlineto{\pgfqpoint{1.818698in}{3.150961in}}%
\pgfpathlineto{\pgfqpoint{1.818698in}{3.192159in}}%
\pgfpathlineto{\pgfqpoint{1.825473in}{3.192159in}}%
\pgfpathlineto{\pgfqpoint{1.825473in}{3.217580in}}%
\pgfpathlineto{\pgfqpoint{1.832248in}{3.217580in}}%
\pgfpathlineto{\pgfqpoint{1.832248in}{3.253519in}}%
\pgfpathlineto{\pgfqpoint{1.839023in}{3.253519in}}%
\pgfpathlineto{\pgfqpoint{1.839023in}{3.348188in}}%
\pgfpathlineto{\pgfqpoint{1.845798in}{3.348188in}}%
\pgfpathlineto{\pgfqpoint{1.845798in}{3.379306in}}%
\pgfpathlineto{\pgfqpoint{1.852574in}{3.379306in}}%
\pgfpathlineto{\pgfqpoint{1.852574in}{3.451184in}}%
\pgfpathlineto{\pgfqpoint{1.859349in}{3.451184in}}%
\pgfpathlineto{\pgfqpoint{1.859349in}{3.500710in}}%
\pgfpathlineto{\pgfqpoint{1.866124in}{3.500710in}}%
\pgfpathlineto{\pgfqpoint{1.866124in}{3.524816in}}%
\pgfpathlineto{\pgfqpoint{1.872899in}{3.524816in}}%
\pgfpathlineto{\pgfqpoint{1.872899in}{3.634386in}}%
\pgfpathlineto{\pgfqpoint{1.879674in}{3.634386in}}%
\pgfpathlineto{\pgfqpoint{1.879674in}{3.684789in}}%
\pgfpathlineto{\pgfqpoint{1.886449in}{3.684789in}}%
\pgfpathlineto{\pgfqpoint{1.886449in}{3.730809in}}%
\pgfpathlineto{\pgfqpoint{1.893224in}{3.730809in}}%
\pgfpathlineto{\pgfqpoint{1.893224in}{3.741327in}}%
\pgfpathlineto{\pgfqpoint{1.900000in}{3.741327in}}%
\pgfpathlineto{\pgfqpoint{1.900000in}{3.814082in}}%
\pgfpathlineto{\pgfqpoint{1.906775in}{3.814082in}}%
\pgfpathlineto{\pgfqpoint{1.906775in}{3.789977in}}%
\pgfpathlineto{\pgfqpoint{1.913550in}{3.789977in}}%
\pgfpathlineto{\pgfqpoint{1.913550in}{3.794360in}}%
\pgfpathlineto{\pgfqpoint{1.920325in}{3.794360in}}%
\pgfpathlineto{\pgfqpoint{1.920325in}{3.709333in}}%
\pgfpathlineto{\pgfqpoint{1.927100in}{3.709333in}}%
\pgfpathlineto{\pgfqpoint{1.927100in}{3.718537in}}%
\pgfpathlineto{\pgfqpoint{1.933875in}{3.718537in}}%
\pgfpathlineto{\pgfqpoint{1.933875in}{3.630880in}}%
\pgfpathlineto{\pgfqpoint{1.940650in}{3.630880in}}%
\pgfpathlineto{\pgfqpoint{1.940650in}{3.594503in}}%
\pgfpathlineto{\pgfqpoint{1.947426in}{3.594503in}}%
\pgfpathlineto{\pgfqpoint{1.947426in}{3.529199in}}%
\pgfpathlineto{\pgfqpoint{1.954201in}{3.529199in}}%
\pgfpathlineto{\pgfqpoint{1.954201in}{3.470469in}}%
\pgfpathlineto{\pgfqpoint{1.960976in}{3.470469in}}%
\pgfpathlineto{\pgfqpoint{1.960976in}{3.379306in}}%
\pgfpathlineto{\pgfqpoint{1.967751in}{3.379306in}}%
\pgfpathlineto{\pgfqpoint{1.967751in}{3.356077in}}%
\pgfpathlineto{\pgfqpoint{1.974526in}{3.356077in}}%
\pgfpathlineto{\pgfqpoint{1.974526in}{3.316632in}}%
\pgfpathlineto{\pgfqpoint{1.981301in}{3.316632in}}%
\pgfpathlineto{\pgfqpoint{1.981301in}{3.275433in}}%
\pgfpathlineto{\pgfqpoint{1.988076in}{3.275433in}}%
\pgfpathlineto{\pgfqpoint{1.988076in}{3.195227in}}%
\pgfpathlineto{\pgfqpoint{1.994852in}{3.195227in}}%
\pgfpathlineto{\pgfqpoint{1.994852in}{3.176381in}}%
\pgfpathlineto{\pgfqpoint{2.001627in}{3.176381in}}%
\pgfpathlineto{\pgfqpoint{2.001627in}{3.134744in}}%
\pgfpathlineto{\pgfqpoint{2.008402in}{3.134744in}}%
\pgfpathlineto{\pgfqpoint{2.008402in}{3.115022in}}%
\pgfpathlineto{\pgfqpoint{2.015177in}{3.115022in}}%
\pgfpathlineto{\pgfqpoint{2.015177in}{3.075576in}}%
\pgfpathlineto{\pgfqpoint{2.021952in}{3.075576in}}%
\pgfpathlineto{\pgfqpoint{2.021952in}{3.054539in}}%
\pgfpathlineto{\pgfqpoint{2.028727in}{3.054539in}}%
\pgfpathlineto{\pgfqpoint{2.028727in}{2.990988in}}%
\pgfpathlineto{\pgfqpoint{2.035502in}{2.990988in}}%
\pgfpathlineto{\pgfqpoint{2.035502in}{2.986166in}}%
\pgfpathlineto{\pgfqpoint{2.042278in}{2.986166in}}%
\pgfpathlineto{\pgfqpoint{2.042278in}{2.973018in}}%
\pgfpathlineto{\pgfqpoint{2.049053in}{2.973018in}}%
\pgfpathlineto{\pgfqpoint{2.049053in}{2.955925in}}%
\pgfpathlineto{\pgfqpoint{2.055828in}{2.955925in}}%
\pgfpathlineto{\pgfqpoint{2.055828in}{2.923930in}}%
\pgfpathlineto{\pgfqpoint{2.062603in}{2.923930in}}%
\pgfpathlineto{\pgfqpoint{2.062603in}{2.943653in}}%
\pgfpathlineto{\pgfqpoint{2.069378in}{2.943653in}}%
\pgfpathlineto{\pgfqpoint{2.069378in}{2.903769in}}%
\pgfpathlineto{\pgfqpoint{2.076153in}{2.903769in}}%
\pgfpathlineto{\pgfqpoint{2.076153in}{2.873090in}}%
\pgfpathlineto{\pgfqpoint{2.082928in}{2.873090in}}%
\pgfpathlineto{\pgfqpoint{2.082928in}{2.860818in}}%
\pgfpathlineto{\pgfqpoint{2.089704in}{2.860818in}}%
\pgfpathlineto{\pgfqpoint{2.089704in}{2.835836in}}%
\pgfpathlineto{\pgfqpoint{2.096479in}{2.835836in}}%
\pgfpathlineto{\pgfqpoint{2.096479in}{2.829261in}}%
\pgfpathlineto{\pgfqpoint{2.103254in}{2.829261in}}%
\pgfpathlineto{\pgfqpoint{2.103254in}{2.814798in}}%
\pgfpathlineto{\pgfqpoint{2.110029in}{2.814798in}}%
\pgfpathlineto{\pgfqpoint{2.110029in}{2.823564in}}%
\pgfpathlineto{\pgfqpoint{2.116804in}{2.823564in}}%
\pgfpathlineto{\pgfqpoint{2.116804in}{2.797267in}}%
\pgfpathlineto{\pgfqpoint{2.123579in}{2.797267in}}%
\pgfpathlineto{\pgfqpoint{2.123579in}{2.769217in}}%
\pgfpathlineto{\pgfqpoint{2.130354in}{2.769217in}}%
\pgfpathlineto{\pgfqpoint{2.130354in}{2.789816in}}%
\pgfpathlineto{\pgfqpoint{2.137130in}{2.789816in}}%
\pgfpathlineto{\pgfqpoint{2.137130in}{2.756068in}}%
\pgfpathlineto{\pgfqpoint{2.143905in}{2.756068in}}%
\pgfpathlineto{\pgfqpoint{2.143905in}{2.748617in}}%
\pgfpathlineto{\pgfqpoint{2.150680in}{2.748617in}}%
\pgfpathlineto{\pgfqpoint{2.150680in}{2.741167in}}%
\pgfpathlineto{\pgfqpoint{2.157455in}{2.741167in}}%
\pgfpathlineto{\pgfqpoint{2.157455in}{2.754315in}}%
\pgfpathlineto{\pgfqpoint{2.164230in}{2.754315in}}%
\pgfpathlineto{\pgfqpoint{2.164230in}{2.721006in}}%
\pgfpathlineto{\pgfqpoint{2.171005in}{2.721006in}}%
\pgfpathlineto{\pgfqpoint{2.171005in}{2.701721in}}%
\pgfpathlineto{\pgfqpoint{2.177780in}{2.701721in}}%
\pgfpathlineto{\pgfqpoint{2.177780in}{2.718376in}}%
\pgfpathlineto{\pgfqpoint{2.184556in}{2.718376in}}%
\pgfpathlineto{\pgfqpoint{2.184556in}{2.688573in}}%
\pgfpathlineto{\pgfqpoint{2.191331in}{2.688573in}}%
\pgfpathlineto{\pgfqpoint{2.191331in}{2.685943in}}%
\pgfpathlineto{\pgfqpoint{2.198106in}{2.685943in}}%
\pgfpathlineto{\pgfqpoint{2.198106in}{2.679807in}}%
\pgfpathlineto{\pgfqpoint{2.204881in}{2.679807in}}%
\pgfpathlineto{\pgfqpoint{2.204881in}{2.673671in}}%
\pgfpathlineto{\pgfqpoint{2.211656in}{2.673671in}}%
\pgfpathlineto{\pgfqpoint{2.211656in}{2.665782in}}%
\pgfpathlineto{\pgfqpoint{2.218431in}{2.665782in}}%
\pgfpathlineto{\pgfqpoint{2.218431in}{2.660961in}}%
\pgfpathlineto{\pgfqpoint{2.225206in}{2.660961in}}%
\pgfpathlineto{\pgfqpoint{2.225206in}{2.650880in}}%
\pgfpathlineto{\pgfqpoint{2.231982in}{2.650880in}}%
\pgfpathlineto{\pgfqpoint{2.231982in}{2.657016in}}%
\pgfpathlineto{\pgfqpoint{2.238757in}{2.657016in}}%
\pgfpathlineto{\pgfqpoint{2.238757in}{2.640362in}}%
\pgfpathlineto{\pgfqpoint{2.245532in}{2.640362in}}%
\pgfpathlineto{\pgfqpoint{2.245532in}{2.628528in}}%
\pgfpathlineto{\pgfqpoint{2.259082in}{2.627651in}}%
\pgfpathlineto{\pgfqpoint{2.259082in}{2.615818in}}%
\pgfpathlineto{\pgfqpoint{2.265857in}{2.615818in}}%
\pgfpathlineto{\pgfqpoint{2.265857in}{2.603546in}}%
\pgfpathlineto{\pgfqpoint{2.272632in}{2.603546in}}%
\pgfpathlineto{\pgfqpoint{2.272632in}{2.616256in}}%
\pgfpathlineto{\pgfqpoint{2.279408in}{2.616256in}}%
\pgfpathlineto{\pgfqpoint{2.279408in}{2.621954in}}%
\pgfpathlineto{\pgfqpoint{2.286183in}{2.621954in}}%
\pgfpathlineto{\pgfqpoint{2.286183in}{2.588206in}}%
\pgfpathlineto{\pgfqpoint{2.292958in}{2.588206in}}%
\pgfpathlineto{\pgfqpoint{2.292958in}{2.613188in}}%
\pgfpathlineto{\pgfqpoint{2.299733in}{2.613188in}}%
\pgfpathlineto{\pgfqpoint{2.299733in}{2.603984in}}%
\pgfpathlineto{\pgfqpoint{2.306508in}{2.603984in}}%
\pgfpathlineto{\pgfqpoint{2.306508in}{2.589959in}}%
\pgfpathlineto{\pgfqpoint{2.313283in}{2.589959in}}%
\pgfpathlineto{\pgfqpoint{2.313283in}{2.596972in}}%
\pgfpathlineto{\pgfqpoint{2.320058in}{2.596972in}}%
\pgfpathlineto{\pgfqpoint{2.320058in}{2.580755in}}%
\pgfpathlineto{\pgfqpoint{2.326834in}{2.580755in}}%
\pgfpathlineto{\pgfqpoint{2.326834in}{2.583823in}}%
\pgfpathlineto{\pgfqpoint{2.333609in}{2.583823in}}%
\pgfpathlineto{\pgfqpoint{2.333609in}{2.574619in}}%
\pgfpathlineto{\pgfqpoint{2.340384in}{2.574619in}}%
\pgfpathlineto{\pgfqpoint{2.340384in}{2.584261in}}%
\pgfpathlineto{\pgfqpoint{2.347159in}{2.584261in}}%
\pgfpathlineto{\pgfqpoint{2.347159in}{2.575496in}}%
\pgfpathlineto{\pgfqpoint{2.353934in}{2.575496in}}%
\pgfpathlineto{\pgfqpoint{2.353934in}{2.573743in}}%
\pgfpathlineto{\pgfqpoint{2.367484in}{2.574181in}}%
\pgfpathlineto{\pgfqpoint{2.367484in}{2.562786in}}%
\pgfpathlineto{\pgfqpoint{2.381035in}{2.561909in}}%
\pgfpathlineto{\pgfqpoint{2.381035in}{2.570236in}}%
\pgfpathlineto{\pgfqpoint{2.387810in}{2.570236in}}%
\pgfpathlineto{\pgfqpoint{2.387810in}{2.557526in}}%
\pgfpathlineto{\pgfqpoint{2.394585in}{2.557526in}}%
\pgfpathlineto{\pgfqpoint{2.394585in}{2.552705in}}%
\pgfpathlineto{\pgfqpoint{2.401360in}{2.552705in}}%
\pgfpathlineto{\pgfqpoint{2.401360in}{2.542625in}}%
\pgfpathlineto{\pgfqpoint{2.408135in}{2.542625in}}%
\pgfpathlineto{\pgfqpoint{2.408135in}{2.547007in}}%
\pgfpathlineto{\pgfqpoint{2.421686in}{2.546131in}}%
\pgfpathlineto{\pgfqpoint{2.421686in}{2.545254in}}%
\pgfpathlineto{\pgfqpoint{2.428461in}{2.545254in}}%
\pgfpathlineto{\pgfqpoint{2.428461in}{2.547884in}}%
\pgfpathlineto{\pgfqpoint{2.435236in}{2.547884in}}%
\pgfpathlineto{\pgfqpoint{2.435236in}{2.538242in}}%
\pgfpathlineto{\pgfqpoint{2.442011in}{2.538242in}}%
\pgfpathlineto{\pgfqpoint{2.442011in}{2.550952in}}%
\pgfpathlineto{\pgfqpoint{2.448786in}{2.550952in}}%
\pgfpathlineto{\pgfqpoint{2.448786in}{2.533859in}}%
\pgfpathlineto{\pgfqpoint{2.455561in}{2.533859in}}%
\pgfpathlineto{\pgfqpoint{2.455561in}{2.544816in}}%
\pgfpathlineto{\pgfqpoint{2.462336in}{2.544816in}}%
\pgfpathlineto{\pgfqpoint{2.462336in}{2.532544in}}%
\pgfpathlineto{\pgfqpoint{2.475887in}{2.531667in}}%
\pgfpathlineto{\pgfqpoint{2.475887in}{2.533859in}}%
\pgfpathlineto{\pgfqpoint{2.489437in}{2.533421in}}%
\pgfpathlineto{\pgfqpoint{2.489437in}{2.531229in}}%
\pgfpathlineto{\pgfqpoint{2.496212in}{2.531229in}}%
\pgfpathlineto{\pgfqpoint{2.496212in}{2.526846in}}%
\pgfpathlineto{\pgfqpoint{2.502987in}{2.526846in}}%
\pgfpathlineto{\pgfqpoint{2.502987in}{2.522025in}}%
\pgfpathlineto{\pgfqpoint{2.509762in}{2.522025in}}%
\pgfpathlineto{\pgfqpoint{2.509762in}{2.526408in}}%
\pgfpathlineto{\pgfqpoint{2.516538in}{2.526408in}}%
\pgfpathlineto{\pgfqpoint{2.516538in}{2.524655in}}%
\pgfpathlineto{\pgfqpoint{2.523313in}{2.524655in}}%
\pgfpathlineto{\pgfqpoint{2.523313in}{2.519834in}}%
\pgfpathlineto{\pgfqpoint{2.530088in}{2.519834in}}%
\pgfpathlineto{\pgfqpoint{2.530088in}{2.517204in}}%
\pgfpathlineto{\pgfqpoint{2.536863in}{2.517204in}}%
\pgfpathlineto{\pgfqpoint{2.536863in}{2.519834in}}%
\pgfpathlineto{\pgfqpoint{2.543638in}{2.519834in}}%
\pgfpathlineto{\pgfqpoint{2.543638in}{2.527723in}}%
\pgfpathlineto{\pgfqpoint{2.550413in}{2.527723in}}%
\pgfpathlineto{\pgfqpoint{2.550413in}{2.525532in}}%
\pgfpathlineto{\pgfqpoint{2.557188in}{2.525532in}}%
\pgfpathlineto{\pgfqpoint{2.557188in}{2.518519in}}%
\pgfpathlineto{\pgfqpoint{2.563964in}{2.518519in}}%
\pgfpathlineto{\pgfqpoint{2.563964in}{2.516766in}}%
\pgfpathlineto{\pgfqpoint{2.570739in}{2.516766in}}%
\pgfpathlineto{\pgfqpoint{2.570739in}{2.515013in}}%
\pgfpathlineto{\pgfqpoint{2.577514in}{2.515013in}}%
\pgfpathlineto{\pgfqpoint{2.577514in}{2.522025in}}%
\pgfpathlineto{\pgfqpoint{2.584289in}{2.522025in}}%
\pgfpathlineto{\pgfqpoint{2.584289in}{2.511945in}}%
\pgfpathlineto{\pgfqpoint{2.604614in}{2.511945in}}%
\pgfpathlineto{\pgfqpoint{2.604614in}{2.509753in}}%
\pgfpathlineto{\pgfqpoint{2.618165in}{2.510630in}}%
\pgfpathlineto{\pgfqpoint{2.618165in}{2.511945in}}%
\pgfpathlineto{\pgfqpoint{2.624940in}{2.511945in}}%
\pgfpathlineto{\pgfqpoint{2.624940in}{2.510192in}}%
\pgfpathlineto{\pgfqpoint{2.638490in}{2.511068in}}%
\pgfpathlineto{\pgfqpoint{2.638490in}{2.506685in}}%
\pgfpathlineto{\pgfqpoint{2.645265in}{2.506685in}}%
\pgfpathlineto{\pgfqpoint{2.645265in}{2.504056in}}%
\pgfpathlineto{\pgfqpoint{2.652040in}{2.504056in}}%
\pgfpathlineto{\pgfqpoint{2.652040in}{2.508438in}}%
\pgfpathlineto{\pgfqpoint{2.658816in}{2.508438in}}%
\pgfpathlineto{\pgfqpoint{2.658816in}{2.504056in}}%
\pgfpathlineto{\pgfqpoint{2.665591in}{2.504056in}}%
\pgfpathlineto{\pgfqpoint{2.665591in}{2.510630in}}%
\pgfpathlineto{\pgfqpoint{2.672366in}{2.510630in}}%
\pgfpathlineto{\pgfqpoint{2.672366in}{2.505809in}}%
\pgfpathlineto{\pgfqpoint{2.679141in}{2.505809in}}%
\pgfpathlineto{\pgfqpoint{2.679141in}{2.502741in}}%
\pgfpathlineto{\pgfqpoint{2.699466in}{2.502741in}}%
\pgfpathlineto{\pgfqpoint{2.699466in}{2.501426in}}%
\pgfpathlineto{\pgfqpoint{2.713017in}{2.500988in}}%
\pgfpathlineto{\pgfqpoint{2.713017in}{2.504932in}}%
\pgfpathlineto{\pgfqpoint{2.733342in}{2.505371in}}%
\pgfpathlineto{\pgfqpoint{2.733342in}{2.501426in}}%
\pgfpathlineto{\pgfqpoint{2.746893in}{2.502303in}}%
\pgfpathlineto{\pgfqpoint{2.746893in}{2.504056in}}%
\pgfpathlineto{\pgfqpoint{2.753668in}{2.504056in}}%
\pgfpathlineto{\pgfqpoint{2.753668in}{2.502303in}}%
\pgfpathlineto{\pgfqpoint{2.760443in}{2.502303in}}%
\pgfpathlineto{\pgfqpoint{2.760443in}{2.500988in}}%
\pgfpathlineto{\pgfqpoint{2.767218in}{2.500988in}}%
\pgfpathlineto{\pgfqpoint{2.767218in}{2.497920in}}%
\pgfpathlineto{\pgfqpoint{2.773993in}{2.497920in}}%
\pgfpathlineto{\pgfqpoint{2.773993in}{2.499673in}}%
\pgfpathlineto{\pgfqpoint{2.780768in}{2.499673in}}%
\pgfpathlineto{\pgfqpoint{2.780768in}{2.498358in}}%
\pgfpathlineto{\pgfqpoint{2.787543in}{2.498358in}}%
\pgfpathlineto{\pgfqpoint{2.787543in}{2.503617in}}%
\pgfpathlineto{\pgfqpoint{2.794319in}{2.503617in}}%
\pgfpathlineto{\pgfqpoint{2.794319in}{2.500988in}}%
\pgfpathlineto{\pgfqpoint{2.814644in}{2.500988in}}%
\pgfpathlineto{\pgfqpoint{2.814644in}{2.497481in}}%
\pgfpathlineto{\pgfqpoint{2.834969in}{2.497920in}}%
\pgfpathlineto{\pgfqpoint{2.834969in}{2.500111in}}%
\pgfpathlineto{\pgfqpoint{2.841745in}{2.500111in}}%
\pgfpathlineto{\pgfqpoint{2.841745in}{2.498796in}}%
\pgfpathlineto{\pgfqpoint{2.862070in}{2.498358in}}%
\pgfpathlineto{\pgfqpoint{2.862070in}{2.500111in}}%
\pgfpathlineto{\pgfqpoint{2.868845in}{2.500111in}}%
\pgfpathlineto{\pgfqpoint{2.868845in}{2.497481in}}%
\pgfpathlineto{\pgfqpoint{2.909496in}{2.497920in}}%
\pgfpathlineto{\pgfqpoint{2.909496in}{2.496167in}}%
\pgfpathlineto{\pgfqpoint{2.909496in}{2.496167in}}%
\pgfusepath{stroke}%
\end{pgfscope}%
\begin{pgfscope}%
\pgfsetrectcap%
\pgfsetmiterjoin%
\pgfsetlinewidth{0.803000pt}%
\definecolor{currentstroke}{rgb}{0.000000,0.000000,0.000000}%
\pgfsetstrokecolor{currentstroke}%
\pgfsetdash{}{0pt}%
\pgfpathmoveto{\pgfqpoint{0.754048in}{2.496167in}}%
\pgfpathlineto{\pgfqpoint{0.754048in}{3.913639in}}%
\pgfusepath{stroke}%
\end{pgfscope}%
\begin{pgfscope}%
\pgfsetrectcap%
\pgfsetmiterjoin%
\pgfsetlinewidth{0.803000pt}%
\definecolor{currentstroke}{rgb}{0.000000,0.000000,0.000000}%
\pgfsetstrokecolor{currentstroke}%
\pgfsetdash{}{0pt}%
\pgfpathmoveto{\pgfqpoint{3.055114in}{2.496167in}}%
\pgfpathlineto{\pgfqpoint{3.055114in}{3.913639in}}%
\pgfusepath{stroke}%
\end{pgfscope}%
\begin{pgfscope}%
\pgfsetrectcap%
\pgfsetmiterjoin%
\pgfsetlinewidth{0.803000pt}%
\definecolor{currentstroke}{rgb}{0.000000,0.000000,0.000000}%
\pgfsetstrokecolor{currentstroke}%
\pgfsetdash{}{0pt}%
\pgfpathmoveto{\pgfqpoint{0.754048in}{2.496167in}}%
\pgfpathlineto{\pgfqpoint{3.055114in}{2.496167in}}%
\pgfusepath{stroke}%
\end{pgfscope}%
\begin{pgfscope}%
\pgfsetrectcap%
\pgfsetmiterjoin%
\pgfsetlinewidth{0.803000pt}%
\definecolor{currentstroke}{rgb}{0.000000,0.000000,0.000000}%
\pgfsetstrokecolor{currentstroke}%
\pgfsetdash{}{0pt}%
\pgfpathmoveto{\pgfqpoint{0.754048in}{3.913639in}}%
\pgfpathlineto{\pgfqpoint{3.055114in}{3.913639in}}%
\pgfusepath{stroke}%
\end{pgfscope}%
\begin{pgfscope}%
\definecolor{textcolor}{rgb}{0.000000,0.000000,0.000000}%
\pgfsetstrokecolor{textcolor}%
\pgfsetfillcolor{textcolor}%
\pgftext[x=0.754048in,y=3.996972in,left,base]{\color{textcolor}\rmfamily\fontsize{10.000000}{12.000000}\selectfont Bin [1.33, 1.5), 127,888 events}%
\end{pgfscope}%
\begin{pgfscope}%
\pgfsetbuttcap%
\pgfsetmiterjoin%
\definecolor{currentfill}{rgb}{1.000000,1.000000,1.000000}%
\pgfsetfillcolor{currentfill}%
\pgfsetfillopacity{0.800000}%
\pgfsetlinewidth{1.003750pt}%
\definecolor{currentstroke}{rgb}{0.800000,0.800000,0.800000}%
\pgfsetstrokecolor{currentstroke}%
\pgfsetstrokeopacity{0.800000}%
\pgfsetdash{}{0pt}%
\pgfpathmoveto{\pgfqpoint{1.961337in}{3.514083in}}%
\pgfpathlineto{\pgfqpoint{2.977337in}{3.514083in}}%
\pgfpathquadraticcurveto{\pgfqpoint{2.999559in}{3.514083in}}{\pgfqpoint{2.999559in}{3.536306in}}%
\pgfpathlineto{\pgfqpoint{2.999559in}{3.835861in}}%
\pgfpathquadraticcurveto{\pgfqpoint{2.999559in}{3.858083in}}{\pgfqpoint{2.977337in}{3.858083in}}%
\pgfpathlineto{\pgfqpoint{1.961337in}{3.858083in}}%
\pgfpathquadraticcurveto{\pgfqpoint{1.939114in}{3.858083in}}{\pgfqpoint{1.939114in}{3.835861in}}%
\pgfpathlineto{\pgfqpoint{1.939114in}{3.536306in}}%
\pgfpathquadraticcurveto{\pgfqpoint{1.939114in}{3.514083in}}{\pgfqpoint{1.961337in}{3.514083in}}%
\pgfpathclose%
\pgfusepath{stroke,fill}%
\end{pgfscope}%
\begin{pgfscope}%
\pgfsetbuttcap%
\pgfsetmiterjoin%
\pgfsetlinewidth{1.003750pt}%
\definecolor{currentstroke}{rgb}{0.313725,0.317647,0.309804}%
\pgfsetstrokecolor{currentstroke}%
\pgfsetdash{}{0pt}%
\pgfpathmoveto{\pgfqpoint{1.983559in}{3.735417in}}%
\pgfpathlineto{\pgfqpoint{2.205781in}{3.735417in}}%
\pgfpathlineto{\pgfqpoint{2.205781in}{3.813194in}}%
\pgfpathlineto{\pgfqpoint{1.983559in}{3.813194in}}%
\pgfpathclose%
\pgfusepath{stroke}%
\end{pgfscope}%
\begin{pgfscope}%
\definecolor{textcolor}{rgb}{0.000000,0.000000,0.000000}%
\pgfsetstrokecolor{textcolor}%
\pgfsetfillcolor{textcolor}%
\pgftext[x=2.294670in,y=3.735417in,left,base]{\color{textcolor}\rmfamily\fontsize{8.000000}{9.600000}\selectfont IQR = 16.57}%
\end{pgfscope}%
\begin{pgfscope}%
\pgfsetbuttcap%
\pgfsetmiterjoin%
\pgfsetlinewidth{1.003750pt}%
\definecolor{currentstroke}{rgb}{0.949020,0.372549,0.360784}%
\pgfsetstrokecolor{currentstroke}%
\pgfsetdash{{1.000000pt}{1.650000pt}}{0.000000pt}%
\pgfpathmoveto{\pgfqpoint{1.983559in}{3.580083in}}%
\pgfpathlineto{\pgfqpoint{2.205781in}{3.580083in}}%
\pgfpathlineto{\pgfqpoint{2.205781in}{3.657861in}}%
\pgfpathlineto{\pgfqpoint{1.983559in}{3.657861in}}%
\pgfpathclose%
\pgfusepath{stroke}%
\end{pgfscope}%
\begin{pgfscope}%
\definecolor{textcolor}{rgb}{0.000000,0.000000,0.000000}%
\pgfsetstrokecolor{textcolor}%
\pgfsetfillcolor{textcolor}%
\pgftext[x=2.294670in,y=3.580083in,left,base]{\color{textcolor}\rmfamily\fontsize{8.000000}{9.600000}\selectfont IQR = 18.8}%
\end{pgfscope}%
\begin{pgfscope}%
\pgfsetbuttcap%
\pgfsetmiterjoin%
\definecolor{currentfill}{rgb}{1.000000,1.000000,1.000000}%
\pgfsetfillcolor{currentfill}%
\pgfsetlinewidth{0.000000pt}%
\definecolor{currentstroke}{rgb}{0.000000,0.000000,0.000000}%
\pgfsetstrokecolor{currentstroke}%
\pgfsetstrokeopacity{0.000000}%
\pgfsetdash{}{0pt}%
\pgfpathmoveto{\pgfqpoint{3.750134in}{2.496167in}}%
\pgfpathlineto{\pgfqpoint{6.051200in}{2.496167in}}%
\pgfpathlineto{\pgfqpoint{6.051200in}{3.913639in}}%
\pgfpathlineto{\pgfqpoint{3.750134in}{3.913639in}}%
\pgfpathclose%
\pgfusepath{fill}%
\end{pgfscope}%
\begin{pgfscope}%
\pgfsetbuttcap%
\pgfsetroundjoin%
\definecolor{currentfill}{rgb}{0.000000,0.000000,0.000000}%
\pgfsetfillcolor{currentfill}%
\pgfsetlinewidth{0.803000pt}%
\definecolor{currentstroke}{rgb}{0.000000,0.000000,0.000000}%
\pgfsetstrokecolor{currentstroke}%
\pgfsetdash{}{0pt}%
\pgfsys@defobject{currentmarker}{\pgfqpoint{0.000000in}{-0.048611in}}{\pgfqpoint{0.000000in}{0.000000in}}{%
\pgfpathmoveto{\pgfqpoint{0.000000in}{0.000000in}}%
\pgfpathlineto{\pgfqpoint{0.000000in}{-0.048611in}}%
\pgfusepath{stroke,fill}%
}%
\begin{pgfscope}%
\pgfsys@transformshift{3.982265in}{2.496167in}%
\pgfsys@useobject{currentmarker}{}%
\end{pgfscope}%
\end{pgfscope}%
\begin{pgfscope}%
\definecolor{textcolor}{rgb}{0.000000,0.000000,0.000000}%
\pgfsetstrokecolor{textcolor}%
\pgfsetfillcolor{textcolor}%
\pgftext[x=3.982265in,y=2.398944in,,top]{\color{textcolor}\rmfamily\fontsize{8.000000}{9.600000}\selectfont \(\displaystyle {-120}\)}%
\end{pgfscope}%
\begin{pgfscope}%
\pgfsetbuttcap%
\pgfsetroundjoin%
\definecolor{currentfill}{rgb}{0.000000,0.000000,0.000000}%
\pgfsetfillcolor{currentfill}%
\pgfsetlinewidth{0.803000pt}%
\definecolor{currentstroke}{rgb}{0.000000,0.000000,0.000000}%
\pgfsetstrokecolor{currentstroke}%
\pgfsetdash{}{0pt}%
\pgfsys@defobject{currentmarker}{\pgfqpoint{0.000000in}{-0.048611in}}{\pgfqpoint{0.000000in}{0.000000in}}{%
\pgfpathmoveto{\pgfqpoint{0.000000in}{0.000000in}}%
\pgfpathlineto{\pgfqpoint{0.000000in}{-0.048611in}}%
\pgfusepath{stroke,fill}%
}%
\begin{pgfscope}%
\pgfsys@transformshift{4.287966in}{2.496167in}%
\pgfsys@useobject{currentmarker}{}%
\end{pgfscope}%
\end{pgfscope}%
\begin{pgfscope}%
\definecolor{textcolor}{rgb}{0.000000,0.000000,0.000000}%
\pgfsetstrokecolor{textcolor}%
\pgfsetfillcolor{textcolor}%
\pgftext[x=4.287966in,y=2.398944in,,top]{\color{textcolor}\rmfamily\fontsize{8.000000}{9.600000}\selectfont \(\displaystyle {-80}\)}%
\end{pgfscope}%
\begin{pgfscope}%
\pgfsetbuttcap%
\pgfsetroundjoin%
\definecolor{currentfill}{rgb}{0.000000,0.000000,0.000000}%
\pgfsetfillcolor{currentfill}%
\pgfsetlinewidth{0.803000pt}%
\definecolor{currentstroke}{rgb}{0.000000,0.000000,0.000000}%
\pgfsetstrokecolor{currentstroke}%
\pgfsetdash{}{0pt}%
\pgfsys@defobject{currentmarker}{\pgfqpoint{0.000000in}{-0.048611in}}{\pgfqpoint{0.000000in}{0.000000in}}{%
\pgfpathmoveto{\pgfqpoint{0.000000in}{0.000000in}}%
\pgfpathlineto{\pgfqpoint{0.000000in}{-0.048611in}}%
\pgfusepath{stroke,fill}%
}%
\begin{pgfscope}%
\pgfsys@transformshift{4.593668in}{2.496167in}%
\pgfsys@useobject{currentmarker}{}%
\end{pgfscope}%
\end{pgfscope}%
\begin{pgfscope}%
\definecolor{textcolor}{rgb}{0.000000,0.000000,0.000000}%
\pgfsetstrokecolor{textcolor}%
\pgfsetfillcolor{textcolor}%
\pgftext[x=4.593668in,y=2.398944in,,top]{\color{textcolor}\rmfamily\fontsize{8.000000}{9.600000}\selectfont \(\displaystyle {-40}\)}%
\end{pgfscope}%
\begin{pgfscope}%
\pgfsetbuttcap%
\pgfsetroundjoin%
\definecolor{currentfill}{rgb}{0.000000,0.000000,0.000000}%
\pgfsetfillcolor{currentfill}%
\pgfsetlinewidth{0.803000pt}%
\definecolor{currentstroke}{rgb}{0.000000,0.000000,0.000000}%
\pgfsetstrokecolor{currentstroke}%
\pgfsetdash{}{0pt}%
\pgfsys@defobject{currentmarker}{\pgfqpoint{0.000000in}{-0.048611in}}{\pgfqpoint{0.000000in}{0.000000in}}{%
\pgfpathmoveto{\pgfqpoint{0.000000in}{0.000000in}}%
\pgfpathlineto{\pgfqpoint{0.000000in}{-0.048611in}}%
\pgfusepath{stroke,fill}%
}%
\begin{pgfscope}%
\pgfsys@transformshift{4.899369in}{2.496167in}%
\pgfsys@useobject{currentmarker}{}%
\end{pgfscope}%
\end{pgfscope}%
\begin{pgfscope}%
\definecolor{textcolor}{rgb}{0.000000,0.000000,0.000000}%
\pgfsetstrokecolor{textcolor}%
\pgfsetfillcolor{textcolor}%
\pgftext[x=4.899369in,y=2.398944in,,top]{\color{textcolor}\rmfamily\fontsize{8.000000}{9.600000}\selectfont \(\displaystyle {0}\)}%
\end{pgfscope}%
\begin{pgfscope}%
\pgfsetbuttcap%
\pgfsetroundjoin%
\definecolor{currentfill}{rgb}{0.000000,0.000000,0.000000}%
\pgfsetfillcolor{currentfill}%
\pgfsetlinewidth{0.803000pt}%
\definecolor{currentstroke}{rgb}{0.000000,0.000000,0.000000}%
\pgfsetstrokecolor{currentstroke}%
\pgfsetdash{}{0pt}%
\pgfsys@defobject{currentmarker}{\pgfqpoint{0.000000in}{-0.048611in}}{\pgfqpoint{0.000000in}{0.000000in}}{%
\pgfpathmoveto{\pgfqpoint{0.000000in}{0.000000in}}%
\pgfpathlineto{\pgfqpoint{0.000000in}{-0.048611in}}%
\pgfusepath{stroke,fill}%
}%
\begin{pgfscope}%
\pgfsys@transformshift{5.205070in}{2.496167in}%
\pgfsys@useobject{currentmarker}{}%
\end{pgfscope}%
\end{pgfscope}%
\begin{pgfscope}%
\definecolor{textcolor}{rgb}{0.000000,0.000000,0.000000}%
\pgfsetstrokecolor{textcolor}%
\pgfsetfillcolor{textcolor}%
\pgftext[x=5.205070in,y=2.398944in,,top]{\color{textcolor}\rmfamily\fontsize{8.000000}{9.600000}\selectfont \(\displaystyle {40}\)}%
\end{pgfscope}%
\begin{pgfscope}%
\pgfsetbuttcap%
\pgfsetroundjoin%
\definecolor{currentfill}{rgb}{0.000000,0.000000,0.000000}%
\pgfsetfillcolor{currentfill}%
\pgfsetlinewidth{0.803000pt}%
\definecolor{currentstroke}{rgb}{0.000000,0.000000,0.000000}%
\pgfsetstrokecolor{currentstroke}%
\pgfsetdash{}{0pt}%
\pgfsys@defobject{currentmarker}{\pgfqpoint{0.000000in}{-0.048611in}}{\pgfqpoint{0.000000in}{0.000000in}}{%
\pgfpathmoveto{\pgfqpoint{0.000000in}{0.000000in}}%
\pgfpathlineto{\pgfqpoint{0.000000in}{-0.048611in}}%
\pgfusepath{stroke,fill}%
}%
\begin{pgfscope}%
\pgfsys@transformshift{5.510772in}{2.496167in}%
\pgfsys@useobject{currentmarker}{}%
\end{pgfscope}%
\end{pgfscope}%
\begin{pgfscope}%
\definecolor{textcolor}{rgb}{0.000000,0.000000,0.000000}%
\pgfsetstrokecolor{textcolor}%
\pgfsetfillcolor{textcolor}%
\pgftext[x=5.510772in,y=2.398944in,,top]{\color{textcolor}\rmfamily\fontsize{8.000000}{9.600000}\selectfont \(\displaystyle {80}\)}%
\end{pgfscope}%
\begin{pgfscope}%
\pgfsetbuttcap%
\pgfsetroundjoin%
\definecolor{currentfill}{rgb}{0.000000,0.000000,0.000000}%
\pgfsetfillcolor{currentfill}%
\pgfsetlinewidth{0.803000pt}%
\definecolor{currentstroke}{rgb}{0.000000,0.000000,0.000000}%
\pgfsetstrokecolor{currentstroke}%
\pgfsetdash{}{0pt}%
\pgfsys@defobject{currentmarker}{\pgfqpoint{0.000000in}{-0.048611in}}{\pgfqpoint{0.000000in}{0.000000in}}{%
\pgfpathmoveto{\pgfqpoint{0.000000in}{0.000000in}}%
\pgfpathlineto{\pgfqpoint{0.000000in}{-0.048611in}}%
\pgfusepath{stroke,fill}%
}%
\begin{pgfscope}%
\pgfsys@transformshift{5.816473in}{2.496167in}%
\pgfsys@useobject{currentmarker}{}%
\end{pgfscope}%
\end{pgfscope}%
\begin{pgfscope}%
\definecolor{textcolor}{rgb}{0.000000,0.000000,0.000000}%
\pgfsetstrokecolor{textcolor}%
\pgfsetfillcolor{textcolor}%
\pgftext[x=5.816473in,y=2.398944in,,top]{\color{textcolor}\rmfamily\fontsize{8.000000}{9.600000}\selectfont \(\displaystyle {120}\)}%
\end{pgfscope}%
\begin{pgfscope}%
\pgfsetbuttcap%
\pgfsetroundjoin%
\definecolor{currentfill}{rgb}{0.000000,0.000000,0.000000}%
\pgfsetfillcolor{currentfill}%
\pgfsetlinewidth{0.803000pt}%
\definecolor{currentstroke}{rgb}{0.000000,0.000000,0.000000}%
\pgfsetstrokecolor{currentstroke}%
\pgfsetdash{}{0pt}%
\pgfsys@defobject{currentmarker}{\pgfqpoint{-0.048611in}{0.000000in}}{\pgfqpoint{-0.000000in}{0.000000in}}{%
\pgfpathmoveto{\pgfqpoint{-0.000000in}{0.000000in}}%
\pgfpathlineto{\pgfqpoint{-0.048611in}{0.000000in}}%
\pgfusepath{stroke,fill}%
}%
\begin{pgfscope}%
\pgfsys@transformshift{3.750134in}{2.496167in}%
\pgfsys@useobject{currentmarker}{}%
\end{pgfscope}%
\end{pgfscope}%
\begin{pgfscope}%
\definecolor{textcolor}{rgb}{0.000000,0.000000,0.000000}%
\pgfsetstrokecolor{textcolor}%
\pgfsetfillcolor{textcolor}%
\pgftext[x=3.384003in, y=2.457611in, left, base]{\color{textcolor}\rmfamily\fontsize{8.000000}{9.600000}\selectfont \(\displaystyle {0.000}\)}%
\end{pgfscope}%
\begin{pgfscope}%
\pgfsetbuttcap%
\pgfsetroundjoin%
\definecolor{currentfill}{rgb}{0.000000,0.000000,0.000000}%
\pgfsetfillcolor{currentfill}%
\pgfsetlinewidth{0.803000pt}%
\definecolor{currentstroke}{rgb}{0.000000,0.000000,0.000000}%
\pgfsetstrokecolor{currentstroke}%
\pgfsetdash{}{0pt}%
\pgfsys@defobject{currentmarker}{\pgfqpoint{-0.048611in}{0.000000in}}{\pgfqpoint{-0.000000in}{0.000000in}}{%
\pgfpathmoveto{\pgfqpoint{-0.000000in}{0.000000in}}%
\pgfpathlineto{\pgfqpoint{-0.048611in}{0.000000in}}%
\pgfusepath{stroke,fill}%
}%
\begin{pgfscope}%
\pgfsys@transformshift{3.750134in}{2.678670in}%
\pgfsys@useobject{currentmarker}{}%
\end{pgfscope}%
\end{pgfscope}%
\begin{pgfscope}%
\definecolor{textcolor}{rgb}{0.000000,0.000000,0.000000}%
\pgfsetstrokecolor{textcolor}%
\pgfsetfillcolor{textcolor}%
\pgftext[x=3.384003in, y=2.640114in, left, base]{\color{textcolor}\rmfamily\fontsize{8.000000}{9.600000}\selectfont \(\displaystyle {0.008}\)}%
\end{pgfscope}%
\begin{pgfscope}%
\pgfsetbuttcap%
\pgfsetroundjoin%
\definecolor{currentfill}{rgb}{0.000000,0.000000,0.000000}%
\pgfsetfillcolor{currentfill}%
\pgfsetlinewidth{0.803000pt}%
\definecolor{currentstroke}{rgb}{0.000000,0.000000,0.000000}%
\pgfsetstrokecolor{currentstroke}%
\pgfsetdash{}{0pt}%
\pgfsys@defobject{currentmarker}{\pgfqpoint{-0.048611in}{0.000000in}}{\pgfqpoint{-0.000000in}{0.000000in}}{%
\pgfpathmoveto{\pgfqpoint{-0.000000in}{0.000000in}}%
\pgfpathlineto{\pgfqpoint{-0.048611in}{0.000000in}}%
\pgfusepath{stroke,fill}%
}%
\begin{pgfscope}%
\pgfsys@transformshift{3.750134in}{2.861173in}%
\pgfsys@useobject{currentmarker}{}%
\end{pgfscope}%
\end{pgfscope}%
\begin{pgfscope}%
\definecolor{textcolor}{rgb}{0.000000,0.000000,0.000000}%
\pgfsetstrokecolor{textcolor}%
\pgfsetfillcolor{textcolor}%
\pgftext[x=3.384003in, y=2.822618in, left, base]{\color{textcolor}\rmfamily\fontsize{8.000000}{9.600000}\selectfont \(\displaystyle {0.016}\)}%
\end{pgfscope}%
\begin{pgfscope}%
\pgfsetbuttcap%
\pgfsetroundjoin%
\definecolor{currentfill}{rgb}{0.000000,0.000000,0.000000}%
\pgfsetfillcolor{currentfill}%
\pgfsetlinewidth{0.803000pt}%
\definecolor{currentstroke}{rgb}{0.000000,0.000000,0.000000}%
\pgfsetstrokecolor{currentstroke}%
\pgfsetdash{}{0pt}%
\pgfsys@defobject{currentmarker}{\pgfqpoint{-0.048611in}{0.000000in}}{\pgfqpoint{-0.000000in}{0.000000in}}{%
\pgfpathmoveto{\pgfqpoint{-0.000000in}{0.000000in}}%
\pgfpathlineto{\pgfqpoint{-0.048611in}{0.000000in}}%
\pgfusepath{stroke,fill}%
}%
\begin{pgfscope}%
\pgfsys@transformshift{3.750134in}{3.043677in}%
\pgfsys@useobject{currentmarker}{}%
\end{pgfscope}%
\end{pgfscope}%
\begin{pgfscope}%
\definecolor{textcolor}{rgb}{0.000000,0.000000,0.000000}%
\pgfsetstrokecolor{textcolor}%
\pgfsetfillcolor{textcolor}%
\pgftext[x=3.384003in, y=3.005121in, left, base]{\color{textcolor}\rmfamily\fontsize{8.000000}{9.600000}\selectfont \(\displaystyle {0.024}\)}%
\end{pgfscope}%
\begin{pgfscope}%
\pgfsetbuttcap%
\pgfsetroundjoin%
\definecolor{currentfill}{rgb}{0.000000,0.000000,0.000000}%
\pgfsetfillcolor{currentfill}%
\pgfsetlinewidth{0.803000pt}%
\definecolor{currentstroke}{rgb}{0.000000,0.000000,0.000000}%
\pgfsetstrokecolor{currentstroke}%
\pgfsetdash{}{0pt}%
\pgfsys@defobject{currentmarker}{\pgfqpoint{-0.048611in}{0.000000in}}{\pgfqpoint{-0.000000in}{0.000000in}}{%
\pgfpathmoveto{\pgfqpoint{-0.000000in}{0.000000in}}%
\pgfpathlineto{\pgfqpoint{-0.048611in}{0.000000in}}%
\pgfusepath{stroke,fill}%
}%
\begin{pgfscope}%
\pgfsys@transformshift{3.750134in}{3.226180in}%
\pgfsys@useobject{currentmarker}{}%
\end{pgfscope}%
\end{pgfscope}%
\begin{pgfscope}%
\definecolor{textcolor}{rgb}{0.000000,0.000000,0.000000}%
\pgfsetstrokecolor{textcolor}%
\pgfsetfillcolor{textcolor}%
\pgftext[x=3.384003in, y=3.187625in, left, base]{\color{textcolor}\rmfamily\fontsize{8.000000}{9.600000}\selectfont \(\displaystyle {0.032}\)}%
\end{pgfscope}%
\begin{pgfscope}%
\pgfsetbuttcap%
\pgfsetroundjoin%
\definecolor{currentfill}{rgb}{0.000000,0.000000,0.000000}%
\pgfsetfillcolor{currentfill}%
\pgfsetlinewidth{0.803000pt}%
\definecolor{currentstroke}{rgb}{0.000000,0.000000,0.000000}%
\pgfsetstrokecolor{currentstroke}%
\pgfsetdash{}{0pt}%
\pgfsys@defobject{currentmarker}{\pgfqpoint{-0.048611in}{0.000000in}}{\pgfqpoint{-0.000000in}{0.000000in}}{%
\pgfpathmoveto{\pgfqpoint{-0.000000in}{0.000000in}}%
\pgfpathlineto{\pgfqpoint{-0.048611in}{0.000000in}}%
\pgfusepath{stroke,fill}%
}%
\begin{pgfscope}%
\pgfsys@transformshift{3.750134in}{3.408684in}%
\pgfsys@useobject{currentmarker}{}%
\end{pgfscope}%
\end{pgfscope}%
\begin{pgfscope}%
\definecolor{textcolor}{rgb}{0.000000,0.000000,0.000000}%
\pgfsetstrokecolor{textcolor}%
\pgfsetfillcolor{textcolor}%
\pgftext[x=3.384003in, y=3.370128in, left, base]{\color{textcolor}\rmfamily\fontsize{8.000000}{9.600000}\selectfont \(\displaystyle {0.040}\)}%
\end{pgfscope}%
\begin{pgfscope}%
\pgfsetbuttcap%
\pgfsetroundjoin%
\definecolor{currentfill}{rgb}{0.000000,0.000000,0.000000}%
\pgfsetfillcolor{currentfill}%
\pgfsetlinewidth{0.803000pt}%
\definecolor{currentstroke}{rgb}{0.000000,0.000000,0.000000}%
\pgfsetstrokecolor{currentstroke}%
\pgfsetdash{}{0pt}%
\pgfsys@defobject{currentmarker}{\pgfqpoint{-0.048611in}{0.000000in}}{\pgfqpoint{-0.000000in}{0.000000in}}{%
\pgfpathmoveto{\pgfqpoint{-0.000000in}{0.000000in}}%
\pgfpathlineto{\pgfqpoint{-0.048611in}{0.000000in}}%
\pgfusepath{stroke,fill}%
}%
\begin{pgfscope}%
\pgfsys@transformshift{3.750134in}{3.591187in}%
\pgfsys@useobject{currentmarker}{}%
\end{pgfscope}%
\end{pgfscope}%
\begin{pgfscope}%
\definecolor{textcolor}{rgb}{0.000000,0.000000,0.000000}%
\pgfsetstrokecolor{textcolor}%
\pgfsetfillcolor{textcolor}%
\pgftext[x=3.384003in, y=3.552632in, left, base]{\color{textcolor}\rmfamily\fontsize{8.000000}{9.600000}\selectfont \(\displaystyle {0.048}\)}%
\end{pgfscope}%
\begin{pgfscope}%
\pgfsetbuttcap%
\pgfsetroundjoin%
\definecolor{currentfill}{rgb}{0.000000,0.000000,0.000000}%
\pgfsetfillcolor{currentfill}%
\pgfsetlinewidth{0.803000pt}%
\definecolor{currentstroke}{rgb}{0.000000,0.000000,0.000000}%
\pgfsetstrokecolor{currentstroke}%
\pgfsetdash{}{0pt}%
\pgfsys@defobject{currentmarker}{\pgfqpoint{-0.048611in}{0.000000in}}{\pgfqpoint{-0.000000in}{0.000000in}}{%
\pgfpathmoveto{\pgfqpoint{-0.000000in}{0.000000in}}%
\pgfpathlineto{\pgfqpoint{-0.048611in}{0.000000in}}%
\pgfusepath{stroke,fill}%
}%
\begin{pgfscope}%
\pgfsys@transformshift{3.750134in}{3.773691in}%
\pgfsys@useobject{currentmarker}{}%
\end{pgfscope}%
\end{pgfscope}%
\begin{pgfscope}%
\definecolor{textcolor}{rgb}{0.000000,0.000000,0.000000}%
\pgfsetstrokecolor{textcolor}%
\pgfsetfillcolor{textcolor}%
\pgftext[x=3.384003in, y=3.735135in, left, base]{\color{textcolor}\rmfamily\fontsize{8.000000}{9.600000}\selectfont \(\displaystyle {0.056}\)}%
\end{pgfscope}%
\begin{pgfscope}%
\definecolor{textcolor}{rgb}{0.000000,0.000000,0.000000}%
\pgfsetstrokecolor{textcolor}%
\pgfsetfillcolor{textcolor}%
\pgftext[x=3.328448in,y=3.204903in,,bottom,rotate=90.000000]{\color{textcolor}\rmfamily\fontsize{10.000000}{12.000000}\selectfont Density}%
\end{pgfscope}%
\begin{pgfscope}%
\pgfpathrectangle{\pgfqpoint{3.750134in}{2.496167in}}{\pgfqpoint{2.301066in}{1.417472in}}%
\pgfusepath{clip}%
\pgfsetbuttcap%
\pgfsetmiterjoin%
\pgfsetlinewidth{1.003750pt}%
\definecolor{currentstroke}{rgb}{0.313725,0.317647,0.309804}%
\pgfsetstrokecolor{currentstroke}%
\pgfsetdash{}{0pt}%
\pgfpathmoveto{\pgfqpoint{3.911232in}{2.496167in}}%
\pgfpathlineto{\pgfqpoint{3.911232in}{2.497098in}}%
\pgfpathlineto{\pgfqpoint{4.186027in}{2.497098in}}%
\pgfpathlineto{\pgfqpoint{4.186027in}{2.498495in}}%
\pgfpathlineto{\pgfqpoint{4.191752in}{2.498495in}}%
\pgfpathlineto{\pgfqpoint{4.191752in}{2.496632in}}%
\pgfpathlineto{\pgfqpoint{4.214652in}{2.497564in}}%
\pgfpathlineto{\pgfqpoint{4.214652in}{2.498029in}}%
\pgfpathlineto{\pgfqpoint{4.237552in}{2.497098in}}%
\pgfpathlineto{\pgfqpoint{4.237552in}{2.501289in}}%
\pgfpathlineto{\pgfqpoint{4.243277in}{2.501289in}}%
\pgfpathlineto{\pgfqpoint{4.243277in}{2.498029in}}%
\pgfpathlineto{\pgfqpoint{4.271901in}{2.498495in}}%
\pgfpathlineto{\pgfqpoint{4.271901in}{2.499427in}}%
\pgfpathlineto{\pgfqpoint{4.283351in}{2.499892in}}%
\pgfpathlineto{\pgfqpoint{4.283351in}{2.501289in}}%
\pgfpathlineto{\pgfqpoint{4.289076in}{2.501289in}}%
\pgfpathlineto{\pgfqpoint{4.289076in}{2.497098in}}%
\pgfpathlineto{\pgfqpoint{4.294801in}{2.497098in}}%
\pgfpathlineto{\pgfqpoint{4.294801in}{2.499427in}}%
\pgfpathlineto{\pgfqpoint{4.306251in}{2.499427in}}%
\pgfpathlineto{\pgfqpoint{4.306251in}{2.501755in}}%
\pgfpathlineto{\pgfqpoint{4.311975in}{2.501755in}}%
\pgfpathlineto{\pgfqpoint{4.311975in}{2.499427in}}%
\pgfpathlineto{\pgfqpoint{4.317700in}{2.499427in}}%
\pgfpathlineto{\pgfqpoint{4.317700in}{2.501755in}}%
\pgfpathlineto{\pgfqpoint{4.346325in}{2.501755in}}%
\pgfpathlineto{\pgfqpoint{4.346325in}{2.499892in}}%
\pgfpathlineto{\pgfqpoint{4.352050in}{2.499892in}}%
\pgfpathlineto{\pgfqpoint{4.352050in}{2.501289in}}%
\pgfpathlineto{\pgfqpoint{4.357775in}{2.501289in}}%
\pgfpathlineto{\pgfqpoint{4.357775in}{2.504084in}}%
\pgfpathlineto{\pgfqpoint{4.363500in}{2.504084in}}%
\pgfpathlineto{\pgfqpoint{4.363500in}{2.505481in}}%
\pgfpathlineto{\pgfqpoint{4.369225in}{2.505481in}}%
\pgfpathlineto{\pgfqpoint{4.369225in}{2.503152in}}%
\pgfpathlineto{\pgfqpoint{4.392124in}{2.503618in}}%
\pgfpathlineto{\pgfqpoint{4.392124in}{2.506412in}}%
\pgfpathlineto{\pgfqpoint{4.397849in}{2.506412in}}%
\pgfpathlineto{\pgfqpoint{4.397849in}{2.503618in}}%
\pgfpathlineto{\pgfqpoint{4.409299in}{2.504550in}}%
\pgfpathlineto{\pgfqpoint{4.409299in}{2.505015in}}%
\pgfpathlineto{\pgfqpoint{4.415024in}{2.505015in}}%
\pgfpathlineto{\pgfqpoint{4.415024in}{2.508275in}}%
\pgfpathlineto{\pgfqpoint{4.460823in}{2.509207in}}%
\pgfpathlineto{\pgfqpoint{4.460823in}{2.509672in}}%
\pgfpathlineto{\pgfqpoint{4.466548in}{2.509672in}}%
\pgfpathlineto{\pgfqpoint{4.466548in}{2.513864in}}%
\pgfpathlineto{\pgfqpoint{4.472273in}{2.513864in}}%
\pgfpathlineto{\pgfqpoint{4.472273in}{2.509207in}}%
\pgfpathlineto{\pgfqpoint{4.477998in}{2.509207in}}%
\pgfpathlineto{\pgfqpoint{4.477998in}{2.518055in}}%
\pgfpathlineto{\pgfqpoint{4.506622in}{2.517124in}}%
\pgfpathlineto{\pgfqpoint{4.506622in}{2.510604in}}%
\pgfpathlineto{\pgfqpoint{4.512347in}{2.510604in}}%
\pgfpathlineto{\pgfqpoint{4.512347in}{2.518521in}}%
\pgfpathlineto{\pgfqpoint{4.518072in}{2.518521in}}%
\pgfpathlineto{\pgfqpoint{4.518072in}{2.522713in}}%
\pgfpathlineto{\pgfqpoint{4.523797in}{2.522713in}}%
\pgfpathlineto{\pgfqpoint{4.523797in}{2.520850in}}%
\pgfpathlineto{\pgfqpoint{4.529522in}{2.520850in}}%
\pgfpathlineto{\pgfqpoint{4.529522in}{2.523178in}}%
\pgfpathlineto{\pgfqpoint{4.540972in}{2.523178in}}%
\pgfpathlineto{\pgfqpoint{4.540972in}{2.524576in}}%
\pgfpathlineto{\pgfqpoint{4.546697in}{2.524576in}}%
\pgfpathlineto{\pgfqpoint{4.546697in}{2.521315in}}%
\pgfpathlineto{\pgfqpoint{4.552422in}{2.521315in}}%
\pgfpathlineto{\pgfqpoint{4.552422in}{2.525041in}}%
\pgfpathlineto{\pgfqpoint{4.563872in}{2.524576in}}%
\pgfpathlineto{\pgfqpoint{4.563872in}{2.528767in}}%
\pgfpathlineto{\pgfqpoint{4.569596in}{2.528767in}}%
\pgfpathlineto{\pgfqpoint{4.569596in}{2.527370in}}%
\pgfpathlineto{\pgfqpoint{4.575321in}{2.527370in}}%
\pgfpathlineto{\pgfqpoint{4.575321in}{2.532027in}}%
\pgfpathlineto{\pgfqpoint{4.581046in}{2.532027in}}%
\pgfpathlineto{\pgfqpoint{4.581046in}{2.540876in}}%
\pgfpathlineto{\pgfqpoint{4.586771in}{2.540876in}}%
\pgfpathlineto{\pgfqpoint{4.586771in}{2.533424in}}%
\pgfpathlineto{\pgfqpoint{4.598221in}{2.532958in}}%
\pgfpathlineto{\pgfqpoint{4.598221in}{2.534821in}}%
\pgfpathlineto{\pgfqpoint{4.603946in}{2.534821in}}%
\pgfpathlineto{\pgfqpoint{4.603946in}{2.545999in}}%
\pgfpathlineto{\pgfqpoint{4.615396in}{2.545067in}}%
\pgfpathlineto{\pgfqpoint{4.615396in}{2.541807in}}%
\pgfpathlineto{\pgfqpoint{4.621121in}{2.541807in}}%
\pgfpathlineto{\pgfqpoint{4.621121in}{2.551587in}}%
\pgfpathlineto{\pgfqpoint{4.626846in}{2.551587in}}%
\pgfpathlineto{\pgfqpoint{4.626846in}{2.547396in}}%
\pgfpathlineto{\pgfqpoint{4.632571in}{2.547396in}}%
\pgfpathlineto{\pgfqpoint{4.632571in}{2.556710in}}%
\pgfpathlineto{\pgfqpoint{4.644020in}{2.556710in}}%
\pgfpathlineto{\pgfqpoint{4.644020in}{2.559970in}}%
\pgfpathlineto{\pgfqpoint{4.649745in}{2.559970in}}%
\pgfpathlineto{\pgfqpoint{4.649745in}{2.558107in}}%
\pgfpathlineto{\pgfqpoint{4.655470in}{2.558107in}}%
\pgfpathlineto{\pgfqpoint{4.655470in}{2.562765in}}%
\pgfpathlineto{\pgfqpoint{4.661195in}{2.562765in}}%
\pgfpathlineto{\pgfqpoint{4.661195in}{2.576270in}}%
\pgfpathlineto{\pgfqpoint{4.666920in}{2.576270in}}%
\pgfpathlineto{\pgfqpoint{4.666920in}{2.569750in}}%
\pgfpathlineto{\pgfqpoint{4.672645in}{2.569750in}}%
\pgfpathlineto{\pgfqpoint{4.672645in}{2.561833in}}%
\pgfpathlineto{\pgfqpoint{4.678370in}{2.561833in}}%
\pgfpathlineto{\pgfqpoint{4.678370in}{2.585119in}}%
\pgfpathlineto{\pgfqpoint{4.684095in}{2.585119in}}%
\pgfpathlineto{\pgfqpoint{4.684095in}{2.579065in}}%
\pgfpathlineto{\pgfqpoint{4.689820in}{2.579065in}}%
\pgfpathlineto{\pgfqpoint{4.689820in}{2.594434in}}%
\pgfpathlineto{\pgfqpoint{4.695545in}{2.594434in}}%
\pgfpathlineto{\pgfqpoint{4.695545in}{2.603748in}}%
\pgfpathlineto{\pgfqpoint{4.701269in}{2.603748in}}%
\pgfpathlineto{\pgfqpoint{4.701269in}{2.598159in}}%
\pgfpathlineto{\pgfqpoint{4.706994in}{2.598159in}}%
\pgfpathlineto{\pgfqpoint{4.706994in}{2.606077in}}%
\pgfpathlineto{\pgfqpoint{4.712719in}{2.606077in}}%
\pgfpathlineto{\pgfqpoint{4.712719in}{2.609802in}}%
\pgfpathlineto{\pgfqpoint{4.718444in}{2.609802in}}%
\pgfpathlineto{\pgfqpoint{4.718444in}{2.615857in}}%
\pgfpathlineto{\pgfqpoint{4.724169in}{2.615857in}}%
\pgfpathlineto{\pgfqpoint{4.724169in}{2.631691in}}%
\pgfpathlineto{\pgfqpoint{4.729894in}{2.631691in}}%
\pgfpathlineto{\pgfqpoint{4.729894in}{2.626103in}}%
\pgfpathlineto{\pgfqpoint{4.735619in}{2.626103in}}%
\pgfpathlineto{\pgfqpoint{4.735619in}{2.646594in}}%
\pgfpathlineto{\pgfqpoint{4.741344in}{2.646594in}}%
\pgfpathlineto{\pgfqpoint{4.741344in}{2.645197in}}%
\pgfpathlineto{\pgfqpoint{4.747069in}{2.645197in}}%
\pgfpathlineto{\pgfqpoint{4.747069in}{2.636814in}}%
\pgfpathlineto{\pgfqpoint{4.752794in}{2.636814in}}%
\pgfpathlineto{\pgfqpoint{4.752794in}{2.666620in}}%
\pgfpathlineto{\pgfqpoint{4.758519in}{2.666620in}}%
\pgfpathlineto{\pgfqpoint{4.758519in}{2.688975in}}%
\pgfpathlineto{\pgfqpoint{4.764243in}{2.688975in}}%
\pgfpathlineto{\pgfqpoint{4.764243in}{2.682455in}}%
\pgfpathlineto{\pgfqpoint{4.769968in}{2.682455in}}%
\pgfpathlineto{\pgfqpoint{4.769968in}{2.699686in}}%
\pgfpathlineto{\pgfqpoint{4.775693in}{2.699686in}}%
\pgfpathlineto{\pgfqpoint{4.775693in}{2.709001in}}%
\pgfpathlineto{\pgfqpoint{4.781418in}{2.709001in}}%
\pgfpathlineto{\pgfqpoint{4.781418in}{2.717384in}}%
\pgfpathlineto{\pgfqpoint{4.787143in}{2.717384in}}%
\pgfpathlineto{\pgfqpoint{4.787143in}{2.759764in}}%
\pgfpathlineto{\pgfqpoint{4.792868in}{2.759764in}}%
\pgfpathlineto{\pgfqpoint{4.792868in}{2.790036in}}%
\pgfpathlineto{\pgfqpoint{4.798593in}{2.790036in}}%
\pgfpathlineto{\pgfqpoint{4.798593in}{2.761627in}}%
\pgfpathlineto{\pgfqpoint{4.804318in}{2.761627in}}%
\pgfpathlineto{\pgfqpoint{4.804318in}{2.810062in}}%
\pgfpathlineto{\pgfqpoint{4.810043in}{2.810062in}}%
\pgfpathlineto{\pgfqpoint{4.810043in}{2.811925in}}%
\pgfpathlineto{\pgfqpoint{4.815768in}{2.811925in}}%
\pgfpathlineto{\pgfqpoint{4.815768in}{2.843128in}}%
\pgfpathlineto{\pgfqpoint{4.821493in}{2.843128in}}%
\pgfpathlineto{\pgfqpoint{4.821493in}{2.904603in}}%
\pgfpathlineto{\pgfqpoint{4.827217in}{2.904603in}}%
\pgfpathlineto{\pgfqpoint{4.827217in}{2.929752in}}%
\pgfpathlineto{\pgfqpoint{4.832942in}{2.929752in}}%
\pgfpathlineto{\pgfqpoint{4.832942in}{2.972133in}}%
\pgfpathlineto{\pgfqpoint{4.838667in}{2.972133in}}%
\pgfpathlineto{\pgfqpoint{4.838667in}{3.010787in}}%
\pgfpathlineto{\pgfqpoint{4.844392in}{3.010787in}}%
\pgfpathlineto{\pgfqpoint{4.844392in}{3.055962in}}%
\pgfpathlineto{\pgfqpoint{4.850117in}{3.055962in}}%
\pgfpathlineto{\pgfqpoint{4.850117in}{3.105329in}}%
\pgfpathlineto{\pgfqpoint{4.855842in}{3.105329in}}%
\pgfpathlineto{\pgfqpoint{4.855842in}{3.187761in}}%
\pgfpathlineto{\pgfqpoint{4.861567in}{3.187761in}}%
\pgfpathlineto{\pgfqpoint{4.861567in}{3.246908in}}%
\pgfpathlineto{\pgfqpoint{4.867292in}{3.246908in}}%
\pgfpathlineto{\pgfqpoint{4.867292in}{3.287891in}}%
\pgfpathlineto{\pgfqpoint{4.873017in}{3.287891in}}%
\pgfpathlineto{\pgfqpoint{4.873017in}{3.341915in}}%
\pgfpathlineto{\pgfqpoint{4.878742in}{3.341915in}}%
\pgfpathlineto{\pgfqpoint{4.878742in}{3.392212in}}%
\pgfpathlineto{\pgfqpoint{4.884467in}{3.392212in}}%
\pgfpathlineto{\pgfqpoint{4.884467in}{3.415498in}}%
\pgfpathlineto{\pgfqpoint{4.890192in}{3.415498in}}%
\pgfpathlineto{\pgfqpoint{4.890192in}{3.448099in}}%
\pgfpathlineto{\pgfqpoint{4.895916in}{3.448099in}}%
\pgfpathlineto{\pgfqpoint{4.895916in}{3.422018in}}%
\pgfpathlineto{\pgfqpoint{4.901641in}{3.422018in}}%
\pgfpathlineto{\pgfqpoint{4.901641in}{3.409444in}}%
\pgfpathlineto{\pgfqpoint{4.907366in}{3.409444in}}%
\pgfpathlineto{\pgfqpoint{4.907366in}{3.396404in}}%
\pgfpathlineto{\pgfqpoint{4.913091in}{3.396404in}}%
\pgfpathlineto{\pgfqpoint{4.913091in}{3.341915in}}%
\pgfpathlineto{\pgfqpoint{4.918816in}{3.341915in}}%
\pgfpathlineto{\pgfqpoint{4.918816in}{3.313506in}}%
\pgfpathlineto{\pgfqpoint{4.924541in}{3.313506in}}%
\pgfpathlineto{\pgfqpoint{4.924541in}{3.276248in}}%
\pgfpathlineto{\pgfqpoint{4.930266in}{3.276248in}}%
\pgfpathlineto{\pgfqpoint{4.930266in}{3.173324in}}%
\pgfpathlineto{\pgfqpoint{4.935991in}{3.173324in}}%
\pgfpathlineto{\pgfqpoint{4.935991in}{3.121163in}}%
\pgfpathlineto{\pgfqpoint{4.941716in}{3.121163in}}%
\pgfpathlineto{\pgfqpoint{4.941716in}{3.075523in}}%
\pgfpathlineto{\pgfqpoint{4.947441in}{3.075523in}}%
\pgfpathlineto{\pgfqpoint{4.947441in}{3.038731in}}%
\pgfpathlineto{\pgfqpoint{4.953166in}{3.038731in}}%
\pgfpathlineto{\pgfqpoint{4.953166in}{2.975858in}}%
\pgfpathlineto{\pgfqpoint{4.958890in}{2.975858in}}%
\pgfpathlineto{\pgfqpoint{4.958890in}{2.923698in}}%
\pgfpathlineto{\pgfqpoint{4.964615in}{2.923698in}}%
\pgfpathlineto{\pgfqpoint{4.964615in}{2.919041in}}%
\pgfpathlineto{\pgfqpoint{4.970340in}{2.919041in}}%
\pgfpathlineto{\pgfqpoint{4.970340in}{2.869674in}}%
\pgfpathlineto{\pgfqpoint{4.976065in}{2.869674in}}%
\pgfpathlineto{\pgfqpoint{4.976065in}{2.831951in}}%
\pgfpathlineto{\pgfqpoint{4.981790in}{2.831951in}}%
\pgfpathlineto{\pgfqpoint{4.981790in}{2.797022in}}%
\pgfpathlineto{\pgfqpoint{4.987515in}{2.797022in}}%
\pgfpathlineto{\pgfqpoint{4.987515in}{2.789570in}}%
\pgfpathlineto{\pgfqpoint{4.993240in}{2.789570in}}%
\pgfpathlineto{\pgfqpoint{4.993240in}{2.785845in}}%
\pgfpathlineto{\pgfqpoint{4.998965in}{2.785845in}}%
\pgfpathlineto{\pgfqpoint{4.998965in}{2.743464in}}%
\pgfpathlineto{\pgfqpoint{5.004690in}{2.743464in}}%
\pgfpathlineto{\pgfqpoint{5.004690in}{2.723438in}}%
\pgfpathlineto{\pgfqpoint{5.010415in}{2.723438in}}%
\pgfpathlineto{\pgfqpoint{5.010415in}{2.721575in}}%
\pgfpathlineto{\pgfqpoint{5.016140in}{2.721575in}}%
\pgfpathlineto{\pgfqpoint{5.016140in}{2.699221in}}%
\pgfpathlineto{\pgfqpoint{5.021864in}{2.699221in}}%
\pgfpathlineto{\pgfqpoint{5.021864in}{2.703878in}}%
\pgfpathlineto{\pgfqpoint{5.027589in}{2.703878in}}%
\pgfpathlineto{\pgfqpoint{5.027589in}{2.672675in}}%
\pgfpathlineto{\pgfqpoint{5.033314in}{2.672675in}}%
\pgfpathlineto{\pgfqpoint{5.033314in}{2.659169in}}%
\pgfpathlineto{\pgfqpoint{5.044764in}{2.659634in}}%
\pgfpathlineto{\pgfqpoint{5.044764in}{2.653114in}}%
\pgfpathlineto{\pgfqpoint{5.050489in}{2.653114in}}%
\pgfpathlineto{\pgfqpoint{5.050489in}{2.624705in}}%
\pgfpathlineto{\pgfqpoint{5.056214in}{2.624705in}}%
\pgfpathlineto{\pgfqpoint{5.056214in}{2.630760in}}%
\pgfpathlineto{\pgfqpoint{5.061939in}{2.630760in}}%
\pgfpathlineto{\pgfqpoint{5.061939in}{2.619582in}}%
\pgfpathlineto{\pgfqpoint{5.067664in}{2.619582in}}%
\pgfpathlineto{\pgfqpoint{5.067664in}{2.624705in}}%
\pgfpathlineto{\pgfqpoint{5.073389in}{2.624705in}}%
\pgfpathlineto{\pgfqpoint{5.073389in}{2.599556in}}%
\pgfpathlineto{\pgfqpoint{5.079114in}{2.599556in}}%
\pgfpathlineto{\pgfqpoint{5.079114in}{2.609337in}}%
\pgfpathlineto{\pgfqpoint{5.084838in}{2.609337in}}%
\pgfpathlineto{\pgfqpoint{5.084838in}{2.594899in}}%
\pgfpathlineto{\pgfqpoint{5.096288in}{2.593968in}}%
\pgfpathlineto{\pgfqpoint{5.096288in}{2.585119in}}%
\pgfpathlineto{\pgfqpoint{5.107738in}{2.584653in}}%
\pgfpathlineto{\pgfqpoint{5.107738in}{2.587913in}}%
\pgfpathlineto{\pgfqpoint{5.113463in}{2.587913in}}%
\pgfpathlineto{\pgfqpoint{5.113463in}{2.575339in}}%
\pgfpathlineto{\pgfqpoint{5.119188in}{2.575339in}}%
\pgfpathlineto{\pgfqpoint{5.119188in}{2.570682in}}%
\pgfpathlineto{\pgfqpoint{5.124913in}{2.570682in}}%
\pgfpathlineto{\pgfqpoint{5.124913in}{2.573476in}}%
\pgfpathlineto{\pgfqpoint{5.130638in}{2.573476in}}%
\pgfpathlineto{\pgfqpoint{5.130638in}{2.563230in}}%
\pgfpathlineto{\pgfqpoint{5.136363in}{2.563230in}}%
\pgfpathlineto{\pgfqpoint{5.136363in}{2.572079in}}%
\pgfpathlineto{\pgfqpoint{5.142088in}{2.572079in}}%
\pgfpathlineto{\pgfqpoint{5.142088in}{2.552519in}}%
\pgfpathlineto{\pgfqpoint{5.147813in}{2.552519in}}%
\pgfpathlineto{\pgfqpoint{5.147813in}{2.559505in}}%
\pgfpathlineto{\pgfqpoint{5.159262in}{2.559039in}}%
\pgfpathlineto{\pgfqpoint{5.159262in}{2.555313in}}%
\pgfpathlineto{\pgfqpoint{5.164987in}{2.555313in}}%
\pgfpathlineto{\pgfqpoint{5.164987in}{2.552053in}}%
\pgfpathlineto{\pgfqpoint{5.170712in}{2.552053in}}%
\pgfpathlineto{\pgfqpoint{5.170712in}{2.546930in}}%
\pgfpathlineto{\pgfqpoint{5.176437in}{2.546930in}}%
\pgfpathlineto{\pgfqpoint{5.176437in}{2.543670in}}%
\pgfpathlineto{\pgfqpoint{5.182162in}{2.543670in}}%
\pgfpathlineto{\pgfqpoint{5.182162in}{2.548793in}}%
\pgfpathlineto{\pgfqpoint{5.187887in}{2.548793in}}%
\pgfpathlineto{\pgfqpoint{5.187887in}{2.544601in}}%
\pgfpathlineto{\pgfqpoint{5.193612in}{2.544601in}}%
\pgfpathlineto{\pgfqpoint{5.193612in}{2.542273in}}%
\pgfpathlineto{\pgfqpoint{5.199337in}{2.542273in}}%
\pgfpathlineto{\pgfqpoint{5.199337in}{2.539479in}}%
\pgfpathlineto{\pgfqpoint{5.205062in}{2.539479in}}%
\pgfpathlineto{\pgfqpoint{5.205062in}{2.548327in}}%
\pgfpathlineto{\pgfqpoint{5.210787in}{2.548327in}}%
\pgfpathlineto{\pgfqpoint{5.210787in}{2.535287in}}%
\pgfpathlineto{\pgfqpoint{5.216511in}{2.535287in}}%
\pgfpathlineto{\pgfqpoint{5.216511in}{2.530630in}}%
\pgfpathlineto{\pgfqpoint{5.222236in}{2.530630in}}%
\pgfpathlineto{\pgfqpoint{5.222236in}{2.528301in}}%
\pgfpathlineto{\pgfqpoint{5.227961in}{2.528301in}}%
\pgfpathlineto{\pgfqpoint{5.227961in}{2.535753in}}%
\pgfpathlineto{\pgfqpoint{5.233686in}{2.535753in}}%
\pgfpathlineto{\pgfqpoint{5.233686in}{2.525507in}}%
\pgfpathlineto{\pgfqpoint{5.239411in}{2.525507in}}%
\pgfpathlineto{\pgfqpoint{5.239411in}{2.527836in}}%
\pgfpathlineto{\pgfqpoint{5.250861in}{2.527370in}}%
\pgfpathlineto{\pgfqpoint{5.250861in}{2.522713in}}%
\pgfpathlineto{\pgfqpoint{5.256586in}{2.522713in}}%
\pgfpathlineto{\pgfqpoint{5.256586in}{2.518987in}}%
\pgfpathlineto{\pgfqpoint{5.268036in}{2.519918in}}%
\pgfpathlineto{\pgfqpoint{5.268036in}{2.524576in}}%
\pgfpathlineto{\pgfqpoint{5.273761in}{2.524576in}}%
\pgfpathlineto{\pgfqpoint{5.273761in}{2.515261in}}%
\pgfpathlineto{\pgfqpoint{5.279485in}{2.515261in}}%
\pgfpathlineto{\pgfqpoint{5.279485in}{2.517590in}}%
\pgfpathlineto{\pgfqpoint{5.285210in}{2.517590in}}%
\pgfpathlineto{\pgfqpoint{5.285210in}{2.519918in}}%
\pgfpathlineto{\pgfqpoint{5.290935in}{2.519918in}}%
\pgfpathlineto{\pgfqpoint{5.290935in}{2.514330in}}%
\pgfpathlineto{\pgfqpoint{5.296660in}{2.514330in}}%
\pgfpathlineto{\pgfqpoint{5.296660in}{2.516658in}}%
\pgfpathlineto{\pgfqpoint{5.302385in}{2.516658in}}%
\pgfpathlineto{\pgfqpoint{5.302385in}{2.514330in}}%
\pgfpathlineto{\pgfqpoint{5.308110in}{2.514330in}}%
\pgfpathlineto{\pgfqpoint{5.308110in}{2.518987in}}%
\pgfpathlineto{\pgfqpoint{5.313835in}{2.518987in}}%
\pgfpathlineto{\pgfqpoint{5.313835in}{2.514795in}}%
\pgfpathlineto{\pgfqpoint{5.319560in}{2.514795in}}%
\pgfpathlineto{\pgfqpoint{5.319560in}{2.508741in}}%
\pgfpathlineto{\pgfqpoint{5.325285in}{2.508741in}}%
\pgfpathlineto{\pgfqpoint{5.325285in}{2.507344in}}%
\pgfpathlineto{\pgfqpoint{5.331010in}{2.507344in}}%
\pgfpathlineto{\pgfqpoint{5.331010in}{2.514795in}}%
\pgfpathlineto{\pgfqpoint{5.336735in}{2.514795in}}%
\pgfpathlineto{\pgfqpoint{5.336735in}{2.509207in}}%
\pgfpathlineto{\pgfqpoint{5.342459in}{2.509207in}}%
\pgfpathlineto{\pgfqpoint{5.342459in}{2.510604in}}%
\pgfpathlineto{\pgfqpoint{5.348184in}{2.510604in}}%
\pgfpathlineto{\pgfqpoint{5.348184in}{2.505947in}}%
\pgfpathlineto{\pgfqpoint{5.353909in}{2.505947in}}%
\pgfpathlineto{\pgfqpoint{5.353909in}{2.508275in}}%
\pgfpathlineto{\pgfqpoint{5.365359in}{2.508741in}}%
\pgfpathlineto{\pgfqpoint{5.365359in}{2.504550in}}%
\pgfpathlineto{\pgfqpoint{5.376809in}{2.505481in}}%
\pgfpathlineto{\pgfqpoint{5.376809in}{2.508275in}}%
\pgfpathlineto{\pgfqpoint{5.388259in}{2.508275in}}%
\pgfpathlineto{\pgfqpoint{5.388259in}{2.505947in}}%
\pgfpathlineto{\pgfqpoint{5.393984in}{2.505947in}}%
\pgfpathlineto{\pgfqpoint{5.393984in}{2.502687in}}%
\pgfpathlineto{\pgfqpoint{5.399709in}{2.502687in}}%
\pgfpathlineto{\pgfqpoint{5.399709in}{2.505015in}}%
\pgfpathlineto{\pgfqpoint{5.411158in}{2.505481in}}%
\pgfpathlineto{\pgfqpoint{5.411158in}{2.503152in}}%
\pgfpathlineto{\pgfqpoint{5.416883in}{2.503152in}}%
\pgfpathlineto{\pgfqpoint{5.416883in}{2.506412in}}%
\pgfpathlineto{\pgfqpoint{5.422608in}{2.506412in}}%
\pgfpathlineto{\pgfqpoint{5.422608in}{2.503618in}}%
\pgfpathlineto{\pgfqpoint{5.434058in}{2.503152in}}%
\pgfpathlineto{\pgfqpoint{5.434058in}{2.500358in}}%
\pgfpathlineto{\pgfqpoint{5.445508in}{2.500824in}}%
\pgfpathlineto{\pgfqpoint{5.445508in}{2.505947in}}%
\pgfpathlineto{\pgfqpoint{5.451233in}{2.505947in}}%
\pgfpathlineto{\pgfqpoint{5.451233in}{2.500824in}}%
\pgfpathlineto{\pgfqpoint{5.456958in}{2.500824in}}%
\pgfpathlineto{\pgfqpoint{5.456958in}{2.502687in}}%
\pgfpathlineto{\pgfqpoint{5.474132in}{2.503618in}}%
\pgfpathlineto{\pgfqpoint{5.474132in}{2.500358in}}%
\pgfpathlineto{\pgfqpoint{5.485582in}{2.500358in}}%
\pgfpathlineto{\pgfqpoint{5.485582in}{2.498961in}}%
\pgfpathlineto{\pgfqpoint{5.497032in}{2.498961in}}%
\pgfpathlineto{\pgfqpoint{5.497032in}{2.501289in}}%
\pgfpathlineto{\pgfqpoint{5.502757in}{2.501289in}}%
\pgfpathlineto{\pgfqpoint{5.502757in}{2.499427in}}%
\pgfpathlineto{\pgfqpoint{5.531382in}{2.498961in}}%
\pgfpathlineto{\pgfqpoint{5.531382in}{2.498029in}}%
\pgfpathlineto{\pgfqpoint{5.537106in}{2.498029in}}%
\pgfpathlineto{\pgfqpoint{5.537106in}{2.499427in}}%
\pgfpathlineto{\pgfqpoint{5.542831in}{2.499427in}}%
\pgfpathlineto{\pgfqpoint{5.542831in}{2.497564in}}%
\pgfpathlineto{\pgfqpoint{5.548556in}{2.497564in}}%
\pgfpathlineto{\pgfqpoint{5.548556in}{2.499427in}}%
\pgfpathlineto{\pgfqpoint{5.554281in}{2.499427in}}%
\pgfpathlineto{\pgfqpoint{5.554281in}{2.498029in}}%
\pgfpathlineto{\pgfqpoint{5.560006in}{2.498029in}}%
\pgfpathlineto{\pgfqpoint{5.560006in}{2.500358in}}%
\pgfpathlineto{\pgfqpoint{5.565731in}{2.500358in}}%
\pgfpathlineto{\pgfqpoint{5.565731in}{2.498029in}}%
\pgfpathlineto{\pgfqpoint{5.577181in}{2.497098in}}%
\pgfpathlineto{\pgfqpoint{5.577181in}{2.496167in}}%
\pgfpathlineto{\pgfqpoint{5.582906in}{2.496167in}}%
\pgfpathlineto{\pgfqpoint{5.582906in}{2.498029in}}%
\pgfpathlineto{\pgfqpoint{5.600080in}{2.497564in}}%
\pgfpathlineto{\pgfqpoint{5.600080in}{2.499892in}}%
\pgfpathlineto{\pgfqpoint{5.605805in}{2.499892in}}%
\pgfpathlineto{\pgfqpoint{5.605805in}{2.497564in}}%
\pgfpathlineto{\pgfqpoint{5.674504in}{2.496632in}}%
\pgfpathlineto{\pgfqpoint{5.674504in}{2.496167in}}%
\pgfpathlineto{\pgfqpoint{5.789003in}{2.496632in}}%
\pgfpathlineto{\pgfqpoint{5.789003in}{2.496167in}}%
\pgfusepath{stroke}%
\end{pgfscope}%
\begin{pgfscope}%
\pgfpathrectangle{\pgfqpoint{3.750134in}{2.496167in}}{\pgfqpoint{2.301066in}{1.417472in}}%
\pgfusepath{clip}%
\pgfsetbuttcap%
\pgfsetmiterjoin%
\pgfsetlinewidth{1.003750pt}%
\definecolor{currentstroke}{rgb}{0.949020,0.372549,0.360784}%
\pgfsetstrokecolor{currentstroke}%
\pgfsetdash{{1.000000pt}{1.650000pt}}{0.000000pt}%
\pgfpathmoveto{\pgfqpoint{3.911232in}{2.496167in}}%
\pgfpathlineto{\pgfqpoint{3.911232in}{2.497099in}}%
\pgfpathlineto{\pgfqpoint{4.031455in}{2.496633in}}%
\pgfpathlineto{\pgfqpoint{4.031455in}{2.498964in}}%
\pgfpathlineto{\pgfqpoint{4.037180in}{2.498964in}}%
\pgfpathlineto{\pgfqpoint{4.037180in}{2.496633in}}%
\pgfpathlineto{\pgfqpoint{4.054354in}{2.497099in}}%
\pgfpathlineto{\pgfqpoint{4.054354in}{2.498498in}}%
\pgfpathlineto{\pgfqpoint{4.094429in}{2.497566in}}%
\pgfpathlineto{\pgfqpoint{4.094429in}{2.500363in}}%
\pgfpathlineto{\pgfqpoint{4.100154in}{2.500363in}}%
\pgfpathlineto{\pgfqpoint{4.100154in}{2.498498in}}%
\pgfpathlineto{\pgfqpoint{4.123053in}{2.497566in}}%
\pgfpathlineto{\pgfqpoint{4.123053in}{2.496633in}}%
\pgfpathlineto{\pgfqpoint{4.128778in}{2.496633in}}%
\pgfpathlineto{\pgfqpoint{4.128778in}{2.499897in}}%
\pgfpathlineto{\pgfqpoint{4.140228in}{2.499431in}}%
\pgfpathlineto{\pgfqpoint{4.140228in}{2.498498in}}%
\pgfpathlineto{\pgfqpoint{4.145953in}{2.498498in}}%
\pgfpathlineto{\pgfqpoint{4.145953in}{2.497099in}}%
\pgfpathlineto{\pgfqpoint{4.151678in}{2.497099in}}%
\pgfpathlineto{\pgfqpoint{4.151678in}{2.498964in}}%
\pgfpathlineto{\pgfqpoint{4.157403in}{2.498964in}}%
\pgfpathlineto{\pgfqpoint{4.157403in}{2.497566in}}%
\pgfpathlineto{\pgfqpoint{4.168853in}{2.498498in}}%
\pgfpathlineto{\pgfqpoint{4.168853in}{2.499897in}}%
\pgfpathlineto{\pgfqpoint{4.174578in}{2.499897in}}%
\pgfpathlineto{\pgfqpoint{4.174578in}{2.502229in}}%
\pgfpathlineto{\pgfqpoint{4.180303in}{2.502229in}}%
\pgfpathlineto{\pgfqpoint{4.180303in}{2.499897in}}%
\pgfpathlineto{\pgfqpoint{4.186027in}{2.499897in}}%
\pgfpathlineto{\pgfqpoint{4.186027in}{2.498498in}}%
\pgfpathlineto{\pgfqpoint{4.191752in}{2.498498in}}%
\pgfpathlineto{\pgfqpoint{4.191752in}{2.499897in}}%
\pgfpathlineto{\pgfqpoint{4.203202in}{2.500830in}}%
\pgfpathlineto{\pgfqpoint{4.203202in}{2.501296in}}%
\pgfpathlineto{\pgfqpoint{4.231827in}{2.502229in}}%
\pgfpathlineto{\pgfqpoint{4.231827in}{2.498498in}}%
\pgfpathlineto{\pgfqpoint{4.237552in}{2.498498in}}%
\pgfpathlineto{\pgfqpoint{4.237552in}{2.505027in}}%
\pgfpathlineto{\pgfqpoint{4.243277in}{2.505027in}}%
\pgfpathlineto{\pgfqpoint{4.243277in}{2.502695in}}%
\pgfpathlineto{\pgfqpoint{4.249001in}{2.502695in}}%
\pgfpathlineto{\pgfqpoint{4.249001in}{2.498498in}}%
\pgfpathlineto{\pgfqpoint{4.254726in}{2.498498in}}%
\pgfpathlineto{\pgfqpoint{4.254726in}{2.503628in}}%
\pgfpathlineto{\pgfqpoint{4.260451in}{2.503628in}}%
\pgfpathlineto{\pgfqpoint{4.260451in}{2.500363in}}%
\pgfpathlineto{\pgfqpoint{4.271901in}{2.501296in}}%
\pgfpathlineto{\pgfqpoint{4.271901in}{2.502229in}}%
\pgfpathlineto{\pgfqpoint{4.300526in}{2.502695in}}%
\pgfpathlineto{\pgfqpoint{4.300526in}{2.504560in}}%
\pgfpathlineto{\pgfqpoint{4.311975in}{2.503628in}}%
\pgfpathlineto{\pgfqpoint{4.311975in}{2.506425in}}%
\pgfpathlineto{\pgfqpoint{4.317700in}{2.506425in}}%
\pgfpathlineto{\pgfqpoint{4.317700in}{2.505027in}}%
\pgfpathlineto{\pgfqpoint{4.329150in}{2.504560in}}%
\pgfpathlineto{\pgfqpoint{4.329150in}{2.499897in}}%
\pgfpathlineto{\pgfqpoint{4.334875in}{2.499897in}}%
\pgfpathlineto{\pgfqpoint{4.334875in}{2.505027in}}%
\pgfpathlineto{\pgfqpoint{4.346325in}{2.504560in}}%
\pgfpathlineto{\pgfqpoint{4.346325in}{2.502695in}}%
\pgfpathlineto{\pgfqpoint{4.352050in}{2.502695in}}%
\pgfpathlineto{\pgfqpoint{4.352050in}{2.506892in}}%
\pgfpathlineto{\pgfqpoint{4.357775in}{2.506892in}}%
\pgfpathlineto{\pgfqpoint{4.357775in}{2.504094in}}%
\pgfpathlineto{\pgfqpoint{4.363500in}{2.504094in}}%
\pgfpathlineto{\pgfqpoint{4.363500in}{2.508757in}}%
\pgfpathlineto{\pgfqpoint{4.369225in}{2.508757in}}%
\pgfpathlineto{\pgfqpoint{4.369225in}{2.507358in}}%
\pgfpathlineto{\pgfqpoint{4.374950in}{2.507358in}}%
\pgfpathlineto{\pgfqpoint{4.374950in}{2.512954in}}%
\pgfpathlineto{\pgfqpoint{4.380674in}{2.512954in}}%
\pgfpathlineto{\pgfqpoint{4.380674in}{2.505027in}}%
\pgfpathlineto{\pgfqpoint{4.386399in}{2.505027in}}%
\pgfpathlineto{\pgfqpoint{4.386399in}{2.512021in}}%
\pgfpathlineto{\pgfqpoint{4.392124in}{2.512021in}}%
\pgfpathlineto{\pgfqpoint{4.392124in}{2.510622in}}%
\pgfpathlineto{\pgfqpoint{4.397849in}{2.510622in}}%
\pgfpathlineto{\pgfqpoint{4.397849in}{2.505027in}}%
\pgfpathlineto{\pgfqpoint{4.403574in}{2.505027in}}%
\pgfpathlineto{\pgfqpoint{4.403574in}{2.514353in}}%
\pgfpathlineto{\pgfqpoint{4.409299in}{2.514353in}}%
\pgfpathlineto{\pgfqpoint{4.409299in}{2.510156in}}%
\pgfpathlineto{\pgfqpoint{4.415024in}{2.510156in}}%
\pgfpathlineto{\pgfqpoint{4.415024in}{2.507824in}}%
\pgfpathlineto{\pgfqpoint{4.420749in}{2.507824in}}%
\pgfpathlineto{\pgfqpoint{4.420749in}{2.516684in}}%
\pgfpathlineto{\pgfqpoint{4.426474in}{2.516684in}}%
\pgfpathlineto{\pgfqpoint{4.426474in}{2.515285in}}%
\pgfpathlineto{\pgfqpoint{4.432199in}{2.515285in}}%
\pgfpathlineto{\pgfqpoint{4.432199in}{2.511089in}}%
\pgfpathlineto{\pgfqpoint{4.437924in}{2.511089in}}%
\pgfpathlineto{\pgfqpoint{4.437924in}{2.514353in}}%
\pgfpathlineto{\pgfqpoint{4.443648in}{2.514353in}}%
\pgfpathlineto{\pgfqpoint{4.443648in}{2.515752in}}%
\pgfpathlineto{\pgfqpoint{4.449373in}{2.515752in}}%
\pgfpathlineto{\pgfqpoint{4.449373in}{2.513420in}}%
\pgfpathlineto{\pgfqpoint{4.455098in}{2.513420in}}%
\pgfpathlineto{\pgfqpoint{4.455098in}{2.515752in}}%
\pgfpathlineto{\pgfqpoint{4.460823in}{2.515752in}}%
\pgfpathlineto{\pgfqpoint{4.460823in}{2.519016in}}%
\pgfpathlineto{\pgfqpoint{4.466548in}{2.519016in}}%
\pgfpathlineto{\pgfqpoint{4.466548in}{2.516218in}}%
\pgfpathlineto{\pgfqpoint{4.472273in}{2.516218in}}%
\pgfpathlineto{\pgfqpoint{4.472273in}{2.512021in}}%
\pgfpathlineto{\pgfqpoint{4.477998in}{2.512021in}}%
\pgfpathlineto{\pgfqpoint{4.477998in}{2.516218in}}%
\pgfpathlineto{\pgfqpoint{4.483723in}{2.516218in}}%
\pgfpathlineto{\pgfqpoint{4.483723in}{2.518083in}}%
\pgfpathlineto{\pgfqpoint{4.489448in}{2.518083in}}%
\pgfpathlineto{\pgfqpoint{4.489448in}{2.523213in}}%
\pgfpathlineto{\pgfqpoint{4.495173in}{2.523213in}}%
\pgfpathlineto{\pgfqpoint{4.495173in}{2.521814in}}%
\pgfpathlineto{\pgfqpoint{4.500898in}{2.521814in}}%
\pgfpathlineto{\pgfqpoint{4.500898in}{2.518550in}}%
\pgfpathlineto{\pgfqpoint{4.506622in}{2.518550in}}%
\pgfpathlineto{\pgfqpoint{4.506622in}{2.521347in}}%
\pgfpathlineto{\pgfqpoint{4.512347in}{2.521347in}}%
\pgfpathlineto{\pgfqpoint{4.512347in}{2.526477in}}%
\pgfpathlineto{\pgfqpoint{4.518072in}{2.526477in}}%
\pgfpathlineto{\pgfqpoint{4.518072in}{2.524612in}}%
\pgfpathlineto{\pgfqpoint{4.523797in}{2.524612in}}%
\pgfpathlineto{\pgfqpoint{4.523797in}{2.519948in}}%
\pgfpathlineto{\pgfqpoint{4.529522in}{2.519948in}}%
\pgfpathlineto{\pgfqpoint{4.529522in}{2.526943in}}%
\pgfpathlineto{\pgfqpoint{4.535247in}{2.526943in}}%
\pgfpathlineto{\pgfqpoint{4.535247in}{2.530207in}}%
\pgfpathlineto{\pgfqpoint{4.540972in}{2.530207in}}%
\pgfpathlineto{\pgfqpoint{4.540972in}{2.525078in}}%
\pgfpathlineto{\pgfqpoint{4.546697in}{2.525078in}}%
\pgfpathlineto{\pgfqpoint{4.546697in}{2.528342in}}%
\pgfpathlineto{\pgfqpoint{4.552422in}{2.528342in}}%
\pgfpathlineto{\pgfqpoint{4.552422in}{2.535803in}}%
\pgfpathlineto{\pgfqpoint{4.558147in}{2.535803in}}%
\pgfpathlineto{\pgfqpoint{4.558147in}{2.533005in}}%
\pgfpathlineto{\pgfqpoint{4.563872in}{2.533005in}}%
\pgfpathlineto{\pgfqpoint{4.563872in}{2.529275in}}%
\pgfpathlineto{\pgfqpoint{4.569596in}{2.529275in}}%
\pgfpathlineto{\pgfqpoint{4.569596in}{2.531606in}}%
\pgfpathlineto{\pgfqpoint{4.575321in}{2.531606in}}%
\pgfpathlineto{\pgfqpoint{4.575321in}{2.535803in}}%
\pgfpathlineto{\pgfqpoint{4.581046in}{2.535803in}}%
\pgfpathlineto{\pgfqpoint{4.581046in}{2.527409in}}%
\pgfpathlineto{\pgfqpoint{4.586771in}{2.527409in}}%
\pgfpathlineto{\pgfqpoint{4.586771in}{2.536736in}}%
\pgfpathlineto{\pgfqpoint{4.592496in}{2.536736in}}%
\pgfpathlineto{\pgfqpoint{4.592496in}{2.540000in}}%
\pgfpathlineto{\pgfqpoint{4.603946in}{2.540000in}}%
\pgfpathlineto{\pgfqpoint{4.603946in}{2.543730in}}%
\pgfpathlineto{\pgfqpoint{4.609671in}{2.543730in}}%
\pgfpathlineto{\pgfqpoint{4.609671in}{2.536269in}}%
\pgfpathlineto{\pgfqpoint{4.615396in}{2.536269in}}%
\pgfpathlineto{\pgfqpoint{4.615396in}{2.540000in}}%
\pgfpathlineto{\pgfqpoint{4.621121in}{2.540000in}}%
\pgfpathlineto{\pgfqpoint{4.621121in}{2.550259in}}%
\pgfpathlineto{\pgfqpoint{4.626846in}{2.550259in}}%
\pgfpathlineto{\pgfqpoint{4.626846in}{2.542331in}}%
\pgfpathlineto{\pgfqpoint{4.632571in}{2.542331in}}%
\pgfpathlineto{\pgfqpoint{4.632571in}{2.553523in}}%
\pgfpathlineto{\pgfqpoint{4.638295in}{2.553523in}}%
\pgfpathlineto{\pgfqpoint{4.638295in}{2.540466in}}%
\pgfpathlineto{\pgfqpoint{4.644020in}{2.540466in}}%
\pgfpathlineto{\pgfqpoint{4.644020in}{2.550725in}}%
\pgfpathlineto{\pgfqpoint{4.649745in}{2.550725in}}%
\pgfpathlineto{\pgfqpoint{4.649745in}{2.544663in}}%
\pgfpathlineto{\pgfqpoint{4.655470in}{2.544663in}}%
\pgfpathlineto{\pgfqpoint{4.655470in}{2.557720in}}%
\pgfpathlineto{\pgfqpoint{4.661195in}{2.557720in}}%
\pgfpathlineto{\pgfqpoint{4.661195in}{2.562849in}}%
\pgfpathlineto{\pgfqpoint{4.666920in}{2.562849in}}%
\pgfpathlineto{\pgfqpoint{4.666920in}{2.555855in}}%
\pgfpathlineto{\pgfqpoint{4.672645in}{2.555855in}}%
\pgfpathlineto{\pgfqpoint{4.672645in}{2.558652in}}%
\pgfpathlineto{\pgfqpoint{4.678370in}{2.558652in}}%
\pgfpathlineto{\pgfqpoint{4.678370in}{2.568911in}}%
\pgfpathlineto{\pgfqpoint{4.684095in}{2.568911in}}%
\pgfpathlineto{\pgfqpoint{4.684095in}{2.573108in}}%
\pgfpathlineto{\pgfqpoint{4.689820in}{2.573108in}}%
\pgfpathlineto{\pgfqpoint{4.689820in}{2.567046in}}%
\pgfpathlineto{\pgfqpoint{4.695545in}{2.567046in}}%
\pgfpathlineto{\pgfqpoint{4.695545in}{2.563782in}}%
\pgfpathlineto{\pgfqpoint{4.701269in}{2.563782in}}%
\pgfpathlineto{\pgfqpoint{4.701269in}{2.574507in}}%
\pgfpathlineto{\pgfqpoint{4.706994in}{2.574507in}}%
\pgfpathlineto{\pgfqpoint{4.706994in}{2.583367in}}%
\pgfpathlineto{\pgfqpoint{4.712719in}{2.583367in}}%
\pgfpathlineto{\pgfqpoint{4.712719in}{2.579636in}}%
\pgfpathlineto{\pgfqpoint{4.718444in}{2.579636in}}%
\pgfpathlineto{\pgfqpoint{4.718444in}{2.589895in}}%
\pgfpathlineto{\pgfqpoint{4.724169in}{2.589895in}}%
\pgfpathlineto{\pgfqpoint{4.724169in}{2.605284in}}%
\pgfpathlineto{\pgfqpoint{4.729894in}{2.605284in}}%
\pgfpathlineto{\pgfqpoint{4.729894in}{2.589429in}}%
\pgfpathlineto{\pgfqpoint{4.735619in}{2.589429in}}%
\pgfpathlineto{\pgfqpoint{4.735619in}{2.595491in}}%
\pgfpathlineto{\pgfqpoint{4.741344in}{2.595491in}}%
\pgfpathlineto{\pgfqpoint{4.741344in}{2.607615in}}%
\pgfpathlineto{\pgfqpoint{4.747069in}{2.607615in}}%
\pgfpathlineto{\pgfqpoint{4.747069in}{2.604351in}}%
\pgfpathlineto{\pgfqpoint{4.752794in}{2.604351in}}%
\pgfpathlineto{\pgfqpoint{4.752794in}{2.611812in}}%
\pgfpathlineto{\pgfqpoint{4.758519in}{2.611812in}}%
\pgfpathlineto{\pgfqpoint{4.758519in}{2.625335in}}%
\pgfpathlineto{\pgfqpoint{4.764243in}{2.625335in}}%
\pgfpathlineto{\pgfqpoint{4.764243in}{2.640257in}}%
\pgfpathlineto{\pgfqpoint{4.769968in}{2.640257in}}%
\pgfpathlineto{\pgfqpoint{4.769968in}{2.648651in}}%
\pgfpathlineto{\pgfqpoint{4.775693in}{2.648651in}}%
\pgfpathlineto{\pgfqpoint{4.775693in}{2.641656in}}%
\pgfpathlineto{\pgfqpoint{4.781418in}{2.641656in}}%
\pgfpathlineto{\pgfqpoint{4.781418in}{2.653314in}}%
\pgfpathlineto{\pgfqpoint{4.787143in}{2.653314in}}%
\pgfpathlineto{\pgfqpoint{4.787143in}{2.655179in}}%
\pgfpathlineto{\pgfqpoint{4.792868in}{2.655179in}}%
\pgfpathlineto{\pgfqpoint{4.792868in}{2.676629in}}%
\pgfpathlineto{\pgfqpoint{4.798593in}{2.676629in}}%
\pgfpathlineto{\pgfqpoint{4.798593in}{2.679894in}}%
\pgfpathlineto{\pgfqpoint{4.804318in}{2.679894in}}%
\pgfpathlineto{\pgfqpoint{4.804318in}{2.705541in}}%
\pgfpathlineto{\pgfqpoint{4.810043in}{2.705541in}}%
\pgfpathlineto{\pgfqpoint{4.810043in}{2.740980in}}%
\pgfpathlineto{\pgfqpoint{4.815768in}{2.740980in}}%
\pgfpathlineto{\pgfqpoint{4.815768in}{2.731188in}}%
\pgfpathlineto{\pgfqpoint{4.821493in}{2.731188in}}%
\pgfpathlineto{\pgfqpoint{4.821493in}{2.762431in}}%
\pgfpathlineto{\pgfqpoint{4.827217in}{2.762431in}}%
\pgfpathlineto{\pgfqpoint{4.827217in}{2.808596in}}%
\pgfpathlineto{\pgfqpoint{4.832942in}{2.808596in}}%
\pgfpathlineto{\pgfqpoint{4.832942in}{2.867817in}}%
\pgfpathlineto{\pgfqpoint{4.838667in}{2.867817in}}%
\pgfpathlineto{\pgfqpoint{4.838667in}{2.865019in}}%
\pgfpathlineto{\pgfqpoint{4.844392in}{2.865019in}}%
\pgfpathlineto{\pgfqpoint{4.844392in}{2.921910in}}%
\pgfpathlineto{\pgfqpoint{4.850117in}{2.921910in}}%
\pgfpathlineto{\pgfqpoint{4.850117in}{2.969007in}}%
\pgfpathlineto{\pgfqpoint{4.855842in}{2.969007in}}%
\pgfpathlineto{\pgfqpoint{4.855842in}{3.030560in}}%
\pgfpathlineto{\pgfqpoint{4.861567in}{3.030560in}}%
\pgfpathlineto{\pgfqpoint{4.861567in}{3.094911in}}%
\pgfpathlineto{\pgfqpoint{4.867292in}{3.094911in}}%
\pgfpathlineto{\pgfqpoint{4.867292in}{3.197966in}}%
\pgfpathlineto{\pgfqpoint{4.873017in}{3.197966in}}%
\pgfpathlineto{\pgfqpoint{4.873017in}{3.349518in}}%
\pgfpathlineto{\pgfqpoint{4.878742in}{3.349518in}}%
\pgfpathlineto{\pgfqpoint{4.878742in}{3.526250in}}%
\pgfpathlineto{\pgfqpoint{4.884467in}{3.526250in}}%
\pgfpathlineto{\pgfqpoint{4.884467in}{3.647491in}}%
\pgfpathlineto{\pgfqpoint{4.890192in}{3.647491in}}%
\pgfpathlineto{\pgfqpoint{4.890192in}{3.795312in}}%
\pgfpathlineto{\pgfqpoint{4.895916in}{3.795312in}}%
\pgfpathlineto{\pgfqpoint{4.895916in}{3.846140in}}%
\pgfpathlineto{\pgfqpoint{4.901641in}{3.846140in}}%
\pgfpathlineto{\pgfqpoint{4.901641in}{3.812566in}}%
\pgfpathlineto{\pgfqpoint{4.907366in}{3.812566in}}%
\pgfpathlineto{\pgfqpoint{4.907366in}{3.757075in}}%
\pgfpathlineto{\pgfqpoint{4.913091in}{3.757075in}}%
\pgfpathlineto{\pgfqpoint{4.913091in}{3.521121in}}%
\pgfpathlineto{\pgfqpoint{4.918816in}{3.521121in}}%
\pgfpathlineto{\pgfqpoint{4.918816in}{3.390553in}}%
\pgfpathlineto{\pgfqpoint{4.924541in}{3.390553in}}%
\pgfpathlineto{\pgfqpoint{4.924541in}{3.264183in}}%
\pgfpathlineto{\pgfqpoint{4.930266in}{3.264183in}}%
\pgfpathlineto{\pgfqpoint{4.930266in}{3.175583in}}%
\pgfpathlineto{\pgfqpoint{4.935991in}{3.175583in}}%
\pgfpathlineto{\pgfqpoint{4.935991in}{3.116828in}}%
\pgfpathlineto{\pgfqpoint{4.941716in}{3.116828in}}%
\pgfpathlineto{\pgfqpoint{4.941716in}{3.033358in}}%
\pgfpathlineto{\pgfqpoint{4.947441in}{3.033358in}}%
\pgfpathlineto{\pgfqpoint{4.947441in}{2.969940in}}%
\pgfpathlineto{\pgfqpoint{4.953166in}{2.969940in}}%
\pgfpathlineto{\pgfqpoint{4.953166in}{2.906521in}}%
\pgfpathlineto{\pgfqpoint{4.958890in}{2.906521in}}%
\pgfpathlineto{\pgfqpoint{4.958890in}{2.859890in}}%
\pgfpathlineto{\pgfqpoint{4.964615in}{2.859890in}}%
\pgfpathlineto{\pgfqpoint{4.964615in}{2.831445in}}%
\pgfpathlineto{\pgfqpoint{4.970340in}{2.831445in}}%
\pgfpathlineto{\pgfqpoint{4.970340in}{2.794606in}}%
\pgfpathlineto{\pgfqpoint{4.976065in}{2.794606in}}%
\pgfpathlineto{\pgfqpoint{4.976065in}{2.782949in}}%
\pgfpathlineto{\pgfqpoint{4.981790in}{2.782949in}}%
\pgfpathlineto{\pgfqpoint{4.981790in}{2.760099in}}%
\pgfpathlineto{\pgfqpoint{4.987515in}{2.760099in}}%
\pgfpathlineto{\pgfqpoint{4.987515in}{2.737250in}}%
\pgfpathlineto{\pgfqpoint{4.993240in}{2.737250in}}%
\pgfpathlineto{\pgfqpoint{4.993240in}{2.735851in}}%
\pgfpathlineto{\pgfqpoint{4.998965in}{2.735851in}}%
\pgfpathlineto{\pgfqpoint{4.998965in}{2.702743in}}%
\pgfpathlineto{\pgfqpoint{5.004690in}{2.702743in}}%
\pgfpathlineto{\pgfqpoint{5.004690in}{2.673365in}}%
\pgfpathlineto{\pgfqpoint{5.010415in}{2.673365in}}%
\pgfpathlineto{\pgfqpoint{5.010415in}{2.686422in}}%
\pgfpathlineto{\pgfqpoint{5.016140in}{2.686422in}}%
\pgfpathlineto{\pgfqpoint{5.016140in}{2.672899in}}%
\pgfpathlineto{\pgfqpoint{5.021864in}{2.672899in}}%
\pgfpathlineto{\pgfqpoint{5.021864in}{2.664972in}}%
\pgfpathlineto{\pgfqpoint{5.027589in}{2.664972in}}%
\pgfpathlineto{\pgfqpoint{5.027589in}{2.641190in}}%
\pgfpathlineto{\pgfqpoint{5.033314in}{2.641190in}}%
\pgfpathlineto{\pgfqpoint{5.033314in}{2.645853in}}%
\pgfpathlineto{\pgfqpoint{5.039039in}{2.645853in}}%
\pgfpathlineto{\pgfqpoint{5.039039in}{2.635594in}}%
\pgfpathlineto{\pgfqpoint{5.044764in}{2.635594in}}%
\pgfpathlineto{\pgfqpoint{5.044764in}{2.621605in}}%
\pgfpathlineto{\pgfqpoint{5.050489in}{2.621605in}}%
\pgfpathlineto{\pgfqpoint{5.050489in}{2.614144in}}%
\pgfpathlineto{\pgfqpoint{5.056214in}{2.614144in}}%
\pgfpathlineto{\pgfqpoint{5.056214in}{2.606683in}}%
\pgfpathlineto{\pgfqpoint{5.067664in}{2.605750in}}%
\pgfpathlineto{\pgfqpoint{5.067664in}{2.610879in}}%
\pgfpathlineto{\pgfqpoint{5.073389in}{2.610879in}}%
\pgfpathlineto{\pgfqpoint{5.073389in}{2.605284in}}%
\pgfpathlineto{\pgfqpoint{5.079114in}{2.605284in}}%
\pgfpathlineto{\pgfqpoint{5.079114in}{2.602952in}}%
\pgfpathlineto{\pgfqpoint{5.084838in}{2.602952in}}%
\pgfpathlineto{\pgfqpoint{5.084838in}{2.599222in}}%
\pgfpathlineto{\pgfqpoint{5.090563in}{2.599222in}}%
\pgfpathlineto{\pgfqpoint{5.090563in}{2.588030in}}%
\pgfpathlineto{\pgfqpoint{5.102013in}{2.588496in}}%
\pgfpathlineto{\pgfqpoint{5.102013in}{2.586165in}}%
\pgfpathlineto{\pgfqpoint{5.107738in}{2.586165in}}%
\pgfpathlineto{\pgfqpoint{5.107738in}{2.588963in}}%
\pgfpathlineto{\pgfqpoint{5.113463in}{2.588963in}}%
\pgfpathlineto{\pgfqpoint{5.113463in}{2.570310in}}%
\pgfpathlineto{\pgfqpoint{5.119188in}{2.570310in}}%
\pgfpathlineto{\pgfqpoint{5.119188in}{2.571709in}}%
\pgfpathlineto{\pgfqpoint{5.124913in}{2.571709in}}%
\pgfpathlineto{\pgfqpoint{5.124913in}{2.577771in}}%
\pgfpathlineto{\pgfqpoint{5.130638in}{2.577771in}}%
\pgfpathlineto{\pgfqpoint{5.130638in}{2.571709in}}%
\pgfpathlineto{\pgfqpoint{5.136363in}{2.571709in}}%
\pgfpathlineto{\pgfqpoint{5.136363in}{2.576839in}}%
\pgfpathlineto{\pgfqpoint{5.142088in}{2.576839in}}%
\pgfpathlineto{\pgfqpoint{5.142088in}{2.561450in}}%
\pgfpathlineto{\pgfqpoint{5.147813in}{2.561450in}}%
\pgfpathlineto{\pgfqpoint{5.147813in}{2.569844in}}%
\pgfpathlineto{\pgfqpoint{5.153537in}{2.569844in}}%
\pgfpathlineto{\pgfqpoint{5.153537in}{2.566113in}}%
\pgfpathlineto{\pgfqpoint{5.159262in}{2.566113in}}%
\pgfpathlineto{\pgfqpoint{5.159262in}{2.560518in}}%
\pgfpathlineto{\pgfqpoint{5.164987in}{2.560518in}}%
\pgfpathlineto{\pgfqpoint{5.164987in}{2.555855in}}%
\pgfpathlineto{\pgfqpoint{5.170712in}{2.555855in}}%
\pgfpathlineto{\pgfqpoint{5.170712in}{2.551658in}}%
\pgfpathlineto{\pgfqpoint{5.176437in}{2.551658in}}%
\pgfpathlineto{\pgfqpoint{5.176437in}{2.554456in}}%
\pgfpathlineto{\pgfqpoint{5.187887in}{2.553523in}}%
\pgfpathlineto{\pgfqpoint{5.187887in}{2.557720in}}%
\pgfpathlineto{\pgfqpoint{5.193612in}{2.557720in}}%
\pgfpathlineto{\pgfqpoint{5.193612in}{2.554456in}}%
\pgfpathlineto{\pgfqpoint{5.199337in}{2.554456in}}%
\pgfpathlineto{\pgfqpoint{5.199337in}{2.555855in}}%
\pgfpathlineto{\pgfqpoint{5.205062in}{2.555855in}}%
\pgfpathlineto{\pgfqpoint{5.205062in}{2.546528in}}%
\pgfpathlineto{\pgfqpoint{5.210787in}{2.546528in}}%
\pgfpathlineto{\pgfqpoint{5.210787in}{2.547927in}}%
\pgfpathlineto{\pgfqpoint{5.216511in}{2.547927in}}%
\pgfpathlineto{\pgfqpoint{5.216511in}{2.550725in}}%
\pgfpathlineto{\pgfqpoint{5.222236in}{2.550725in}}%
\pgfpathlineto{\pgfqpoint{5.222236in}{2.543264in}}%
\pgfpathlineto{\pgfqpoint{5.227961in}{2.543264in}}%
\pgfpathlineto{\pgfqpoint{5.227961in}{2.541865in}}%
\pgfpathlineto{\pgfqpoint{5.233686in}{2.541865in}}%
\pgfpathlineto{\pgfqpoint{5.233686in}{2.543264in}}%
\pgfpathlineto{\pgfqpoint{5.239411in}{2.543264in}}%
\pgfpathlineto{\pgfqpoint{5.239411in}{2.545129in}}%
\pgfpathlineto{\pgfqpoint{5.245136in}{2.545129in}}%
\pgfpathlineto{\pgfqpoint{5.245136in}{2.542798in}}%
\pgfpathlineto{\pgfqpoint{5.250861in}{2.542798in}}%
\pgfpathlineto{\pgfqpoint{5.250861in}{2.541399in}}%
\pgfpathlineto{\pgfqpoint{5.256586in}{2.541399in}}%
\pgfpathlineto{\pgfqpoint{5.256586in}{2.540000in}}%
\pgfpathlineto{\pgfqpoint{5.262311in}{2.540000in}}%
\pgfpathlineto{\pgfqpoint{5.262311in}{2.534870in}}%
\pgfpathlineto{\pgfqpoint{5.268036in}{2.534870in}}%
\pgfpathlineto{\pgfqpoint{5.268036in}{2.529275in}}%
\pgfpathlineto{\pgfqpoint{5.273761in}{2.529275in}}%
\pgfpathlineto{\pgfqpoint{5.273761in}{2.545129in}}%
\pgfpathlineto{\pgfqpoint{5.279485in}{2.545129in}}%
\pgfpathlineto{\pgfqpoint{5.279485in}{2.530674in}}%
\pgfpathlineto{\pgfqpoint{5.285210in}{2.530674in}}%
\pgfpathlineto{\pgfqpoint{5.285210in}{2.537202in}}%
\pgfpathlineto{\pgfqpoint{5.290935in}{2.537202in}}%
\pgfpathlineto{\pgfqpoint{5.290935in}{2.540466in}}%
\pgfpathlineto{\pgfqpoint{5.296660in}{2.540466in}}%
\pgfpathlineto{\pgfqpoint{5.296660in}{2.534404in}}%
\pgfpathlineto{\pgfqpoint{5.302385in}{2.534404in}}%
\pgfpathlineto{\pgfqpoint{5.302385in}{2.526011in}}%
\pgfpathlineto{\pgfqpoint{5.308110in}{2.526011in}}%
\pgfpathlineto{\pgfqpoint{5.308110in}{2.533472in}}%
\pgfpathlineto{\pgfqpoint{5.313835in}{2.533472in}}%
\pgfpathlineto{\pgfqpoint{5.313835in}{2.528342in}}%
\pgfpathlineto{\pgfqpoint{5.319560in}{2.528342in}}%
\pgfpathlineto{\pgfqpoint{5.319560in}{2.530207in}}%
\pgfpathlineto{\pgfqpoint{5.325285in}{2.530207in}}%
\pgfpathlineto{\pgfqpoint{5.325285in}{2.528808in}}%
\pgfpathlineto{\pgfqpoint{5.331010in}{2.528808in}}%
\pgfpathlineto{\pgfqpoint{5.331010in}{2.525544in}}%
\pgfpathlineto{\pgfqpoint{5.336735in}{2.525544in}}%
\pgfpathlineto{\pgfqpoint{5.336735in}{2.521814in}}%
\pgfpathlineto{\pgfqpoint{5.342459in}{2.521814in}}%
\pgfpathlineto{\pgfqpoint{5.342459in}{2.519948in}}%
\pgfpathlineto{\pgfqpoint{5.348184in}{2.519948in}}%
\pgfpathlineto{\pgfqpoint{5.348184in}{2.524612in}}%
\pgfpathlineto{\pgfqpoint{5.353909in}{2.524612in}}%
\pgfpathlineto{\pgfqpoint{5.353909in}{2.521814in}}%
\pgfpathlineto{\pgfqpoint{5.359634in}{2.521814in}}%
\pgfpathlineto{\pgfqpoint{5.359634in}{2.517151in}}%
\pgfpathlineto{\pgfqpoint{5.365359in}{2.517151in}}%
\pgfpathlineto{\pgfqpoint{5.365359in}{2.519948in}}%
\pgfpathlineto{\pgfqpoint{5.371084in}{2.519948in}}%
\pgfpathlineto{\pgfqpoint{5.371084in}{2.516684in}}%
\pgfpathlineto{\pgfqpoint{5.376809in}{2.516684in}}%
\pgfpathlineto{\pgfqpoint{5.376809in}{2.522280in}}%
\pgfpathlineto{\pgfqpoint{5.382534in}{2.522280in}}%
\pgfpathlineto{\pgfqpoint{5.382534in}{2.520415in}}%
\pgfpathlineto{\pgfqpoint{5.388259in}{2.520415in}}%
\pgfpathlineto{\pgfqpoint{5.388259in}{2.522746in}}%
\pgfpathlineto{\pgfqpoint{5.393984in}{2.522746in}}%
\pgfpathlineto{\pgfqpoint{5.393984in}{2.518550in}}%
\pgfpathlineto{\pgfqpoint{5.399709in}{2.518550in}}%
\pgfpathlineto{\pgfqpoint{5.399709in}{2.523679in}}%
\pgfpathlineto{\pgfqpoint{5.405434in}{2.523679in}}%
\pgfpathlineto{\pgfqpoint{5.405434in}{2.515285in}}%
\pgfpathlineto{\pgfqpoint{5.411158in}{2.515285in}}%
\pgfpathlineto{\pgfqpoint{5.411158in}{2.513886in}}%
\pgfpathlineto{\pgfqpoint{5.416883in}{2.513886in}}%
\pgfpathlineto{\pgfqpoint{5.416883in}{2.510156in}}%
\pgfpathlineto{\pgfqpoint{5.422608in}{2.510156in}}%
\pgfpathlineto{\pgfqpoint{5.422608in}{2.512954in}}%
\pgfpathlineto{\pgfqpoint{5.428333in}{2.512954in}}%
\pgfpathlineto{\pgfqpoint{5.428333in}{2.520415in}}%
\pgfpathlineto{\pgfqpoint{5.434058in}{2.520415in}}%
\pgfpathlineto{\pgfqpoint{5.434058in}{2.514819in}}%
\pgfpathlineto{\pgfqpoint{5.445508in}{2.514819in}}%
\pgfpathlineto{\pgfqpoint{5.445508in}{2.516684in}}%
\pgfpathlineto{\pgfqpoint{5.451233in}{2.516684in}}%
\pgfpathlineto{\pgfqpoint{5.451233in}{2.512954in}}%
\pgfpathlineto{\pgfqpoint{5.456958in}{2.512954in}}%
\pgfpathlineto{\pgfqpoint{5.456958in}{2.516684in}}%
\pgfpathlineto{\pgfqpoint{5.462683in}{2.516684in}}%
\pgfpathlineto{\pgfqpoint{5.462683in}{2.508291in}}%
\pgfpathlineto{\pgfqpoint{5.468408in}{2.508291in}}%
\pgfpathlineto{\pgfqpoint{5.468408in}{2.512487in}}%
\pgfpathlineto{\pgfqpoint{5.474132in}{2.512487in}}%
\pgfpathlineto{\pgfqpoint{5.474132in}{2.507358in}}%
\pgfpathlineto{\pgfqpoint{5.485582in}{2.506425in}}%
\pgfpathlineto{\pgfqpoint{5.485582in}{2.510156in}}%
\pgfpathlineto{\pgfqpoint{5.491307in}{2.510156in}}%
\pgfpathlineto{\pgfqpoint{5.491307in}{2.505959in}}%
\pgfpathlineto{\pgfqpoint{5.497032in}{2.505959in}}%
\pgfpathlineto{\pgfqpoint{5.497032in}{2.510156in}}%
\pgfpathlineto{\pgfqpoint{5.502757in}{2.510156in}}%
\pgfpathlineto{\pgfqpoint{5.502757in}{2.506892in}}%
\pgfpathlineto{\pgfqpoint{5.508482in}{2.506892in}}%
\pgfpathlineto{\pgfqpoint{5.508482in}{2.502695in}}%
\pgfpathlineto{\pgfqpoint{5.514207in}{2.502695in}}%
\pgfpathlineto{\pgfqpoint{5.514207in}{2.509223in}}%
\pgfpathlineto{\pgfqpoint{5.519932in}{2.509223in}}%
\pgfpathlineto{\pgfqpoint{5.519932in}{2.507358in}}%
\pgfpathlineto{\pgfqpoint{5.531382in}{2.506892in}}%
\pgfpathlineto{\pgfqpoint{5.531382in}{2.504560in}}%
\pgfpathlineto{\pgfqpoint{5.542831in}{2.505027in}}%
\pgfpathlineto{\pgfqpoint{5.542831in}{2.507824in}}%
\pgfpathlineto{\pgfqpoint{5.560006in}{2.507358in}}%
\pgfpathlineto{\pgfqpoint{5.560006in}{2.505959in}}%
\pgfpathlineto{\pgfqpoint{5.582906in}{2.505959in}}%
\pgfpathlineto{\pgfqpoint{5.582906in}{2.501296in}}%
\pgfpathlineto{\pgfqpoint{5.588631in}{2.501296in}}%
\pgfpathlineto{\pgfqpoint{5.588631in}{2.503628in}}%
\pgfpathlineto{\pgfqpoint{5.594356in}{2.503628in}}%
\pgfpathlineto{\pgfqpoint{5.594356in}{2.501762in}}%
\pgfpathlineto{\pgfqpoint{5.600080in}{2.501762in}}%
\pgfpathlineto{\pgfqpoint{5.600080in}{2.503161in}}%
\pgfpathlineto{\pgfqpoint{5.605805in}{2.503161in}}%
\pgfpathlineto{\pgfqpoint{5.605805in}{2.505493in}}%
\pgfpathlineto{\pgfqpoint{5.611530in}{2.505493in}}%
\pgfpathlineto{\pgfqpoint{5.611530in}{2.501762in}}%
\pgfpathlineto{\pgfqpoint{5.622980in}{2.501762in}}%
\pgfpathlineto{\pgfqpoint{5.622980in}{2.503161in}}%
\pgfpathlineto{\pgfqpoint{5.634430in}{2.502229in}}%
\pgfpathlineto{\pgfqpoint{5.634430in}{2.498964in}}%
\pgfpathlineto{\pgfqpoint{5.640155in}{2.498964in}}%
\pgfpathlineto{\pgfqpoint{5.640155in}{2.505493in}}%
\pgfpathlineto{\pgfqpoint{5.645880in}{2.505493in}}%
\pgfpathlineto{\pgfqpoint{5.645880in}{2.502695in}}%
\pgfpathlineto{\pgfqpoint{5.663055in}{2.501762in}}%
\pgfpathlineto{\pgfqpoint{5.663055in}{2.499897in}}%
\pgfpathlineto{\pgfqpoint{5.668779in}{2.499897in}}%
\pgfpathlineto{\pgfqpoint{5.668779in}{2.501296in}}%
\pgfpathlineto{\pgfqpoint{5.674504in}{2.501296in}}%
\pgfpathlineto{\pgfqpoint{5.674504in}{2.504560in}}%
\pgfpathlineto{\pgfqpoint{5.680229in}{2.504560in}}%
\pgfpathlineto{\pgfqpoint{5.680229in}{2.501762in}}%
\pgfpathlineto{\pgfqpoint{5.697404in}{2.502695in}}%
\pgfpathlineto{\pgfqpoint{5.697404in}{2.500363in}}%
\pgfpathlineto{\pgfqpoint{5.714579in}{2.499897in}}%
\pgfpathlineto{\pgfqpoint{5.714579in}{2.498498in}}%
\pgfpathlineto{\pgfqpoint{5.720304in}{2.498498in}}%
\pgfpathlineto{\pgfqpoint{5.720304in}{2.499897in}}%
\pgfpathlineto{\pgfqpoint{5.726029in}{2.499897in}}%
\pgfpathlineto{\pgfqpoint{5.726029in}{2.498032in}}%
\pgfpathlineto{\pgfqpoint{5.731753in}{2.498032in}}%
\pgfpathlineto{\pgfqpoint{5.731753in}{2.496167in}}%
\pgfpathlineto{\pgfqpoint{5.737478in}{2.496167in}}%
\pgfpathlineto{\pgfqpoint{5.737478in}{2.500363in}}%
\pgfpathlineto{\pgfqpoint{5.743203in}{2.500363in}}%
\pgfpathlineto{\pgfqpoint{5.743203in}{2.498498in}}%
\pgfpathlineto{\pgfqpoint{5.748928in}{2.498498in}}%
\pgfpathlineto{\pgfqpoint{5.748928in}{2.497099in}}%
\pgfpathlineto{\pgfqpoint{5.754653in}{2.497099in}}%
\pgfpathlineto{\pgfqpoint{5.754653in}{2.499431in}}%
\pgfpathlineto{\pgfqpoint{5.760378in}{2.499431in}}%
\pgfpathlineto{\pgfqpoint{5.760378in}{2.497566in}}%
\pgfpathlineto{\pgfqpoint{5.789003in}{2.498498in}}%
\pgfpathlineto{\pgfqpoint{5.789003in}{2.496167in}}%
\pgfpathlineto{\pgfqpoint{5.789003in}{2.496167in}}%
\pgfusepath{stroke}%
\end{pgfscope}%
\begin{pgfscope}%
\pgfsetrectcap%
\pgfsetmiterjoin%
\pgfsetlinewidth{0.803000pt}%
\definecolor{currentstroke}{rgb}{0.000000,0.000000,0.000000}%
\pgfsetstrokecolor{currentstroke}%
\pgfsetdash{}{0pt}%
\pgfpathmoveto{\pgfqpoint{3.750134in}{2.496167in}}%
\pgfpathlineto{\pgfqpoint{3.750134in}{3.913639in}}%
\pgfusepath{stroke}%
\end{pgfscope}%
\begin{pgfscope}%
\pgfsetrectcap%
\pgfsetmiterjoin%
\pgfsetlinewidth{0.803000pt}%
\definecolor{currentstroke}{rgb}{0.000000,0.000000,0.000000}%
\pgfsetstrokecolor{currentstroke}%
\pgfsetdash{}{0pt}%
\pgfpathmoveto{\pgfqpoint{6.051200in}{2.496167in}}%
\pgfpathlineto{\pgfqpoint{6.051200in}{3.913639in}}%
\pgfusepath{stroke}%
\end{pgfscope}%
\begin{pgfscope}%
\pgfsetrectcap%
\pgfsetmiterjoin%
\pgfsetlinewidth{0.803000pt}%
\definecolor{currentstroke}{rgb}{0.000000,0.000000,0.000000}%
\pgfsetstrokecolor{currentstroke}%
\pgfsetdash{}{0pt}%
\pgfpathmoveto{\pgfqpoint{3.750134in}{2.496167in}}%
\pgfpathlineto{\pgfqpoint{6.051200in}{2.496167in}}%
\pgfusepath{stroke}%
\end{pgfscope}%
\begin{pgfscope}%
\pgfsetrectcap%
\pgfsetmiterjoin%
\pgfsetlinewidth{0.803000pt}%
\definecolor{currentstroke}{rgb}{0.000000,0.000000,0.000000}%
\pgfsetstrokecolor{currentstroke}%
\pgfsetdash{}{0pt}%
\pgfpathmoveto{\pgfqpoint{3.750134in}{3.913639in}}%
\pgfpathlineto{\pgfqpoint{6.051200in}{3.913639in}}%
\pgfusepath{stroke}%
\end{pgfscope}%
\begin{pgfscope}%
\definecolor{textcolor}{rgb}{0.000000,0.000000,0.000000}%
\pgfsetstrokecolor{textcolor}%
\pgfsetfillcolor{textcolor}%
\pgftext[x=3.750134in,y=3.996972in,left,base]{\color{textcolor}\rmfamily\fontsize{10.000000}{12.000000}\selectfont Bin [1.83, 2.0), 65,392 events}%
\end{pgfscope}%
\begin{pgfscope}%
\pgfsetbuttcap%
\pgfsetmiterjoin%
\definecolor{currentfill}{rgb}{1.000000,1.000000,1.000000}%
\pgfsetfillcolor{currentfill}%
\pgfsetfillopacity{0.800000}%
\pgfsetlinewidth{1.003750pt}%
\definecolor{currentstroke}{rgb}{0.800000,0.800000,0.800000}%
\pgfsetstrokecolor{currentstroke}%
\pgfsetstrokeopacity{0.800000}%
\pgfsetdash{}{0pt}%
\pgfpathmoveto{\pgfqpoint{4.957422in}{3.514083in}}%
\pgfpathlineto{\pgfqpoint{5.973422in}{3.514083in}}%
\pgfpathquadraticcurveto{\pgfqpoint{5.995644in}{3.514083in}}{\pgfqpoint{5.995644in}{3.536306in}}%
\pgfpathlineto{\pgfqpoint{5.995644in}{3.835861in}}%
\pgfpathquadraticcurveto{\pgfqpoint{5.995644in}{3.858083in}}{\pgfqpoint{5.973422in}{3.858083in}}%
\pgfpathlineto{\pgfqpoint{4.957422in}{3.858083in}}%
\pgfpathquadraticcurveto{\pgfqpoint{4.935200in}{3.858083in}}{\pgfqpoint{4.935200in}{3.835861in}}%
\pgfpathlineto{\pgfqpoint{4.935200in}{3.536306in}}%
\pgfpathquadraticcurveto{\pgfqpoint{4.935200in}{3.514083in}}{\pgfqpoint{4.957422in}{3.514083in}}%
\pgfpathclose%
\pgfusepath{stroke,fill}%
\end{pgfscope}%
\begin{pgfscope}%
\pgfsetbuttcap%
\pgfsetmiterjoin%
\pgfsetlinewidth{1.003750pt}%
\definecolor{currentstroke}{rgb}{0.313725,0.317647,0.309804}%
\pgfsetstrokecolor{currentstroke}%
\pgfsetdash{}{0pt}%
\pgfpathmoveto{\pgfqpoint{4.979644in}{3.735417in}}%
\pgfpathlineto{\pgfqpoint{5.201867in}{3.735417in}}%
\pgfpathlineto{\pgfqpoint{5.201867in}{3.813194in}}%
\pgfpathlineto{\pgfqpoint{4.979644in}{3.813194in}}%
\pgfpathclose%
\pgfusepath{stroke}%
\end{pgfscope}%
\begin{pgfscope}%
\definecolor{textcolor}{rgb}{0.000000,0.000000,0.000000}%
\pgfsetstrokecolor{textcolor}%
\pgfsetfillcolor{textcolor}%
\pgftext[x=5.290756in,y=3.735417in,left,base]{\color{textcolor}\rmfamily\fontsize{8.000000}{9.600000}\selectfont IQR = 11.21}%
\end{pgfscope}%
\begin{pgfscope}%
\pgfsetbuttcap%
\pgfsetmiterjoin%
\pgfsetlinewidth{1.003750pt}%
\definecolor{currentstroke}{rgb}{0.949020,0.372549,0.360784}%
\pgfsetstrokecolor{currentstroke}%
\pgfsetdash{{1.000000pt}{1.650000pt}}{0.000000pt}%
\pgfpathmoveto{\pgfqpoint{4.979644in}{3.580083in}}%
\pgfpathlineto{\pgfqpoint{5.201867in}{3.580083in}}%
\pgfpathlineto{\pgfqpoint{5.201867in}{3.657861in}}%
\pgfpathlineto{\pgfqpoint{4.979644in}{3.657861in}}%
\pgfpathclose%
\pgfusepath{stroke}%
\end{pgfscope}%
\begin{pgfscope}%
\definecolor{textcolor}{rgb}{0.000000,0.000000,0.000000}%
\pgfsetstrokecolor{textcolor}%
\pgfsetfillcolor{textcolor}%
\pgftext[x=5.290756in,y=3.580083in,left,base]{\color{textcolor}\rmfamily\fontsize{8.000000}{9.600000}\selectfont IQR = 9.69}%
\end{pgfscope}%
\begin{pgfscope}%
\pgfsetbuttcap%
\pgfsetmiterjoin%
\definecolor{currentfill}{rgb}{1.000000,1.000000,1.000000}%
\pgfsetfillcolor{currentfill}%
\pgfsetlinewidth{0.000000pt}%
\definecolor{currentstroke}{rgb}{0.000000,0.000000,0.000000}%
\pgfsetstrokecolor{currentstroke}%
\pgfsetstrokeopacity{0.000000}%
\pgfsetdash{}{0pt}%
\pgfpathmoveto{\pgfqpoint{0.754048in}{0.545305in}}%
\pgfpathlineto{\pgfqpoint{3.055114in}{0.545305in}}%
\pgfpathlineto{\pgfqpoint{3.055114in}{1.962778in}}%
\pgfpathlineto{\pgfqpoint{0.754048in}{1.962778in}}%
\pgfpathclose%
\pgfusepath{fill}%
\end{pgfscope}%
\begin{pgfscope}%
\pgfsetbuttcap%
\pgfsetroundjoin%
\definecolor{currentfill}{rgb}{0.000000,0.000000,0.000000}%
\pgfsetfillcolor{currentfill}%
\pgfsetlinewidth{0.803000pt}%
\definecolor{currentstroke}{rgb}{0.000000,0.000000,0.000000}%
\pgfsetstrokecolor{currentstroke}%
\pgfsetdash{}{0pt}%
\pgfsys@defobject{currentmarker}{\pgfqpoint{0.000000in}{-0.048611in}}{\pgfqpoint{0.000000in}{0.000000in}}{%
\pgfpathmoveto{\pgfqpoint{0.000000in}{0.000000in}}%
\pgfpathlineto{\pgfqpoint{0.000000in}{-0.048611in}}%
\pgfusepath{stroke,fill}%
}%
\begin{pgfscope}%
\pgfsys@transformshift{0.986179in}{0.545305in}%
\pgfsys@useobject{currentmarker}{}%
\end{pgfscope}%
\end{pgfscope}%
\begin{pgfscope}%
\definecolor{textcolor}{rgb}{0.000000,0.000000,0.000000}%
\pgfsetstrokecolor{textcolor}%
\pgfsetfillcolor{textcolor}%
\pgftext[x=0.986179in,y=0.448083in,,top]{\color{textcolor}\rmfamily\fontsize{8.000000}{9.600000}\selectfont \(\displaystyle {-120}\)}%
\end{pgfscope}%
\begin{pgfscope}%
\pgfsetbuttcap%
\pgfsetroundjoin%
\definecolor{currentfill}{rgb}{0.000000,0.000000,0.000000}%
\pgfsetfillcolor{currentfill}%
\pgfsetlinewidth{0.803000pt}%
\definecolor{currentstroke}{rgb}{0.000000,0.000000,0.000000}%
\pgfsetstrokecolor{currentstroke}%
\pgfsetdash{}{0pt}%
\pgfsys@defobject{currentmarker}{\pgfqpoint{0.000000in}{-0.048611in}}{\pgfqpoint{0.000000in}{0.000000in}}{%
\pgfpathmoveto{\pgfqpoint{0.000000in}{0.000000in}}%
\pgfpathlineto{\pgfqpoint{0.000000in}{-0.048611in}}%
\pgfusepath{stroke,fill}%
}%
\begin{pgfscope}%
\pgfsys@transformshift{1.291881in}{0.545305in}%
\pgfsys@useobject{currentmarker}{}%
\end{pgfscope}%
\end{pgfscope}%
\begin{pgfscope}%
\definecolor{textcolor}{rgb}{0.000000,0.000000,0.000000}%
\pgfsetstrokecolor{textcolor}%
\pgfsetfillcolor{textcolor}%
\pgftext[x=1.291881in,y=0.448083in,,top]{\color{textcolor}\rmfamily\fontsize{8.000000}{9.600000}\selectfont \(\displaystyle {-80}\)}%
\end{pgfscope}%
\begin{pgfscope}%
\pgfsetbuttcap%
\pgfsetroundjoin%
\definecolor{currentfill}{rgb}{0.000000,0.000000,0.000000}%
\pgfsetfillcolor{currentfill}%
\pgfsetlinewidth{0.803000pt}%
\definecolor{currentstroke}{rgb}{0.000000,0.000000,0.000000}%
\pgfsetstrokecolor{currentstroke}%
\pgfsetdash{}{0pt}%
\pgfsys@defobject{currentmarker}{\pgfqpoint{0.000000in}{-0.048611in}}{\pgfqpoint{0.000000in}{0.000000in}}{%
\pgfpathmoveto{\pgfqpoint{0.000000in}{0.000000in}}%
\pgfpathlineto{\pgfqpoint{0.000000in}{-0.048611in}}%
\pgfusepath{stroke,fill}%
}%
\begin{pgfscope}%
\pgfsys@transformshift{1.597582in}{0.545305in}%
\pgfsys@useobject{currentmarker}{}%
\end{pgfscope}%
\end{pgfscope}%
\begin{pgfscope}%
\definecolor{textcolor}{rgb}{0.000000,0.000000,0.000000}%
\pgfsetstrokecolor{textcolor}%
\pgfsetfillcolor{textcolor}%
\pgftext[x=1.597582in,y=0.448083in,,top]{\color{textcolor}\rmfamily\fontsize{8.000000}{9.600000}\selectfont \(\displaystyle {-40}\)}%
\end{pgfscope}%
\begin{pgfscope}%
\pgfsetbuttcap%
\pgfsetroundjoin%
\definecolor{currentfill}{rgb}{0.000000,0.000000,0.000000}%
\pgfsetfillcolor{currentfill}%
\pgfsetlinewidth{0.803000pt}%
\definecolor{currentstroke}{rgb}{0.000000,0.000000,0.000000}%
\pgfsetstrokecolor{currentstroke}%
\pgfsetdash{}{0pt}%
\pgfsys@defobject{currentmarker}{\pgfqpoint{0.000000in}{-0.048611in}}{\pgfqpoint{0.000000in}{0.000000in}}{%
\pgfpathmoveto{\pgfqpoint{0.000000in}{0.000000in}}%
\pgfpathlineto{\pgfqpoint{0.000000in}{-0.048611in}}%
\pgfusepath{stroke,fill}%
}%
\begin{pgfscope}%
\pgfsys@transformshift{1.903283in}{0.545305in}%
\pgfsys@useobject{currentmarker}{}%
\end{pgfscope}%
\end{pgfscope}%
\begin{pgfscope}%
\definecolor{textcolor}{rgb}{0.000000,0.000000,0.000000}%
\pgfsetstrokecolor{textcolor}%
\pgfsetfillcolor{textcolor}%
\pgftext[x=1.903283in,y=0.448083in,,top]{\color{textcolor}\rmfamily\fontsize{8.000000}{9.600000}\selectfont \(\displaystyle {0}\)}%
\end{pgfscope}%
\begin{pgfscope}%
\pgfsetbuttcap%
\pgfsetroundjoin%
\definecolor{currentfill}{rgb}{0.000000,0.000000,0.000000}%
\pgfsetfillcolor{currentfill}%
\pgfsetlinewidth{0.803000pt}%
\definecolor{currentstroke}{rgb}{0.000000,0.000000,0.000000}%
\pgfsetstrokecolor{currentstroke}%
\pgfsetdash{}{0pt}%
\pgfsys@defobject{currentmarker}{\pgfqpoint{0.000000in}{-0.048611in}}{\pgfqpoint{0.000000in}{0.000000in}}{%
\pgfpathmoveto{\pgfqpoint{0.000000in}{0.000000in}}%
\pgfpathlineto{\pgfqpoint{0.000000in}{-0.048611in}}%
\pgfusepath{stroke,fill}%
}%
\begin{pgfscope}%
\pgfsys@transformshift{2.208985in}{0.545305in}%
\pgfsys@useobject{currentmarker}{}%
\end{pgfscope}%
\end{pgfscope}%
\begin{pgfscope}%
\definecolor{textcolor}{rgb}{0.000000,0.000000,0.000000}%
\pgfsetstrokecolor{textcolor}%
\pgfsetfillcolor{textcolor}%
\pgftext[x=2.208985in,y=0.448083in,,top]{\color{textcolor}\rmfamily\fontsize{8.000000}{9.600000}\selectfont \(\displaystyle {40}\)}%
\end{pgfscope}%
\begin{pgfscope}%
\pgfsetbuttcap%
\pgfsetroundjoin%
\definecolor{currentfill}{rgb}{0.000000,0.000000,0.000000}%
\pgfsetfillcolor{currentfill}%
\pgfsetlinewidth{0.803000pt}%
\definecolor{currentstroke}{rgb}{0.000000,0.000000,0.000000}%
\pgfsetstrokecolor{currentstroke}%
\pgfsetdash{}{0pt}%
\pgfsys@defobject{currentmarker}{\pgfqpoint{0.000000in}{-0.048611in}}{\pgfqpoint{0.000000in}{0.000000in}}{%
\pgfpathmoveto{\pgfqpoint{0.000000in}{0.000000in}}%
\pgfpathlineto{\pgfqpoint{0.000000in}{-0.048611in}}%
\pgfusepath{stroke,fill}%
}%
\begin{pgfscope}%
\pgfsys@transformshift{2.514686in}{0.545305in}%
\pgfsys@useobject{currentmarker}{}%
\end{pgfscope}%
\end{pgfscope}%
\begin{pgfscope}%
\definecolor{textcolor}{rgb}{0.000000,0.000000,0.000000}%
\pgfsetstrokecolor{textcolor}%
\pgfsetfillcolor{textcolor}%
\pgftext[x=2.514686in,y=0.448083in,,top]{\color{textcolor}\rmfamily\fontsize{8.000000}{9.600000}\selectfont \(\displaystyle {80}\)}%
\end{pgfscope}%
\begin{pgfscope}%
\pgfsetbuttcap%
\pgfsetroundjoin%
\definecolor{currentfill}{rgb}{0.000000,0.000000,0.000000}%
\pgfsetfillcolor{currentfill}%
\pgfsetlinewidth{0.803000pt}%
\definecolor{currentstroke}{rgb}{0.000000,0.000000,0.000000}%
\pgfsetstrokecolor{currentstroke}%
\pgfsetdash{}{0pt}%
\pgfsys@defobject{currentmarker}{\pgfqpoint{0.000000in}{-0.048611in}}{\pgfqpoint{0.000000in}{0.000000in}}{%
\pgfpathmoveto{\pgfqpoint{0.000000in}{0.000000in}}%
\pgfpathlineto{\pgfqpoint{0.000000in}{-0.048611in}}%
\pgfusepath{stroke,fill}%
}%
\begin{pgfscope}%
\pgfsys@transformshift{2.820387in}{0.545305in}%
\pgfsys@useobject{currentmarker}{}%
\end{pgfscope}%
\end{pgfscope}%
\begin{pgfscope}%
\definecolor{textcolor}{rgb}{0.000000,0.000000,0.000000}%
\pgfsetstrokecolor{textcolor}%
\pgfsetfillcolor{textcolor}%
\pgftext[x=2.820387in,y=0.448083in,,top]{\color{textcolor}\rmfamily\fontsize{8.000000}{9.600000}\selectfont \(\displaystyle {120}\)}%
\end{pgfscope}%
\begin{pgfscope}%
\definecolor{textcolor}{rgb}{0.000000,0.000000,0.000000}%
\pgfsetstrokecolor{textcolor}%
\pgfsetfillcolor{textcolor}%
\pgftext[x=1.904581in,y=0.293861in,,top]{\color{textcolor}\rmfamily\fontsize{10.000000}{12.000000}\selectfont \(\displaystyle \theta_{\textup{truth}} - \theta_{\textup{prediction}} \, [\textup{deg}]\)}%
\end{pgfscope}%
\begin{pgfscope}%
\pgfsetbuttcap%
\pgfsetroundjoin%
\definecolor{currentfill}{rgb}{0.000000,0.000000,0.000000}%
\pgfsetfillcolor{currentfill}%
\pgfsetlinewidth{0.803000pt}%
\definecolor{currentstroke}{rgb}{0.000000,0.000000,0.000000}%
\pgfsetstrokecolor{currentstroke}%
\pgfsetdash{}{0pt}%
\pgfsys@defobject{currentmarker}{\pgfqpoint{-0.048611in}{0.000000in}}{\pgfqpoint{-0.000000in}{0.000000in}}{%
\pgfpathmoveto{\pgfqpoint{-0.000000in}{0.000000in}}%
\pgfpathlineto{\pgfqpoint{-0.048611in}{0.000000in}}%
\pgfusepath{stroke,fill}%
}%
\begin{pgfscope}%
\pgfsys@transformshift{0.754048in}{0.545305in}%
\pgfsys@useobject{currentmarker}{}%
\end{pgfscope}%
\end{pgfscope}%
\begin{pgfscope}%
\definecolor{textcolor}{rgb}{0.000000,0.000000,0.000000}%
\pgfsetstrokecolor{textcolor}%
\pgfsetfillcolor{textcolor}%
\pgftext[x=0.387917in, y=0.506750in, left, base]{\color{textcolor}\rmfamily\fontsize{8.000000}{9.600000}\selectfont \(\displaystyle {0.000}\)}%
\end{pgfscope}%
\begin{pgfscope}%
\pgfsetbuttcap%
\pgfsetroundjoin%
\definecolor{currentfill}{rgb}{0.000000,0.000000,0.000000}%
\pgfsetfillcolor{currentfill}%
\pgfsetlinewidth{0.803000pt}%
\definecolor{currentstroke}{rgb}{0.000000,0.000000,0.000000}%
\pgfsetstrokecolor{currentstroke}%
\pgfsetdash{}{0pt}%
\pgfsys@defobject{currentmarker}{\pgfqpoint{-0.048611in}{0.000000in}}{\pgfqpoint{-0.000000in}{0.000000in}}{%
\pgfpathmoveto{\pgfqpoint{-0.000000in}{0.000000in}}%
\pgfpathlineto{\pgfqpoint{-0.048611in}{0.000000in}}%
\pgfusepath{stroke,fill}%
}%
\begin{pgfscope}%
\pgfsys@transformshift{0.754048in}{0.700828in}%
\pgfsys@useobject{currentmarker}{}%
\end{pgfscope}%
\end{pgfscope}%
\begin{pgfscope}%
\definecolor{textcolor}{rgb}{0.000000,0.000000,0.000000}%
\pgfsetstrokecolor{textcolor}%
\pgfsetfillcolor{textcolor}%
\pgftext[x=0.387917in, y=0.662272in, left, base]{\color{textcolor}\rmfamily\fontsize{8.000000}{9.600000}\selectfont \(\displaystyle {0.015}\)}%
\end{pgfscope}%
\begin{pgfscope}%
\pgfsetbuttcap%
\pgfsetroundjoin%
\definecolor{currentfill}{rgb}{0.000000,0.000000,0.000000}%
\pgfsetfillcolor{currentfill}%
\pgfsetlinewidth{0.803000pt}%
\definecolor{currentstroke}{rgb}{0.000000,0.000000,0.000000}%
\pgfsetstrokecolor{currentstroke}%
\pgfsetdash{}{0pt}%
\pgfsys@defobject{currentmarker}{\pgfqpoint{-0.048611in}{0.000000in}}{\pgfqpoint{-0.000000in}{0.000000in}}{%
\pgfpathmoveto{\pgfqpoint{-0.000000in}{0.000000in}}%
\pgfpathlineto{\pgfqpoint{-0.048611in}{0.000000in}}%
\pgfusepath{stroke,fill}%
}%
\begin{pgfscope}%
\pgfsys@transformshift{0.754048in}{0.856351in}%
\pgfsys@useobject{currentmarker}{}%
\end{pgfscope}%
\end{pgfscope}%
\begin{pgfscope}%
\definecolor{textcolor}{rgb}{0.000000,0.000000,0.000000}%
\pgfsetstrokecolor{textcolor}%
\pgfsetfillcolor{textcolor}%
\pgftext[x=0.387917in, y=0.817795in, left, base]{\color{textcolor}\rmfamily\fontsize{8.000000}{9.600000}\selectfont \(\displaystyle {0.030}\)}%
\end{pgfscope}%
\begin{pgfscope}%
\pgfsetbuttcap%
\pgfsetroundjoin%
\definecolor{currentfill}{rgb}{0.000000,0.000000,0.000000}%
\pgfsetfillcolor{currentfill}%
\pgfsetlinewidth{0.803000pt}%
\definecolor{currentstroke}{rgb}{0.000000,0.000000,0.000000}%
\pgfsetstrokecolor{currentstroke}%
\pgfsetdash{}{0pt}%
\pgfsys@defobject{currentmarker}{\pgfqpoint{-0.048611in}{0.000000in}}{\pgfqpoint{-0.000000in}{0.000000in}}{%
\pgfpathmoveto{\pgfqpoint{-0.000000in}{0.000000in}}%
\pgfpathlineto{\pgfqpoint{-0.048611in}{0.000000in}}%
\pgfusepath{stroke,fill}%
}%
\begin{pgfscope}%
\pgfsys@transformshift{0.754048in}{1.011873in}%
\pgfsys@useobject{currentmarker}{}%
\end{pgfscope}%
\end{pgfscope}%
\begin{pgfscope}%
\definecolor{textcolor}{rgb}{0.000000,0.000000,0.000000}%
\pgfsetstrokecolor{textcolor}%
\pgfsetfillcolor{textcolor}%
\pgftext[x=0.387917in, y=0.973318in, left, base]{\color{textcolor}\rmfamily\fontsize{8.000000}{9.600000}\selectfont \(\displaystyle {0.045}\)}%
\end{pgfscope}%
\begin{pgfscope}%
\pgfsetbuttcap%
\pgfsetroundjoin%
\definecolor{currentfill}{rgb}{0.000000,0.000000,0.000000}%
\pgfsetfillcolor{currentfill}%
\pgfsetlinewidth{0.803000pt}%
\definecolor{currentstroke}{rgb}{0.000000,0.000000,0.000000}%
\pgfsetstrokecolor{currentstroke}%
\pgfsetdash{}{0pt}%
\pgfsys@defobject{currentmarker}{\pgfqpoint{-0.048611in}{0.000000in}}{\pgfqpoint{-0.000000in}{0.000000in}}{%
\pgfpathmoveto{\pgfqpoint{-0.000000in}{0.000000in}}%
\pgfpathlineto{\pgfqpoint{-0.048611in}{0.000000in}}%
\pgfusepath{stroke,fill}%
}%
\begin{pgfscope}%
\pgfsys@transformshift{0.754048in}{1.167396in}%
\pgfsys@useobject{currentmarker}{}%
\end{pgfscope}%
\end{pgfscope}%
\begin{pgfscope}%
\definecolor{textcolor}{rgb}{0.000000,0.000000,0.000000}%
\pgfsetstrokecolor{textcolor}%
\pgfsetfillcolor{textcolor}%
\pgftext[x=0.387917in, y=1.128840in, left, base]{\color{textcolor}\rmfamily\fontsize{8.000000}{9.600000}\selectfont \(\displaystyle {0.060}\)}%
\end{pgfscope}%
\begin{pgfscope}%
\pgfsetbuttcap%
\pgfsetroundjoin%
\definecolor{currentfill}{rgb}{0.000000,0.000000,0.000000}%
\pgfsetfillcolor{currentfill}%
\pgfsetlinewidth{0.803000pt}%
\definecolor{currentstroke}{rgb}{0.000000,0.000000,0.000000}%
\pgfsetstrokecolor{currentstroke}%
\pgfsetdash{}{0pt}%
\pgfsys@defobject{currentmarker}{\pgfqpoint{-0.048611in}{0.000000in}}{\pgfqpoint{-0.000000in}{0.000000in}}{%
\pgfpathmoveto{\pgfqpoint{-0.000000in}{0.000000in}}%
\pgfpathlineto{\pgfqpoint{-0.048611in}{0.000000in}}%
\pgfusepath{stroke,fill}%
}%
\begin{pgfscope}%
\pgfsys@transformshift{0.754048in}{1.322918in}%
\pgfsys@useobject{currentmarker}{}%
\end{pgfscope}%
\end{pgfscope}%
\begin{pgfscope}%
\definecolor{textcolor}{rgb}{0.000000,0.000000,0.000000}%
\pgfsetstrokecolor{textcolor}%
\pgfsetfillcolor{textcolor}%
\pgftext[x=0.387917in, y=1.284363in, left, base]{\color{textcolor}\rmfamily\fontsize{8.000000}{9.600000}\selectfont \(\displaystyle {0.075}\)}%
\end{pgfscope}%
\begin{pgfscope}%
\pgfsetbuttcap%
\pgfsetroundjoin%
\definecolor{currentfill}{rgb}{0.000000,0.000000,0.000000}%
\pgfsetfillcolor{currentfill}%
\pgfsetlinewidth{0.803000pt}%
\definecolor{currentstroke}{rgb}{0.000000,0.000000,0.000000}%
\pgfsetstrokecolor{currentstroke}%
\pgfsetdash{}{0pt}%
\pgfsys@defobject{currentmarker}{\pgfqpoint{-0.048611in}{0.000000in}}{\pgfqpoint{-0.000000in}{0.000000in}}{%
\pgfpathmoveto{\pgfqpoint{-0.000000in}{0.000000in}}%
\pgfpathlineto{\pgfqpoint{-0.048611in}{0.000000in}}%
\pgfusepath{stroke,fill}%
}%
\begin{pgfscope}%
\pgfsys@transformshift{0.754048in}{1.478441in}%
\pgfsys@useobject{currentmarker}{}%
\end{pgfscope}%
\end{pgfscope}%
\begin{pgfscope}%
\definecolor{textcolor}{rgb}{0.000000,0.000000,0.000000}%
\pgfsetstrokecolor{textcolor}%
\pgfsetfillcolor{textcolor}%
\pgftext[x=0.387917in, y=1.439885in, left, base]{\color{textcolor}\rmfamily\fontsize{8.000000}{9.600000}\selectfont \(\displaystyle {0.090}\)}%
\end{pgfscope}%
\begin{pgfscope}%
\pgfsetbuttcap%
\pgfsetroundjoin%
\definecolor{currentfill}{rgb}{0.000000,0.000000,0.000000}%
\pgfsetfillcolor{currentfill}%
\pgfsetlinewidth{0.803000pt}%
\definecolor{currentstroke}{rgb}{0.000000,0.000000,0.000000}%
\pgfsetstrokecolor{currentstroke}%
\pgfsetdash{}{0pt}%
\pgfsys@defobject{currentmarker}{\pgfqpoint{-0.048611in}{0.000000in}}{\pgfqpoint{-0.000000in}{0.000000in}}{%
\pgfpathmoveto{\pgfqpoint{-0.000000in}{0.000000in}}%
\pgfpathlineto{\pgfqpoint{-0.048611in}{0.000000in}}%
\pgfusepath{stroke,fill}%
}%
\begin{pgfscope}%
\pgfsys@transformshift{0.754048in}{1.633963in}%
\pgfsys@useobject{currentmarker}{}%
\end{pgfscope}%
\end{pgfscope}%
\begin{pgfscope}%
\definecolor{textcolor}{rgb}{0.000000,0.000000,0.000000}%
\pgfsetstrokecolor{textcolor}%
\pgfsetfillcolor{textcolor}%
\pgftext[x=0.387917in, y=1.595408in, left, base]{\color{textcolor}\rmfamily\fontsize{8.000000}{9.600000}\selectfont \(\displaystyle {0.105}\)}%
\end{pgfscope}%
\begin{pgfscope}%
\pgfsetbuttcap%
\pgfsetroundjoin%
\definecolor{currentfill}{rgb}{0.000000,0.000000,0.000000}%
\pgfsetfillcolor{currentfill}%
\pgfsetlinewidth{0.803000pt}%
\definecolor{currentstroke}{rgb}{0.000000,0.000000,0.000000}%
\pgfsetstrokecolor{currentstroke}%
\pgfsetdash{}{0pt}%
\pgfsys@defobject{currentmarker}{\pgfqpoint{-0.048611in}{0.000000in}}{\pgfqpoint{-0.000000in}{0.000000in}}{%
\pgfpathmoveto{\pgfqpoint{-0.000000in}{0.000000in}}%
\pgfpathlineto{\pgfqpoint{-0.048611in}{0.000000in}}%
\pgfusepath{stroke,fill}%
}%
\begin{pgfscope}%
\pgfsys@transformshift{0.754048in}{1.789486in}%
\pgfsys@useobject{currentmarker}{}%
\end{pgfscope}%
\end{pgfscope}%
\begin{pgfscope}%
\definecolor{textcolor}{rgb}{0.000000,0.000000,0.000000}%
\pgfsetstrokecolor{textcolor}%
\pgfsetfillcolor{textcolor}%
\pgftext[x=0.387917in, y=1.750930in, left, base]{\color{textcolor}\rmfamily\fontsize{8.000000}{9.600000}\selectfont \(\displaystyle {0.120}\)}%
\end{pgfscope}%
\begin{pgfscope}%
\pgfsetbuttcap%
\pgfsetroundjoin%
\definecolor{currentfill}{rgb}{0.000000,0.000000,0.000000}%
\pgfsetfillcolor{currentfill}%
\pgfsetlinewidth{0.803000pt}%
\definecolor{currentstroke}{rgb}{0.000000,0.000000,0.000000}%
\pgfsetstrokecolor{currentstroke}%
\pgfsetdash{}{0pt}%
\pgfsys@defobject{currentmarker}{\pgfqpoint{-0.048611in}{0.000000in}}{\pgfqpoint{-0.000000in}{0.000000in}}{%
\pgfpathmoveto{\pgfqpoint{-0.000000in}{0.000000in}}%
\pgfpathlineto{\pgfqpoint{-0.048611in}{0.000000in}}%
\pgfusepath{stroke,fill}%
}%
\begin{pgfscope}%
\pgfsys@transformshift{0.754048in}{1.945009in}%
\pgfsys@useobject{currentmarker}{}%
\end{pgfscope}%
\end{pgfscope}%
\begin{pgfscope}%
\definecolor{textcolor}{rgb}{0.000000,0.000000,0.000000}%
\pgfsetstrokecolor{textcolor}%
\pgfsetfillcolor{textcolor}%
\pgftext[x=0.387917in, y=1.906453in, left, base]{\color{textcolor}\rmfamily\fontsize{8.000000}{9.600000}\selectfont \(\displaystyle {0.135}\)}%
\end{pgfscope}%
\begin{pgfscope}%
\definecolor{textcolor}{rgb}{0.000000,0.000000,0.000000}%
\pgfsetstrokecolor{textcolor}%
\pgfsetfillcolor{textcolor}%
\pgftext[x=0.332362in,y=1.254042in,,bottom,rotate=90.000000]{\color{textcolor}\rmfamily\fontsize{10.000000}{12.000000}\selectfont Density}%
\end{pgfscope}%
\begin{pgfscope}%
\pgfpathrectangle{\pgfqpoint{0.754048in}{0.545305in}}{\pgfqpoint{2.301066in}{1.417472in}}%
\pgfusepath{clip}%
\pgfsetbuttcap%
\pgfsetmiterjoin%
\pgfsetlinewidth{1.003750pt}%
\definecolor{currentstroke}{rgb}{0.313725,0.317647,0.309804}%
\pgfsetstrokecolor{currentstroke}%
\pgfsetdash{}{0pt}%
\pgfpathmoveto{\pgfqpoint{0.858642in}{0.545305in}}%
\pgfpathlineto{\pgfqpoint{0.858642in}{0.545907in}}%
\pgfpathlineto{\pgfqpoint{1.166643in}{0.545907in}}%
\pgfpathlineto{\pgfqpoint{1.166643in}{0.547111in}}%
\pgfpathlineto{\pgfqpoint{1.185500in}{0.546509in}}%
\pgfpathlineto{\pgfqpoint{1.185500in}{0.545907in}}%
\pgfpathlineto{\pgfqpoint{1.216929in}{0.545907in}}%
\pgfpathlineto{\pgfqpoint{1.216929in}{0.547111in}}%
\pgfpathlineto{\pgfqpoint{1.223214in}{0.547111in}}%
\pgfpathlineto{\pgfqpoint{1.223214in}{0.545907in}}%
\pgfpathlineto{\pgfqpoint{1.229500in}{0.545907in}}%
\pgfpathlineto{\pgfqpoint{1.229500in}{0.547713in}}%
\pgfpathlineto{\pgfqpoint{1.235786in}{0.547713in}}%
\pgfpathlineto{\pgfqpoint{1.235786in}{0.545907in}}%
\pgfpathlineto{\pgfqpoint{1.254643in}{0.545907in}}%
\pgfpathlineto{\pgfqpoint{1.254643in}{0.547713in}}%
\pgfpathlineto{\pgfqpoint{1.260929in}{0.547713in}}%
\pgfpathlineto{\pgfqpoint{1.260929in}{0.548916in}}%
\pgfpathlineto{\pgfqpoint{1.267214in}{0.548916in}}%
\pgfpathlineto{\pgfqpoint{1.267214in}{0.545907in}}%
\pgfpathlineto{\pgfqpoint{1.273500in}{0.545907in}}%
\pgfpathlineto{\pgfqpoint{1.273500in}{0.548314in}}%
\pgfpathlineto{\pgfqpoint{1.286072in}{0.547713in}}%
\pgfpathlineto{\pgfqpoint{1.286072in}{0.546509in}}%
\pgfpathlineto{\pgfqpoint{1.298643in}{0.547111in}}%
\pgfpathlineto{\pgfqpoint{1.298643in}{0.547713in}}%
\pgfpathlineto{\pgfqpoint{1.311215in}{0.548314in}}%
\pgfpathlineto{\pgfqpoint{1.311215in}{0.550120in}}%
\pgfpathlineto{\pgfqpoint{1.317500in}{0.550120in}}%
\pgfpathlineto{\pgfqpoint{1.317500in}{0.547713in}}%
\pgfpathlineto{\pgfqpoint{1.323786in}{0.547713in}}%
\pgfpathlineto{\pgfqpoint{1.323786in}{0.550120in}}%
\pgfpathlineto{\pgfqpoint{1.330072in}{0.550120in}}%
\pgfpathlineto{\pgfqpoint{1.330072in}{0.548916in}}%
\pgfpathlineto{\pgfqpoint{1.336357in}{0.548916in}}%
\pgfpathlineto{\pgfqpoint{1.336357in}{0.547713in}}%
\pgfpathlineto{\pgfqpoint{1.342643in}{0.547713in}}%
\pgfpathlineto{\pgfqpoint{1.342643in}{0.550721in}}%
\pgfpathlineto{\pgfqpoint{1.355215in}{0.550120in}}%
\pgfpathlineto{\pgfqpoint{1.355215in}{0.548916in}}%
\pgfpathlineto{\pgfqpoint{1.361500in}{0.548916in}}%
\pgfpathlineto{\pgfqpoint{1.361500in}{0.550721in}}%
\pgfpathlineto{\pgfqpoint{1.367786in}{0.550721in}}%
\pgfpathlineto{\pgfqpoint{1.367786in}{0.548314in}}%
\pgfpathlineto{\pgfqpoint{1.374072in}{0.548314in}}%
\pgfpathlineto{\pgfqpoint{1.374072in}{0.550721in}}%
\pgfpathlineto{\pgfqpoint{1.380358in}{0.550721in}}%
\pgfpathlineto{\pgfqpoint{1.380358in}{0.548314in}}%
\pgfpathlineto{\pgfqpoint{1.386643in}{0.548314in}}%
\pgfpathlineto{\pgfqpoint{1.386643in}{0.551323in}}%
\pgfpathlineto{\pgfqpoint{1.392929in}{0.551323in}}%
\pgfpathlineto{\pgfqpoint{1.392929in}{0.554332in}}%
\pgfpathlineto{\pgfqpoint{1.399215in}{0.554332in}}%
\pgfpathlineto{\pgfqpoint{1.399215in}{0.547713in}}%
\pgfpathlineto{\pgfqpoint{1.405501in}{0.547713in}}%
\pgfpathlineto{\pgfqpoint{1.405501in}{0.550721in}}%
\pgfpathlineto{\pgfqpoint{1.418072in}{0.550721in}}%
\pgfpathlineto{\pgfqpoint{1.418072in}{0.549518in}}%
\pgfpathlineto{\pgfqpoint{1.430643in}{0.548916in}}%
\pgfpathlineto{\pgfqpoint{1.430643in}{0.551925in}}%
\pgfpathlineto{\pgfqpoint{1.443215in}{0.551323in}}%
\pgfpathlineto{\pgfqpoint{1.443215in}{0.553129in}}%
\pgfpathlineto{\pgfqpoint{1.449501in}{0.553129in}}%
\pgfpathlineto{\pgfqpoint{1.449501in}{0.550120in}}%
\pgfpathlineto{\pgfqpoint{1.455786in}{0.550120in}}%
\pgfpathlineto{\pgfqpoint{1.455786in}{0.552527in}}%
\pgfpathlineto{\pgfqpoint{1.474644in}{0.553129in}}%
\pgfpathlineto{\pgfqpoint{1.474644in}{0.554332in}}%
\pgfpathlineto{\pgfqpoint{1.487215in}{0.553730in}}%
\pgfpathlineto{\pgfqpoint{1.487215in}{0.558545in}}%
\pgfpathlineto{\pgfqpoint{1.493501in}{0.558545in}}%
\pgfpathlineto{\pgfqpoint{1.493501in}{0.556138in}}%
\pgfpathlineto{\pgfqpoint{1.499787in}{0.556138in}}%
\pgfpathlineto{\pgfqpoint{1.499787in}{0.553129in}}%
\pgfpathlineto{\pgfqpoint{1.512358in}{0.553730in}}%
\pgfpathlineto{\pgfqpoint{1.512358in}{0.554934in}}%
\pgfpathlineto{\pgfqpoint{1.518644in}{0.554934in}}%
\pgfpathlineto{\pgfqpoint{1.518644in}{0.557341in}}%
\pgfpathlineto{\pgfqpoint{1.524929in}{0.557341in}}%
\pgfpathlineto{\pgfqpoint{1.524929in}{0.552527in}}%
\pgfpathlineto{\pgfqpoint{1.531215in}{0.552527in}}%
\pgfpathlineto{\pgfqpoint{1.531215in}{0.558545in}}%
\pgfpathlineto{\pgfqpoint{1.537501in}{0.558545in}}%
\pgfpathlineto{\pgfqpoint{1.537501in}{0.557341in}}%
\pgfpathlineto{\pgfqpoint{1.543787in}{0.557341in}}%
\pgfpathlineto{\pgfqpoint{1.543787in}{0.558545in}}%
\pgfpathlineto{\pgfqpoint{1.550072in}{0.558545in}}%
\pgfpathlineto{\pgfqpoint{1.550072in}{0.562155in}}%
\pgfpathlineto{\pgfqpoint{1.556358in}{0.562155in}}%
\pgfpathlineto{\pgfqpoint{1.556358in}{0.560350in}}%
\pgfpathlineto{\pgfqpoint{1.562644in}{0.560350in}}%
\pgfpathlineto{\pgfqpoint{1.562644in}{0.559146in}}%
\pgfpathlineto{\pgfqpoint{1.568930in}{0.559146in}}%
\pgfpathlineto{\pgfqpoint{1.568930in}{0.562155in}}%
\pgfpathlineto{\pgfqpoint{1.575215in}{0.562155in}}%
\pgfpathlineto{\pgfqpoint{1.575215in}{0.554332in}}%
\pgfpathlineto{\pgfqpoint{1.581501in}{0.554332in}}%
\pgfpathlineto{\pgfqpoint{1.581501in}{0.559748in}}%
\pgfpathlineto{\pgfqpoint{1.587787in}{0.559748in}}%
\pgfpathlineto{\pgfqpoint{1.587787in}{0.565766in}}%
\pgfpathlineto{\pgfqpoint{1.594073in}{0.565766in}}%
\pgfpathlineto{\pgfqpoint{1.594073in}{0.566970in}}%
\pgfpathlineto{\pgfqpoint{1.600358in}{0.566970in}}%
\pgfpathlineto{\pgfqpoint{1.600358in}{0.560952in}}%
\pgfpathlineto{\pgfqpoint{1.606644in}{0.560952in}}%
\pgfpathlineto{\pgfqpoint{1.606644in}{0.569979in}}%
\pgfpathlineto{\pgfqpoint{1.612930in}{0.569979in}}%
\pgfpathlineto{\pgfqpoint{1.612930in}{0.568775in}}%
\pgfpathlineto{\pgfqpoint{1.619215in}{0.568775in}}%
\pgfpathlineto{\pgfqpoint{1.619215in}{0.570580in}}%
\pgfpathlineto{\pgfqpoint{1.625501in}{0.570580in}}%
\pgfpathlineto{\pgfqpoint{1.625501in}{0.569377in}}%
\pgfpathlineto{\pgfqpoint{1.631787in}{0.569377in}}%
\pgfpathlineto{\pgfqpoint{1.631787in}{0.563359in}}%
\pgfpathlineto{\pgfqpoint{1.638073in}{0.563359in}}%
\pgfpathlineto{\pgfqpoint{1.638073in}{0.567571in}}%
\pgfpathlineto{\pgfqpoint{1.644358in}{0.567571in}}%
\pgfpathlineto{\pgfqpoint{1.644358in}{0.569377in}}%
\pgfpathlineto{\pgfqpoint{1.650644in}{0.569377in}}%
\pgfpathlineto{\pgfqpoint{1.650644in}{0.573589in}}%
\pgfpathlineto{\pgfqpoint{1.656930in}{0.573589in}}%
\pgfpathlineto{\pgfqpoint{1.656930in}{0.575395in}}%
\pgfpathlineto{\pgfqpoint{1.663216in}{0.575395in}}%
\pgfpathlineto{\pgfqpoint{1.663216in}{0.566368in}}%
\pgfpathlineto{\pgfqpoint{1.669501in}{0.566368in}}%
\pgfpathlineto{\pgfqpoint{1.669501in}{0.575395in}}%
\pgfpathlineto{\pgfqpoint{1.675787in}{0.575395in}}%
\pgfpathlineto{\pgfqpoint{1.675787in}{0.577802in}}%
\pgfpathlineto{\pgfqpoint{1.694644in}{0.578404in}}%
\pgfpathlineto{\pgfqpoint{1.694644in}{0.589236in}}%
\pgfpathlineto{\pgfqpoint{1.700930in}{0.589236in}}%
\pgfpathlineto{\pgfqpoint{1.700930in}{0.585023in}}%
\pgfpathlineto{\pgfqpoint{1.707216in}{0.585023in}}%
\pgfpathlineto{\pgfqpoint{1.707216in}{0.583218in}}%
\pgfpathlineto{\pgfqpoint{1.713501in}{0.583218in}}%
\pgfpathlineto{\pgfqpoint{1.713501in}{0.580811in}}%
\pgfpathlineto{\pgfqpoint{1.719787in}{0.580811in}}%
\pgfpathlineto{\pgfqpoint{1.719787in}{0.588032in}}%
\pgfpathlineto{\pgfqpoint{1.726073in}{0.588032in}}%
\pgfpathlineto{\pgfqpoint{1.726073in}{0.590439in}}%
\pgfpathlineto{\pgfqpoint{1.732359in}{0.590439in}}%
\pgfpathlineto{\pgfqpoint{1.732359in}{0.589236in}}%
\pgfpathlineto{\pgfqpoint{1.738644in}{0.589236in}}%
\pgfpathlineto{\pgfqpoint{1.738644in}{0.605484in}}%
\pgfpathlineto{\pgfqpoint{1.744930in}{0.605484in}}%
\pgfpathlineto{\pgfqpoint{1.744930in}{0.599466in}}%
\pgfpathlineto{\pgfqpoint{1.751216in}{0.599466in}}%
\pgfpathlineto{\pgfqpoint{1.751216in}{0.600670in}}%
\pgfpathlineto{\pgfqpoint{1.757502in}{0.600670in}}%
\pgfpathlineto{\pgfqpoint{1.757502in}{0.606687in}}%
\pgfpathlineto{\pgfqpoint{1.763787in}{0.606687in}}%
\pgfpathlineto{\pgfqpoint{1.763787in}{0.612104in}}%
\pgfpathlineto{\pgfqpoint{1.770073in}{0.612104in}}%
\pgfpathlineto{\pgfqpoint{1.770073in}{0.610900in}}%
\pgfpathlineto{\pgfqpoint{1.776359in}{0.610900in}}%
\pgfpathlineto{\pgfqpoint{1.776359in}{0.625945in}}%
\pgfpathlineto{\pgfqpoint{1.782644in}{0.625945in}}%
\pgfpathlineto{\pgfqpoint{1.782644in}{0.618121in}}%
\pgfpathlineto{\pgfqpoint{1.788930in}{0.618121in}}%
\pgfpathlineto{\pgfqpoint{1.788930in}{0.626546in}}%
\pgfpathlineto{\pgfqpoint{1.795216in}{0.626546in}}%
\pgfpathlineto{\pgfqpoint{1.795216in}{0.649414in}}%
\pgfpathlineto{\pgfqpoint{1.807787in}{0.649414in}}%
\pgfpathlineto{\pgfqpoint{1.807787in}{0.652423in}}%
\pgfpathlineto{\pgfqpoint{1.814073in}{0.652423in}}%
\pgfpathlineto{\pgfqpoint{1.814073in}{0.657839in}}%
\pgfpathlineto{\pgfqpoint{1.820359in}{0.657839in}}%
\pgfpathlineto{\pgfqpoint{1.820359in}{0.684318in}}%
\pgfpathlineto{\pgfqpoint{1.826645in}{0.684318in}}%
\pgfpathlineto{\pgfqpoint{1.826645in}{0.703575in}}%
\pgfpathlineto{\pgfqpoint{1.832930in}{0.703575in}}%
\pgfpathlineto{\pgfqpoint{1.832930in}{0.719823in}}%
\pgfpathlineto{\pgfqpoint{1.839216in}{0.719823in}}%
\pgfpathlineto{\pgfqpoint{1.839216in}{0.744496in}}%
\pgfpathlineto{\pgfqpoint{1.845502in}{0.744496in}}%
\pgfpathlineto{\pgfqpoint{1.845502in}{0.767364in}}%
\pgfpathlineto{\pgfqpoint{1.851788in}{0.767364in}}%
\pgfpathlineto{\pgfqpoint{1.851788in}{0.794444in}}%
\pgfpathlineto{\pgfqpoint{1.858073in}{0.794444in}}%
\pgfpathlineto{\pgfqpoint{1.858073in}{0.855225in}}%
\pgfpathlineto{\pgfqpoint{1.864359in}{0.855225in}}%
\pgfpathlineto{\pgfqpoint{1.864359in}{0.882305in}}%
\pgfpathlineto{\pgfqpoint{1.870645in}{0.882305in}}%
\pgfpathlineto{\pgfqpoint{1.870645in}{0.958731in}}%
\pgfpathlineto{\pgfqpoint{1.876930in}{0.958731in}}%
\pgfpathlineto{\pgfqpoint{1.876930in}{0.987617in}}%
\pgfpathlineto{\pgfqpoint{1.883216in}{0.987617in}}%
\pgfpathlineto{\pgfqpoint{1.883216in}{1.050203in}}%
\pgfpathlineto{\pgfqpoint{1.889502in}{1.050203in}}%
\pgfpathlineto{\pgfqpoint{1.889502in}{1.083301in}}%
\pgfpathlineto{\pgfqpoint{1.895788in}{1.083301in}}%
\pgfpathlineto{\pgfqpoint{1.895788in}{1.138063in}}%
\pgfpathlineto{\pgfqpoint{1.902073in}{1.138063in}}%
\pgfpathlineto{\pgfqpoint{1.902073in}{1.115195in}}%
\pgfpathlineto{\pgfqpoint{1.908359in}{1.115195in}}%
\pgfpathlineto{\pgfqpoint{1.908359in}{1.101956in}}%
\pgfpathlineto{\pgfqpoint{1.914645in}{1.101956in}}%
\pgfpathlineto{\pgfqpoint{1.914645in}{1.089921in}}%
\pgfpathlineto{\pgfqpoint{1.920931in}{1.089921in}}%
\pgfpathlineto{\pgfqpoint{1.920931in}{1.036362in}}%
\pgfpathlineto{\pgfqpoint{1.927216in}{1.036362in}}%
\pgfpathlineto{\pgfqpoint{1.927216in}{0.938873in}}%
\pgfpathlineto{\pgfqpoint{1.933502in}{0.938873in}}%
\pgfpathlineto{\pgfqpoint{1.933502in}{0.930448in}}%
\pgfpathlineto{\pgfqpoint{1.939788in}{0.930448in}}%
\pgfpathlineto{\pgfqpoint{1.939788in}{0.847401in}}%
\pgfpathlineto{\pgfqpoint{1.946074in}{0.847401in}}%
\pgfpathlineto{\pgfqpoint{1.946074in}{0.829950in}}%
\pgfpathlineto{\pgfqpoint{1.952359in}{0.829950in}}%
\pgfpathlineto{\pgfqpoint{1.952359in}{0.787223in}}%
\pgfpathlineto{\pgfqpoint{1.958645in}{0.787223in}}%
\pgfpathlineto{\pgfqpoint{1.958645in}{0.756532in}}%
\pgfpathlineto{\pgfqpoint{1.964931in}{0.756532in}}%
\pgfpathlineto{\pgfqpoint{1.964931in}{0.707787in}}%
\pgfpathlineto{\pgfqpoint{1.971216in}{0.707787in}}%
\pgfpathlineto{\pgfqpoint{1.971216in}{0.694548in}}%
\pgfpathlineto{\pgfqpoint{1.977502in}{0.694548in}}%
\pgfpathlineto{\pgfqpoint{1.977502in}{0.682512in}}%
\pgfpathlineto{\pgfqpoint{1.983788in}{0.682512in}}%
\pgfpathlineto{\pgfqpoint{1.983788in}{0.671680in}}%
\pgfpathlineto{\pgfqpoint{1.990074in}{0.671680in}}%
\pgfpathlineto{\pgfqpoint{1.990074in}{0.651821in}}%
\pgfpathlineto{\pgfqpoint{1.996359in}{0.651821in}}%
\pgfpathlineto{\pgfqpoint{1.996359in}{0.642795in}}%
\pgfpathlineto{\pgfqpoint{2.002645in}{0.642795in}}%
\pgfpathlineto{\pgfqpoint{2.002645in}{0.633768in}}%
\pgfpathlineto{\pgfqpoint{2.008931in}{0.633768in}}%
\pgfpathlineto{\pgfqpoint{2.008931in}{0.632564in}}%
\pgfpathlineto{\pgfqpoint{2.015217in}{0.632564in}}%
\pgfpathlineto{\pgfqpoint{2.015217in}{0.624741in}}%
\pgfpathlineto{\pgfqpoint{2.021502in}{0.624741in}}%
\pgfpathlineto{\pgfqpoint{2.021502in}{0.601271in}}%
\pgfpathlineto{\pgfqpoint{2.034074in}{0.600670in}}%
\pgfpathlineto{\pgfqpoint{2.034074in}{0.607891in}}%
\pgfpathlineto{\pgfqpoint{2.040360in}{0.607891in}}%
\pgfpathlineto{\pgfqpoint{2.040360in}{0.604280in}}%
\pgfpathlineto{\pgfqpoint{2.046645in}{0.604280in}}%
\pgfpathlineto{\pgfqpoint{2.046645in}{0.598864in}}%
\pgfpathlineto{\pgfqpoint{2.052931in}{0.598864in}}%
\pgfpathlineto{\pgfqpoint{2.052931in}{0.582616in}}%
\pgfpathlineto{\pgfqpoint{2.059217in}{0.582616in}}%
\pgfpathlineto{\pgfqpoint{2.059217in}{0.594050in}}%
\pgfpathlineto{\pgfqpoint{2.065502in}{0.594050in}}%
\pgfpathlineto{\pgfqpoint{2.065502in}{0.590439in}}%
\pgfpathlineto{\pgfqpoint{2.071788in}{0.590439in}}%
\pgfpathlineto{\pgfqpoint{2.071788in}{0.586227in}}%
\pgfpathlineto{\pgfqpoint{2.078074in}{0.586227in}}%
\pgfpathlineto{\pgfqpoint{2.078074in}{0.575996in}}%
\pgfpathlineto{\pgfqpoint{2.084360in}{0.575996in}}%
\pgfpathlineto{\pgfqpoint{2.084360in}{0.584421in}}%
\pgfpathlineto{\pgfqpoint{2.090645in}{0.584421in}}%
\pgfpathlineto{\pgfqpoint{2.090645in}{0.572988in}}%
\pgfpathlineto{\pgfqpoint{2.096931in}{0.572988in}}%
\pgfpathlineto{\pgfqpoint{2.096931in}{0.577200in}}%
\pgfpathlineto{\pgfqpoint{2.103217in}{0.577200in}}%
\pgfpathlineto{\pgfqpoint{2.103217in}{0.569979in}}%
\pgfpathlineto{\pgfqpoint{2.109503in}{0.569979in}}%
\pgfpathlineto{\pgfqpoint{2.109503in}{0.568173in}}%
\pgfpathlineto{\pgfqpoint{2.115788in}{0.568173in}}%
\pgfpathlineto{\pgfqpoint{2.115788in}{0.571784in}}%
\pgfpathlineto{\pgfqpoint{2.122074in}{0.571784in}}%
\pgfpathlineto{\pgfqpoint{2.122074in}{0.577200in}}%
\pgfpathlineto{\pgfqpoint{2.128360in}{0.577200in}}%
\pgfpathlineto{\pgfqpoint{2.128360in}{0.563961in}}%
\pgfpathlineto{\pgfqpoint{2.147217in}{0.563359in}}%
\pgfpathlineto{\pgfqpoint{2.147217in}{0.562757in}}%
\pgfpathlineto{\pgfqpoint{2.153503in}{0.562757in}}%
\pgfpathlineto{\pgfqpoint{2.153503in}{0.560350in}}%
\pgfpathlineto{\pgfqpoint{2.159788in}{0.560350in}}%
\pgfpathlineto{\pgfqpoint{2.159788in}{0.564563in}}%
\pgfpathlineto{\pgfqpoint{2.166074in}{0.564563in}}%
\pgfpathlineto{\pgfqpoint{2.166074in}{0.560952in}}%
\pgfpathlineto{\pgfqpoint{2.172360in}{0.560952in}}%
\pgfpathlineto{\pgfqpoint{2.172360in}{0.562757in}}%
\pgfpathlineto{\pgfqpoint{2.184931in}{0.562757in}}%
\pgfpathlineto{\pgfqpoint{2.184931in}{0.560952in}}%
\pgfpathlineto{\pgfqpoint{2.197503in}{0.561554in}}%
\pgfpathlineto{\pgfqpoint{2.197503in}{0.562155in}}%
\pgfpathlineto{\pgfqpoint{2.203789in}{0.562155in}}%
\pgfpathlineto{\pgfqpoint{2.203789in}{0.563359in}}%
\pgfpathlineto{\pgfqpoint{2.210074in}{0.563359in}}%
\pgfpathlineto{\pgfqpoint{2.210074in}{0.560952in}}%
\pgfpathlineto{\pgfqpoint{2.216360in}{0.560952in}}%
\pgfpathlineto{\pgfqpoint{2.216360in}{0.556138in}}%
\pgfpathlineto{\pgfqpoint{2.222646in}{0.556138in}}%
\pgfpathlineto{\pgfqpoint{2.222646in}{0.554332in}}%
\pgfpathlineto{\pgfqpoint{2.228931in}{0.554332in}}%
\pgfpathlineto{\pgfqpoint{2.228931in}{0.556739in}}%
\pgfpathlineto{\pgfqpoint{2.235217in}{0.556739in}}%
\pgfpathlineto{\pgfqpoint{2.235217in}{0.554934in}}%
\pgfpathlineto{\pgfqpoint{2.241503in}{0.554934in}}%
\pgfpathlineto{\pgfqpoint{2.241503in}{0.551323in}}%
\pgfpathlineto{\pgfqpoint{2.247789in}{0.551323in}}%
\pgfpathlineto{\pgfqpoint{2.247789in}{0.553129in}}%
\pgfpathlineto{\pgfqpoint{2.254074in}{0.553129in}}%
\pgfpathlineto{\pgfqpoint{2.254074in}{0.554332in}}%
\pgfpathlineto{\pgfqpoint{2.260360in}{0.554332in}}%
\pgfpathlineto{\pgfqpoint{2.260360in}{0.552527in}}%
\pgfpathlineto{\pgfqpoint{2.266646in}{0.552527in}}%
\pgfpathlineto{\pgfqpoint{2.266646in}{0.550721in}}%
\pgfpathlineto{\pgfqpoint{2.272932in}{0.550721in}}%
\pgfpathlineto{\pgfqpoint{2.272932in}{0.554934in}}%
\pgfpathlineto{\pgfqpoint{2.279217in}{0.554934in}}%
\pgfpathlineto{\pgfqpoint{2.279217in}{0.552527in}}%
\pgfpathlineto{\pgfqpoint{2.285503in}{0.552527in}}%
\pgfpathlineto{\pgfqpoint{2.285503in}{0.548314in}}%
\pgfpathlineto{\pgfqpoint{2.291789in}{0.548314in}}%
\pgfpathlineto{\pgfqpoint{2.291789in}{0.550120in}}%
\pgfpathlineto{\pgfqpoint{2.298075in}{0.550120in}}%
\pgfpathlineto{\pgfqpoint{2.298075in}{0.552527in}}%
\pgfpathlineto{\pgfqpoint{2.304360in}{0.552527in}}%
\pgfpathlineto{\pgfqpoint{2.304360in}{0.557341in}}%
\pgfpathlineto{\pgfqpoint{2.310646in}{0.557341in}}%
\pgfpathlineto{\pgfqpoint{2.310646in}{0.550721in}}%
\pgfpathlineto{\pgfqpoint{2.323217in}{0.550120in}}%
\pgfpathlineto{\pgfqpoint{2.323217in}{0.549518in}}%
\pgfpathlineto{\pgfqpoint{2.329503in}{0.549518in}}%
\pgfpathlineto{\pgfqpoint{2.329503in}{0.551323in}}%
\pgfpathlineto{\pgfqpoint{2.335789in}{0.551323in}}%
\pgfpathlineto{\pgfqpoint{2.335789in}{0.548916in}}%
\pgfpathlineto{\pgfqpoint{2.348360in}{0.548314in}}%
\pgfpathlineto{\pgfqpoint{2.348360in}{0.551323in}}%
\pgfpathlineto{\pgfqpoint{2.354646in}{0.551323in}}%
\pgfpathlineto{\pgfqpoint{2.354646in}{0.549518in}}%
\pgfpathlineto{\pgfqpoint{2.373503in}{0.549518in}}%
\pgfpathlineto{\pgfqpoint{2.373503in}{0.548314in}}%
\pgfpathlineto{\pgfqpoint{2.379789in}{0.548314in}}%
\pgfpathlineto{\pgfqpoint{2.379789in}{0.550721in}}%
\pgfpathlineto{\pgfqpoint{2.386075in}{0.550721in}}%
\pgfpathlineto{\pgfqpoint{2.386075in}{0.547713in}}%
\pgfpathlineto{\pgfqpoint{2.392361in}{0.547713in}}%
\pgfpathlineto{\pgfqpoint{2.392361in}{0.551925in}}%
\pgfpathlineto{\pgfqpoint{2.398646in}{0.551925in}}%
\pgfpathlineto{\pgfqpoint{2.398646in}{0.549518in}}%
\pgfpathlineto{\pgfqpoint{2.404932in}{0.549518in}}%
\pgfpathlineto{\pgfqpoint{2.404932in}{0.547713in}}%
\pgfpathlineto{\pgfqpoint{2.423789in}{0.547713in}}%
\pgfpathlineto{\pgfqpoint{2.423789in}{0.546509in}}%
\pgfpathlineto{\pgfqpoint{2.430075in}{0.546509in}}%
\pgfpathlineto{\pgfqpoint{2.430075in}{0.548916in}}%
\pgfpathlineto{\pgfqpoint{2.455218in}{0.548314in}}%
\pgfpathlineto{\pgfqpoint{2.455218in}{0.546509in}}%
\pgfpathlineto{\pgfqpoint{2.461504in}{0.546509in}}%
\pgfpathlineto{\pgfqpoint{2.461504in}{0.548314in}}%
\pgfpathlineto{\pgfqpoint{2.474075in}{0.548916in}}%
\pgfpathlineto{\pgfqpoint{2.474075in}{0.546509in}}%
\pgfpathlineto{\pgfqpoint{2.480361in}{0.546509in}}%
\pgfpathlineto{\pgfqpoint{2.480361in}{0.548314in}}%
\pgfpathlineto{\pgfqpoint{2.486647in}{0.548314in}}%
\pgfpathlineto{\pgfqpoint{2.486647in}{0.547111in}}%
\pgfpathlineto{\pgfqpoint{2.511789in}{0.547713in}}%
\pgfpathlineto{\pgfqpoint{2.511789in}{0.545305in}}%
\pgfpathlineto{\pgfqpoint{2.518075in}{0.545305in}}%
\pgfpathlineto{\pgfqpoint{2.518075in}{0.547713in}}%
\pgfpathlineto{\pgfqpoint{2.524361in}{0.547713in}}%
\pgfpathlineto{\pgfqpoint{2.524361in}{0.546509in}}%
\pgfpathlineto{\pgfqpoint{2.536932in}{0.546509in}}%
\pgfpathlineto{\pgfqpoint{2.536932in}{0.547713in}}%
\pgfpathlineto{\pgfqpoint{2.543218in}{0.547713in}}%
\pgfpathlineto{\pgfqpoint{2.543218in}{0.546509in}}%
\pgfpathlineto{\pgfqpoint{2.555790in}{0.546509in}}%
\pgfpathlineto{\pgfqpoint{2.555790in}{0.545305in}}%
\pgfpathlineto{\pgfqpoint{2.562075in}{0.545305in}}%
\pgfpathlineto{\pgfqpoint{2.562075in}{0.546509in}}%
\pgfpathlineto{\pgfqpoint{2.568361in}{0.546509in}}%
\pgfpathlineto{\pgfqpoint{2.568361in}{0.545305in}}%
\pgfpathlineto{\pgfqpoint{2.593504in}{0.545907in}}%
\pgfpathlineto{\pgfqpoint{2.593504in}{0.546509in}}%
\pgfpathlineto{\pgfqpoint{2.599790in}{0.546509in}}%
\pgfpathlineto{\pgfqpoint{2.599790in}{0.545305in}}%
\pgfpathlineto{\pgfqpoint{2.650076in}{0.545305in}}%
\pgfpathlineto{\pgfqpoint{2.650076in}{0.546509in}}%
\pgfpathlineto{\pgfqpoint{2.656361in}{0.546509in}}%
\pgfpathlineto{\pgfqpoint{2.656361in}{0.545305in}}%
\pgfpathlineto{\pgfqpoint{2.775790in}{0.545907in}}%
\pgfpathlineto{\pgfqpoint{2.775790in}{0.545305in}}%
\pgfusepath{stroke}%
\end{pgfscope}%
\begin{pgfscope}%
\pgfpathrectangle{\pgfqpoint{0.754048in}{0.545305in}}{\pgfqpoint{2.301066in}{1.417472in}}%
\pgfusepath{clip}%
\pgfsetbuttcap%
\pgfsetmiterjoin%
\pgfsetlinewidth{1.003750pt}%
\definecolor{currentstroke}{rgb}{0.949020,0.372549,0.360784}%
\pgfsetstrokecolor{currentstroke}%
\pgfsetdash{{1.000000pt}{1.650000pt}}{0.000000pt}%
\pgfpathmoveto{\pgfqpoint{0.858642in}{0.545305in}}%
\pgfpathlineto{\pgfqpoint{0.858642in}{0.545908in}}%
\pgfpathlineto{\pgfqpoint{1.128928in}{0.545908in}}%
\pgfpathlineto{\pgfqpoint{1.128928in}{0.547114in}}%
\pgfpathlineto{\pgfqpoint{1.135214in}{0.547114in}}%
\pgfpathlineto{\pgfqpoint{1.135214in}{0.548320in}}%
\pgfpathlineto{\pgfqpoint{1.141500in}{0.548320in}}%
\pgfpathlineto{\pgfqpoint{1.141500in}{0.546511in}}%
\pgfpathlineto{\pgfqpoint{1.147786in}{0.546511in}}%
\pgfpathlineto{\pgfqpoint{1.147786in}{0.545305in}}%
\pgfpathlineto{\pgfqpoint{1.154071in}{0.545305in}}%
\pgfpathlineto{\pgfqpoint{1.154071in}{0.547114in}}%
\pgfpathlineto{\pgfqpoint{1.185500in}{0.547114in}}%
\pgfpathlineto{\pgfqpoint{1.185500in}{0.548320in}}%
\pgfpathlineto{\pgfqpoint{1.191786in}{0.548320in}}%
\pgfpathlineto{\pgfqpoint{1.191786in}{0.545908in}}%
\pgfpathlineto{\pgfqpoint{1.198071in}{0.545908in}}%
\pgfpathlineto{\pgfqpoint{1.198071in}{0.548320in}}%
\pgfpathlineto{\pgfqpoint{1.204357in}{0.548320in}}%
\pgfpathlineto{\pgfqpoint{1.204357in}{0.545305in}}%
\pgfpathlineto{\pgfqpoint{1.216929in}{0.545908in}}%
\pgfpathlineto{\pgfqpoint{1.216929in}{0.547114in}}%
\pgfpathlineto{\pgfqpoint{1.223214in}{0.547114in}}%
\pgfpathlineto{\pgfqpoint{1.223214in}{0.549526in}}%
\pgfpathlineto{\pgfqpoint{1.229500in}{0.549526in}}%
\pgfpathlineto{\pgfqpoint{1.229500in}{0.547717in}}%
\pgfpathlineto{\pgfqpoint{1.235786in}{0.547717in}}%
\pgfpathlineto{\pgfqpoint{1.235786in}{0.546511in}}%
\pgfpathlineto{\pgfqpoint{1.267214in}{0.546511in}}%
\pgfpathlineto{\pgfqpoint{1.267214in}{0.551335in}}%
\pgfpathlineto{\pgfqpoint{1.273500in}{0.551335in}}%
\pgfpathlineto{\pgfqpoint{1.273500in}{0.547114in}}%
\pgfpathlineto{\pgfqpoint{1.279786in}{0.547114in}}%
\pgfpathlineto{\pgfqpoint{1.279786in}{0.548320in}}%
\pgfpathlineto{\pgfqpoint{1.286072in}{0.548320in}}%
\pgfpathlineto{\pgfqpoint{1.286072in}{0.549526in}}%
\pgfpathlineto{\pgfqpoint{1.292357in}{0.549526in}}%
\pgfpathlineto{\pgfqpoint{1.292357in}{0.548320in}}%
\pgfpathlineto{\pgfqpoint{1.298643in}{0.548320in}}%
\pgfpathlineto{\pgfqpoint{1.298643in}{0.549526in}}%
\pgfpathlineto{\pgfqpoint{1.304929in}{0.549526in}}%
\pgfpathlineto{\pgfqpoint{1.304929in}{0.551335in}}%
\pgfpathlineto{\pgfqpoint{1.311215in}{0.551335in}}%
\pgfpathlineto{\pgfqpoint{1.311215in}{0.545908in}}%
\pgfpathlineto{\pgfqpoint{1.317500in}{0.545908in}}%
\pgfpathlineto{\pgfqpoint{1.317500in}{0.547717in}}%
\pgfpathlineto{\pgfqpoint{1.330072in}{0.548320in}}%
\pgfpathlineto{\pgfqpoint{1.330072in}{0.549526in}}%
\pgfpathlineto{\pgfqpoint{1.342643in}{0.550129in}}%
\pgfpathlineto{\pgfqpoint{1.342643in}{0.550732in}}%
\pgfpathlineto{\pgfqpoint{1.348929in}{0.550732in}}%
\pgfpathlineto{\pgfqpoint{1.348929in}{0.548320in}}%
\pgfpathlineto{\pgfqpoint{1.361500in}{0.548923in}}%
\pgfpathlineto{\pgfqpoint{1.361500in}{0.550129in}}%
\pgfpathlineto{\pgfqpoint{1.374072in}{0.550732in}}%
\pgfpathlineto{\pgfqpoint{1.374072in}{0.547717in}}%
\pgfpathlineto{\pgfqpoint{1.380358in}{0.547717in}}%
\pgfpathlineto{\pgfqpoint{1.380358in}{0.549526in}}%
\pgfpathlineto{\pgfqpoint{1.386643in}{0.549526in}}%
\pgfpathlineto{\pgfqpoint{1.386643in}{0.548320in}}%
\pgfpathlineto{\pgfqpoint{1.392929in}{0.548320in}}%
\pgfpathlineto{\pgfqpoint{1.392929in}{0.549526in}}%
\pgfpathlineto{\pgfqpoint{1.405501in}{0.550129in}}%
\pgfpathlineto{\pgfqpoint{1.405501in}{0.547114in}}%
\pgfpathlineto{\pgfqpoint{1.411786in}{0.547114in}}%
\pgfpathlineto{\pgfqpoint{1.411786in}{0.554349in}}%
\pgfpathlineto{\pgfqpoint{1.418072in}{0.554349in}}%
\pgfpathlineto{\pgfqpoint{1.418072in}{0.548923in}}%
\pgfpathlineto{\pgfqpoint{1.424358in}{0.548923in}}%
\pgfpathlineto{\pgfqpoint{1.424358in}{0.553747in}}%
\pgfpathlineto{\pgfqpoint{1.430643in}{0.553747in}}%
\pgfpathlineto{\pgfqpoint{1.430643in}{0.549526in}}%
\pgfpathlineto{\pgfqpoint{1.436929in}{0.549526in}}%
\pgfpathlineto{\pgfqpoint{1.436929in}{0.551335in}}%
\pgfpathlineto{\pgfqpoint{1.443215in}{0.551335in}}%
\pgfpathlineto{\pgfqpoint{1.443215in}{0.550129in}}%
\pgfpathlineto{\pgfqpoint{1.449501in}{0.550129in}}%
\pgfpathlineto{\pgfqpoint{1.449501in}{0.553144in}}%
\pgfpathlineto{\pgfqpoint{1.455786in}{0.553144in}}%
\pgfpathlineto{\pgfqpoint{1.455786in}{0.551335in}}%
\pgfpathlineto{\pgfqpoint{1.462072in}{0.551335in}}%
\pgfpathlineto{\pgfqpoint{1.462072in}{0.552541in}}%
\pgfpathlineto{\pgfqpoint{1.468358in}{0.552541in}}%
\pgfpathlineto{\pgfqpoint{1.468358in}{0.551335in}}%
\pgfpathlineto{\pgfqpoint{1.474644in}{0.551335in}}%
\pgfpathlineto{\pgfqpoint{1.474644in}{0.554349in}}%
\pgfpathlineto{\pgfqpoint{1.480929in}{0.554349in}}%
\pgfpathlineto{\pgfqpoint{1.480929in}{0.553144in}}%
\pgfpathlineto{\pgfqpoint{1.487215in}{0.553144in}}%
\pgfpathlineto{\pgfqpoint{1.487215in}{0.555555in}}%
\pgfpathlineto{\pgfqpoint{1.499787in}{0.556158in}}%
\pgfpathlineto{\pgfqpoint{1.499787in}{0.552541in}}%
\pgfpathlineto{\pgfqpoint{1.506072in}{0.552541in}}%
\pgfpathlineto{\pgfqpoint{1.506072in}{0.549526in}}%
\pgfpathlineto{\pgfqpoint{1.512358in}{0.549526in}}%
\pgfpathlineto{\pgfqpoint{1.512358in}{0.552541in}}%
\pgfpathlineto{\pgfqpoint{1.518644in}{0.552541in}}%
\pgfpathlineto{\pgfqpoint{1.518644in}{0.557364in}}%
\pgfpathlineto{\pgfqpoint{1.524929in}{0.557364in}}%
\pgfpathlineto{\pgfqpoint{1.524929in}{0.551335in}}%
\pgfpathlineto{\pgfqpoint{1.531215in}{0.551335in}}%
\pgfpathlineto{\pgfqpoint{1.531215in}{0.549526in}}%
\pgfpathlineto{\pgfqpoint{1.537501in}{0.549526in}}%
\pgfpathlineto{\pgfqpoint{1.537501in}{0.559776in}}%
\pgfpathlineto{\pgfqpoint{1.543787in}{0.559776in}}%
\pgfpathlineto{\pgfqpoint{1.543787in}{0.558570in}}%
\pgfpathlineto{\pgfqpoint{1.550072in}{0.558570in}}%
\pgfpathlineto{\pgfqpoint{1.550072in}{0.556761in}}%
\pgfpathlineto{\pgfqpoint{1.556358in}{0.556761in}}%
\pgfpathlineto{\pgfqpoint{1.556358in}{0.552541in}}%
\pgfpathlineto{\pgfqpoint{1.562644in}{0.552541in}}%
\pgfpathlineto{\pgfqpoint{1.562644in}{0.556158in}}%
\pgfpathlineto{\pgfqpoint{1.581501in}{0.556761in}}%
\pgfpathlineto{\pgfqpoint{1.581501in}{0.560379in}}%
\pgfpathlineto{\pgfqpoint{1.587787in}{0.560379in}}%
\pgfpathlineto{\pgfqpoint{1.587787in}{0.564599in}}%
\pgfpathlineto{\pgfqpoint{1.594073in}{0.564599in}}%
\pgfpathlineto{\pgfqpoint{1.594073in}{0.560379in}}%
\pgfpathlineto{\pgfqpoint{1.600358in}{0.560379in}}%
\pgfpathlineto{\pgfqpoint{1.600358in}{0.563996in}}%
\pgfpathlineto{\pgfqpoint{1.606644in}{0.563996in}}%
\pgfpathlineto{\pgfqpoint{1.606644in}{0.556761in}}%
\pgfpathlineto{\pgfqpoint{1.619215in}{0.556158in}}%
\pgfpathlineto{\pgfqpoint{1.619215in}{0.560982in}}%
\pgfpathlineto{\pgfqpoint{1.625501in}{0.560982in}}%
\pgfpathlineto{\pgfqpoint{1.625501in}{0.559776in}}%
\pgfpathlineto{\pgfqpoint{1.631787in}{0.559776in}}%
\pgfpathlineto{\pgfqpoint{1.631787in}{0.555555in}}%
\pgfpathlineto{\pgfqpoint{1.638073in}{0.555555in}}%
\pgfpathlineto{\pgfqpoint{1.638073in}{0.563393in}}%
\pgfpathlineto{\pgfqpoint{1.644358in}{0.563393in}}%
\pgfpathlineto{\pgfqpoint{1.644358in}{0.564599in}}%
\pgfpathlineto{\pgfqpoint{1.656930in}{0.563996in}}%
\pgfpathlineto{\pgfqpoint{1.656930in}{0.562791in}}%
\pgfpathlineto{\pgfqpoint{1.663216in}{0.562791in}}%
\pgfpathlineto{\pgfqpoint{1.663216in}{0.557364in}}%
\pgfpathlineto{\pgfqpoint{1.669501in}{0.557364in}}%
\pgfpathlineto{\pgfqpoint{1.669501in}{0.567011in}}%
\pgfpathlineto{\pgfqpoint{1.675787in}{0.567011in}}%
\pgfpathlineto{\pgfqpoint{1.675787in}{0.565805in}}%
\pgfpathlineto{\pgfqpoint{1.682073in}{0.565805in}}%
\pgfpathlineto{\pgfqpoint{1.682073in}{0.572438in}}%
\pgfpathlineto{\pgfqpoint{1.688359in}{0.572438in}}%
\pgfpathlineto{\pgfqpoint{1.688359in}{0.571232in}}%
\pgfpathlineto{\pgfqpoint{1.694644in}{0.571232in}}%
\pgfpathlineto{\pgfqpoint{1.694644in}{0.567011in}}%
\pgfpathlineto{\pgfqpoint{1.713501in}{0.567011in}}%
\pgfpathlineto{\pgfqpoint{1.713501in}{0.568217in}}%
\pgfpathlineto{\pgfqpoint{1.719787in}{0.568217in}}%
\pgfpathlineto{\pgfqpoint{1.719787in}{0.572438in}}%
\pgfpathlineto{\pgfqpoint{1.726073in}{0.572438in}}%
\pgfpathlineto{\pgfqpoint{1.726073in}{0.579673in}}%
\pgfpathlineto{\pgfqpoint{1.732359in}{0.579673in}}%
\pgfpathlineto{\pgfqpoint{1.732359in}{0.567011in}}%
\pgfpathlineto{\pgfqpoint{1.738644in}{0.567011in}}%
\pgfpathlineto{\pgfqpoint{1.738644in}{0.568820in}}%
\pgfpathlineto{\pgfqpoint{1.744930in}{0.568820in}}%
\pgfpathlineto{\pgfqpoint{1.744930in}{0.577261in}}%
\pgfpathlineto{\pgfqpoint{1.757502in}{0.577864in}}%
\pgfpathlineto{\pgfqpoint{1.757502in}{0.582687in}}%
\pgfpathlineto{\pgfqpoint{1.763787in}{0.582687in}}%
\pgfpathlineto{\pgfqpoint{1.763787in}{0.587511in}}%
\pgfpathlineto{\pgfqpoint{1.776359in}{0.586908in}}%
\pgfpathlineto{\pgfqpoint{1.776359in}{0.589320in}}%
\pgfpathlineto{\pgfqpoint{1.782644in}{0.589320in}}%
\pgfpathlineto{\pgfqpoint{1.782644in}{0.593540in}}%
\pgfpathlineto{\pgfqpoint{1.795216in}{0.594143in}}%
\pgfpathlineto{\pgfqpoint{1.795216in}{0.604393in}}%
\pgfpathlineto{\pgfqpoint{1.807787in}{0.604393in}}%
\pgfpathlineto{\pgfqpoint{1.807787in}{0.609217in}}%
\pgfpathlineto{\pgfqpoint{1.814073in}{0.609217in}}%
\pgfpathlineto{\pgfqpoint{1.814073in}{0.619467in}}%
\pgfpathlineto{\pgfqpoint{1.820359in}{0.619467in}}%
\pgfpathlineto{\pgfqpoint{1.820359in}{0.614040in}}%
\pgfpathlineto{\pgfqpoint{1.826645in}{0.614040in}}%
\pgfpathlineto{\pgfqpoint{1.826645in}{0.639363in}}%
\pgfpathlineto{\pgfqpoint{1.832930in}{0.639363in}}%
\pgfpathlineto{\pgfqpoint{1.832930in}{0.629716in}}%
\pgfpathlineto{\pgfqpoint{1.839216in}{0.629716in}}%
\pgfpathlineto{\pgfqpoint{1.839216in}{0.662275in}}%
\pgfpathlineto{\pgfqpoint{1.845502in}{0.662275in}}%
\pgfpathlineto{\pgfqpoint{1.845502in}{0.673128in}}%
\pgfpathlineto{\pgfqpoint{1.851788in}{0.673128in}}%
\pgfpathlineto{\pgfqpoint{1.851788in}{0.697848in}}%
\pgfpathlineto{\pgfqpoint{1.858073in}{0.697848in}}%
\pgfpathlineto{\pgfqpoint{1.858073in}{0.729201in}}%
\pgfpathlineto{\pgfqpoint{1.864359in}{0.729201in}}%
\pgfpathlineto{\pgfqpoint{1.864359in}{0.798539in}}%
\pgfpathlineto{\pgfqpoint{1.870645in}{0.798539in}}%
\pgfpathlineto{\pgfqpoint{1.870645in}{0.847979in}}%
\pgfpathlineto{\pgfqpoint{1.876930in}{0.847979in}}%
\pgfpathlineto{\pgfqpoint{1.876930in}{0.970375in}}%
\pgfpathlineto{\pgfqpoint{1.883216in}{0.970375in}}%
\pgfpathlineto{\pgfqpoint{1.883216in}{1.166932in}}%
\pgfpathlineto{\pgfqpoint{1.889502in}{1.166932in}}%
\pgfpathlineto{\pgfqpoint{1.889502in}{1.406901in}}%
\pgfpathlineto{\pgfqpoint{1.895788in}{1.406901in}}%
\pgfpathlineto{\pgfqpoint{1.895788in}{1.810868in}}%
\pgfpathlineto{\pgfqpoint{1.902073in}{1.810868in}}%
\pgfpathlineto{\pgfqpoint{1.902073in}{1.895279in}}%
\pgfpathlineto{\pgfqpoint{1.908359in}{1.895279in}}%
\pgfpathlineto{\pgfqpoint{1.908359in}{1.607076in}}%
\pgfpathlineto{\pgfqpoint{1.914645in}{1.607076in}}%
\pgfpathlineto{\pgfqpoint{1.914645in}{1.228432in}}%
\pgfpathlineto{\pgfqpoint{1.920931in}{1.228432in}}%
\pgfpathlineto{\pgfqpoint{1.920931in}{1.034286in}}%
\pgfpathlineto{\pgfqpoint{1.927216in}{1.034286in}}%
\pgfpathlineto{\pgfqpoint{1.927216in}{0.896214in}}%
\pgfpathlineto{\pgfqpoint{1.933502in}{0.896214in}}%
\pgfpathlineto{\pgfqpoint{1.933502in}{0.802156in}}%
\pgfpathlineto{\pgfqpoint{1.939788in}{0.802156in}}%
\pgfpathlineto{\pgfqpoint{1.939788in}{0.748495in}}%
\pgfpathlineto{\pgfqpoint{1.946074in}{0.748495in}}%
\pgfpathlineto{\pgfqpoint{1.946074in}{0.705686in}}%
\pgfpathlineto{\pgfqpoint{1.952359in}{0.705686in}}%
\pgfpathlineto{\pgfqpoint{1.952359in}{0.679157in}}%
\pgfpathlineto{\pgfqpoint{1.958645in}{0.679157in}}%
\pgfpathlineto{\pgfqpoint{1.958645in}{0.658054in}}%
\pgfpathlineto{\pgfqpoint{1.964931in}{0.658054in}}%
\pgfpathlineto{\pgfqpoint{1.964931in}{0.641172in}}%
\pgfpathlineto{\pgfqpoint{1.971216in}{0.641172in}}%
\pgfpathlineto{\pgfqpoint{1.971216in}{0.619467in}}%
\pgfpathlineto{\pgfqpoint{1.977502in}{0.619467in}}%
\pgfpathlineto{\pgfqpoint{1.977502in}{0.604996in}}%
\pgfpathlineto{\pgfqpoint{1.983788in}{0.604996in}}%
\pgfpathlineto{\pgfqpoint{1.983788in}{0.622481in}}%
\pgfpathlineto{\pgfqpoint{1.990074in}{0.622481in}}%
\pgfpathlineto{\pgfqpoint{1.990074in}{0.614643in}}%
\pgfpathlineto{\pgfqpoint{1.996359in}{0.614643in}}%
\pgfpathlineto{\pgfqpoint{1.996359in}{0.608614in}}%
\pgfpathlineto{\pgfqpoint{2.002645in}{0.608614in}}%
\pgfpathlineto{\pgfqpoint{2.002645in}{0.592937in}}%
\pgfpathlineto{\pgfqpoint{2.008931in}{0.592937in}}%
\pgfpathlineto{\pgfqpoint{2.008931in}{0.595952in}}%
\pgfpathlineto{\pgfqpoint{2.015217in}{0.595952in}}%
\pgfpathlineto{\pgfqpoint{2.015217in}{0.589923in}}%
\pgfpathlineto{\pgfqpoint{2.021502in}{0.589923in}}%
\pgfpathlineto{\pgfqpoint{2.021502in}{0.577864in}}%
\pgfpathlineto{\pgfqpoint{2.027788in}{0.577864in}}%
\pgfpathlineto{\pgfqpoint{2.027788in}{0.586908in}}%
\pgfpathlineto{\pgfqpoint{2.034074in}{0.586908in}}%
\pgfpathlineto{\pgfqpoint{2.034074in}{0.582085in}}%
\pgfpathlineto{\pgfqpoint{2.040360in}{0.582085in}}%
\pgfpathlineto{\pgfqpoint{2.040360in}{0.579070in}}%
\pgfpathlineto{\pgfqpoint{2.046645in}{0.579070in}}%
\pgfpathlineto{\pgfqpoint{2.046645in}{0.582687in}}%
\pgfpathlineto{\pgfqpoint{2.052931in}{0.582687in}}%
\pgfpathlineto{\pgfqpoint{2.052931in}{0.575452in}}%
\pgfpathlineto{\pgfqpoint{2.059217in}{0.575452in}}%
\pgfpathlineto{\pgfqpoint{2.059217in}{0.570629in}}%
\pgfpathlineto{\pgfqpoint{2.065502in}{0.570629in}}%
\pgfpathlineto{\pgfqpoint{2.065502in}{0.577864in}}%
\pgfpathlineto{\pgfqpoint{2.071788in}{0.577864in}}%
\pgfpathlineto{\pgfqpoint{2.071788in}{0.564599in}}%
\pgfpathlineto{\pgfqpoint{2.078074in}{0.564599in}}%
\pgfpathlineto{\pgfqpoint{2.078074in}{0.568820in}}%
\pgfpathlineto{\pgfqpoint{2.084360in}{0.568820in}}%
\pgfpathlineto{\pgfqpoint{2.084360in}{0.574849in}}%
\pgfpathlineto{\pgfqpoint{2.090645in}{0.574849in}}%
\pgfpathlineto{\pgfqpoint{2.090645in}{0.568820in}}%
\pgfpathlineto{\pgfqpoint{2.096931in}{0.568820in}}%
\pgfpathlineto{\pgfqpoint{2.096931in}{0.571232in}}%
\pgfpathlineto{\pgfqpoint{2.109503in}{0.570629in}}%
\pgfpathlineto{\pgfqpoint{2.109503in}{0.573643in}}%
\pgfpathlineto{\pgfqpoint{2.115788in}{0.573643in}}%
\pgfpathlineto{\pgfqpoint{2.115788in}{0.567011in}}%
\pgfpathlineto{\pgfqpoint{2.122074in}{0.567011in}}%
\pgfpathlineto{\pgfqpoint{2.122074in}{0.565805in}}%
\pgfpathlineto{\pgfqpoint{2.128360in}{0.565805in}}%
\pgfpathlineto{\pgfqpoint{2.128360in}{0.563393in}}%
\pgfpathlineto{\pgfqpoint{2.134645in}{0.563393in}}%
\pgfpathlineto{\pgfqpoint{2.134645in}{0.557967in}}%
\pgfpathlineto{\pgfqpoint{2.140931in}{0.557967in}}%
\pgfpathlineto{\pgfqpoint{2.140931in}{0.563996in}}%
\pgfpathlineto{\pgfqpoint{2.147217in}{0.563996in}}%
\pgfpathlineto{\pgfqpoint{2.147217in}{0.565805in}}%
\pgfpathlineto{\pgfqpoint{2.153503in}{0.565805in}}%
\pgfpathlineto{\pgfqpoint{2.153503in}{0.559173in}}%
\pgfpathlineto{\pgfqpoint{2.159788in}{0.559173in}}%
\pgfpathlineto{\pgfqpoint{2.159788in}{0.555555in}}%
\pgfpathlineto{\pgfqpoint{2.166074in}{0.555555in}}%
\pgfpathlineto{\pgfqpoint{2.166074in}{0.563393in}}%
\pgfpathlineto{\pgfqpoint{2.178646in}{0.563393in}}%
\pgfpathlineto{\pgfqpoint{2.178646in}{0.560982in}}%
\pgfpathlineto{\pgfqpoint{2.184931in}{0.560982in}}%
\pgfpathlineto{\pgfqpoint{2.184931in}{0.562188in}}%
\pgfpathlineto{\pgfqpoint{2.191217in}{0.562188in}}%
\pgfpathlineto{\pgfqpoint{2.191217in}{0.565202in}}%
\pgfpathlineto{\pgfqpoint{2.197503in}{0.565202in}}%
\pgfpathlineto{\pgfqpoint{2.197503in}{0.562791in}}%
\pgfpathlineto{\pgfqpoint{2.203789in}{0.562791in}}%
\pgfpathlineto{\pgfqpoint{2.203789in}{0.560379in}}%
\pgfpathlineto{\pgfqpoint{2.210074in}{0.560379in}}%
\pgfpathlineto{\pgfqpoint{2.210074in}{0.558570in}}%
\pgfpathlineto{\pgfqpoint{2.216360in}{0.558570in}}%
\pgfpathlineto{\pgfqpoint{2.216360in}{0.555555in}}%
\pgfpathlineto{\pgfqpoint{2.222646in}{0.555555in}}%
\pgfpathlineto{\pgfqpoint{2.222646in}{0.561585in}}%
\pgfpathlineto{\pgfqpoint{2.228931in}{0.561585in}}%
\pgfpathlineto{\pgfqpoint{2.228931in}{0.560379in}}%
\pgfpathlineto{\pgfqpoint{2.235217in}{0.560379in}}%
\pgfpathlineto{\pgfqpoint{2.235217in}{0.557967in}}%
\pgfpathlineto{\pgfqpoint{2.254074in}{0.557967in}}%
\pgfpathlineto{\pgfqpoint{2.254074in}{0.552541in}}%
\pgfpathlineto{\pgfqpoint{2.260360in}{0.552541in}}%
\pgfpathlineto{\pgfqpoint{2.260360in}{0.554952in}}%
\pgfpathlineto{\pgfqpoint{2.266646in}{0.554952in}}%
\pgfpathlineto{\pgfqpoint{2.266646in}{0.553144in}}%
\pgfpathlineto{\pgfqpoint{2.272932in}{0.553144in}}%
\pgfpathlineto{\pgfqpoint{2.272932in}{0.557364in}}%
\pgfpathlineto{\pgfqpoint{2.279217in}{0.557364in}}%
\pgfpathlineto{\pgfqpoint{2.279217in}{0.556158in}}%
\pgfpathlineto{\pgfqpoint{2.291789in}{0.556761in}}%
\pgfpathlineto{\pgfqpoint{2.291789in}{0.558570in}}%
\pgfpathlineto{\pgfqpoint{2.304360in}{0.559173in}}%
\pgfpathlineto{\pgfqpoint{2.304360in}{0.553144in}}%
\pgfpathlineto{\pgfqpoint{2.310646in}{0.553144in}}%
\pgfpathlineto{\pgfqpoint{2.310646in}{0.556158in}}%
\pgfpathlineto{\pgfqpoint{2.316932in}{0.556158in}}%
\pgfpathlineto{\pgfqpoint{2.316932in}{0.551938in}}%
\pgfpathlineto{\pgfqpoint{2.329503in}{0.552541in}}%
\pgfpathlineto{\pgfqpoint{2.329503in}{0.553144in}}%
\pgfpathlineto{\pgfqpoint{2.342075in}{0.553747in}}%
\pgfpathlineto{\pgfqpoint{2.342075in}{0.556761in}}%
\pgfpathlineto{\pgfqpoint{2.348360in}{0.556761in}}%
\pgfpathlineto{\pgfqpoint{2.348360in}{0.553144in}}%
\pgfpathlineto{\pgfqpoint{2.392361in}{0.553747in}}%
\pgfpathlineto{\pgfqpoint{2.392361in}{0.549526in}}%
\pgfpathlineto{\pgfqpoint{2.398646in}{0.549526in}}%
\pgfpathlineto{\pgfqpoint{2.398646in}{0.553144in}}%
\pgfpathlineto{\pgfqpoint{2.404932in}{0.553144in}}%
\pgfpathlineto{\pgfqpoint{2.404932in}{0.548320in}}%
\pgfpathlineto{\pgfqpoint{2.411218in}{0.548320in}}%
\pgfpathlineto{\pgfqpoint{2.411218in}{0.551335in}}%
\pgfpathlineto{\pgfqpoint{2.417503in}{0.551335in}}%
\pgfpathlineto{\pgfqpoint{2.417503in}{0.550129in}}%
\pgfpathlineto{\pgfqpoint{2.430075in}{0.549526in}}%
\pgfpathlineto{\pgfqpoint{2.430075in}{0.551335in}}%
\pgfpathlineto{\pgfqpoint{2.436361in}{0.551335in}}%
\pgfpathlineto{\pgfqpoint{2.436361in}{0.549526in}}%
\pgfpathlineto{\pgfqpoint{2.448932in}{0.550129in}}%
\pgfpathlineto{\pgfqpoint{2.448932in}{0.551335in}}%
\pgfpathlineto{\pgfqpoint{2.461504in}{0.551938in}}%
\pgfpathlineto{\pgfqpoint{2.461504in}{0.550129in}}%
\pgfpathlineto{\pgfqpoint{2.467789in}{0.550129in}}%
\pgfpathlineto{\pgfqpoint{2.467789in}{0.548320in}}%
\pgfpathlineto{\pgfqpoint{2.474075in}{0.548320in}}%
\pgfpathlineto{\pgfqpoint{2.474075in}{0.554952in}}%
\pgfpathlineto{\pgfqpoint{2.480361in}{0.554952in}}%
\pgfpathlineto{\pgfqpoint{2.480361in}{0.550129in}}%
\pgfpathlineto{\pgfqpoint{2.492932in}{0.550732in}}%
\pgfpathlineto{\pgfqpoint{2.492932in}{0.551938in}}%
\pgfpathlineto{\pgfqpoint{2.499218in}{0.551938in}}%
\pgfpathlineto{\pgfqpoint{2.499218in}{0.549526in}}%
\pgfpathlineto{\pgfqpoint{2.524361in}{0.550129in}}%
\pgfpathlineto{\pgfqpoint{2.524361in}{0.548320in}}%
\pgfpathlineto{\pgfqpoint{2.530647in}{0.548320in}}%
\pgfpathlineto{\pgfqpoint{2.530647in}{0.549526in}}%
\pgfpathlineto{\pgfqpoint{2.536932in}{0.549526in}}%
\pgfpathlineto{\pgfqpoint{2.536932in}{0.551335in}}%
\pgfpathlineto{\pgfqpoint{2.543218in}{0.551335in}}%
\pgfpathlineto{\pgfqpoint{2.543218in}{0.549526in}}%
\pgfpathlineto{\pgfqpoint{2.549504in}{0.549526in}}%
\pgfpathlineto{\pgfqpoint{2.549504in}{0.551938in}}%
\pgfpathlineto{\pgfqpoint{2.562075in}{0.551335in}}%
\pgfpathlineto{\pgfqpoint{2.562075in}{0.550732in}}%
\pgfpathlineto{\pgfqpoint{2.568361in}{0.550732in}}%
\pgfpathlineto{\pgfqpoint{2.568361in}{0.548320in}}%
\pgfpathlineto{\pgfqpoint{2.587218in}{0.547717in}}%
\pgfpathlineto{\pgfqpoint{2.587218in}{0.549526in}}%
\pgfpathlineto{\pgfqpoint{2.599790in}{0.550129in}}%
\pgfpathlineto{\pgfqpoint{2.599790in}{0.547114in}}%
\pgfpathlineto{\pgfqpoint{2.606075in}{0.547114in}}%
\pgfpathlineto{\pgfqpoint{2.606075in}{0.548923in}}%
\pgfpathlineto{\pgfqpoint{2.612361in}{0.548923in}}%
\pgfpathlineto{\pgfqpoint{2.612361in}{0.550129in}}%
\pgfpathlineto{\pgfqpoint{2.618647in}{0.550129in}}%
\pgfpathlineto{\pgfqpoint{2.618647in}{0.547717in}}%
\pgfpathlineto{\pgfqpoint{2.624933in}{0.547717in}}%
\pgfpathlineto{\pgfqpoint{2.624933in}{0.549526in}}%
\pgfpathlineto{\pgfqpoint{2.637504in}{0.550129in}}%
\pgfpathlineto{\pgfqpoint{2.637504in}{0.545908in}}%
\pgfpathlineto{\pgfqpoint{2.650076in}{0.545908in}}%
\pgfpathlineto{\pgfqpoint{2.650076in}{0.548320in}}%
\pgfpathlineto{\pgfqpoint{2.668933in}{0.548320in}}%
\pgfpathlineto{\pgfqpoint{2.668933in}{0.546511in}}%
\pgfpathlineto{\pgfqpoint{2.675218in}{0.546511in}}%
\pgfpathlineto{\pgfqpoint{2.675218in}{0.547717in}}%
\pgfpathlineto{\pgfqpoint{2.687790in}{0.547717in}}%
\pgfpathlineto{\pgfqpoint{2.687790in}{0.546511in}}%
\pgfpathlineto{\pgfqpoint{2.719219in}{0.547114in}}%
\pgfpathlineto{\pgfqpoint{2.719219in}{0.547717in}}%
\pgfpathlineto{\pgfqpoint{2.725504in}{0.547717in}}%
\pgfpathlineto{\pgfqpoint{2.725504in}{0.546511in}}%
\pgfpathlineto{\pgfqpoint{2.763219in}{0.545908in}}%
\pgfpathlineto{\pgfqpoint{2.763219in}{0.545305in}}%
\pgfpathlineto{\pgfqpoint{2.775790in}{0.545305in}}%
\pgfpathlineto{\pgfqpoint{2.775790in}{0.545305in}}%
\pgfusepath{stroke}%
\end{pgfscope}%
\begin{pgfscope}%
\pgfsetrectcap%
\pgfsetmiterjoin%
\pgfsetlinewidth{0.803000pt}%
\definecolor{currentstroke}{rgb}{0.000000,0.000000,0.000000}%
\pgfsetstrokecolor{currentstroke}%
\pgfsetdash{}{0pt}%
\pgfpathmoveto{\pgfqpoint{0.754048in}{0.545305in}}%
\pgfpathlineto{\pgfqpoint{0.754048in}{1.962778in}}%
\pgfusepath{stroke}%
\end{pgfscope}%
\begin{pgfscope}%
\pgfsetrectcap%
\pgfsetmiterjoin%
\pgfsetlinewidth{0.803000pt}%
\definecolor{currentstroke}{rgb}{0.000000,0.000000,0.000000}%
\pgfsetstrokecolor{currentstroke}%
\pgfsetdash{}{0pt}%
\pgfpathmoveto{\pgfqpoint{3.055114in}{0.545305in}}%
\pgfpathlineto{\pgfqpoint{3.055114in}{1.962778in}}%
\pgfusepath{stroke}%
\end{pgfscope}%
\begin{pgfscope}%
\pgfsetrectcap%
\pgfsetmiterjoin%
\pgfsetlinewidth{0.803000pt}%
\definecolor{currentstroke}{rgb}{0.000000,0.000000,0.000000}%
\pgfsetstrokecolor{currentstroke}%
\pgfsetdash{}{0pt}%
\pgfpathmoveto{\pgfqpoint{0.754048in}{0.545305in}}%
\pgfpathlineto{\pgfqpoint{3.055114in}{0.545305in}}%
\pgfusepath{stroke}%
\end{pgfscope}%
\begin{pgfscope}%
\pgfsetrectcap%
\pgfsetmiterjoin%
\pgfsetlinewidth{0.803000pt}%
\definecolor{currentstroke}{rgb}{0.000000,0.000000,0.000000}%
\pgfsetstrokecolor{currentstroke}%
\pgfsetdash{}{0pt}%
\pgfpathmoveto{\pgfqpoint{0.754048in}{1.962778in}}%
\pgfpathlineto{\pgfqpoint{3.055114in}{1.962778in}}%
\pgfusepath{stroke}%
\end{pgfscope}%
\begin{pgfscope}%
\definecolor{textcolor}{rgb}{0.000000,0.000000,0.000000}%
\pgfsetstrokecolor{textcolor}%
\pgfsetfillcolor{textcolor}%
\pgftext[x=0.754048in,y=2.046111in,left,base]{\color{textcolor}\rmfamily\fontsize{10.000000}{12.000000}\selectfont Bin [2.33, 2.5), 20,948 events}%
\end{pgfscope}%
\begin{pgfscope}%
\pgfsetbuttcap%
\pgfsetmiterjoin%
\definecolor{currentfill}{rgb}{1.000000,1.000000,1.000000}%
\pgfsetfillcolor{currentfill}%
\pgfsetfillopacity{0.800000}%
\pgfsetlinewidth{1.003750pt}%
\definecolor{currentstroke}{rgb}{0.800000,0.800000,0.800000}%
\pgfsetstrokecolor{currentstroke}%
\pgfsetstrokeopacity{0.800000}%
\pgfsetdash{}{0pt}%
\pgfpathmoveto{\pgfqpoint{2.020337in}{1.563222in}}%
\pgfpathlineto{\pgfqpoint{2.977337in}{1.563222in}}%
\pgfpathquadraticcurveto{\pgfqpoint{2.999559in}{1.563222in}}{\pgfqpoint{2.999559in}{1.585444in}}%
\pgfpathlineto{\pgfqpoint{2.999559in}{1.885000in}}%
\pgfpathquadraticcurveto{\pgfqpoint{2.999559in}{1.907222in}}{\pgfqpoint{2.977337in}{1.907222in}}%
\pgfpathlineto{\pgfqpoint{2.020337in}{1.907222in}}%
\pgfpathquadraticcurveto{\pgfqpoint{1.998114in}{1.907222in}}{\pgfqpoint{1.998114in}{1.885000in}}%
\pgfpathlineto{\pgfqpoint{1.998114in}{1.585444in}}%
\pgfpathquadraticcurveto{\pgfqpoint{1.998114in}{1.563222in}}{\pgfqpoint{2.020337in}{1.563222in}}%
\pgfpathclose%
\pgfusepath{stroke,fill}%
\end{pgfscope}%
\begin{pgfscope}%
\pgfsetbuttcap%
\pgfsetmiterjoin%
\pgfsetlinewidth{1.003750pt}%
\definecolor{currentstroke}{rgb}{0.313725,0.317647,0.309804}%
\pgfsetstrokecolor{currentstroke}%
\pgfsetdash{}{0pt}%
\pgfpathmoveto{\pgfqpoint{2.042559in}{1.784555in}}%
\pgfpathlineto{\pgfqpoint{2.264781in}{1.784555in}}%
\pgfpathlineto{\pgfqpoint{2.264781in}{1.862333in}}%
\pgfpathlineto{\pgfqpoint{2.042559in}{1.862333in}}%
\pgfpathclose%
\pgfusepath{stroke}%
\end{pgfscope}%
\begin{pgfscope}%
\definecolor{textcolor}{rgb}{0.000000,0.000000,0.000000}%
\pgfsetstrokecolor{textcolor}%
\pgfsetfillcolor{textcolor}%
\pgftext[x=2.353670in,y=1.784555in,left,base]{\color{textcolor}\rmfamily\fontsize{8.000000}{9.600000}\selectfont IQR = 8.41}%
\end{pgfscope}%
\begin{pgfscope}%
\pgfsetbuttcap%
\pgfsetmiterjoin%
\pgfsetlinewidth{1.003750pt}%
\definecolor{currentstroke}{rgb}{0.949020,0.372549,0.360784}%
\pgfsetstrokecolor{currentstroke}%
\pgfsetdash{{1.000000pt}{1.650000pt}}{0.000000pt}%
\pgfpathmoveto{\pgfqpoint{2.042559in}{1.629222in}}%
\pgfpathlineto{\pgfqpoint{2.264781in}{1.629222in}}%
\pgfpathlineto{\pgfqpoint{2.264781in}{1.707000in}}%
\pgfpathlineto{\pgfqpoint{2.042559in}{1.707000in}}%
\pgfpathclose%
\pgfusepath{stroke}%
\end{pgfscope}%
\begin{pgfscope}%
\definecolor{textcolor}{rgb}{0.000000,0.000000,0.000000}%
\pgfsetstrokecolor{textcolor}%
\pgfsetfillcolor{textcolor}%
\pgftext[x=2.353670in,y=1.629222in,left,base]{\color{textcolor}\rmfamily\fontsize{8.000000}{9.600000}\selectfont IQR = 4.22}%
\end{pgfscope}%
\begin{pgfscope}%
\pgfsetbuttcap%
\pgfsetmiterjoin%
\definecolor{currentfill}{rgb}{1.000000,1.000000,1.000000}%
\pgfsetfillcolor{currentfill}%
\pgfsetlinewidth{0.000000pt}%
\definecolor{currentstroke}{rgb}{0.000000,0.000000,0.000000}%
\pgfsetstrokecolor{currentstroke}%
\pgfsetstrokeopacity{0.000000}%
\pgfsetdash{}{0pt}%
\pgfpathmoveto{\pgfqpoint{3.750134in}{0.545305in}}%
\pgfpathlineto{\pgfqpoint{6.051200in}{0.545305in}}%
\pgfpathlineto{\pgfqpoint{6.051200in}{1.962778in}}%
\pgfpathlineto{\pgfqpoint{3.750134in}{1.962778in}}%
\pgfpathclose%
\pgfusepath{fill}%
\end{pgfscope}%
\begin{pgfscope}%
\pgfsetbuttcap%
\pgfsetroundjoin%
\definecolor{currentfill}{rgb}{0.000000,0.000000,0.000000}%
\pgfsetfillcolor{currentfill}%
\pgfsetlinewidth{0.803000pt}%
\definecolor{currentstroke}{rgb}{0.000000,0.000000,0.000000}%
\pgfsetstrokecolor{currentstroke}%
\pgfsetdash{}{0pt}%
\pgfsys@defobject{currentmarker}{\pgfqpoint{0.000000in}{-0.048611in}}{\pgfqpoint{0.000000in}{0.000000in}}{%
\pgfpathmoveto{\pgfqpoint{0.000000in}{0.000000in}}%
\pgfpathlineto{\pgfqpoint{0.000000in}{-0.048611in}}%
\pgfusepath{stroke,fill}%
}%
\begin{pgfscope}%
\pgfsys@transformshift{3.982265in}{0.545305in}%
\pgfsys@useobject{currentmarker}{}%
\end{pgfscope}%
\end{pgfscope}%
\begin{pgfscope}%
\definecolor{textcolor}{rgb}{0.000000,0.000000,0.000000}%
\pgfsetstrokecolor{textcolor}%
\pgfsetfillcolor{textcolor}%
\pgftext[x=3.982265in,y=0.448083in,,top]{\color{textcolor}\rmfamily\fontsize{8.000000}{9.600000}\selectfont \(\displaystyle {-120}\)}%
\end{pgfscope}%
\begin{pgfscope}%
\pgfsetbuttcap%
\pgfsetroundjoin%
\definecolor{currentfill}{rgb}{0.000000,0.000000,0.000000}%
\pgfsetfillcolor{currentfill}%
\pgfsetlinewidth{0.803000pt}%
\definecolor{currentstroke}{rgb}{0.000000,0.000000,0.000000}%
\pgfsetstrokecolor{currentstroke}%
\pgfsetdash{}{0pt}%
\pgfsys@defobject{currentmarker}{\pgfqpoint{0.000000in}{-0.048611in}}{\pgfqpoint{0.000000in}{0.000000in}}{%
\pgfpathmoveto{\pgfqpoint{0.000000in}{0.000000in}}%
\pgfpathlineto{\pgfqpoint{0.000000in}{-0.048611in}}%
\pgfusepath{stroke,fill}%
}%
\begin{pgfscope}%
\pgfsys@transformshift{4.287966in}{0.545305in}%
\pgfsys@useobject{currentmarker}{}%
\end{pgfscope}%
\end{pgfscope}%
\begin{pgfscope}%
\definecolor{textcolor}{rgb}{0.000000,0.000000,0.000000}%
\pgfsetstrokecolor{textcolor}%
\pgfsetfillcolor{textcolor}%
\pgftext[x=4.287966in,y=0.448083in,,top]{\color{textcolor}\rmfamily\fontsize{8.000000}{9.600000}\selectfont \(\displaystyle {-80}\)}%
\end{pgfscope}%
\begin{pgfscope}%
\pgfsetbuttcap%
\pgfsetroundjoin%
\definecolor{currentfill}{rgb}{0.000000,0.000000,0.000000}%
\pgfsetfillcolor{currentfill}%
\pgfsetlinewidth{0.803000pt}%
\definecolor{currentstroke}{rgb}{0.000000,0.000000,0.000000}%
\pgfsetstrokecolor{currentstroke}%
\pgfsetdash{}{0pt}%
\pgfsys@defobject{currentmarker}{\pgfqpoint{0.000000in}{-0.048611in}}{\pgfqpoint{0.000000in}{0.000000in}}{%
\pgfpathmoveto{\pgfqpoint{0.000000in}{0.000000in}}%
\pgfpathlineto{\pgfqpoint{0.000000in}{-0.048611in}}%
\pgfusepath{stroke,fill}%
}%
\begin{pgfscope}%
\pgfsys@transformshift{4.593668in}{0.545305in}%
\pgfsys@useobject{currentmarker}{}%
\end{pgfscope}%
\end{pgfscope}%
\begin{pgfscope}%
\definecolor{textcolor}{rgb}{0.000000,0.000000,0.000000}%
\pgfsetstrokecolor{textcolor}%
\pgfsetfillcolor{textcolor}%
\pgftext[x=4.593668in,y=0.448083in,,top]{\color{textcolor}\rmfamily\fontsize{8.000000}{9.600000}\selectfont \(\displaystyle {-40}\)}%
\end{pgfscope}%
\begin{pgfscope}%
\pgfsetbuttcap%
\pgfsetroundjoin%
\definecolor{currentfill}{rgb}{0.000000,0.000000,0.000000}%
\pgfsetfillcolor{currentfill}%
\pgfsetlinewidth{0.803000pt}%
\definecolor{currentstroke}{rgb}{0.000000,0.000000,0.000000}%
\pgfsetstrokecolor{currentstroke}%
\pgfsetdash{}{0pt}%
\pgfsys@defobject{currentmarker}{\pgfqpoint{0.000000in}{-0.048611in}}{\pgfqpoint{0.000000in}{0.000000in}}{%
\pgfpathmoveto{\pgfqpoint{0.000000in}{0.000000in}}%
\pgfpathlineto{\pgfqpoint{0.000000in}{-0.048611in}}%
\pgfusepath{stroke,fill}%
}%
\begin{pgfscope}%
\pgfsys@transformshift{4.899369in}{0.545305in}%
\pgfsys@useobject{currentmarker}{}%
\end{pgfscope}%
\end{pgfscope}%
\begin{pgfscope}%
\definecolor{textcolor}{rgb}{0.000000,0.000000,0.000000}%
\pgfsetstrokecolor{textcolor}%
\pgfsetfillcolor{textcolor}%
\pgftext[x=4.899369in,y=0.448083in,,top]{\color{textcolor}\rmfamily\fontsize{8.000000}{9.600000}\selectfont \(\displaystyle {0}\)}%
\end{pgfscope}%
\begin{pgfscope}%
\pgfsetbuttcap%
\pgfsetroundjoin%
\definecolor{currentfill}{rgb}{0.000000,0.000000,0.000000}%
\pgfsetfillcolor{currentfill}%
\pgfsetlinewidth{0.803000pt}%
\definecolor{currentstroke}{rgb}{0.000000,0.000000,0.000000}%
\pgfsetstrokecolor{currentstroke}%
\pgfsetdash{}{0pt}%
\pgfsys@defobject{currentmarker}{\pgfqpoint{0.000000in}{-0.048611in}}{\pgfqpoint{0.000000in}{0.000000in}}{%
\pgfpathmoveto{\pgfqpoint{0.000000in}{0.000000in}}%
\pgfpathlineto{\pgfqpoint{0.000000in}{-0.048611in}}%
\pgfusepath{stroke,fill}%
}%
\begin{pgfscope}%
\pgfsys@transformshift{5.205070in}{0.545305in}%
\pgfsys@useobject{currentmarker}{}%
\end{pgfscope}%
\end{pgfscope}%
\begin{pgfscope}%
\definecolor{textcolor}{rgb}{0.000000,0.000000,0.000000}%
\pgfsetstrokecolor{textcolor}%
\pgfsetfillcolor{textcolor}%
\pgftext[x=5.205070in,y=0.448083in,,top]{\color{textcolor}\rmfamily\fontsize{8.000000}{9.600000}\selectfont \(\displaystyle {40}\)}%
\end{pgfscope}%
\begin{pgfscope}%
\pgfsetbuttcap%
\pgfsetroundjoin%
\definecolor{currentfill}{rgb}{0.000000,0.000000,0.000000}%
\pgfsetfillcolor{currentfill}%
\pgfsetlinewidth{0.803000pt}%
\definecolor{currentstroke}{rgb}{0.000000,0.000000,0.000000}%
\pgfsetstrokecolor{currentstroke}%
\pgfsetdash{}{0pt}%
\pgfsys@defobject{currentmarker}{\pgfqpoint{0.000000in}{-0.048611in}}{\pgfqpoint{0.000000in}{0.000000in}}{%
\pgfpathmoveto{\pgfqpoint{0.000000in}{0.000000in}}%
\pgfpathlineto{\pgfqpoint{0.000000in}{-0.048611in}}%
\pgfusepath{stroke,fill}%
}%
\begin{pgfscope}%
\pgfsys@transformshift{5.510772in}{0.545305in}%
\pgfsys@useobject{currentmarker}{}%
\end{pgfscope}%
\end{pgfscope}%
\begin{pgfscope}%
\definecolor{textcolor}{rgb}{0.000000,0.000000,0.000000}%
\pgfsetstrokecolor{textcolor}%
\pgfsetfillcolor{textcolor}%
\pgftext[x=5.510772in,y=0.448083in,,top]{\color{textcolor}\rmfamily\fontsize{8.000000}{9.600000}\selectfont \(\displaystyle {80}\)}%
\end{pgfscope}%
\begin{pgfscope}%
\pgfsetbuttcap%
\pgfsetroundjoin%
\definecolor{currentfill}{rgb}{0.000000,0.000000,0.000000}%
\pgfsetfillcolor{currentfill}%
\pgfsetlinewidth{0.803000pt}%
\definecolor{currentstroke}{rgb}{0.000000,0.000000,0.000000}%
\pgfsetstrokecolor{currentstroke}%
\pgfsetdash{}{0pt}%
\pgfsys@defobject{currentmarker}{\pgfqpoint{0.000000in}{-0.048611in}}{\pgfqpoint{0.000000in}{0.000000in}}{%
\pgfpathmoveto{\pgfqpoint{0.000000in}{0.000000in}}%
\pgfpathlineto{\pgfqpoint{0.000000in}{-0.048611in}}%
\pgfusepath{stroke,fill}%
}%
\begin{pgfscope}%
\pgfsys@transformshift{5.816473in}{0.545305in}%
\pgfsys@useobject{currentmarker}{}%
\end{pgfscope}%
\end{pgfscope}%
\begin{pgfscope}%
\definecolor{textcolor}{rgb}{0.000000,0.000000,0.000000}%
\pgfsetstrokecolor{textcolor}%
\pgfsetfillcolor{textcolor}%
\pgftext[x=5.816473in,y=0.448083in,,top]{\color{textcolor}\rmfamily\fontsize{8.000000}{9.600000}\selectfont \(\displaystyle {120}\)}%
\end{pgfscope}%
\begin{pgfscope}%
\definecolor{textcolor}{rgb}{0.000000,0.000000,0.000000}%
\pgfsetstrokecolor{textcolor}%
\pgfsetfillcolor{textcolor}%
\pgftext[x=4.900667in,y=0.293861in,,top]{\color{textcolor}\rmfamily\fontsize{10.000000}{12.000000}\selectfont \(\displaystyle \theta_{\textup{truth}} - \theta_{\textup{prediction}} \, [\textup{deg}]\)}%
\end{pgfscope}%
\begin{pgfscope}%
\pgfsetbuttcap%
\pgfsetroundjoin%
\definecolor{currentfill}{rgb}{0.000000,0.000000,0.000000}%
\pgfsetfillcolor{currentfill}%
\pgfsetlinewidth{0.803000pt}%
\definecolor{currentstroke}{rgb}{0.000000,0.000000,0.000000}%
\pgfsetstrokecolor{currentstroke}%
\pgfsetdash{}{0pt}%
\pgfsys@defobject{currentmarker}{\pgfqpoint{-0.048611in}{0.000000in}}{\pgfqpoint{-0.000000in}{0.000000in}}{%
\pgfpathmoveto{\pgfqpoint{-0.000000in}{0.000000in}}%
\pgfpathlineto{\pgfqpoint{-0.048611in}{0.000000in}}%
\pgfusepath{stroke,fill}%
}%
\begin{pgfscope}%
\pgfsys@transformshift{3.750134in}{0.545305in}%
\pgfsys@useobject{currentmarker}{}%
\end{pgfscope}%
\end{pgfscope}%
\begin{pgfscope}%
\definecolor{textcolor}{rgb}{0.000000,0.000000,0.000000}%
\pgfsetstrokecolor{textcolor}%
\pgfsetfillcolor{textcolor}%
\pgftext[x=3.443032in, y=0.506750in, left, base]{\color{textcolor}\rmfamily\fontsize{8.000000}{9.600000}\selectfont \(\displaystyle {0.00}\)}%
\end{pgfscope}%
\begin{pgfscope}%
\pgfsetbuttcap%
\pgfsetroundjoin%
\definecolor{currentfill}{rgb}{0.000000,0.000000,0.000000}%
\pgfsetfillcolor{currentfill}%
\pgfsetlinewidth{0.803000pt}%
\definecolor{currentstroke}{rgb}{0.000000,0.000000,0.000000}%
\pgfsetstrokecolor{currentstroke}%
\pgfsetdash{}{0pt}%
\pgfsys@defobject{currentmarker}{\pgfqpoint{-0.048611in}{0.000000in}}{\pgfqpoint{-0.000000in}{0.000000in}}{%
\pgfpathmoveto{\pgfqpoint{-0.000000in}{0.000000in}}%
\pgfpathlineto{\pgfqpoint{-0.048611in}{0.000000in}}%
\pgfusepath{stroke,fill}%
}%
\begin{pgfscope}%
\pgfsys@transformshift{3.750134in}{0.704675in}%
\pgfsys@useobject{currentmarker}{}%
\end{pgfscope}%
\end{pgfscope}%
\begin{pgfscope}%
\definecolor{textcolor}{rgb}{0.000000,0.000000,0.000000}%
\pgfsetstrokecolor{textcolor}%
\pgfsetfillcolor{textcolor}%
\pgftext[x=3.443032in, y=0.666119in, left, base]{\color{textcolor}\rmfamily\fontsize{8.000000}{9.600000}\selectfont \(\displaystyle {0.02}\)}%
\end{pgfscope}%
\begin{pgfscope}%
\pgfsetbuttcap%
\pgfsetroundjoin%
\definecolor{currentfill}{rgb}{0.000000,0.000000,0.000000}%
\pgfsetfillcolor{currentfill}%
\pgfsetlinewidth{0.803000pt}%
\definecolor{currentstroke}{rgb}{0.000000,0.000000,0.000000}%
\pgfsetstrokecolor{currentstroke}%
\pgfsetdash{}{0pt}%
\pgfsys@defobject{currentmarker}{\pgfqpoint{-0.048611in}{0.000000in}}{\pgfqpoint{-0.000000in}{0.000000in}}{%
\pgfpathmoveto{\pgfqpoint{-0.000000in}{0.000000in}}%
\pgfpathlineto{\pgfqpoint{-0.048611in}{0.000000in}}%
\pgfusepath{stroke,fill}%
}%
\begin{pgfscope}%
\pgfsys@transformshift{3.750134in}{0.864044in}%
\pgfsys@useobject{currentmarker}{}%
\end{pgfscope}%
\end{pgfscope}%
\begin{pgfscope}%
\definecolor{textcolor}{rgb}{0.000000,0.000000,0.000000}%
\pgfsetstrokecolor{textcolor}%
\pgfsetfillcolor{textcolor}%
\pgftext[x=3.443032in, y=0.825488in, left, base]{\color{textcolor}\rmfamily\fontsize{8.000000}{9.600000}\selectfont \(\displaystyle {0.04}\)}%
\end{pgfscope}%
\begin{pgfscope}%
\pgfsetbuttcap%
\pgfsetroundjoin%
\definecolor{currentfill}{rgb}{0.000000,0.000000,0.000000}%
\pgfsetfillcolor{currentfill}%
\pgfsetlinewidth{0.803000pt}%
\definecolor{currentstroke}{rgb}{0.000000,0.000000,0.000000}%
\pgfsetstrokecolor{currentstroke}%
\pgfsetdash{}{0pt}%
\pgfsys@defobject{currentmarker}{\pgfqpoint{-0.048611in}{0.000000in}}{\pgfqpoint{-0.000000in}{0.000000in}}{%
\pgfpathmoveto{\pgfqpoint{-0.000000in}{0.000000in}}%
\pgfpathlineto{\pgfqpoint{-0.048611in}{0.000000in}}%
\pgfusepath{stroke,fill}%
}%
\begin{pgfscope}%
\pgfsys@transformshift{3.750134in}{1.023413in}%
\pgfsys@useobject{currentmarker}{}%
\end{pgfscope}%
\end{pgfscope}%
\begin{pgfscope}%
\definecolor{textcolor}{rgb}{0.000000,0.000000,0.000000}%
\pgfsetstrokecolor{textcolor}%
\pgfsetfillcolor{textcolor}%
\pgftext[x=3.443032in, y=0.984857in, left, base]{\color{textcolor}\rmfamily\fontsize{8.000000}{9.600000}\selectfont \(\displaystyle {0.06}\)}%
\end{pgfscope}%
\begin{pgfscope}%
\pgfsetbuttcap%
\pgfsetroundjoin%
\definecolor{currentfill}{rgb}{0.000000,0.000000,0.000000}%
\pgfsetfillcolor{currentfill}%
\pgfsetlinewidth{0.803000pt}%
\definecolor{currentstroke}{rgb}{0.000000,0.000000,0.000000}%
\pgfsetstrokecolor{currentstroke}%
\pgfsetdash{}{0pt}%
\pgfsys@defobject{currentmarker}{\pgfqpoint{-0.048611in}{0.000000in}}{\pgfqpoint{-0.000000in}{0.000000in}}{%
\pgfpathmoveto{\pgfqpoint{-0.000000in}{0.000000in}}%
\pgfpathlineto{\pgfqpoint{-0.048611in}{0.000000in}}%
\pgfusepath{stroke,fill}%
}%
\begin{pgfscope}%
\pgfsys@transformshift{3.750134in}{1.182782in}%
\pgfsys@useobject{currentmarker}{}%
\end{pgfscope}%
\end{pgfscope}%
\begin{pgfscope}%
\definecolor{textcolor}{rgb}{0.000000,0.000000,0.000000}%
\pgfsetstrokecolor{textcolor}%
\pgfsetfillcolor{textcolor}%
\pgftext[x=3.443032in, y=1.144227in, left, base]{\color{textcolor}\rmfamily\fontsize{8.000000}{9.600000}\selectfont \(\displaystyle {0.08}\)}%
\end{pgfscope}%
\begin{pgfscope}%
\pgfsetbuttcap%
\pgfsetroundjoin%
\definecolor{currentfill}{rgb}{0.000000,0.000000,0.000000}%
\pgfsetfillcolor{currentfill}%
\pgfsetlinewidth{0.803000pt}%
\definecolor{currentstroke}{rgb}{0.000000,0.000000,0.000000}%
\pgfsetstrokecolor{currentstroke}%
\pgfsetdash{}{0pt}%
\pgfsys@defobject{currentmarker}{\pgfqpoint{-0.048611in}{0.000000in}}{\pgfqpoint{-0.000000in}{0.000000in}}{%
\pgfpathmoveto{\pgfqpoint{-0.000000in}{0.000000in}}%
\pgfpathlineto{\pgfqpoint{-0.048611in}{0.000000in}}%
\pgfusepath{stroke,fill}%
}%
\begin{pgfscope}%
\pgfsys@transformshift{3.750134in}{1.342151in}%
\pgfsys@useobject{currentmarker}{}%
\end{pgfscope}%
\end{pgfscope}%
\begin{pgfscope}%
\definecolor{textcolor}{rgb}{0.000000,0.000000,0.000000}%
\pgfsetstrokecolor{textcolor}%
\pgfsetfillcolor{textcolor}%
\pgftext[x=3.443032in, y=1.303596in, left, base]{\color{textcolor}\rmfamily\fontsize{8.000000}{9.600000}\selectfont \(\displaystyle {0.10}\)}%
\end{pgfscope}%
\begin{pgfscope}%
\pgfsetbuttcap%
\pgfsetroundjoin%
\definecolor{currentfill}{rgb}{0.000000,0.000000,0.000000}%
\pgfsetfillcolor{currentfill}%
\pgfsetlinewidth{0.803000pt}%
\definecolor{currentstroke}{rgb}{0.000000,0.000000,0.000000}%
\pgfsetstrokecolor{currentstroke}%
\pgfsetdash{}{0pt}%
\pgfsys@defobject{currentmarker}{\pgfqpoint{-0.048611in}{0.000000in}}{\pgfqpoint{-0.000000in}{0.000000in}}{%
\pgfpathmoveto{\pgfqpoint{-0.000000in}{0.000000in}}%
\pgfpathlineto{\pgfqpoint{-0.048611in}{0.000000in}}%
\pgfusepath{stroke,fill}%
}%
\begin{pgfscope}%
\pgfsys@transformshift{3.750134in}{1.501520in}%
\pgfsys@useobject{currentmarker}{}%
\end{pgfscope}%
\end{pgfscope}%
\begin{pgfscope}%
\definecolor{textcolor}{rgb}{0.000000,0.000000,0.000000}%
\pgfsetstrokecolor{textcolor}%
\pgfsetfillcolor{textcolor}%
\pgftext[x=3.443032in, y=1.462965in, left, base]{\color{textcolor}\rmfamily\fontsize{8.000000}{9.600000}\selectfont \(\displaystyle {0.12}\)}%
\end{pgfscope}%
\begin{pgfscope}%
\pgfsetbuttcap%
\pgfsetroundjoin%
\definecolor{currentfill}{rgb}{0.000000,0.000000,0.000000}%
\pgfsetfillcolor{currentfill}%
\pgfsetlinewidth{0.803000pt}%
\definecolor{currentstroke}{rgb}{0.000000,0.000000,0.000000}%
\pgfsetstrokecolor{currentstroke}%
\pgfsetdash{}{0pt}%
\pgfsys@defobject{currentmarker}{\pgfqpoint{-0.048611in}{0.000000in}}{\pgfqpoint{-0.000000in}{0.000000in}}{%
\pgfpathmoveto{\pgfqpoint{-0.000000in}{0.000000in}}%
\pgfpathlineto{\pgfqpoint{-0.048611in}{0.000000in}}%
\pgfusepath{stroke,fill}%
}%
\begin{pgfscope}%
\pgfsys@transformshift{3.750134in}{1.660890in}%
\pgfsys@useobject{currentmarker}{}%
\end{pgfscope}%
\end{pgfscope}%
\begin{pgfscope}%
\definecolor{textcolor}{rgb}{0.000000,0.000000,0.000000}%
\pgfsetstrokecolor{textcolor}%
\pgfsetfillcolor{textcolor}%
\pgftext[x=3.443032in, y=1.622334in, left, base]{\color{textcolor}\rmfamily\fontsize{8.000000}{9.600000}\selectfont \(\displaystyle {0.14}\)}%
\end{pgfscope}%
\begin{pgfscope}%
\pgfsetbuttcap%
\pgfsetroundjoin%
\definecolor{currentfill}{rgb}{0.000000,0.000000,0.000000}%
\pgfsetfillcolor{currentfill}%
\pgfsetlinewidth{0.803000pt}%
\definecolor{currentstroke}{rgb}{0.000000,0.000000,0.000000}%
\pgfsetstrokecolor{currentstroke}%
\pgfsetdash{}{0pt}%
\pgfsys@defobject{currentmarker}{\pgfqpoint{-0.048611in}{0.000000in}}{\pgfqpoint{-0.000000in}{0.000000in}}{%
\pgfpathmoveto{\pgfqpoint{-0.000000in}{0.000000in}}%
\pgfpathlineto{\pgfqpoint{-0.048611in}{0.000000in}}%
\pgfusepath{stroke,fill}%
}%
\begin{pgfscope}%
\pgfsys@transformshift{3.750134in}{1.820259in}%
\pgfsys@useobject{currentmarker}{}%
\end{pgfscope}%
\end{pgfscope}%
\begin{pgfscope}%
\definecolor{textcolor}{rgb}{0.000000,0.000000,0.000000}%
\pgfsetstrokecolor{textcolor}%
\pgfsetfillcolor{textcolor}%
\pgftext[x=3.443032in, y=1.781703in, left, base]{\color{textcolor}\rmfamily\fontsize{8.000000}{9.600000}\selectfont \(\displaystyle {0.16}\)}%
\end{pgfscope}%
\begin{pgfscope}%
\definecolor{textcolor}{rgb}{0.000000,0.000000,0.000000}%
\pgfsetstrokecolor{textcolor}%
\pgfsetfillcolor{textcolor}%
\pgftext[x=3.387476in,y=1.254042in,,bottom,rotate=90.000000]{\color{textcolor}\rmfamily\fontsize{10.000000}{12.000000}\selectfont Density}%
\end{pgfscope}%
\begin{pgfscope}%
\pgfpathrectangle{\pgfqpoint{3.750134in}{0.545305in}}{\pgfqpoint{2.301066in}{1.417472in}}%
\pgfusepath{clip}%
\pgfsetbuttcap%
\pgfsetmiterjoin%
\pgfsetlinewidth{1.003750pt}%
\definecolor{currentstroke}{rgb}{0.313725,0.317647,0.309804}%
\pgfsetstrokecolor{currentstroke}%
\pgfsetdash{}{0pt}%
\pgfpathmoveto{\pgfqpoint{4.139826in}{0.545305in}}%
\pgfpathlineto{\pgfqpoint{4.139826in}{0.547330in}}%
\pgfpathlineto{\pgfqpoint{4.151433in}{0.547330in}}%
\pgfpathlineto{\pgfqpoint{4.151433in}{0.545305in}}%
\pgfpathlineto{\pgfqpoint{4.244296in}{0.545305in}}%
\pgfpathlineto{\pgfqpoint{4.244296in}{0.547330in}}%
\pgfpathlineto{\pgfqpoint{4.279119in}{0.548343in}}%
\pgfpathlineto{\pgfqpoint{4.279119in}{0.545305in}}%
\pgfpathlineto{\pgfqpoint{4.290727in}{0.545305in}}%
\pgfpathlineto{\pgfqpoint{4.290727in}{0.549355in}}%
\pgfpathlineto{\pgfqpoint{4.325551in}{0.549355in}}%
\pgfpathlineto{\pgfqpoint{4.325551in}{0.551380in}}%
\pgfpathlineto{\pgfqpoint{4.337158in}{0.551380in}}%
\pgfpathlineto{\pgfqpoint{4.337158in}{0.549355in}}%
\pgfpathlineto{\pgfqpoint{4.348766in}{0.549355in}}%
\pgfpathlineto{\pgfqpoint{4.348766in}{0.554417in}}%
\pgfpathlineto{\pgfqpoint{4.360374in}{0.554417in}}%
\pgfpathlineto{\pgfqpoint{4.360374in}{0.549355in}}%
\pgfpathlineto{\pgfqpoint{4.406805in}{0.550368in}}%
\pgfpathlineto{\pgfqpoint{4.406805in}{0.554417in}}%
\pgfpathlineto{\pgfqpoint{4.418413in}{0.554417in}}%
\pgfpathlineto{\pgfqpoint{4.418413in}{0.549355in}}%
\pgfpathlineto{\pgfqpoint{4.430021in}{0.549355in}}%
\pgfpathlineto{\pgfqpoint{4.430021in}{0.553405in}}%
\pgfpathlineto{\pgfqpoint{4.441629in}{0.553405in}}%
\pgfpathlineto{\pgfqpoint{4.441629in}{0.551380in}}%
\pgfpathlineto{\pgfqpoint{4.453237in}{0.551380in}}%
\pgfpathlineto{\pgfqpoint{4.453237in}{0.555430in}}%
\pgfpathlineto{\pgfqpoint{4.464844in}{0.555430in}}%
\pgfpathlineto{\pgfqpoint{4.464844in}{0.552392in}}%
\pgfpathlineto{\pgfqpoint{4.499668in}{0.553405in}}%
\pgfpathlineto{\pgfqpoint{4.499668in}{0.556442in}}%
\pgfpathlineto{\pgfqpoint{4.522884in}{0.556442in}}%
\pgfpathlineto{\pgfqpoint{4.522884in}{0.559479in}}%
\pgfpathlineto{\pgfqpoint{4.534491in}{0.559479in}}%
\pgfpathlineto{\pgfqpoint{4.534491in}{0.565554in}}%
\pgfpathlineto{\pgfqpoint{4.557707in}{0.564542in}}%
\pgfpathlineto{\pgfqpoint{4.557707in}{0.562517in}}%
\pgfpathlineto{\pgfqpoint{4.580923in}{0.562517in}}%
\pgfpathlineto{\pgfqpoint{4.580923in}{0.556442in}}%
\pgfpathlineto{\pgfqpoint{4.592530in}{0.556442in}}%
\pgfpathlineto{\pgfqpoint{4.592530in}{0.568591in}}%
\pgfpathlineto{\pgfqpoint{4.604138in}{0.568591in}}%
\pgfpathlineto{\pgfqpoint{4.604138in}{0.572641in}}%
\pgfpathlineto{\pgfqpoint{4.615746in}{0.572641in}}%
\pgfpathlineto{\pgfqpoint{4.615746in}{0.566566in}}%
\pgfpathlineto{\pgfqpoint{4.627354in}{0.566566in}}%
\pgfpathlineto{\pgfqpoint{4.627354in}{0.580740in}}%
\pgfpathlineto{\pgfqpoint{4.638962in}{0.580740in}}%
\pgfpathlineto{\pgfqpoint{4.638962in}{0.574666in}}%
\pgfpathlineto{\pgfqpoint{4.650570in}{0.574666in}}%
\pgfpathlineto{\pgfqpoint{4.650570in}{0.584790in}}%
\pgfpathlineto{\pgfqpoint{4.662177in}{0.584790in}}%
\pgfpathlineto{\pgfqpoint{4.662177in}{0.575678in}}%
\pgfpathlineto{\pgfqpoint{4.673785in}{0.575678in}}%
\pgfpathlineto{\pgfqpoint{4.673785in}{0.585802in}}%
\pgfpathlineto{\pgfqpoint{4.708609in}{0.585802in}}%
\pgfpathlineto{\pgfqpoint{4.708609in}{0.588840in}}%
\pgfpathlineto{\pgfqpoint{4.731824in}{0.587827in}}%
\pgfpathlineto{\pgfqpoint{4.731824in}{0.608076in}}%
\pgfpathlineto{\pgfqpoint{4.743432in}{0.608076in}}%
\pgfpathlineto{\pgfqpoint{4.743432in}{0.590865in}}%
\pgfpathlineto{\pgfqpoint{4.755040in}{0.590865in}}%
\pgfpathlineto{\pgfqpoint{4.755040in}{0.606051in}}%
\pgfpathlineto{\pgfqpoint{4.766648in}{0.606051in}}%
\pgfpathlineto{\pgfqpoint{4.766648in}{0.609088in}}%
\pgfpathlineto{\pgfqpoint{4.778256in}{0.609088in}}%
\pgfpathlineto{\pgfqpoint{4.778256in}{0.604026in}}%
\pgfpathlineto{\pgfqpoint{4.789863in}{0.604026in}}%
\pgfpathlineto{\pgfqpoint{4.789863in}{0.619213in}}%
\pgfpathlineto{\pgfqpoint{4.801471in}{0.619213in}}%
\pgfpathlineto{\pgfqpoint{4.801471in}{0.638449in}}%
\pgfpathlineto{\pgfqpoint{4.813079in}{0.638449in}}%
\pgfpathlineto{\pgfqpoint{4.813079in}{0.646548in}}%
\pgfpathlineto{\pgfqpoint{4.824687in}{0.646548in}}%
\pgfpathlineto{\pgfqpoint{4.824687in}{0.679958in}}%
\pgfpathlineto{\pgfqpoint{4.836295in}{0.679958in}}%
\pgfpathlineto{\pgfqpoint{4.836295in}{0.699194in}}%
\pgfpathlineto{\pgfqpoint{4.847903in}{0.699194in}}%
\pgfpathlineto{\pgfqpoint{4.847903in}{0.731592in}}%
\pgfpathlineto{\pgfqpoint{4.859510in}{0.731592in}}%
\pgfpathlineto{\pgfqpoint{4.859510in}{0.781201in}}%
\pgfpathlineto{\pgfqpoint{4.871118in}{0.781201in}}%
\pgfpathlineto{\pgfqpoint{4.871118in}{0.875357in}}%
\pgfpathlineto{\pgfqpoint{4.882726in}{0.875357in}}%
\pgfpathlineto{\pgfqpoint{4.882726in}{0.931040in}}%
\pgfpathlineto{\pgfqpoint{4.894334in}{0.931040in}}%
\pgfpathlineto{\pgfqpoint{4.894334in}{0.960400in}}%
\pgfpathlineto{\pgfqpoint{4.917549in}{0.961413in}}%
\pgfpathlineto{\pgfqpoint{4.917549in}{0.898642in}}%
\pgfpathlineto{\pgfqpoint{4.929157in}{0.898642in}}%
\pgfpathlineto{\pgfqpoint{4.929157in}{0.824735in}}%
\pgfpathlineto{\pgfqpoint{4.940765in}{0.824735in}}%
\pgfpathlineto{\pgfqpoint{4.940765in}{0.763990in}}%
\pgfpathlineto{\pgfqpoint{4.952373in}{0.763990in}}%
\pgfpathlineto{\pgfqpoint{4.952373in}{0.703244in}}%
\pgfpathlineto{\pgfqpoint{4.963981in}{0.703244in}}%
\pgfpathlineto{\pgfqpoint{4.963981in}{0.658697in}}%
\pgfpathlineto{\pgfqpoint{4.975589in}{0.658697in}}%
\pgfpathlineto{\pgfqpoint{4.975589in}{0.638449in}}%
\pgfpathlineto{\pgfqpoint{4.987196in}{0.638449in}}%
\pgfpathlineto{\pgfqpoint{4.987196in}{0.620225in}}%
\pgfpathlineto{\pgfqpoint{4.998804in}{0.620225in}}%
\pgfpathlineto{\pgfqpoint{4.998804in}{0.617188in}}%
\pgfpathlineto{\pgfqpoint{5.010412in}{0.617188in}}%
\pgfpathlineto{\pgfqpoint{5.010412in}{0.599976in}}%
\pgfpathlineto{\pgfqpoint{5.022020in}{0.599976in}}%
\pgfpathlineto{\pgfqpoint{5.022020in}{0.588840in}}%
\pgfpathlineto{\pgfqpoint{5.033628in}{0.588840in}}%
\pgfpathlineto{\pgfqpoint{5.033628in}{0.577703in}}%
\pgfpathlineto{\pgfqpoint{5.056843in}{0.577703in}}%
\pgfpathlineto{\pgfqpoint{5.056843in}{0.568591in}}%
\pgfpathlineto{\pgfqpoint{5.068451in}{0.568591in}}%
\pgfpathlineto{\pgfqpoint{5.068451in}{0.575678in}}%
\pgfpathlineto{\pgfqpoint{5.080059in}{0.575678in}}%
\pgfpathlineto{\pgfqpoint{5.080059in}{0.569604in}}%
\pgfpathlineto{\pgfqpoint{5.091667in}{0.569604in}}%
\pgfpathlineto{\pgfqpoint{5.091667in}{0.559479in}}%
\pgfpathlineto{\pgfqpoint{5.103275in}{0.559479in}}%
\pgfpathlineto{\pgfqpoint{5.103275in}{0.566566in}}%
\pgfpathlineto{\pgfqpoint{5.114882in}{0.566566in}}%
\pgfpathlineto{\pgfqpoint{5.114882in}{0.552392in}}%
\pgfpathlineto{\pgfqpoint{5.126490in}{0.552392in}}%
\pgfpathlineto{\pgfqpoint{5.126490in}{0.564542in}}%
\pgfpathlineto{\pgfqpoint{5.138098in}{0.564542in}}%
\pgfpathlineto{\pgfqpoint{5.138098in}{0.551380in}}%
\pgfpathlineto{\pgfqpoint{5.149706in}{0.551380in}}%
\pgfpathlineto{\pgfqpoint{5.149706in}{0.554417in}}%
\pgfpathlineto{\pgfqpoint{5.161314in}{0.554417in}}%
\pgfpathlineto{\pgfqpoint{5.161314in}{0.551380in}}%
\pgfpathlineto{\pgfqpoint{5.196137in}{0.552392in}}%
\pgfpathlineto{\pgfqpoint{5.196137in}{0.553405in}}%
\pgfpathlineto{\pgfqpoint{5.207745in}{0.553405in}}%
\pgfpathlineto{\pgfqpoint{5.207745in}{0.546318in}}%
\pgfpathlineto{\pgfqpoint{5.219353in}{0.546318in}}%
\pgfpathlineto{\pgfqpoint{5.219353in}{0.549355in}}%
\pgfpathlineto{\pgfqpoint{5.242568in}{0.550368in}}%
\pgfpathlineto{\pgfqpoint{5.242568in}{0.547330in}}%
\pgfpathlineto{\pgfqpoint{5.277392in}{0.547330in}}%
\pgfpathlineto{\pgfqpoint{5.277392in}{0.551380in}}%
\pgfpathlineto{\pgfqpoint{5.289000in}{0.551380in}}%
\pgfpathlineto{\pgfqpoint{5.289000in}{0.549355in}}%
\pgfpathlineto{\pgfqpoint{5.300607in}{0.549355in}}%
\pgfpathlineto{\pgfqpoint{5.300607in}{0.546318in}}%
\pgfpathlineto{\pgfqpoint{5.312215in}{0.546318in}}%
\pgfpathlineto{\pgfqpoint{5.312215in}{0.548343in}}%
\pgfpathlineto{\pgfqpoint{5.323823in}{0.548343in}}%
\pgfpathlineto{\pgfqpoint{5.323823in}{0.546318in}}%
\pgfpathlineto{\pgfqpoint{5.347039in}{0.546318in}}%
\pgfpathlineto{\pgfqpoint{5.347039in}{0.551380in}}%
\pgfpathlineto{\pgfqpoint{5.358647in}{0.551380in}}%
\pgfpathlineto{\pgfqpoint{5.358647in}{0.546318in}}%
\pgfpathlineto{\pgfqpoint{5.370254in}{0.546318in}}%
\pgfpathlineto{\pgfqpoint{5.370254in}{0.548343in}}%
\pgfpathlineto{\pgfqpoint{5.393470in}{0.547330in}}%
\pgfpathlineto{\pgfqpoint{5.393470in}{0.546318in}}%
\pgfpathlineto{\pgfqpoint{5.439901in}{0.546318in}}%
\pgfpathlineto{\pgfqpoint{5.439901in}{0.548343in}}%
\pgfpathlineto{\pgfqpoint{5.451509in}{0.548343in}}%
\pgfpathlineto{\pgfqpoint{5.451509in}{0.546318in}}%
\pgfpathlineto{\pgfqpoint{5.764920in}{0.546318in}}%
\pgfpathlineto{\pgfqpoint{5.764920in}{0.545305in}}%
\pgfusepath{stroke}%
\end{pgfscope}%
\begin{pgfscope}%
\pgfpathrectangle{\pgfqpoint{3.750134in}{0.545305in}}{\pgfqpoint{2.301066in}{1.417472in}}%
\pgfusepath{clip}%
\pgfsetbuttcap%
\pgfsetmiterjoin%
\pgfsetlinewidth{1.003750pt}%
\definecolor{currentstroke}{rgb}{0.949020,0.372549,0.360784}%
\pgfsetstrokecolor{currentstroke}%
\pgfsetdash{{1.000000pt}{1.650000pt}}{0.000000pt}%
\pgfpathmoveto{\pgfqpoint{4.139826in}{0.545305in}}%
\pgfpathlineto{\pgfqpoint{4.139826in}{0.546322in}}%
\pgfpathlineto{\pgfqpoint{4.163041in}{0.545305in}}%
\pgfpathlineto{\pgfqpoint{4.163041in}{0.548355in}}%
\pgfpathlineto{\pgfqpoint{4.174649in}{0.548355in}}%
\pgfpathlineto{\pgfqpoint{4.174649in}{0.546322in}}%
\pgfpathlineto{\pgfqpoint{4.209472in}{0.546322in}}%
\pgfpathlineto{\pgfqpoint{4.209472in}{0.548355in}}%
\pgfpathlineto{\pgfqpoint{4.221080in}{0.548355in}}%
\pgfpathlineto{\pgfqpoint{4.221080in}{0.546322in}}%
\pgfpathlineto{\pgfqpoint{4.232688in}{0.546322in}}%
\pgfpathlineto{\pgfqpoint{4.232688in}{0.548355in}}%
\pgfpathlineto{\pgfqpoint{4.290727in}{0.549372in}}%
\pgfpathlineto{\pgfqpoint{4.290727in}{0.546322in}}%
\pgfpathlineto{\pgfqpoint{4.302335in}{0.546322in}}%
\pgfpathlineto{\pgfqpoint{4.302335in}{0.548355in}}%
\pgfpathlineto{\pgfqpoint{4.360374in}{0.548355in}}%
\pgfpathlineto{\pgfqpoint{4.360374in}{0.546322in}}%
\pgfpathlineto{\pgfqpoint{4.383590in}{0.547339in}}%
\pgfpathlineto{\pgfqpoint{4.383590in}{0.552421in}}%
\pgfpathlineto{\pgfqpoint{4.406805in}{0.551405in}}%
\pgfpathlineto{\pgfqpoint{4.406805in}{0.550388in}}%
\pgfpathlineto{\pgfqpoint{4.418413in}{0.550388in}}%
\pgfpathlineto{\pgfqpoint{4.418413in}{0.554454in}}%
\pgfpathlineto{\pgfqpoint{4.430021in}{0.554454in}}%
\pgfpathlineto{\pgfqpoint{4.430021in}{0.548355in}}%
\pgfpathlineto{\pgfqpoint{4.441629in}{0.548355in}}%
\pgfpathlineto{\pgfqpoint{4.441629in}{0.555471in}}%
\pgfpathlineto{\pgfqpoint{4.464844in}{0.555471in}}%
\pgfpathlineto{\pgfqpoint{4.464844in}{0.553438in}}%
\pgfpathlineto{\pgfqpoint{4.476452in}{0.553438in}}%
\pgfpathlineto{\pgfqpoint{4.476452in}{0.549372in}}%
\pgfpathlineto{\pgfqpoint{4.488060in}{0.549372in}}%
\pgfpathlineto{\pgfqpoint{4.488060in}{0.557504in}}%
\pgfpathlineto{\pgfqpoint{4.499668in}{0.557504in}}%
\pgfpathlineto{\pgfqpoint{4.499668in}{0.555471in}}%
\pgfpathlineto{\pgfqpoint{4.534491in}{0.556487in}}%
\pgfpathlineto{\pgfqpoint{4.534491in}{0.552421in}}%
\pgfpathlineto{\pgfqpoint{4.546099in}{0.552421in}}%
\pgfpathlineto{\pgfqpoint{4.546099in}{0.556487in}}%
\pgfpathlineto{\pgfqpoint{4.569315in}{0.556487in}}%
\pgfpathlineto{\pgfqpoint{4.569315in}{0.554454in}}%
\pgfpathlineto{\pgfqpoint{4.580923in}{0.554454in}}%
\pgfpathlineto{\pgfqpoint{4.580923in}{0.558521in}}%
\pgfpathlineto{\pgfqpoint{4.592530in}{0.558521in}}%
\pgfpathlineto{\pgfqpoint{4.592530in}{0.555471in}}%
\pgfpathlineto{\pgfqpoint{4.604138in}{0.555471in}}%
\pgfpathlineto{\pgfqpoint{4.604138in}{0.563603in}}%
\pgfpathlineto{\pgfqpoint{4.615746in}{0.563603in}}%
\pgfpathlineto{\pgfqpoint{4.615746in}{0.558521in}}%
\pgfpathlineto{\pgfqpoint{4.627354in}{0.558521in}}%
\pgfpathlineto{\pgfqpoint{4.627354in}{0.561570in}}%
\pgfpathlineto{\pgfqpoint{4.662177in}{0.561570in}}%
\pgfpathlineto{\pgfqpoint{4.662177in}{0.565636in}}%
\pgfpathlineto{\pgfqpoint{4.673785in}{0.565636in}}%
\pgfpathlineto{\pgfqpoint{4.673785in}{0.562587in}}%
\pgfpathlineto{\pgfqpoint{4.685393in}{0.562587in}}%
\pgfpathlineto{\pgfqpoint{4.685393in}{0.559537in}}%
\pgfpathlineto{\pgfqpoint{4.697001in}{0.559537in}}%
\pgfpathlineto{\pgfqpoint{4.697001in}{0.570719in}}%
\pgfpathlineto{\pgfqpoint{4.708609in}{0.570719in}}%
\pgfpathlineto{\pgfqpoint{4.708609in}{0.563603in}}%
\pgfpathlineto{\pgfqpoint{4.720217in}{0.563603in}}%
\pgfpathlineto{\pgfqpoint{4.720217in}{0.579868in}}%
\pgfpathlineto{\pgfqpoint{4.731824in}{0.579868in}}%
\pgfpathlineto{\pgfqpoint{4.731824in}{0.571736in}}%
\pgfpathlineto{\pgfqpoint{4.743432in}{0.571736in}}%
\pgfpathlineto{\pgfqpoint{4.743432in}{0.567669in}}%
\pgfpathlineto{\pgfqpoint{4.755040in}{0.567669in}}%
\pgfpathlineto{\pgfqpoint{4.755040in}{0.577835in}}%
\pgfpathlineto{\pgfqpoint{4.789863in}{0.577835in}}%
\pgfpathlineto{\pgfqpoint{4.789863in}{0.593083in}}%
\pgfpathlineto{\pgfqpoint{4.801471in}{0.593083in}}%
\pgfpathlineto{\pgfqpoint{4.801471in}{0.600199in}}%
\pgfpathlineto{\pgfqpoint{4.813079in}{0.600199in}}%
\pgfpathlineto{\pgfqpoint{4.813079in}{0.602232in}}%
\pgfpathlineto{\pgfqpoint{4.824687in}{0.602232in}}%
\pgfpathlineto{\pgfqpoint{4.824687in}{0.613414in}}%
\pgfpathlineto{\pgfqpoint{4.836295in}{0.613414in}}%
\pgfpathlineto{\pgfqpoint{4.836295in}{0.628662in}}%
\pgfpathlineto{\pgfqpoint{4.847903in}{0.628662in}}%
\pgfpathlineto{\pgfqpoint{4.847903in}{0.654076in}}%
\pgfpathlineto{\pgfqpoint{4.859510in}{0.654076in}}%
\pgfpathlineto{\pgfqpoint{4.859510in}{0.702870in}}%
\pgfpathlineto{\pgfqpoint{4.871118in}{0.702870in}}%
\pgfpathlineto{\pgfqpoint{4.871118in}{0.835021in}}%
\pgfpathlineto{\pgfqpoint{4.882726in}{0.835021in}}%
\pgfpathlineto{\pgfqpoint{4.882726in}{1.234524in}}%
\pgfpathlineto{\pgfqpoint{4.894334in}{1.234524in}}%
\pgfpathlineto{\pgfqpoint{4.894334in}{1.895279in}}%
\pgfpathlineto{\pgfqpoint{4.905942in}{1.895279in}}%
\pgfpathlineto{\pgfqpoint{4.905942in}{1.175564in}}%
\pgfpathlineto{\pgfqpoint{4.917549in}{1.175564in}}%
\pgfpathlineto{\pgfqpoint{4.917549in}{0.811641in}}%
\pgfpathlineto{\pgfqpoint{4.929157in}{0.811641in}}%
\pgfpathlineto{\pgfqpoint{4.929157in}{0.687622in}}%
\pgfpathlineto{\pgfqpoint{4.940765in}{0.687622in}}%
\pgfpathlineto{\pgfqpoint{4.940765in}{0.630695in}}%
\pgfpathlineto{\pgfqpoint{4.952373in}{0.630695in}}%
\pgfpathlineto{\pgfqpoint{4.952373in}{0.618497in}}%
\pgfpathlineto{\pgfqpoint{4.963981in}{0.618497in}}%
\pgfpathlineto{\pgfqpoint{4.963981in}{0.592067in}}%
\pgfpathlineto{\pgfqpoint{4.975589in}{0.592067in}}%
\pgfpathlineto{\pgfqpoint{4.975589in}{0.578851in}}%
\pgfpathlineto{\pgfqpoint{4.987196in}{0.578851in}}%
\pgfpathlineto{\pgfqpoint{4.987196in}{0.581901in}}%
\pgfpathlineto{\pgfqpoint{4.998804in}{0.581901in}}%
\pgfpathlineto{\pgfqpoint{4.998804in}{0.579868in}}%
\pgfpathlineto{\pgfqpoint{5.010412in}{0.579868in}}%
\pgfpathlineto{\pgfqpoint{5.010412in}{0.566653in}}%
\pgfpathlineto{\pgfqpoint{5.022020in}{0.566653in}}%
\pgfpathlineto{\pgfqpoint{5.022020in}{0.568686in}}%
\pgfpathlineto{\pgfqpoint{5.033628in}{0.568686in}}%
\pgfpathlineto{\pgfqpoint{5.033628in}{0.566653in}}%
\pgfpathlineto{\pgfqpoint{5.045235in}{0.566653in}}%
\pgfpathlineto{\pgfqpoint{5.045235in}{0.561570in}}%
\pgfpathlineto{\pgfqpoint{5.056843in}{0.561570in}}%
\pgfpathlineto{\pgfqpoint{5.056843in}{0.566653in}}%
\pgfpathlineto{\pgfqpoint{5.068451in}{0.566653in}}%
\pgfpathlineto{\pgfqpoint{5.068451in}{0.558521in}}%
\pgfpathlineto{\pgfqpoint{5.080059in}{0.558521in}}%
\pgfpathlineto{\pgfqpoint{5.080059in}{0.556487in}}%
\pgfpathlineto{\pgfqpoint{5.091667in}{0.556487in}}%
\pgfpathlineto{\pgfqpoint{5.091667in}{0.558521in}}%
\pgfpathlineto{\pgfqpoint{5.103275in}{0.558521in}}%
\pgfpathlineto{\pgfqpoint{5.103275in}{0.562587in}}%
\pgfpathlineto{\pgfqpoint{5.114882in}{0.562587in}}%
\pgfpathlineto{\pgfqpoint{5.114882in}{0.556487in}}%
\pgfpathlineto{\pgfqpoint{5.138098in}{0.556487in}}%
\pgfpathlineto{\pgfqpoint{5.138098in}{0.559537in}}%
\pgfpathlineto{\pgfqpoint{5.149706in}{0.559537in}}%
\pgfpathlineto{\pgfqpoint{5.149706in}{0.554454in}}%
\pgfpathlineto{\pgfqpoint{5.161314in}{0.554454in}}%
\pgfpathlineto{\pgfqpoint{5.161314in}{0.558521in}}%
\pgfpathlineto{\pgfqpoint{5.172921in}{0.558521in}}%
\pgfpathlineto{\pgfqpoint{5.172921in}{0.551405in}}%
\pgfpathlineto{\pgfqpoint{5.230961in}{0.551405in}}%
\pgfpathlineto{\pgfqpoint{5.230961in}{0.561570in}}%
\pgfpathlineto{\pgfqpoint{5.242568in}{0.561570in}}%
\pgfpathlineto{\pgfqpoint{5.242568in}{0.556487in}}%
\pgfpathlineto{\pgfqpoint{5.254176in}{0.556487in}}%
\pgfpathlineto{\pgfqpoint{5.254176in}{0.552421in}}%
\pgfpathlineto{\pgfqpoint{5.277392in}{0.552421in}}%
\pgfpathlineto{\pgfqpoint{5.277392in}{0.548355in}}%
\pgfpathlineto{\pgfqpoint{5.289000in}{0.548355in}}%
\pgfpathlineto{\pgfqpoint{5.289000in}{0.552421in}}%
\pgfpathlineto{\pgfqpoint{5.312215in}{0.552421in}}%
\pgfpathlineto{\pgfqpoint{5.312215in}{0.555471in}}%
\pgfpathlineto{\pgfqpoint{5.323823in}{0.555471in}}%
\pgfpathlineto{\pgfqpoint{5.323823in}{0.548355in}}%
\pgfpathlineto{\pgfqpoint{5.335431in}{0.548355in}}%
\pgfpathlineto{\pgfqpoint{5.335431in}{0.553438in}}%
\pgfpathlineto{\pgfqpoint{5.358647in}{0.552421in}}%
\pgfpathlineto{\pgfqpoint{5.358647in}{0.551405in}}%
\pgfpathlineto{\pgfqpoint{5.370254in}{0.551405in}}%
\pgfpathlineto{\pgfqpoint{5.370254in}{0.548355in}}%
\pgfpathlineto{\pgfqpoint{5.405078in}{0.548355in}}%
\pgfpathlineto{\pgfqpoint{5.405078in}{0.552421in}}%
\pgfpathlineto{\pgfqpoint{5.416686in}{0.552421in}}%
\pgfpathlineto{\pgfqpoint{5.416686in}{0.548355in}}%
\pgfpathlineto{\pgfqpoint{5.497940in}{0.548355in}}%
\pgfpathlineto{\pgfqpoint{5.497940in}{0.550388in}}%
\pgfpathlineto{\pgfqpoint{5.509548in}{0.550388in}}%
\pgfpathlineto{\pgfqpoint{5.509548in}{0.545305in}}%
\pgfpathlineto{\pgfqpoint{5.521156in}{0.545305in}}%
\pgfpathlineto{\pgfqpoint{5.521156in}{0.550388in}}%
\pgfpathlineto{\pgfqpoint{5.544372in}{0.549372in}}%
\pgfpathlineto{\pgfqpoint{5.544372in}{0.545305in}}%
\pgfpathlineto{\pgfqpoint{5.555980in}{0.545305in}}%
\pgfpathlineto{\pgfqpoint{5.555980in}{0.547339in}}%
\pgfpathlineto{\pgfqpoint{5.590803in}{0.548355in}}%
\pgfpathlineto{\pgfqpoint{5.590803in}{0.550388in}}%
\pgfpathlineto{\pgfqpoint{5.602411in}{0.550388in}}%
\pgfpathlineto{\pgfqpoint{5.602411in}{0.546322in}}%
\pgfpathlineto{\pgfqpoint{5.706881in}{0.545305in}}%
\pgfpathlineto{\pgfqpoint{5.706881in}{0.548355in}}%
\pgfpathlineto{\pgfqpoint{5.718489in}{0.548355in}}%
\pgfpathlineto{\pgfqpoint{5.718489in}{0.546322in}}%
\pgfpathlineto{\pgfqpoint{5.764920in}{0.546322in}}%
\pgfpathlineto{\pgfqpoint{5.764920in}{0.545305in}}%
\pgfusepath{stroke}%
\end{pgfscope}%
\begin{pgfscope}%
\pgfsetrectcap%
\pgfsetmiterjoin%
\pgfsetlinewidth{0.803000pt}%
\definecolor{currentstroke}{rgb}{0.000000,0.000000,0.000000}%
\pgfsetstrokecolor{currentstroke}%
\pgfsetdash{}{0pt}%
\pgfpathmoveto{\pgfqpoint{3.750134in}{0.545305in}}%
\pgfpathlineto{\pgfqpoint{3.750134in}{1.962778in}}%
\pgfusepath{stroke}%
\end{pgfscope}%
\begin{pgfscope}%
\pgfsetrectcap%
\pgfsetmiterjoin%
\pgfsetlinewidth{0.803000pt}%
\definecolor{currentstroke}{rgb}{0.000000,0.000000,0.000000}%
\pgfsetstrokecolor{currentstroke}%
\pgfsetdash{}{0pt}%
\pgfpathmoveto{\pgfqpoint{6.051200in}{0.545305in}}%
\pgfpathlineto{\pgfqpoint{6.051200in}{1.962778in}}%
\pgfusepath{stroke}%
\end{pgfscope}%
\begin{pgfscope}%
\pgfsetrectcap%
\pgfsetmiterjoin%
\pgfsetlinewidth{0.803000pt}%
\definecolor{currentstroke}{rgb}{0.000000,0.000000,0.000000}%
\pgfsetstrokecolor{currentstroke}%
\pgfsetdash{}{0pt}%
\pgfpathmoveto{\pgfqpoint{3.750134in}{0.545305in}}%
\pgfpathlineto{\pgfqpoint{6.051200in}{0.545305in}}%
\pgfusepath{stroke}%
\end{pgfscope}%
\begin{pgfscope}%
\pgfsetrectcap%
\pgfsetmiterjoin%
\pgfsetlinewidth{0.803000pt}%
\definecolor{currentstroke}{rgb}{0.000000,0.000000,0.000000}%
\pgfsetstrokecolor{currentstroke}%
\pgfsetdash{}{0pt}%
\pgfpathmoveto{\pgfqpoint{3.750134in}{1.962778in}}%
\pgfpathlineto{\pgfqpoint{6.051200in}{1.962778in}}%
\pgfusepath{stroke}%
\end{pgfscope}%
\begin{pgfscope}%
\definecolor{textcolor}{rgb}{0.000000,0.000000,0.000000}%
\pgfsetstrokecolor{textcolor}%
\pgfsetfillcolor{textcolor}%
\pgftext[x=3.750134in,y=2.046111in,left,base]{\color{textcolor}\rmfamily\fontsize{10.000000}{12.000000}\selectfont Bin [2.83, 3.0], 5,182 events}%
\end{pgfscope}%
\begin{pgfscope}%
\pgfsetbuttcap%
\pgfsetmiterjoin%
\definecolor{currentfill}{rgb}{1.000000,1.000000,1.000000}%
\pgfsetfillcolor{currentfill}%
\pgfsetfillopacity{0.800000}%
\pgfsetlinewidth{1.003750pt}%
\definecolor{currentstroke}{rgb}{0.800000,0.800000,0.800000}%
\pgfsetstrokecolor{currentstroke}%
\pgfsetstrokeopacity{0.800000}%
\pgfsetdash{}{0pt}%
\pgfpathmoveto{\pgfqpoint{5.016422in}{1.563222in}}%
\pgfpathlineto{\pgfqpoint{5.973422in}{1.563222in}}%
\pgfpathquadraticcurveto{\pgfqpoint{5.995644in}{1.563222in}}{\pgfqpoint{5.995644in}{1.585444in}}%
\pgfpathlineto{\pgfqpoint{5.995644in}{1.885000in}}%
\pgfpathquadraticcurveto{\pgfqpoint{5.995644in}{1.907222in}}{\pgfqpoint{5.973422in}{1.907222in}}%
\pgfpathlineto{\pgfqpoint{5.016422in}{1.907222in}}%
\pgfpathquadraticcurveto{\pgfqpoint{4.994200in}{1.907222in}}{\pgfqpoint{4.994200in}{1.885000in}}%
\pgfpathlineto{\pgfqpoint{4.994200in}{1.585444in}}%
\pgfpathquadraticcurveto{\pgfqpoint{4.994200in}{1.563222in}}{\pgfqpoint{5.016422in}{1.563222in}}%
\pgfpathclose%
\pgfusepath{stroke,fill}%
\end{pgfscope}%
\begin{pgfscope}%
\pgfsetbuttcap%
\pgfsetmiterjoin%
\pgfsetlinewidth{1.003750pt}%
\definecolor{currentstroke}{rgb}{0.313725,0.317647,0.309804}%
\pgfsetstrokecolor{currentstroke}%
\pgfsetdash{}{0pt}%
\pgfpathmoveto{\pgfqpoint{5.038644in}{1.784555in}}%
\pgfpathlineto{\pgfqpoint{5.260867in}{1.784555in}}%
\pgfpathlineto{\pgfqpoint{5.260867in}{1.862333in}}%
\pgfpathlineto{\pgfqpoint{5.038644in}{1.862333in}}%
\pgfpathclose%
\pgfusepath{stroke}%
\end{pgfscope}%
\begin{pgfscope}%
\definecolor{textcolor}{rgb}{0.000000,0.000000,0.000000}%
\pgfsetstrokecolor{textcolor}%
\pgfsetfillcolor{textcolor}%
\pgftext[x=5.349756in,y=1.784555in,left,base]{\color{textcolor}\rmfamily\fontsize{8.000000}{9.600000}\selectfont IQR = 9.79}%
\end{pgfscope}%
\begin{pgfscope}%
\pgfsetbuttcap%
\pgfsetmiterjoin%
\pgfsetlinewidth{1.003750pt}%
\definecolor{currentstroke}{rgb}{0.949020,0.372549,0.360784}%
\pgfsetstrokecolor{currentstroke}%
\pgfsetdash{{1.000000pt}{1.650000pt}}{0.000000pt}%
\pgfpathmoveto{\pgfqpoint{5.038644in}{1.629222in}}%
\pgfpathlineto{\pgfqpoint{5.260867in}{1.629222in}}%
\pgfpathlineto{\pgfqpoint{5.260867in}{1.707000in}}%
\pgfpathlineto{\pgfqpoint{5.038644in}{1.707000in}}%
\pgfpathclose%
\pgfusepath{stroke}%
\end{pgfscope}%
\begin{pgfscope}%
\definecolor{textcolor}{rgb}{0.000000,0.000000,0.000000}%
\pgfsetstrokecolor{textcolor}%
\pgfsetfillcolor{textcolor}%
\pgftext[x=5.349756in,y=1.629222in,left,base]{\color{textcolor}\rmfamily\fontsize{8.000000}{9.600000}\selectfont IQR = 3.32}%
\end{pgfscope}%
\begin{pgfscope}%
\definecolor{textcolor}{rgb}{0.000000,0.000000,0.000000}%
\pgfsetstrokecolor{textcolor}%
\pgfsetfillcolor{textcolor}%
\pgftext[x=0.620120in,y=6.370000in,left,top]{\color{textcolor}\rmfamily\fontsize{12.000000}{14.400000}\selectfont Zenith error distribution in selected bins}%
\end{pgfscope}%
\begin{pgfscope}%
\pgfsetbuttcap%
\pgfsetmiterjoin%
\definecolor{currentfill}{rgb}{1.000000,1.000000,1.000000}%
\pgfsetfillcolor{currentfill}%
\pgfsetfillopacity{0.800000}%
\pgfsetlinewidth{1.003750pt}%
\definecolor{currentstroke}{rgb}{0.800000,0.800000,0.800000}%
\pgfsetstrokecolor{currentstroke}%
\pgfsetstrokeopacity{0.800000}%
\pgfsetdash{}{0pt}%
\pgfpathmoveto{\pgfqpoint{5.187533in}{6.101333in}}%
\pgfpathlineto{\pgfqpoint{6.123422in}{6.101333in}}%
\pgfpathquadraticcurveto{\pgfqpoint{6.145644in}{6.101333in}}{\pgfqpoint{6.145644in}{6.123556in}}%
\pgfpathlineto{\pgfqpoint{6.145644in}{6.422222in}}%
\pgfpathquadraticcurveto{\pgfqpoint{6.145644in}{6.444444in}}{\pgfqpoint{6.123422in}{6.444444in}}%
\pgfpathlineto{\pgfqpoint{5.187533in}{6.444444in}}%
\pgfpathquadraticcurveto{\pgfqpoint{5.165311in}{6.444444in}}{\pgfqpoint{5.165311in}{6.422222in}}%
\pgfpathlineto{\pgfqpoint{5.165311in}{6.123556in}}%
\pgfpathquadraticcurveto{\pgfqpoint{5.165311in}{6.101333in}}{\pgfqpoint{5.187533in}{6.101333in}}%
\pgfpathclose%
\pgfusepath{stroke,fill}%
\end{pgfscope}%
\begin{pgfscope}%
\pgfsetbuttcap%
\pgfsetmiterjoin%
\pgfsetlinewidth{1.003750pt}%
\definecolor{currentstroke}{rgb}{0.313725,0.317647,0.309804}%
\pgfsetstrokecolor{currentstroke}%
\pgfsetdash{}{0pt}%
\pgfpathmoveto{\pgfqpoint{5.209756in}{6.322222in}}%
\pgfpathlineto{\pgfqpoint{5.431978in}{6.322222in}}%
\pgfpathlineto{\pgfqpoint{5.431978in}{6.400000in}}%
\pgfpathlineto{\pgfqpoint{5.209756in}{6.400000in}}%
\pgfpathclose%
\pgfusepath{stroke}%
\end{pgfscope}%
\begin{pgfscope}%
\definecolor{textcolor}{rgb}{0.000000,0.000000,0.000000}%
\pgfsetstrokecolor{textcolor}%
\pgfsetfillcolor{textcolor}%
\pgftext[x=5.520867in,y=6.322222in,left,base]{\color{textcolor}\rmfamily\fontsize{8.000000}{9.600000}\selectfont CubeFlow}%
\end{pgfscope}%
\begin{pgfscope}%
\pgfsetbuttcap%
\pgfsetmiterjoin%
\pgfsetlinewidth{1.003750pt}%
\definecolor{currentstroke}{rgb}{0.949020,0.372549,0.360784}%
\pgfsetstrokecolor{currentstroke}%
\pgfsetdash{{1.000000pt}{1.650000pt}}{0.000000pt}%
\pgfpathmoveto{\pgfqpoint{5.209756in}{6.167333in}}%
\pgfpathlineto{\pgfqpoint{5.431978in}{6.167333in}}%
\pgfpathlineto{\pgfqpoint{5.431978in}{6.245111in}}%
\pgfpathlineto{\pgfqpoint{5.209756in}{6.245111in}}%
\pgfpathclose%
\pgfusepath{stroke}%
\end{pgfscope}%
\begin{pgfscope}%
\definecolor{textcolor}{rgb}{0.000000,0.000000,0.000000}%
\pgfsetstrokecolor{textcolor}%
\pgfsetfillcolor{textcolor}%
\pgftext[x=5.520867in,y=6.167333in,left,base]{\color{textcolor}\rmfamily\fontsize{8.000000}{9.600000}\selectfont Retro Reco}%
\end{pgfscope}%
\end{pgfpicture}%
\makeatother%
\endgroup%

    \caption{Zenith error distribution in certain selected bins, representative of the overall performance difference.
    Contrary to the energy case, there is no large inherent bias in zenith reconstruction, at least as a function of true energy.
    The width of the CubeFlow error distribution can be seen to be narrower than the Retro Reco case.}\label{fig:zenith_bins}
\end{figure}

\Vref{fig:energy_bins,fig:zenith_bins} shows error distributions from selected bins, covering where CubeFlow performs best, and where Retro Reco does; it is at once comforting and amazing to see how similar the errors are distributed by energy.
Comforting because the two wildly different algorithms seem to face the same challenges, amazing because one algorithm (Retro Reco) a priori knows physics, while the other (CubeFlow) entirely does not, but infers it based on the truth.

In the energy case, both algorithms skew at low and high energies, predicting too high values at low energy, too low at high.
One might surmise that CubeFlow suffers there because of the lack of training data in these ranges, but the similar performance of Retro Reco suggests that might not be the whole story, and maybe the problem is endemic to the detector.

This bias is also clearly seen in~\vref{fig:energy_2d}, both the top and bottom histograms.
CubeFlow, while biased similarly, is less so at the most populated energy range around \SIrange{10}{50}{\giga\electronvolt}.

\begin{figure}
    \centering
    \input{./images/results/energy_2d.pgf}
    \caption{Energy prediction (top row) and error (bottom row) 2D histograms for CubeFlow (left column) and Retro Reco (right column).
    In the energy case, these two types of histograms are basically the same, only shifted.
    However, for consistency both are provided, and the top histograms are in a sense unprocessed (we calculate no error metric), so any issues with the error calculation would show up here; thus they are good to have.
    The bias can, as expected, be seen in both types of histograms, and the top row makes it very clear that CubeFlow lies along the diagonal line more than Retro Reco, indicating less bias.
    The black dots on the top row histograms show the median in the bin defined by what would typically be the \( x \) error bars.
    The \( y \) error bars show the IQR in that same bin, while the dotted red diagonal line traces out \( y = x \), where perfect reconstructions lie.
    For the error histograms (bottom row) the solid red line shows the median, while the dotted red lines are \( 1 \sigma \).
    The grey dotted line lies along \( y = 0 \), where the reconstruction is error-free.}\label{fig:energy_2d}
\end{figure}

\Vref{fig:zenith_2d} shows the zenith angle reconstruction and error histograms.
These are not biased at lower energies, although performance of course suffers there because low energy events do not leave a long track-like signature.
However, they are biased at as a function of the true polar angle.
This effect is most pronounced for CubeFlow, and may be caused by the fact that the dataset is mostly composed of up-going neutrinos, leaving less training data for down-going (low zenith value) events.
Relatively fewer neutrinos are coming straight up (as seen in~\vref{fig:deepcore_truth_distributions}) explaining the worse performance at high zenith values.

\begin{figure}
    \centering
    \input{./images/results/zenith_2d.pgf}
    \caption{Zenith prediction (top row) and error (bottom row) 2D histograms for CubeFlow (left column) and Retro Reco (right column).
    The top row histograms clearly show a bias at low and high zenith values, most pronounced for CubeFlow.
    Bottom row histograms show no bias as a function of energy, while the resolution is seen to narrow with higher energy.
    The black dots on the top row histograms show the median in the bin defined by what would typically be the \( x \) error bars.
    The \( y \) error bars show the IQR in that same bin, while the dotted red diagonal line traces out \( y = x \), where perfect reconstructions lie.
    For the error histograms (bottom row) the solid red line shows the median, while the dotted red lines are \( 1 \sigma \).
    The grey dotted line lies along \( y = 0 \), where the reconstruction is error-free.}\label{fig:zenith_2d}
\end{figure}

It should be noted that the lower histograms in~\vref{fig:energy_2d,fig:zenith_2d} is basically a two dimensional view of the data summarized in~\vref{fig:energy_comparison,fig:zenith_comparison} and all other figures in that vein.

\begin{figure}
    \centering
    %% Creator: Matplotlib, PGF backend
%%
%% To include the figure in your LaTeX document, write
%%   \input{<filename>.pgf}
%%
%% Make sure the required packages are loaded in your preamble
%%   \usepackage{pgf}
%%
%% and, on pdftex
%%   \usepackage[utf8]{inputenc}\DeclareUnicodeCharacter{2212}{-}
%%
%% or, on luatex and xetex
%%   \usepackage{unicode-math}
%%
%% Figures using additional raster images can only be included by \input if
%% they are in the same directory as the main LaTeX file. For loading figures
%% from other directories you can use the `import` package
%%   \usepackage{import}
%%
%% and then include the figures with
%%   \import{<path to file>}{<filename>.pgf}
%%
%% Matplotlib used the following preamble
%%   \usepackage{siunitx} \usepackage{amsmath} \usepackage{bm}
%%   \usepackage{fontspec}
%%
\begingroup%
\makeatletter%
\begin{pgfpicture}%
\pgfpathrectangle{\pgfpointorigin}{\pgfqpoint{6.201200in}{3.500000in}}%
\pgfusepath{use as bounding box, clip}%
\begin{pgfscope}%
\pgfsetbuttcap%
\pgfsetmiterjoin%
\definecolor{currentfill}{rgb}{1.000000,1.000000,1.000000}%
\pgfsetfillcolor{currentfill}%
\pgfsetlinewidth{0.000000pt}%
\definecolor{currentstroke}{rgb}{1.000000,1.000000,1.000000}%
\pgfsetstrokecolor{currentstroke}%
\pgfsetdash{}{0pt}%
\pgfpathmoveto{\pgfqpoint{0.000000in}{0.000000in}}%
\pgfpathlineto{\pgfqpoint{6.201200in}{0.000000in}}%
\pgfpathlineto{\pgfqpoint{6.201200in}{3.500000in}}%
\pgfpathlineto{\pgfqpoint{0.000000in}{3.500000in}}%
\pgfpathclose%
\pgfusepath{fill}%
\end{pgfscope}%
\begin{pgfscope}%
\pgfsetbuttcap%
\pgfsetmiterjoin%
\definecolor{currentfill}{rgb}{1.000000,1.000000,1.000000}%
\pgfsetfillcolor{currentfill}%
\pgfsetlinewidth{0.000000pt}%
\definecolor{currentstroke}{rgb}{0.000000,0.000000,0.000000}%
\pgfsetstrokecolor{currentstroke}%
\pgfsetstrokeopacity{0.000000}%
\pgfsetdash{}{0pt}%
\pgfpathmoveto{\pgfqpoint{0.664741in}{1.389617in}}%
\pgfpathlineto{\pgfqpoint{5.587445in}{1.389617in}}%
\pgfpathlineto{\pgfqpoint{5.587445in}{3.149333in}}%
\pgfpathlineto{\pgfqpoint{0.664741in}{3.149333in}}%
\pgfpathclose%
\pgfusepath{fill}%
\end{pgfscope}%
\begin{pgfscope}%
\pgfpathrectangle{\pgfqpoint{0.664741in}{1.389617in}}{\pgfqpoint{4.922705in}{1.759717in}}%
\pgfusepath{clip}%
\pgfsetbuttcap%
\pgfsetroundjoin%
\pgfsetlinewidth{0.501875pt}%
\definecolor{currentstroke}{rgb}{0.690196,0.690196,0.690196}%
\pgfsetstrokecolor{currentstroke}%
\pgfsetstrokeopacity{0.500000}%
\pgfsetdash{{0.500000pt}{0.825000pt}}{0.000000pt}%
\pgfpathmoveto{\pgfqpoint{0.888500in}{1.389617in}}%
\pgfpathlineto{\pgfqpoint{0.888500in}{3.149333in}}%
\pgfusepath{stroke}%
\end{pgfscope}%
\begin{pgfscope}%
\pgfsetbuttcap%
\pgfsetroundjoin%
\definecolor{currentfill}{rgb}{0.000000,0.000000,0.000000}%
\pgfsetfillcolor{currentfill}%
\pgfsetlinewidth{0.803000pt}%
\definecolor{currentstroke}{rgb}{0.000000,0.000000,0.000000}%
\pgfsetstrokecolor{currentstroke}%
\pgfsetdash{}{0pt}%
\pgfsys@defobject{currentmarker}{\pgfqpoint{0.000000in}{-0.048611in}}{\pgfqpoint{0.000000in}{0.000000in}}{%
\pgfpathmoveto{\pgfqpoint{0.000000in}{0.000000in}}%
\pgfpathlineto{\pgfqpoint{0.000000in}{-0.048611in}}%
\pgfusepath{stroke,fill}%
}%
\begin{pgfscope}%
\pgfsys@transformshift{0.888500in}{1.389617in}%
\pgfsys@useobject{currentmarker}{}%
\end{pgfscope}%
\end{pgfscope}%
\begin{pgfscope}%
\pgfpathrectangle{\pgfqpoint{0.664741in}{1.389617in}}{\pgfqpoint{4.922705in}{1.759717in}}%
\pgfusepath{clip}%
\pgfsetbuttcap%
\pgfsetroundjoin%
\pgfsetlinewidth{0.501875pt}%
\definecolor{currentstroke}{rgb}{0.690196,0.690196,0.690196}%
\pgfsetstrokecolor{currentstroke}%
\pgfsetstrokeopacity{0.500000}%
\pgfsetdash{{0.500000pt}{0.825000pt}}{0.000000pt}%
\pgfpathmoveto{\pgfqpoint{1.634364in}{1.389617in}}%
\pgfpathlineto{\pgfqpoint{1.634364in}{3.149333in}}%
\pgfusepath{stroke}%
\end{pgfscope}%
\begin{pgfscope}%
\pgfsetbuttcap%
\pgfsetroundjoin%
\definecolor{currentfill}{rgb}{0.000000,0.000000,0.000000}%
\pgfsetfillcolor{currentfill}%
\pgfsetlinewidth{0.803000pt}%
\definecolor{currentstroke}{rgb}{0.000000,0.000000,0.000000}%
\pgfsetstrokecolor{currentstroke}%
\pgfsetdash{}{0pt}%
\pgfsys@defobject{currentmarker}{\pgfqpoint{0.000000in}{-0.048611in}}{\pgfqpoint{0.000000in}{0.000000in}}{%
\pgfpathmoveto{\pgfqpoint{0.000000in}{0.000000in}}%
\pgfpathlineto{\pgfqpoint{0.000000in}{-0.048611in}}%
\pgfusepath{stroke,fill}%
}%
\begin{pgfscope}%
\pgfsys@transformshift{1.634364in}{1.389617in}%
\pgfsys@useobject{currentmarker}{}%
\end{pgfscope}%
\end{pgfscope}%
\begin{pgfscope}%
\pgfpathrectangle{\pgfqpoint{0.664741in}{1.389617in}}{\pgfqpoint{4.922705in}{1.759717in}}%
\pgfusepath{clip}%
\pgfsetbuttcap%
\pgfsetroundjoin%
\pgfsetlinewidth{0.501875pt}%
\definecolor{currentstroke}{rgb}{0.690196,0.690196,0.690196}%
\pgfsetstrokecolor{currentstroke}%
\pgfsetstrokeopacity{0.500000}%
\pgfsetdash{{0.500000pt}{0.825000pt}}{0.000000pt}%
\pgfpathmoveto{\pgfqpoint{2.380229in}{1.389617in}}%
\pgfpathlineto{\pgfqpoint{2.380229in}{3.149333in}}%
\pgfusepath{stroke}%
\end{pgfscope}%
\begin{pgfscope}%
\pgfsetbuttcap%
\pgfsetroundjoin%
\definecolor{currentfill}{rgb}{0.000000,0.000000,0.000000}%
\pgfsetfillcolor{currentfill}%
\pgfsetlinewidth{0.803000pt}%
\definecolor{currentstroke}{rgb}{0.000000,0.000000,0.000000}%
\pgfsetstrokecolor{currentstroke}%
\pgfsetdash{}{0pt}%
\pgfsys@defobject{currentmarker}{\pgfqpoint{0.000000in}{-0.048611in}}{\pgfqpoint{0.000000in}{0.000000in}}{%
\pgfpathmoveto{\pgfqpoint{0.000000in}{0.000000in}}%
\pgfpathlineto{\pgfqpoint{0.000000in}{-0.048611in}}%
\pgfusepath{stroke,fill}%
}%
\begin{pgfscope}%
\pgfsys@transformshift{2.380229in}{1.389617in}%
\pgfsys@useobject{currentmarker}{}%
\end{pgfscope}%
\end{pgfscope}%
\begin{pgfscope}%
\pgfpathrectangle{\pgfqpoint{0.664741in}{1.389617in}}{\pgfqpoint{4.922705in}{1.759717in}}%
\pgfusepath{clip}%
\pgfsetbuttcap%
\pgfsetroundjoin%
\pgfsetlinewidth{0.501875pt}%
\definecolor{currentstroke}{rgb}{0.690196,0.690196,0.690196}%
\pgfsetstrokecolor{currentstroke}%
\pgfsetstrokeopacity{0.500000}%
\pgfsetdash{{0.500000pt}{0.825000pt}}{0.000000pt}%
\pgfpathmoveto{\pgfqpoint{3.126093in}{1.389617in}}%
\pgfpathlineto{\pgfqpoint{3.126093in}{3.149333in}}%
\pgfusepath{stroke}%
\end{pgfscope}%
\begin{pgfscope}%
\pgfsetbuttcap%
\pgfsetroundjoin%
\definecolor{currentfill}{rgb}{0.000000,0.000000,0.000000}%
\pgfsetfillcolor{currentfill}%
\pgfsetlinewidth{0.803000pt}%
\definecolor{currentstroke}{rgb}{0.000000,0.000000,0.000000}%
\pgfsetstrokecolor{currentstroke}%
\pgfsetdash{}{0pt}%
\pgfsys@defobject{currentmarker}{\pgfqpoint{0.000000in}{-0.048611in}}{\pgfqpoint{0.000000in}{0.000000in}}{%
\pgfpathmoveto{\pgfqpoint{0.000000in}{0.000000in}}%
\pgfpathlineto{\pgfqpoint{0.000000in}{-0.048611in}}%
\pgfusepath{stroke,fill}%
}%
\begin{pgfscope}%
\pgfsys@transformshift{3.126093in}{1.389617in}%
\pgfsys@useobject{currentmarker}{}%
\end{pgfscope}%
\end{pgfscope}%
\begin{pgfscope}%
\pgfpathrectangle{\pgfqpoint{0.664741in}{1.389617in}}{\pgfqpoint{4.922705in}{1.759717in}}%
\pgfusepath{clip}%
\pgfsetbuttcap%
\pgfsetroundjoin%
\pgfsetlinewidth{0.501875pt}%
\definecolor{currentstroke}{rgb}{0.690196,0.690196,0.690196}%
\pgfsetstrokecolor{currentstroke}%
\pgfsetstrokeopacity{0.500000}%
\pgfsetdash{{0.500000pt}{0.825000pt}}{0.000000pt}%
\pgfpathmoveto{\pgfqpoint{3.871957in}{1.389617in}}%
\pgfpathlineto{\pgfqpoint{3.871957in}{3.149333in}}%
\pgfusepath{stroke}%
\end{pgfscope}%
\begin{pgfscope}%
\pgfsetbuttcap%
\pgfsetroundjoin%
\definecolor{currentfill}{rgb}{0.000000,0.000000,0.000000}%
\pgfsetfillcolor{currentfill}%
\pgfsetlinewidth{0.803000pt}%
\definecolor{currentstroke}{rgb}{0.000000,0.000000,0.000000}%
\pgfsetstrokecolor{currentstroke}%
\pgfsetdash{}{0pt}%
\pgfsys@defobject{currentmarker}{\pgfqpoint{0.000000in}{-0.048611in}}{\pgfqpoint{0.000000in}{0.000000in}}{%
\pgfpathmoveto{\pgfqpoint{0.000000in}{0.000000in}}%
\pgfpathlineto{\pgfqpoint{0.000000in}{-0.048611in}}%
\pgfusepath{stroke,fill}%
}%
\begin{pgfscope}%
\pgfsys@transformshift{3.871957in}{1.389617in}%
\pgfsys@useobject{currentmarker}{}%
\end{pgfscope}%
\end{pgfscope}%
\begin{pgfscope}%
\pgfpathrectangle{\pgfqpoint{0.664741in}{1.389617in}}{\pgfqpoint{4.922705in}{1.759717in}}%
\pgfusepath{clip}%
\pgfsetbuttcap%
\pgfsetroundjoin%
\pgfsetlinewidth{0.501875pt}%
\definecolor{currentstroke}{rgb}{0.690196,0.690196,0.690196}%
\pgfsetstrokecolor{currentstroke}%
\pgfsetstrokeopacity{0.500000}%
\pgfsetdash{{0.500000pt}{0.825000pt}}{0.000000pt}%
\pgfpathmoveto{\pgfqpoint{4.617822in}{1.389617in}}%
\pgfpathlineto{\pgfqpoint{4.617822in}{3.149333in}}%
\pgfusepath{stroke}%
\end{pgfscope}%
\begin{pgfscope}%
\pgfsetbuttcap%
\pgfsetroundjoin%
\definecolor{currentfill}{rgb}{0.000000,0.000000,0.000000}%
\pgfsetfillcolor{currentfill}%
\pgfsetlinewidth{0.803000pt}%
\definecolor{currentstroke}{rgb}{0.000000,0.000000,0.000000}%
\pgfsetstrokecolor{currentstroke}%
\pgfsetdash{}{0pt}%
\pgfsys@defobject{currentmarker}{\pgfqpoint{0.000000in}{-0.048611in}}{\pgfqpoint{0.000000in}{0.000000in}}{%
\pgfpathmoveto{\pgfqpoint{0.000000in}{0.000000in}}%
\pgfpathlineto{\pgfqpoint{0.000000in}{-0.048611in}}%
\pgfusepath{stroke,fill}%
}%
\begin{pgfscope}%
\pgfsys@transformshift{4.617822in}{1.389617in}%
\pgfsys@useobject{currentmarker}{}%
\end{pgfscope}%
\end{pgfscope}%
\begin{pgfscope}%
\pgfpathrectangle{\pgfqpoint{0.664741in}{1.389617in}}{\pgfqpoint{4.922705in}{1.759717in}}%
\pgfusepath{clip}%
\pgfsetbuttcap%
\pgfsetroundjoin%
\pgfsetlinewidth{0.501875pt}%
\definecolor{currentstroke}{rgb}{0.690196,0.690196,0.690196}%
\pgfsetstrokecolor{currentstroke}%
\pgfsetstrokeopacity{0.500000}%
\pgfsetdash{{0.500000pt}{0.825000pt}}{0.000000pt}%
\pgfpathmoveto{\pgfqpoint{5.363686in}{1.389617in}}%
\pgfpathlineto{\pgfqpoint{5.363686in}{3.149333in}}%
\pgfusepath{stroke}%
\end{pgfscope}%
\begin{pgfscope}%
\pgfsetbuttcap%
\pgfsetroundjoin%
\definecolor{currentfill}{rgb}{0.000000,0.000000,0.000000}%
\pgfsetfillcolor{currentfill}%
\pgfsetlinewidth{0.803000pt}%
\definecolor{currentstroke}{rgb}{0.000000,0.000000,0.000000}%
\pgfsetstrokecolor{currentstroke}%
\pgfsetdash{}{0pt}%
\pgfsys@defobject{currentmarker}{\pgfqpoint{0.000000in}{-0.048611in}}{\pgfqpoint{0.000000in}{0.000000in}}{%
\pgfpathmoveto{\pgfqpoint{0.000000in}{0.000000in}}%
\pgfpathlineto{\pgfqpoint{0.000000in}{-0.048611in}}%
\pgfusepath{stroke,fill}%
}%
\begin{pgfscope}%
\pgfsys@transformshift{5.363686in}{1.389617in}%
\pgfsys@useobject{currentmarker}{}%
\end{pgfscope}%
\end{pgfscope}%
\begin{pgfscope}%
\pgfpathrectangle{\pgfqpoint{0.664741in}{1.389617in}}{\pgfqpoint{4.922705in}{1.759717in}}%
\pgfusepath{clip}%
\pgfsetbuttcap%
\pgfsetroundjoin%
\pgfsetlinewidth{0.501875pt}%
\definecolor{currentstroke}{rgb}{0.690196,0.690196,0.690196}%
\pgfsetstrokecolor{currentstroke}%
\pgfsetstrokeopacity{0.500000}%
\pgfsetdash{{0.500000pt}{0.825000pt}}{0.000000pt}%
\pgfpathmoveto{\pgfqpoint{0.664741in}{1.619249in}}%
\pgfpathlineto{\pgfqpoint{5.587445in}{1.619249in}}%
\pgfusepath{stroke}%
\end{pgfscope}%
\begin{pgfscope}%
\pgfsetbuttcap%
\pgfsetroundjoin%
\definecolor{currentfill}{rgb}{0.000000,0.000000,0.000000}%
\pgfsetfillcolor{currentfill}%
\pgfsetlinewidth{0.803000pt}%
\definecolor{currentstroke}{rgb}{0.000000,0.000000,0.000000}%
\pgfsetstrokecolor{currentstroke}%
\pgfsetdash{}{0pt}%
\pgfsys@defobject{currentmarker}{\pgfqpoint{-0.048611in}{0.000000in}}{\pgfqpoint{-0.000000in}{0.000000in}}{%
\pgfpathmoveto{\pgfqpoint{-0.000000in}{0.000000in}}%
\pgfpathlineto{\pgfqpoint{-0.048611in}{0.000000in}}%
\pgfusepath{stroke,fill}%
}%
\begin{pgfscope}%
\pgfsys@transformshift{0.664741in}{1.619249in}%
\pgfsys@useobject{currentmarker}{}%
\end{pgfscope}%
\end{pgfscope}%
\begin{pgfscope}%
\definecolor{textcolor}{rgb}{0.000000,0.000000,0.000000}%
\pgfsetstrokecolor{textcolor}%
\pgfsetfillcolor{textcolor}%
\pgftext[x=0.298610in, y=1.580694in, left, base]{\color{textcolor}\rmfamily\fontsize{8.000000}{9.600000}\selectfont \(\displaystyle {0.150}\)}%
\end{pgfscope}%
\begin{pgfscope}%
\pgfpathrectangle{\pgfqpoint{0.664741in}{1.389617in}}{\pgfqpoint{4.922705in}{1.759717in}}%
\pgfusepath{clip}%
\pgfsetbuttcap%
\pgfsetroundjoin%
\pgfsetlinewidth{0.501875pt}%
\definecolor{currentstroke}{rgb}{0.690196,0.690196,0.690196}%
\pgfsetstrokecolor{currentstroke}%
\pgfsetstrokeopacity{0.500000}%
\pgfsetdash{{0.500000pt}{0.825000pt}}{0.000000pt}%
\pgfpathmoveto{\pgfqpoint{0.664741in}{1.914993in}}%
\pgfpathlineto{\pgfqpoint{5.587445in}{1.914993in}}%
\pgfusepath{stroke}%
\end{pgfscope}%
\begin{pgfscope}%
\pgfsetbuttcap%
\pgfsetroundjoin%
\definecolor{currentfill}{rgb}{0.000000,0.000000,0.000000}%
\pgfsetfillcolor{currentfill}%
\pgfsetlinewidth{0.803000pt}%
\definecolor{currentstroke}{rgb}{0.000000,0.000000,0.000000}%
\pgfsetstrokecolor{currentstroke}%
\pgfsetdash{}{0pt}%
\pgfsys@defobject{currentmarker}{\pgfqpoint{-0.048611in}{0.000000in}}{\pgfqpoint{-0.000000in}{0.000000in}}{%
\pgfpathmoveto{\pgfqpoint{-0.000000in}{0.000000in}}%
\pgfpathlineto{\pgfqpoint{-0.048611in}{0.000000in}}%
\pgfusepath{stroke,fill}%
}%
\begin{pgfscope}%
\pgfsys@transformshift{0.664741in}{1.914993in}%
\pgfsys@useobject{currentmarker}{}%
\end{pgfscope}%
\end{pgfscope}%
\begin{pgfscope}%
\definecolor{textcolor}{rgb}{0.000000,0.000000,0.000000}%
\pgfsetstrokecolor{textcolor}%
\pgfsetfillcolor{textcolor}%
\pgftext[x=0.298610in, y=1.876437in, left, base]{\color{textcolor}\rmfamily\fontsize{8.000000}{9.600000}\selectfont \(\displaystyle {0.175}\)}%
\end{pgfscope}%
\begin{pgfscope}%
\pgfpathrectangle{\pgfqpoint{0.664741in}{1.389617in}}{\pgfqpoint{4.922705in}{1.759717in}}%
\pgfusepath{clip}%
\pgfsetbuttcap%
\pgfsetroundjoin%
\pgfsetlinewidth{0.501875pt}%
\definecolor{currentstroke}{rgb}{0.690196,0.690196,0.690196}%
\pgfsetstrokecolor{currentstroke}%
\pgfsetstrokeopacity{0.500000}%
\pgfsetdash{{0.500000pt}{0.825000pt}}{0.000000pt}%
\pgfpathmoveto{\pgfqpoint{0.664741in}{2.210737in}}%
\pgfpathlineto{\pgfqpoint{5.587445in}{2.210737in}}%
\pgfusepath{stroke}%
\end{pgfscope}%
\begin{pgfscope}%
\pgfsetbuttcap%
\pgfsetroundjoin%
\definecolor{currentfill}{rgb}{0.000000,0.000000,0.000000}%
\pgfsetfillcolor{currentfill}%
\pgfsetlinewidth{0.803000pt}%
\definecolor{currentstroke}{rgb}{0.000000,0.000000,0.000000}%
\pgfsetstrokecolor{currentstroke}%
\pgfsetdash{}{0pt}%
\pgfsys@defobject{currentmarker}{\pgfqpoint{-0.048611in}{0.000000in}}{\pgfqpoint{-0.000000in}{0.000000in}}{%
\pgfpathmoveto{\pgfqpoint{-0.000000in}{0.000000in}}%
\pgfpathlineto{\pgfqpoint{-0.048611in}{0.000000in}}%
\pgfusepath{stroke,fill}%
}%
\begin{pgfscope}%
\pgfsys@transformshift{0.664741in}{2.210737in}%
\pgfsys@useobject{currentmarker}{}%
\end{pgfscope}%
\end{pgfscope}%
\begin{pgfscope}%
\definecolor{textcolor}{rgb}{0.000000,0.000000,0.000000}%
\pgfsetstrokecolor{textcolor}%
\pgfsetfillcolor{textcolor}%
\pgftext[x=0.298610in, y=2.172181in, left, base]{\color{textcolor}\rmfamily\fontsize{8.000000}{9.600000}\selectfont \(\displaystyle {0.200}\)}%
\end{pgfscope}%
\begin{pgfscope}%
\pgfpathrectangle{\pgfqpoint{0.664741in}{1.389617in}}{\pgfqpoint{4.922705in}{1.759717in}}%
\pgfusepath{clip}%
\pgfsetbuttcap%
\pgfsetroundjoin%
\pgfsetlinewidth{0.501875pt}%
\definecolor{currentstroke}{rgb}{0.690196,0.690196,0.690196}%
\pgfsetstrokecolor{currentstroke}%
\pgfsetstrokeopacity{0.500000}%
\pgfsetdash{{0.500000pt}{0.825000pt}}{0.000000pt}%
\pgfpathmoveto{\pgfqpoint{0.664741in}{2.506481in}}%
\pgfpathlineto{\pgfqpoint{5.587445in}{2.506481in}}%
\pgfusepath{stroke}%
\end{pgfscope}%
\begin{pgfscope}%
\pgfsetbuttcap%
\pgfsetroundjoin%
\definecolor{currentfill}{rgb}{0.000000,0.000000,0.000000}%
\pgfsetfillcolor{currentfill}%
\pgfsetlinewidth{0.803000pt}%
\definecolor{currentstroke}{rgb}{0.000000,0.000000,0.000000}%
\pgfsetstrokecolor{currentstroke}%
\pgfsetdash{}{0pt}%
\pgfsys@defobject{currentmarker}{\pgfqpoint{-0.048611in}{0.000000in}}{\pgfqpoint{-0.000000in}{0.000000in}}{%
\pgfpathmoveto{\pgfqpoint{-0.000000in}{0.000000in}}%
\pgfpathlineto{\pgfqpoint{-0.048611in}{0.000000in}}%
\pgfusepath{stroke,fill}%
}%
\begin{pgfscope}%
\pgfsys@transformshift{0.664741in}{2.506481in}%
\pgfsys@useobject{currentmarker}{}%
\end{pgfscope}%
\end{pgfscope}%
\begin{pgfscope}%
\definecolor{textcolor}{rgb}{0.000000,0.000000,0.000000}%
\pgfsetstrokecolor{textcolor}%
\pgfsetfillcolor{textcolor}%
\pgftext[x=0.298610in, y=2.467925in, left, base]{\color{textcolor}\rmfamily\fontsize{8.000000}{9.600000}\selectfont \(\displaystyle {0.225}\)}%
\end{pgfscope}%
\begin{pgfscope}%
\pgfpathrectangle{\pgfqpoint{0.664741in}{1.389617in}}{\pgfqpoint{4.922705in}{1.759717in}}%
\pgfusepath{clip}%
\pgfsetbuttcap%
\pgfsetroundjoin%
\pgfsetlinewidth{0.501875pt}%
\definecolor{currentstroke}{rgb}{0.690196,0.690196,0.690196}%
\pgfsetstrokecolor{currentstroke}%
\pgfsetstrokeopacity{0.500000}%
\pgfsetdash{{0.500000pt}{0.825000pt}}{0.000000pt}%
\pgfpathmoveto{\pgfqpoint{0.664741in}{2.802225in}}%
\pgfpathlineto{\pgfqpoint{5.587445in}{2.802225in}}%
\pgfusepath{stroke}%
\end{pgfscope}%
\begin{pgfscope}%
\pgfsetbuttcap%
\pgfsetroundjoin%
\definecolor{currentfill}{rgb}{0.000000,0.000000,0.000000}%
\pgfsetfillcolor{currentfill}%
\pgfsetlinewidth{0.803000pt}%
\definecolor{currentstroke}{rgb}{0.000000,0.000000,0.000000}%
\pgfsetstrokecolor{currentstroke}%
\pgfsetdash{}{0pt}%
\pgfsys@defobject{currentmarker}{\pgfqpoint{-0.048611in}{0.000000in}}{\pgfqpoint{-0.000000in}{0.000000in}}{%
\pgfpathmoveto{\pgfqpoint{-0.000000in}{0.000000in}}%
\pgfpathlineto{\pgfqpoint{-0.048611in}{0.000000in}}%
\pgfusepath{stroke,fill}%
}%
\begin{pgfscope}%
\pgfsys@transformshift{0.664741in}{2.802225in}%
\pgfsys@useobject{currentmarker}{}%
\end{pgfscope}%
\end{pgfscope}%
\begin{pgfscope}%
\definecolor{textcolor}{rgb}{0.000000,0.000000,0.000000}%
\pgfsetstrokecolor{textcolor}%
\pgfsetfillcolor{textcolor}%
\pgftext[x=0.298610in, y=2.763669in, left, base]{\color{textcolor}\rmfamily\fontsize{8.000000}{9.600000}\selectfont \(\displaystyle {0.250}\)}%
\end{pgfscope}%
\begin{pgfscope}%
\pgfpathrectangle{\pgfqpoint{0.664741in}{1.389617in}}{\pgfqpoint{4.922705in}{1.759717in}}%
\pgfusepath{clip}%
\pgfsetbuttcap%
\pgfsetroundjoin%
\pgfsetlinewidth{0.501875pt}%
\definecolor{currentstroke}{rgb}{0.690196,0.690196,0.690196}%
\pgfsetstrokecolor{currentstroke}%
\pgfsetstrokeopacity{0.500000}%
\pgfsetdash{{0.500000pt}{0.825000pt}}{0.000000pt}%
\pgfpathmoveto{\pgfqpoint{0.664741in}{3.097968in}}%
\pgfpathlineto{\pgfqpoint{5.587445in}{3.097968in}}%
\pgfusepath{stroke}%
\end{pgfscope}%
\begin{pgfscope}%
\pgfsetbuttcap%
\pgfsetroundjoin%
\definecolor{currentfill}{rgb}{0.000000,0.000000,0.000000}%
\pgfsetfillcolor{currentfill}%
\pgfsetlinewidth{0.803000pt}%
\definecolor{currentstroke}{rgb}{0.000000,0.000000,0.000000}%
\pgfsetstrokecolor{currentstroke}%
\pgfsetdash{}{0pt}%
\pgfsys@defobject{currentmarker}{\pgfqpoint{-0.048611in}{0.000000in}}{\pgfqpoint{-0.000000in}{0.000000in}}{%
\pgfpathmoveto{\pgfqpoint{-0.000000in}{0.000000in}}%
\pgfpathlineto{\pgfqpoint{-0.048611in}{0.000000in}}%
\pgfusepath{stroke,fill}%
}%
\begin{pgfscope}%
\pgfsys@transformshift{0.664741in}{3.097968in}%
\pgfsys@useobject{currentmarker}{}%
\end{pgfscope}%
\end{pgfscope}%
\begin{pgfscope}%
\definecolor{textcolor}{rgb}{0.000000,0.000000,0.000000}%
\pgfsetstrokecolor{textcolor}%
\pgfsetfillcolor{textcolor}%
\pgftext[x=0.298610in, y=3.059413in, left, base]{\color{textcolor}\rmfamily\fontsize{8.000000}{9.600000}\selectfont \(\displaystyle {0.275}\)}%
\end{pgfscope}%
\begin{pgfscope}%
\definecolor{textcolor}{rgb}{0.000000,0.000000,0.000000}%
\pgfsetstrokecolor{textcolor}%
\pgfsetfillcolor{textcolor}%
\pgftext[x=0.243055in,y=2.269475in,,bottom,rotate=90.000000]{\color{textcolor}\rmfamily\fontsize{10.950000}{13.140000}\selectfont IQR / 1.349 \(\displaystyle \left[ E / \textup{GeV} \right]\)}%
\end{pgfscope}%
\begin{pgfscope}%
\pgfpathrectangle{\pgfqpoint{0.664741in}{1.389617in}}{\pgfqpoint{4.922705in}{1.759717in}}%
\pgfusepath{clip}%
\pgfsetbuttcap%
\pgfsetroundjoin%
\pgfsetlinewidth{1.505625pt}%
\definecolor{currentstroke}{rgb}{0.313725,0.317647,0.309804}%
\pgfsetstrokecolor{currentstroke}%
\pgfsetstrokeopacity{0.900000}%
\pgfsetdash{}{0pt}%
\pgfpathmoveto{\pgfqpoint{0.888500in}{2.868858in}}%
\pgfpathlineto{\pgfqpoint{1.137121in}{2.868858in}}%
\pgfusepath{stroke}%
\end{pgfscope}%
\begin{pgfscope}%
\pgfpathrectangle{\pgfqpoint{0.664741in}{1.389617in}}{\pgfqpoint{4.922705in}{1.759717in}}%
\pgfusepath{clip}%
\pgfsetbuttcap%
\pgfsetroundjoin%
\pgfsetlinewidth{1.505625pt}%
\definecolor{currentstroke}{rgb}{0.313725,0.317647,0.309804}%
\pgfsetstrokecolor{currentstroke}%
\pgfsetstrokeopacity{0.900000}%
\pgfsetdash{}{0pt}%
\pgfpathmoveto{\pgfqpoint{1.137121in}{2.716306in}}%
\pgfpathlineto{\pgfqpoint{1.385743in}{2.716306in}}%
\pgfusepath{stroke}%
\end{pgfscope}%
\begin{pgfscope}%
\pgfpathrectangle{\pgfqpoint{0.664741in}{1.389617in}}{\pgfqpoint{4.922705in}{1.759717in}}%
\pgfusepath{clip}%
\pgfsetbuttcap%
\pgfsetroundjoin%
\pgfsetlinewidth{1.505625pt}%
\definecolor{currentstroke}{rgb}{0.313725,0.317647,0.309804}%
\pgfsetstrokecolor{currentstroke}%
\pgfsetstrokeopacity{0.900000}%
\pgfsetdash{}{0pt}%
\pgfpathmoveto{\pgfqpoint{1.385743in}{2.557950in}}%
\pgfpathlineto{\pgfqpoint{1.634364in}{2.557950in}}%
\pgfusepath{stroke}%
\end{pgfscope}%
\begin{pgfscope}%
\pgfpathrectangle{\pgfqpoint{0.664741in}{1.389617in}}{\pgfqpoint{4.922705in}{1.759717in}}%
\pgfusepath{clip}%
\pgfsetbuttcap%
\pgfsetroundjoin%
\pgfsetlinewidth{1.505625pt}%
\definecolor{currentstroke}{rgb}{0.313725,0.317647,0.309804}%
\pgfsetstrokecolor{currentstroke}%
\pgfsetstrokeopacity{0.900000}%
\pgfsetdash{}{0pt}%
\pgfpathmoveto{\pgfqpoint{1.634364in}{2.437089in}}%
\pgfpathlineto{\pgfqpoint{1.882986in}{2.437089in}}%
\pgfusepath{stroke}%
\end{pgfscope}%
\begin{pgfscope}%
\pgfpathrectangle{\pgfqpoint{0.664741in}{1.389617in}}{\pgfqpoint{4.922705in}{1.759717in}}%
\pgfusepath{clip}%
\pgfsetbuttcap%
\pgfsetroundjoin%
\pgfsetlinewidth{1.505625pt}%
\definecolor{currentstroke}{rgb}{0.313725,0.317647,0.309804}%
\pgfsetstrokecolor{currentstroke}%
\pgfsetstrokeopacity{0.900000}%
\pgfsetdash{}{0pt}%
\pgfpathmoveto{\pgfqpoint{1.882986in}{2.323877in}}%
\pgfpathlineto{\pgfqpoint{2.131607in}{2.323877in}}%
\pgfusepath{stroke}%
\end{pgfscope}%
\begin{pgfscope}%
\pgfpathrectangle{\pgfqpoint{0.664741in}{1.389617in}}{\pgfqpoint{4.922705in}{1.759717in}}%
\pgfusepath{clip}%
\pgfsetbuttcap%
\pgfsetroundjoin%
\pgfsetlinewidth{1.505625pt}%
\definecolor{currentstroke}{rgb}{0.313725,0.317647,0.309804}%
\pgfsetstrokecolor{currentstroke}%
\pgfsetstrokeopacity{0.900000}%
\pgfsetdash{}{0pt}%
\pgfpathmoveto{\pgfqpoint{2.131607in}{2.224214in}}%
\pgfpathlineto{\pgfqpoint{2.380229in}{2.224214in}}%
\pgfusepath{stroke}%
\end{pgfscope}%
\begin{pgfscope}%
\pgfpathrectangle{\pgfqpoint{0.664741in}{1.389617in}}{\pgfqpoint{4.922705in}{1.759717in}}%
\pgfusepath{clip}%
\pgfsetbuttcap%
\pgfsetroundjoin%
\pgfsetlinewidth{1.505625pt}%
\definecolor{currentstroke}{rgb}{0.313725,0.317647,0.309804}%
\pgfsetstrokecolor{currentstroke}%
\pgfsetstrokeopacity{0.900000}%
\pgfsetdash{}{0pt}%
\pgfpathmoveto{\pgfqpoint{2.380229in}{2.093596in}}%
\pgfpathlineto{\pgfqpoint{2.628850in}{2.093596in}}%
\pgfusepath{stroke}%
\end{pgfscope}%
\begin{pgfscope}%
\pgfpathrectangle{\pgfqpoint{0.664741in}{1.389617in}}{\pgfqpoint{4.922705in}{1.759717in}}%
\pgfusepath{clip}%
\pgfsetbuttcap%
\pgfsetroundjoin%
\pgfsetlinewidth{1.505625pt}%
\definecolor{currentstroke}{rgb}{0.313725,0.317647,0.309804}%
\pgfsetstrokecolor{currentstroke}%
\pgfsetstrokeopacity{0.900000}%
\pgfsetdash{}{0pt}%
\pgfpathmoveto{\pgfqpoint{2.628850in}{1.976669in}}%
\pgfpathlineto{\pgfqpoint{2.877472in}{1.976669in}}%
\pgfusepath{stroke}%
\end{pgfscope}%
\begin{pgfscope}%
\pgfpathrectangle{\pgfqpoint{0.664741in}{1.389617in}}{\pgfqpoint{4.922705in}{1.759717in}}%
\pgfusepath{clip}%
\pgfsetbuttcap%
\pgfsetroundjoin%
\pgfsetlinewidth{1.505625pt}%
\definecolor{currentstroke}{rgb}{0.313725,0.317647,0.309804}%
\pgfsetstrokecolor{currentstroke}%
\pgfsetstrokeopacity{0.900000}%
\pgfsetdash{}{0pt}%
\pgfpathmoveto{\pgfqpoint{2.877472in}{1.915815in}}%
\pgfpathlineto{\pgfqpoint{3.126093in}{1.915815in}}%
\pgfusepath{stroke}%
\end{pgfscope}%
\begin{pgfscope}%
\pgfpathrectangle{\pgfqpoint{0.664741in}{1.389617in}}{\pgfqpoint{4.922705in}{1.759717in}}%
\pgfusepath{clip}%
\pgfsetbuttcap%
\pgfsetroundjoin%
\pgfsetlinewidth{1.505625pt}%
\definecolor{currentstroke}{rgb}{0.313725,0.317647,0.309804}%
\pgfsetstrokecolor{currentstroke}%
\pgfsetstrokeopacity{0.900000}%
\pgfsetdash{}{0pt}%
\pgfpathmoveto{\pgfqpoint{3.126093in}{1.921293in}}%
\pgfpathlineto{\pgfqpoint{3.374715in}{1.921293in}}%
\pgfusepath{stroke}%
\end{pgfscope}%
\begin{pgfscope}%
\pgfpathrectangle{\pgfqpoint{0.664741in}{1.389617in}}{\pgfqpoint{4.922705in}{1.759717in}}%
\pgfusepath{clip}%
\pgfsetbuttcap%
\pgfsetroundjoin%
\pgfsetlinewidth{1.505625pt}%
\definecolor{currentstroke}{rgb}{0.313725,0.317647,0.309804}%
\pgfsetstrokecolor{currentstroke}%
\pgfsetstrokeopacity{0.900000}%
\pgfsetdash{}{0pt}%
\pgfpathmoveto{\pgfqpoint{3.374715in}{1.886080in}}%
\pgfpathlineto{\pgfqpoint{3.623336in}{1.886080in}}%
\pgfusepath{stroke}%
\end{pgfscope}%
\begin{pgfscope}%
\pgfpathrectangle{\pgfqpoint{0.664741in}{1.389617in}}{\pgfqpoint{4.922705in}{1.759717in}}%
\pgfusepath{clip}%
\pgfsetbuttcap%
\pgfsetroundjoin%
\pgfsetlinewidth{1.505625pt}%
\definecolor{currentstroke}{rgb}{0.313725,0.317647,0.309804}%
\pgfsetstrokecolor{currentstroke}%
\pgfsetstrokeopacity{0.900000}%
\pgfsetdash{}{0pt}%
\pgfpathmoveto{\pgfqpoint{3.623336in}{1.866379in}}%
\pgfpathlineto{\pgfqpoint{3.871957in}{1.866379in}}%
\pgfusepath{stroke}%
\end{pgfscope}%
\begin{pgfscope}%
\pgfpathrectangle{\pgfqpoint{0.664741in}{1.389617in}}{\pgfqpoint{4.922705in}{1.759717in}}%
\pgfusepath{clip}%
\pgfsetbuttcap%
\pgfsetroundjoin%
\pgfsetlinewidth{1.505625pt}%
\definecolor{currentstroke}{rgb}{0.313725,0.317647,0.309804}%
\pgfsetstrokecolor{currentstroke}%
\pgfsetstrokeopacity{0.900000}%
\pgfsetdash{}{0pt}%
\pgfpathmoveto{\pgfqpoint{3.871957in}{1.877324in}}%
\pgfpathlineto{\pgfqpoint{4.120579in}{1.877324in}}%
\pgfusepath{stroke}%
\end{pgfscope}%
\begin{pgfscope}%
\pgfpathrectangle{\pgfqpoint{0.664741in}{1.389617in}}{\pgfqpoint{4.922705in}{1.759717in}}%
\pgfusepath{clip}%
\pgfsetbuttcap%
\pgfsetroundjoin%
\pgfsetlinewidth{1.505625pt}%
\definecolor{currentstroke}{rgb}{0.313725,0.317647,0.309804}%
\pgfsetstrokecolor{currentstroke}%
\pgfsetstrokeopacity{0.900000}%
\pgfsetdash{}{0pt}%
\pgfpathmoveto{\pgfqpoint{4.120579in}{1.969561in}}%
\pgfpathlineto{\pgfqpoint{4.369200in}{1.969561in}}%
\pgfusepath{stroke}%
\end{pgfscope}%
\begin{pgfscope}%
\pgfpathrectangle{\pgfqpoint{0.664741in}{1.389617in}}{\pgfqpoint{4.922705in}{1.759717in}}%
\pgfusepath{clip}%
\pgfsetbuttcap%
\pgfsetroundjoin%
\pgfsetlinewidth{1.505625pt}%
\definecolor{currentstroke}{rgb}{0.313725,0.317647,0.309804}%
\pgfsetstrokecolor{currentstroke}%
\pgfsetstrokeopacity{0.900000}%
\pgfsetdash{}{0pt}%
\pgfpathmoveto{\pgfqpoint{4.369200in}{2.129492in}}%
\pgfpathlineto{\pgfqpoint{4.617822in}{2.129492in}}%
\pgfusepath{stroke}%
\end{pgfscope}%
\begin{pgfscope}%
\pgfpathrectangle{\pgfqpoint{0.664741in}{1.389617in}}{\pgfqpoint{4.922705in}{1.759717in}}%
\pgfusepath{clip}%
\pgfsetbuttcap%
\pgfsetroundjoin%
\pgfsetlinewidth{1.505625pt}%
\definecolor{currentstroke}{rgb}{0.313725,0.317647,0.309804}%
\pgfsetstrokecolor{currentstroke}%
\pgfsetstrokeopacity{0.900000}%
\pgfsetdash{}{0pt}%
\pgfpathmoveto{\pgfqpoint{4.617822in}{2.346695in}}%
\pgfpathlineto{\pgfqpoint{4.866443in}{2.346695in}}%
\pgfusepath{stroke}%
\end{pgfscope}%
\begin{pgfscope}%
\pgfpathrectangle{\pgfqpoint{0.664741in}{1.389617in}}{\pgfqpoint{4.922705in}{1.759717in}}%
\pgfusepath{clip}%
\pgfsetbuttcap%
\pgfsetroundjoin%
\pgfsetlinewidth{1.505625pt}%
\definecolor{currentstroke}{rgb}{0.313725,0.317647,0.309804}%
\pgfsetstrokecolor{currentstroke}%
\pgfsetstrokeopacity{0.900000}%
\pgfsetdash{}{0pt}%
\pgfpathmoveto{\pgfqpoint{4.866443in}{2.608543in}}%
\pgfpathlineto{\pgfqpoint{5.115065in}{2.608543in}}%
\pgfusepath{stroke}%
\end{pgfscope}%
\begin{pgfscope}%
\pgfpathrectangle{\pgfqpoint{0.664741in}{1.389617in}}{\pgfqpoint{4.922705in}{1.759717in}}%
\pgfusepath{clip}%
\pgfsetbuttcap%
\pgfsetroundjoin%
\pgfsetlinewidth{1.505625pt}%
\definecolor{currentstroke}{rgb}{0.313725,0.317647,0.309804}%
\pgfsetstrokecolor{currentstroke}%
\pgfsetstrokeopacity{0.900000}%
\pgfsetdash{}{0pt}%
\pgfpathmoveto{\pgfqpoint{5.115065in}{2.958940in}}%
\pgfpathlineto{\pgfqpoint{5.363686in}{2.958940in}}%
\pgfusepath{stroke}%
\end{pgfscope}%
\begin{pgfscope}%
\pgfpathrectangle{\pgfqpoint{0.664741in}{1.389617in}}{\pgfqpoint{4.922705in}{1.759717in}}%
\pgfusepath{clip}%
\pgfsetbuttcap%
\pgfsetroundjoin%
\pgfsetlinewidth{1.505625pt}%
\definecolor{currentstroke}{rgb}{0.313725,0.317647,0.309804}%
\pgfsetstrokecolor{currentstroke}%
\pgfsetstrokeopacity{0.900000}%
\pgfsetdash{}{0pt}%
\pgfpathmoveto{\pgfqpoint{1.012811in}{2.814328in}}%
\pgfpathlineto{\pgfqpoint{1.012811in}{2.945722in}}%
\pgfusepath{stroke}%
\end{pgfscope}%
\begin{pgfscope}%
\pgfpathrectangle{\pgfqpoint{0.664741in}{1.389617in}}{\pgfqpoint{4.922705in}{1.759717in}}%
\pgfusepath{clip}%
\pgfsetbuttcap%
\pgfsetroundjoin%
\pgfsetlinewidth{1.505625pt}%
\definecolor{currentstroke}{rgb}{0.313725,0.317647,0.309804}%
\pgfsetstrokecolor{currentstroke}%
\pgfsetstrokeopacity{0.900000}%
\pgfsetdash{}{0pt}%
\pgfpathmoveto{\pgfqpoint{1.261432in}{2.685015in}}%
\pgfpathlineto{\pgfqpoint{1.261432in}{2.743674in}}%
\pgfusepath{stroke}%
\end{pgfscope}%
\begin{pgfscope}%
\pgfpathrectangle{\pgfqpoint{0.664741in}{1.389617in}}{\pgfqpoint{4.922705in}{1.759717in}}%
\pgfusepath{clip}%
\pgfsetbuttcap%
\pgfsetroundjoin%
\pgfsetlinewidth{1.505625pt}%
\definecolor{currentstroke}{rgb}{0.313725,0.317647,0.309804}%
\pgfsetstrokecolor{currentstroke}%
\pgfsetstrokeopacity{0.900000}%
\pgfsetdash{}{0pt}%
\pgfpathmoveto{\pgfqpoint{1.510054in}{2.543225in}}%
\pgfpathlineto{\pgfqpoint{1.510054in}{2.576122in}}%
\pgfusepath{stroke}%
\end{pgfscope}%
\begin{pgfscope}%
\pgfpathrectangle{\pgfqpoint{0.664741in}{1.389617in}}{\pgfqpoint{4.922705in}{1.759717in}}%
\pgfusepath{clip}%
\pgfsetbuttcap%
\pgfsetroundjoin%
\pgfsetlinewidth{1.505625pt}%
\definecolor{currentstroke}{rgb}{0.313725,0.317647,0.309804}%
\pgfsetstrokecolor{currentstroke}%
\pgfsetstrokeopacity{0.900000}%
\pgfsetdash{}{0pt}%
\pgfpathmoveto{\pgfqpoint{1.758675in}{2.427171in}}%
\pgfpathlineto{\pgfqpoint{1.758675in}{2.448503in}}%
\pgfusepath{stroke}%
\end{pgfscope}%
\begin{pgfscope}%
\pgfpathrectangle{\pgfqpoint{0.664741in}{1.389617in}}{\pgfqpoint{4.922705in}{1.759717in}}%
\pgfusepath{clip}%
\pgfsetbuttcap%
\pgfsetroundjoin%
\pgfsetlinewidth{1.505625pt}%
\definecolor{currentstroke}{rgb}{0.313725,0.317647,0.309804}%
\pgfsetstrokecolor{currentstroke}%
\pgfsetstrokeopacity{0.900000}%
\pgfsetdash{}{0pt}%
\pgfpathmoveto{\pgfqpoint{2.007297in}{2.315996in}}%
\pgfpathlineto{\pgfqpoint{2.007297in}{2.331770in}}%
\pgfusepath{stroke}%
\end{pgfscope}%
\begin{pgfscope}%
\pgfpathrectangle{\pgfqpoint{0.664741in}{1.389617in}}{\pgfqpoint{4.922705in}{1.759717in}}%
\pgfusepath{clip}%
\pgfsetbuttcap%
\pgfsetroundjoin%
\pgfsetlinewidth{1.505625pt}%
\definecolor{currentstroke}{rgb}{0.313725,0.317647,0.309804}%
\pgfsetstrokecolor{currentstroke}%
\pgfsetstrokeopacity{0.900000}%
\pgfsetdash{}{0pt}%
\pgfpathmoveto{\pgfqpoint{2.255918in}{2.217588in}}%
\pgfpathlineto{\pgfqpoint{2.255918in}{2.230733in}}%
\pgfusepath{stroke}%
\end{pgfscope}%
\begin{pgfscope}%
\pgfpathrectangle{\pgfqpoint{0.664741in}{1.389617in}}{\pgfqpoint{4.922705in}{1.759717in}}%
\pgfusepath{clip}%
\pgfsetbuttcap%
\pgfsetroundjoin%
\pgfsetlinewidth{1.505625pt}%
\definecolor{currentstroke}{rgb}{0.313725,0.317647,0.309804}%
\pgfsetstrokecolor{currentstroke}%
\pgfsetstrokeopacity{0.900000}%
\pgfsetdash{}{0pt}%
\pgfpathmoveto{\pgfqpoint{2.504539in}{2.088007in}}%
\pgfpathlineto{\pgfqpoint{2.504539in}{2.100000in}}%
\pgfusepath{stroke}%
\end{pgfscope}%
\begin{pgfscope}%
\pgfpathrectangle{\pgfqpoint{0.664741in}{1.389617in}}{\pgfqpoint{4.922705in}{1.759717in}}%
\pgfusepath{clip}%
\pgfsetbuttcap%
\pgfsetroundjoin%
\pgfsetlinewidth{1.505625pt}%
\definecolor{currentstroke}{rgb}{0.313725,0.317647,0.309804}%
\pgfsetstrokecolor{currentstroke}%
\pgfsetstrokeopacity{0.900000}%
\pgfsetdash{}{0pt}%
\pgfpathmoveto{\pgfqpoint{2.753161in}{1.971325in}}%
\pgfpathlineto{\pgfqpoint{2.753161in}{1.982133in}}%
\pgfusepath{stroke}%
\end{pgfscope}%
\begin{pgfscope}%
\pgfpathrectangle{\pgfqpoint{0.664741in}{1.389617in}}{\pgfqpoint{4.922705in}{1.759717in}}%
\pgfusepath{clip}%
\pgfsetbuttcap%
\pgfsetroundjoin%
\pgfsetlinewidth{1.505625pt}%
\definecolor{currentstroke}{rgb}{0.313725,0.317647,0.309804}%
\pgfsetstrokecolor{currentstroke}%
\pgfsetstrokeopacity{0.900000}%
\pgfsetdash{}{0pt}%
\pgfpathmoveto{\pgfqpoint{3.001782in}{1.910498in}}%
\pgfpathlineto{\pgfqpoint{3.001782in}{1.920861in}}%
\pgfusepath{stroke}%
\end{pgfscope}%
\begin{pgfscope}%
\pgfpathrectangle{\pgfqpoint{0.664741in}{1.389617in}}{\pgfqpoint{4.922705in}{1.759717in}}%
\pgfusepath{clip}%
\pgfsetbuttcap%
\pgfsetroundjoin%
\pgfsetlinewidth{1.505625pt}%
\definecolor{currentstroke}{rgb}{0.313725,0.317647,0.309804}%
\pgfsetstrokecolor{currentstroke}%
\pgfsetstrokeopacity{0.900000}%
\pgfsetdash{}{0pt}%
\pgfpathmoveto{\pgfqpoint{3.250404in}{1.915697in}}%
\pgfpathlineto{\pgfqpoint{3.250404in}{1.927216in}}%
\pgfusepath{stroke}%
\end{pgfscope}%
\begin{pgfscope}%
\pgfpathrectangle{\pgfqpoint{0.664741in}{1.389617in}}{\pgfqpoint{4.922705in}{1.759717in}}%
\pgfusepath{clip}%
\pgfsetbuttcap%
\pgfsetroundjoin%
\pgfsetlinewidth{1.505625pt}%
\definecolor{currentstroke}{rgb}{0.313725,0.317647,0.309804}%
\pgfsetstrokecolor{currentstroke}%
\pgfsetstrokeopacity{0.900000}%
\pgfsetdash{}{0pt}%
\pgfpathmoveto{\pgfqpoint{3.499025in}{1.879960in}}%
\pgfpathlineto{\pgfqpoint{3.499025in}{1.893082in}}%
\pgfusepath{stroke}%
\end{pgfscope}%
\begin{pgfscope}%
\pgfpathrectangle{\pgfqpoint{0.664741in}{1.389617in}}{\pgfqpoint{4.922705in}{1.759717in}}%
\pgfusepath{clip}%
\pgfsetbuttcap%
\pgfsetroundjoin%
\pgfsetlinewidth{1.505625pt}%
\definecolor{currentstroke}{rgb}{0.313725,0.317647,0.309804}%
\pgfsetstrokecolor{currentstroke}%
\pgfsetstrokeopacity{0.900000}%
\pgfsetdash{}{0pt}%
\pgfpathmoveto{\pgfqpoint{3.747647in}{1.858893in}}%
\pgfpathlineto{\pgfqpoint{3.747647in}{1.873116in}}%
\pgfusepath{stroke}%
\end{pgfscope}%
\begin{pgfscope}%
\pgfpathrectangle{\pgfqpoint{0.664741in}{1.389617in}}{\pgfqpoint{4.922705in}{1.759717in}}%
\pgfusepath{clip}%
\pgfsetbuttcap%
\pgfsetroundjoin%
\pgfsetlinewidth{1.505625pt}%
\definecolor{currentstroke}{rgb}{0.313725,0.317647,0.309804}%
\pgfsetstrokecolor{currentstroke}%
\pgfsetstrokeopacity{0.900000}%
\pgfsetdash{}{0pt}%
\pgfpathmoveto{\pgfqpoint{3.996268in}{1.868215in}}%
\pgfpathlineto{\pgfqpoint{3.996268in}{1.886928in}}%
\pgfusepath{stroke}%
\end{pgfscope}%
\begin{pgfscope}%
\pgfpathrectangle{\pgfqpoint{0.664741in}{1.389617in}}{\pgfqpoint{4.922705in}{1.759717in}}%
\pgfusepath{clip}%
\pgfsetbuttcap%
\pgfsetroundjoin%
\pgfsetlinewidth{1.505625pt}%
\definecolor{currentstroke}{rgb}{0.313725,0.317647,0.309804}%
\pgfsetstrokecolor{currentstroke}%
\pgfsetstrokeopacity{0.900000}%
\pgfsetdash{}{0pt}%
\pgfpathmoveto{\pgfqpoint{4.244890in}{1.955799in}}%
\pgfpathlineto{\pgfqpoint{4.244890in}{1.981813in}}%
\pgfusepath{stroke}%
\end{pgfscope}%
\begin{pgfscope}%
\pgfpathrectangle{\pgfqpoint{0.664741in}{1.389617in}}{\pgfqpoint{4.922705in}{1.759717in}}%
\pgfusepath{clip}%
\pgfsetbuttcap%
\pgfsetroundjoin%
\pgfsetlinewidth{1.505625pt}%
\definecolor{currentstroke}{rgb}{0.313725,0.317647,0.309804}%
\pgfsetstrokecolor{currentstroke}%
\pgfsetstrokeopacity{0.900000}%
\pgfsetdash{}{0pt}%
\pgfpathmoveto{\pgfqpoint{4.493511in}{2.110796in}}%
\pgfpathlineto{\pgfqpoint{4.493511in}{2.147134in}}%
\pgfusepath{stroke}%
\end{pgfscope}%
\begin{pgfscope}%
\pgfpathrectangle{\pgfqpoint{0.664741in}{1.389617in}}{\pgfqpoint{4.922705in}{1.759717in}}%
\pgfusepath{clip}%
\pgfsetbuttcap%
\pgfsetroundjoin%
\pgfsetlinewidth{1.505625pt}%
\definecolor{currentstroke}{rgb}{0.313725,0.317647,0.309804}%
\pgfsetstrokecolor{currentstroke}%
\pgfsetstrokeopacity{0.900000}%
\pgfsetdash{}{0pt}%
\pgfpathmoveto{\pgfqpoint{4.742132in}{2.322512in}}%
\pgfpathlineto{\pgfqpoint{4.742132in}{2.369090in}}%
\pgfusepath{stroke}%
\end{pgfscope}%
\begin{pgfscope}%
\pgfpathrectangle{\pgfqpoint{0.664741in}{1.389617in}}{\pgfqpoint{4.922705in}{1.759717in}}%
\pgfusepath{clip}%
\pgfsetbuttcap%
\pgfsetroundjoin%
\pgfsetlinewidth{1.505625pt}%
\definecolor{currentstroke}{rgb}{0.313725,0.317647,0.309804}%
\pgfsetstrokecolor{currentstroke}%
\pgfsetstrokeopacity{0.900000}%
\pgfsetdash{}{0pt}%
\pgfpathmoveto{\pgfqpoint{4.990754in}{2.580057in}}%
\pgfpathlineto{\pgfqpoint{4.990754in}{2.639852in}}%
\pgfusepath{stroke}%
\end{pgfscope}%
\begin{pgfscope}%
\pgfpathrectangle{\pgfqpoint{0.664741in}{1.389617in}}{\pgfqpoint{4.922705in}{1.759717in}}%
\pgfusepath{clip}%
\pgfsetbuttcap%
\pgfsetroundjoin%
\pgfsetlinewidth{1.505625pt}%
\definecolor{currentstroke}{rgb}{0.313725,0.317647,0.309804}%
\pgfsetstrokecolor{currentstroke}%
\pgfsetstrokeopacity{0.900000}%
\pgfsetdash{}{0pt}%
\pgfpathmoveto{\pgfqpoint{5.239375in}{2.920196in}}%
\pgfpathlineto{\pgfqpoint{5.239375in}{2.999856in}}%
\pgfusepath{stroke}%
\end{pgfscope}%
\begin{pgfscope}%
\pgfpathrectangle{\pgfqpoint{0.664741in}{1.389617in}}{\pgfqpoint{4.922705in}{1.759717in}}%
\pgfusepath{clip}%
\pgfsetbuttcap%
\pgfsetroundjoin%
\pgfsetlinewidth{1.505625pt}%
\definecolor{currentstroke}{rgb}{0.949020,0.372549,0.360784}%
\pgfsetstrokecolor{currentstroke}%
\pgfsetstrokeopacity{0.900000}%
\pgfsetdash{}{0pt}%
\pgfpathmoveto{\pgfqpoint{0.888500in}{1.502716in}}%
\pgfpathlineto{\pgfqpoint{1.137121in}{1.502716in}}%
\pgfusepath{stroke}%
\end{pgfscope}%
\begin{pgfscope}%
\pgfpathrectangle{\pgfqpoint{0.664741in}{1.389617in}}{\pgfqpoint{4.922705in}{1.759717in}}%
\pgfusepath{clip}%
\pgfsetbuttcap%
\pgfsetroundjoin%
\pgfsetlinewidth{1.505625pt}%
\definecolor{currentstroke}{rgb}{0.949020,0.372549,0.360784}%
\pgfsetstrokecolor{currentstroke}%
\pgfsetstrokeopacity{0.900000}%
\pgfsetdash{}{0pt}%
\pgfpathmoveto{\pgfqpoint{1.137121in}{1.525704in}}%
\pgfpathlineto{\pgfqpoint{1.385743in}{1.525704in}}%
\pgfusepath{stroke}%
\end{pgfscope}%
\begin{pgfscope}%
\pgfpathrectangle{\pgfqpoint{0.664741in}{1.389617in}}{\pgfqpoint{4.922705in}{1.759717in}}%
\pgfusepath{clip}%
\pgfsetbuttcap%
\pgfsetroundjoin%
\pgfsetlinewidth{1.505625pt}%
\definecolor{currentstroke}{rgb}{0.949020,0.372549,0.360784}%
\pgfsetstrokecolor{currentstroke}%
\pgfsetstrokeopacity{0.900000}%
\pgfsetdash{}{0pt}%
\pgfpathmoveto{\pgfqpoint{1.385743in}{1.561847in}}%
\pgfpathlineto{\pgfqpoint{1.634364in}{1.561847in}}%
\pgfusepath{stroke}%
\end{pgfscope}%
\begin{pgfscope}%
\pgfpathrectangle{\pgfqpoint{0.664741in}{1.389617in}}{\pgfqpoint{4.922705in}{1.759717in}}%
\pgfusepath{clip}%
\pgfsetbuttcap%
\pgfsetroundjoin%
\pgfsetlinewidth{1.505625pt}%
\definecolor{currentstroke}{rgb}{0.949020,0.372549,0.360784}%
\pgfsetstrokecolor{currentstroke}%
\pgfsetstrokeopacity{0.900000}%
\pgfsetdash{}{0pt}%
\pgfpathmoveto{\pgfqpoint{1.634364in}{1.607916in}}%
\pgfpathlineto{\pgfqpoint{1.882986in}{1.607916in}}%
\pgfusepath{stroke}%
\end{pgfscope}%
\begin{pgfscope}%
\pgfpathrectangle{\pgfqpoint{0.664741in}{1.389617in}}{\pgfqpoint{4.922705in}{1.759717in}}%
\pgfusepath{clip}%
\pgfsetbuttcap%
\pgfsetroundjoin%
\pgfsetlinewidth{1.505625pt}%
\definecolor{currentstroke}{rgb}{0.949020,0.372549,0.360784}%
\pgfsetstrokecolor{currentstroke}%
\pgfsetstrokeopacity{0.900000}%
\pgfsetdash{}{0pt}%
\pgfpathmoveto{\pgfqpoint{1.882986in}{1.690719in}}%
\pgfpathlineto{\pgfqpoint{2.131607in}{1.690719in}}%
\pgfusepath{stroke}%
\end{pgfscope}%
\begin{pgfscope}%
\pgfpathrectangle{\pgfqpoint{0.664741in}{1.389617in}}{\pgfqpoint{4.922705in}{1.759717in}}%
\pgfusepath{clip}%
\pgfsetbuttcap%
\pgfsetroundjoin%
\pgfsetlinewidth{1.505625pt}%
\definecolor{currentstroke}{rgb}{0.949020,0.372549,0.360784}%
\pgfsetstrokecolor{currentstroke}%
\pgfsetstrokeopacity{0.900000}%
\pgfsetdash{}{0pt}%
\pgfpathmoveto{\pgfqpoint{2.131607in}{1.792934in}}%
\pgfpathlineto{\pgfqpoint{2.380229in}{1.792934in}}%
\pgfusepath{stroke}%
\end{pgfscope}%
\begin{pgfscope}%
\pgfpathrectangle{\pgfqpoint{0.664741in}{1.389617in}}{\pgfqpoint{4.922705in}{1.759717in}}%
\pgfusepath{clip}%
\pgfsetbuttcap%
\pgfsetroundjoin%
\pgfsetlinewidth{1.505625pt}%
\definecolor{currentstroke}{rgb}{0.949020,0.372549,0.360784}%
\pgfsetstrokecolor{currentstroke}%
\pgfsetstrokeopacity{0.900000}%
\pgfsetdash{}{0pt}%
\pgfpathmoveto{\pgfqpoint{2.380229in}{1.843062in}}%
\pgfpathlineto{\pgfqpoint{2.628850in}{1.843062in}}%
\pgfusepath{stroke}%
\end{pgfscope}%
\begin{pgfscope}%
\pgfpathrectangle{\pgfqpoint{0.664741in}{1.389617in}}{\pgfqpoint{4.922705in}{1.759717in}}%
\pgfusepath{clip}%
\pgfsetbuttcap%
\pgfsetroundjoin%
\pgfsetlinewidth{1.505625pt}%
\definecolor{currentstroke}{rgb}{0.949020,0.372549,0.360784}%
\pgfsetstrokecolor{currentstroke}%
\pgfsetstrokeopacity{0.900000}%
\pgfsetdash{}{0pt}%
\pgfpathmoveto{\pgfqpoint{2.628850in}{1.812470in}}%
\pgfpathlineto{\pgfqpoint{2.877472in}{1.812470in}}%
\pgfusepath{stroke}%
\end{pgfscope}%
\begin{pgfscope}%
\pgfpathrectangle{\pgfqpoint{0.664741in}{1.389617in}}{\pgfqpoint{4.922705in}{1.759717in}}%
\pgfusepath{clip}%
\pgfsetbuttcap%
\pgfsetroundjoin%
\pgfsetlinewidth{1.505625pt}%
\definecolor{currentstroke}{rgb}{0.949020,0.372549,0.360784}%
\pgfsetstrokecolor{currentstroke}%
\pgfsetstrokeopacity{0.900000}%
\pgfsetdash{}{0pt}%
\pgfpathmoveto{\pgfqpoint{2.877472in}{1.819471in}}%
\pgfpathlineto{\pgfqpoint{3.126093in}{1.819471in}}%
\pgfusepath{stroke}%
\end{pgfscope}%
\begin{pgfscope}%
\pgfpathrectangle{\pgfqpoint{0.664741in}{1.389617in}}{\pgfqpoint{4.922705in}{1.759717in}}%
\pgfusepath{clip}%
\pgfsetbuttcap%
\pgfsetroundjoin%
\pgfsetlinewidth{1.505625pt}%
\definecolor{currentstroke}{rgb}{0.949020,0.372549,0.360784}%
\pgfsetstrokecolor{currentstroke}%
\pgfsetstrokeopacity{0.900000}%
\pgfsetdash{}{0pt}%
\pgfpathmoveto{\pgfqpoint{3.126093in}{1.838381in}}%
\pgfpathlineto{\pgfqpoint{3.374715in}{1.838381in}}%
\pgfusepath{stroke}%
\end{pgfscope}%
\begin{pgfscope}%
\pgfpathrectangle{\pgfqpoint{0.664741in}{1.389617in}}{\pgfqpoint{4.922705in}{1.759717in}}%
\pgfusepath{clip}%
\pgfsetbuttcap%
\pgfsetroundjoin%
\pgfsetlinewidth{1.505625pt}%
\definecolor{currentstroke}{rgb}{0.949020,0.372549,0.360784}%
\pgfsetstrokecolor{currentstroke}%
\pgfsetstrokeopacity{0.900000}%
\pgfsetdash{}{0pt}%
\pgfpathmoveto{\pgfqpoint{3.374715in}{1.849312in}}%
\pgfpathlineto{\pgfqpoint{3.623336in}{1.849312in}}%
\pgfusepath{stroke}%
\end{pgfscope}%
\begin{pgfscope}%
\pgfpathrectangle{\pgfqpoint{0.664741in}{1.389617in}}{\pgfqpoint{4.922705in}{1.759717in}}%
\pgfusepath{clip}%
\pgfsetbuttcap%
\pgfsetroundjoin%
\pgfsetlinewidth{1.505625pt}%
\definecolor{currentstroke}{rgb}{0.949020,0.372549,0.360784}%
\pgfsetstrokecolor{currentstroke}%
\pgfsetstrokeopacity{0.900000}%
\pgfsetdash{}{0pt}%
\pgfpathmoveto{\pgfqpoint{3.623336in}{1.904850in}}%
\pgfpathlineto{\pgfqpoint{3.871957in}{1.904850in}}%
\pgfusepath{stroke}%
\end{pgfscope}%
\begin{pgfscope}%
\pgfpathrectangle{\pgfqpoint{0.664741in}{1.389617in}}{\pgfqpoint{4.922705in}{1.759717in}}%
\pgfusepath{clip}%
\pgfsetbuttcap%
\pgfsetroundjoin%
\pgfsetlinewidth{1.505625pt}%
\definecolor{currentstroke}{rgb}{0.949020,0.372549,0.360784}%
\pgfsetstrokecolor{currentstroke}%
\pgfsetstrokeopacity{0.900000}%
\pgfsetdash{}{0pt}%
\pgfpathmoveto{\pgfqpoint{3.871957in}{2.025758in}}%
\pgfpathlineto{\pgfqpoint{4.120579in}{2.025758in}}%
\pgfusepath{stroke}%
\end{pgfscope}%
\begin{pgfscope}%
\pgfpathrectangle{\pgfqpoint{0.664741in}{1.389617in}}{\pgfqpoint{4.922705in}{1.759717in}}%
\pgfusepath{clip}%
\pgfsetbuttcap%
\pgfsetroundjoin%
\pgfsetlinewidth{1.505625pt}%
\definecolor{currentstroke}{rgb}{0.949020,0.372549,0.360784}%
\pgfsetstrokecolor{currentstroke}%
\pgfsetstrokeopacity{0.900000}%
\pgfsetdash{}{0pt}%
\pgfpathmoveto{\pgfqpoint{4.120579in}{2.266408in}}%
\pgfpathlineto{\pgfqpoint{4.369200in}{2.266408in}}%
\pgfusepath{stroke}%
\end{pgfscope}%
\begin{pgfscope}%
\pgfpathrectangle{\pgfqpoint{0.664741in}{1.389617in}}{\pgfqpoint{4.922705in}{1.759717in}}%
\pgfusepath{clip}%
\pgfsetbuttcap%
\pgfsetroundjoin%
\pgfsetlinewidth{1.505625pt}%
\definecolor{currentstroke}{rgb}{0.949020,0.372549,0.360784}%
\pgfsetstrokecolor{currentstroke}%
\pgfsetstrokeopacity{0.900000}%
\pgfsetdash{}{0pt}%
\pgfpathmoveto{\pgfqpoint{4.369200in}{2.558696in}}%
\pgfpathlineto{\pgfqpoint{4.617822in}{2.558696in}}%
\pgfusepath{stroke}%
\end{pgfscope}%
\begin{pgfscope}%
\pgfpathrectangle{\pgfqpoint{0.664741in}{1.389617in}}{\pgfqpoint{4.922705in}{1.759717in}}%
\pgfusepath{clip}%
\pgfsetbuttcap%
\pgfsetroundjoin%
\pgfsetlinewidth{1.505625pt}%
\definecolor{currentstroke}{rgb}{0.949020,0.372549,0.360784}%
\pgfsetstrokecolor{currentstroke}%
\pgfsetstrokeopacity{0.900000}%
\pgfsetdash{}{0pt}%
\pgfpathmoveto{\pgfqpoint{4.617822in}{2.789303in}}%
\pgfpathlineto{\pgfqpoint{4.866443in}{2.789303in}}%
\pgfusepath{stroke}%
\end{pgfscope}%
\begin{pgfscope}%
\pgfpathrectangle{\pgfqpoint{0.664741in}{1.389617in}}{\pgfqpoint{4.922705in}{1.759717in}}%
\pgfusepath{clip}%
\pgfsetbuttcap%
\pgfsetroundjoin%
\pgfsetlinewidth{1.505625pt}%
\definecolor{currentstroke}{rgb}{0.949020,0.372549,0.360784}%
\pgfsetstrokecolor{currentstroke}%
\pgfsetstrokeopacity{0.900000}%
\pgfsetdash{}{0pt}%
\pgfpathmoveto{\pgfqpoint{4.866443in}{2.949469in}}%
\pgfpathlineto{\pgfqpoint{5.115065in}{2.949469in}}%
\pgfusepath{stroke}%
\end{pgfscope}%
\begin{pgfscope}%
\pgfpathrectangle{\pgfqpoint{0.664741in}{1.389617in}}{\pgfqpoint{4.922705in}{1.759717in}}%
\pgfusepath{clip}%
\pgfsetbuttcap%
\pgfsetroundjoin%
\pgfsetlinewidth{1.505625pt}%
\definecolor{currentstroke}{rgb}{0.949020,0.372549,0.360784}%
\pgfsetstrokecolor{currentstroke}%
\pgfsetstrokeopacity{0.900000}%
\pgfsetdash{}{0pt}%
\pgfpathmoveto{\pgfqpoint{5.115065in}{3.019010in}}%
\pgfpathlineto{\pgfqpoint{5.363686in}{3.019010in}}%
\pgfusepath{stroke}%
\end{pgfscope}%
\begin{pgfscope}%
\pgfpathrectangle{\pgfqpoint{0.664741in}{1.389617in}}{\pgfqpoint{4.922705in}{1.759717in}}%
\pgfusepath{clip}%
\pgfsetbuttcap%
\pgfsetroundjoin%
\pgfsetlinewidth{1.505625pt}%
\definecolor{currentstroke}{rgb}{0.949020,0.372549,0.360784}%
\pgfsetstrokecolor{currentstroke}%
\pgfsetstrokeopacity{0.900000}%
\pgfsetdash{}{0pt}%
\pgfpathmoveto{\pgfqpoint{1.012811in}{1.469604in}}%
\pgfpathlineto{\pgfqpoint{1.012811in}{1.532845in}}%
\pgfusepath{stroke}%
\end{pgfscope}%
\begin{pgfscope}%
\pgfpathrectangle{\pgfqpoint{0.664741in}{1.389617in}}{\pgfqpoint{4.922705in}{1.759717in}}%
\pgfusepath{clip}%
\pgfsetbuttcap%
\pgfsetroundjoin%
\pgfsetlinewidth{1.505625pt}%
\definecolor{currentstroke}{rgb}{0.949020,0.372549,0.360784}%
\pgfsetstrokecolor{currentstroke}%
\pgfsetstrokeopacity{0.900000}%
\pgfsetdash{}{0pt}%
\pgfpathmoveto{\pgfqpoint{1.261432in}{1.509142in}}%
\pgfpathlineto{\pgfqpoint{1.261432in}{1.542650in}}%
\pgfusepath{stroke}%
\end{pgfscope}%
\begin{pgfscope}%
\pgfpathrectangle{\pgfqpoint{0.664741in}{1.389617in}}{\pgfqpoint{4.922705in}{1.759717in}}%
\pgfusepath{clip}%
\pgfsetbuttcap%
\pgfsetroundjoin%
\pgfsetlinewidth{1.505625pt}%
\definecolor{currentstroke}{rgb}{0.949020,0.372549,0.360784}%
\pgfsetstrokecolor{currentstroke}%
\pgfsetstrokeopacity{0.900000}%
\pgfsetdash{}{0pt}%
\pgfpathmoveto{\pgfqpoint{1.510054in}{1.551997in}}%
\pgfpathlineto{\pgfqpoint{1.510054in}{1.571054in}}%
\pgfusepath{stroke}%
\end{pgfscope}%
\begin{pgfscope}%
\pgfpathrectangle{\pgfqpoint{0.664741in}{1.389617in}}{\pgfqpoint{4.922705in}{1.759717in}}%
\pgfusepath{clip}%
\pgfsetbuttcap%
\pgfsetroundjoin%
\pgfsetlinewidth{1.505625pt}%
\definecolor{currentstroke}{rgb}{0.949020,0.372549,0.360784}%
\pgfsetstrokecolor{currentstroke}%
\pgfsetstrokeopacity{0.900000}%
\pgfsetdash{}{0pt}%
\pgfpathmoveto{\pgfqpoint{1.758675in}{1.600859in}}%
\pgfpathlineto{\pgfqpoint{1.758675in}{1.615567in}}%
\pgfusepath{stroke}%
\end{pgfscope}%
\begin{pgfscope}%
\pgfpathrectangle{\pgfqpoint{0.664741in}{1.389617in}}{\pgfqpoint{4.922705in}{1.759717in}}%
\pgfusepath{clip}%
\pgfsetbuttcap%
\pgfsetroundjoin%
\pgfsetlinewidth{1.505625pt}%
\definecolor{currentstroke}{rgb}{0.949020,0.372549,0.360784}%
\pgfsetstrokecolor{currentstroke}%
\pgfsetstrokeopacity{0.900000}%
\pgfsetdash{}{0pt}%
\pgfpathmoveto{\pgfqpoint{2.007297in}{1.685004in}}%
\pgfpathlineto{\pgfqpoint{2.007297in}{1.697054in}}%
\pgfusepath{stroke}%
\end{pgfscope}%
\begin{pgfscope}%
\pgfpathrectangle{\pgfqpoint{0.664741in}{1.389617in}}{\pgfqpoint{4.922705in}{1.759717in}}%
\pgfusepath{clip}%
\pgfsetbuttcap%
\pgfsetroundjoin%
\pgfsetlinewidth{1.505625pt}%
\definecolor{currentstroke}{rgb}{0.949020,0.372549,0.360784}%
\pgfsetstrokecolor{currentstroke}%
\pgfsetstrokeopacity{0.900000}%
\pgfsetdash{}{0pt}%
\pgfpathmoveto{\pgfqpoint{2.255918in}{1.787532in}}%
\pgfpathlineto{\pgfqpoint{2.255918in}{1.798124in}}%
\pgfusepath{stroke}%
\end{pgfscope}%
\begin{pgfscope}%
\pgfpathrectangle{\pgfqpoint{0.664741in}{1.389617in}}{\pgfqpoint{4.922705in}{1.759717in}}%
\pgfusepath{clip}%
\pgfsetbuttcap%
\pgfsetroundjoin%
\pgfsetlinewidth{1.505625pt}%
\definecolor{currentstroke}{rgb}{0.949020,0.372549,0.360784}%
\pgfsetstrokecolor{currentstroke}%
\pgfsetstrokeopacity{0.900000}%
\pgfsetdash{}{0pt}%
\pgfpathmoveto{\pgfqpoint{2.504539in}{1.838122in}}%
\pgfpathlineto{\pgfqpoint{2.504539in}{1.847400in}}%
\pgfusepath{stroke}%
\end{pgfscope}%
\begin{pgfscope}%
\pgfpathrectangle{\pgfqpoint{0.664741in}{1.389617in}}{\pgfqpoint{4.922705in}{1.759717in}}%
\pgfusepath{clip}%
\pgfsetbuttcap%
\pgfsetroundjoin%
\pgfsetlinewidth{1.505625pt}%
\definecolor{currentstroke}{rgb}{0.949020,0.372549,0.360784}%
\pgfsetstrokecolor{currentstroke}%
\pgfsetstrokeopacity{0.900000}%
\pgfsetdash{}{0pt}%
\pgfpathmoveto{\pgfqpoint{2.753161in}{1.807711in}}%
\pgfpathlineto{\pgfqpoint{2.753161in}{1.816967in}}%
\pgfusepath{stroke}%
\end{pgfscope}%
\begin{pgfscope}%
\pgfpathrectangle{\pgfqpoint{0.664741in}{1.389617in}}{\pgfqpoint{4.922705in}{1.759717in}}%
\pgfusepath{clip}%
\pgfsetbuttcap%
\pgfsetroundjoin%
\pgfsetlinewidth{1.505625pt}%
\definecolor{currentstroke}{rgb}{0.949020,0.372549,0.360784}%
\pgfsetstrokecolor{currentstroke}%
\pgfsetstrokeopacity{0.900000}%
\pgfsetdash{}{0pt}%
\pgfpathmoveto{\pgfqpoint{3.001782in}{1.815004in}}%
\pgfpathlineto{\pgfqpoint{3.001782in}{1.824267in}}%
\pgfusepath{stroke}%
\end{pgfscope}%
\begin{pgfscope}%
\pgfpathrectangle{\pgfqpoint{0.664741in}{1.389617in}}{\pgfqpoint{4.922705in}{1.759717in}}%
\pgfusepath{clip}%
\pgfsetbuttcap%
\pgfsetroundjoin%
\pgfsetlinewidth{1.505625pt}%
\definecolor{currentstroke}{rgb}{0.949020,0.372549,0.360784}%
\pgfsetstrokecolor{currentstroke}%
\pgfsetstrokeopacity{0.900000}%
\pgfsetdash{}{0pt}%
\pgfpathmoveto{\pgfqpoint{3.250404in}{1.832851in}}%
\pgfpathlineto{\pgfqpoint{3.250404in}{1.843511in}}%
\pgfusepath{stroke}%
\end{pgfscope}%
\begin{pgfscope}%
\pgfpathrectangle{\pgfqpoint{0.664741in}{1.389617in}}{\pgfqpoint{4.922705in}{1.759717in}}%
\pgfusepath{clip}%
\pgfsetbuttcap%
\pgfsetroundjoin%
\pgfsetlinewidth{1.505625pt}%
\definecolor{currentstroke}{rgb}{0.949020,0.372549,0.360784}%
\pgfsetstrokecolor{currentstroke}%
\pgfsetstrokeopacity{0.900000}%
\pgfsetdash{}{0pt}%
\pgfpathmoveto{\pgfqpoint{3.499025in}{1.843107in}}%
\pgfpathlineto{\pgfqpoint{3.499025in}{1.855880in}}%
\pgfusepath{stroke}%
\end{pgfscope}%
\begin{pgfscope}%
\pgfpathrectangle{\pgfqpoint{0.664741in}{1.389617in}}{\pgfqpoint{4.922705in}{1.759717in}}%
\pgfusepath{clip}%
\pgfsetbuttcap%
\pgfsetroundjoin%
\pgfsetlinewidth{1.505625pt}%
\definecolor{currentstroke}{rgb}{0.949020,0.372549,0.360784}%
\pgfsetstrokecolor{currentstroke}%
\pgfsetstrokeopacity{0.900000}%
\pgfsetdash{}{0pt}%
\pgfpathmoveto{\pgfqpoint{3.747647in}{1.896963in}}%
\pgfpathlineto{\pgfqpoint{3.747647in}{1.912785in}}%
\pgfusepath{stroke}%
\end{pgfscope}%
\begin{pgfscope}%
\pgfpathrectangle{\pgfqpoint{0.664741in}{1.389617in}}{\pgfqpoint{4.922705in}{1.759717in}}%
\pgfusepath{clip}%
\pgfsetbuttcap%
\pgfsetroundjoin%
\pgfsetlinewidth{1.505625pt}%
\definecolor{currentstroke}{rgb}{0.949020,0.372549,0.360784}%
\pgfsetstrokecolor{currentstroke}%
\pgfsetstrokeopacity{0.900000}%
\pgfsetdash{}{0pt}%
\pgfpathmoveto{\pgfqpoint{3.996268in}{2.016466in}}%
\pgfpathlineto{\pgfqpoint{3.996268in}{2.035955in}}%
\pgfusepath{stroke}%
\end{pgfscope}%
\begin{pgfscope}%
\pgfpathrectangle{\pgfqpoint{0.664741in}{1.389617in}}{\pgfqpoint{4.922705in}{1.759717in}}%
\pgfusepath{clip}%
\pgfsetbuttcap%
\pgfsetroundjoin%
\pgfsetlinewidth{1.505625pt}%
\definecolor{currentstroke}{rgb}{0.949020,0.372549,0.360784}%
\pgfsetstrokecolor{currentstroke}%
\pgfsetstrokeopacity{0.900000}%
\pgfsetdash{}{0pt}%
\pgfpathmoveto{\pgfqpoint{4.244890in}{2.252290in}}%
\pgfpathlineto{\pgfqpoint{4.244890in}{2.280236in}}%
\pgfusepath{stroke}%
\end{pgfscope}%
\begin{pgfscope}%
\pgfpathrectangle{\pgfqpoint{0.664741in}{1.389617in}}{\pgfqpoint{4.922705in}{1.759717in}}%
\pgfusepath{clip}%
\pgfsetbuttcap%
\pgfsetroundjoin%
\pgfsetlinewidth{1.505625pt}%
\definecolor{currentstroke}{rgb}{0.949020,0.372549,0.360784}%
\pgfsetstrokecolor{currentstroke}%
\pgfsetstrokeopacity{0.900000}%
\pgfsetdash{}{0pt}%
\pgfpathmoveto{\pgfqpoint{4.493511in}{2.540381in}}%
\pgfpathlineto{\pgfqpoint{4.493511in}{2.578914in}}%
\pgfusepath{stroke}%
\end{pgfscope}%
\begin{pgfscope}%
\pgfpathrectangle{\pgfqpoint{0.664741in}{1.389617in}}{\pgfqpoint{4.922705in}{1.759717in}}%
\pgfusepath{clip}%
\pgfsetbuttcap%
\pgfsetroundjoin%
\pgfsetlinewidth{1.505625pt}%
\definecolor{currentstroke}{rgb}{0.949020,0.372549,0.360784}%
\pgfsetstrokecolor{currentstroke}%
\pgfsetstrokeopacity{0.900000}%
\pgfsetdash{}{0pt}%
\pgfpathmoveto{\pgfqpoint{4.742132in}{2.765424in}}%
\pgfpathlineto{\pgfqpoint{4.742132in}{2.817944in}}%
\pgfusepath{stroke}%
\end{pgfscope}%
\begin{pgfscope}%
\pgfpathrectangle{\pgfqpoint{0.664741in}{1.389617in}}{\pgfqpoint{4.922705in}{1.759717in}}%
\pgfusepath{clip}%
\pgfsetbuttcap%
\pgfsetroundjoin%
\pgfsetlinewidth{1.505625pt}%
\definecolor{currentstroke}{rgb}{0.949020,0.372549,0.360784}%
\pgfsetstrokecolor{currentstroke}%
\pgfsetstrokeopacity{0.900000}%
\pgfsetdash{}{0pt}%
\pgfpathmoveto{\pgfqpoint{4.990754in}{2.914855in}}%
\pgfpathlineto{\pgfqpoint{4.990754in}{2.981301in}}%
\pgfusepath{stroke}%
\end{pgfscope}%
\begin{pgfscope}%
\pgfpathrectangle{\pgfqpoint{0.664741in}{1.389617in}}{\pgfqpoint{4.922705in}{1.759717in}}%
\pgfusepath{clip}%
\pgfsetbuttcap%
\pgfsetroundjoin%
\pgfsetlinewidth{1.505625pt}%
\definecolor{currentstroke}{rgb}{0.949020,0.372549,0.360784}%
\pgfsetstrokecolor{currentstroke}%
\pgfsetstrokeopacity{0.900000}%
\pgfsetdash{}{0pt}%
\pgfpathmoveto{\pgfqpoint{5.239375in}{2.970032in}}%
\pgfpathlineto{\pgfqpoint{5.239375in}{3.069346in}}%
\pgfusepath{stroke}%
\end{pgfscope}%
\begin{pgfscope}%
\pgfpathrectangle{\pgfqpoint{0.664741in}{1.389617in}}{\pgfqpoint{4.922705in}{1.759717in}}%
\pgfusepath{clip}%
\pgfsetbuttcap%
\pgfsetroundjoin%
\definecolor{currentfill}{rgb}{0.313725,0.317647,0.309804}%
\pgfsetfillcolor{currentfill}%
\pgfsetfillopacity{0.900000}%
\pgfsetlinewidth{1.003750pt}%
\definecolor{currentstroke}{rgb}{0.313725,0.317647,0.309804}%
\pgfsetstrokecolor{currentstroke}%
\pgfsetstrokeopacity{0.900000}%
\pgfsetdash{}{0pt}%
\pgfsys@defobject{currentmarker}{\pgfqpoint{0.000000in}{-0.013889in}}{\pgfqpoint{0.000000in}{0.013889in}}{%
\pgfpathmoveto{\pgfqpoint{0.000000in}{-0.013889in}}%
\pgfpathlineto{\pgfqpoint{0.000000in}{0.013889in}}%
\pgfusepath{stroke,fill}%
}%
\begin{pgfscope}%
\pgfsys@transformshift{0.888500in}{2.868858in}%
\pgfsys@useobject{currentmarker}{}%
\end{pgfscope}%
\begin{pgfscope}%
\pgfsys@transformshift{1.137121in}{2.716306in}%
\pgfsys@useobject{currentmarker}{}%
\end{pgfscope}%
\begin{pgfscope}%
\pgfsys@transformshift{1.385743in}{2.557950in}%
\pgfsys@useobject{currentmarker}{}%
\end{pgfscope}%
\begin{pgfscope}%
\pgfsys@transformshift{1.634364in}{2.437089in}%
\pgfsys@useobject{currentmarker}{}%
\end{pgfscope}%
\begin{pgfscope}%
\pgfsys@transformshift{1.882986in}{2.323877in}%
\pgfsys@useobject{currentmarker}{}%
\end{pgfscope}%
\begin{pgfscope}%
\pgfsys@transformshift{2.131607in}{2.224214in}%
\pgfsys@useobject{currentmarker}{}%
\end{pgfscope}%
\begin{pgfscope}%
\pgfsys@transformshift{2.380229in}{2.093596in}%
\pgfsys@useobject{currentmarker}{}%
\end{pgfscope}%
\begin{pgfscope}%
\pgfsys@transformshift{2.628850in}{1.976669in}%
\pgfsys@useobject{currentmarker}{}%
\end{pgfscope}%
\begin{pgfscope}%
\pgfsys@transformshift{2.877472in}{1.915815in}%
\pgfsys@useobject{currentmarker}{}%
\end{pgfscope}%
\begin{pgfscope}%
\pgfsys@transformshift{3.126093in}{1.921293in}%
\pgfsys@useobject{currentmarker}{}%
\end{pgfscope}%
\begin{pgfscope}%
\pgfsys@transformshift{3.374715in}{1.886080in}%
\pgfsys@useobject{currentmarker}{}%
\end{pgfscope}%
\begin{pgfscope}%
\pgfsys@transformshift{3.623336in}{1.866379in}%
\pgfsys@useobject{currentmarker}{}%
\end{pgfscope}%
\begin{pgfscope}%
\pgfsys@transformshift{3.871957in}{1.877324in}%
\pgfsys@useobject{currentmarker}{}%
\end{pgfscope}%
\begin{pgfscope}%
\pgfsys@transformshift{4.120579in}{1.969561in}%
\pgfsys@useobject{currentmarker}{}%
\end{pgfscope}%
\begin{pgfscope}%
\pgfsys@transformshift{4.369200in}{2.129492in}%
\pgfsys@useobject{currentmarker}{}%
\end{pgfscope}%
\begin{pgfscope}%
\pgfsys@transformshift{4.617822in}{2.346695in}%
\pgfsys@useobject{currentmarker}{}%
\end{pgfscope}%
\begin{pgfscope}%
\pgfsys@transformshift{4.866443in}{2.608543in}%
\pgfsys@useobject{currentmarker}{}%
\end{pgfscope}%
\begin{pgfscope}%
\pgfsys@transformshift{5.115065in}{2.958940in}%
\pgfsys@useobject{currentmarker}{}%
\end{pgfscope}%
\end{pgfscope}%
\begin{pgfscope}%
\pgfpathrectangle{\pgfqpoint{0.664741in}{1.389617in}}{\pgfqpoint{4.922705in}{1.759717in}}%
\pgfusepath{clip}%
\pgfsetbuttcap%
\pgfsetroundjoin%
\definecolor{currentfill}{rgb}{0.313725,0.317647,0.309804}%
\pgfsetfillcolor{currentfill}%
\pgfsetfillopacity{0.900000}%
\pgfsetlinewidth{1.003750pt}%
\definecolor{currentstroke}{rgb}{0.313725,0.317647,0.309804}%
\pgfsetstrokecolor{currentstroke}%
\pgfsetstrokeopacity{0.900000}%
\pgfsetdash{}{0pt}%
\pgfsys@defobject{currentmarker}{\pgfqpoint{0.000000in}{-0.013889in}}{\pgfqpoint{0.000000in}{0.013889in}}{%
\pgfpathmoveto{\pgfqpoint{0.000000in}{-0.013889in}}%
\pgfpathlineto{\pgfqpoint{0.000000in}{0.013889in}}%
\pgfusepath{stroke,fill}%
}%
\begin{pgfscope}%
\pgfsys@transformshift{1.137121in}{2.868858in}%
\pgfsys@useobject{currentmarker}{}%
\end{pgfscope}%
\begin{pgfscope}%
\pgfsys@transformshift{1.385743in}{2.716306in}%
\pgfsys@useobject{currentmarker}{}%
\end{pgfscope}%
\begin{pgfscope}%
\pgfsys@transformshift{1.634364in}{2.557950in}%
\pgfsys@useobject{currentmarker}{}%
\end{pgfscope}%
\begin{pgfscope}%
\pgfsys@transformshift{1.882986in}{2.437089in}%
\pgfsys@useobject{currentmarker}{}%
\end{pgfscope}%
\begin{pgfscope}%
\pgfsys@transformshift{2.131607in}{2.323877in}%
\pgfsys@useobject{currentmarker}{}%
\end{pgfscope}%
\begin{pgfscope}%
\pgfsys@transformshift{2.380229in}{2.224214in}%
\pgfsys@useobject{currentmarker}{}%
\end{pgfscope}%
\begin{pgfscope}%
\pgfsys@transformshift{2.628850in}{2.093596in}%
\pgfsys@useobject{currentmarker}{}%
\end{pgfscope}%
\begin{pgfscope}%
\pgfsys@transformshift{2.877472in}{1.976669in}%
\pgfsys@useobject{currentmarker}{}%
\end{pgfscope}%
\begin{pgfscope}%
\pgfsys@transformshift{3.126093in}{1.915815in}%
\pgfsys@useobject{currentmarker}{}%
\end{pgfscope}%
\begin{pgfscope}%
\pgfsys@transformshift{3.374715in}{1.921293in}%
\pgfsys@useobject{currentmarker}{}%
\end{pgfscope}%
\begin{pgfscope}%
\pgfsys@transformshift{3.623336in}{1.886080in}%
\pgfsys@useobject{currentmarker}{}%
\end{pgfscope}%
\begin{pgfscope}%
\pgfsys@transformshift{3.871957in}{1.866379in}%
\pgfsys@useobject{currentmarker}{}%
\end{pgfscope}%
\begin{pgfscope}%
\pgfsys@transformshift{4.120579in}{1.877324in}%
\pgfsys@useobject{currentmarker}{}%
\end{pgfscope}%
\begin{pgfscope}%
\pgfsys@transformshift{4.369200in}{1.969561in}%
\pgfsys@useobject{currentmarker}{}%
\end{pgfscope}%
\begin{pgfscope}%
\pgfsys@transformshift{4.617822in}{2.129492in}%
\pgfsys@useobject{currentmarker}{}%
\end{pgfscope}%
\begin{pgfscope}%
\pgfsys@transformshift{4.866443in}{2.346695in}%
\pgfsys@useobject{currentmarker}{}%
\end{pgfscope}%
\begin{pgfscope}%
\pgfsys@transformshift{5.115065in}{2.608543in}%
\pgfsys@useobject{currentmarker}{}%
\end{pgfscope}%
\begin{pgfscope}%
\pgfsys@transformshift{5.363686in}{2.958940in}%
\pgfsys@useobject{currentmarker}{}%
\end{pgfscope}%
\end{pgfscope}%
\begin{pgfscope}%
\pgfpathrectangle{\pgfqpoint{0.664741in}{1.389617in}}{\pgfqpoint{4.922705in}{1.759717in}}%
\pgfusepath{clip}%
\pgfsetbuttcap%
\pgfsetroundjoin%
\definecolor{currentfill}{rgb}{0.313725,0.317647,0.309804}%
\pgfsetfillcolor{currentfill}%
\pgfsetfillopacity{0.900000}%
\pgfsetlinewidth{1.003750pt}%
\definecolor{currentstroke}{rgb}{0.313725,0.317647,0.309804}%
\pgfsetstrokecolor{currentstroke}%
\pgfsetstrokeopacity{0.900000}%
\pgfsetdash{}{0pt}%
\pgfsys@defobject{currentmarker}{\pgfqpoint{-0.013889in}{-0.000000in}}{\pgfqpoint{0.013889in}{0.000000in}}{%
\pgfpathmoveto{\pgfqpoint{0.013889in}{-0.000000in}}%
\pgfpathlineto{\pgfqpoint{-0.013889in}{0.000000in}}%
\pgfusepath{stroke,fill}%
}%
\begin{pgfscope}%
\pgfsys@transformshift{1.012811in}{2.814328in}%
\pgfsys@useobject{currentmarker}{}%
\end{pgfscope}%
\begin{pgfscope}%
\pgfsys@transformshift{1.261432in}{2.685015in}%
\pgfsys@useobject{currentmarker}{}%
\end{pgfscope}%
\begin{pgfscope}%
\pgfsys@transformshift{1.510054in}{2.543225in}%
\pgfsys@useobject{currentmarker}{}%
\end{pgfscope}%
\begin{pgfscope}%
\pgfsys@transformshift{1.758675in}{2.427171in}%
\pgfsys@useobject{currentmarker}{}%
\end{pgfscope}%
\begin{pgfscope}%
\pgfsys@transformshift{2.007297in}{2.315996in}%
\pgfsys@useobject{currentmarker}{}%
\end{pgfscope}%
\begin{pgfscope}%
\pgfsys@transformshift{2.255918in}{2.217588in}%
\pgfsys@useobject{currentmarker}{}%
\end{pgfscope}%
\begin{pgfscope}%
\pgfsys@transformshift{2.504539in}{2.088007in}%
\pgfsys@useobject{currentmarker}{}%
\end{pgfscope}%
\begin{pgfscope}%
\pgfsys@transformshift{2.753161in}{1.971325in}%
\pgfsys@useobject{currentmarker}{}%
\end{pgfscope}%
\begin{pgfscope}%
\pgfsys@transformshift{3.001782in}{1.910498in}%
\pgfsys@useobject{currentmarker}{}%
\end{pgfscope}%
\begin{pgfscope}%
\pgfsys@transformshift{3.250404in}{1.915697in}%
\pgfsys@useobject{currentmarker}{}%
\end{pgfscope}%
\begin{pgfscope}%
\pgfsys@transformshift{3.499025in}{1.879960in}%
\pgfsys@useobject{currentmarker}{}%
\end{pgfscope}%
\begin{pgfscope}%
\pgfsys@transformshift{3.747647in}{1.858893in}%
\pgfsys@useobject{currentmarker}{}%
\end{pgfscope}%
\begin{pgfscope}%
\pgfsys@transformshift{3.996268in}{1.868215in}%
\pgfsys@useobject{currentmarker}{}%
\end{pgfscope}%
\begin{pgfscope}%
\pgfsys@transformshift{4.244890in}{1.955799in}%
\pgfsys@useobject{currentmarker}{}%
\end{pgfscope}%
\begin{pgfscope}%
\pgfsys@transformshift{4.493511in}{2.110796in}%
\pgfsys@useobject{currentmarker}{}%
\end{pgfscope}%
\begin{pgfscope}%
\pgfsys@transformshift{4.742132in}{2.322512in}%
\pgfsys@useobject{currentmarker}{}%
\end{pgfscope}%
\begin{pgfscope}%
\pgfsys@transformshift{4.990754in}{2.580057in}%
\pgfsys@useobject{currentmarker}{}%
\end{pgfscope}%
\begin{pgfscope}%
\pgfsys@transformshift{5.239375in}{2.920196in}%
\pgfsys@useobject{currentmarker}{}%
\end{pgfscope}%
\end{pgfscope}%
\begin{pgfscope}%
\pgfpathrectangle{\pgfqpoint{0.664741in}{1.389617in}}{\pgfqpoint{4.922705in}{1.759717in}}%
\pgfusepath{clip}%
\pgfsetbuttcap%
\pgfsetroundjoin%
\definecolor{currentfill}{rgb}{0.313725,0.317647,0.309804}%
\pgfsetfillcolor{currentfill}%
\pgfsetfillopacity{0.900000}%
\pgfsetlinewidth{1.003750pt}%
\definecolor{currentstroke}{rgb}{0.313725,0.317647,0.309804}%
\pgfsetstrokecolor{currentstroke}%
\pgfsetstrokeopacity{0.900000}%
\pgfsetdash{}{0pt}%
\pgfsys@defobject{currentmarker}{\pgfqpoint{-0.013889in}{-0.000000in}}{\pgfqpoint{0.013889in}{0.000000in}}{%
\pgfpathmoveto{\pgfqpoint{0.013889in}{-0.000000in}}%
\pgfpathlineto{\pgfqpoint{-0.013889in}{0.000000in}}%
\pgfusepath{stroke,fill}%
}%
\begin{pgfscope}%
\pgfsys@transformshift{1.012811in}{2.945722in}%
\pgfsys@useobject{currentmarker}{}%
\end{pgfscope}%
\begin{pgfscope}%
\pgfsys@transformshift{1.261432in}{2.743674in}%
\pgfsys@useobject{currentmarker}{}%
\end{pgfscope}%
\begin{pgfscope}%
\pgfsys@transformshift{1.510054in}{2.576122in}%
\pgfsys@useobject{currentmarker}{}%
\end{pgfscope}%
\begin{pgfscope}%
\pgfsys@transformshift{1.758675in}{2.448503in}%
\pgfsys@useobject{currentmarker}{}%
\end{pgfscope}%
\begin{pgfscope}%
\pgfsys@transformshift{2.007297in}{2.331770in}%
\pgfsys@useobject{currentmarker}{}%
\end{pgfscope}%
\begin{pgfscope}%
\pgfsys@transformshift{2.255918in}{2.230733in}%
\pgfsys@useobject{currentmarker}{}%
\end{pgfscope}%
\begin{pgfscope}%
\pgfsys@transformshift{2.504539in}{2.100000in}%
\pgfsys@useobject{currentmarker}{}%
\end{pgfscope}%
\begin{pgfscope}%
\pgfsys@transformshift{2.753161in}{1.982133in}%
\pgfsys@useobject{currentmarker}{}%
\end{pgfscope}%
\begin{pgfscope}%
\pgfsys@transformshift{3.001782in}{1.920861in}%
\pgfsys@useobject{currentmarker}{}%
\end{pgfscope}%
\begin{pgfscope}%
\pgfsys@transformshift{3.250404in}{1.927216in}%
\pgfsys@useobject{currentmarker}{}%
\end{pgfscope}%
\begin{pgfscope}%
\pgfsys@transformshift{3.499025in}{1.893082in}%
\pgfsys@useobject{currentmarker}{}%
\end{pgfscope}%
\begin{pgfscope}%
\pgfsys@transformshift{3.747647in}{1.873116in}%
\pgfsys@useobject{currentmarker}{}%
\end{pgfscope}%
\begin{pgfscope}%
\pgfsys@transformshift{3.996268in}{1.886928in}%
\pgfsys@useobject{currentmarker}{}%
\end{pgfscope}%
\begin{pgfscope}%
\pgfsys@transformshift{4.244890in}{1.981813in}%
\pgfsys@useobject{currentmarker}{}%
\end{pgfscope}%
\begin{pgfscope}%
\pgfsys@transformshift{4.493511in}{2.147134in}%
\pgfsys@useobject{currentmarker}{}%
\end{pgfscope}%
\begin{pgfscope}%
\pgfsys@transformshift{4.742132in}{2.369090in}%
\pgfsys@useobject{currentmarker}{}%
\end{pgfscope}%
\begin{pgfscope}%
\pgfsys@transformshift{4.990754in}{2.639852in}%
\pgfsys@useobject{currentmarker}{}%
\end{pgfscope}%
\begin{pgfscope}%
\pgfsys@transformshift{5.239375in}{2.999856in}%
\pgfsys@useobject{currentmarker}{}%
\end{pgfscope}%
\end{pgfscope}%
\begin{pgfscope}%
\pgfpathrectangle{\pgfqpoint{0.664741in}{1.389617in}}{\pgfqpoint{4.922705in}{1.759717in}}%
\pgfusepath{clip}%
\pgfsetbuttcap%
\pgfsetroundjoin%
\definecolor{currentfill}{rgb}{0.949020,0.372549,0.360784}%
\pgfsetfillcolor{currentfill}%
\pgfsetfillopacity{0.900000}%
\pgfsetlinewidth{1.003750pt}%
\definecolor{currentstroke}{rgb}{0.949020,0.372549,0.360784}%
\pgfsetstrokecolor{currentstroke}%
\pgfsetstrokeopacity{0.900000}%
\pgfsetdash{}{0pt}%
\pgfsys@defobject{currentmarker}{\pgfqpoint{0.000000in}{-0.013889in}}{\pgfqpoint{0.000000in}{0.013889in}}{%
\pgfpathmoveto{\pgfqpoint{0.000000in}{-0.013889in}}%
\pgfpathlineto{\pgfqpoint{0.000000in}{0.013889in}}%
\pgfusepath{stroke,fill}%
}%
\begin{pgfscope}%
\pgfsys@transformshift{0.888500in}{1.502716in}%
\pgfsys@useobject{currentmarker}{}%
\end{pgfscope}%
\begin{pgfscope}%
\pgfsys@transformshift{1.137121in}{1.525704in}%
\pgfsys@useobject{currentmarker}{}%
\end{pgfscope}%
\begin{pgfscope}%
\pgfsys@transformshift{1.385743in}{1.561847in}%
\pgfsys@useobject{currentmarker}{}%
\end{pgfscope}%
\begin{pgfscope}%
\pgfsys@transformshift{1.634364in}{1.607916in}%
\pgfsys@useobject{currentmarker}{}%
\end{pgfscope}%
\begin{pgfscope}%
\pgfsys@transformshift{1.882986in}{1.690719in}%
\pgfsys@useobject{currentmarker}{}%
\end{pgfscope}%
\begin{pgfscope}%
\pgfsys@transformshift{2.131607in}{1.792934in}%
\pgfsys@useobject{currentmarker}{}%
\end{pgfscope}%
\begin{pgfscope}%
\pgfsys@transformshift{2.380229in}{1.843062in}%
\pgfsys@useobject{currentmarker}{}%
\end{pgfscope}%
\begin{pgfscope}%
\pgfsys@transformshift{2.628850in}{1.812470in}%
\pgfsys@useobject{currentmarker}{}%
\end{pgfscope}%
\begin{pgfscope}%
\pgfsys@transformshift{2.877472in}{1.819471in}%
\pgfsys@useobject{currentmarker}{}%
\end{pgfscope}%
\begin{pgfscope}%
\pgfsys@transformshift{3.126093in}{1.838381in}%
\pgfsys@useobject{currentmarker}{}%
\end{pgfscope}%
\begin{pgfscope}%
\pgfsys@transformshift{3.374715in}{1.849312in}%
\pgfsys@useobject{currentmarker}{}%
\end{pgfscope}%
\begin{pgfscope}%
\pgfsys@transformshift{3.623336in}{1.904850in}%
\pgfsys@useobject{currentmarker}{}%
\end{pgfscope}%
\begin{pgfscope}%
\pgfsys@transformshift{3.871957in}{2.025758in}%
\pgfsys@useobject{currentmarker}{}%
\end{pgfscope}%
\begin{pgfscope}%
\pgfsys@transformshift{4.120579in}{2.266408in}%
\pgfsys@useobject{currentmarker}{}%
\end{pgfscope}%
\begin{pgfscope}%
\pgfsys@transformshift{4.369200in}{2.558696in}%
\pgfsys@useobject{currentmarker}{}%
\end{pgfscope}%
\begin{pgfscope}%
\pgfsys@transformshift{4.617822in}{2.789303in}%
\pgfsys@useobject{currentmarker}{}%
\end{pgfscope}%
\begin{pgfscope}%
\pgfsys@transformshift{4.866443in}{2.949469in}%
\pgfsys@useobject{currentmarker}{}%
\end{pgfscope}%
\begin{pgfscope}%
\pgfsys@transformshift{5.115065in}{3.019010in}%
\pgfsys@useobject{currentmarker}{}%
\end{pgfscope}%
\end{pgfscope}%
\begin{pgfscope}%
\pgfpathrectangle{\pgfqpoint{0.664741in}{1.389617in}}{\pgfqpoint{4.922705in}{1.759717in}}%
\pgfusepath{clip}%
\pgfsetbuttcap%
\pgfsetroundjoin%
\definecolor{currentfill}{rgb}{0.949020,0.372549,0.360784}%
\pgfsetfillcolor{currentfill}%
\pgfsetfillopacity{0.900000}%
\pgfsetlinewidth{1.003750pt}%
\definecolor{currentstroke}{rgb}{0.949020,0.372549,0.360784}%
\pgfsetstrokecolor{currentstroke}%
\pgfsetstrokeopacity{0.900000}%
\pgfsetdash{}{0pt}%
\pgfsys@defobject{currentmarker}{\pgfqpoint{0.000000in}{-0.013889in}}{\pgfqpoint{0.000000in}{0.013889in}}{%
\pgfpathmoveto{\pgfqpoint{0.000000in}{-0.013889in}}%
\pgfpathlineto{\pgfqpoint{0.000000in}{0.013889in}}%
\pgfusepath{stroke,fill}%
}%
\begin{pgfscope}%
\pgfsys@transformshift{1.137121in}{1.502716in}%
\pgfsys@useobject{currentmarker}{}%
\end{pgfscope}%
\begin{pgfscope}%
\pgfsys@transformshift{1.385743in}{1.525704in}%
\pgfsys@useobject{currentmarker}{}%
\end{pgfscope}%
\begin{pgfscope}%
\pgfsys@transformshift{1.634364in}{1.561847in}%
\pgfsys@useobject{currentmarker}{}%
\end{pgfscope}%
\begin{pgfscope}%
\pgfsys@transformshift{1.882986in}{1.607916in}%
\pgfsys@useobject{currentmarker}{}%
\end{pgfscope}%
\begin{pgfscope}%
\pgfsys@transformshift{2.131607in}{1.690719in}%
\pgfsys@useobject{currentmarker}{}%
\end{pgfscope}%
\begin{pgfscope}%
\pgfsys@transformshift{2.380229in}{1.792934in}%
\pgfsys@useobject{currentmarker}{}%
\end{pgfscope}%
\begin{pgfscope}%
\pgfsys@transformshift{2.628850in}{1.843062in}%
\pgfsys@useobject{currentmarker}{}%
\end{pgfscope}%
\begin{pgfscope}%
\pgfsys@transformshift{2.877472in}{1.812470in}%
\pgfsys@useobject{currentmarker}{}%
\end{pgfscope}%
\begin{pgfscope}%
\pgfsys@transformshift{3.126093in}{1.819471in}%
\pgfsys@useobject{currentmarker}{}%
\end{pgfscope}%
\begin{pgfscope}%
\pgfsys@transformshift{3.374715in}{1.838381in}%
\pgfsys@useobject{currentmarker}{}%
\end{pgfscope}%
\begin{pgfscope}%
\pgfsys@transformshift{3.623336in}{1.849312in}%
\pgfsys@useobject{currentmarker}{}%
\end{pgfscope}%
\begin{pgfscope}%
\pgfsys@transformshift{3.871957in}{1.904850in}%
\pgfsys@useobject{currentmarker}{}%
\end{pgfscope}%
\begin{pgfscope}%
\pgfsys@transformshift{4.120579in}{2.025758in}%
\pgfsys@useobject{currentmarker}{}%
\end{pgfscope}%
\begin{pgfscope}%
\pgfsys@transformshift{4.369200in}{2.266408in}%
\pgfsys@useobject{currentmarker}{}%
\end{pgfscope}%
\begin{pgfscope}%
\pgfsys@transformshift{4.617822in}{2.558696in}%
\pgfsys@useobject{currentmarker}{}%
\end{pgfscope}%
\begin{pgfscope}%
\pgfsys@transformshift{4.866443in}{2.789303in}%
\pgfsys@useobject{currentmarker}{}%
\end{pgfscope}%
\begin{pgfscope}%
\pgfsys@transformshift{5.115065in}{2.949469in}%
\pgfsys@useobject{currentmarker}{}%
\end{pgfscope}%
\begin{pgfscope}%
\pgfsys@transformshift{5.363686in}{3.019010in}%
\pgfsys@useobject{currentmarker}{}%
\end{pgfscope}%
\end{pgfscope}%
\begin{pgfscope}%
\pgfpathrectangle{\pgfqpoint{0.664741in}{1.389617in}}{\pgfqpoint{4.922705in}{1.759717in}}%
\pgfusepath{clip}%
\pgfsetbuttcap%
\pgfsetroundjoin%
\definecolor{currentfill}{rgb}{0.949020,0.372549,0.360784}%
\pgfsetfillcolor{currentfill}%
\pgfsetfillopacity{0.900000}%
\pgfsetlinewidth{1.003750pt}%
\definecolor{currentstroke}{rgb}{0.949020,0.372549,0.360784}%
\pgfsetstrokecolor{currentstroke}%
\pgfsetstrokeopacity{0.900000}%
\pgfsetdash{}{0pt}%
\pgfsys@defobject{currentmarker}{\pgfqpoint{-0.013889in}{-0.000000in}}{\pgfqpoint{0.013889in}{0.000000in}}{%
\pgfpathmoveto{\pgfqpoint{0.013889in}{-0.000000in}}%
\pgfpathlineto{\pgfqpoint{-0.013889in}{0.000000in}}%
\pgfusepath{stroke,fill}%
}%
\begin{pgfscope}%
\pgfsys@transformshift{1.012811in}{1.469604in}%
\pgfsys@useobject{currentmarker}{}%
\end{pgfscope}%
\begin{pgfscope}%
\pgfsys@transformshift{1.261432in}{1.509142in}%
\pgfsys@useobject{currentmarker}{}%
\end{pgfscope}%
\begin{pgfscope}%
\pgfsys@transformshift{1.510054in}{1.551997in}%
\pgfsys@useobject{currentmarker}{}%
\end{pgfscope}%
\begin{pgfscope}%
\pgfsys@transformshift{1.758675in}{1.600859in}%
\pgfsys@useobject{currentmarker}{}%
\end{pgfscope}%
\begin{pgfscope}%
\pgfsys@transformshift{2.007297in}{1.685004in}%
\pgfsys@useobject{currentmarker}{}%
\end{pgfscope}%
\begin{pgfscope}%
\pgfsys@transformshift{2.255918in}{1.787532in}%
\pgfsys@useobject{currentmarker}{}%
\end{pgfscope}%
\begin{pgfscope}%
\pgfsys@transformshift{2.504539in}{1.838122in}%
\pgfsys@useobject{currentmarker}{}%
\end{pgfscope}%
\begin{pgfscope}%
\pgfsys@transformshift{2.753161in}{1.807711in}%
\pgfsys@useobject{currentmarker}{}%
\end{pgfscope}%
\begin{pgfscope}%
\pgfsys@transformshift{3.001782in}{1.815004in}%
\pgfsys@useobject{currentmarker}{}%
\end{pgfscope}%
\begin{pgfscope}%
\pgfsys@transformshift{3.250404in}{1.832851in}%
\pgfsys@useobject{currentmarker}{}%
\end{pgfscope}%
\begin{pgfscope}%
\pgfsys@transformshift{3.499025in}{1.843107in}%
\pgfsys@useobject{currentmarker}{}%
\end{pgfscope}%
\begin{pgfscope}%
\pgfsys@transformshift{3.747647in}{1.896963in}%
\pgfsys@useobject{currentmarker}{}%
\end{pgfscope}%
\begin{pgfscope}%
\pgfsys@transformshift{3.996268in}{2.016466in}%
\pgfsys@useobject{currentmarker}{}%
\end{pgfscope}%
\begin{pgfscope}%
\pgfsys@transformshift{4.244890in}{2.252290in}%
\pgfsys@useobject{currentmarker}{}%
\end{pgfscope}%
\begin{pgfscope}%
\pgfsys@transformshift{4.493511in}{2.540381in}%
\pgfsys@useobject{currentmarker}{}%
\end{pgfscope}%
\begin{pgfscope}%
\pgfsys@transformshift{4.742132in}{2.765424in}%
\pgfsys@useobject{currentmarker}{}%
\end{pgfscope}%
\begin{pgfscope}%
\pgfsys@transformshift{4.990754in}{2.914855in}%
\pgfsys@useobject{currentmarker}{}%
\end{pgfscope}%
\begin{pgfscope}%
\pgfsys@transformshift{5.239375in}{2.970032in}%
\pgfsys@useobject{currentmarker}{}%
\end{pgfscope}%
\end{pgfscope}%
\begin{pgfscope}%
\pgfpathrectangle{\pgfqpoint{0.664741in}{1.389617in}}{\pgfqpoint{4.922705in}{1.759717in}}%
\pgfusepath{clip}%
\pgfsetbuttcap%
\pgfsetroundjoin%
\definecolor{currentfill}{rgb}{0.949020,0.372549,0.360784}%
\pgfsetfillcolor{currentfill}%
\pgfsetfillopacity{0.900000}%
\pgfsetlinewidth{1.003750pt}%
\definecolor{currentstroke}{rgb}{0.949020,0.372549,0.360784}%
\pgfsetstrokecolor{currentstroke}%
\pgfsetstrokeopacity{0.900000}%
\pgfsetdash{}{0pt}%
\pgfsys@defobject{currentmarker}{\pgfqpoint{-0.013889in}{-0.000000in}}{\pgfqpoint{0.013889in}{0.000000in}}{%
\pgfpathmoveto{\pgfqpoint{0.013889in}{-0.000000in}}%
\pgfpathlineto{\pgfqpoint{-0.013889in}{0.000000in}}%
\pgfusepath{stroke,fill}%
}%
\begin{pgfscope}%
\pgfsys@transformshift{1.012811in}{1.532845in}%
\pgfsys@useobject{currentmarker}{}%
\end{pgfscope}%
\begin{pgfscope}%
\pgfsys@transformshift{1.261432in}{1.542650in}%
\pgfsys@useobject{currentmarker}{}%
\end{pgfscope}%
\begin{pgfscope}%
\pgfsys@transformshift{1.510054in}{1.571054in}%
\pgfsys@useobject{currentmarker}{}%
\end{pgfscope}%
\begin{pgfscope}%
\pgfsys@transformshift{1.758675in}{1.615567in}%
\pgfsys@useobject{currentmarker}{}%
\end{pgfscope}%
\begin{pgfscope}%
\pgfsys@transformshift{2.007297in}{1.697054in}%
\pgfsys@useobject{currentmarker}{}%
\end{pgfscope}%
\begin{pgfscope}%
\pgfsys@transformshift{2.255918in}{1.798124in}%
\pgfsys@useobject{currentmarker}{}%
\end{pgfscope}%
\begin{pgfscope}%
\pgfsys@transformshift{2.504539in}{1.847400in}%
\pgfsys@useobject{currentmarker}{}%
\end{pgfscope}%
\begin{pgfscope}%
\pgfsys@transformshift{2.753161in}{1.816967in}%
\pgfsys@useobject{currentmarker}{}%
\end{pgfscope}%
\begin{pgfscope}%
\pgfsys@transformshift{3.001782in}{1.824267in}%
\pgfsys@useobject{currentmarker}{}%
\end{pgfscope}%
\begin{pgfscope}%
\pgfsys@transformshift{3.250404in}{1.843511in}%
\pgfsys@useobject{currentmarker}{}%
\end{pgfscope}%
\begin{pgfscope}%
\pgfsys@transformshift{3.499025in}{1.855880in}%
\pgfsys@useobject{currentmarker}{}%
\end{pgfscope}%
\begin{pgfscope}%
\pgfsys@transformshift{3.747647in}{1.912785in}%
\pgfsys@useobject{currentmarker}{}%
\end{pgfscope}%
\begin{pgfscope}%
\pgfsys@transformshift{3.996268in}{2.035955in}%
\pgfsys@useobject{currentmarker}{}%
\end{pgfscope}%
\begin{pgfscope}%
\pgfsys@transformshift{4.244890in}{2.280236in}%
\pgfsys@useobject{currentmarker}{}%
\end{pgfscope}%
\begin{pgfscope}%
\pgfsys@transformshift{4.493511in}{2.578914in}%
\pgfsys@useobject{currentmarker}{}%
\end{pgfscope}%
\begin{pgfscope}%
\pgfsys@transformshift{4.742132in}{2.817944in}%
\pgfsys@useobject{currentmarker}{}%
\end{pgfscope}%
\begin{pgfscope}%
\pgfsys@transformshift{4.990754in}{2.981301in}%
\pgfsys@useobject{currentmarker}{}%
\end{pgfscope}%
\begin{pgfscope}%
\pgfsys@transformshift{5.239375in}{3.069346in}%
\pgfsys@useobject{currentmarker}{}%
\end{pgfscope}%
\end{pgfscope}%
\begin{pgfscope}%
\pgfsetrectcap%
\pgfsetmiterjoin%
\pgfsetlinewidth{0.803000pt}%
\definecolor{currentstroke}{rgb}{0.000000,0.000000,0.000000}%
\pgfsetstrokecolor{currentstroke}%
\pgfsetdash{}{0pt}%
\pgfpathmoveto{\pgfqpoint{0.664741in}{1.389617in}}%
\pgfpathlineto{\pgfqpoint{0.664741in}{3.149333in}}%
\pgfusepath{stroke}%
\end{pgfscope}%
\begin{pgfscope}%
\pgfsetrectcap%
\pgfsetmiterjoin%
\pgfsetlinewidth{0.803000pt}%
\definecolor{currentstroke}{rgb}{0.000000,0.000000,0.000000}%
\pgfsetstrokecolor{currentstroke}%
\pgfsetdash{}{0pt}%
\pgfpathmoveto{\pgfqpoint{5.587445in}{1.389617in}}%
\pgfpathlineto{\pgfqpoint{5.587445in}{3.149333in}}%
\pgfusepath{stroke}%
\end{pgfscope}%
\begin{pgfscope}%
\pgfsetrectcap%
\pgfsetmiterjoin%
\pgfsetlinewidth{0.803000pt}%
\definecolor{currentstroke}{rgb}{0.000000,0.000000,0.000000}%
\pgfsetstrokecolor{currentstroke}%
\pgfsetdash{}{0pt}%
\pgfpathmoveto{\pgfqpoint{0.664741in}{1.389617in}}%
\pgfpathlineto{\pgfqpoint{5.587445in}{1.389617in}}%
\pgfusepath{stroke}%
\end{pgfscope}%
\begin{pgfscope}%
\pgfsetrectcap%
\pgfsetmiterjoin%
\pgfsetlinewidth{0.803000pt}%
\definecolor{currentstroke}{rgb}{0.000000,0.000000,0.000000}%
\pgfsetstrokecolor{currentstroke}%
\pgfsetdash{}{0pt}%
\pgfpathmoveto{\pgfqpoint{0.664741in}{3.149333in}}%
\pgfpathlineto{\pgfqpoint{5.587445in}{3.149333in}}%
\pgfusepath{stroke}%
\end{pgfscope}%
\begin{pgfscope}%
\definecolor{textcolor}{rgb}{0.000000,0.000000,0.000000}%
\pgfsetstrokecolor{textcolor}%
\pgfsetfillcolor{textcolor}%
\pgftext[x=0.664741in,y=3.232667in,left,base]{\color{textcolor}\rmfamily\fontsize{12.000000}{14.400000}\selectfont Energy performance}%
\end{pgfscope}%
\begin{pgfscope}%
\pgfsetbuttcap%
\pgfsetmiterjoin%
\definecolor{currentfill}{rgb}{1.000000,1.000000,1.000000}%
\pgfsetfillcolor{currentfill}%
\pgfsetfillopacity{0.800000}%
\pgfsetlinewidth{1.003750pt}%
\definecolor{currentstroke}{rgb}{0.800000,0.800000,0.800000}%
\pgfsetstrokecolor{currentstroke}%
\pgfsetstrokeopacity{0.800000}%
\pgfsetdash{}{0pt}%
\pgfpathmoveto{\pgfqpoint{4.360223in}{1.445172in}}%
\pgfpathlineto{\pgfqpoint{5.509668in}{1.445172in}}%
\pgfpathquadraticcurveto{\pgfqpoint{5.531890in}{1.445172in}}{\pgfqpoint{5.531890in}{1.467394in}}%
\pgfpathlineto{\pgfqpoint{5.531890in}{1.922172in}}%
\pgfpathquadraticcurveto{\pgfqpoint{5.531890in}{1.944394in}}{\pgfqpoint{5.509668in}{1.944394in}}%
\pgfpathlineto{\pgfqpoint{4.360223in}{1.944394in}}%
\pgfpathquadraticcurveto{\pgfqpoint{4.338001in}{1.944394in}}{\pgfqpoint{4.338001in}{1.922172in}}%
\pgfpathlineto{\pgfqpoint{4.338001in}{1.467394in}}%
\pgfpathquadraticcurveto{\pgfqpoint{4.338001in}{1.445172in}}{\pgfqpoint{4.360223in}{1.445172in}}%
\pgfpathclose%
\pgfusepath{stroke,fill}%
\end{pgfscope}%
\begin{pgfscope}%
\pgfsetbuttcap%
\pgfsetmiterjoin%
\definecolor{currentfill}{rgb}{0.501961,0.501961,0.501961}%
\pgfsetfillcolor{currentfill}%
\pgfsetfillopacity{0.200000}%
\pgfsetlinewidth{0.000000pt}%
\definecolor{currentstroke}{rgb}{0.000000,0.000000,0.000000}%
\pgfsetstrokecolor{currentstroke}%
\pgfsetstrokeopacity{0.200000}%
\pgfsetdash{}{0pt}%
\pgfpathmoveto{\pgfqpoint{4.382445in}{1.822172in}}%
\pgfpathlineto{\pgfqpoint{4.604668in}{1.822172in}}%
\pgfpathlineto{\pgfqpoint{4.604668in}{1.899950in}}%
\pgfpathlineto{\pgfqpoint{4.382445in}{1.899950in}}%
\pgfpathclose%
\pgfusepath{fill}%
\end{pgfscope}%
\begin{pgfscope}%
\definecolor{textcolor}{rgb}{0.000000,0.000000,0.000000}%
\pgfsetstrokecolor{textcolor}%
\pgfsetfillcolor{textcolor}%
\pgftext[x=4.693557in,y=1.822172in,left,base]{\color{textcolor}\rmfamily\fontsize{8.000000}{9.600000}\selectfont Training events}%
\end{pgfscope}%
\begin{pgfscope}%
\pgfsetbuttcap%
\pgfsetroundjoin%
\pgfsetlinewidth{1.505625pt}%
\definecolor{currentstroke}{rgb}{0.313725,0.317647,0.309804}%
\pgfsetstrokecolor{currentstroke}%
\pgfsetstrokeopacity{0.900000}%
\pgfsetdash{}{0pt}%
\pgfpathmoveto{\pgfqpoint{4.438001in}{1.704950in}}%
\pgfpathlineto{\pgfqpoint{4.549112in}{1.704950in}}%
\pgfusepath{stroke}%
\end{pgfscope}%
\begin{pgfscope}%
\pgfsetbuttcap%
\pgfsetroundjoin%
\pgfsetlinewidth{1.505625pt}%
\definecolor{currentstroke}{rgb}{0.313725,0.317647,0.309804}%
\pgfsetstrokecolor{currentstroke}%
\pgfsetstrokeopacity{0.900000}%
\pgfsetdash{}{0pt}%
\pgfpathmoveto{\pgfqpoint{4.493557in}{1.649394in}}%
\pgfpathlineto{\pgfqpoint{4.493557in}{1.760506in}}%
\pgfusepath{stroke}%
\end{pgfscope}%
\begin{pgfscope}%
\pgfsetbuttcap%
\pgfsetroundjoin%
\definecolor{currentfill}{rgb}{0.313725,0.317647,0.309804}%
\pgfsetfillcolor{currentfill}%
\pgfsetfillopacity{0.900000}%
\pgfsetlinewidth{1.003750pt}%
\definecolor{currentstroke}{rgb}{0.313725,0.317647,0.309804}%
\pgfsetstrokecolor{currentstroke}%
\pgfsetstrokeopacity{0.900000}%
\pgfsetdash{}{0pt}%
\pgfsys@defobject{currentmarker}{\pgfqpoint{0.000000in}{-0.013889in}}{\pgfqpoint{0.000000in}{0.013889in}}{%
\pgfpathmoveto{\pgfqpoint{0.000000in}{-0.013889in}}%
\pgfpathlineto{\pgfqpoint{0.000000in}{0.013889in}}%
\pgfusepath{stroke,fill}%
}%
\begin{pgfscope}%
\pgfsys@transformshift{4.438001in}{1.704950in}%
\pgfsys@useobject{currentmarker}{}%
\end{pgfscope}%
\end{pgfscope}%
\begin{pgfscope}%
\pgfsetbuttcap%
\pgfsetroundjoin%
\definecolor{currentfill}{rgb}{0.313725,0.317647,0.309804}%
\pgfsetfillcolor{currentfill}%
\pgfsetfillopacity{0.900000}%
\pgfsetlinewidth{1.003750pt}%
\definecolor{currentstroke}{rgb}{0.313725,0.317647,0.309804}%
\pgfsetstrokecolor{currentstroke}%
\pgfsetstrokeopacity{0.900000}%
\pgfsetdash{}{0pt}%
\pgfsys@defobject{currentmarker}{\pgfqpoint{0.000000in}{-0.013889in}}{\pgfqpoint{0.000000in}{0.013889in}}{%
\pgfpathmoveto{\pgfqpoint{0.000000in}{-0.013889in}}%
\pgfpathlineto{\pgfqpoint{0.000000in}{0.013889in}}%
\pgfusepath{stroke,fill}%
}%
\begin{pgfscope}%
\pgfsys@transformshift{4.549112in}{1.704950in}%
\pgfsys@useobject{currentmarker}{}%
\end{pgfscope}%
\end{pgfscope}%
\begin{pgfscope}%
\pgfsetbuttcap%
\pgfsetroundjoin%
\definecolor{currentfill}{rgb}{0.313725,0.317647,0.309804}%
\pgfsetfillcolor{currentfill}%
\pgfsetfillopacity{0.900000}%
\pgfsetlinewidth{1.003750pt}%
\definecolor{currentstroke}{rgb}{0.313725,0.317647,0.309804}%
\pgfsetstrokecolor{currentstroke}%
\pgfsetstrokeopacity{0.900000}%
\pgfsetdash{}{0pt}%
\pgfsys@defobject{currentmarker}{\pgfqpoint{-0.013889in}{-0.000000in}}{\pgfqpoint{0.013889in}{0.000000in}}{%
\pgfpathmoveto{\pgfqpoint{0.013889in}{-0.000000in}}%
\pgfpathlineto{\pgfqpoint{-0.013889in}{0.000000in}}%
\pgfusepath{stroke,fill}%
}%
\begin{pgfscope}%
\pgfsys@transformshift{4.493557in}{1.649394in}%
\pgfsys@useobject{currentmarker}{}%
\end{pgfscope}%
\end{pgfscope}%
\begin{pgfscope}%
\pgfsetbuttcap%
\pgfsetroundjoin%
\definecolor{currentfill}{rgb}{0.313725,0.317647,0.309804}%
\pgfsetfillcolor{currentfill}%
\pgfsetfillopacity{0.900000}%
\pgfsetlinewidth{1.003750pt}%
\definecolor{currentstroke}{rgb}{0.313725,0.317647,0.309804}%
\pgfsetstrokecolor{currentstroke}%
\pgfsetstrokeopacity{0.900000}%
\pgfsetdash{}{0pt}%
\pgfsys@defobject{currentmarker}{\pgfqpoint{-0.013889in}{-0.000000in}}{\pgfqpoint{0.013889in}{0.000000in}}{%
\pgfpathmoveto{\pgfqpoint{0.013889in}{-0.000000in}}%
\pgfpathlineto{\pgfqpoint{-0.013889in}{0.000000in}}%
\pgfusepath{stroke,fill}%
}%
\begin{pgfscope}%
\pgfsys@transformshift{4.493557in}{1.760506in}%
\pgfsys@useobject{currentmarker}{}%
\end{pgfscope}%
\end{pgfscope}%
\begin{pgfscope}%
\definecolor{textcolor}{rgb}{0.000000,0.000000,0.000000}%
\pgfsetstrokecolor{textcolor}%
\pgfsetfillcolor{textcolor}%
\pgftext[x=4.693557in,y=1.666061in,left,base]{\color{textcolor}\rmfamily\fontsize{8.000000}{9.600000}\selectfont Retro Reco}%
\end{pgfscope}%
\begin{pgfscope}%
\pgfsetbuttcap%
\pgfsetroundjoin%
\pgfsetlinewidth{1.505625pt}%
\definecolor{currentstroke}{rgb}{0.949020,0.372549,0.360784}%
\pgfsetstrokecolor{currentstroke}%
\pgfsetstrokeopacity{0.900000}%
\pgfsetdash{}{0pt}%
\pgfpathmoveto{\pgfqpoint{4.438001in}{1.550061in}}%
\pgfpathlineto{\pgfqpoint{4.549112in}{1.550061in}}%
\pgfusepath{stroke}%
\end{pgfscope}%
\begin{pgfscope}%
\pgfsetbuttcap%
\pgfsetroundjoin%
\pgfsetlinewidth{1.505625pt}%
\definecolor{currentstroke}{rgb}{0.949020,0.372549,0.360784}%
\pgfsetstrokecolor{currentstroke}%
\pgfsetstrokeopacity{0.900000}%
\pgfsetdash{}{0pt}%
\pgfpathmoveto{\pgfqpoint{4.493557in}{1.494506in}}%
\pgfpathlineto{\pgfqpoint{4.493557in}{1.605617in}}%
\pgfusepath{stroke}%
\end{pgfscope}%
\begin{pgfscope}%
\pgfsetbuttcap%
\pgfsetroundjoin%
\definecolor{currentfill}{rgb}{0.949020,0.372549,0.360784}%
\pgfsetfillcolor{currentfill}%
\pgfsetfillopacity{0.900000}%
\pgfsetlinewidth{1.003750pt}%
\definecolor{currentstroke}{rgb}{0.949020,0.372549,0.360784}%
\pgfsetstrokecolor{currentstroke}%
\pgfsetstrokeopacity{0.900000}%
\pgfsetdash{}{0pt}%
\pgfsys@defobject{currentmarker}{\pgfqpoint{0.000000in}{-0.013889in}}{\pgfqpoint{0.000000in}{0.013889in}}{%
\pgfpathmoveto{\pgfqpoint{0.000000in}{-0.013889in}}%
\pgfpathlineto{\pgfqpoint{0.000000in}{0.013889in}}%
\pgfusepath{stroke,fill}%
}%
\begin{pgfscope}%
\pgfsys@transformshift{4.438001in}{1.550061in}%
\pgfsys@useobject{currentmarker}{}%
\end{pgfscope}%
\end{pgfscope}%
\begin{pgfscope}%
\pgfsetbuttcap%
\pgfsetroundjoin%
\definecolor{currentfill}{rgb}{0.949020,0.372549,0.360784}%
\pgfsetfillcolor{currentfill}%
\pgfsetfillopacity{0.900000}%
\pgfsetlinewidth{1.003750pt}%
\definecolor{currentstroke}{rgb}{0.949020,0.372549,0.360784}%
\pgfsetstrokecolor{currentstroke}%
\pgfsetstrokeopacity{0.900000}%
\pgfsetdash{}{0pt}%
\pgfsys@defobject{currentmarker}{\pgfqpoint{0.000000in}{-0.013889in}}{\pgfqpoint{0.000000in}{0.013889in}}{%
\pgfpathmoveto{\pgfqpoint{0.000000in}{-0.013889in}}%
\pgfpathlineto{\pgfqpoint{0.000000in}{0.013889in}}%
\pgfusepath{stroke,fill}%
}%
\begin{pgfscope}%
\pgfsys@transformshift{4.549112in}{1.550061in}%
\pgfsys@useobject{currentmarker}{}%
\end{pgfscope}%
\end{pgfscope}%
\begin{pgfscope}%
\pgfsetbuttcap%
\pgfsetroundjoin%
\definecolor{currentfill}{rgb}{0.949020,0.372549,0.360784}%
\pgfsetfillcolor{currentfill}%
\pgfsetfillopacity{0.900000}%
\pgfsetlinewidth{1.003750pt}%
\definecolor{currentstroke}{rgb}{0.949020,0.372549,0.360784}%
\pgfsetstrokecolor{currentstroke}%
\pgfsetstrokeopacity{0.900000}%
\pgfsetdash{}{0pt}%
\pgfsys@defobject{currentmarker}{\pgfqpoint{-0.013889in}{-0.000000in}}{\pgfqpoint{0.013889in}{0.000000in}}{%
\pgfpathmoveto{\pgfqpoint{0.013889in}{-0.000000in}}%
\pgfpathlineto{\pgfqpoint{-0.013889in}{0.000000in}}%
\pgfusepath{stroke,fill}%
}%
\begin{pgfscope}%
\pgfsys@transformshift{4.493557in}{1.494506in}%
\pgfsys@useobject{currentmarker}{}%
\end{pgfscope}%
\end{pgfscope}%
\begin{pgfscope}%
\pgfsetbuttcap%
\pgfsetroundjoin%
\definecolor{currentfill}{rgb}{0.949020,0.372549,0.360784}%
\pgfsetfillcolor{currentfill}%
\pgfsetfillopacity{0.900000}%
\pgfsetlinewidth{1.003750pt}%
\definecolor{currentstroke}{rgb}{0.949020,0.372549,0.360784}%
\pgfsetstrokecolor{currentstroke}%
\pgfsetstrokeopacity{0.900000}%
\pgfsetdash{}{0pt}%
\pgfsys@defobject{currentmarker}{\pgfqpoint{-0.013889in}{-0.000000in}}{\pgfqpoint{0.013889in}{0.000000in}}{%
\pgfpathmoveto{\pgfqpoint{0.013889in}{-0.000000in}}%
\pgfpathlineto{\pgfqpoint{-0.013889in}{0.000000in}}%
\pgfusepath{stroke,fill}%
}%
\begin{pgfscope}%
\pgfsys@transformshift{4.493557in}{1.605617in}%
\pgfsys@useobject{currentmarker}{}%
\end{pgfscope}%
\end{pgfscope}%
\begin{pgfscope}%
\definecolor{textcolor}{rgb}{0.000000,0.000000,0.000000}%
\pgfsetstrokecolor{textcolor}%
\pgfsetfillcolor{textcolor}%
\pgftext[x=4.693557in,y=1.511172in,left,base]{\color{textcolor}\rmfamily\fontsize{8.000000}{9.600000}\selectfont CubeFlow}%
\end{pgfscope}%
\begin{pgfscope}%
\pgfsetbuttcap%
\pgfsetmiterjoin%
\definecolor{currentfill}{rgb}{1.000000,1.000000,1.000000}%
\pgfsetfillcolor{currentfill}%
\pgfsetlinewidth{0.000000pt}%
\definecolor{currentstroke}{rgb}{0.000000,0.000000,0.000000}%
\pgfsetstrokecolor{currentstroke}%
\pgfsetstrokeopacity{0.000000}%
\pgfsetdash{}{0pt}%
\pgfpathmoveto{\pgfqpoint{0.664741in}{0.553781in}}%
\pgfpathlineto{\pgfqpoint{5.587445in}{0.553781in}}%
\pgfpathlineto{\pgfqpoint{5.587445in}{1.140353in}}%
\pgfpathlineto{\pgfqpoint{0.664741in}{1.140353in}}%
\pgfpathclose%
\pgfusepath{fill}%
\end{pgfscope}%
\begin{pgfscope}%
\pgfpathrectangle{\pgfqpoint{0.664741in}{0.553781in}}{\pgfqpoint{4.922705in}{0.586572in}}%
\pgfusepath{clip}%
\pgfsetbuttcap%
\pgfsetroundjoin%
\definecolor{currentfill}{rgb}{0.313725,0.317647,0.309804}%
\pgfsetfillcolor{currentfill}%
\pgfsetlinewidth{1.003750pt}%
\definecolor{currentstroke}{rgb}{0.313725,0.317647,0.309804}%
\pgfsetstrokecolor{currentstroke}%
\pgfsetdash{}{0pt}%
\pgfsys@defobject{currentmarker}{\pgfqpoint{-0.020833in}{-0.020833in}}{\pgfqpoint{0.020833in}{0.020833in}}{%
\pgfpathmoveto{\pgfqpoint{0.000000in}{-0.020833in}}%
\pgfpathcurveto{\pgfqpoint{0.005525in}{-0.020833in}}{\pgfqpoint{0.010825in}{-0.018638in}}{\pgfqpoint{0.014731in}{-0.014731in}}%
\pgfpathcurveto{\pgfqpoint{0.018638in}{-0.010825in}}{\pgfqpoint{0.020833in}{-0.005525in}}{\pgfqpoint{0.020833in}{0.000000in}}%
\pgfpathcurveto{\pgfqpoint{0.020833in}{0.005525in}}{\pgfqpoint{0.018638in}{0.010825in}}{\pgfqpoint{0.014731in}{0.014731in}}%
\pgfpathcurveto{\pgfqpoint{0.010825in}{0.018638in}}{\pgfqpoint{0.005525in}{0.020833in}}{\pgfqpoint{0.000000in}{0.020833in}}%
\pgfpathcurveto{\pgfqpoint{-0.005525in}{0.020833in}}{\pgfqpoint{-0.010825in}{0.018638in}}{\pgfqpoint{-0.014731in}{0.014731in}}%
\pgfpathcurveto{\pgfqpoint{-0.018638in}{0.010825in}}{\pgfqpoint{-0.020833in}{0.005525in}}{\pgfqpoint{-0.020833in}{0.000000in}}%
\pgfpathcurveto{\pgfqpoint{-0.020833in}{-0.005525in}}{\pgfqpoint{-0.018638in}{-0.010825in}}{\pgfqpoint{-0.014731in}{-0.014731in}}%
\pgfpathcurveto{\pgfqpoint{-0.010825in}{-0.018638in}}{\pgfqpoint{-0.005525in}{-0.020833in}}{\pgfqpoint{0.000000in}{-0.020833in}}%
\pgfpathclose%
\pgfusepath{stroke,fill}%
}%
\begin{pgfscope}%
\pgfsys@transformshift{1.012811in}{0.979560in}%
\pgfsys@useobject{currentmarker}{}%
\end{pgfscope}%
\begin{pgfscope}%
\pgfsys@transformshift{1.261432in}{0.968670in}%
\pgfsys@useobject{currentmarker}{}%
\end{pgfscope}%
\begin{pgfscope}%
\pgfsys@transformshift{1.510054in}{0.954743in}%
\pgfsys@useobject{currentmarker}{}%
\end{pgfscope}%
\begin{pgfscope}%
\pgfsys@transformshift{1.758675in}{0.940877in}%
\pgfsys@useobject{currentmarker}{}%
\end{pgfscope}%
\begin{pgfscope}%
\pgfsys@transformshift{2.007297in}{0.921972in}%
\pgfsys@useobject{currentmarker}{}%
\end{pgfscope}%
\begin{pgfscope}%
\pgfsys@transformshift{2.255918in}{0.900226in}%
\pgfsys@useobject{currentmarker}{}%
\end{pgfscope}%
\begin{pgfscope}%
\pgfsys@transformshift{2.504539in}{0.879741in}%
\pgfsys@useobject{currentmarker}{}%
\end{pgfscope}%
\begin{pgfscope}%
\pgfsys@transformshift{2.753161in}{0.869656in}%
\pgfsys@useobject{currentmarker}{}%
\end{pgfscope}%
\begin{pgfscope}%
\pgfsys@transformshift{3.001782in}{0.860710in}%
\pgfsys@useobject{currentmarker}{}%
\end{pgfscope}%
\begin{pgfscope}%
\pgfsys@transformshift{3.250404in}{0.858777in}%
\pgfsys@useobject{currentmarker}{}%
\end{pgfscope}%
\begin{pgfscope}%
\pgfsys@transformshift{3.499025in}{0.852349in}%
\pgfsys@useobject{currentmarker}{}%
\end{pgfscope}%
\begin{pgfscope}%
\pgfsys@transformshift{3.747647in}{0.841485in}%
\pgfsys@useobject{currentmarker}{}%
\end{pgfscope}%
\begin{pgfscope}%
\pgfsys@transformshift{3.996268in}{0.825648in}%
\pgfsys@useobject{currentmarker}{}%
\end{pgfscope}%
\begin{pgfscope}%
\pgfsys@transformshift{4.244890in}{0.806092in}%
\pgfsys@useobject{currentmarker}{}%
\end{pgfscope}%
\begin{pgfscope}%
\pgfsys@transformshift{4.493511in}{0.791970in}%
\pgfsys@useobject{currentmarker}{}%
\end{pgfscope}%
\begin{pgfscope}%
\pgfsys@transformshift{4.742132in}{0.795182in}%
\pgfsys@useobject{currentmarker}{}%
\end{pgfscope}%
\begin{pgfscope}%
\pgfsys@transformshift{4.990754in}{0.810888in}%
\pgfsys@useobject{currentmarker}{}%
\end{pgfscope}%
\begin{pgfscope}%
\pgfsys@transformshift{5.239375in}{0.841409in}%
\pgfsys@useobject{currentmarker}{}%
\end{pgfscope}%
\end{pgfscope}%
\begin{pgfscope}%
\pgfsetbuttcap%
\pgfsetroundjoin%
\definecolor{currentfill}{rgb}{0.000000,0.000000,0.000000}%
\pgfsetfillcolor{currentfill}%
\pgfsetlinewidth{0.803000pt}%
\definecolor{currentstroke}{rgb}{0.000000,0.000000,0.000000}%
\pgfsetstrokecolor{currentstroke}%
\pgfsetdash{}{0pt}%
\pgfsys@defobject{currentmarker}{\pgfqpoint{0.000000in}{-0.048611in}}{\pgfqpoint{0.000000in}{0.000000in}}{%
\pgfpathmoveto{\pgfqpoint{0.000000in}{0.000000in}}%
\pgfpathlineto{\pgfqpoint{0.000000in}{-0.048611in}}%
\pgfusepath{stroke,fill}%
}%
\begin{pgfscope}%
\pgfsys@transformshift{0.888500in}{0.553781in}%
\pgfsys@useobject{currentmarker}{}%
\end{pgfscope}%
\end{pgfscope}%
\begin{pgfscope}%
\definecolor{textcolor}{rgb}{0.000000,0.000000,0.000000}%
\pgfsetstrokecolor{textcolor}%
\pgfsetfillcolor{textcolor}%
\pgftext[x=0.888500in,y=0.456558in,,top]{\color{textcolor}\rmfamily\fontsize{8.000000}{9.600000}\selectfont \(\displaystyle {0.0}\)}%
\end{pgfscope}%
\begin{pgfscope}%
\pgfsetbuttcap%
\pgfsetroundjoin%
\definecolor{currentfill}{rgb}{0.000000,0.000000,0.000000}%
\pgfsetfillcolor{currentfill}%
\pgfsetlinewidth{0.803000pt}%
\definecolor{currentstroke}{rgb}{0.000000,0.000000,0.000000}%
\pgfsetstrokecolor{currentstroke}%
\pgfsetdash{}{0pt}%
\pgfsys@defobject{currentmarker}{\pgfqpoint{0.000000in}{-0.048611in}}{\pgfqpoint{0.000000in}{0.000000in}}{%
\pgfpathmoveto{\pgfqpoint{0.000000in}{0.000000in}}%
\pgfpathlineto{\pgfqpoint{0.000000in}{-0.048611in}}%
\pgfusepath{stroke,fill}%
}%
\begin{pgfscope}%
\pgfsys@transformshift{1.634364in}{0.553781in}%
\pgfsys@useobject{currentmarker}{}%
\end{pgfscope}%
\end{pgfscope}%
\begin{pgfscope}%
\definecolor{textcolor}{rgb}{0.000000,0.000000,0.000000}%
\pgfsetstrokecolor{textcolor}%
\pgfsetfillcolor{textcolor}%
\pgftext[x=1.634364in,y=0.456558in,,top]{\color{textcolor}\rmfamily\fontsize{8.000000}{9.600000}\selectfont \(\displaystyle {0.5}\)}%
\end{pgfscope}%
\begin{pgfscope}%
\pgfsetbuttcap%
\pgfsetroundjoin%
\definecolor{currentfill}{rgb}{0.000000,0.000000,0.000000}%
\pgfsetfillcolor{currentfill}%
\pgfsetlinewidth{0.803000pt}%
\definecolor{currentstroke}{rgb}{0.000000,0.000000,0.000000}%
\pgfsetstrokecolor{currentstroke}%
\pgfsetdash{}{0pt}%
\pgfsys@defobject{currentmarker}{\pgfqpoint{0.000000in}{-0.048611in}}{\pgfqpoint{0.000000in}{0.000000in}}{%
\pgfpathmoveto{\pgfqpoint{0.000000in}{0.000000in}}%
\pgfpathlineto{\pgfqpoint{0.000000in}{-0.048611in}}%
\pgfusepath{stroke,fill}%
}%
\begin{pgfscope}%
\pgfsys@transformshift{2.380229in}{0.553781in}%
\pgfsys@useobject{currentmarker}{}%
\end{pgfscope}%
\end{pgfscope}%
\begin{pgfscope}%
\definecolor{textcolor}{rgb}{0.000000,0.000000,0.000000}%
\pgfsetstrokecolor{textcolor}%
\pgfsetfillcolor{textcolor}%
\pgftext[x=2.380229in,y=0.456558in,,top]{\color{textcolor}\rmfamily\fontsize{8.000000}{9.600000}\selectfont \(\displaystyle {1.0}\)}%
\end{pgfscope}%
\begin{pgfscope}%
\pgfsetbuttcap%
\pgfsetroundjoin%
\definecolor{currentfill}{rgb}{0.000000,0.000000,0.000000}%
\pgfsetfillcolor{currentfill}%
\pgfsetlinewidth{0.803000pt}%
\definecolor{currentstroke}{rgb}{0.000000,0.000000,0.000000}%
\pgfsetstrokecolor{currentstroke}%
\pgfsetdash{}{0pt}%
\pgfsys@defobject{currentmarker}{\pgfqpoint{0.000000in}{-0.048611in}}{\pgfqpoint{0.000000in}{0.000000in}}{%
\pgfpathmoveto{\pgfqpoint{0.000000in}{0.000000in}}%
\pgfpathlineto{\pgfqpoint{0.000000in}{-0.048611in}}%
\pgfusepath{stroke,fill}%
}%
\begin{pgfscope}%
\pgfsys@transformshift{3.126093in}{0.553781in}%
\pgfsys@useobject{currentmarker}{}%
\end{pgfscope}%
\end{pgfscope}%
\begin{pgfscope}%
\definecolor{textcolor}{rgb}{0.000000,0.000000,0.000000}%
\pgfsetstrokecolor{textcolor}%
\pgfsetfillcolor{textcolor}%
\pgftext[x=3.126093in,y=0.456558in,,top]{\color{textcolor}\rmfamily\fontsize{8.000000}{9.600000}\selectfont \(\displaystyle {1.5}\)}%
\end{pgfscope}%
\begin{pgfscope}%
\pgfsetbuttcap%
\pgfsetroundjoin%
\definecolor{currentfill}{rgb}{0.000000,0.000000,0.000000}%
\pgfsetfillcolor{currentfill}%
\pgfsetlinewidth{0.803000pt}%
\definecolor{currentstroke}{rgb}{0.000000,0.000000,0.000000}%
\pgfsetstrokecolor{currentstroke}%
\pgfsetdash{}{0pt}%
\pgfsys@defobject{currentmarker}{\pgfqpoint{0.000000in}{-0.048611in}}{\pgfqpoint{0.000000in}{0.000000in}}{%
\pgfpathmoveto{\pgfqpoint{0.000000in}{0.000000in}}%
\pgfpathlineto{\pgfqpoint{0.000000in}{-0.048611in}}%
\pgfusepath{stroke,fill}%
}%
\begin{pgfscope}%
\pgfsys@transformshift{3.871957in}{0.553781in}%
\pgfsys@useobject{currentmarker}{}%
\end{pgfscope}%
\end{pgfscope}%
\begin{pgfscope}%
\definecolor{textcolor}{rgb}{0.000000,0.000000,0.000000}%
\pgfsetstrokecolor{textcolor}%
\pgfsetfillcolor{textcolor}%
\pgftext[x=3.871957in,y=0.456558in,,top]{\color{textcolor}\rmfamily\fontsize{8.000000}{9.600000}\selectfont \(\displaystyle {2.0}\)}%
\end{pgfscope}%
\begin{pgfscope}%
\pgfsetbuttcap%
\pgfsetroundjoin%
\definecolor{currentfill}{rgb}{0.000000,0.000000,0.000000}%
\pgfsetfillcolor{currentfill}%
\pgfsetlinewidth{0.803000pt}%
\definecolor{currentstroke}{rgb}{0.000000,0.000000,0.000000}%
\pgfsetstrokecolor{currentstroke}%
\pgfsetdash{}{0pt}%
\pgfsys@defobject{currentmarker}{\pgfqpoint{0.000000in}{-0.048611in}}{\pgfqpoint{0.000000in}{0.000000in}}{%
\pgfpathmoveto{\pgfqpoint{0.000000in}{0.000000in}}%
\pgfpathlineto{\pgfqpoint{0.000000in}{-0.048611in}}%
\pgfusepath{stroke,fill}%
}%
\begin{pgfscope}%
\pgfsys@transformshift{4.617822in}{0.553781in}%
\pgfsys@useobject{currentmarker}{}%
\end{pgfscope}%
\end{pgfscope}%
\begin{pgfscope}%
\definecolor{textcolor}{rgb}{0.000000,0.000000,0.000000}%
\pgfsetstrokecolor{textcolor}%
\pgfsetfillcolor{textcolor}%
\pgftext[x=4.617822in,y=0.456558in,,top]{\color{textcolor}\rmfamily\fontsize{8.000000}{9.600000}\selectfont \(\displaystyle {2.5}\)}%
\end{pgfscope}%
\begin{pgfscope}%
\pgfsetbuttcap%
\pgfsetroundjoin%
\definecolor{currentfill}{rgb}{0.000000,0.000000,0.000000}%
\pgfsetfillcolor{currentfill}%
\pgfsetlinewidth{0.803000pt}%
\definecolor{currentstroke}{rgb}{0.000000,0.000000,0.000000}%
\pgfsetstrokecolor{currentstroke}%
\pgfsetdash{}{0pt}%
\pgfsys@defobject{currentmarker}{\pgfqpoint{0.000000in}{-0.048611in}}{\pgfqpoint{0.000000in}{0.000000in}}{%
\pgfpathmoveto{\pgfqpoint{0.000000in}{0.000000in}}%
\pgfpathlineto{\pgfqpoint{0.000000in}{-0.048611in}}%
\pgfusepath{stroke,fill}%
}%
\begin{pgfscope}%
\pgfsys@transformshift{5.363686in}{0.553781in}%
\pgfsys@useobject{currentmarker}{}%
\end{pgfscope}%
\end{pgfscope}%
\begin{pgfscope}%
\definecolor{textcolor}{rgb}{0.000000,0.000000,0.000000}%
\pgfsetstrokecolor{textcolor}%
\pgfsetfillcolor{textcolor}%
\pgftext[x=5.363686in,y=0.456558in,,top]{\color{textcolor}\rmfamily\fontsize{8.000000}{9.600000}\selectfont \(\displaystyle {3.0}\)}%
\end{pgfscope}%
\begin{pgfscope}%
\definecolor{textcolor}{rgb}{0.000000,0.000000,0.000000}%
\pgfsetstrokecolor{textcolor}%
\pgfsetfillcolor{textcolor}%
\pgftext[x=3.126093in,y=0.302336in,,top]{\color{textcolor}\rmfamily\fontsize{10.950000}{13.140000}\selectfont \(\displaystyle \log_{10}(E_{\textup{true}}) \, \left[ E / \textup{GeV} \right]\)}%
\end{pgfscope}%
\begin{pgfscope}%
\pgfsetbuttcap%
\pgfsetroundjoin%
\definecolor{currentfill}{rgb}{0.000000,0.000000,0.000000}%
\pgfsetfillcolor{currentfill}%
\pgfsetlinewidth{0.803000pt}%
\definecolor{currentstroke}{rgb}{0.000000,0.000000,0.000000}%
\pgfsetstrokecolor{currentstroke}%
\pgfsetdash{}{0pt}%
\pgfsys@defobject{currentmarker}{\pgfqpoint{-0.048611in}{0.000000in}}{\pgfqpoint{-0.000000in}{0.000000in}}{%
\pgfpathmoveto{\pgfqpoint{-0.000000in}{0.000000in}}%
\pgfpathlineto{\pgfqpoint{-0.048611in}{0.000000in}}%
\pgfusepath{stroke,fill}%
}%
\begin{pgfscope}%
\pgfsys@transformshift{0.664741in}{0.553781in}%
\pgfsys@useobject{currentmarker}{}%
\end{pgfscope}%
\end{pgfscope}%
\begin{pgfscope}%
\definecolor{textcolor}{rgb}{0.000000,0.000000,0.000000}%
\pgfsetstrokecolor{textcolor}%
\pgfsetfillcolor{textcolor}%
\pgftext[x=0.416712in, y=0.515225in, left, base]{\color{textcolor}\rmfamily\fontsize{8.000000}{9.600000}\selectfont \(\displaystyle {−1}\)}%
\end{pgfscope}%
\begin{pgfscope}%
\pgfsetbuttcap%
\pgfsetroundjoin%
\definecolor{currentfill}{rgb}{0.000000,0.000000,0.000000}%
\pgfsetfillcolor{currentfill}%
\pgfsetlinewidth{0.803000pt}%
\definecolor{currentstroke}{rgb}{0.000000,0.000000,0.000000}%
\pgfsetstrokecolor{currentstroke}%
\pgfsetdash{}{0pt}%
\pgfsys@defobject{currentmarker}{\pgfqpoint{-0.048611in}{0.000000in}}{\pgfqpoint{-0.000000in}{0.000000in}}{%
\pgfpathmoveto{\pgfqpoint{-0.000000in}{0.000000in}}%
\pgfpathlineto{\pgfqpoint{-0.048611in}{0.000000in}}%
\pgfusepath{stroke,fill}%
}%
\begin{pgfscope}%
\pgfsys@transformshift{0.664741in}{0.847067in}%
\pgfsys@useobject{currentmarker}{}%
\end{pgfscope}%
\end{pgfscope}%
\begin{pgfscope}%
\definecolor{textcolor}{rgb}{0.000000,0.000000,0.000000}%
\pgfsetstrokecolor{textcolor}%
\pgfsetfillcolor{textcolor}%
\pgftext[x=0.508490in, y=0.808511in, left, base]{\color{textcolor}\rmfamily\fontsize{8.000000}{9.600000}\selectfont \(\displaystyle {0}\)}%
\end{pgfscope}%
\begin{pgfscope}%
\pgfsetbuttcap%
\pgfsetroundjoin%
\definecolor{currentfill}{rgb}{0.000000,0.000000,0.000000}%
\pgfsetfillcolor{currentfill}%
\pgfsetlinewidth{0.803000pt}%
\definecolor{currentstroke}{rgb}{0.000000,0.000000,0.000000}%
\pgfsetstrokecolor{currentstroke}%
\pgfsetdash{}{0pt}%
\pgfsys@defobject{currentmarker}{\pgfqpoint{-0.048611in}{0.000000in}}{\pgfqpoint{-0.000000in}{0.000000in}}{%
\pgfpathmoveto{\pgfqpoint{-0.000000in}{0.000000in}}%
\pgfpathlineto{\pgfqpoint{-0.048611in}{0.000000in}}%
\pgfusepath{stroke,fill}%
}%
\begin{pgfscope}%
\pgfsys@transformshift{0.664741in}{1.140353in}%
\pgfsys@useobject{currentmarker}{}%
\end{pgfscope}%
\end{pgfscope}%
\begin{pgfscope}%
\definecolor{textcolor}{rgb}{0.000000,0.000000,0.000000}%
\pgfsetstrokecolor{textcolor}%
\pgfsetfillcolor{textcolor}%
\pgftext[x=0.508490in, y=1.101797in, left, base]{\color{textcolor}\rmfamily\fontsize{8.000000}{9.600000}\selectfont \(\displaystyle {1}\)}%
\end{pgfscope}%
\begin{pgfscope}%
\definecolor{textcolor}{rgb}{0.000000,0.000000,0.000000}%
\pgfsetstrokecolor{textcolor}%
\pgfsetfillcolor{textcolor}%
\pgftext[x=0.361157in,y=0.847067in,,bottom,rotate=90.000000]{\color{textcolor}\rmfamily\fontsize{10.950000}{13.140000}\selectfont Rel. imp.}%
\end{pgfscope}%
\begin{pgfscope}%
\pgfpathrectangle{\pgfqpoint{0.664741in}{0.553781in}}{\pgfqpoint{4.922705in}{0.586572in}}%
\pgfusepath{clip}%
\pgfsetbuttcap%
\pgfsetroundjoin%
\pgfsetlinewidth{1.003750pt}%
\definecolor{currentstroke}{rgb}{0.501961,0.501961,0.501961}%
\pgfsetstrokecolor{currentstroke}%
\pgfsetdash{{3.700000pt}{1.600000pt}}{0.000000pt}%
\pgfpathmoveto{\pgfqpoint{0.664741in}{0.847067in}}%
\pgfpathlineto{\pgfqpoint{5.587445in}{0.847067in}}%
\pgfusepath{stroke}%
\end{pgfscope}%
\begin{pgfscope}%
\pgfsetrectcap%
\pgfsetmiterjoin%
\pgfsetlinewidth{0.803000pt}%
\definecolor{currentstroke}{rgb}{0.000000,0.000000,0.000000}%
\pgfsetstrokecolor{currentstroke}%
\pgfsetdash{}{0pt}%
\pgfpathmoveto{\pgfqpoint{0.664741in}{0.553781in}}%
\pgfpathlineto{\pgfqpoint{0.664741in}{1.140353in}}%
\pgfusepath{stroke}%
\end{pgfscope}%
\begin{pgfscope}%
\pgfsetrectcap%
\pgfsetmiterjoin%
\pgfsetlinewidth{0.803000pt}%
\definecolor{currentstroke}{rgb}{0.000000,0.000000,0.000000}%
\pgfsetstrokecolor{currentstroke}%
\pgfsetdash{}{0pt}%
\pgfpathmoveto{\pgfqpoint{5.587445in}{0.553781in}}%
\pgfpathlineto{\pgfqpoint{5.587445in}{1.140353in}}%
\pgfusepath{stroke}%
\end{pgfscope}%
\begin{pgfscope}%
\pgfsetrectcap%
\pgfsetmiterjoin%
\pgfsetlinewidth{0.803000pt}%
\definecolor{currentstroke}{rgb}{0.000000,0.000000,0.000000}%
\pgfsetstrokecolor{currentstroke}%
\pgfsetdash{}{0pt}%
\pgfpathmoveto{\pgfqpoint{0.664741in}{0.553781in}}%
\pgfpathlineto{\pgfqpoint{5.587445in}{0.553781in}}%
\pgfusepath{stroke}%
\end{pgfscope}%
\begin{pgfscope}%
\pgfsetrectcap%
\pgfsetmiterjoin%
\pgfsetlinewidth{0.803000pt}%
\definecolor{currentstroke}{rgb}{0.000000,0.000000,0.000000}%
\pgfsetstrokecolor{currentstroke}%
\pgfsetdash{}{0pt}%
\pgfpathmoveto{\pgfqpoint{0.664741in}{1.140353in}}%
\pgfpathlineto{\pgfqpoint{5.587445in}{1.140353in}}%
\pgfusepath{stroke}%
\end{pgfscope}%
\begin{pgfscope}%
\pgfpathrectangle{\pgfqpoint{0.664741in}{1.389617in}}{\pgfqpoint{4.922705in}{1.759717in}}%
\pgfusepath{clip}%
\pgfsetbuttcap%
\pgfsetmiterjoin%
\definecolor{currentfill}{rgb}{0.501961,0.501961,0.501961}%
\pgfsetfillcolor{currentfill}%
\pgfsetfillopacity{0.200000}%
\pgfsetlinewidth{0.000000pt}%
\definecolor{currentstroke}{rgb}{0.000000,0.000000,0.000000}%
\pgfsetstrokecolor{currentstroke}%
\pgfsetstrokeopacity{0.200000}%
\pgfsetdash{}{0pt}%
\pgfpathmoveto{\pgfqpoint{0.888500in}{1.389617in}}%
\pgfpathlineto{\pgfqpoint{1.137121in}{1.389617in}}%
\pgfpathlineto{\pgfqpoint{1.137121in}{2.537817in}}%
\pgfpathlineto{\pgfqpoint{0.888500in}{2.537817in}}%
\pgfpathclose%
\pgfusepath{fill}%
\end{pgfscope}%
\begin{pgfscope}%
\pgfpathrectangle{\pgfqpoint{0.664741in}{1.389617in}}{\pgfqpoint{4.922705in}{1.759717in}}%
\pgfusepath{clip}%
\pgfsetbuttcap%
\pgfsetmiterjoin%
\definecolor{currentfill}{rgb}{0.501961,0.501961,0.501961}%
\pgfsetfillcolor{currentfill}%
\pgfsetfillopacity{0.200000}%
\pgfsetlinewidth{0.000000pt}%
\definecolor{currentstroke}{rgb}{0.000000,0.000000,0.000000}%
\pgfsetstrokecolor{currentstroke}%
\pgfsetstrokeopacity{0.200000}%
\pgfsetdash{}{0pt}%
\pgfpathmoveto{\pgfqpoint{1.137121in}{1.389617in}}%
\pgfpathlineto{\pgfqpoint{1.385742in}{1.389617in}}%
\pgfpathlineto{\pgfqpoint{1.385742in}{2.704582in}}%
\pgfpathlineto{\pgfqpoint{1.137121in}{2.704582in}}%
\pgfpathclose%
\pgfusepath{fill}%
\end{pgfscope}%
\begin{pgfscope}%
\pgfpathrectangle{\pgfqpoint{0.664741in}{1.389617in}}{\pgfqpoint{4.922705in}{1.759717in}}%
\pgfusepath{clip}%
\pgfsetbuttcap%
\pgfsetmiterjoin%
\definecolor{currentfill}{rgb}{0.501961,0.501961,0.501961}%
\pgfsetfillcolor{currentfill}%
\pgfsetfillopacity{0.200000}%
\pgfsetlinewidth{0.000000pt}%
\definecolor{currentstroke}{rgb}{0.000000,0.000000,0.000000}%
\pgfsetstrokecolor{currentstroke}%
\pgfsetstrokeopacity{0.200000}%
\pgfsetdash{}{0pt}%
\pgfpathmoveto{\pgfqpoint{1.385742in}{1.389617in}}%
\pgfpathlineto{\pgfqpoint{1.634362in}{1.389617in}}%
\pgfpathlineto{\pgfqpoint{1.634362in}{2.834445in}}%
\pgfpathlineto{\pgfqpoint{1.385742in}{2.834445in}}%
\pgfpathclose%
\pgfusepath{fill}%
\end{pgfscope}%
\begin{pgfscope}%
\pgfpathrectangle{\pgfqpoint{0.664741in}{1.389617in}}{\pgfqpoint{4.922705in}{1.759717in}}%
\pgfusepath{clip}%
\pgfsetbuttcap%
\pgfsetmiterjoin%
\definecolor{currentfill}{rgb}{0.501961,0.501961,0.501961}%
\pgfsetfillcolor{currentfill}%
\pgfsetfillopacity{0.200000}%
\pgfsetlinewidth{0.000000pt}%
\definecolor{currentstroke}{rgb}{0.000000,0.000000,0.000000}%
\pgfsetstrokecolor{currentstroke}%
\pgfsetstrokeopacity{0.200000}%
\pgfsetdash{}{0pt}%
\pgfpathmoveto{\pgfqpoint{1.634362in}{1.389617in}}%
\pgfpathlineto{\pgfqpoint{1.882983in}{1.389617in}}%
\pgfpathlineto{\pgfqpoint{1.882983in}{2.929819in}}%
\pgfpathlineto{\pgfqpoint{1.634362in}{2.929819in}}%
\pgfpathclose%
\pgfusepath{fill}%
\end{pgfscope}%
\begin{pgfscope}%
\pgfpathrectangle{\pgfqpoint{0.664741in}{1.389617in}}{\pgfqpoint{4.922705in}{1.759717in}}%
\pgfusepath{clip}%
\pgfsetbuttcap%
\pgfsetmiterjoin%
\definecolor{currentfill}{rgb}{0.501961,0.501961,0.501961}%
\pgfsetfillcolor{currentfill}%
\pgfsetfillopacity{0.200000}%
\pgfsetlinewidth{0.000000pt}%
\definecolor{currentstroke}{rgb}{0.000000,0.000000,0.000000}%
\pgfsetstrokecolor{currentstroke}%
\pgfsetstrokeopacity{0.200000}%
\pgfsetdash{}{0pt}%
\pgfpathmoveto{\pgfqpoint{1.882983in}{1.389617in}}%
\pgfpathlineto{\pgfqpoint{2.131604in}{1.389617in}}%
\pgfpathlineto{\pgfqpoint{2.131604in}{2.993403in}}%
\pgfpathlineto{\pgfqpoint{1.882983in}{2.993403in}}%
\pgfpathclose%
\pgfusepath{fill}%
\end{pgfscope}%
\begin{pgfscope}%
\pgfpathrectangle{\pgfqpoint{0.664741in}{1.389617in}}{\pgfqpoint{4.922705in}{1.759717in}}%
\pgfusepath{clip}%
\pgfsetbuttcap%
\pgfsetmiterjoin%
\definecolor{currentfill}{rgb}{0.501961,0.501961,0.501961}%
\pgfsetfillcolor{currentfill}%
\pgfsetfillopacity{0.200000}%
\pgfsetlinewidth{0.000000pt}%
\definecolor{currentstroke}{rgb}{0.000000,0.000000,0.000000}%
\pgfsetstrokecolor{currentstroke}%
\pgfsetstrokeopacity{0.200000}%
\pgfsetdash{}{0pt}%
\pgfpathmoveto{\pgfqpoint{2.131604in}{1.389617in}}%
\pgfpathlineto{\pgfqpoint{2.380224in}{1.389617in}}%
\pgfpathlineto{\pgfqpoint{2.380224in}{3.035875in}}%
\pgfpathlineto{\pgfqpoint{2.131604in}{3.035875in}}%
\pgfpathclose%
\pgfusepath{fill}%
\end{pgfscope}%
\begin{pgfscope}%
\pgfpathrectangle{\pgfqpoint{0.664741in}{1.389617in}}{\pgfqpoint{4.922705in}{1.759717in}}%
\pgfusepath{clip}%
\pgfsetbuttcap%
\pgfsetmiterjoin%
\definecolor{currentfill}{rgb}{0.501961,0.501961,0.501961}%
\pgfsetfillcolor{currentfill}%
\pgfsetfillopacity{0.200000}%
\pgfsetlinewidth{0.000000pt}%
\definecolor{currentstroke}{rgb}{0.000000,0.000000,0.000000}%
\pgfsetstrokecolor{currentstroke}%
\pgfsetstrokeopacity{0.200000}%
\pgfsetdash{}{0pt}%
\pgfpathmoveto{\pgfqpoint{2.380224in}{1.389617in}}%
\pgfpathlineto{\pgfqpoint{2.628845in}{1.389617in}}%
\pgfpathlineto{\pgfqpoint{2.628845in}{3.059103in}}%
\pgfpathlineto{\pgfqpoint{2.380224in}{3.059103in}}%
\pgfpathclose%
\pgfusepath{fill}%
\end{pgfscope}%
\begin{pgfscope}%
\pgfpathrectangle{\pgfqpoint{0.664741in}{1.389617in}}{\pgfqpoint{4.922705in}{1.759717in}}%
\pgfusepath{clip}%
\pgfsetbuttcap%
\pgfsetmiterjoin%
\definecolor{currentfill}{rgb}{0.501961,0.501961,0.501961}%
\pgfsetfillcolor{currentfill}%
\pgfsetfillopacity{0.200000}%
\pgfsetlinewidth{0.000000pt}%
\definecolor{currentstroke}{rgb}{0.000000,0.000000,0.000000}%
\pgfsetstrokecolor{currentstroke}%
\pgfsetstrokeopacity{0.200000}%
\pgfsetdash{}{0pt}%
\pgfpathmoveto{\pgfqpoint{2.628845in}{1.389617in}}%
\pgfpathlineto{\pgfqpoint{2.877466in}{1.389617in}}%
\pgfpathlineto{\pgfqpoint{2.877466in}{3.065537in}}%
\pgfpathlineto{\pgfqpoint{2.628845in}{3.065537in}}%
\pgfpathclose%
\pgfusepath{fill}%
\end{pgfscope}%
\begin{pgfscope}%
\pgfpathrectangle{\pgfqpoint{0.664741in}{1.389617in}}{\pgfqpoint{4.922705in}{1.759717in}}%
\pgfusepath{clip}%
\pgfsetbuttcap%
\pgfsetmiterjoin%
\definecolor{currentfill}{rgb}{0.501961,0.501961,0.501961}%
\pgfsetfillcolor{currentfill}%
\pgfsetfillopacity{0.200000}%
\pgfsetlinewidth{0.000000pt}%
\definecolor{currentstroke}{rgb}{0.000000,0.000000,0.000000}%
\pgfsetstrokecolor{currentstroke}%
\pgfsetstrokeopacity{0.200000}%
\pgfsetdash{}{0pt}%
\pgfpathmoveto{\pgfqpoint{2.877466in}{1.389617in}}%
\pgfpathlineto{\pgfqpoint{3.126086in}{1.389617in}}%
\pgfpathlineto{\pgfqpoint{3.126086in}{3.057480in}}%
\pgfpathlineto{\pgfqpoint{2.877466in}{3.057480in}}%
\pgfpathclose%
\pgfusepath{fill}%
\end{pgfscope}%
\begin{pgfscope}%
\pgfpathrectangle{\pgfqpoint{0.664741in}{1.389617in}}{\pgfqpoint{4.922705in}{1.759717in}}%
\pgfusepath{clip}%
\pgfsetbuttcap%
\pgfsetmiterjoin%
\definecolor{currentfill}{rgb}{0.501961,0.501961,0.501961}%
\pgfsetfillcolor{currentfill}%
\pgfsetfillopacity{0.200000}%
\pgfsetlinewidth{0.000000pt}%
\definecolor{currentstroke}{rgb}{0.000000,0.000000,0.000000}%
\pgfsetstrokecolor{currentstroke}%
\pgfsetstrokeopacity{0.200000}%
\pgfsetdash{}{0pt}%
\pgfpathmoveto{\pgfqpoint{3.126086in}{1.389617in}}%
\pgfpathlineto{\pgfqpoint{3.374707in}{1.389617in}}%
\pgfpathlineto{\pgfqpoint{3.374707in}{3.038785in}}%
\pgfpathlineto{\pgfqpoint{3.126086in}{3.038785in}}%
\pgfpathclose%
\pgfusepath{fill}%
\end{pgfscope}%
\begin{pgfscope}%
\pgfpathrectangle{\pgfqpoint{0.664741in}{1.389617in}}{\pgfqpoint{4.922705in}{1.759717in}}%
\pgfusepath{clip}%
\pgfsetbuttcap%
\pgfsetmiterjoin%
\definecolor{currentfill}{rgb}{0.501961,0.501961,0.501961}%
\pgfsetfillcolor{currentfill}%
\pgfsetfillopacity{0.200000}%
\pgfsetlinewidth{0.000000pt}%
\definecolor{currentstroke}{rgb}{0.000000,0.000000,0.000000}%
\pgfsetstrokecolor{currentstroke}%
\pgfsetstrokeopacity{0.200000}%
\pgfsetdash{}{0pt}%
\pgfpathmoveto{\pgfqpoint{3.374707in}{1.389617in}}%
\pgfpathlineto{\pgfqpoint{3.623328in}{1.389617in}}%
\pgfpathlineto{\pgfqpoint{3.623328in}{3.010615in}}%
\pgfpathlineto{\pgfqpoint{3.374707in}{3.010615in}}%
\pgfpathclose%
\pgfusepath{fill}%
\end{pgfscope}%
\begin{pgfscope}%
\pgfpathrectangle{\pgfqpoint{0.664741in}{1.389617in}}{\pgfqpoint{4.922705in}{1.759717in}}%
\pgfusepath{clip}%
\pgfsetbuttcap%
\pgfsetmiterjoin%
\definecolor{currentfill}{rgb}{0.501961,0.501961,0.501961}%
\pgfsetfillcolor{currentfill}%
\pgfsetfillopacity{0.200000}%
\pgfsetlinewidth{0.000000pt}%
\definecolor{currentstroke}{rgb}{0.000000,0.000000,0.000000}%
\pgfsetstrokecolor{currentstroke}%
\pgfsetstrokeopacity{0.200000}%
\pgfsetdash{}{0pt}%
\pgfpathmoveto{\pgfqpoint{3.623328in}{1.389617in}}%
\pgfpathlineto{\pgfqpoint{3.871948in}{1.389617in}}%
\pgfpathlineto{\pgfqpoint{3.871948in}{2.973571in}}%
\pgfpathlineto{\pgfqpoint{3.623328in}{2.973571in}}%
\pgfpathclose%
\pgfusepath{fill}%
\end{pgfscope}%
\begin{pgfscope}%
\pgfpathrectangle{\pgfqpoint{0.664741in}{1.389617in}}{\pgfqpoint{4.922705in}{1.759717in}}%
\pgfusepath{clip}%
\pgfsetbuttcap%
\pgfsetmiterjoin%
\definecolor{currentfill}{rgb}{0.501961,0.501961,0.501961}%
\pgfsetfillcolor{currentfill}%
\pgfsetfillopacity{0.200000}%
\pgfsetlinewidth{0.000000pt}%
\definecolor{currentstroke}{rgb}{0.000000,0.000000,0.000000}%
\pgfsetstrokecolor{currentstroke}%
\pgfsetstrokeopacity{0.200000}%
\pgfsetdash{}{0pt}%
\pgfpathmoveto{\pgfqpoint{3.871948in}{1.389617in}}%
\pgfpathlineto{\pgfqpoint{4.120569in}{1.389617in}}%
\pgfpathlineto{\pgfqpoint{4.120569in}{2.931005in}}%
\pgfpathlineto{\pgfqpoint{3.871948in}{2.931005in}}%
\pgfpathclose%
\pgfusepath{fill}%
\end{pgfscope}%
\begin{pgfscope}%
\pgfpathrectangle{\pgfqpoint{0.664741in}{1.389617in}}{\pgfqpoint{4.922705in}{1.759717in}}%
\pgfusepath{clip}%
\pgfsetbuttcap%
\pgfsetmiterjoin%
\definecolor{currentfill}{rgb}{0.501961,0.501961,0.501961}%
\pgfsetfillcolor{currentfill}%
\pgfsetfillopacity{0.200000}%
\pgfsetlinewidth{0.000000pt}%
\definecolor{currentstroke}{rgb}{0.000000,0.000000,0.000000}%
\pgfsetstrokecolor{currentstroke}%
\pgfsetstrokeopacity{0.200000}%
\pgfsetdash{}{0pt}%
\pgfpathmoveto{\pgfqpoint{4.120569in}{1.389617in}}%
\pgfpathlineto{\pgfqpoint{4.369189in}{1.389617in}}%
\pgfpathlineto{\pgfqpoint{4.369189in}{2.884289in}}%
\pgfpathlineto{\pgfqpoint{4.120569in}{2.884289in}}%
\pgfpathclose%
\pgfusepath{fill}%
\end{pgfscope}%
\begin{pgfscope}%
\pgfpathrectangle{\pgfqpoint{0.664741in}{1.389617in}}{\pgfqpoint{4.922705in}{1.759717in}}%
\pgfusepath{clip}%
\pgfsetbuttcap%
\pgfsetmiterjoin%
\definecolor{currentfill}{rgb}{0.501961,0.501961,0.501961}%
\pgfsetfillcolor{currentfill}%
\pgfsetfillopacity{0.200000}%
\pgfsetlinewidth{0.000000pt}%
\definecolor{currentstroke}{rgb}{0.000000,0.000000,0.000000}%
\pgfsetstrokecolor{currentstroke}%
\pgfsetstrokeopacity{0.200000}%
\pgfsetdash{}{0pt}%
\pgfpathmoveto{\pgfqpoint{4.369189in}{1.389617in}}%
\pgfpathlineto{\pgfqpoint{4.617810in}{1.389617in}}%
\pgfpathlineto{\pgfqpoint{4.617810in}{2.832214in}}%
\pgfpathlineto{\pgfqpoint{4.369189in}{2.832214in}}%
\pgfpathclose%
\pgfusepath{fill}%
\end{pgfscope}%
\begin{pgfscope}%
\pgfpathrectangle{\pgfqpoint{0.664741in}{1.389617in}}{\pgfqpoint{4.922705in}{1.759717in}}%
\pgfusepath{clip}%
\pgfsetbuttcap%
\pgfsetmiterjoin%
\definecolor{currentfill}{rgb}{0.501961,0.501961,0.501961}%
\pgfsetfillcolor{currentfill}%
\pgfsetfillopacity{0.200000}%
\pgfsetlinewidth{0.000000pt}%
\definecolor{currentstroke}{rgb}{0.000000,0.000000,0.000000}%
\pgfsetstrokecolor{currentstroke}%
\pgfsetstrokeopacity{0.200000}%
\pgfsetdash{}{0pt}%
\pgfpathmoveto{\pgfqpoint{4.617810in}{1.389617in}}%
\pgfpathlineto{\pgfqpoint{4.866431in}{1.389617in}}%
\pgfpathlineto{\pgfqpoint{4.866431in}{2.777309in}}%
\pgfpathlineto{\pgfqpoint{4.617810in}{2.777309in}}%
\pgfpathclose%
\pgfusepath{fill}%
\end{pgfscope}%
\begin{pgfscope}%
\pgfpathrectangle{\pgfqpoint{0.664741in}{1.389617in}}{\pgfqpoint{4.922705in}{1.759717in}}%
\pgfusepath{clip}%
\pgfsetbuttcap%
\pgfsetmiterjoin%
\definecolor{currentfill}{rgb}{0.501961,0.501961,0.501961}%
\pgfsetfillcolor{currentfill}%
\pgfsetfillopacity{0.200000}%
\pgfsetlinewidth{0.000000pt}%
\definecolor{currentstroke}{rgb}{0.000000,0.000000,0.000000}%
\pgfsetstrokecolor{currentstroke}%
\pgfsetstrokeopacity{0.200000}%
\pgfsetdash{}{0pt}%
\pgfpathmoveto{\pgfqpoint{4.866431in}{1.389617in}}%
\pgfpathlineto{\pgfqpoint{5.115051in}{1.389617in}}%
\pgfpathlineto{\pgfqpoint{5.115051in}{2.718603in}}%
\pgfpathlineto{\pgfqpoint{4.866431in}{2.718603in}}%
\pgfpathclose%
\pgfusepath{fill}%
\end{pgfscope}%
\begin{pgfscope}%
\pgfpathrectangle{\pgfqpoint{0.664741in}{1.389617in}}{\pgfqpoint{4.922705in}{1.759717in}}%
\pgfusepath{clip}%
\pgfsetbuttcap%
\pgfsetmiterjoin%
\definecolor{currentfill}{rgb}{0.501961,0.501961,0.501961}%
\pgfsetfillcolor{currentfill}%
\pgfsetfillopacity{0.200000}%
\pgfsetlinewidth{0.000000pt}%
\definecolor{currentstroke}{rgb}{0.000000,0.000000,0.000000}%
\pgfsetstrokecolor{currentstroke}%
\pgfsetstrokeopacity{0.200000}%
\pgfsetdash{}{0pt}%
\pgfpathmoveto{\pgfqpoint{5.115051in}{1.389617in}}%
\pgfpathlineto{\pgfqpoint{5.363672in}{1.389617in}}%
\pgfpathlineto{\pgfqpoint{5.363672in}{2.656762in}}%
\pgfpathlineto{\pgfqpoint{5.115051in}{2.656762in}}%
\pgfpathclose%
\pgfusepath{fill}%
\end{pgfscope}%
\begin{pgfscope}%
\pgfsetbuttcap%
\pgfsetroundjoin%
\definecolor{currentfill}{rgb}{0.000000,0.000000,0.000000}%
\pgfsetfillcolor{currentfill}%
\pgfsetlinewidth{0.803000pt}%
\definecolor{currentstroke}{rgb}{0.000000,0.000000,0.000000}%
\pgfsetstrokecolor{currentstroke}%
\pgfsetdash{}{0pt}%
\pgfsys@defobject{currentmarker}{\pgfqpoint{0.000000in}{0.000000in}}{\pgfqpoint{0.048611in}{0.000000in}}{%
\pgfpathmoveto{\pgfqpoint{0.000000in}{0.000000in}}%
\pgfpathlineto{\pgfqpoint{0.048611in}{0.000000in}}%
\pgfusepath{stroke,fill}%
}%
\begin{pgfscope}%
\pgfsys@transformshift{5.587445in}{1.389617in}%
\pgfsys@useobject{currentmarker}{}%
\end{pgfscope}%
\end{pgfscope}%
\begin{pgfscope}%
\definecolor{textcolor}{rgb}{0.000000,0.000000,0.000000}%
\pgfsetstrokecolor{textcolor}%
\pgfsetfillcolor{textcolor}%
\pgftext[x=5.684668in, y=1.350464in, left, base]{\color{textcolor}\rmfamily\fontsize{8.000000}{9.600000}\selectfont \(\displaystyle {10^{0}}\)}%
\end{pgfscope}%
\begin{pgfscope}%
\pgfsetbuttcap%
\pgfsetroundjoin%
\definecolor{currentfill}{rgb}{0.000000,0.000000,0.000000}%
\pgfsetfillcolor{currentfill}%
\pgfsetlinewidth{0.803000pt}%
\definecolor{currentstroke}{rgb}{0.000000,0.000000,0.000000}%
\pgfsetstrokecolor{currentstroke}%
\pgfsetdash{}{0pt}%
\pgfsys@defobject{currentmarker}{\pgfqpoint{0.000000in}{0.000000in}}{\pgfqpoint{0.048611in}{0.000000in}}{%
\pgfpathmoveto{\pgfqpoint{0.000000in}{0.000000in}}%
\pgfpathlineto{\pgfqpoint{0.048611in}{0.000000in}}%
\pgfusepath{stroke,fill}%
}%
\begin{pgfscope}%
\pgfsys@transformshift{5.587445in}{1.964099in}%
\pgfsys@useobject{currentmarker}{}%
\end{pgfscope}%
\end{pgfscope}%
\begin{pgfscope}%
\definecolor{textcolor}{rgb}{0.000000,0.000000,0.000000}%
\pgfsetstrokecolor{textcolor}%
\pgfsetfillcolor{textcolor}%
\pgftext[x=5.684668in, y=1.924946in, left, base]{\color{textcolor}\rmfamily\fontsize{8.000000}{9.600000}\selectfont \(\displaystyle {10^{2}}\)}%
\end{pgfscope}%
\begin{pgfscope}%
\pgfsetbuttcap%
\pgfsetroundjoin%
\definecolor{currentfill}{rgb}{0.000000,0.000000,0.000000}%
\pgfsetfillcolor{currentfill}%
\pgfsetlinewidth{0.803000pt}%
\definecolor{currentstroke}{rgb}{0.000000,0.000000,0.000000}%
\pgfsetstrokecolor{currentstroke}%
\pgfsetdash{}{0pt}%
\pgfsys@defobject{currentmarker}{\pgfqpoint{0.000000in}{0.000000in}}{\pgfqpoint{0.048611in}{0.000000in}}{%
\pgfpathmoveto{\pgfqpoint{0.000000in}{0.000000in}}%
\pgfpathlineto{\pgfqpoint{0.048611in}{0.000000in}}%
\pgfusepath{stroke,fill}%
}%
\begin{pgfscope}%
\pgfsys@transformshift{5.587445in}{2.538581in}%
\pgfsys@useobject{currentmarker}{}%
\end{pgfscope}%
\end{pgfscope}%
\begin{pgfscope}%
\definecolor{textcolor}{rgb}{0.000000,0.000000,0.000000}%
\pgfsetstrokecolor{textcolor}%
\pgfsetfillcolor{textcolor}%
\pgftext[x=5.684668in, y=2.499428in, left, base]{\color{textcolor}\rmfamily\fontsize{8.000000}{9.600000}\selectfont \(\displaystyle {10^{4}}\)}%
\end{pgfscope}%
\begin{pgfscope}%
\pgfsetbuttcap%
\pgfsetroundjoin%
\definecolor{currentfill}{rgb}{0.000000,0.000000,0.000000}%
\pgfsetfillcolor{currentfill}%
\pgfsetlinewidth{0.803000pt}%
\definecolor{currentstroke}{rgb}{0.000000,0.000000,0.000000}%
\pgfsetstrokecolor{currentstroke}%
\pgfsetdash{}{0pt}%
\pgfsys@defobject{currentmarker}{\pgfqpoint{0.000000in}{0.000000in}}{\pgfqpoint{0.048611in}{0.000000in}}{%
\pgfpathmoveto{\pgfqpoint{0.000000in}{0.000000in}}%
\pgfpathlineto{\pgfqpoint{0.048611in}{0.000000in}}%
\pgfusepath{stroke,fill}%
}%
\begin{pgfscope}%
\pgfsys@transformshift{5.587445in}{3.113063in}%
\pgfsys@useobject{currentmarker}{}%
\end{pgfscope}%
\end{pgfscope}%
\begin{pgfscope}%
\definecolor{textcolor}{rgb}{0.000000,0.000000,0.000000}%
\pgfsetstrokecolor{textcolor}%
\pgfsetfillcolor{textcolor}%
\pgftext[x=5.684668in, y=3.073910in, left, base]{\color{textcolor}\rmfamily\fontsize{8.000000}{9.600000}\selectfont \(\displaystyle {10^{6}}\)}%
\end{pgfscope}%
\begin{pgfscope}%
\definecolor{textcolor}{rgb}{0.000000,0.000000,0.000000}%
\pgfsetstrokecolor{textcolor}%
\pgfsetfillcolor{textcolor}%
\pgftext[x=5.916150in,y=2.269475in,,top,rotate=90.000000]{\color{textcolor}\rmfamily\fontsize{10.950000}{13.140000}\selectfont Events}%
\end{pgfscope}%
\begin{pgfscope}%
\pgfsetrectcap%
\pgfsetmiterjoin%
\pgfsetlinewidth{0.803000pt}%
\definecolor{currentstroke}{rgb}{0.000000,0.000000,0.000000}%
\pgfsetstrokecolor{currentstroke}%
\pgfsetdash{}{0pt}%
\pgfpathmoveto{\pgfqpoint{0.664741in}{1.389617in}}%
\pgfpathlineto{\pgfqpoint{0.664741in}{3.149333in}}%
\pgfusepath{stroke}%
\end{pgfscope}%
\begin{pgfscope}%
\pgfsetrectcap%
\pgfsetmiterjoin%
\pgfsetlinewidth{0.803000pt}%
\definecolor{currentstroke}{rgb}{0.000000,0.000000,0.000000}%
\pgfsetstrokecolor{currentstroke}%
\pgfsetdash{}{0pt}%
\pgfpathmoveto{\pgfqpoint{5.587445in}{1.389617in}}%
\pgfpathlineto{\pgfqpoint{5.587445in}{3.149333in}}%
\pgfusepath{stroke}%
\end{pgfscope}%
\begin{pgfscope}%
\pgfsetrectcap%
\pgfsetmiterjoin%
\pgfsetlinewidth{0.803000pt}%
\definecolor{currentstroke}{rgb}{0.000000,0.000000,0.000000}%
\pgfsetstrokecolor{currentstroke}%
\pgfsetdash{}{0pt}%
\pgfpathmoveto{\pgfqpoint{0.664741in}{1.389617in}}%
\pgfpathlineto{\pgfqpoint{5.587445in}{1.389617in}}%
\pgfusepath{stroke}%
\end{pgfscope}%
\begin{pgfscope}%
\pgfsetrectcap%
\pgfsetmiterjoin%
\pgfsetlinewidth{0.803000pt}%
\definecolor{currentstroke}{rgb}{0.000000,0.000000,0.000000}%
\pgfsetstrokecolor{currentstroke}%
\pgfsetdash{}{0pt}%
\pgfpathmoveto{\pgfqpoint{0.664741in}{3.149333in}}%
\pgfpathlineto{\pgfqpoint{5.587445in}{3.149333in}}%
\pgfusepath{stroke}%
\end{pgfscope}%
\end{pgfpicture}%
\makeatother%
\endgroup%

    \caption{The resolution performance of CubeFlow (red) and Retro Reco (black) for energy reconstruction.
    CubeFlow performs much better at low energies, while Retro Reco wins at higher energies.
    The amount of training events in each energy bin is superimposed (using a log scale) with grey bars.
    The lower plot shows the improvement of CubeFlow relative to Retro Reco, constrained between \SI{-100}{\percent} and \SI{100}{\percent}.}\label{fig:energy_comparison}
\end{figure}

\begin{figure}
    \centering
    %% Creator: Matplotlib, PGF backend
%%
%% To include the figure in your LaTeX document, write
%%   \input{<filename>.pgf}
%%
%% Make sure the required packages are loaded in your preamble
%%   \usepackage{pgf}
%%
%% and, on pdftex
%%   \usepackage[utf8]{inputenc}\DeclareUnicodeCharacter{2212}{-}
%%
%% or, on luatex and xetex
%%   \usepackage{unicode-math}
%%
%% Figures using additional raster images can only be included by \input if
%% they are in the same directory as the main LaTeX file. For loading figures
%% from other directories you can use the `import` package
%%   \usepackage{import}
%%
%% and then include the figures with
%%   \import{<path to file>}{<filename>.pgf}
%%
%% Matplotlib used the following preamble
%%   \usepackage{siunitx} \usepackage{amsmath} \usepackage{bm}
%%   \usepackage{fontspec}
%%
\begingroup%
\makeatletter%
\begin{pgfpicture}%
\pgfpathrectangle{\pgfpointorigin}{\pgfqpoint{6.201200in}{3.500000in}}%
\pgfusepath{use as bounding box, clip}%
\begin{pgfscope}%
\pgfsetbuttcap%
\pgfsetmiterjoin%
\definecolor{currentfill}{rgb}{1.000000,1.000000,1.000000}%
\pgfsetfillcolor{currentfill}%
\pgfsetlinewidth{0.000000pt}%
\definecolor{currentstroke}{rgb}{1.000000,1.000000,1.000000}%
\pgfsetstrokecolor{currentstroke}%
\pgfsetdash{}{0pt}%
\pgfpathmoveto{\pgfqpoint{0.000000in}{0.000000in}}%
\pgfpathlineto{\pgfqpoint{6.201200in}{0.000000in}}%
\pgfpathlineto{\pgfqpoint{6.201200in}{3.500000in}}%
\pgfpathlineto{\pgfqpoint{0.000000in}{3.500000in}}%
\pgfpathclose%
\pgfusepath{fill}%
\end{pgfscope}%
\begin{pgfscope}%
\pgfsetbuttcap%
\pgfsetmiterjoin%
\definecolor{currentfill}{rgb}{1.000000,1.000000,1.000000}%
\pgfsetfillcolor{currentfill}%
\pgfsetlinewidth{0.000000pt}%
\definecolor{currentstroke}{rgb}{0.000000,0.000000,0.000000}%
\pgfsetstrokecolor{currentstroke}%
\pgfsetstrokeopacity{0.000000}%
\pgfsetdash{}{0pt}%
\pgfpathmoveto{\pgfqpoint{0.588634in}{1.389617in}}%
\pgfpathlineto{\pgfqpoint{5.587445in}{1.389617in}}%
\pgfpathlineto{\pgfqpoint{5.587445in}{3.149333in}}%
\pgfpathlineto{\pgfqpoint{0.588634in}{3.149333in}}%
\pgfpathclose%
\pgfusepath{fill}%
\end{pgfscope}%
\begin{pgfscope}%
\pgfpathrectangle{\pgfqpoint{0.588634in}{1.389617in}}{\pgfqpoint{4.998811in}{1.759717in}}%
\pgfusepath{clip}%
\pgfsetbuttcap%
\pgfsetroundjoin%
\pgfsetlinewidth{0.501875pt}%
\definecolor{currentstroke}{rgb}{0.690196,0.690196,0.690196}%
\pgfsetstrokecolor{currentstroke}%
\pgfsetstrokeopacity{0.500000}%
\pgfsetdash{{0.500000pt}{0.825000pt}}{0.000000pt}%
\pgfpathmoveto{\pgfqpoint{0.815853in}{1.389617in}}%
\pgfpathlineto{\pgfqpoint{0.815853in}{3.149333in}}%
\pgfusepath{stroke}%
\end{pgfscope}%
\begin{pgfscope}%
\pgfsetbuttcap%
\pgfsetroundjoin%
\definecolor{currentfill}{rgb}{0.000000,0.000000,0.000000}%
\pgfsetfillcolor{currentfill}%
\pgfsetlinewidth{0.803000pt}%
\definecolor{currentstroke}{rgb}{0.000000,0.000000,0.000000}%
\pgfsetstrokecolor{currentstroke}%
\pgfsetdash{}{0pt}%
\pgfsys@defobject{currentmarker}{\pgfqpoint{0.000000in}{-0.048611in}}{\pgfqpoint{0.000000in}{0.000000in}}{%
\pgfpathmoveto{\pgfqpoint{0.000000in}{0.000000in}}%
\pgfpathlineto{\pgfqpoint{0.000000in}{-0.048611in}}%
\pgfusepath{stroke,fill}%
}%
\begin{pgfscope}%
\pgfsys@transformshift{0.815853in}{1.389617in}%
\pgfsys@useobject{currentmarker}{}%
\end{pgfscope}%
\end{pgfscope}%
\begin{pgfscope}%
\pgfpathrectangle{\pgfqpoint{0.588634in}{1.389617in}}{\pgfqpoint{4.998811in}{1.759717in}}%
\pgfusepath{clip}%
\pgfsetbuttcap%
\pgfsetroundjoin%
\pgfsetlinewidth{0.501875pt}%
\definecolor{currentstroke}{rgb}{0.690196,0.690196,0.690196}%
\pgfsetstrokecolor{currentstroke}%
\pgfsetstrokeopacity{0.500000}%
\pgfsetdash{{0.500000pt}{0.825000pt}}{0.000000pt}%
\pgfpathmoveto{\pgfqpoint{1.573249in}{1.389617in}}%
\pgfpathlineto{\pgfqpoint{1.573249in}{3.149333in}}%
\pgfusepath{stroke}%
\end{pgfscope}%
\begin{pgfscope}%
\pgfsetbuttcap%
\pgfsetroundjoin%
\definecolor{currentfill}{rgb}{0.000000,0.000000,0.000000}%
\pgfsetfillcolor{currentfill}%
\pgfsetlinewidth{0.803000pt}%
\definecolor{currentstroke}{rgb}{0.000000,0.000000,0.000000}%
\pgfsetstrokecolor{currentstroke}%
\pgfsetdash{}{0pt}%
\pgfsys@defobject{currentmarker}{\pgfqpoint{0.000000in}{-0.048611in}}{\pgfqpoint{0.000000in}{0.000000in}}{%
\pgfpathmoveto{\pgfqpoint{0.000000in}{0.000000in}}%
\pgfpathlineto{\pgfqpoint{0.000000in}{-0.048611in}}%
\pgfusepath{stroke,fill}%
}%
\begin{pgfscope}%
\pgfsys@transformshift{1.573249in}{1.389617in}%
\pgfsys@useobject{currentmarker}{}%
\end{pgfscope}%
\end{pgfscope}%
\begin{pgfscope}%
\pgfpathrectangle{\pgfqpoint{0.588634in}{1.389617in}}{\pgfqpoint{4.998811in}{1.759717in}}%
\pgfusepath{clip}%
\pgfsetbuttcap%
\pgfsetroundjoin%
\pgfsetlinewidth{0.501875pt}%
\definecolor{currentstroke}{rgb}{0.690196,0.690196,0.690196}%
\pgfsetstrokecolor{currentstroke}%
\pgfsetstrokeopacity{0.500000}%
\pgfsetdash{{0.500000pt}{0.825000pt}}{0.000000pt}%
\pgfpathmoveto{\pgfqpoint{2.330644in}{1.389617in}}%
\pgfpathlineto{\pgfqpoint{2.330644in}{3.149333in}}%
\pgfusepath{stroke}%
\end{pgfscope}%
\begin{pgfscope}%
\pgfsetbuttcap%
\pgfsetroundjoin%
\definecolor{currentfill}{rgb}{0.000000,0.000000,0.000000}%
\pgfsetfillcolor{currentfill}%
\pgfsetlinewidth{0.803000pt}%
\definecolor{currentstroke}{rgb}{0.000000,0.000000,0.000000}%
\pgfsetstrokecolor{currentstroke}%
\pgfsetdash{}{0pt}%
\pgfsys@defobject{currentmarker}{\pgfqpoint{0.000000in}{-0.048611in}}{\pgfqpoint{0.000000in}{0.000000in}}{%
\pgfpathmoveto{\pgfqpoint{0.000000in}{0.000000in}}%
\pgfpathlineto{\pgfqpoint{0.000000in}{-0.048611in}}%
\pgfusepath{stroke,fill}%
}%
\begin{pgfscope}%
\pgfsys@transformshift{2.330644in}{1.389617in}%
\pgfsys@useobject{currentmarker}{}%
\end{pgfscope}%
\end{pgfscope}%
\begin{pgfscope}%
\pgfpathrectangle{\pgfqpoint{0.588634in}{1.389617in}}{\pgfqpoint{4.998811in}{1.759717in}}%
\pgfusepath{clip}%
\pgfsetbuttcap%
\pgfsetroundjoin%
\pgfsetlinewidth{0.501875pt}%
\definecolor{currentstroke}{rgb}{0.690196,0.690196,0.690196}%
\pgfsetstrokecolor{currentstroke}%
\pgfsetstrokeopacity{0.500000}%
\pgfsetdash{{0.500000pt}{0.825000pt}}{0.000000pt}%
\pgfpathmoveto{\pgfqpoint{3.088040in}{1.389617in}}%
\pgfpathlineto{\pgfqpoint{3.088040in}{3.149333in}}%
\pgfusepath{stroke}%
\end{pgfscope}%
\begin{pgfscope}%
\pgfsetbuttcap%
\pgfsetroundjoin%
\definecolor{currentfill}{rgb}{0.000000,0.000000,0.000000}%
\pgfsetfillcolor{currentfill}%
\pgfsetlinewidth{0.803000pt}%
\definecolor{currentstroke}{rgb}{0.000000,0.000000,0.000000}%
\pgfsetstrokecolor{currentstroke}%
\pgfsetdash{}{0pt}%
\pgfsys@defobject{currentmarker}{\pgfqpoint{0.000000in}{-0.048611in}}{\pgfqpoint{0.000000in}{0.000000in}}{%
\pgfpathmoveto{\pgfqpoint{0.000000in}{0.000000in}}%
\pgfpathlineto{\pgfqpoint{0.000000in}{-0.048611in}}%
\pgfusepath{stroke,fill}%
}%
\begin{pgfscope}%
\pgfsys@transformshift{3.088040in}{1.389617in}%
\pgfsys@useobject{currentmarker}{}%
\end{pgfscope}%
\end{pgfscope}%
\begin{pgfscope}%
\pgfpathrectangle{\pgfqpoint{0.588634in}{1.389617in}}{\pgfqpoint{4.998811in}{1.759717in}}%
\pgfusepath{clip}%
\pgfsetbuttcap%
\pgfsetroundjoin%
\pgfsetlinewidth{0.501875pt}%
\definecolor{currentstroke}{rgb}{0.690196,0.690196,0.690196}%
\pgfsetstrokecolor{currentstroke}%
\pgfsetstrokeopacity{0.500000}%
\pgfsetdash{{0.500000pt}{0.825000pt}}{0.000000pt}%
\pgfpathmoveto{\pgfqpoint{3.845435in}{1.389617in}}%
\pgfpathlineto{\pgfqpoint{3.845435in}{3.149333in}}%
\pgfusepath{stroke}%
\end{pgfscope}%
\begin{pgfscope}%
\pgfsetbuttcap%
\pgfsetroundjoin%
\definecolor{currentfill}{rgb}{0.000000,0.000000,0.000000}%
\pgfsetfillcolor{currentfill}%
\pgfsetlinewidth{0.803000pt}%
\definecolor{currentstroke}{rgb}{0.000000,0.000000,0.000000}%
\pgfsetstrokecolor{currentstroke}%
\pgfsetdash{}{0pt}%
\pgfsys@defobject{currentmarker}{\pgfqpoint{0.000000in}{-0.048611in}}{\pgfqpoint{0.000000in}{0.000000in}}{%
\pgfpathmoveto{\pgfqpoint{0.000000in}{0.000000in}}%
\pgfpathlineto{\pgfqpoint{0.000000in}{-0.048611in}}%
\pgfusepath{stroke,fill}%
}%
\begin{pgfscope}%
\pgfsys@transformshift{3.845435in}{1.389617in}%
\pgfsys@useobject{currentmarker}{}%
\end{pgfscope}%
\end{pgfscope}%
\begin{pgfscope}%
\pgfpathrectangle{\pgfqpoint{0.588634in}{1.389617in}}{\pgfqpoint{4.998811in}{1.759717in}}%
\pgfusepath{clip}%
\pgfsetbuttcap%
\pgfsetroundjoin%
\pgfsetlinewidth{0.501875pt}%
\definecolor{currentstroke}{rgb}{0.690196,0.690196,0.690196}%
\pgfsetstrokecolor{currentstroke}%
\pgfsetstrokeopacity{0.500000}%
\pgfsetdash{{0.500000pt}{0.825000pt}}{0.000000pt}%
\pgfpathmoveto{\pgfqpoint{4.602831in}{1.389617in}}%
\pgfpathlineto{\pgfqpoint{4.602831in}{3.149333in}}%
\pgfusepath{stroke}%
\end{pgfscope}%
\begin{pgfscope}%
\pgfsetbuttcap%
\pgfsetroundjoin%
\definecolor{currentfill}{rgb}{0.000000,0.000000,0.000000}%
\pgfsetfillcolor{currentfill}%
\pgfsetlinewidth{0.803000pt}%
\definecolor{currentstroke}{rgb}{0.000000,0.000000,0.000000}%
\pgfsetstrokecolor{currentstroke}%
\pgfsetdash{}{0pt}%
\pgfsys@defobject{currentmarker}{\pgfqpoint{0.000000in}{-0.048611in}}{\pgfqpoint{0.000000in}{0.000000in}}{%
\pgfpathmoveto{\pgfqpoint{0.000000in}{0.000000in}}%
\pgfpathlineto{\pgfqpoint{0.000000in}{-0.048611in}}%
\pgfusepath{stroke,fill}%
}%
\begin{pgfscope}%
\pgfsys@transformshift{4.602831in}{1.389617in}%
\pgfsys@useobject{currentmarker}{}%
\end{pgfscope}%
\end{pgfscope}%
\begin{pgfscope}%
\pgfpathrectangle{\pgfqpoint{0.588634in}{1.389617in}}{\pgfqpoint{4.998811in}{1.759717in}}%
\pgfusepath{clip}%
\pgfsetbuttcap%
\pgfsetroundjoin%
\pgfsetlinewidth{0.501875pt}%
\definecolor{currentstroke}{rgb}{0.690196,0.690196,0.690196}%
\pgfsetstrokecolor{currentstroke}%
\pgfsetstrokeopacity{0.500000}%
\pgfsetdash{{0.500000pt}{0.825000pt}}{0.000000pt}%
\pgfpathmoveto{\pgfqpoint{5.360227in}{1.389617in}}%
\pgfpathlineto{\pgfqpoint{5.360227in}{3.149333in}}%
\pgfusepath{stroke}%
\end{pgfscope}%
\begin{pgfscope}%
\pgfsetbuttcap%
\pgfsetroundjoin%
\definecolor{currentfill}{rgb}{0.000000,0.000000,0.000000}%
\pgfsetfillcolor{currentfill}%
\pgfsetlinewidth{0.803000pt}%
\definecolor{currentstroke}{rgb}{0.000000,0.000000,0.000000}%
\pgfsetstrokecolor{currentstroke}%
\pgfsetdash{}{0pt}%
\pgfsys@defobject{currentmarker}{\pgfqpoint{0.000000in}{-0.048611in}}{\pgfqpoint{0.000000in}{0.000000in}}{%
\pgfpathmoveto{\pgfqpoint{0.000000in}{0.000000in}}%
\pgfpathlineto{\pgfqpoint{0.000000in}{-0.048611in}}%
\pgfusepath{stroke,fill}%
}%
\begin{pgfscope}%
\pgfsys@transformshift{5.360227in}{1.389617in}%
\pgfsys@useobject{currentmarker}{}%
\end{pgfscope}%
\end{pgfscope}%
\begin{pgfscope}%
\pgfpathrectangle{\pgfqpoint{0.588634in}{1.389617in}}{\pgfqpoint{4.998811in}{1.759717in}}%
\pgfusepath{clip}%
\pgfsetbuttcap%
\pgfsetroundjoin%
\pgfsetlinewidth{0.501875pt}%
\definecolor{currentstroke}{rgb}{0.690196,0.690196,0.690196}%
\pgfsetstrokecolor{currentstroke}%
\pgfsetstrokeopacity{0.500000}%
\pgfsetdash{{0.500000pt}{0.825000pt}}{0.000000pt}%
\pgfpathmoveto{\pgfqpoint{0.588634in}{1.583808in}}%
\pgfpathlineto{\pgfqpoint{5.587445in}{1.583808in}}%
\pgfusepath{stroke}%
\end{pgfscope}%
\begin{pgfscope}%
\pgfsetbuttcap%
\pgfsetroundjoin%
\definecolor{currentfill}{rgb}{0.000000,0.000000,0.000000}%
\pgfsetfillcolor{currentfill}%
\pgfsetlinewidth{0.803000pt}%
\definecolor{currentstroke}{rgb}{0.000000,0.000000,0.000000}%
\pgfsetstrokecolor{currentstroke}%
\pgfsetdash{}{0pt}%
\pgfsys@defobject{currentmarker}{\pgfqpoint{-0.048611in}{0.000000in}}{\pgfqpoint{-0.000000in}{0.000000in}}{%
\pgfpathmoveto{\pgfqpoint{-0.000000in}{0.000000in}}%
\pgfpathlineto{\pgfqpoint{-0.048611in}{0.000000in}}%
\pgfusepath{stroke,fill}%
}%
\begin{pgfscope}%
\pgfsys@transformshift{0.588634in}{1.583808in}%
\pgfsys@useobject{currentmarker}{}%
\end{pgfscope}%
\end{pgfscope}%
\begin{pgfscope}%
\definecolor{textcolor}{rgb}{0.000000,0.000000,0.000000}%
\pgfsetstrokecolor{textcolor}%
\pgfsetfillcolor{textcolor}%
\pgftext[x=0.432383in, y=1.545252in, left, base]{\color{textcolor}\rmfamily\fontsize{8.000000}{9.600000}\selectfont \(\displaystyle {5}\)}%
\end{pgfscope}%
\begin{pgfscope}%
\pgfpathrectangle{\pgfqpoint{0.588634in}{1.389617in}}{\pgfqpoint{4.998811in}{1.759717in}}%
\pgfusepath{clip}%
\pgfsetbuttcap%
\pgfsetroundjoin%
\pgfsetlinewidth{0.501875pt}%
\definecolor{currentstroke}{rgb}{0.690196,0.690196,0.690196}%
\pgfsetstrokecolor{currentstroke}%
\pgfsetstrokeopacity{0.500000}%
\pgfsetdash{{0.500000pt}{0.825000pt}}{0.000000pt}%
\pgfpathmoveto{\pgfqpoint{0.588634in}{1.876782in}}%
\pgfpathlineto{\pgfqpoint{5.587445in}{1.876782in}}%
\pgfusepath{stroke}%
\end{pgfscope}%
\begin{pgfscope}%
\pgfsetbuttcap%
\pgfsetroundjoin%
\definecolor{currentfill}{rgb}{0.000000,0.000000,0.000000}%
\pgfsetfillcolor{currentfill}%
\pgfsetlinewidth{0.803000pt}%
\definecolor{currentstroke}{rgb}{0.000000,0.000000,0.000000}%
\pgfsetstrokecolor{currentstroke}%
\pgfsetdash{}{0pt}%
\pgfsys@defobject{currentmarker}{\pgfqpoint{-0.048611in}{0.000000in}}{\pgfqpoint{-0.000000in}{0.000000in}}{%
\pgfpathmoveto{\pgfqpoint{-0.000000in}{0.000000in}}%
\pgfpathlineto{\pgfqpoint{-0.048611in}{0.000000in}}%
\pgfusepath{stroke,fill}%
}%
\begin{pgfscope}%
\pgfsys@transformshift{0.588634in}{1.876782in}%
\pgfsys@useobject{currentmarker}{}%
\end{pgfscope}%
\end{pgfscope}%
\begin{pgfscope}%
\definecolor{textcolor}{rgb}{0.000000,0.000000,0.000000}%
\pgfsetstrokecolor{textcolor}%
\pgfsetfillcolor{textcolor}%
\pgftext[x=0.373355in, y=1.838227in, left, base]{\color{textcolor}\rmfamily\fontsize{8.000000}{9.600000}\selectfont \(\displaystyle {10}\)}%
\end{pgfscope}%
\begin{pgfscope}%
\pgfpathrectangle{\pgfqpoint{0.588634in}{1.389617in}}{\pgfqpoint{4.998811in}{1.759717in}}%
\pgfusepath{clip}%
\pgfsetbuttcap%
\pgfsetroundjoin%
\pgfsetlinewidth{0.501875pt}%
\definecolor{currentstroke}{rgb}{0.690196,0.690196,0.690196}%
\pgfsetstrokecolor{currentstroke}%
\pgfsetstrokeopacity{0.500000}%
\pgfsetdash{{0.500000pt}{0.825000pt}}{0.000000pt}%
\pgfpathmoveto{\pgfqpoint{0.588634in}{2.169757in}}%
\pgfpathlineto{\pgfqpoint{5.587445in}{2.169757in}}%
\pgfusepath{stroke}%
\end{pgfscope}%
\begin{pgfscope}%
\pgfsetbuttcap%
\pgfsetroundjoin%
\definecolor{currentfill}{rgb}{0.000000,0.000000,0.000000}%
\pgfsetfillcolor{currentfill}%
\pgfsetlinewidth{0.803000pt}%
\definecolor{currentstroke}{rgb}{0.000000,0.000000,0.000000}%
\pgfsetstrokecolor{currentstroke}%
\pgfsetdash{}{0pt}%
\pgfsys@defobject{currentmarker}{\pgfqpoint{-0.048611in}{0.000000in}}{\pgfqpoint{-0.000000in}{0.000000in}}{%
\pgfpathmoveto{\pgfqpoint{-0.000000in}{0.000000in}}%
\pgfpathlineto{\pgfqpoint{-0.048611in}{0.000000in}}%
\pgfusepath{stroke,fill}%
}%
\begin{pgfscope}%
\pgfsys@transformshift{0.588634in}{2.169757in}%
\pgfsys@useobject{currentmarker}{}%
\end{pgfscope}%
\end{pgfscope}%
\begin{pgfscope}%
\definecolor{textcolor}{rgb}{0.000000,0.000000,0.000000}%
\pgfsetstrokecolor{textcolor}%
\pgfsetfillcolor{textcolor}%
\pgftext[x=0.373355in, y=2.131201in, left, base]{\color{textcolor}\rmfamily\fontsize{8.000000}{9.600000}\selectfont \(\displaystyle {15}\)}%
\end{pgfscope}%
\begin{pgfscope}%
\pgfpathrectangle{\pgfqpoint{0.588634in}{1.389617in}}{\pgfqpoint{4.998811in}{1.759717in}}%
\pgfusepath{clip}%
\pgfsetbuttcap%
\pgfsetroundjoin%
\pgfsetlinewidth{0.501875pt}%
\definecolor{currentstroke}{rgb}{0.690196,0.690196,0.690196}%
\pgfsetstrokecolor{currentstroke}%
\pgfsetstrokeopacity{0.500000}%
\pgfsetdash{{0.500000pt}{0.825000pt}}{0.000000pt}%
\pgfpathmoveto{\pgfqpoint{0.588634in}{2.462731in}}%
\pgfpathlineto{\pgfqpoint{5.587445in}{2.462731in}}%
\pgfusepath{stroke}%
\end{pgfscope}%
\begin{pgfscope}%
\pgfsetbuttcap%
\pgfsetroundjoin%
\definecolor{currentfill}{rgb}{0.000000,0.000000,0.000000}%
\pgfsetfillcolor{currentfill}%
\pgfsetlinewidth{0.803000pt}%
\definecolor{currentstroke}{rgb}{0.000000,0.000000,0.000000}%
\pgfsetstrokecolor{currentstroke}%
\pgfsetdash{}{0pt}%
\pgfsys@defobject{currentmarker}{\pgfqpoint{-0.048611in}{0.000000in}}{\pgfqpoint{-0.000000in}{0.000000in}}{%
\pgfpathmoveto{\pgfqpoint{-0.000000in}{0.000000in}}%
\pgfpathlineto{\pgfqpoint{-0.048611in}{0.000000in}}%
\pgfusepath{stroke,fill}%
}%
\begin{pgfscope}%
\pgfsys@transformshift{0.588634in}{2.462731in}%
\pgfsys@useobject{currentmarker}{}%
\end{pgfscope}%
\end{pgfscope}%
\begin{pgfscope}%
\definecolor{textcolor}{rgb}{0.000000,0.000000,0.000000}%
\pgfsetstrokecolor{textcolor}%
\pgfsetfillcolor{textcolor}%
\pgftext[x=0.373355in, y=2.424175in, left, base]{\color{textcolor}\rmfamily\fontsize{8.000000}{9.600000}\selectfont \(\displaystyle {20}\)}%
\end{pgfscope}%
\begin{pgfscope}%
\pgfpathrectangle{\pgfqpoint{0.588634in}{1.389617in}}{\pgfqpoint{4.998811in}{1.759717in}}%
\pgfusepath{clip}%
\pgfsetbuttcap%
\pgfsetroundjoin%
\pgfsetlinewidth{0.501875pt}%
\definecolor{currentstroke}{rgb}{0.690196,0.690196,0.690196}%
\pgfsetstrokecolor{currentstroke}%
\pgfsetstrokeopacity{0.500000}%
\pgfsetdash{{0.500000pt}{0.825000pt}}{0.000000pt}%
\pgfpathmoveto{\pgfqpoint{0.588634in}{2.755705in}}%
\pgfpathlineto{\pgfqpoint{5.587445in}{2.755705in}}%
\pgfusepath{stroke}%
\end{pgfscope}%
\begin{pgfscope}%
\pgfsetbuttcap%
\pgfsetroundjoin%
\definecolor{currentfill}{rgb}{0.000000,0.000000,0.000000}%
\pgfsetfillcolor{currentfill}%
\pgfsetlinewidth{0.803000pt}%
\definecolor{currentstroke}{rgb}{0.000000,0.000000,0.000000}%
\pgfsetstrokecolor{currentstroke}%
\pgfsetdash{}{0pt}%
\pgfsys@defobject{currentmarker}{\pgfqpoint{-0.048611in}{0.000000in}}{\pgfqpoint{-0.000000in}{0.000000in}}{%
\pgfpathmoveto{\pgfqpoint{-0.000000in}{0.000000in}}%
\pgfpathlineto{\pgfqpoint{-0.048611in}{0.000000in}}%
\pgfusepath{stroke,fill}%
}%
\begin{pgfscope}%
\pgfsys@transformshift{0.588634in}{2.755705in}%
\pgfsys@useobject{currentmarker}{}%
\end{pgfscope}%
\end{pgfscope}%
\begin{pgfscope}%
\definecolor{textcolor}{rgb}{0.000000,0.000000,0.000000}%
\pgfsetstrokecolor{textcolor}%
\pgfsetfillcolor{textcolor}%
\pgftext[x=0.373355in, y=2.717150in, left, base]{\color{textcolor}\rmfamily\fontsize{8.000000}{9.600000}\selectfont \(\displaystyle {25}\)}%
\end{pgfscope}%
\begin{pgfscope}%
\pgfpathrectangle{\pgfqpoint{0.588634in}{1.389617in}}{\pgfqpoint{4.998811in}{1.759717in}}%
\pgfusepath{clip}%
\pgfsetbuttcap%
\pgfsetroundjoin%
\pgfsetlinewidth{0.501875pt}%
\definecolor{currentstroke}{rgb}{0.690196,0.690196,0.690196}%
\pgfsetstrokecolor{currentstroke}%
\pgfsetstrokeopacity{0.500000}%
\pgfsetdash{{0.500000pt}{0.825000pt}}{0.000000pt}%
\pgfpathmoveto{\pgfqpoint{0.588634in}{3.048679in}}%
\pgfpathlineto{\pgfqpoint{5.587445in}{3.048679in}}%
\pgfusepath{stroke}%
\end{pgfscope}%
\begin{pgfscope}%
\pgfsetbuttcap%
\pgfsetroundjoin%
\definecolor{currentfill}{rgb}{0.000000,0.000000,0.000000}%
\pgfsetfillcolor{currentfill}%
\pgfsetlinewidth{0.803000pt}%
\definecolor{currentstroke}{rgb}{0.000000,0.000000,0.000000}%
\pgfsetstrokecolor{currentstroke}%
\pgfsetdash{}{0pt}%
\pgfsys@defobject{currentmarker}{\pgfqpoint{-0.048611in}{0.000000in}}{\pgfqpoint{-0.000000in}{0.000000in}}{%
\pgfpathmoveto{\pgfqpoint{-0.000000in}{0.000000in}}%
\pgfpathlineto{\pgfqpoint{-0.048611in}{0.000000in}}%
\pgfusepath{stroke,fill}%
}%
\begin{pgfscope}%
\pgfsys@transformshift{0.588634in}{3.048679in}%
\pgfsys@useobject{currentmarker}{}%
\end{pgfscope}%
\end{pgfscope}%
\begin{pgfscope}%
\definecolor{textcolor}{rgb}{0.000000,0.000000,0.000000}%
\pgfsetstrokecolor{textcolor}%
\pgfsetfillcolor{textcolor}%
\pgftext[x=0.373355in, y=3.010124in, left, base]{\color{textcolor}\rmfamily\fontsize{8.000000}{9.600000}\selectfont \(\displaystyle {30}\)}%
\end{pgfscope}%
\begin{pgfscope}%
\definecolor{textcolor}{rgb}{0.000000,0.000000,0.000000}%
\pgfsetstrokecolor{textcolor}%
\pgfsetfillcolor{textcolor}%
\pgftext[x=0.317799in,y=2.269475in,,bottom,rotate=90.000000]{\color{textcolor}\rmfamily\fontsize{10.950000}{13.140000}\selectfont IQR / 1.349 \(\displaystyle \left[ \textup{deg} \right]\)}%
\end{pgfscope}%
\begin{pgfscope}%
\pgfpathrectangle{\pgfqpoint{0.588634in}{1.389617in}}{\pgfqpoint{4.998811in}{1.759717in}}%
\pgfusepath{clip}%
\pgfsetbuttcap%
\pgfsetroundjoin%
\pgfsetlinewidth{1.505625pt}%
\definecolor{currentstroke}{rgb}{0.313725,0.317647,0.309804}%
\pgfsetstrokecolor{currentstroke}%
\pgfsetstrokeopacity{0.900000}%
\pgfsetdash{}{0pt}%
\pgfpathmoveto{\pgfqpoint{0.815853in}{3.034810in}}%
\pgfpathlineto{\pgfqpoint{1.068318in}{3.034810in}}%
\pgfusepath{stroke}%
\end{pgfscope}%
\begin{pgfscope}%
\pgfpathrectangle{\pgfqpoint{0.588634in}{1.389617in}}{\pgfqpoint{4.998811in}{1.759717in}}%
\pgfusepath{clip}%
\pgfsetbuttcap%
\pgfsetroundjoin%
\pgfsetlinewidth{1.505625pt}%
\definecolor{currentstroke}{rgb}{0.313725,0.317647,0.309804}%
\pgfsetstrokecolor{currentstroke}%
\pgfsetstrokeopacity{0.900000}%
\pgfsetdash{}{0pt}%
\pgfpathmoveto{\pgfqpoint{1.068318in}{2.946714in}}%
\pgfpathlineto{\pgfqpoint{1.320783in}{2.946714in}}%
\pgfusepath{stroke}%
\end{pgfscope}%
\begin{pgfscope}%
\pgfpathrectangle{\pgfqpoint{0.588634in}{1.389617in}}{\pgfqpoint{4.998811in}{1.759717in}}%
\pgfusepath{clip}%
\pgfsetbuttcap%
\pgfsetroundjoin%
\pgfsetlinewidth{1.505625pt}%
\definecolor{currentstroke}{rgb}{0.313725,0.317647,0.309804}%
\pgfsetstrokecolor{currentstroke}%
\pgfsetstrokeopacity{0.900000}%
\pgfsetdash{}{0pt}%
\pgfpathmoveto{\pgfqpoint{1.320783in}{2.885802in}}%
\pgfpathlineto{\pgfqpoint{1.573249in}{2.885802in}}%
\pgfusepath{stroke}%
\end{pgfscope}%
\begin{pgfscope}%
\pgfpathrectangle{\pgfqpoint{0.588634in}{1.389617in}}{\pgfqpoint{4.998811in}{1.759717in}}%
\pgfusepath{clip}%
\pgfsetbuttcap%
\pgfsetroundjoin%
\pgfsetlinewidth{1.505625pt}%
\definecolor{currentstroke}{rgb}{0.313725,0.317647,0.309804}%
\pgfsetstrokecolor{currentstroke}%
\pgfsetstrokeopacity{0.900000}%
\pgfsetdash{}{0pt}%
\pgfpathmoveto{\pgfqpoint{1.573249in}{2.850759in}}%
\pgfpathlineto{\pgfqpoint{1.825714in}{2.850759in}}%
\pgfusepath{stroke}%
\end{pgfscope}%
\begin{pgfscope}%
\pgfpathrectangle{\pgfqpoint{0.588634in}{1.389617in}}{\pgfqpoint{4.998811in}{1.759717in}}%
\pgfusepath{clip}%
\pgfsetbuttcap%
\pgfsetroundjoin%
\pgfsetlinewidth{1.505625pt}%
\definecolor{currentstroke}{rgb}{0.313725,0.317647,0.309804}%
\pgfsetstrokecolor{currentstroke}%
\pgfsetstrokeopacity{0.900000}%
\pgfsetdash{}{0pt}%
\pgfpathmoveto{\pgfqpoint{1.825714in}{2.826646in}}%
\pgfpathlineto{\pgfqpoint{2.078179in}{2.826646in}}%
\pgfusepath{stroke}%
\end{pgfscope}%
\begin{pgfscope}%
\pgfpathrectangle{\pgfqpoint{0.588634in}{1.389617in}}{\pgfqpoint{4.998811in}{1.759717in}}%
\pgfusepath{clip}%
\pgfsetbuttcap%
\pgfsetroundjoin%
\pgfsetlinewidth{1.505625pt}%
\definecolor{currentstroke}{rgb}{0.313725,0.317647,0.309804}%
\pgfsetstrokecolor{currentstroke}%
\pgfsetstrokeopacity{0.900000}%
\pgfsetdash{}{0pt}%
\pgfpathmoveto{\pgfqpoint{2.078179in}{2.791834in}}%
\pgfpathlineto{\pgfqpoint{2.330644in}{2.791834in}}%
\pgfusepath{stroke}%
\end{pgfscope}%
\begin{pgfscope}%
\pgfpathrectangle{\pgfqpoint{0.588634in}{1.389617in}}{\pgfqpoint{4.998811in}{1.759717in}}%
\pgfusepath{clip}%
\pgfsetbuttcap%
\pgfsetroundjoin%
\pgfsetlinewidth{1.505625pt}%
\definecolor{currentstroke}{rgb}{0.313725,0.317647,0.309804}%
\pgfsetstrokecolor{currentstroke}%
\pgfsetstrokeopacity{0.900000}%
\pgfsetdash{}{0pt}%
\pgfpathmoveto{\pgfqpoint{2.330644in}{2.708700in}}%
\pgfpathlineto{\pgfqpoint{2.583109in}{2.708700in}}%
\pgfusepath{stroke}%
\end{pgfscope}%
\begin{pgfscope}%
\pgfpathrectangle{\pgfqpoint{0.588634in}{1.389617in}}{\pgfqpoint{4.998811in}{1.759717in}}%
\pgfusepath{clip}%
\pgfsetbuttcap%
\pgfsetroundjoin%
\pgfsetlinewidth{1.505625pt}%
\definecolor{currentstroke}{rgb}{0.313725,0.317647,0.309804}%
\pgfsetstrokecolor{currentstroke}%
\pgfsetstrokeopacity{0.900000}%
\pgfsetdash{}{0pt}%
\pgfpathmoveto{\pgfqpoint{2.583109in}{2.569857in}}%
\pgfpathlineto{\pgfqpoint{2.835575in}{2.569857in}}%
\pgfusepath{stroke}%
\end{pgfscope}%
\begin{pgfscope}%
\pgfpathrectangle{\pgfqpoint{0.588634in}{1.389617in}}{\pgfqpoint{4.998811in}{1.759717in}}%
\pgfusepath{clip}%
\pgfsetbuttcap%
\pgfsetroundjoin%
\pgfsetlinewidth{1.505625pt}%
\definecolor{currentstroke}{rgb}{0.313725,0.317647,0.309804}%
\pgfsetstrokecolor{currentstroke}%
\pgfsetstrokeopacity{0.900000}%
\pgfsetdash{}{0pt}%
\pgfpathmoveto{\pgfqpoint{2.835575in}{2.385982in}}%
\pgfpathlineto{\pgfqpoint{3.088040in}{2.385982in}}%
\pgfusepath{stroke}%
\end{pgfscope}%
\begin{pgfscope}%
\pgfpathrectangle{\pgfqpoint{0.588634in}{1.389617in}}{\pgfqpoint{4.998811in}{1.759717in}}%
\pgfusepath{clip}%
\pgfsetbuttcap%
\pgfsetroundjoin%
\pgfsetlinewidth{1.505625pt}%
\definecolor{currentstroke}{rgb}{0.313725,0.317647,0.309804}%
\pgfsetstrokecolor{currentstroke}%
\pgfsetstrokeopacity{0.900000}%
\pgfsetdash{}{0pt}%
\pgfpathmoveto{\pgfqpoint{3.088040in}{2.194775in}}%
\pgfpathlineto{\pgfqpoint{3.340505in}{2.194775in}}%
\pgfusepath{stroke}%
\end{pgfscope}%
\begin{pgfscope}%
\pgfpathrectangle{\pgfqpoint{0.588634in}{1.389617in}}{\pgfqpoint{4.998811in}{1.759717in}}%
\pgfusepath{clip}%
\pgfsetbuttcap%
\pgfsetroundjoin%
\pgfsetlinewidth{1.505625pt}%
\definecolor{currentstroke}{rgb}{0.313725,0.317647,0.309804}%
\pgfsetstrokecolor{currentstroke}%
\pgfsetstrokeopacity{0.900000}%
\pgfsetdash{}{0pt}%
\pgfpathmoveto{\pgfqpoint{3.340505in}{2.014706in}}%
\pgfpathlineto{\pgfqpoint{3.592970in}{2.014706in}}%
\pgfusepath{stroke}%
\end{pgfscope}%
\begin{pgfscope}%
\pgfpathrectangle{\pgfqpoint{0.588634in}{1.389617in}}{\pgfqpoint{4.998811in}{1.759717in}}%
\pgfusepath{clip}%
\pgfsetbuttcap%
\pgfsetroundjoin%
\pgfsetlinewidth{1.505625pt}%
\definecolor{currentstroke}{rgb}{0.313725,0.317647,0.309804}%
\pgfsetstrokecolor{currentstroke}%
\pgfsetstrokeopacity{0.900000}%
\pgfsetdash{}{0pt}%
\pgfpathmoveto{\pgfqpoint{3.592970in}{1.857604in}}%
\pgfpathlineto{\pgfqpoint{3.845435in}{1.857604in}}%
\pgfusepath{stroke}%
\end{pgfscope}%
\begin{pgfscope}%
\pgfpathrectangle{\pgfqpoint{0.588634in}{1.389617in}}{\pgfqpoint{4.998811in}{1.759717in}}%
\pgfusepath{clip}%
\pgfsetbuttcap%
\pgfsetroundjoin%
\pgfsetlinewidth{1.505625pt}%
\definecolor{currentstroke}{rgb}{0.313725,0.317647,0.309804}%
\pgfsetstrokecolor{currentstroke}%
\pgfsetstrokeopacity{0.900000}%
\pgfsetdash{}{0pt}%
\pgfpathmoveto{\pgfqpoint{3.845435in}{1.718820in}}%
\pgfpathlineto{\pgfqpoint{4.097901in}{1.718820in}}%
\pgfusepath{stroke}%
\end{pgfscope}%
\begin{pgfscope}%
\pgfpathrectangle{\pgfqpoint{0.588634in}{1.389617in}}{\pgfqpoint{4.998811in}{1.759717in}}%
\pgfusepath{clip}%
\pgfsetbuttcap%
\pgfsetroundjoin%
\pgfsetlinewidth{1.505625pt}%
\definecolor{currentstroke}{rgb}{0.313725,0.317647,0.309804}%
\pgfsetstrokecolor{currentstroke}%
\pgfsetstrokeopacity{0.900000}%
\pgfsetdash{}{0pt}%
\pgfpathmoveto{\pgfqpoint{4.097901in}{1.613108in}}%
\pgfpathlineto{\pgfqpoint{4.350366in}{1.613108in}}%
\pgfusepath{stroke}%
\end{pgfscope}%
\begin{pgfscope}%
\pgfpathrectangle{\pgfqpoint{0.588634in}{1.389617in}}{\pgfqpoint{4.998811in}{1.759717in}}%
\pgfusepath{clip}%
\pgfsetbuttcap%
\pgfsetroundjoin%
\pgfsetlinewidth{1.505625pt}%
\definecolor{currentstroke}{rgb}{0.313725,0.317647,0.309804}%
\pgfsetstrokecolor{currentstroke}%
\pgfsetstrokeopacity{0.900000}%
\pgfsetdash{}{0pt}%
\pgfpathmoveto{\pgfqpoint{4.350366in}{1.545347in}}%
\pgfpathlineto{\pgfqpoint{4.602831in}{1.545347in}}%
\pgfusepath{stroke}%
\end{pgfscope}%
\begin{pgfscope}%
\pgfpathrectangle{\pgfqpoint{0.588634in}{1.389617in}}{\pgfqpoint{4.998811in}{1.759717in}}%
\pgfusepath{clip}%
\pgfsetbuttcap%
\pgfsetroundjoin%
\pgfsetlinewidth{1.505625pt}%
\definecolor{currentstroke}{rgb}{0.313725,0.317647,0.309804}%
\pgfsetstrokecolor{currentstroke}%
\pgfsetstrokeopacity{0.900000}%
\pgfsetdash{}{0pt}%
\pgfpathmoveto{\pgfqpoint{4.602831in}{1.501270in}}%
\pgfpathlineto{\pgfqpoint{4.855296in}{1.501270in}}%
\pgfusepath{stroke}%
\end{pgfscope}%
\begin{pgfscope}%
\pgfpathrectangle{\pgfqpoint{0.588634in}{1.389617in}}{\pgfqpoint{4.998811in}{1.759717in}}%
\pgfusepath{clip}%
\pgfsetbuttcap%
\pgfsetroundjoin%
\pgfsetlinewidth{1.505625pt}%
\definecolor{currentstroke}{rgb}{0.313725,0.317647,0.309804}%
\pgfsetstrokecolor{currentstroke}%
\pgfsetstrokeopacity{0.900000}%
\pgfsetdash{}{0pt}%
\pgfpathmoveto{\pgfqpoint{4.855296in}{1.478361in}}%
\pgfpathlineto{\pgfqpoint{5.107762in}{1.478361in}}%
\pgfusepath{stroke}%
\end{pgfscope}%
\begin{pgfscope}%
\pgfpathrectangle{\pgfqpoint{0.588634in}{1.389617in}}{\pgfqpoint{4.998811in}{1.759717in}}%
\pgfusepath{clip}%
\pgfsetbuttcap%
\pgfsetroundjoin%
\pgfsetlinewidth{1.505625pt}%
\definecolor{currentstroke}{rgb}{0.313725,0.317647,0.309804}%
\pgfsetstrokecolor{currentstroke}%
\pgfsetstrokeopacity{0.900000}%
\pgfsetdash{}{0pt}%
\pgfpathmoveto{\pgfqpoint{5.107762in}{1.474320in}}%
\pgfpathlineto{\pgfqpoint{5.360227in}{1.474320in}}%
\pgfusepath{stroke}%
\end{pgfscope}%
\begin{pgfscope}%
\pgfpathrectangle{\pgfqpoint{0.588634in}{1.389617in}}{\pgfqpoint{4.998811in}{1.759717in}}%
\pgfusepath{clip}%
\pgfsetbuttcap%
\pgfsetroundjoin%
\pgfsetlinewidth{1.505625pt}%
\definecolor{currentstroke}{rgb}{0.313725,0.317647,0.309804}%
\pgfsetstrokecolor{currentstroke}%
\pgfsetstrokeopacity{0.900000}%
\pgfsetdash{}{0pt}%
\pgfpathmoveto{\pgfqpoint{0.942085in}{3.002883in}}%
\pgfpathlineto{\pgfqpoint{0.942085in}{3.069346in}}%
\pgfusepath{stroke}%
\end{pgfscope}%
\begin{pgfscope}%
\pgfpathrectangle{\pgfqpoint{0.588634in}{1.389617in}}{\pgfqpoint{4.998811in}{1.759717in}}%
\pgfusepath{clip}%
\pgfsetbuttcap%
\pgfsetroundjoin%
\pgfsetlinewidth{1.505625pt}%
\definecolor{currentstroke}{rgb}{0.313725,0.317647,0.309804}%
\pgfsetstrokecolor{currentstroke}%
\pgfsetstrokeopacity{0.900000}%
\pgfsetdash{}{0pt}%
\pgfpathmoveto{\pgfqpoint{1.194551in}{2.930261in}}%
\pgfpathlineto{\pgfqpoint{1.194551in}{2.965490in}}%
\pgfusepath{stroke}%
\end{pgfscope}%
\begin{pgfscope}%
\pgfpathrectangle{\pgfqpoint{0.588634in}{1.389617in}}{\pgfqpoint{4.998811in}{1.759717in}}%
\pgfusepath{clip}%
\pgfsetbuttcap%
\pgfsetroundjoin%
\pgfsetlinewidth{1.505625pt}%
\definecolor{currentstroke}{rgb}{0.313725,0.317647,0.309804}%
\pgfsetstrokecolor{currentstroke}%
\pgfsetstrokeopacity{0.900000}%
\pgfsetdash{}{0pt}%
\pgfpathmoveto{\pgfqpoint{1.447016in}{2.876310in}}%
\pgfpathlineto{\pgfqpoint{1.447016in}{2.895422in}}%
\pgfusepath{stroke}%
\end{pgfscope}%
\begin{pgfscope}%
\pgfpathrectangle{\pgfqpoint{0.588634in}{1.389617in}}{\pgfqpoint{4.998811in}{1.759717in}}%
\pgfusepath{clip}%
\pgfsetbuttcap%
\pgfsetroundjoin%
\pgfsetlinewidth{1.505625pt}%
\definecolor{currentstroke}{rgb}{0.313725,0.317647,0.309804}%
\pgfsetstrokecolor{currentstroke}%
\pgfsetstrokeopacity{0.900000}%
\pgfsetdash{}{0pt}%
\pgfpathmoveto{\pgfqpoint{1.699481in}{2.843992in}}%
\pgfpathlineto{\pgfqpoint{1.699481in}{2.857742in}}%
\pgfusepath{stroke}%
\end{pgfscope}%
\begin{pgfscope}%
\pgfpathrectangle{\pgfqpoint{0.588634in}{1.389617in}}{\pgfqpoint{4.998811in}{1.759717in}}%
\pgfusepath{clip}%
\pgfsetbuttcap%
\pgfsetroundjoin%
\pgfsetlinewidth{1.505625pt}%
\definecolor{currentstroke}{rgb}{0.313725,0.317647,0.309804}%
\pgfsetstrokecolor{currentstroke}%
\pgfsetstrokeopacity{0.900000}%
\pgfsetdash{}{0pt}%
\pgfpathmoveto{\pgfqpoint{1.951946in}{2.821497in}}%
\pgfpathlineto{\pgfqpoint{1.951946in}{2.832210in}}%
\pgfusepath{stroke}%
\end{pgfscope}%
\begin{pgfscope}%
\pgfpathrectangle{\pgfqpoint{0.588634in}{1.389617in}}{\pgfqpoint{4.998811in}{1.759717in}}%
\pgfusepath{clip}%
\pgfsetbuttcap%
\pgfsetroundjoin%
\pgfsetlinewidth{1.505625pt}%
\definecolor{currentstroke}{rgb}{0.313725,0.317647,0.309804}%
\pgfsetstrokecolor{currentstroke}%
\pgfsetstrokeopacity{0.900000}%
\pgfsetdash{}{0pt}%
\pgfpathmoveto{\pgfqpoint{2.204412in}{2.787875in}}%
\pgfpathlineto{\pgfqpoint{2.204412in}{2.796370in}}%
\pgfusepath{stroke}%
\end{pgfscope}%
\begin{pgfscope}%
\pgfpathrectangle{\pgfqpoint{0.588634in}{1.389617in}}{\pgfqpoint{4.998811in}{1.759717in}}%
\pgfusepath{clip}%
\pgfsetbuttcap%
\pgfsetroundjoin%
\pgfsetlinewidth{1.505625pt}%
\definecolor{currentstroke}{rgb}{0.313725,0.317647,0.309804}%
\pgfsetstrokecolor{currentstroke}%
\pgfsetstrokeopacity{0.900000}%
\pgfsetdash{}{0pt}%
\pgfpathmoveto{\pgfqpoint{2.456877in}{2.704695in}}%
\pgfpathlineto{\pgfqpoint{2.456877in}{2.713055in}}%
\pgfusepath{stroke}%
\end{pgfscope}%
\begin{pgfscope}%
\pgfpathrectangle{\pgfqpoint{0.588634in}{1.389617in}}{\pgfqpoint{4.998811in}{1.759717in}}%
\pgfusepath{clip}%
\pgfsetbuttcap%
\pgfsetroundjoin%
\pgfsetlinewidth{1.505625pt}%
\definecolor{currentstroke}{rgb}{0.313725,0.317647,0.309804}%
\pgfsetstrokecolor{currentstroke}%
\pgfsetstrokeopacity{0.900000}%
\pgfsetdash{}{0pt}%
\pgfpathmoveto{\pgfqpoint{2.709342in}{2.566070in}}%
\pgfpathlineto{\pgfqpoint{2.709342in}{2.573671in}}%
\pgfusepath{stroke}%
\end{pgfscope}%
\begin{pgfscope}%
\pgfpathrectangle{\pgfqpoint{0.588634in}{1.389617in}}{\pgfqpoint{4.998811in}{1.759717in}}%
\pgfusepath{clip}%
\pgfsetbuttcap%
\pgfsetroundjoin%
\pgfsetlinewidth{1.505625pt}%
\definecolor{currentstroke}{rgb}{0.313725,0.317647,0.309804}%
\pgfsetstrokecolor{currentstroke}%
\pgfsetstrokeopacity{0.900000}%
\pgfsetdash{}{0pt}%
\pgfpathmoveto{\pgfqpoint{2.961807in}{2.382686in}}%
\pgfpathlineto{\pgfqpoint{2.961807in}{2.389853in}}%
\pgfusepath{stroke}%
\end{pgfscope}%
\begin{pgfscope}%
\pgfpathrectangle{\pgfqpoint{0.588634in}{1.389617in}}{\pgfqpoint{4.998811in}{1.759717in}}%
\pgfusepath{clip}%
\pgfsetbuttcap%
\pgfsetroundjoin%
\pgfsetlinewidth{1.505625pt}%
\definecolor{currentstroke}{rgb}{0.313725,0.317647,0.309804}%
\pgfsetstrokecolor{currentstroke}%
\pgfsetstrokeopacity{0.900000}%
\pgfsetdash{}{0pt}%
\pgfpathmoveto{\pgfqpoint{3.214272in}{2.191793in}}%
\pgfpathlineto{\pgfqpoint{3.214272in}{2.198589in}}%
\pgfusepath{stroke}%
\end{pgfscope}%
\begin{pgfscope}%
\pgfpathrectangle{\pgfqpoint{0.588634in}{1.389617in}}{\pgfqpoint{4.998811in}{1.759717in}}%
\pgfusepath{clip}%
\pgfsetbuttcap%
\pgfsetroundjoin%
\pgfsetlinewidth{1.505625pt}%
\definecolor{currentstroke}{rgb}{0.313725,0.317647,0.309804}%
\pgfsetstrokecolor{currentstroke}%
\pgfsetstrokeopacity{0.900000}%
\pgfsetdash{}{0pt}%
\pgfpathmoveto{\pgfqpoint{3.466738in}{2.010947in}}%
\pgfpathlineto{\pgfqpoint{3.466738in}{2.018226in}}%
\pgfusepath{stroke}%
\end{pgfscope}%
\begin{pgfscope}%
\pgfpathrectangle{\pgfqpoint{0.588634in}{1.389617in}}{\pgfqpoint{4.998811in}{1.759717in}}%
\pgfusepath{clip}%
\pgfsetbuttcap%
\pgfsetroundjoin%
\pgfsetlinewidth{1.505625pt}%
\definecolor{currentstroke}{rgb}{0.313725,0.317647,0.309804}%
\pgfsetstrokecolor{currentstroke}%
\pgfsetstrokeopacity{0.900000}%
\pgfsetdash{}{0pt}%
\pgfpathmoveto{\pgfqpoint{3.719203in}{1.854270in}}%
\pgfpathlineto{\pgfqpoint{3.719203in}{1.860886in}}%
\pgfusepath{stroke}%
\end{pgfscope}%
\begin{pgfscope}%
\pgfpathrectangle{\pgfqpoint{0.588634in}{1.389617in}}{\pgfqpoint{4.998811in}{1.759717in}}%
\pgfusepath{clip}%
\pgfsetbuttcap%
\pgfsetroundjoin%
\pgfsetlinewidth{1.505625pt}%
\definecolor{currentstroke}{rgb}{0.313725,0.317647,0.309804}%
\pgfsetstrokecolor{currentstroke}%
\pgfsetstrokeopacity{0.900000}%
\pgfsetdash{}{0pt}%
\pgfpathmoveto{\pgfqpoint{3.971668in}{1.715839in}}%
\pgfpathlineto{\pgfqpoint{3.971668in}{1.721851in}}%
\pgfusepath{stroke}%
\end{pgfscope}%
\begin{pgfscope}%
\pgfpathrectangle{\pgfqpoint{0.588634in}{1.389617in}}{\pgfqpoint{4.998811in}{1.759717in}}%
\pgfusepath{clip}%
\pgfsetbuttcap%
\pgfsetroundjoin%
\pgfsetlinewidth{1.505625pt}%
\definecolor{currentstroke}{rgb}{0.313725,0.317647,0.309804}%
\pgfsetstrokecolor{currentstroke}%
\pgfsetstrokeopacity{0.900000}%
\pgfsetdash{}{0pt}%
\pgfpathmoveto{\pgfqpoint{4.224133in}{1.610515in}}%
\pgfpathlineto{\pgfqpoint{4.224133in}{1.615955in}}%
\pgfusepath{stroke}%
\end{pgfscope}%
\begin{pgfscope}%
\pgfpathrectangle{\pgfqpoint{0.588634in}{1.389617in}}{\pgfqpoint{4.998811in}{1.759717in}}%
\pgfusepath{clip}%
\pgfsetbuttcap%
\pgfsetroundjoin%
\pgfsetlinewidth{1.505625pt}%
\definecolor{currentstroke}{rgb}{0.313725,0.317647,0.309804}%
\pgfsetstrokecolor{currentstroke}%
\pgfsetstrokeopacity{0.900000}%
\pgfsetdash{}{0pt}%
\pgfpathmoveto{\pgfqpoint{4.476598in}{1.542482in}}%
\pgfpathlineto{\pgfqpoint{4.476598in}{1.548024in}}%
\pgfusepath{stroke}%
\end{pgfscope}%
\begin{pgfscope}%
\pgfpathrectangle{\pgfqpoint{0.588634in}{1.389617in}}{\pgfqpoint{4.998811in}{1.759717in}}%
\pgfusepath{clip}%
\pgfsetbuttcap%
\pgfsetroundjoin%
\pgfsetlinewidth{1.505625pt}%
\definecolor{currentstroke}{rgb}{0.313725,0.317647,0.309804}%
\pgfsetstrokecolor{currentstroke}%
\pgfsetstrokeopacity{0.900000}%
\pgfsetdash{}{0pt}%
\pgfpathmoveto{\pgfqpoint{4.729064in}{1.498565in}}%
\pgfpathlineto{\pgfqpoint{4.729064in}{1.504058in}}%
\pgfusepath{stroke}%
\end{pgfscope}%
\begin{pgfscope}%
\pgfpathrectangle{\pgfqpoint{0.588634in}{1.389617in}}{\pgfqpoint{4.998811in}{1.759717in}}%
\pgfusepath{clip}%
\pgfsetbuttcap%
\pgfsetroundjoin%
\pgfsetlinewidth{1.505625pt}%
\definecolor{currentstroke}{rgb}{0.313725,0.317647,0.309804}%
\pgfsetstrokecolor{currentstroke}%
\pgfsetstrokeopacity{0.900000}%
\pgfsetdash{}{0pt}%
\pgfpathmoveto{\pgfqpoint{4.981529in}{1.475142in}}%
\pgfpathlineto{\pgfqpoint{4.981529in}{1.481234in}}%
\pgfusepath{stroke}%
\end{pgfscope}%
\begin{pgfscope}%
\pgfpathrectangle{\pgfqpoint{0.588634in}{1.389617in}}{\pgfqpoint{4.998811in}{1.759717in}}%
\pgfusepath{clip}%
\pgfsetbuttcap%
\pgfsetroundjoin%
\pgfsetlinewidth{1.505625pt}%
\definecolor{currentstroke}{rgb}{0.313725,0.317647,0.309804}%
\pgfsetstrokecolor{currentstroke}%
\pgfsetstrokeopacity{0.900000}%
\pgfsetdash{}{0pt}%
\pgfpathmoveto{\pgfqpoint{5.233994in}{1.469604in}}%
\pgfpathlineto{\pgfqpoint{5.233994in}{1.478997in}}%
\pgfusepath{stroke}%
\end{pgfscope}%
\begin{pgfscope}%
\pgfpathrectangle{\pgfqpoint{0.588634in}{1.389617in}}{\pgfqpoint{4.998811in}{1.759717in}}%
\pgfusepath{clip}%
\pgfsetbuttcap%
\pgfsetroundjoin%
\pgfsetlinewidth{1.505625pt}%
\definecolor{currentstroke}{rgb}{0.949020,0.372549,0.360784}%
\pgfsetstrokecolor{currentstroke}%
\pgfsetstrokeopacity{0.900000}%
\pgfsetdash{}{0pt}%
\pgfpathmoveto{\pgfqpoint{0.815853in}{2.691087in}}%
\pgfpathlineto{\pgfqpoint{1.068318in}{2.691087in}}%
\pgfusepath{stroke}%
\end{pgfscope}%
\begin{pgfscope}%
\pgfpathrectangle{\pgfqpoint{0.588634in}{1.389617in}}{\pgfqpoint{4.998811in}{1.759717in}}%
\pgfusepath{clip}%
\pgfsetbuttcap%
\pgfsetroundjoin%
\pgfsetlinewidth{1.505625pt}%
\definecolor{currentstroke}{rgb}{0.949020,0.372549,0.360784}%
\pgfsetstrokecolor{currentstroke}%
\pgfsetstrokeopacity{0.900000}%
\pgfsetdash{}{0pt}%
\pgfpathmoveto{\pgfqpoint{1.068318in}{2.614381in}}%
\pgfpathlineto{\pgfqpoint{1.320783in}{2.614381in}}%
\pgfusepath{stroke}%
\end{pgfscope}%
\begin{pgfscope}%
\pgfpathrectangle{\pgfqpoint{0.588634in}{1.389617in}}{\pgfqpoint{4.998811in}{1.759717in}}%
\pgfusepath{clip}%
\pgfsetbuttcap%
\pgfsetroundjoin%
\pgfsetlinewidth{1.505625pt}%
\definecolor{currentstroke}{rgb}{0.949020,0.372549,0.360784}%
\pgfsetstrokecolor{currentstroke}%
\pgfsetstrokeopacity{0.900000}%
\pgfsetdash{}{0pt}%
\pgfpathmoveto{\pgfqpoint{1.320783in}{2.600045in}}%
\pgfpathlineto{\pgfqpoint{1.573249in}{2.600045in}}%
\pgfusepath{stroke}%
\end{pgfscope}%
\begin{pgfscope}%
\pgfpathrectangle{\pgfqpoint{0.588634in}{1.389617in}}{\pgfqpoint{4.998811in}{1.759717in}}%
\pgfusepath{clip}%
\pgfsetbuttcap%
\pgfsetroundjoin%
\pgfsetlinewidth{1.505625pt}%
\definecolor{currentstroke}{rgb}{0.949020,0.372549,0.360784}%
\pgfsetstrokecolor{currentstroke}%
\pgfsetstrokeopacity{0.900000}%
\pgfsetdash{}{0pt}%
\pgfpathmoveto{\pgfqpoint{1.573249in}{2.581993in}}%
\pgfpathlineto{\pgfqpoint{1.825714in}{2.581993in}}%
\pgfusepath{stroke}%
\end{pgfscope}%
\begin{pgfscope}%
\pgfpathrectangle{\pgfqpoint{0.588634in}{1.389617in}}{\pgfqpoint{4.998811in}{1.759717in}}%
\pgfusepath{clip}%
\pgfsetbuttcap%
\pgfsetroundjoin%
\pgfsetlinewidth{1.505625pt}%
\definecolor{currentstroke}{rgb}{0.949020,0.372549,0.360784}%
\pgfsetstrokecolor{currentstroke}%
\pgfsetstrokeopacity{0.900000}%
\pgfsetdash{}{0pt}%
\pgfpathmoveto{\pgfqpoint{1.825714in}{2.570113in}}%
\pgfpathlineto{\pgfqpoint{2.078179in}{2.570113in}}%
\pgfusepath{stroke}%
\end{pgfscope}%
\begin{pgfscope}%
\pgfpathrectangle{\pgfqpoint{0.588634in}{1.389617in}}{\pgfqpoint{4.998811in}{1.759717in}}%
\pgfusepath{clip}%
\pgfsetbuttcap%
\pgfsetroundjoin%
\pgfsetlinewidth{1.505625pt}%
\definecolor{currentstroke}{rgb}{0.949020,0.372549,0.360784}%
\pgfsetstrokecolor{currentstroke}%
\pgfsetstrokeopacity{0.900000}%
\pgfsetdash{}{0pt}%
\pgfpathmoveto{\pgfqpoint{2.078179in}{2.535488in}}%
\pgfpathlineto{\pgfqpoint{2.330644in}{2.535488in}}%
\pgfusepath{stroke}%
\end{pgfscope}%
\begin{pgfscope}%
\pgfpathrectangle{\pgfqpoint{0.588634in}{1.389617in}}{\pgfqpoint{4.998811in}{1.759717in}}%
\pgfusepath{clip}%
\pgfsetbuttcap%
\pgfsetroundjoin%
\pgfsetlinewidth{1.505625pt}%
\definecolor{currentstroke}{rgb}{0.949020,0.372549,0.360784}%
\pgfsetstrokecolor{currentstroke}%
\pgfsetstrokeopacity{0.900000}%
\pgfsetdash{}{0pt}%
\pgfpathmoveto{\pgfqpoint{2.330644in}{2.464800in}}%
\pgfpathlineto{\pgfqpoint{2.583109in}{2.464800in}}%
\pgfusepath{stroke}%
\end{pgfscope}%
\begin{pgfscope}%
\pgfpathrectangle{\pgfqpoint{0.588634in}{1.389617in}}{\pgfqpoint{4.998811in}{1.759717in}}%
\pgfusepath{clip}%
\pgfsetbuttcap%
\pgfsetroundjoin%
\pgfsetlinewidth{1.505625pt}%
\definecolor{currentstroke}{rgb}{0.949020,0.372549,0.360784}%
\pgfsetstrokecolor{currentstroke}%
\pgfsetstrokeopacity{0.900000}%
\pgfsetdash{}{0pt}%
\pgfpathmoveto{\pgfqpoint{2.583109in}{2.363436in}}%
\pgfpathlineto{\pgfqpoint{2.835575in}{2.363436in}}%
\pgfusepath{stroke}%
\end{pgfscope}%
\begin{pgfscope}%
\pgfpathrectangle{\pgfqpoint{0.588634in}{1.389617in}}{\pgfqpoint{4.998811in}{1.759717in}}%
\pgfusepath{clip}%
\pgfsetbuttcap%
\pgfsetroundjoin%
\pgfsetlinewidth{1.505625pt}%
\definecolor{currentstroke}{rgb}{0.949020,0.372549,0.360784}%
\pgfsetstrokecolor{currentstroke}%
\pgfsetstrokeopacity{0.900000}%
\pgfsetdash{}{0pt}%
\pgfpathmoveto{\pgfqpoint{2.835575in}{2.237061in}}%
\pgfpathlineto{\pgfqpoint{3.088040in}{2.237061in}}%
\pgfusepath{stroke}%
\end{pgfscope}%
\begin{pgfscope}%
\pgfpathrectangle{\pgfqpoint{0.588634in}{1.389617in}}{\pgfqpoint{4.998811in}{1.759717in}}%
\pgfusepath{clip}%
\pgfsetbuttcap%
\pgfsetroundjoin%
\pgfsetlinewidth{1.505625pt}%
\definecolor{currentstroke}{rgb}{0.949020,0.372549,0.360784}%
\pgfsetstrokecolor{currentstroke}%
\pgfsetstrokeopacity{0.900000}%
\pgfsetdash{}{0pt}%
\pgfpathmoveto{\pgfqpoint{3.088040in}{2.117601in}}%
\pgfpathlineto{\pgfqpoint{3.340505in}{2.117601in}}%
\pgfusepath{stroke}%
\end{pgfscope}%
\begin{pgfscope}%
\pgfpathrectangle{\pgfqpoint{0.588634in}{1.389617in}}{\pgfqpoint{4.998811in}{1.759717in}}%
\pgfusepath{clip}%
\pgfsetbuttcap%
\pgfsetroundjoin%
\pgfsetlinewidth{1.505625pt}%
\definecolor{currentstroke}{rgb}{0.949020,0.372549,0.360784}%
\pgfsetstrokecolor{currentstroke}%
\pgfsetstrokeopacity{0.900000}%
\pgfsetdash{}{0pt}%
\pgfpathmoveto{\pgfqpoint{3.340505in}{2.013777in}}%
\pgfpathlineto{\pgfqpoint{3.592970in}{2.013777in}}%
\pgfusepath{stroke}%
\end{pgfscope}%
\begin{pgfscope}%
\pgfpathrectangle{\pgfqpoint{0.588634in}{1.389617in}}{\pgfqpoint{4.998811in}{1.759717in}}%
\pgfusepath{clip}%
\pgfsetbuttcap%
\pgfsetroundjoin%
\pgfsetlinewidth{1.505625pt}%
\definecolor{currentstroke}{rgb}{0.949020,0.372549,0.360784}%
\pgfsetstrokecolor{currentstroke}%
\pgfsetstrokeopacity{0.900000}%
\pgfsetdash{}{0pt}%
\pgfpathmoveto{\pgfqpoint{3.592970in}{1.927419in}}%
\pgfpathlineto{\pgfqpoint{3.845435in}{1.927419in}}%
\pgfusepath{stroke}%
\end{pgfscope}%
\begin{pgfscope}%
\pgfpathrectangle{\pgfqpoint{0.588634in}{1.389617in}}{\pgfqpoint{4.998811in}{1.759717in}}%
\pgfusepath{clip}%
\pgfsetbuttcap%
\pgfsetroundjoin%
\pgfsetlinewidth{1.505625pt}%
\definecolor{currentstroke}{rgb}{0.949020,0.372549,0.360784}%
\pgfsetstrokecolor{currentstroke}%
\pgfsetstrokeopacity{0.900000}%
\pgfsetdash{}{0pt}%
\pgfpathmoveto{\pgfqpoint{3.845435in}{1.859305in}}%
\pgfpathlineto{\pgfqpoint{4.097901in}{1.859305in}}%
\pgfusepath{stroke}%
\end{pgfscope}%
\begin{pgfscope}%
\pgfpathrectangle{\pgfqpoint{0.588634in}{1.389617in}}{\pgfqpoint{4.998811in}{1.759717in}}%
\pgfusepath{clip}%
\pgfsetbuttcap%
\pgfsetroundjoin%
\pgfsetlinewidth{1.505625pt}%
\definecolor{currentstroke}{rgb}{0.949020,0.372549,0.360784}%
\pgfsetstrokecolor{currentstroke}%
\pgfsetstrokeopacity{0.900000}%
\pgfsetdash{}{0pt}%
\pgfpathmoveto{\pgfqpoint{4.097901in}{1.806542in}}%
\pgfpathlineto{\pgfqpoint{4.350366in}{1.806542in}}%
\pgfusepath{stroke}%
\end{pgfscope}%
\begin{pgfscope}%
\pgfpathrectangle{\pgfqpoint{0.588634in}{1.389617in}}{\pgfqpoint{4.998811in}{1.759717in}}%
\pgfusepath{clip}%
\pgfsetbuttcap%
\pgfsetroundjoin%
\pgfsetlinewidth{1.505625pt}%
\definecolor{currentstroke}{rgb}{0.949020,0.372549,0.360784}%
\pgfsetstrokecolor{currentstroke}%
\pgfsetstrokeopacity{0.900000}%
\pgfsetdash{}{0pt}%
\pgfpathmoveto{\pgfqpoint{4.350366in}{1.774888in}}%
\pgfpathlineto{\pgfqpoint{4.602831in}{1.774888in}}%
\pgfusepath{stroke}%
\end{pgfscope}%
\begin{pgfscope}%
\pgfpathrectangle{\pgfqpoint{0.588634in}{1.389617in}}{\pgfqpoint{4.998811in}{1.759717in}}%
\pgfusepath{clip}%
\pgfsetbuttcap%
\pgfsetroundjoin%
\pgfsetlinewidth{1.505625pt}%
\definecolor{currentstroke}{rgb}{0.949020,0.372549,0.360784}%
\pgfsetstrokecolor{currentstroke}%
\pgfsetstrokeopacity{0.900000}%
\pgfsetdash{}{0pt}%
\pgfpathmoveto{\pgfqpoint{4.602831in}{1.764581in}}%
\pgfpathlineto{\pgfqpoint{4.855296in}{1.764581in}}%
\pgfusepath{stroke}%
\end{pgfscope}%
\begin{pgfscope}%
\pgfpathrectangle{\pgfqpoint{0.588634in}{1.389617in}}{\pgfqpoint{4.998811in}{1.759717in}}%
\pgfusepath{clip}%
\pgfsetbuttcap%
\pgfsetroundjoin%
\pgfsetlinewidth{1.505625pt}%
\definecolor{currentstroke}{rgb}{0.949020,0.372549,0.360784}%
\pgfsetstrokecolor{currentstroke}%
\pgfsetstrokeopacity{0.900000}%
\pgfsetdash{}{0pt}%
\pgfpathmoveto{\pgfqpoint{4.855296in}{1.779623in}}%
\pgfpathlineto{\pgfqpoint{5.107762in}{1.779623in}}%
\pgfusepath{stroke}%
\end{pgfscope}%
\begin{pgfscope}%
\pgfpathrectangle{\pgfqpoint{0.588634in}{1.389617in}}{\pgfqpoint{4.998811in}{1.759717in}}%
\pgfusepath{clip}%
\pgfsetbuttcap%
\pgfsetroundjoin%
\pgfsetlinewidth{1.505625pt}%
\definecolor{currentstroke}{rgb}{0.949020,0.372549,0.360784}%
\pgfsetstrokecolor{currentstroke}%
\pgfsetstrokeopacity{0.900000}%
\pgfsetdash{}{0pt}%
\pgfpathmoveto{\pgfqpoint{5.107762in}{1.827544in}}%
\pgfpathlineto{\pgfqpoint{5.360227in}{1.827544in}}%
\pgfusepath{stroke}%
\end{pgfscope}%
\begin{pgfscope}%
\pgfpathrectangle{\pgfqpoint{0.588634in}{1.389617in}}{\pgfqpoint{4.998811in}{1.759717in}}%
\pgfusepath{clip}%
\pgfsetbuttcap%
\pgfsetroundjoin%
\pgfsetlinewidth{1.505625pt}%
\definecolor{currentstroke}{rgb}{0.949020,0.372549,0.360784}%
\pgfsetstrokecolor{currentstroke}%
\pgfsetstrokeopacity{0.900000}%
\pgfsetdash{}{0pt}%
\pgfpathmoveto{\pgfqpoint{0.942085in}{2.664218in}}%
\pgfpathlineto{\pgfqpoint{0.942085in}{2.720548in}}%
\pgfusepath{stroke}%
\end{pgfscope}%
\begin{pgfscope}%
\pgfpathrectangle{\pgfqpoint{0.588634in}{1.389617in}}{\pgfqpoint{4.998811in}{1.759717in}}%
\pgfusepath{clip}%
\pgfsetbuttcap%
\pgfsetroundjoin%
\pgfsetlinewidth{1.505625pt}%
\definecolor{currentstroke}{rgb}{0.949020,0.372549,0.360784}%
\pgfsetstrokecolor{currentstroke}%
\pgfsetstrokeopacity{0.900000}%
\pgfsetdash{}{0pt}%
\pgfpathmoveto{\pgfqpoint{1.194551in}{2.599780in}}%
\pgfpathlineto{\pgfqpoint{1.194551in}{2.627940in}}%
\pgfusepath{stroke}%
\end{pgfscope}%
\begin{pgfscope}%
\pgfpathrectangle{\pgfqpoint{0.588634in}{1.389617in}}{\pgfqpoint{4.998811in}{1.759717in}}%
\pgfusepath{clip}%
\pgfsetbuttcap%
\pgfsetroundjoin%
\pgfsetlinewidth{1.505625pt}%
\definecolor{currentstroke}{rgb}{0.949020,0.372549,0.360784}%
\pgfsetstrokecolor{currentstroke}%
\pgfsetstrokeopacity{0.900000}%
\pgfsetdash{}{0pt}%
\pgfpathmoveto{\pgfqpoint{1.447016in}{2.592267in}}%
\pgfpathlineto{\pgfqpoint{1.447016in}{2.608231in}}%
\pgfusepath{stroke}%
\end{pgfscope}%
\begin{pgfscope}%
\pgfpathrectangle{\pgfqpoint{0.588634in}{1.389617in}}{\pgfqpoint{4.998811in}{1.759717in}}%
\pgfusepath{clip}%
\pgfsetbuttcap%
\pgfsetroundjoin%
\pgfsetlinewidth{1.505625pt}%
\definecolor{currentstroke}{rgb}{0.949020,0.372549,0.360784}%
\pgfsetstrokecolor{currentstroke}%
\pgfsetstrokeopacity{0.900000}%
\pgfsetdash{}{0pt}%
\pgfpathmoveto{\pgfqpoint{1.699481in}{2.576719in}}%
\pgfpathlineto{\pgfqpoint{1.699481in}{2.586945in}}%
\pgfusepath{stroke}%
\end{pgfscope}%
\begin{pgfscope}%
\pgfpathrectangle{\pgfqpoint{0.588634in}{1.389617in}}{\pgfqpoint{4.998811in}{1.759717in}}%
\pgfusepath{clip}%
\pgfsetbuttcap%
\pgfsetroundjoin%
\pgfsetlinewidth{1.505625pt}%
\definecolor{currentstroke}{rgb}{0.949020,0.372549,0.360784}%
\pgfsetstrokecolor{currentstroke}%
\pgfsetstrokeopacity{0.900000}%
\pgfsetdash{}{0pt}%
\pgfpathmoveto{\pgfqpoint{1.951946in}{2.565662in}}%
\pgfpathlineto{\pgfqpoint{1.951946in}{2.574195in}}%
\pgfusepath{stroke}%
\end{pgfscope}%
\begin{pgfscope}%
\pgfpathrectangle{\pgfqpoint{0.588634in}{1.389617in}}{\pgfqpoint{4.998811in}{1.759717in}}%
\pgfusepath{clip}%
\pgfsetbuttcap%
\pgfsetroundjoin%
\pgfsetlinewidth{1.505625pt}%
\definecolor{currentstroke}{rgb}{0.949020,0.372549,0.360784}%
\pgfsetstrokecolor{currentstroke}%
\pgfsetstrokeopacity{0.900000}%
\pgfsetdash{}{0pt}%
\pgfpathmoveto{\pgfqpoint{2.204412in}{2.532346in}}%
\pgfpathlineto{\pgfqpoint{2.204412in}{2.539470in}}%
\pgfusepath{stroke}%
\end{pgfscope}%
\begin{pgfscope}%
\pgfpathrectangle{\pgfqpoint{0.588634in}{1.389617in}}{\pgfqpoint{4.998811in}{1.759717in}}%
\pgfusepath{clip}%
\pgfsetbuttcap%
\pgfsetroundjoin%
\pgfsetlinewidth{1.505625pt}%
\definecolor{currentstroke}{rgb}{0.949020,0.372549,0.360784}%
\pgfsetstrokecolor{currentstroke}%
\pgfsetstrokeopacity{0.900000}%
\pgfsetdash{}{0pt}%
\pgfpathmoveto{\pgfqpoint{2.456877in}{2.461478in}}%
\pgfpathlineto{\pgfqpoint{2.456877in}{2.468020in}}%
\pgfusepath{stroke}%
\end{pgfscope}%
\begin{pgfscope}%
\pgfpathrectangle{\pgfqpoint{0.588634in}{1.389617in}}{\pgfqpoint{4.998811in}{1.759717in}}%
\pgfusepath{clip}%
\pgfsetbuttcap%
\pgfsetroundjoin%
\pgfsetlinewidth{1.505625pt}%
\definecolor{currentstroke}{rgb}{0.949020,0.372549,0.360784}%
\pgfsetstrokecolor{currentstroke}%
\pgfsetstrokeopacity{0.900000}%
\pgfsetdash{}{0pt}%
\pgfpathmoveto{\pgfqpoint{2.709342in}{2.360531in}}%
\pgfpathlineto{\pgfqpoint{2.709342in}{2.366405in}}%
\pgfusepath{stroke}%
\end{pgfscope}%
\begin{pgfscope}%
\pgfpathrectangle{\pgfqpoint{0.588634in}{1.389617in}}{\pgfqpoint{4.998811in}{1.759717in}}%
\pgfusepath{clip}%
\pgfsetbuttcap%
\pgfsetroundjoin%
\pgfsetlinewidth{1.505625pt}%
\definecolor{currentstroke}{rgb}{0.949020,0.372549,0.360784}%
\pgfsetstrokecolor{currentstroke}%
\pgfsetstrokeopacity{0.900000}%
\pgfsetdash{}{0pt}%
\pgfpathmoveto{\pgfqpoint{2.961807in}{2.234249in}}%
\pgfpathlineto{\pgfqpoint{2.961807in}{2.239883in}}%
\pgfusepath{stroke}%
\end{pgfscope}%
\begin{pgfscope}%
\pgfpathrectangle{\pgfqpoint{0.588634in}{1.389617in}}{\pgfqpoint{4.998811in}{1.759717in}}%
\pgfusepath{clip}%
\pgfsetbuttcap%
\pgfsetroundjoin%
\pgfsetlinewidth{1.505625pt}%
\definecolor{currentstroke}{rgb}{0.949020,0.372549,0.360784}%
\pgfsetstrokecolor{currentstroke}%
\pgfsetstrokeopacity{0.900000}%
\pgfsetdash{}{0pt}%
\pgfpathmoveto{\pgfqpoint{3.214272in}{2.114966in}}%
\pgfpathlineto{\pgfqpoint{3.214272in}{2.120217in}}%
\pgfusepath{stroke}%
\end{pgfscope}%
\begin{pgfscope}%
\pgfpathrectangle{\pgfqpoint{0.588634in}{1.389617in}}{\pgfqpoint{4.998811in}{1.759717in}}%
\pgfusepath{clip}%
\pgfsetbuttcap%
\pgfsetroundjoin%
\pgfsetlinewidth{1.505625pt}%
\definecolor{currentstroke}{rgb}{0.949020,0.372549,0.360784}%
\pgfsetstrokecolor{currentstroke}%
\pgfsetstrokeopacity{0.900000}%
\pgfsetdash{}{0pt}%
\pgfpathmoveto{\pgfqpoint{3.466738in}{2.011185in}}%
\pgfpathlineto{\pgfqpoint{3.466738in}{2.016142in}}%
\pgfusepath{stroke}%
\end{pgfscope}%
\begin{pgfscope}%
\pgfpathrectangle{\pgfqpoint{0.588634in}{1.389617in}}{\pgfqpoint{4.998811in}{1.759717in}}%
\pgfusepath{clip}%
\pgfsetbuttcap%
\pgfsetroundjoin%
\pgfsetlinewidth{1.505625pt}%
\definecolor{currentstroke}{rgb}{0.949020,0.372549,0.360784}%
\pgfsetstrokecolor{currentstroke}%
\pgfsetstrokeopacity{0.900000}%
\pgfsetdash{}{0pt}%
\pgfpathmoveto{\pgfqpoint{3.719203in}{1.924525in}}%
\pgfpathlineto{\pgfqpoint{3.719203in}{1.930439in}}%
\pgfusepath{stroke}%
\end{pgfscope}%
\begin{pgfscope}%
\pgfpathrectangle{\pgfqpoint{0.588634in}{1.389617in}}{\pgfqpoint{4.998811in}{1.759717in}}%
\pgfusepath{clip}%
\pgfsetbuttcap%
\pgfsetroundjoin%
\pgfsetlinewidth{1.505625pt}%
\definecolor{currentstroke}{rgb}{0.949020,0.372549,0.360784}%
\pgfsetstrokecolor{currentstroke}%
\pgfsetstrokeopacity{0.900000}%
\pgfsetdash{}{0pt}%
\pgfpathmoveto{\pgfqpoint{3.971668in}{1.856427in}}%
\pgfpathlineto{\pgfqpoint{3.971668in}{1.862066in}}%
\pgfusepath{stroke}%
\end{pgfscope}%
\begin{pgfscope}%
\pgfpathrectangle{\pgfqpoint{0.588634in}{1.389617in}}{\pgfqpoint{4.998811in}{1.759717in}}%
\pgfusepath{clip}%
\pgfsetbuttcap%
\pgfsetroundjoin%
\pgfsetlinewidth{1.505625pt}%
\definecolor{currentstroke}{rgb}{0.949020,0.372549,0.360784}%
\pgfsetstrokecolor{currentstroke}%
\pgfsetstrokeopacity{0.900000}%
\pgfsetdash{}{0pt}%
\pgfpathmoveto{\pgfqpoint{4.224133in}{1.803206in}}%
\pgfpathlineto{\pgfqpoint{4.224133in}{1.810016in}}%
\pgfusepath{stroke}%
\end{pgfscope}%
\begin{pgfscope}%
\pgfpathrectangle{\pgfqpoint{0.588634in}{1.389617in}}{\pgfqpoint{4.998811in}{1.759717in}}%
\pgfusepath{clip}%
\pgfsetbuttcap%
\pgfsetroundjoin%
\pgfsetlinewidth{1.505625pt}%
\definecolor{currentstroke}{rgb}{0.949020,0.372549,0.360784}%
\pgfsetstrokecolor{currentstroke}%
\pgfsetstrokeopacity{0.900000}%
\pgfsetdash{}{0pt}%
\pgfpathmoveto{\pgfqpoint{4.476598in}{1.770883in}}%
\pgfpathlineto{\pgfqpoint{4.476598in}{1.778451in}}%
\pgfusepath{stroke}%
\end{pgfscope}%
\begin{pgfscope}%
\pgfpathrectangle{\pgfqpoint{0.588634in}{1.389617in}}{\pgfqpoint{4.998811in}{1.759717in}}%
\pgfusepath{clip}%
\pgfsetbuttcap%
\pgfsetroundjoin%
\pgfsetlinewidth{1.505625pt}%
\definecolor{currentstroke}{rgb}{0.949020,0.372549,0.360784}%
\pgfsetstrokecolor{currentstroke}%
\pgfsetstrokeopacity{0.900000}%
\pgfsetdash{}{0pt}%
\pgfpathmoveto{\pgfqpoint{4.729064in}{1.759715in}}%
\pgfpathlineto{\pgfqpoint{4.729064in}{1.769286in}}%
\pgfusepath{stroke}%
\end{pgfscope}%
\begin{pgfscope}%
\pgfpathrectangle{\pgfqpoint{0.588634in}{1.389617in}}{\pgfqpoint{4.998811in}{1.759717in}}%
\pgfusepath{clip}%
\pgfsetbuttcap%
\pgfsetroundjoin%
\pgfsetlinewidth{1.505625pt}%
\definecolor{currentstroke}{rgb}{0.949020,0.372549,0.360784}%
\pgfsetstrokecolor{currentstroke}%
\pgfsetstrokeopacity{0.900000}%
\pgfsetdash{}{0pt}%
\pgfpathmoveto{\pgfqpoint{4.981529in}{1.772613in}}%
\pgfpathlineto{\pgfqpoint{4.981529in}{1.786778in}}%
\pgfusepath{stroke}%
\end{pgfscope}%
\begin{pgfscope}%
\pgfpathrectangle{\pgfqpoint{0.588634in}{1.389617in}}{\pgfqpoint{4.998811in}{1.759717in}}%
\pgfusepath{clip}%
\pgfsetbuttcap%
\pgfsetroundjoin%
\pgfsetlinewidth{1.505625pt}%
\definecolor{currentstroke}{rgb}{0.949020,0.372549,0.360784}%
\pgfsetstrokecolor{currentstroke}%
\pgfsetstrokeopacity{0.900000}%
\pgfsetdash{}{0pt}%
\pgfpathmoveto{\pgfqpoint{5.233994in}{1.816051in}}%
\pgfpathlineto{\pgfqpoint{5.233994in}{1.839855in}}%
\pgfusepath{stroke}%
\end{pgfscope}%
\begin{pgfscope}%
\pgfpathrectangle{\pgfqpoint{0.588634in}{1.389617in}}{\pgfqpoint{4.998811in}{1.759717in}}%
\pgfusepath{clip}%
\pgfsetbuttcap%
\pgfsetroundjoin%
\definecolor{currentfill}{rgb}{0.313725,0.317647,0.309804}%
\pgfsetfillcolor{currentfill}%
\pgfsetfillopacity{0.900000}%
\pgfsetlinewidth{1.003750pt}%
\definecolor{currentstroke}{rgb}{0.313725,0.317647,0.309804}%
\pgfsetstrokecolor{currentstroke}%
\pgfsetstrokeopacity{0.900000}%
\pgfsetdash{}{0pt}%
\pgfsys@defobject{currentmarker}{\pgfqpoint{0.000000in}{-0.013889in}}{\pgfqpoint{0.000000in}{0.013889in}}{%
\pgfpathmoveto{\pgfqpoint{0.000000in}{-0.013889in}}%
\pgfpathlineto{\pgfqpoint{0.000000in}{0.013889in}}%
\pgfusepath{stroke,fill}%
}%
\begin{pgfscope}%
\pgfsys@transformshift{0.815853in}{3.034810in}%
\pgfsys@useobject{currentmarker}{}%
\end{pgfscope}%
\begin{pgfscope}%
\pgfsys@transformshift{1.068318in}{2.946714in}%
\pgfsys@useobject{currentmarker}{}%
\end{pgfscope}%
\begin{pgfscope}%
\pgfsys@transformshift{1.320783in}{2.885802in}%
\pgfsys@useobject{currentmarker}{}%
\end{pgfscope}%
\begin{pgfscope}%
\pgfsys@transformshift{1.573249in}{2.850759in}%
\pgfsys@useobject{currentmarker}{}%
\end{pgfscope}%
\begin{pgfscope}%
\pgfsys@transformshift{1.825714in}{2.826646in}%
\pgfsys@useobject{currentmarker}{}%
\end{pgfscope}%
\begin{pgfscope}%
\pgfsys@transformshift{2.078179in}{2.791834in}%
\pgfsys@useobject{currentmarker}{}%
\end{pgfscope}%
\begin{pgfscope}%
\pgfsys@transformshift{2.330644in}{2.708700in}%
\pgfsys@useobject{currentmarker}{}%
\end{pgfscope}%
\begin{pgfscope}%
\pgfsys@transformshift{2.583109in}{2.569857in}%
\pgfsys@useobject{currentmarker}{}%
\end{pgfscope}%
\begin{pgfscope}%
\pgfsys@transformshift{2.835575in}{2.385982in}%
\pgfsys@useobject{currentmarker}{}%
\end{pgfscope}%
\begin{pgfscope}%
\pgfsys@transformshift{3.088040in}{2.194775in}%
\pgfsys@useobject{currentmarker}{}%
\end{pgfscope}%
\begin{pgfscope}%
\pgfsys@transformshift{3.340505in}{2.014706in}%
\pgfsys@useobject{currentmarker}{}%
\end{pgfscope}%
\begin{pgfscope}%
\pgfsys@transformshift{3.592970in}{1.857604in}%
\pgfsys@useobject{currentmarker}{}%
\end{pgfscope}%
\begin{pgfscope}%
\pgfsys@transformshift{3.845435in}{1.718820in}%
\pgfsys@useobject{currentmarker}{}%
\end{pgfscope}%
\begin{pgfscope}%
\pgfsys@transformshift{4.097901in}{1.613108in}%
\pgfsys@useobject{currentmarker}{}%
\end{pgfscope}%
\begin{pgfscope}%
\pgfsys@transformshift{4.350366in}{1.545347in}%
\pgfsys@useobject{currentmarker}{}%
\end{pgfscope}%
\begin{pgfscope}%
\pgfsys@transformshift{4.602831in}{1.501270in}%
\pgfsys@useobject{currentmarker}{}%
\end{pgfscope}%
\begin{pgfscope}%
\pgfsys@transformshift{4.855296in}{1.478361in}%
\pgfsys@useobject{currentmarker}{}%
\end{pgfscope}%
\begin{pgfscope}%
\pgfsys@transformshift{5.107762in}{1.474320in}%
\pgfsys@useobject{currentmarker}{}%
\end{pgfscope}%
\end{pgfscope}%
\begin{pgfscope}%
\pgfpathrectangle{\pgfqpoint{0.588634in}{1.389617in}}{\pgfqpoint{4.998811in}{1.759717in}}%
\pgfusepath{clip}%
\pgfsetbuttcap%
\pgfsetroundjoin%
\definecolor{currentfill}{rgb}{0.313725,0.317647,0.309804}%
\pgfsetfillcolor{currentfill}%
\pgfsetfillopacity{0.900000}%
\pgfsetlinewidth{1.003750pt}%
\definecolor{currentstroke}{rgb}{0.313725,0.317647,0.309804}%
\pgfsetstrokecolor{currentstroke}%
\pgfsetstrokeopacity{0.900000}%
\pgfsetdash{}{0pt}%
\pgfsys@defobject{currentmarker}{\pgfqpoint{0.000000in}{-0.013889in}}{\pgfqpoint{0.000000in}{0.013889in}}{%
\pgfpathmoveto{\pgfqpoint{0.000000in}{-0.013889in}}%
\pgfpathlineto{\pgfqpoint{0.000000in}{0.013889in}}%
\pgfusepath{stroke,fill}%
}%
\begin{pgfscope}%
\pgfsys@transformshift{1.068318in}{3.034810in}%
\pgfsys@useobject{currentmarker}{}%
\end{pgfscope}%
\begin{pgfscope}%
\pgfsys@transformshift{1.320783in}{2.946714in}%
\pgfsys@useobject{currentmarker}{}%
\end{pgfscope}%
\begin{pgfscope}%
\pgfsys@transformshift{1.573249in}{2.885802in}%
\pgfsys@useobject{currentmarker}{}%
\end{pgfscope}%
\begin{pgfscope}%
\pgfsys@transformshift{1.825714in}{2.850759in}%
\pgfsys@useobject{currentmarker}{}%
\end{pgfscope}%
\begin{pgfscope}%
\pgfsys@transformshift{2.078179in}{2.826646in}%
\pgfsys@useobject{currentmarker}{}%
\end{pgfscope}%
\begin{pgfscope}%
\pgfsys@transformshift{2.330644in}{2.791834in}%
\pgfsys@useobject{currentmarker}{}%
\end{pgfscope}%
\begin{pgfscope}%
\pgfsys@transformshift{2.583109in}{2.708700in}%
\pgfsys@useobject{currentmarker}{}%
\end{pgfscope}%
\begin{pgfscope}%
\pgfsys@transformshift{2.835575in}{2.569857in}%
\pgfsys@useobject{currentmarker}{}%
\end{pgfscope}%
\begin{pgfscope}%
\pgfsys@transformshift{3.088040in}{2.385982in}%
\pgfsys@useobject{currentmarker}{}%
\end{pgfscope}%
\begin{pgfscope}%
\pgfsys@transformshift{3.340505in}{2.194775in}%
\pgfsys@useobject{currentmarker}{}%
\end{pgfscope}%
\begin{pgfscope}%
\pgfsys@transformshift{3.592970in}{2.014706in}%
\pgfsys@useobject{currentmarker}{}%
\end{pgfscope}%
\begin{pgfscope}%
\pgfsys@transformshift{3.845435in}{1.857604in}%
\pgfsys@useobject{currentmarker}{}%
\end{pgfscope}%
\begin{pgfscope}%
\pgfsys@transformshift{4.097901in}{1.718820in}%
\pgfsys@useobject{currentmarker}{}%
\end{pgfscope}%
\begin{pgfscope}%
\pgfsys@transformshift{4.350366in}{1.613108in}%
\pgfsys@useobject{currentmarker}{}%
\end{pgfscope}%
\begin{pgfscope}%
\pgfsys@transformshift{4.602831in}{1.545347in}%
\pgfsys@useobject{currentmarker}{}%
\end{pgfscope}%
\begin{pgfscope}%
\pgfsys@transformshift{4.855296in}{1.501270in}%
\pgfsys@useobject{currentmarker}{}%
\end{pgfscope}%
\begin{pgfscope}%
\pgfsys@transformshift{5.107762in}{1.478361in}%
\pgfsys@useobject{currentmarker}{}%
\end{pgfscope}%
\begin{pgfscope}%
\pgfsys@transformshift{5.360227in}{1.474320in}%
\pgfsys@useobject{currentmarker}{}%
\end{pgfscope}%
\end{pgfscope}%
\begin{pgfscope}%
\pgfpathrectangle{\pgfqpoint{0.588634in}{1.389617in}}{\pgfqpoint{4.998811in}{1.759717in}}%
\pgfusepath{clip}%
\pgfsetbuttcap%
\pgfsetroundjoin%
\definecolor{currentfill}{rgb}{0.313725,0.317647,0.309804}%
\pgfsetfillcolor{currentfill}%
\pgfsetfillopacity{0.900000}%
\pgfsetlinewidth{1.003750pt}%
\definecolor{currentstroke}{rgb}{0.313725,0.317647,0.309804}%
\pgfsetstrokecolor{currentstroke}%
\pgfsetstrokeopacity{0.900000}%
\pgfsetdash{}{0pt}%
\pgfsys@defobject{currentmarker}{\pgfqpoint{-0.013889in}{-0.000000in}}{\pgfqpoint{0.013889in}{0.000000in}}{%
\pgfpathmoveto{\pgfqpoint{0.013889in}{-0.000000in}}%
\pgfpathlineto{\pgfqpoint{-0.013889in}{0.000000in}}%
\pgfusepath{stroke,fill}%
}%
\begin{pgfscope}%
\pgfsys@transformshift{0.942085in}{3.002883in}%
\pgfsys@useobject{currentmarker}{}%
\end{pgfscope}%
\begin{pgfscope}%
\pgfsys@transformshift{1.194551in}{2.930261in}%
\pgfsys@useobject{currentmarker}{}%
\end{pgfscope}%
\begin{pgfscope}%
\pgfsys@transformshift{1.447016in}{2.876310in}%
\pgfsys@useobject{currentmarker}{}%
\end{pgfscope}%
\begin{pgfscope}%
\pgfsys@transformshift{1.699481in}{2.843992in}%
\pgfsys@useobject{currentmarker}{}%
\end{pgfscope}%
\begin{pgfscope}%
\pgfsys@transformshift{1.951946in}{2.821497in}%
\pgfsys@useobject{currentmarker}{}%
\end{pgfscope}%
\begin{pgfscope}%
\pgfsys@transformshift{2.204412in}{2.787875in}%
\pgfsys@useobject{currentmarker}{}%
\end{pgfscope}%
\begin{pgfscope}%
\pgfsys@transformshift{2.456877in}{2.704695in}%
\pgfsys@useobject{currentmarker}{}%
\end{pgfscope}%
\begin{pgfscope}%
\pgfsys@transformshift{2.709342in}{2.566070in}%
\pgfsys@useobject{currentmarker}{}%
\end{pgfscope}%
\begin{pgfscope}%
\pgfsys@transformshift{2.961807in}{2.382686in}%
\pgfsys@useobject{currentmarker}{}%
\end{pgfscope}%
\begin{pgfscope}%
\pgfsys@transformshift{3.214272in}{2.191793in}%
\pgfsys@useobject{currentmarker}{}%
\end{pgfscope}%
\begin{pgfscope}%
\pgfsys@transformshift{3.466738in}{2.010947in}%
\pgfsys@useobject{currentmarker}{}%
\end{pgfscope}%
\begin{pgfscope}%
\pgfsys@transformshift{3.719203in}{1.854270in}%
\pgfsys@useobject{currentmarker}{}%
\end{pgfscope}%
\begin{pgfscope}%
\pgfsys@transformshift{3.971668in}{1.715839in}%
\pgfsys@useobject{currentmarker}{}%
\end{pgfscope}%
\begin{pgfscope}%
\pgfsys@transformshift{4.224133in}{1.610515in}%
\pgfsys@useobject{currentmarker}{}%
\end{pgfscope}%
\begin{pgfscope}%
\pgfsys@transformshift{4.476598in}{1.542482in}%
\pgfsys@useobject{currentmarker}{}%
\end{pgfscope}%
\begin{pgfscope}%
\pgfsys@transformshift{4.729064in}{1.498565in}%
\pgfsys@useobject{currentmarker}{}%
\end{pgfscope}%
\begin{pgfscope}%
\pgfsys@transformshift{4.981529in}{1.475142in}%
\pgfsys@useobject{currentmarker}{}%
\end{pgfscope}%
\begin{pgfscope}%
\pgfsys@transformshift{5.233994in}{1.469604in}%
\pgfsys@useobject{currentmarker}{}%
\end{pgfscope}%
\end{pgfscope}%
\begin{pgfscope}%
\pgfpathrectangle{\pgfqpoint{0.588634in}{1.389617in}}{\pgfqpoint{4.998811in}{1.759717in}}%
\pgfusepath{clip}%
\pgfsetbuttcap%
\pgfsetroundjoin%
\definecolor{currentfill}{rgb}{0.313725,0.317647,0.309804}%
\pgfsetfillcolor{currentfill}%
\pgfsetfillopacity{0.900000}%
\pgfsetlinewidth{1.003750pt}%
\definecolor{currentstroke}{rgb}{0.313725,0.317647,0.309804}%
\pgfsetstrokecolor{currentstroke}%
\pgfsetstrokeopacity{0.900000}%
\pgfsetdash{}{0pt}%
\pgfsys@defobject{currentmarker}{\pgfqpoint{-0.013889in}{-0.000000in}}{\pgfqpoint{0.013889in}{0.000000in}}{%
\pgfpathmoveto{\pgfqpoint{0.013889in}{-0.000000in}}%
\pgfpathlineto{\pgfqpoint{-0.013889in}{0.000000in}}%
\pgfusepath{stroke,fill}%
}%
\begin{pgfscope}%
\pgfsys@transformshift{0.942085in}{3.069346in}%
\pgfsys@useobject{currentmarker}{}%
\end{pgfscope}%
\begin{pgfscope}%
\pgfsys@transformshift{1.194551in}{2.965490in}%
\pgfsys@useobject{currentmarker}{}%
\end{pgfscope}%
\begin{pgfscope}%
\pgfsys@transformshift{1.447016in}{2.895422in}%
\pgfsys@useobject{currentmarker}{}%
\end{pgfscope}%
\begin{pgfscope}%
\pgfsys@transformshift{1.699481in}{2.857742in}%
\pgfsys@useobject{currentmarker}{}%
\end{pgfscope}%
\begin{pgfscope}%
\pgfsys@transformshift{1.951946in}{2.832210in}%
\pgfsys@useobject{currentmarker}{}%
\end{pgfscope}%
\begin{pgfscope}%
\pgfsys@transformshift{2.204412in}{2.796370in}%
\pgfsys@useobject{currentmarker}{}%
\end{pgfscope}%
\begin{pgfscope}%
\pgfsys@transformshift{2.456877in}{2.713055in}%
\pgfsys@useobject{currentmarker}{}%
\end{pgfscope}%
\begin{pgfscope}%
\pgfsys@transformshift{2.709342in}{2.573671in}%
\pgfsys@useobject{currentmarker}{}%
\end{pgfscope}%
\begin{pgfscope}%
\pgfsys@transformshift{2.961807in}{2.389853in}%
\pgfsys@useobject{currentmarker}{}%
\end{pgfscope}%
\begin{pgfscope}%
\pgfsys@transformshift{3.214272in}{2.198589in}%
\pgfsys@useobject{currentmarker}{}%
\end{pgfscope}%
\begin{pgfscope}%
\pgfsys@transformshift{3.466738in}{2.018226in}%
\pgfsys@useobject{currentmarker}{}%
\end{pgfscope}%
\begin{pgfscope}%
\pgfsys@transformshift{3.719203in}{1.860886in}%
\pgfsys@useobject{currentmarker}{}%
\end{pgfscope}%
\begin{pgfscope}%
\pgfsys@transformshift{3.971668in}{1.721851in}%
\pgfsys@useobject{currentmarker}{}%
\end{pgfscope}%
\begin{pgfscope}%
\pgfsys@transformshift{4.224133in}{1.615955in}%
\pgfsys@useobject{currentmarker}{}%
\end{pgfscope}%
\begin{pgfscope}%
\pgfsys@transformshift{4.476598in}{1.548024in}%
\pgfsys@useobject{currentmarker}{}%
\end{pgfscope}%
\begin{pgfscope}%
\pgfsys@transformshift{4.729064in}{1.504058in}%
\pgfsys@useobject{currentmarker}{}%
\end{pgfscope}%
\begin{pgfscope}%
\pgfsys@transformshift{4.981529in}{1.481234in}%
\pgfsys@useobject{currentmarker}{}%
\end{pgfscope}%
\begin{pgfscope}%
\pgfsys@transformshift{5.233994in}{1.478997in}%
\pgfsys@useobject{currentmarker}{}%
\end{pgfscope}%
\end{pgfscope}%
\begin{pgfscope}%
\pgfpathrectangle{\pgfqpoint{0.588634in}{1.389617in}}{\pgfqpoint{4.998811in}{1.759717in}}%
\pgfusepath{clip}%
\pgfsetbuttcap%
\pgfsetroundjoin%
\definecolor{currentfill}{rgb}{0.949020,0.372549,0.360784}%
\pgfsetfillcolor{currentfill}%
\pgfsetfillopacity{0.900000}%
\pgfsetlinewidth{1.003750pt}%
\definecolor{currentstroke}{rgb}{0.949020,0.372549,0.360784}%
\pgfsetstrokecolor{currentstroke}%
\pgfsetstrokeopacity{0.900000}%
\pgfsetdash{}{0pt}%
\pgfsys@defobject{currentmarker}{\pgfqpoint{0.000000in}{-0.013889in}}{\pgfqpoint{0.000000in}{0.013889in}}{%
\pgfpathmoveto{\pgfqpoint{0.000000in}{-0.013889in}}%
\pgfpathlineto{\pgfqpoint{0.000000in}{0.013889in}}%
\pgfusepath{stroke,fill}%
}%
\begin{pgfscope}%
\pgfsys@transformshift{0.815853in}{2.691087in}%
\pgfsys@useobject{currentmarker}{}%
\end{pgfscope}%
\begin{pgfscope}%
\pgfsys@transformshift{1.068318in}{2.614381in}%
\pgfsys@useobject{currentmarker}{}%
\end{pgfscope}%
\begin{pgfscope}%
\pgfsys@transformshift{1.320783in}{2.600045in}%
\pgfsys@useobject{currentmarker}{}%
\end{pgfscope}%
\begin{pgfscope}%
\pgfsys@transformshift{1.573249in}{2.581993in}%
\pgfsys@useobject{currentmarker}{}%
\end{pgfscope}%
\begin{pgfscope}%
\pgfsys@transformshift{1.825714in}{2.570113in}%
\pgfsys@useobject{currentmarker}{}%
\end{pgfscope}%
\begin{pgfscope}%
\pgfsys@transformshift{2.078179in}{2.535488in}%
\pgfsys@useobject{currentmarker}{}%
\end{pgfscope}%
\begin{pgfscope}%
\pgfsys@transformshift{2.330644in}{2.464800in}%
\pgfsys@useobject{currentmarker}{}%
\end{pgfscope}%
\begin{pgfscope}%
\pgfsys@transformshift{2.583109in}{2.363436in}%
\pgfsys@useobject{currentmarker}{}%
\end{pgfscope}%
\begin{pgfscope}%
\pgfsys@transformshift{2.835575in}{2.237061in}%
\pgfsys@useobject{currentmarker}{}%
\end{pgfscope}%
\begin{pgfscope}%
\pgfsys@transformshift{3.088040in}{2.117601in}%
\pgfsys@useobject{currentmarker}{}%
\end{pgfscope}%
\begin{pgfscope}%
\pgfsys@transformshift{3.340505in}{2.013777in}%
\pgfsys@useobject{currentmarker}{}%
\end{pgfscope}%
\begin{pgfscope}%
\pgfsys@transformshift{3.592970in}{1.927419in}%
\pgfsys@useobject{currentmarker}{}%
\end{pgfscope}%
\begin{pgfscope}%
\pgfsys@transformshift{3.845435in}{1.859305in}%
\pgfsys@useobject{currentmarker}{}%
\end{pgfscope}%
\begin{pgfscope}%
\pgfsys@transformshift{4.097901in}{1.806542in}%
\pgfsys@useobject{currentmarker}{}%
\end{pgfscope}%
\begin{pgfscope}%
\pgfsys@transformshift{4.350366in}{1.774888in}%
\pgfsys@useobject{currentmarker}{}%
\end{pgfscope}%
\begin{pgfscope}%
\pgfsys@transformshift{4.602831in}{1.764581in}%
\pgfsys@useobject{currentmarker}{}%
\end{pgfscope}%
\begin{pgfscope}%
\pgfsys@transformshift{4.855296in}{1.779623in}%
\pgfsys@useobject{currentmarker}{}%
\end{pgfscope}%
\begin{pgfscope}%
\pgfsys@transformshift{5.107762in}{1.827544in}%
\pgfsys@useobject{currentmarker}{}%
\end{pgfscope}%
\end{pgfscope}%
\begin{pgfscope}%
\pgfpathrectangle{\pgfqpoint{0.588634in}{1.389617in}}{\pgfqpoint{4.998811in}{1.759717in}}%
\pgfusepath{clip}%
\pgfsetbuttcap%
\pgfsetroundjoin%
\definecolor{currentfill}{rgb}{0.949020,0.372549,0.360784}%
\pgfsetfillcolor{currentfill}%
\pgfsetfillopacity{0.900000}%
\pgfsetlinewidth{1.003750pt}%
\definecolor{currentstroke}{rgb}{0.949020,0.372549,0.360784}%
\pgfsetstrokecolor{currentstroke}%
\pgfsetstrokeopacity{0.900000}%
\pgfsetdash{}{0pt}%
\pgfsys@defobject{currentmarker}{\pgfqpoint{0.000000in}{-0.013889in}}{\pgfqpoint{0.000000in}{0.013889in}}{%
\pgfpathmoveto{\pgfqpoint{0.000000in}{-0.013889in}}%
\pgfpathlineto{\pgfqpoint{0.000000in}{0.013889in}}%
\pgfusepath{stroke,fill}%
}%
\begin{pgfscope}%
\pgfsys@transformshift{1.068318in}{2.691087in}%
\pgfsys@useobject{currentmarker}{}%
\end{pgfscope}%
\begin{pgfscope}%
\pgfsys@transformshift{1.320783in}{2.614381in}%
\pgfsys@useobject{currentmarker}{}%
\end{pgfscope}%
\begin{pgfscope}%
\pgfsys@transformshift{1.573249in}{2.600045in}%
\pgfsys@useobject{currentmarker}{}%
\end{pgfscope}%
\begin{pgfscope}%
\pgfsys@transformshift{1.825714in}{2.581993in}%
\pgfsys@useobject{currentmarker}{}%
\end{pgfscope}%
\begin{pgfscope}%
\pgfsys@transformshift{2.078179in}{2.570113in}%
\pgfsys@useobject{currentmarker}{}%
\end{pgfscope}%
\begin{pgfscope}%
\pgfsys@transformshift{2.330644in}{2.535488in}%
\pgfsys@useobject{currentmarker}{}%
\end{pgfscope}%
\begin{pgfscope}%
\pgfsys@transformshift{2.583109in}{2.464800in}%
\pgfsys@useobject{currentmarker}{}%
\end{pgfscope}%
\begin{pgfscope}%
\pgfsys@transformshift{2.835575in}{2.363436in}%
\pgfsys@useobject{currentmarker}{}%
\end{pgfscope}%
\begin{pgfscope}%
\pgfsys@transformshift{3.088040in}{2.237061in}%
\pgfsys@useobject{currentmarker}{}%
\end{pgfscope}%
\begin{pgfscope}%
\pgfsys@transformshift{3.340505in}{2.117601in}%
\pgfsys@useobject{currentmarker}{}%
\end{pgfscope}%
\begin{pgfscope}%
\pgfsys@transformshift{3.592970in}{2.013777in}%
\pgfsys@useobject{currentmarker}{}%
\end{pgfscope}%
\begin{pgfscope}%
\pgfsys@transformshift{3.845435in}{1.927419in}%
\pgfsys@useobject{currentmarker}{}%
\end{pgfscope}%
\begin{pgfscope}%
\pgfsys@transformshift{4.097901in}{1.859305in}%
\pgfsys@useobject{currentmarker}{}%
\end{pgfscope}%
\begin{pgfscope}%
\pgfsys@transformshift{4.350366in}{1.806542in}%
\pgfsys@useobject{currentmarker}{}%
\end{pgfscope}%
\begin{pgfscope}%
\pgfsys@transformshift{4.602831in}{1.774888in}%
\pgfsys@useobject{currentmarker}{}%
\end{pgfscope}%
\begin{pgfscope}%
\pgfsys@transformshift{4.855296in}{1.764581in}%
\pgfsys@useobject{currentmarker}{}%
\end{pgfscope}%
\begin{pgfscope}%
\pgfsys@transformshift{5.107762in}{1.779623in}%
\pgfsys@useobject{currentmarker}{}%
\end{pgfscope}%
\begin{pgfscope}%
\pgfsys@transformshift{5.360227in}{1.827544in}%
\pgfsys@useobject{currentmarker}{}%
\end{pgfscope}%
\end{pgfscope}%
\begin{pgfscope}%
\pgfpathrectangle{\pgfqpoint{0.588634in}{1.389617in}}{\pgfqpoint{4.998811in}{1.759717in}}%
\pgfusepath{clip}%
\pgfsetbuttcap%
\pgfsetroundjoin%
\definecolor{currentfill}{rgb}{0.949020,0.372549,0.360784}%
\pgfsetfillcolor{currentfill}%
\pgfsetfillopacity{0.900000}%
\pgfsetlinewidth{1.003750pt}%
\definecolor{currentstroke}{rgb}{0.949020,0.372549,0.360784}%
\pgfsetstrokecolor{currentstroke}%
\pgfsetstrokeopacity{0.900000}%
\pgfsetdash{}{0pt}%
\pgfsys@defobject{currentmarker}{\pgfqpoint{-0.013889in}{-0.000000in}}{\pgfqpoint{0.013889in}{0.000000in}}{%
\pgfpathmoveto{\pgfqpoint{0.013889in}{-0.000000in}}%
\pgfpathlineto{\pgfqpoint{-0.013889in}{0.000000in}}%
\pgfusepath{stroke,fill}%
}%
\begin{pgfscope}%
\pgfsys@transformshift{0.942085in}{2.664218in}%
\pgfsys@useobject{currentmarker}{}%
\end{pgfscope}%
\begin{pgfscope}%
\pgfsys@transformshift{1.194551in}{2.599780in}%
\pgfsys@useobject{currentmarker}{}%
\end{pgfscope}%
\begin{pgfscope}%
\pgfsys@transformshift{1.447016in}{2.592267in}%
\pgfsys@useobject{currentmarker}{}%
\end{pgfscope}%
\begin{pgfscope}%
\pgfsys@transformshift{1.699481in}{2.576719in}%
\pgfsys@useobject{currentmarker}{}%
\end{pgfscope}%
\begin{pgfscope}%
\pgfsys@transformshift{1.951946in}{2.565662in}%
\pgfsys@useobject{currentmarker}{}%
\end{pgfscope}%
\begin{pgfscope}%
\pgfsys@transformshift{2.204412in}{2.532346in}%
\pgfsys@useobject{currentmarker}{}%
\end{pgfscope}%
\begin{pgfscope}%
\pgfsys@transformshift{2.456877in}{2.461478in}%
\pgfsys@useobject{currentmarker}{}%
\end{pgfscope}%
\begin{pgfscope}%
\pgfsys@transformshift{2.709342in}{2.360531in}%
\pgfsys@useobject{currentmarker}{}%
\end{pgfscope}%
\begin{pgfscope}%
\pgfsys@transformshift{2.961807in}{2.234249in}%
\pgfsys@useobject{currentmarker}{}%
\end{pgfscope}%
\begin{pgfscope}%
\pgfsys@transformshift{3.214272in}{2.114966in}%
\pgfsys@useobject{currentmarker}{}%
\end{pgfscope}%
\begin{pgfscope}%
\pgfsys@transformshift{3.466738in}{2.011185in}%
\pgfsys@useobject{currentmarker}{}%
\end{pgfscope}%
\begin{pgfscope}%
\pgfsys@transformshift{3.719203in}{1.924525in}%
\pgfsys@useobject{currentmarker}{}%
\end{pgfscope}%
\begin{pgfscope}%
\pgfsys@transformshift{3.971668in}{1.856427in}%
\pgfsys@useobject{currentmarker}{}%
\end{pgfscope}%
\begin{pgfscope}%
\pgfsys@transformshift{4.224133in}{1.803206in}%
\pgfsys@useobject{currentmarker}{}%
\end{pgfscope}%
\begin{pgfscope}%
\pgfsys@transformshift{4.476598in}{1.770883in}%
\pgfsys@useobject{currentmarker}{}%
\end{pgfscope}%
\begin{pgfscope}%
\pgfsys@transformshift{4.729064in}{1.759715in}%
\pgfsys@useobject{currentmarker}{}%
\end{pgfscope}%
\begin{pgfscope}%
\pgfsys@transformshift{4.981529in}{1.772613in}%
\pgfsys@useobject{currentmarker}{}%
\end{pgfscope}%
\begin{pgfscope}%
\pgfsys@transformshift{5.233994in}{1.816051in}%
\pgfsys@useobject{currentmarker}{}%
\end{pgfscope}%
\end{pgfscope}%
\begin{pgfscope}%
\pgfpathrectangle{\pgfqpoint{0.588634in}{1.389617in}}{\pgfqpoint{4.998811in}{1.759717in}}%
\pgfusepath{clip}%
\pgfsetbuttcap%
\pgfsetroundjoin%
\definecolor{currentfill}{rgb}{0.949020,0.372549,0.360784}%
\pgfsetfillcolor{currentfill}%
\pgfsetfillopacity{0.900000}%
\pgfsetlinewidth{1.003750pt}%
\definecolor{currentstroke}{rgb}{0.949020,0.372549,0.360784}%
\pgfsetstrokecolor{currentstroke}%
\pgfsetstrokeopacity{0.900000}%
\pgfsetdash{}{0pt}%
\pgfsys@defobject{currentmarker}{\pgfqpoint{-0.013889in}{-0.000000in}}{\pgfqpoint{0.013889in}{0.000000in}}{%
\pgfpathmoveto{\pgfqpoint{0.013889in}{-0.000000in}}%
\pgfpathlineto{\pgfqpoint{-0.013889in}{0.000000in}}%
\pgfusepath{stroke,fill}%
}%
\begin{pgfscope}%
\pgfsys@transformshift{0.942085in}{2.720548in}%
\pgfsys@useobject{currentmarker}{}%
\end{pgfscope}%
\begin{pgfscope}%
\pgfsys@transformshift{1.194551in}{2.627940in}%
\pgfsys@useobject{currentmarker}{}%
\end{pgfscope}%
\begin{pgfscope}%
\pgfsys@transformshift{1.447016in}{2.608231in}%
\pgfsys@useobject{currentmarker}{}%
\end{pgfscope}%
\begin{pgfscope}%
\pgfsys@transformshift{1.699481in}{2.586945in}%
\pgfsys@useobject{currentmarker}{}%
\end{pgfscope}%
\begin{pgfscope}%
\pgfsys@transformshift{1.951946in}{2.574195in}%
\pgfsys@useobject{currentmarker}{}%
\end{pgfscope}%
\begin{pgfscope}%
\pgfsys@transformshift{2.204412in}{2.539470in}%
\pgfsys@useobject{currentmarker}{}%
\end{pgfscope}%
\begin{pgfscope}%
\pgfsys@transformshift{2.456877in}{2.468020in}%
\pgfsys@useobject{currentmarker}{}%
\end{pgfscope}%
\begin{pgfscope}%
\pgfsys@transformshift{2.709342in}{2.366405in}%
\pgfsys@useobject{currentmarker}{}%
\end{pgfscope}%
\begin{pgfscope}%
\pgfsys@transformshift{2.961807in}{2.239883in}%
\pgfsys@useobject{currentmarker}{}%
\end{pgfscope}%
\begin{pgfscope}%
\pgfsys@transformshift{3.214272in}{2.120217in}%
\pgfsys@useobject{currentmarker}{}%
\end{pgfscope}%
\begin{pgfscope}%
\pgfsys@transformshift{3.466738in}{2.016142in}%
\pgfsys@useobject{currentmarker}{}%
\end{pgfscope}%
\begin{pgfscope}%
\pgfsys@transformshift{3.719203in}{1.930439in}%
\pgfsys@useobject{currentmarker}{}%
\end{pgfscope}%
\begin{pgfscope}%
\pgfsys@transformshift{3.971668in}{1.862066in}%
\pgfsys@useobject{currentmarker}{}%
\end{pgfscope}%
\begin{pgfscope}%
\pgfsys@transformshift{4.224133in}{1.810016in}%
\pgfsys@useobject{currentmarker}{}%
\end{pgfscope}%
\begin{pgfscope}%
\pgfsys@transformshift{4.476598in}{1.778451in}%
\pgfsys@useobject{currentmarker}{}%
\end{pgfscope}%
\begin{pgfscope}%
\pgfsys@transformshift{4.729064in}{1.769286in}%
\pgfsys@useobject{currentmarker}{}%
\end{pgfscope}%
\begin{pgfscope}%
\pgfsys@transformshift{4.981529in}{1.786778in}%
\pgfsys@useobject{currentmarker}{}%
\end{pgfscope}%
\begin{pgfscope}%
\pgfsys@transformshift{5.233994in}{1.839855in}%
\pgfsys@useobject{currentmarker}{}%
\end{pgfscope}%
\end{pgfscope}%
\begin{pgfscope}%
\pgfsetrectcap%
\pgfsetmiterjoin%
\pgfsetlinewidth{0.803000pt}%
\definecolor{currentstroke}{rgb}{0.000000,0.000000,0.000000}%
\pgfsetstrokecolor{currentstroke}%
\pgfsetdash{}{0pt}%
\pgfpathmoveto{\pgfqpoint{0.588634in}{1.389617in}}%
\pgfpathlineto{\pgfqpoint{0.588634in}{3.149333in}}%
\pgfusepath{stroke}%
\end{pgfscope}%
\begin{pgfscope}%
\pgfsetrectcap%
\pgfsetmiterjoin%
\pgfsetlinewidth{0.803000pt}%
\definecolor{currentstroke}{rgb}{0.000000,0.000000,0.000000}%
\pgfsetstrokecolor{currentstroke}%
\pgfsetdash{}{0pt}%
\pgfpathmoveto{\pgfqpoint{5.587445in}{1.389617in}}%
\pgfpathlineto{\pgfqpoint{5.587445in}{3.149333in}}%
\pgfusepath{stroke}%
\end{pgfscope}%
\begin{pgfscope}%
\pgfsetrectcap%
\pgfsetmiterjoin%
\pgfsetlinewidth{0.803000pt}%
\definecolor{currentstroke}{rgb}{0.000000,0.000000,0.000000}%
\pgfsetstrokecolor{currentstroke}%
\pgfsetdash{}{0pt}%
\pgfpathmoveto{\pgfqpoint{0.588634in}{1.389617in}}%
\pgfpathlineto{\pgfqpoint{5.587445in}{1.389617in}}%
\pgfusepath{stroke}%
\end{pgfscope}%
\begin{pgfscope}%
\pgfsetrectcap%
\pgfsetmiterjoin%
\pgfsetlinewidth{0.803000pt}%
\definecolor{currentstroke}{rgb}{0.000000,0.000000,0.000000}%
\pgfsetstrokecolor{currentstroke}%
\pgfsetdash{}{0pt}%
\pgfpathmoveto{\pgfqpoint{0.588634in}{3.149333in}}%
\pgfpathlineto{\pgfqpoint{5.587445in}{3.149333in}}%
\pgfusepath{stroke}%
\end{pgfscope}%
\begin{pgfscope}%
\definecolor{textcolor}{rgb}{0.000000,0.000000,0.000000}%
\pgfsetstrokecolor{textcolor}%
\pgfsetfillcolor{textcolor}%
\pgftext[x=0.588634in,y=3.232667in,left,base]{\color{textcolor}\rmfamily\fontsize{12.000000}{14.400000}\selectfont Zenith performance}%
\end{pgfscope}%
\begin{pgfscope}%
\pgfsetbuttcap%
\pgfsetmiterjoin%
\definecolor{currentfill}{rgb}{1.000000,1.000000,1.000000}%
\pgfsetfillcolor{currentfill}%
\pgfsetfillopacity{0.800000}%
\pgfsetlinewidth{1.003750pt}%
\definecolor{currentstroke}{rgb}{0.800000,0.800000,0.800000}%
\pgfsetstrokecolor{currentstroke}%
\pgfsetstrokeopacity{0.800000}%
\pgfsetdash{}{0pt}%
\pgfpathmoveto{\pgfqpoint{4.360223in}{2.594556in}}%
\pgfpathlineto{\pgfqpoint{5.509668in}{2.594556in}}%
\pgfpathquadraticcurveto{\pgfqpoint{5.531890in}{2.594556in}}{\pgfqpoint{5.531890in}{2.616778in}}%
\pgfpathlineto{\pgfqpoint{5.531890in}{3.071556in}}%
\pgfpathquadraticcurveto{\pgfqpoint{5.531890in}{3.093778in}}{\pgfqpoint{5.509668in}{3.093778in}}%
\pgfpathlineto{\pgfqpoint{4.360223in}{3.093778in}}%
\pgfpathquadraticcurveto{\pgfqpoint{4.338001in}{3.093778in}}{\pgfqpoint{4.338001in}{3.071556in}}%
\pgfpathlineto{\pgfqpoint{4.338001in}{2.616778in}}%
\pgfpathquadraticcurveto{\pgfqpoint{4.338001in}{2.594556in}}{\pgfqpoint{4.360223in}{2.594556in}}%
\pgfpathclose%
\pgfusepath{stroke,fill}%
\end{pgfscope}%
\begin{pgfscope}%
\pgfsetbuttcap%
\pgfsetmiterjoin%
\definecolor{currentfill}{rgb}{0.501961,0.501961,0.501961}%
\pgfsetfillcolor{currentfill}%
\pgfsetfillopacity{0.200000}%
\pgfsetlinewidth{0.000000pt}%
\definecolor{currentstroke}{rgb}{0.000000,0.000000,0.000000}%
\pgfsetstrokecolor{currentstroke}%
\pgfsetstrokeopacity{0.200000}%
\pgfsetdash{}{0pt}%
\pgfpathmoveto{\pgfqpoint{4.382445in}{2.971556in}}%
\pgfpathlineto{\pgfqpoint{4.604668in}{2.971556in}}%
\pgfpathlineto{\pgfqpoint{4.604668in}{3.049333in}}%
\pgfpathlineto{\pgfqpoint{4.382445in}{3.049333in}}%
\pgfpathclose%
\pgfusepath{fill}%
\end{pgfscope}%
\begin{pgfscope}%
\definecolor{textcolor}{rgb}{0.000000,0.000000,0.000000}%
\pgfsetstrokecolor{textcolor}%
\pgfsetfillcolor{textcolor}%
\pgftext[x=4.693557in,y=2.971556in,left,base]{\color{textcolor}\rmfamily\fontsize{8.000000}{9.600000}\selectfont Training events}%
\end{pgfscope}%
\begin{pgfscope}%
\pgfsetbuttcap%
\pgfsetroundjoin%
\pgfsetlinewidth{1.505625pt}%
\definecolor{currentstroke}{rgb}{0.313725,0.317647,0.309804}%
\pgfsetstrokecolor{currentstroke}%
\pgfsetstrokeopacity{0.900000}%
\pgfsetdash{}{0pt}%
\pgfpathmoveto{\pgfqpoint{4.438001in}{2.854333in}}%
\pgfpathlineto{\pgfqpoint{4.549112in}{2.854333in}}%
\pgfusepath{stroke}%
\end{pgfscope}%
\begin{pgfscope}%
\pgfsetbuttcap%
\pgfsetroundjoin%
\pgfsetlinewidth{1.505625pt}%
\definecolor{currentstroke}{rgb}{0.313725,0.317647,0.309804}%
\pgfsetstrokecolor{currentstroke}%
\pgfsetstrokeopacity{0.900000}%
\pgfsetdash{}{0pt}%
\pgfpathmoveto{\pgfqpoint{4.493557in}{2.798778in}}%
\pgfpathlineto{\pgfqpoint{4.493557in}{2.909889in}}%
\pgfusepath{stroke}%
\end{pgfscope}%
\begin{pgfscope}%
\pgfsetbuttcap%
\pgfsetroundjoin%
\definecolor{currentfill}{rgb}{0.313725,0.317647,0.309804}%
\pgfsetfillcolor{currentfill}%
\pgfsetfillopacity{0.900000}%
\pgfsetlinewidth{1.003750pt}%
\definecolor{currentstroke}{rgb}{0.313725,0.317647,0.309804}%
\pgfsetstrokecolor{currentstroke}%
\pgfsetstrokeopacity{0.900000}%
\pgfsetdash{}{0pt}%
\pgfsys@defobject{currentmarker}{\pgfqpoint{0.000000in}{-0.013889in}}{\pgfqpoint{0.000000in}{0.013889in}}{%
\pgfpathmoveto{\pgfqpoint{0.000000in}{-0.013889in}}%
\pgfpathlineto{\pgfqpoint{0.000000in}{0.013889in}}%
\pgfusepath{stroke,fill}%
}%
\begin{pgfscope}%
\pgfsys@transformshift{4.438001in}{2.854333in}%
\pgfsys@useobject{currentmarker}{}%
\end{pgfscope}%
\end{pgfscope}%
\begin{pgfscope}%
\pgfsetbuttcap%
\pgfsetroundjoin%
\definecolor{currentfill}{rgb}{0.313725,0.317647,0.309804}%
\pgfsetfillcolor{currentfill}%
\pgfsetfillopacity{0.900000}%
\pgfsetlinewidth{1.003750pt}%
\definecolor{currentstroke}{rgb}{0.313725,0.317647,0.309804}%
\pgfsetstrokecolor{currentstroke}%
\pgfsetstrokeopacity{0.900000}%
\pgfsetdash{}{0pt}%
\pgfsys@defobject{currentmarker}{\pgfqpoint{0.000000in}{-0.013889in}}{\pgfqpoint{0.000000in}{0.013889in}}{%
\pgfpathmoveto{\pgfqpoint{0.000000in}{-0.013889in}}%
\pgfpathlineto{\pgfqpoint{0.000000in}{0.013889in}}%
\pgfusepath{stroke,fill}%
}%
\begin{pgfscope}%
\pgfsys@transformshift{4.549112in}{2.854333in}%
\pgfsys@useobject{currentmarker}{}%
\end{pgfscope}%
\end{pgfscope}%
\begin{pgfscope}%
\pgfsetbuttcap%
\pgfsetroundjoin%
\definecolor{currentfill}{rgb}{0.313725,0.317647,0.309804}%
\pgfsetfillcolor{currentfill}%
\pgfsetfillopacity{0.900000}%
\pgfsetlinewidth{1.003750pt}%
\definecolor{currentstroke}{rgb}{0.313725,0.317647,0.309804}%
\pgfsetstrokecolor{currentstroke}%
\pgfsetstrokeopacity{0.900000}%
\pgfsetdash{}{0pt}%
\pgfsys@defobject{currentmarker}{\pgfqpoint{-0.013889in}{-0.000000in}}{\pgfqpoint{0.013889in}{0.000000in}}{%
\pgfpathmoveto{\pgfqpoint{0.013889in}{-0.000000in}}%
\pgfpathlineto{\pgfqpoint{-0.013889in}{0.000000in}}%
\pgfusepath{stroke,fill}%
}%
\begin{pgfscope}%
\pgfsys@transformshift{4.493557in}{2.798778in}%
\pgfsys@useobject{currentmarker}{}%
\end{pgfscope}%
\end{pgfscope}%
\begin{pgfscope}%
\pgfsetbuttcap%
\pgfsetroundjoin%
\definecolor{currentfill}{rgb}{0.313725,0.317647,0.309804}%
\pgfsetfillcolor{currentfill}%
\pgfsetfillopacity{0.900000}%
\pgfsetlinewidth{1.003750pt}%
\definecolor{currentstroke}{rgb}{0.313725,0.317647,0.309804}%
\pgfsetstrokecolor{currentstroke}%
\pgfsetstrokeopacity{0.900000}%
\pgfsetdash{}{0pt}%
\pgfsys@defobject{currentmarker}{\pgfqpoint{-0.013889in}{-0.000000in}}{\pgfqpoint{0.013889in}{0.000000in}}{%
\pgfpathmoveto{\pgfqpoint{0.013889in}{-0.000000in}}%
\pgfpathlineto{\pgfqpoint{-0.013889in}{0.000000in}}%
\pgfusepath{stroke,fill}%
}%
\begin{pgfscope}%
\pgfsys@transformshift{4.493557in}{2.909889in}%
\pgfsys@useobject{currentmarker}{}%
\end{pgfscope}%
\end{pgfscope}%
\begin{pgfscope}%
\definecolor{textcolor}{rgb}{0.000000,0.000000,0.000000}%
\pgfsetstrokecolor{textcolor}%
\pgfsetfillcolor{textcolor}%
\pgftext[x=4.693557in,y=2.815444in,left,base]{\color{textcolor}\rmfamily\fontsize{8.000000}{9.600000}\selectfont Retro Reco}%
\end{pgfscope}%
\begin{pgfscope}%
\pgfsetbuttcap%
\pgfsetroundjoin%
\pgfsetlinewidth{1.505625pt}%
\definecolor{currentstroke}{rgb}{0.949020,0.372549,0.360784}%
\pgfsetstrokecolor{currentstroke}%
\pgfsetstrokeopacity{0.900000}%
\pgfsetdash{}{0pt}%
\pgfpathmoveto{\pgfqpoint{4.438001in}{2.699444in}}%
\pgfpathlineto{\pgfqpoint{4.549112in}{2.699444in}}%
\pgfusepath{stroke}%
\end{pgfscope}%
\begin{pgfscope}%
\pgfsetbuttcap%
\pgfsetroundjoin%
\pgfsetlinewidth{1.505625pt}%
\definecolor{currentstroke}{rgb}{0.949020,0.372549,0.360784}%
\pgfsetstrokecolor{currentstroke}%
\pgfsetstrokeopacity{0.900000}%
\pgfsetdash{}{0pt}%
\pgfpathmoveto{\pgfqpoint{4.493557in}{2.643889in}}%
\pgfpathlineto{\pgfqpoint{4.493557in}{2.755000in}}%
\pgfusepath{stroke}%
\end{pgfscope}%
\begin{pgfscope}%
\pgfsetbuttcap%
\pgfsetroundjoin%
\definecolor{currentfill}{rgb}{0.949020,0.372549,0.360784}%
\pgfsetfillcolor{currentfill}%
\pgfsetfillopacity{0.900000}%
\pgfsetlinewidth{1.003750pt}%
\definecolor{currentstroke}{rgb}{0.949020,0.372549,0.360784}%
\pgfsetstrokecolor{currentstroke}%
\pgfsetstrokeopacity{0.900000}%
\pgfsetdash{}{0pt}%
\pgfsys@defobject{currentmarker}{\pgfqpoint{0.000000in}{-0.013889in}}{\pgfqpoint{0.000000in}{0.013889in}}{%
\pgfpathmoveto{\pgfqpoint{0.000000in}{-0.013889in}}%
\pgfpathlineto{\pgfqpoint{0.000000in}{0.013889in}}%
\pgfusepath{stroke,fill}%
}%
\begin{pgfscope}%
\pgfsys@transformshift{4.438001in}{2.699444in}%
\pgfsys@useobject{currentmarker}{}%
\end{pgfscope}%
\end{pgfscope}%
\begin{pgfscope}%
\pgfsetbuttcap%
\pgfsetroundjoin%
\definecolor{currentfill}{rgb}{0.949020,0.372549,0.360784}%
\pgfsetfillcolor{currentfill}%
\pgfsetfillopacity{0.900000}%
\pgfsetlinewidth{1.003750pt}%
\definecolor{currentstroke}{rgb}{0.949020,0.372549,0.360784}%
\pgfsetstrokecolor{currentstroke}%
\pgfsetstrokeopacity{0.900000}%
\pgfsetdash{}{0pt}%
\pgfsys@defobject{currentmarker}{\pgfqpoint{0.000000in}{-0.013889in}}{\pgfqpoint{0.000000in}{0.013889in}}{%
\pgfpathmoveto{\pgfqpoint{0.000000in}{-0.013889in}}%
\pgfpathlineto{\pgfqpoint{0.000000in}{0.013889in}}%
\pgfusepath{stroke,fill}%
}%
\begin{pgfscope}%
\pgfsys@transformshift{4.549112in}{2.699444in}%
\pgfsys@useobject{currentmarker}{}%
\end{pgfscope}%
\end{pgfscope}%
\begin{pgfscope}%
\pgfsetbuttcap%
\pgfsetroundjoin%
\definecolor{currentfill}{rgb}{0.949020,0.372549,0.360784}%
\pgfsetfillcolor{currentfill}%
\pgfsetfillopacity{0.900000}%
\pgfsetlinewidth{1.003750pt}%
\definecolor{currentstroke}{rgb}{0.949020,0.372549,0.360784}%
\pgfsetstrokecolor{currentstroke}%
\pgfsetstrokeopacity{0.900000}%
\pgfsetdash{}{0pt}%
\pgfsys@defobject{currentmarker}{\pgfqpoint{-0.013889in}{-0.000000in}}{\pgfqpoint{0.013889in}{0.000000in}}{%
\pgfpathmoveto{\pgfqpoint{0.013889in}{-0.000000in}}%
\pgfpathlineto{\pgfqpoint{-0.013889in}{0.000000in}}%
\pgfusepath{stroke,fill}%
}%
\begin{pgfscope}%
\pgfsys@transformshift{4.493557in}{2.643889in}%
\pgfsys@useobject{currentmarker}{}%
\end{pgfscope}%
\end{pgfscope}%
\begin{pgfscope}%
\pgfsetbuttcap%
\pgfsetroundjoin%
\definecolor{currentfill}{rgb}{0.949020,0.372549,0.360784}%
\pgfsetfillcolor{currentfill}%
\pgfsetfillopacity{0.900000}%
\pgfsetlinewidth{1.003750pt}%
\definecolor{currentstroke}{rgb}{0.949020,0.372549,0.360784}%
\pgfsetstrokecolor{currentstroke}%
\pgfsetstrokeopacity{0.900000}%
\pgfsetdash{}{0pt}%
\pgfsys@defobject{currentmarker}{\pgfqpoint{-0.013889in}{-0.000000in}}{\pgfqpoint{0.013889in}{0.000000in}}{%
\pgfpathmoveto{\pgfqpoint{0.013889in}{-0.000000in}}%
\pgfpathlineto{\pgfqpoint{-0.013889in}{0.000000in}}%
\pgfusepath{stroke,fill}%
}%
\begin{pgfscope}%
\pgfsys@transformshift{4.493557in}{2.755000in}%
\pgfsys@useobject{currentmarker}{}%
\end{pgfscope}%
\end{pgfscope}%
\begin{pgfscope}%
\definecolor{textcolor}{rgb}{0.000000,0.000000,0.000000}%
\pgfsetstrokecolor{textcolor}%
\pgfsetfillcolor{textcolor}%
\pgftext[x=4.693557in,y=2.660556in,left,base]{\color{textcolor}\rmfamily\fontsize{8.000000}{9.600000}\selectfont CubeFlow}%
\end{pgfscope}%
\begin{pgfscope}%
\pgfsetbuttcap%
\pgfsetmiterjoin%
\definecolor{currentfill}{rgb}{1.000000,1.000000,1.000000}%
\pgfsetfillcolor{currentfill}%
\pgfsetlinewidth{0.000000pt}%
\definecolor{currentstroke}{rgb}{0.000000,0.000000,0.000000}%
\pgfsetstrokecolor{currentstroke}%
\pgfsetstrokeopacity{0.000000}%
\pgfsetdash{}{0pt}%
\pgfpathmoveto{\pgfqpoint{0.588634in}{0.553781in}}%
\pgfpathlineto{\pgfqpoint{5.587445in}{0.553781in}}%
\pgfpathlineto{\pgfqpoint{5.587445in}{1.140353in}}%
\pgfpathlineto{\pgfqpoint{0.588634in}{1.140353in}}%
\pgfpathclose%
\pgfusepath{fill}%
\end{pgfscope}%
\begin{pgfscope}%
\pgfpathrectangle{\pgfqpoint{0.588634in}{0.553781in}}{\pgfqpoint{4.998811in}{0.586572in}}%
\pgfusepath{clip}%
\pgfsetbuttcap%
\pgfsetroundjoin%
\definecolor{currentfill}{rgb}{0.313725,0.317647,0.309804}%
\pgfsetfillcolor{currentfill}%
\pgfsetlinewidth{1.003750pt}%
\definecolor{currentstroke}{rgb}{0.313725,0.317647,0.309804}%
\pgfsetstrokecolor{currentstroke}%
\pgfsetdash{}{0pt}%
\pgfsys@defobject{currentmarker}{\pgfqpoint{-0.020833in}{-0.020833in}}{\pgfqpoint{0.020833in}{0.020833in}}{%
\pgfpathmoveto{\pgfqpoint{0.000000in}{-0.020833in}}%
\pgfpathcurveto{\pgfqpoint{0.005525in}{-0.020833in}}{\pgfqpoint{0.010825in}{-0.018638in}}{\pgfqpoint{0.014731in}{-0.014731in}}%
\pgfpathcurveto{\pgfqpoint{0.018638in}{-0.010825in}}{\pgfqpoint{0.020833in}{-0.005525in}}{\pgfqpoint{0.020833in}{0.000000in}}%
\pgfpathcurveto{\pgfqpoint{0.020833in}{0.005525in}}{\pgfqpoint{0.018638in}{0.010825in}}{\pgfqpoint{0.014731in}{0.014731in}}%
\pgfpathcurveto{\pgfqpoint{0.010825in}{0.018638in}}{\pgfqpoint{0.005525in}{0.020833in}}{\pgfqpoint{0.000000in}{0.020833in}}%
\pgfpathcurveto{\pgfqpoint{-0.005525in}{0.020833in}}{\pgfqpoint{-0.010825in}{0.018638in}}{\pgfqpoint{-0.014731in}{0.014731in}}%
\pgfpathcurveto{\pgfqpoint{-0.018638in}{0.010825in}}{\pgfqpoint{-0.020833in}{0.005525in}}{\pgfqpoint{-0.020833in}{0.000000in}}%
\pgfpathcurveto{\pgfqpoint{-0.020833in}{-0.005525in}}{\pgfqpoint{-0.018638in}{-0.010825in}}{\pgfqpoint{-0.014731in}{-0.014731in}}%
\pgfpathcurveto{\pgfqpoint{-0.010825in}{-0.018638in}}{\pgfqpoint{-0.005525in}{-0.020833in}}{\pgfqpoint{0.000000in}{-0.020833in}}%
\pgfpathclose%
\pgfusepath{stroke,fill}%
}%
\begin{pgfscope}%
\pgfsys@transformshift{0.942085in}{0.904871in}%
\pgfsys@useobject{currentmarker}{}%
\end{pgfscope}%
\begin{pgfscope}%
\pgfsys@transformshift{1.194551in}{0.905929in}%
\pgfsys@useobject{currentmarker}{}%
\end{pgfscope}%
\begin{pgfscope}%
\pgfsys@transformshift{1.447016in}{0.899612in}%
\pgfsys@useobject{currentmarker}{}%
\end{pgfscope}%
\begin{pgfscope}%
\pgfsys@transformshift{1.699481in}{0.897598in}%
\pgfsys@useobject{currentmarker}{}%
\end{pgfscope}%
\begin{pgfscope}%
\pgfsys@transformshift{1.951946in}{0.896056in}%
\pgfsys@useobject{currentmarker}{}%
\end{pgfscope}%
\begin{pgfscope}%
\pgfsys@transformshift{2.204412in}{0.897155in}%
\pgfsys@useobject{currentmarker}{}%
\end{pgfscope}%
\begin{pgfscope}%
\pgfsys@transformshift{2.456877in}{0.897518in}%
\pgfsys@useobject{currentmarker}{}%
\end{pgfscope}%
\begin{pgfscope}%
\pgfsys@transformshift{2.709342in}{0.894400in}%
\pgfsys@useobject{currentmarker}{}%
\end{pgfscope}%
\begin{pgfscope}%
\pgfsys@transformshift{2.961807in}{0.886949in}%
\pgfsys@useobject{currentmarker}{}%
\end{pgfscope}%
\begin{pgfscope}%
\pgfsys@transformshift{3.214272in}{0.872106in}%
\pgfsys@useobject{currentmarker}{}%
\end{pgfscope}%
\begin{pgfscope}%
\pgfsys@transformshift{3.466738in}{0.847443in}%
\pgfsys@useobject{currentmarker}{}%
\end{pgfscope}%
\begin{pgfscope}%
\pgfsys@transformshift{3.719203in}{0.810939in}%
\pgfsys@useobject{currentmarker}{}%
\end{pgfscope}%
\begin{pgfscope}%
\pgfsys@transformshift{3.971668in}{0.750797in}%
\pgfsys@useobject{currentmarker}{}%
\end{pgfscope}%
\begin{pgfscope}%
\pgfsys@transformshift{4.224133in}{0.671032in}%
\pgfsys@useobject{currentmarker}{}%
\end{pgfscope}%
\begin{pgfscope}%
\pgfsys@transformshift{4.476598in}{0.582557in}%
\pgfsys@useobject{currentmarker}{}%
\end{pgfscope}%
\begin{pgfscope}%
\pgfsys@transformshift{4.729064in}{0.480090in}%
\pgfsys@useobject{currentmarker}{}%
\end{pgfscope}%
\begin{pgfscope}%
\pgfsys@transformshift{4.981529in}{0.375902in}%
\pgfsys@useobject{currentmarker}{}%
\end{pgfscope}%
\begin{pgfscope}%
\pgfsys@transformshift{5.233994in}{0.282469in}%
\pgfsys@useobject{currentmarker}{}%
\end{pgfscope}%
\end{pgfscope}%
\begin{pgfscope}%
\pgfsetbuttcap%
\pgfsetroundjoin%
\definecolor{currentfill}{rgb}{0.000000,0.000000,0.000000}%
\pgfsetfillcolor{currentfill}%
\pgfsetlinewidth{0.803000pt}%
\definecolor{currentstroke}{rgb}{0.000000,0.000000,0.000000}%
\pgfsetstrokecolor{currentstroke}%
\pgfsetdash{}{0pt}%
\pgfsys@defobject{currentmarker}{\pgfqpoint{0.000000in}{-0.048611in}}{\pgfqpoint{0.000000in}{0.000000in}}{%
\pgfpathmoveto{\pgfqpoint{0.000000in}{0.000000in}}%
\pgfpathlineto{\pgfqpoint{0.000000in}{-0.048611in}}%
\pgfusepath{stroke,fill}%
}%
\begin{pgfscope}%
\pgfsys@transformshift{0.815853in}{0.553781in}%
\pgfsys@useobject{currentmarker}{}%
\end{pgfscope}%
\end{pgfscope}%
\begin{pgfscope}%
\definecolor{textcolor}{rgb}{0.000000,0.000000,0.000000}%
\pgfsetstrokecolor{textcolor}%
\pgfsetfillcolor{textcolor}%
\pgftext[x=0.815853in,y=0.456558in,,top]{\color{textcolor}\rmfamily\fontsize{8.000000}{9.600000}\selectfont \(\displaystyle {0.0}\)}%
\end{pgfscope}%
\begin{pgfscope}%
\pgfsetbuttcap%
\pgfsetroundjoin%
\definecolor{currentfill}{rgb}{0.000000,0.000000,0.000000}%
\pgfsetfillcolor{currentfill}%
\pgfsetlinewidth{0.803000pt}%
\definecolor{currentstroke}{rgb}{0.000000,0.000000,0.000000}%
\pgfsetstrokecolor{currentstroke}%
\pgfsetdash{}{0pt}%
\pgfsys@defobject{currentmarker}{\pgfqpoint{0.000000in}{-0.048611in}}{\pgfqpoint{0.000000in}{0.000000in}}{%
\pgfpathmoveto{\pgfqpoint{0.000000in}{0.000000in}}%
\pgfpathlineto{\pgfqpoint{0.000000in}{-0.048611in}}%
\pgfusepath{stroke,fill}%
}%
\begin{pgfscope}%
\pgfsys@transformshift{1.573249in}{0.553781in}%
\pgfsys@useobject{currentmarker}{}%
\end{pgfscope}%
\end{pgfscope}%
\begin{pgfscope}%
\definecolor{textcolor}{rgb}{0.000000,0.000000,0.000000}%
\pgfsetstrokecolor{textcolor}%
\pgfsetfillcolor{textcolor}%
\pgftext[x=1.573249in,y=0.456558in,,top]{\color{textcolor}\rmfamily\fontsize{8.000000}{9.600000}\selectfont \(\displaystyle {0.5}\)}%
\end{pgfscope}%
\begin{pgfscope}%
\pgfsetbuttcap%
\pgfsetroundjoin%
\definecolor{currentfill}{rgb}{0.000000,0.000000,0.000000}%
\pgfsetfillcolor{currentfill}%
\pgfsetlinewidth{0.803000pt}%
\definecolor{currentstroke}{rgb}{0.000000,0.000000,0.000000}%
\pgfsetstrokecolor{currentstroke}%
\pgfsetdash{}{0pt}%
\pgfsys@defobject{currentmarker}{\pgfqpoint{0.000000in}{-0.048611in}}{\pgfqpoint{0.000000in}{0.000000in}}{%
\pgfpathmoveto{\pgfqpoint{0.000000in}{0.000000in}}%
\pgfpathlineto{\pgfqpoint{0.000000in}{-0.048611in}}%
\pgfusepath{stroke,fill}%
}%
\begin{pgfscope}%
\pgfsys@transformshift{2.330644in}{0.553781in}%
\pgfsys@useobject{currentmarker}{}%
\end{pgfscope}%
\end{pgfscope}%
\begin{pgfscope}%
\definecolor{textcolor}{rgb}{0.000000,0.000000,0.000000}%
\pgfsetstrokecolor{textcolor}%
\pgfsetfillcolor{textcolor}%
\pgftext[x=2.330644in,y=0.456558in,,top]{\color{textcolor}\rmfamily\fontsize{8.000000}{9.600000}\selectfont \(\displaystyle {1.0}\)}%
\end{pgfscope}%
\begin{pgfscope}%
\pgfsetbuttcap%
\pgfsetroundjoin%
\definecolor{currentfill}{rgb}{0.000000,0.000000,0.000000}%
\pgfsetfillcolor{currentfill}%
\pgfsetlinewidth{0.803000pt}%
\definecolor{currentstroke}{rgb}{0.000000,0.000000,0.000000}%
\pgfsetstrokecolor{currentstroke}%
\pgfsetdash{}{0pt}%
\pgfsys@defobject{currentmarker}{\pgfqpoint{0.000000in}{-0.048611in}}{\pgfqpoint{0.000000in}{0.000000in}}{%
\pgfpathmoveto{\pgfqpoint{0.000000in}{0.000000in}}%
\pgfpathlineto{\pgfqpoint{0.000000in}{-0.048611in}}%
\pgfusepath{stroke,fill}%
}%
\begin{pgfscope}%
\pgfsys@transformshift{3.088040in}{0.553781in}%
\pgfsys@useobject{currentmarker}{}%
\end{pgfscope}%
\end{pgfscope}%
\begin{pgfscope}%
\definecolor{textcolor}{rgb}{0.000000,0.000000,0.000000}%
\pgfsetstrokecolor{textcolor}%
\pgfsetfillcolor{textcolor}%
\pgftext[x=3.088040in,y=0.456558in,,top]{\color{textcolor}\rmfamily\fontsize{8.000000}{9.600000}\selectfont \(\displaystyle {1.5}\)}%
\end{pgfscope}%
\begin{pgfscope}%
\pgfsetbuttcap%
\pgfsetroundjoin%
\definecolor{currentfill}{rgb}{0.000000,0.000000,0.000000}%
\pgfsetfillcolor{currentfill}%
\pgfsetlinewidth{0.803000pt}%
\definecolor{currentstroke}{rgb}{0.000000,0.000000,0.000000}%
\pgfsetstrokecolor{currentstroke}%
\pgfsetdash{}{0pt}%
\pgfsys@defobject{currentmarker}{\pgfqpoint{0.000000in}{-0.048611in}}{\pgfqpoint{0.000000in}{0.000000in}}{%
\pgfpathmoveto{\pgfqpoint{0.000000in}{0.000000in}}%
\pgfpathlineto{\pgfqpoint{0.000000in}{-0.048611in}}%
\pgfusepath{stroke,fill}%
}%
\begin{pgfscope}%
\pgfsys@transformshift{3.845435in}{0.553781in}%
\pgfsys@useobject{currentmarker}{}%
\end{pgfscope}%
\end{pgfscope}%
\begin{pgfscope}%
\definecolor{textcolor}{rgb}{0.000000,0.000000,0.000000}%
\pgfsetstrokecolor{textcolor}%
\pgfsetfillcolor{textcolor}%
\pgftext[x=3.845435in,y=0.456558in,,top]{\color{textcolor}\rmfamily\fontsize{8.000000}{9.600000}\selectfont \(\displaystyle {2.0}\)}%
\end{pgfscope}%
\begin{pgfscope}%
\pgfsetbuttcap%
\pgfsetroundjoin%
\definecolor{currentfill}{rgb}{0.000000,0.000000,0.000000}%
\pgfsetfillcolor{currentfill}%
\pgfsetlinewidth{0.803000pt}%
\definecolor{currentstroke}{rgb}{0.000000,0.000000,0.000000}%
\pgfsetstrokecolor{currentstroke}%
\pgfsetdash{}{0pt}%
\pgfsys@defobject{currentmarker}{\pgfqpoint{0.000000in}{-0.048611in}}{\pgfqpoint{0.000000in}{0.000000in}}{%
\pgfpathmoveto{\pgfqpoint{0.000000in}{0.000000in}}%
\pgfpathlineto{\pgfqpoint{0.000000in}{-0.048611in}}%
\pgfusepath{stroke,fill}%
}%
\begin{pgfscope}%
\pgfsys@transformshift{4.602831in}{0.553781in}%
\pgfsys@useobject{currentmarker}{}%
\end{pgfscope}%
\end{pgfscope}%
\begin{pgfscope}%
\definecolor{textcolor}{rgb}{0.000000,0.000000,0.000000}%
\pgfsetstrokecolor{textcolor}%
\pgfsetfillcolor{textcolor}%
\pgftext[x=4.602831in,y=0.456558in,,top]{\color{textcolor}\rmfamily\fontsize{8.000000}{9.600000}\selectfont \(\displaystyle {2.5}\)}%
\end{pgfscope}%
\begin{pgfscope}%
\pgfsetbuttcap%
\pgfsetroundjoin%
\definecolor{currentfill}{rgb}{0.000000,0.000000,0.000000}%
\pgfsetfillcolor{currentfill}%
\pgfsetlinewidth{0.803000pt}%
\definecolor{currentstroke}{rgb}{0.000000,0.000000,0.000000}%
\pgfsetstrokecolor{currentstroke}%
\pgfsetdash{}{0pt}%
\pgfsys@defobject{currentmarker}{\pgfqpoint{0.000000in}{-0.048611in}}{\pgfqpoint{0.000000in}{0.000000in}}{%
\pgfpathmoveto{\pgfqpoint{0.000000in}{0.000000in}}%
\pgfpathlineto{\pgfqpoint{0.000000in}{-0.048611in}}%
\pgfusepath{stroke,fill}%
}%
\begin{pgfscope}%
\pgfsys@transformshift{5.360227in}{0.553781in}%
\pgfsys@useobject{currentmarker}{}%
\end{pgfscope}%
\end{pgfscope}%
\begin{pgfscope}%
\definecolor{textcolor}{rgb}{0.000000,0.000000,0.000000}%
\pgfsetstrokecolor{textcolor}%
\pgfsetfillcolor{textcolor}%
\pgftext[x=5.360227in,y=0.456558in,,top]{\color{textcolor}\rmfamily\fontsize{8.000000}{9.600000}\selectfont \(\displaystyle {3.0}\)}%
\end{pgfscope}%
\begin{pgfscope}%
\definecolor{textcolor}{rgb}{0.000000,0.000000,0.000000}%
\pgfsetstrokecolor{textcolor}%
\pgfsetfillcolor{textcolor}%
\pgftext[x=3.088040in,y=0.302336in,,top]{\color{textcolor}\rmfamily\fontsize{10.950000}{13.140000}\selectfont \(\displaystyle \log_{10}(E_{\textup{true}}) \, \left[ E / \textup{GeV} \right]\)}%
\end{pgfscope}%
\begin{pgfscope}%
\pgfsetbuttcap%
\pgfsetroundjoin%
\definecolor{currentfill}{rgb}{0.000000,0.000000,0.000000}%
\pgfsetfillcolor{currentfill}%
\pgfsetlinewidth{0.803000pt}%
\definecolor{currentstroke}{rgb}{0.000000,0.000000,0.000000}%
\pgfsetstrokecolor{currentstroke}%
\pgfsetdash{}{0pt}%
\pgfsys@defobject{currentmarker}{\pgfqpoint{-0.048611in}{0.000000in}}{\pgfqpoint{-0.000000in}{0.000000in}}{%
\pgfpathmoveto{\pgfqpoint{-0.000000in}{0.000000in}}%
\pgfpathlineto{\pgfqpoint{-0.048611in}{0.000000in}}%
\pgfusepath{stroke,fill}%
}%
\begin{pgfscope}%
\pgfsys@transformshift{0.588634in}{0.553781in}%
\pgfsys@useobject{currentmarker}{}%
\end{pgfscope}%
\end{pgfscope}%
\begin{pgfscope}%
\definecolor{textcolor}{rgb}{0.000000,0.000000,0.000000}%
\pgfsetstrokecolor{textcolor}%
\pgfsetfillcolor{textcolor}%
\pgftext[x=0.340606in, y=0.515225in, left, base]{\color{textcolor}\rmfamily\fontsize{8.000000}{9.600000}\selectfont \(\displaystyle {−1}\)}%
\end{pgfscope}%
\begin{pgfscope}%
\pgfsetbuttcap%
\pgfsetroundjoin%
\definecolor{currentfill}{rgb}{0.000000,0.000000,0.000000}%
\pgfsetfillcolor{currentfill}%
\pgfsetlinewidth{0.803000pt}%
\definecolor{currentstroke}{rgb}{0.000000,0.000000,0.000000}%
\pgfsetstrokecolor{currentstroke}%
\pgfsetdash{}{0pt}%
\pgfsys@defobject{currentmarker}{\pgfqpoint{-0.048611in}{0.000000in}}{\pgfqpoint{-0.000000in}{0.000000in}}{%
\pgfpathmoveto{\pgfqpoint{-0.000000in}{0.000000in}}%
\pgfpathlineto{\pgfqpoint{-0.048611in}{0.000000in}}%
\pgfusepath{stroke,fill}%
}%
\begin{pgfscope}%
\pgfsys@transformshift{0.588634in}{0.847067in}%
\pgfsys@useobject{currentmarker}{}%
\end{pgfscope}%
\end{pgfscope}%
\begin{pgfscope}%
\definecolor{textcolor}{rgb}{0.000000,0.000000,0.000000}%
\pgfsetstrokecolor{textcolor}%
\pgfsetfillcolor{textcolor}%
\pgftext[x=0.432383in, y=0.808511in, left, base]{\color{textcolor}\rmfamily\fontsize{8.000000}{9.600000}\selectfont \(\displaystyle {0}\)}%
\end{pgfscope}%
\begin{pgfscope}%
\pgfsetbuttcap%
\pgfsetroundjoin%
\definecolor{currentfill}{rgb}{0.000000,0.000000,0.000000}%
\pgfsetfillcolor{currentfill}%
\pgfsetlinewidth{0.803000pt}%
\definecolor{currentstroke}{rgb}{0.000000,0.000000,0.000000}%
\pgfsetstrokecolor{currentstroke}%
\pgfsetdash{}{0pt}%
\pgfsys@defobject{currentmarker}{\pgfqpoint{-0.048611in}{0.000000in}}{\pgfqpoint{-0.000000in}{0.000000in}}{%
\pgfpathmoveto{\pgfqpoint{-0.000000in}{0.000000in}}%
\pgfpathlineto{\pgfqpoint{-0.048611in}{0.000000in}}%
\pgfusepath{stroke,fill}%
}%
\begin{pgfscope}%
\pgfsys@transformshift{0.588634in}{1.140353in}%
\pgfsys@useobject{currentmarker}{}%
\end{pgfscope}%
\end{pgfscope}%
\begin{pgfscope}%
\definecolor{textcolor}{rgb}{0.000000,0.000000,0.000000}%
\pgfsetstrokecolor{textcolor}%
\pgfsetfillcolor{textcolor}%
\pgftext[x=0.432383in, y=1.101797in, left, base]{\color{textcolor}\rmfamily\fontsize{8.000000}{9.600000}\selectfont \(\displaystyle {1}\)}%
\end{pgfscope}%
\begin{pgfscope}%
\definecolor{textcolor}{rgb}{0.000000,0.000000,0.000000}%
\pgfsetstrokecolor{textcolor}%
\pgfsetfillcolor{textcolor}%
\pgftext[x=0.285050in,y=0.847067in,,bottom,rotate=90.000000]{\color{textcolor}\rmfamily\fontsize{10.950000}{13.140000}\selectfont Rel. imp.}%
\end{pgfscope}%
\begin{pgfscope}%
\pgfpathrectangle{\pgfqpoint{0.588634in}{0.553781in}}{\pgfqpoint{4.998811in}{0.586572in}}%
\pgfusepath{clip}%
\pgfsetbuttcap%
\pgfsetroundjoin%
\pgfsetlinewidth{1.003750pt}%
\definecolor{currentstroke}{rgb}{0.501961,0.501961,0.501961}%
\pgfsetstrokecolor{currentstroke}%
\pgfsetdash{{3.700000pt}{1.600000pt}}{0.000000pt}%
\pgfpathmoveto{\pgfqpoint{0.588634in}{0.847067in}}%
\pgfpathlineto{\pgfqpoint{5.587445in}{0.847067in}}%
\pgfusepath{stroke}%
\end{pgfscope}%
\begin{pgfscope}%
\pgfsetrectcap%
\pgfsetmiterjoin%
\pgfsetlinewidth{0.803000pt}%
\definecolor{currentstroke}{rgb}{0.000000,0.000000,0.000000}%
\pgfsetstrokecolor{currentstroke}%
\pgfsetdash{}{0pt}%
\pgfpathmoveto{\pgfqpoint{0.588634in}{0.553781in}}%
\pgfpathlineto{\pgfqpoint{0.588634in}{1.140353in}}%
\pgfusepath{stroke}%
\end{pgfscope}%
\begin{pgfscope}%
\pgfsetrectcap%
\pgfsetmiterjoin%
\pgfsetlinewidth{0.803000pt}%
\definecolor{currentstroke}{rgb}{0.000000,0.000000,0.000000}%
\pgfsetstrokecolor{currentstroke}%
\pgfsetdash{}{0pt}%
\pgfpathmoveto{\pgfqpoint{5.587445in}{0.553781in}}%
\pgfpathlineto{\pgfqpoint{5.587445in}{1.140353in}}%
\pgfusepath{stroke}%
\end{pgfscope}%
\begin{pgfscope}%
\pgfsetrectcap%
\pgfsetmiterjoin%
\pgfsetlinewidth{0.803000pt}%
\definecolor{currentstroke}{rgb}{0.000000,0.000000,0.000000}%
\pgfsetstrokecolor{currentstroke}%
\pgfsetdash{}{0pt}%
\pgfpathmoveto{\pgfqpoint{0.588634in}{0.553781in}}%
\pgfpathlineto{\pgfqpoint{5.587445in}{0.553781in}}%
\pgfusepath{stroke}%
\end{pgfscope}%
\begin{pgfscope}%
\pgfsetrectcap%
\pgfsetmiterjoin%
\pgfsetlinewidth{0.803000pt}%
\definecolor{currentstroke}{rgb}{0.000000,0.000000,0.000000}%
\pgfsetstrokecolor{currentstroke}%
\pgfsetdash{}{0pt}%
\pgfpathmoveto{\pgfqpoint{0.588634in}{1.140353in}}%
\pgfpathlineto{\pgfqpoint{5.587445in}{1.140353in}}%
\pgfusepath{stroke}%
\end{pgfscope}%
\begin{pgfscope}%
\pgfpathrectangle{\pgfqpoint{0.588634in}{1.389617in}}{\pgfqpoint{4.998811in}{1.759717in}}%
\pgfusepath{clip}%
\pgfsetbuttcap%
\pgfsetmiterjoin%
\definecolor{currentfill}{rgb}{0.501961,0.501961,0.501961}%
\pgfsetfillcolor{currentfill}%
\pgfsetfillopacity{0.200000}%
\pgfsetlinewidth{0.000000pt}%
\definecolor{currentstroke}{rgb}{0.000000,0.000000,0.000000}%
\pgfsetstrokecolor{currentstroke}%
\pgfsetstrokeopacity{0.200000}%
\pgfsetdash{}{0pt}%
\pgfpathmoveto{\pgfqpoint{0.815853in}{1.389617in}}%
\pgfpathlineto{\pgfqpoint{1.068318in}{1.389617in}}%
\pgfpathlineto{\pgfqpoint{1.068318in}{2.537817in}}%
\pgfpathlineto{\pgfqpoint{0.815853in}{2.537817in}}%
\pgfpathclose%
\pgfusepath{fill}%
\end{pgfscope}%
\begin{pgfscope}%
\pgfpathrectangle{\pgfqpoint{0.588634in}{1.389617in}}{\pgfqpoint{4.998811in}{1.759717in}}%
\pgfusepath{clip}%
\pgfsetbuttcap%
\pgfsetmiterjoin%
\definecolor{currentfill}{rgb}{0.501961,0.501961,0.501961}%
\pgfsetfillcolor{currentfill}%
\pgfsetfillopacity{0.200000}%
\pgfsetlinewidth{0.000000pt}%
\definecolor{currentstroke}{rgb}{0.000000,0.000000,0.000000}%
\pgfsetstrokecolor{currentstroke}%
\pgfsetstrokeopacity{0.200000}%
\pgfsetdash{}{0pt}%
\pgfpathmoveto{\pgfqpoint{1.068318in}{1.389617in}}%
\pgfpathlineto{\pgfqpoint{1.320782in}{1.389617in}}%
\pgfpathlineto{\pgfqpoint{1.320782in}{2.704582in}}%
\pgfpathlineto{\pgfqpoint{1.068318in}{2.704582in}}%
\pgfpathclose%
\pgfusepath{fill}%
\end{pgfscope}%
\begin{pgfscope}%
\pgfpathrectangle{\pgfqpoint{0.588634in}{1.389617in}}{\pgfqpoint{4.998811in}{1.759717in}}%
\pgfusepath{clip}%
\pgfsetbuttcap%
\pgfsetmiterjoin%
\definecolor{currentfill}{rgb}{0.501961,0.501961,0.501961}%
\pgfsetfillcolor{currentfill}%
\pgfsetfillopacity{0.200000}%
\pgfsetlinewidth{0.000000pt}%
\definecolor{currentstroke}{rgb}{0.000000,0.000000,0.000000}%
\pgfsetstrokecolor{currentstroke}%
\pgfsetstrokeopacity{0.200000}%
\pgfsetdash{}{0pt}%
\pgfpathmoveto{\pgfqpoint{1.320782in}{1.389617in}}%
\pgfpathlineto{\pgfqpoint{1.573247in}{1.389617in}}%
\pgfpathlineto{\pgfqpoint{1.573247in}{2.834445in}}%
\pgfpathlineto{\pgfqpoint{1.320782in}{2.834445in}}%
\pgfpathclose%
\pgfusepath{fill}%
\end{pgfscope}%
\begin{pgfscope}%
\pgfpathrectangle{\pgfqpoint{0.588634in}{1.389617in}}{\pgfqpoint{4.998811in}{1.759717in}}%
\pgfusepath{clip}%
\pgfsetbuttcap%
\pgfsetmiterjoin%
\definecolor{currentfill}{rgb}{0.501961,0.501961,0.501961}%
\pgfsetfillcolor{currentfill}%
\pgfsetfillopacity{0.200000}%
\pgfsetlinewidth{0.000000pt}%
\definecolor{currentstroke}{rgb}{0.000000,0.000000,0.000000}%
\pgfsetstrokecolor{currentstroke}%
\pgfsetstrokeopacity{0.200000}%
\pgfsetdash{}{0pt}%
\pgfpathmoveto{\pgfqpoint{1.573247in}{1.389617in}}%
\pgfpathlineto{\pgfqpoint{1.825711in}{1.389617in}}%
\pgfpathlineto{\pgfqpoint{1.825711in}{2.929819in}}%
\pgfpathlineto{\pgfqpoint{1.573247in}{2.929819in}}%
\pgfpathclose%
\pgfusepath{fill}%
\end{pgfscope}%
\begin{pgfscope}%
\pgfpathrectangle{\pgfqpoint{0.588634in}{1.389617in}}{\pgfqpoint{4.998811in}{1.759717in}}%
\pgfusepath{clip}%
\pgfsetbuttcap%
\pgfsetmiterjoin%
\definecolor{currentfill}{rgb}{0.501961,0.501961,0.501961}%
\pgfsetfillcolor{currentfill}%
\pgfsetfillopacity{0.200000}%
\pgfsetlinewidth{0.000000pt}%
\definecolor{currentstroke}{rgb}{0.000000,0.000000,0.000000}%
\pgfsetstrokecolor{currentstroke}%
\pgfsetstrokeopacity{0.200000}%
\pgfsetdash{}{0pt}%
\pgfpathmoveto{\pgfqpoint{1.825711in}{1.389617in}}%
\pgfpathlineto{\pgfqpoint{2.078175in}{1.389617in}}%
\pgfpathlineto{\pgfqpoint{2.078175in}{2.993403in}}%
\pgfpathlineto{\pgfqpoint{1.825711in}{2.993403in}}%
\pgfpathclose%
\pgfusepath{fill}%
\end{pgfscope}%
\begin{pgfscope}%
\pgfpathrectangle{\pgfqpoint{0.588634in}{1.389617in}}{\pgfqpoint{4.998811in}{1.759717in}}%
\pgfusepath{clip}%
\pgfsetbuttcap%
\pgfsetmiterjoin%
\definecolor{currentfill}{rgb}{0.501961,0.501961,0.501961}%
\pgfsetfillcolor{currentfill}%
\pgfsetfillopacity{0.200000}%
\pgfsetlinewidth{0.000000pt}%
\definecolor{currentstroke}{rgb}{0.000000,0.000000,0.000000}%
\pgfsetstrokecolor{currentstroke}%
\pgfsetstrokeopacity{0.200000}%
\pgfsetdash{}{0pt}%
\pgfpathmoveto{\pgfqpoint{2.078175in}{1.389617in}}%
\pgfpathlineto{\pgfqpoint{2.330640in}{1.389617in}}%
\pgfpathlineto{\pgfqpoint{2.330640in}{3.035875in}}%
\pgfpathlineto{\pgfqpoint{2.078175in}{3.035875in}}%
\pgfpathclose%
\pgfusepath{fill}%
\end{pgfscope}%
\begin{pgfscope}%
\pgfpathrectangle{\pgfqpoint{0.588634in}{1.389617in}}{\pgfqpoint{4.998811in}{1.759717in}}%
\pgfusepath{clip}%
\pgfsetbuttcap%
\pgfsetmiterjoin%
\definecolor{currentfill}{rgb}{0.501961,0.501961,0.501961}%
\pgfsetfillcolor{currentfill}%
\pgfsetfillopacity{0.200000}%
\pgfsetlinewidth{0.000000pt}%
\definecolor{currentstroke}{rgb}{0.000000,0.000000,0.000000}%
\pgfsetstrokecolor{currentstroke}%
\pgfsetstrokeopacity{0.200000}%
\pgfsetdash{}{0pt}%
\pgfpathmoveto{\pgfqpoint{2.330640in}{1.389617in}}%
\pgfpathlineto{\pgfqpoint{2.583104in}{1.389617in}}%
\pgfpathlineto{\pgfqpoint{2.583104in}{3.059103in}}%
\pgfpathlineto{\pgfqpoint{2.330640in}{3.059103in}}%
\pgfpathclose%
\pgfusepath{fill}%
\end{pgfscope}%
\begin{pgfscope}%
\pgfpathrectangle{\pgfqpoint{0.588634in}{1.389617in}}{\pgfqpoint{4.998811in}{1.759717in}}%
\pgfusepath{clip}%
\pgfsetbuttcap%
\pgfsetmiterjoin%
\definecolor{currentfill}{rgb}{0.501961,0.501961,0.501961}%
\pgfsetfillcolor{currentfill}%
\pgfsetfillopacity{0.200000}%
\pgfsetlinewidth{0.000000pt}%
\definecolor{currentstroke}{rgb}{0.000000,0.000000,0.000000}%
\pgfsetstrokecolor{currentstroke}%
\pgfsetstrokeopacity{0.200000}%
\pgfsetdash{}{0pt}%
\pgfpathmoveto{\pgfqpoint{2.583104in}{1.389617in}}%
\pgfpathlineto{\pgfqpoint{2.835569in}{1.389617in}}%
\pgfpathlineto{\pgfqpoint{2.835569in}{3.065537in}}%
\pgfpathlineto{\pgfqpoint{2.583104in}{3.065537in}}%
\pgfpathclose%
\pgfusepath{fill}%
\end{pgfscope}%
\begin{pgfscope}%
\pgfpathrectangle{\pgfqpoint{0.588634in}{1.389617in}}{\pgfqpoint{4.998811in}{1.759717in}}%
\pgfusepath{clip}%
\pgfsetbuttcap%
\pgfsetmiterjoin%
\definecolor{currentfill}{rgb}{0.501961,0.501961,0.501961}%
\pgfsetfillcolor{currentfill}%
\pgfsetfillopacity{0.200000}%
\pgfsetlinewidth{0.000000pt}%
\definecolor{currentstroke}{rgb}{0.000000,0.000000,0.000000}%
\pgfsetstrokecolor{currentstroke}%
\pgfsetstrokeopacity{0.200000}%
\pgfsetdash{}{0pt}%
\pgfpathmoveto{\pgfqpoint{2.835569in}{1.389617in}}%
\pgfpathlineto{\pgfqpoint{3.088033in}{1.389617in}}%
\pgfpathlineto{\pgfqpoint{3.088033in}{3.057480in}}%
\pgfpathlineto{\pgfqpoint{2.835569in}{3.057480in}}%
\pgfpathclose%
\pgfusepath{fill}%
\end{pgfscope}%
\begin{pgfscope}%
\pgfpathrectangle{\pgfqpoint{0.588634in}{1.389617in}}{\pgfqpoint{4.998811in}{1.759717in}}%
\pgfusepath{clip}%
\pgfsetbuttcap%
\pgfsetmiterjoin%
\definecolor{currentfill}{rgb}{0.501961,0.501961,0.501961}%
\pgfsetfillcolor{currentfill}%
\pgfsetfillopacity{0.200000}%
\pgfsetlinewidth{0.000000pt}%
\definecolor{currentstroke}{rgb}{0.000000,0.000000,0.000000}%
\pgfsetstrokecolor{currentstroke}%
\pgfsetstrokeopacity{0.200000}%
\pgfsetdash{}{0pt}%
\pgfpathmoveto{\pgfqpoint{3.088033in}{1.389617in}}%
\pgfpathlineto{\pgfqpoint{3.340497in}{1.389617in}}%
\pgfpathlineto{\pgfqpoint{3.340497in}{3.038785in}}%
\pgfpathlineto{\pgfqpoint{3.088033in}{3.038785in}}%
\pgfpathclose%
\pgfusepath{fill}%
\end{pgfscope}%
\begin{pgfscope}%
\pgfpathrectangle{\pgfqpoint{0.588634in}{1.389617in}}{\pgfqpoint{4.998811in}{1.759717in}}%
\pgfusepath{clip}%
\pgfsetbuttcap%
\pgfsetmiterjoin%
\definecolor{currentfill}{rgb}{0.501961,0.501961,0.501961}%
\pgfsetfillcolor{currentfill}%
\pgfsetfillopacity{0.200000}%
\pgfsetlinewidth{0.000000pt}%
\definecolor{currentstroke}{rgb}{0.000000,0.000000,0.000000}%
\pgfsetstrokecolor{currentstroke}%
\pgfsetstrokeopacity{0.200000}%
\pgfsetdash{}{0pt}%
\pgfpathmoveto{\pgfqpoint{3.340497in}{1.389617in}}%
\pgfpathlineto{\pgfqpoint{3.592962in}{1.389617in}}%
\pgfpathlineto{\pgfqpoint{3.592962in}{3.010615in}}%
\pgfpathlineto{\pgfqpoint{3.340497in}{3.010615in}}%
\pgfpathclose%
\pgfusepath{fill}%
\end{pgfscope}%
\begin{pgfscope}%
\pgfpathrectangle{\pgfqpoint{0.588634in}{1.389617in}}{\pgfqpoint{4.998811in}{1.759717in}}%
\pgfusepath{clip}%
\pgfsetbuttcap%
\pgfsetmiterjoin%
\definecolor{currentfill}{rgb}{0.501961,0.501961,0.501961}%
\pgfsetfillcolor{currentfill}%
\pgfsetfillopacity{0.200000}%
\pgfsetlinewidth{0.000000pt}%
\definecolor{currentstroke}{rgb}{0.000000,0.000000,0.000000}%
\pgfsetstrokecolor{currentstroke}%
\pgfsetstrokeopacity{0.200000}%
\pgfsetdash{}{0pt}%
\pgfpathmoveto{\pgfqpoint{3.592962in}{1.389617in}}%
\pgfpathlineto{\pgfqpoint{3.845426in}{1.389617in}}%
\pgfpathlineto{\pgfqpoint{3.845426in}{2.973571in}}%
\pgfpathlineto{\pgfqpoint{3.592962in}{2.973571in}}%
\pgfpathclose%
\pgfusepath{fill}%
\end{pgfscope}%
\begin{pgfscope}%
\pgfpathrectangle{\pgfqpoint{0.588634in}{1.389617in}}{\pgfqpoint{4.998811in}{1.759717in}}%
\pgfusepath{clip}%
\pgfsetbuttcap%
\pgfsetmiterjoin%
\definecolor{currentfill}{rgb}{0.501961,0.501961,0.501961}%
\pgfsetfillcolor{currentfill}%
\pgfsetfillopacity{0.200000}%
\pgfsetlinewidth{0.000000pt}%
\definecolor{currentstroke}{rgb}{0.000000,0.000000,0.000000}%
\pgfsetstrokecolor{currentstroke}%
\pgfsetstrokeopacity{0.200000}%
\pgfsetdash{}{0pt}%
\pgfpathmoveto{\pgfqpoint{3.845426in}{1.389617in}}%
\pgfpathlineto{\pgfqpoint{4.097890in}{1.389617in}}%
\pgfpathlineto{\pgfqpoint{4.097890in}{2.931005in}}%
\pgfpathlineto{\pgfqpoint{3.845426in}{2.931005in}}%
\pgfpathclose%
\pgfusepath{fill}%
\end{pgfscope}%
\begin{pgfscope}%
\pgfpathrectangle{\pgfqpoint{0.588634in}{1.389617in}}{\pgfqpoint{4.998811in}{1.759717in}}%
\pgfusepath{clip}%
\pgfsetbuttcap%
\pgfsetmiterjoin%
\definecolor{currentfill}{rgb}{0.501961,0.501961,0.501961}%
\pgfsetfillcolor{currentfill}%
\pgfsetfillopacity{0.200000}%
\pgfsetlinewidth{0.000000pt}%
\definecolor{currentstroke}{rgb}{0.000000,0.000000,0.000000}%
\pgfsetstrokecolor{currentstroke}%
\pgfsetstrokeopacity{0.200000}%
\pgfsetdash{}{0pt}%
\pgfpathmoveto{\pgfqpoint{4.097890in}{1.389617in}}%
\pgfpathlineto{\pgfqpoint{4.350355in}{1.389617in}}%
\pgfpathlineto{\pgfqpoint{4.350355in}{2.884289in}}%
\pgfpathlineto{\pgfqpoint{4.097890in}{2.884289in}}%
\pgfpathclose%
\pgfusepath{fill}%
\end{pgfscope}%
\begin{pgfscope}%
\pgfpathrectangle{\pgfqpoint{0.588634in}{1.389617in}}{\pgfqpoint{4.998811in}{1.759717in}}%
\pgfusepath{clip}%
\pgfsetbuttcap%
\pgfsetmiterjoin%
\definecolor{currentfill}{rgb}{0.501961,0.501961,0.501961}%
\pgfsetfillcolor{currentfill}%
\pgfsetfillopacity{0.200000}%
\pgfsetlinewidth{0.000000pt}%
\definecolor{currentstroke}{rgb}{0.000000,0.000000,0.000000}%
\pgfsetstrokecolor{currentstroke}%
\pgfsetstrokeopacity{0.200000}%
\pgfsetdash{}{0pt}%
\pgfpathmoveto{\pgfqpoint{4.350355in}{1.389617in}}%
\pgfpathlineto{\pgfqpoint{4.602819in}{1.389617in}}%
\pgfpathlineto{\pgfqpoint{4.602819in}{2.832214in}}%
\pgfpathlineto{\pgfqpoint{4.350355in}{2.832214in}}%
\pgfpathclose%
\pgfusepath{fill}%
\end{pgfscope}%
\begin{pgfscope}%
\pgfpathrectangle{\pgfqpoint{0.588634in}{1.389617in}}{\pgfqpoint{4.998811in}{1.759717in}}%
\pgfusepath{clip}%
\pgfsetbuttcap%
\pgfsetmiterjoin%
\definecolor{currentfill}{rgb}{0.501961,0.501961,0.501961}%
\pgfsetfillcolor{currentfill}%
\pgfsetfillopacity{0.200000}%
\pgfsetlinewidth{0.000000pt}%
\definecolor{currentstroke}{rgb}{0.000000,0.000000,0.000000}%
\pgfsetstrokecolor{currentstroke}%
\pgfsetstrokeopacity{0.200000}%
\pgfsetdash{}{0pt}%
\pgfpathmoveto{\pgfqpoint{4.602819in}{1.389617in}}%
\pgfpathlineto{\pgfqpoint{4.855284in}{1.389617in}}%
\pgfpathlineto{\pgfqpoint{4.855284in}{2.777309in}}%
\pgfpathlineto{\pgfqpoint{4.602819in}{2.777309in}}%
\pgfpathclose%
\pgfusepath{fill}%
\end{pgfscope}%
\begin{pgfscope}%
\pgfpathrectangle{\pgfqpoint{0.588634in}{1.389617in}}{\pgfqpoint{4.998811in}{1.759717in}}%
\pgfusepath{clip}%
\pgfsetbuttcap%
\pgfsetmiterjoin%
\definecolor{currentfill}{rgb}{0.501961,0.501961,0.501961}%
\pgfsetfillcolor{currentfill}%
\pgfsetfillopacity{0.200000}%
\pgfsetlinewidth{0.000000pt}%
\definecolor{currentstroke}{rgb}{0.000000,0.000000,0.000000}%
\pgfsetstrokecolor{currentstroke}%
\pgfsetstrokeopacity{0.200000}%
\pgfsetdash{}{0pt}%
\pgfpathmoveto{\pgfqpoint{4.855284in}{1.389617in}}%
\pgfpathlineto{\pgfqpoint{5.107748in}{1.389617in}}%
\pgfpathlineto{\pgfqpoint{5.107748in}{2.718603in}}%
\pgfpathlineto{\pgfqpoint{4.855284in}{2.718603in}}%
\pgfpathclose%
\pgfusepath{fill}%
\end{pgfscope}%
\begin{pgfscope}%
\pgfpathrectangle{\pgfqpoint{0.588634in}{1.389617in}}{\pgfqpoint{4.998811in}{1.759717in}}%
\pgfusepath{clip}%
\pgfsetbuttcap%
\pgfsetmiterjoin%
\definecolor{currentfill}{rgb}{0.501961,0.501961,0.501961}%
\pgfsetfillcolor{currentfill}%
\pgfsetfillopacity{0.200000}%
\pgfsetlinewidth{0.000000pt}%
\definecolor{currentstroke}{rgb}{0.000000,0.000000,0.000000}%
\pgfsetstrokecolor{currentstroke}%
\pgfsetstrokeopacity{0.200000}%
\pgfsetdash{}{0pt}%
\pgfpathmoveto{\pgfqpoint{5.107748in}{1.389617in}}%
\pgfpathlineto{\pgfqpoint{5.360212in}{1.389617in}}%
\pgfpathlineto{\pgfqpoint{5.360212in}{2.656762in}}%
\pgfpathlineto{\pgfqpoint{5.107748in}{2.656762in}}%
\pgfpathclose%
\pgfusepath{fill}%
\end{pgfscope}%
\begin{pgfscope}%
\pgfsetbuttcap%
\pgfsetroundjoin%
\definecolor{currentfill}{rgb}{0.000000,0.000000,0.000000}%
\pgfsetfillcolor{currentfill}%
\pgfsetlinewidth{0.803000pt}%
\definecolor{currentstroke}{rgb}{0.000000,0.000000,0.000000}%
\pgfsetstrokecolor{currentstroke}%
\pgfsetdash{}{0pt}%
\pgfsys@defobject{currentmarker}{\pgfqpoint{0.000000in}{0.000000in}}{\pgfqpoint{0.048611in}{0.000000in}}{%
\pgfpathmoveto{\pgfqpoint{0.000000in}{0.000000in}}%
\pgfpathlineto{\pgfqpoint{0.048611in}{0.000000in}}%
\pgfusepath{stroke,fill}%
}%
\begin{pgfscope}%
\pgfsys@transformshift{5.587445in}{1.389617in}%
\pgfsys@useobject{currentmarker}{}%
\end{pgfscope}%
\end{pgfscope}%
\begin{pgfscope}%
\definecolor{textcolor}{rgb}{0.000000,0.000000,0.000000}%
\pgfsetstrokecolor{textcolor}%
\pgfsetfillcolor{textcolor}%
\pgftext[x=5.684668in, y=1.350464in, left, base]{\color{textcolor}\rmfamily\fontsize{8.000000}{9.600000}\selectfont \(\displaystyle {10^{0}}\)}%
\end{pgfscope}%
\begin{pgfscope}%
\pgfsetbuttcap%
\pgfsetroundjoin%
\definecolor{currentfill}{rgb}{0.000000,0.000000,0.000000}%
\pgfsetfillcolor{currentfill}%
\pgfsetlinewidth{0.803000pt}%
\definecolor{currentstroke}{rgb}{0.000000,0.000000,0.000000}%
\pgfsetstrokecolor{currentstroke}%
\pgfsetdash{}{0pt}%
\pgfsys@defobject{currentmarker}{\pgfqpoint{0.000000in}{0.000000in}}{\pgfqpoint{0.048611in}{0.000000in}}{%
\pgfpathmoveto{\pgfqpoint{0.000000in}{0.000000in}}%
\pgfpathlineto{\pgfqpoint{0.048611in}{0.000000in}}%
\pgfusepath{stroke,fill}%
}%
\begin{pgfscope}%
\pgfsys@transformshift{5.587445in}{1.964099in}%
\pgfsys@useobject{currentmarker}{}%
\end{pgfscope}%
\end{pgfscope}%
\begin{pgfscope}%
\definecolor{textcolor}{rgb}{0.000000,0.000000,0.000000}%
\pgfsetstrokecolor{textcolor}%
\pgfsetfillcolor{textcolor}%
\pgftext[x=5.684668in, y=1.924946in, left, base]{\color{textcolor}\rmfamily\fontsize{8.000000}{9.600000}\selectfont \(\displaystyle {10^{2}}\)}%
\end{pgfscope}%
\begin{pgfscope}%
\pgfsetbuttcap%
\pgfsetroundjoin%
\definecolor{currentfill}{rgb}{0.000000,0.000000,0.000000}%
\pgfsetfillcolor{currentfill}%
\pgfsetlinewidth{0.803000pt}%
\definecolor{currentstroke}{rgb}{0.000000,0.000000,0.000000}%
\pgfsetstrokecolor{currentstroke}%
\pgfsetdash{}{0pt}%
\pgfsys@defobject{currentmarker}{\pgfqpoint{0.000000in}{0.000000in}}{\pgfqpoint{0.048611in}{0.000000in}}{%
\pgfpathmoveto{\pgfqpoint{0.000000in}{0.000000in}}%
\pgfpathlineto{\pgfqpoint{0.048611in}{0.000000in}}%
\pgfusepath{stroke,fill}%
}%
\begin{pgfscope}%
\pgfsys@transformshift{5.587445in}{2.538581in}%
\pgfsys@useobject{currentmarker}{}%
\end{pgfscope}%
\end{pgfscope}%
\begin{pgfscope}%
\definecolor{textcolor}{rgb}{0.000000,0.000000,0.000000}%
\pgfsetstrokecolor{textcolor}%
\pgfsetfillcolor{textcolor}%
\pgftext[x=5.684668in, y=2.499428in, left, base]{\color{textcolor}\rmfamily\fontsize{8.000000}{9.600000}\selectfont \(\displaystyle {10^{4}}\)}%
\end{pgfscope}%
\begin{pgfscope}%
\pgfsetbuttcap%
\pgfsetroundjoin%
\definecolor{currentfill}{rgb}{0.000000,0.000000,0.000000}%
\pgfsetfillcolor{currentfill}%
\pgfsetlinewidth{0.803000pt}%
\definecolor{currentstroke}{rgb}{0.000000,0.000000,0.000000}%
\pgfsetstrokecolor{currentstroke}%
\pgfsetdash{}{0pt}%
\pgfsys@defobject{currentmarker}{\pgfqpoint{0.000000in}{0.000000in}}{\pgfqpoint{0.048611in}{0.000000in}}{%
\pgfpathmoveto{\pgfqpoint{0.000000in}{0.000000in}}%
\pgfpathlineto{\pgfqpoint{0.048611in}{0.000000in}}%
\pgfusepath{stroke,fill}%
}%
\begin{pgfscope}%
\pgfsys@transformshift{5.587445in}{3.113063in}%
\pgfsys@useobject{currentmarker}{}%
\end{pgfscope}%
\end{pgfscope}%
\begin{pgfscope}%
\definecolor{textcolor}{rgb}{0.000000,0.000000,0.000000}%
\pgfsetstrokecolor{textcolor}%
\pgfsetfillcolor{textcolor}%
\pgftext[x=5.684668in, y=3.073910in, left, base]{\color{textcolor}\rmfamily\fontsize{8.000000}{9.600000}\selectfont \(\displaystyle {10^{6}}\)}%
\end{pgfscope}%
\begin{pgfscope}%
\definecolor{textcolor}{rgb}{0.000000,0.000000,0.000000}%
\pgfsetstrokecolor{textcolor}%
\pgfsetfillcolor{textcolor}%
\pgftext[x=5.916150in,y=2.269475in,,top,rotate=90.000000]{\color{textcolor}\rmfamily\fontsize{10.950000}{13.140000}\selectfont Events}%
\end{pgfscope}%
\begin{pgfscope}%
\pgfsetrectcap%
\pgfsetmiterjoin%
\pgfsetlinewidth{0.803000pt}%
\definecolor{currentstroke}{rgb}{0.000000,0.000000,0.000000}%
\pgfsetstrokecolor{currentstroke}%
\pgfsetdash{}{0pt}%
\pgfpathmoveto{\pgfqpoint{0.588634in}{1.389617in}}%
\pgfpathlineto{\pgfqpoint{0.588634in}{3.149333in}}%
\pgfusepath{stroke}%
\end{pgfscope}%
\begin{pgfscope}%
\pgfsetrectcap%
\pgfsetmiterjoin%
\pgfsetlinewidth{0.803000pt}%
\definecolor{currentstroke}{rgb}{0.000000,0.000000,0.000000}%
\pgfsetstrokecolor{currentstroke}%
\pgfsetdash{}{0pt}%
\pgfpathmoveto{\pgfqpoint{5.587445in}{1.389617in}}%
\pgfpathlineto{\pgfqpoint{5.587445in}{3.149333in}}%
\pgfusepath{stroke}%
\end{pgfscope}%
\begin{pgfscope}%
\pgfsetrectcap%
\pgfsetmiterjoin%
\pgfsetlinewidth{0.803000pt}%
\definecolor{currentstroke}{rgb}{0.000000,0.000000,0.000000}%
\pgfsetstrokecolor{currentstroke}%
\pgfsetdash{}{0pt}%
\pgfpathmoveto{\pgfqpoint{0.588634in}{1.389617in}}%
\pgfpathlineto{\pgfqpoint{5.587445in}{1.389617in}}%
\pgfusepath{stroke}%
\end{pgfscope}%
\begin{pgfscope}%
\pgfsetrectcap%
\pgfsetmiterjoin%
\pgfsetlinewidth{0.803000pt}%
\definecolor{currentstroke}{rgb}{0.000000,0.000000,0.000000}%
\pgfsetstrokecolor{currentstroke}%
\pgfsetdash{}{0pt}%
\pgfpathmoveto{\pgfqpoint{0.588634in}{3.149333in}}%
\pgfpathlineto{\pgfqpoint{5.587445in}{3.149333in}}%
\pgfusepath{stroke}%
\end{pgfscope}%
\end{pgfpicture}%
\makeatother%
\endgroup%

    \caption{The resolution performance of CubeFlow (red) and Retro Reco (black) for zenith reconstruction.
    CubeFlow performs better at low energies, while Retro Reco wins at higher energies.
    The amount of training events in each energy bin is superimposed (using a log scale) with grey bars.
    The lower plot shows the improvement of CubeFlow relative to Retro Reco, constrained between \SI{-100}{\percent} and \SI{100}{\percent}.}\label{fig:zenith_comparison}
\end{figure}

\begin{figure}
    \centering
    %% Creator: Matplotlib, PGF backend
%%
%% To include the figure in your LaTeX document, write
%%   \input{<filename>.pgf}
%%
%% Make sure the required packages are loaded in your preamble
%%   \usepackage{pgf}
%%
%% and, on pdftex
%%   \usepackage[utf8]{inputenc}\DeclareUnicodeCharacter{2212}{-}
%%
%% or, on luatex and xetex
%%   \usepackage{unicode-math}
%%
%% Figures using additional raster images can only be included by \input if
%% they are in the same directory as the main LaTeX file. For loading figures
%% from other directories you can use the `import` package
%%   \usepackage{import}
%%
%% and then include the figures with
%%   \import{<path to file>}{<filename>.pgf}
%%
%% Matplotlib used the following preamble
%%   \usepackage{siunitx} \usepackage{amsmath} \usepackage{bm}
%%   \usepackage{fontspec}
%%
\begingroup%
\makeatletter%
\begin{pgfpicture}%
\pgfpathrectangle{\pgfpointorigin}{\pgfqpoint{6.201200in}{3.500000in}}%
\pgfusepath{use as bounding box, clip}%
\begin{pgfscope}%
\pgfsetbuttcap%
\pgfsetmiterjoin%
\definecolor{currentfill}{rgb}{1.000000,1.000000,1.000000}%
\pgfsetfillcolor{currentfill}%
\pgfsetlinewidth{0.000000pt}%
\definecolor{currentstroke}{rgb}{1.000000,1.000000,1.000000}%
\pgfsetstrokecolor{currentstroke}%
\pgfsetdash{}{0pt}%
\pgfpathmoveto{\pgfqpoint{0.000000in}{0.000000in}}%
\pgfpathlineto{\pgfqpoint{6.201200in}{0.000000in}}%
\pgfpathlineto{\pgfqpoint{6.201200in}{3.500000in}}%
\pgfpathlineto{\pgfqpoint{0.000000in}{3.500000in}}%
\pgfpathclose%
\pgfusepath{fill}%
\end{pgfscope}%
\begin{pgfscope}%
\pgfsetbuttcap%
\pgfsetmiterjoin%
\definecolor{currentfill}{rgb}{1.000000,1.000000,1.000000}%
\pgfsetfillcolor{currentfill}%
\pgfsetlinewidth{0.000000pt}%
\definecolor{currentstroke}{rgb}{0.000000,0.000000,0.000000}%
\pgfsetstrokecolor{currentstroke}%
\pgfsetstrokeopacity{0.000000}%
\pgfsetdash{}{0pt}%
\pgfpathmoveto{\pgfqpoint{0.935556in}{0.536494in}}%
\pgfpathlineto{\pgfqpoint{3.365908in}{0.536494in}}%
\pgfpathlineto{\pgfqpoint{3.365908in}{3.151000in}}%
\pgfpathlineto{\pgfqpoint{0.935556in}{3.151000in}}%
\pgfpathclose%
\pgfusepath{fill}%
\end{pgfscope}%
\begin{pgfscope}%
\pgfpathrectangle{\pgfqpoint{0.935556in}{0.536494in}}{\pgfqpoint{2.430352in}{2.614506in}}%
\pgfusepath{clip}%
\pgfsetbuttcap%
\pgfsetmiterjoin%
\definecolor{currentfill}{rgb}{0.313725,0.317647,0.309804}%
\pgfsetfillcolor{currentfill}%
\pgfsetlinewidth{0.000000pt}%
\definecolor{currentstroke}{rgb}{0.000000,0.000000,0.000000}%
\pgfsetstrokecolor{currentstroke}%
\pgfsetstrokeopacity{0.000000}%
\pgfsetdash{}{0pt}%
\pgfpathmoveto{\pgfqpoint{0.935556in}{0.655336in}}%
\pgfpathlineto{\pgfqpoint{1.464448in}{0.655336in}}%
\pgfpathlineto{\pgfqpoint{1.464448in}{0.983173in}}%
\pgfpathlineto{\pgfqpoint{0.935556in}{0.983173in}}%
\pgfpathclose%
\pgfusepath{fill}%
\end{pgfscope}%
\begin{pgfscope}%
\pgfpathrectangle{\pgfqpoint{0.935556in}{0.536494in}}{\pgfqpoint{2.430352in}{2.614506in}}%
\pgfusepath{clip}%
\pgfsetbuttcap%
\pgfsetmiterjoin%
\definecolor{currentfill}{rgb}{0.313725,0.317647,0.309804}%
\pgfsetfillcolor{currentfill}%
\pgfsetlinewidth{0.000000pt}%
\definecolor{currentstroke}{rgb}{0.000000,0.000000,0.000000}%
\pgfsetstrokecolor{currentstroke}%
\pgfsetstrokeopacity{0.000000}%
\pgfsetdash{}{0pt}%
\pgfpathmoveto{\pgfqpoint{0.935556in}{1.065133in}}%
\pgfpathlineto{\pgfqpoint{1.519342in}{1.065133in}}%
\pgfpathlineto{\pgfqpoint{1.519342in}{1.392970in}}%
\pgfpathlineto{\pgfqpoint{0.935556in}{1.392970in}}%
\pgfpathclose%
\pgfusepath{fill}%
\end{pgfscope}%
\begin{pgfscope}%
\pgfpathrectangle{\pgfqpoint{0.935556in}{0.536494in}}{\pgfqpoint{2.430352in}{2.614506in}}%
\pgfusepath{clip}%
\pgfsetbuttcap%
\pgfsetmiterjoin%
\definecolor{currentfill}{rgb}{0.313725,0.317647,0.309804}%
\pgfsetfillcolor{currentfill}%
\pgfsetlinewidth{0.000000pt}%
\definecolor{currentstroke}{rgb}{0.000000,0.000000,0.000000}%
\pgfsetstrokecolor{currentstroke}%
\pgfsetstrokeopacity{0.000000}%
\pgfsetdash{}{0pt}%
\pgfpathmoveto{\pgfqpoint{0.935556in}{1.474930in}}%
\pgfpathlineto{\pgfqpoint{3.250177in}{1.474930in}}%
\pgfpathlineto{\pgfqpoint{3.250177in}{1.802768in}}%
\pgfpathlineto{\pgfqpoint{0.935556in}{1.802768in}}%
\pgfpathclose%
\pgfusepath{fill}%
\end{pgfscope}%
\begin{pgfscope}%
\pgfpathrectangle{\pgfqpoint{0.935556in}{0.536494in}}{\pgfqpoint{2.430352in}{2.614506in}}%
\pgfusepath{clip}%
\pgfsetbuttcap%
\pgfsetmiterjoin%
\definecolor{currentfill}{rgb}{0.313725,0.317647,0.309804}%
\pgfsetfillcolor{currentfill}%
\pgfsetlinewidth{0.000000pt}%
\definecolor{currentstroke}{rgb}{0.000000,0.000000,0.000000}%
\pgfsetstrokecolor{currentstroke}%
\pgfsetstrokeopacity{0.000000}%
\pgfsetdash{}{0pt}%
\pgfpathmoveto{\pgfqpoint{0.935556in}{1.884727in}}%
\pgfpathlineto{\pgfqpoint{2.856007in}{1.884727in}}%
\pgfpathlineto{\pgfqpoint{2.856007in}{2.212565in}}%
\pgfpathlineto{\pgfqpoint{0.935556in}{2.212565in}}%
\pgfpathclose%
\pgfusepath{fill}%
\end{pgfscope}%
\begin{pgfscope}%
\pgfpathrectangle{\pgfqpoint{0.935556in}{0.536494in}}{\pgfqpoint{2.430352in}{2.614506in}}%
\pgfusepath{clip}%
\pgfsetbuttcap%
\pgfsetmiterjoin%
\definecolor{currentfill}{rgb}{0.313725,0.317647,0.309804}%
\pgfsetfillcolor{currentfill}%
\pgfsetlinewidth{0.000000pt}%
\definecolor{currentstroke}{rgb}{0.000000,0.000000,0.000000}%
\pgfsetstrokecolor{currentstroke}%
\pgfsetstrokeopacity{0.000000}%
\pgfsetdash{}{0pt}%
\pgfpathmoveto{\pgfqpoint{0.935556in}{2.294524in}}%
\pgfpathlineto{\pgfqpoint{1.494289in}{2.294524in}}%
\pgfpathlineto{\pgfqpoint{1.494289in}{2.622362in}}%
\pgfpathlineto{\pgfqpoint{0.935556in}{2.622362in}}%
\pgfpathclose%
\pgfusepath{fill}%
\end{pgfscope}%
\begin{pgfscope}%
\pgfpathrectangle{\pgfqpoint{0.935556in}{0.536494in}}{\pgfqpoint{2.430352in}{2.614506in}}%
\pgfusepath{clip}%
\pgfsetbuttcap%
\pgfsetmiterjoin%
\definecolor{currentfill}{rgb}{0.313725,0.317647,0.309804}%
\pgfsetfillcolor{currentfill}%
\pgfsetlinewidth{0.000000pt}%
\definecolor{currentstroke}{rgb}{0.000000,0.000000,0.000000}%
\pgfsetstrokecolor{currentstroke}%
\pgfsetstrokeopacity{0.000000}%
\pgfsetdash{}{0pt}%
\pgfpathmoveto{\pgfqpoint{0.935556in}{2.704321in}}%
\pgfpathlineto{\pgfqpoint{1.661538in}{2.704321in}}%
\pgfpathlineto{\pgfqpoint{1.661538in}{3.032159in}}%
\pgfpathlineto{\pgfqpoint{0.935556in}{3.032159in}}%
\pgfpathclose%
\pgfusepath{fill}%
\end{pgfscope}%
\begin{pgfscope}%
\pgfsetbuttcap%
\pgfsetroundjoin%
\definecolor{currentfill}{rgb}{0.000000,0.000000,0.000000}%
\pgfsetfillcolor{currentfill}%
\pgfsetlinewidth{0.803000pt}%
\definecolor{currentstroke}{rgb}{0.000000,0.000000,0.000000}%
\pgfsetstrokecolor{currentstroke}%
\pgfsetdash{}{0pt}%
\pgfsys@defobject{currentmarker}{\pgfqpoint{0.000000in}{-0.048611in}}{\pgfqpoint{0.000000in}{0.000000in}}{%
\pgfpathmoveto{\pgfqpoint{0.000000in}{0.000000in}}%
\pgfpathlineto{\pgfqpoint{0.000000in}{-0.048611in}}%
\pgfusepath{stroke,fill}%
}%
\begin{pgfscope}%
\pgfsys@transformshift{0.935556in}{0.536494in}%
\pgfsys@useobject{currentmarker}{}%
\end{pgfscope}%
\end{pgfscope}%
\begin{pgfscope}%
\definecolor{textcolor}{rgb}{0.000000,0.000000,0.000000}%
\pgfsetstrokecolor{textcolor}%
\pgfsetfillcolor{textcolor}%
\pgftext[x=0.935556in,y=0.439272in,,top]{\color{textcolor}\rmfamily\fontsize{8.000000}{9.600000}\selectfont \(\displaystyle {0}\)}%
\end{pgfscope}%
\begin{pgfscope}%
\pgfsetbuttcap%
\pgfsetroundjoin%
\definecolor{currentfill}{rgb}{0.000000,0.000000,0.000000}%
\pgfsetfillcolor{currentfill}%
\pgfsetlinewidth{0.803000pt}%
\definecolor{currentstroke}{rgb}{0.000000,0.000000,0.000000}%
\pgfsetstrokecolor{currentstroke}%
\pgfsetdash{}{0pt}%
\pgfsys@defobject{currentmarker}{\pgfqpoint{0.000000in}{-0.048611in}}{\pgfqpoint{0.000000in}{0.000000in}}{%
\pgfpathmoveto{\pgfqpoint{0.000000in}{0.000000in}}%
\pgfpathlineto{\pgfqpoint{0.000000in}{-0.048611in}}%
\pgfusepath{stroke,fill}%
}%
\begin{pgfscope}%
\pgfsys@transformshift{1.464442in}{0.536494in}%
\pgfsys@useobject{currentmarker}{}%
\end{pgfscope}%
\end{pgfscope}%
\begin{pgfscope}%
\definecolor{textcolor}{rgb}{0.000000,0.000000,0.000000}%
\pgfsetstrokecolor{textcolor}%
\pgfsetfillcolor{textcolor}%
\pgftext[x=1.464442in,y=0.439272in,,top]{\color{textcolor}\rmfamily\fontsize{8.000000}{9.600000}\selectfont \(\displaystyle {1}\)}%
\end{pgfscope}%
\begin{pgfscope}%
\pgfsetbuttcap%
\pgfsetroundjoin%
\definecolor{currentfill}{rgb}{0.000000,0.000000,0.000000}%
\pgfsetfillcolor{currentfill}%
\pgfsetlinewidth{0.803000pt}%
\definecolor{currentstroke}{rgb}{0.000000,0.000000,0.000000}%
\pgfsetstrokecolor{currentstroke}%
\pgfsetdash{}{0pt}%
\pgfsys@defobject{currentmarker}{\pgfqpoint{0.000000in}{-0.048611in}}{\pgfqpoint{0.000000in}{0.000000in}}{%
\pgfpathmoveto{\pgfqpoint{0.000000in}{0.000000in}}%
\pgfpathlineto{\pgfqpoint{0.000000in}{-0.048611in}}%
\pgfusepath{stroke,fill}%
}%
\begin{pgfscope}%
\pgfsys@transformshift{1.993328in}{0.536494in}%
\pgfsys@useobject{currentmarker}{}%
\end{pgfscope}%
\end{pgfscope}%
\begin{pgfscope}%
\definecolor{textcolor}{rgb}{0.000000,0.000000,0.000000}%
\pgfsetstrokecolor{textcolor}%
\pgfsetfillcolor{textcolor}%
\pgftext[x=1.993328in,y=0.439272in,,top]{\color{textcolor}\rmfamily\fontsize{8.000000}{9.600000}\selectfont \(\displaystyle {2}\)}%
\end{pgfscope}%
\begin{pgfscope}%
\pgfsetbuttcap%
\pgfsetroundjoin%
\definecolor{currentfill}{rgb}{0.000000,0.000000,0.000000}%
\pgfsetfillcolor{currentfill}%
\pgfsetlinewidth{0.803000pt}%
\definecolor{currentstroke}{rgb}{0.000000,0.000000,0.000000}%
\pgfsetstrokecolor{currentstroke}%
\pgfsetdash{}{0pt}%
\pgfsys@defobject{currentmarker}{\pgfqpoint{0.000000in}{-0.048611in}}{\pgfqpoint{0.000000in}{0.000000in}}{%
\pgfpathmoveto{\pgfqpoint{0.000000in}{0.000000in}}%
\pgfpathlineto{\pgfqpoint{0.000000in}{-0.048611in}}%
\pgfusepath{stroke,fill}%
}%
\begin{pgfscope}%
\pgfsys@transformshift{2.522214in}{0.536494in}%
\pgfsys@useobject{currentmarker}{}%
\end{pgfscope}%
\end{pgfscope}%
\begin{pgfscope}%
\definecolor{textcolor}{rgb}{0.000000,0.000000,0.000000}%
\pgfsetstrokecolor{textcolor}%
\pgfsetfillcolor{textcolor}%
\pgftext[x=2.522214in,y=0.439272in,,top]{\color{textcolor}\rmfamily\fontsize{8.000000}{9.600000}\selectfont \(\displaystyle {3}\)}%
\end{pgfscope}%
\begin{pgfscope}%
\pgfsetbuttcap%
\pgfsetroundjoin%
\definecolor{currentfill}{rgb}{0.000000,0.000000,0.000000}%
\pgfsetfillcolor{currentfill}%
\pgfsetlinewidth{0.803000pt}%
\definecolor{currentstroke}{rgb}{0.000000,0.000000,0.000000}%
\pgfsetstrokecolor{currentstroke}%
\pgfsetdash{}{0pt}%
\pgfsys@defobject{currentmarker}{\pgfqpoint{0.000000in}{-0.048611in}}{\pgfqpoint{0.000000in}{0.000000in}}{%
\pgfpathmoveto{\pgfqpoint{0.000000in}{0.000000in}}%
\pgfpathlineto{\pgfqpoint{0.000000in}{-0.048611in}}%
\pgfusepath{stroke,fill}%
}%
\begin{pgfscope}%
\pgfsys@transformshift{3.051100in}{0.536494in}%
\pgfsys@useobject{currentmarker}{}%
\end{pgfscope}%
\end{pgfscope}%
\begin{pgfscope}%
\definecolor{textcolor}{rgb}{0.000000,0.000000,0.000000}%
\pgfsetstrokecolor{textcolor}%
\pgfsetfillcolor{textcolor}%
\pgftext[x=3.051100in,y=0.439272in,,top]{\color{textcolor}\rmfamily\fontsize{8.000000}{9.600000}\selectfont \(\displaystyle {4}\)}%
\end{pgfscope}%
\begin{pgfscope}%
\definecolor{textcolor}{rgb}{0.000000,0.000000,0.000000}%
\pgfsetstrokecolor{textcolor}%
\pgfsetfillcolor{textcolor}%
\pgftext[x=2.150732in,y=0.285050in,,top]{\color{textcolor}\rmfamily\fontsize{10.950000}{13.140000}\selectfont Permutation importance}%
\end{pgfscope}%
\begin{pgfscope}%
\pgfsetbuttcap%
\pgfsetroundjoin%
\definecolor{currentfill}{rgb}{0.000000,0.000000,0.000000}%
\pgfsetfillcolor{currentfill}%
\pgfsetlinewidth{0.803000pt}%
\definecolor{currentstroke}{rgb}{0.000000,0.000000,0.000000}%
\pgfsetstrokecolor{currentstroke}%
\pgfsetdash{}{0pt}%
\pgfsys@defobject{currentmarker}{\pgfqpoint{-0.048611in}{0.000000in}}{\pgfqpoint{-0.000000in}{0.000000in}}{%
\pgfpathmoveto{\pgfqpoint{-0.000000in}{0.000000in}}%
\pgfpathlineto{\pgfqpoint{-0.048611in}{0.000000in}}%
\pgfusepath{stroke,fill}%
}%
\begin{pgfscope}%
\pgfsys@transformshift{0.935556in}{0.819254in}%
\pgfsys@useobject{currentmarker}{}%
\end{pgfscope}%
\end{pgfscope}%
\begin{pgfscope}%
\definecolor{textcolor}{rgb}{0.000000,0.000000,0.000000}%
\pgfsetstrokecolor{textcolor}%
\pgfsetfillcolor{textcolor}%
\pgftext[x=0.464667in, y=0.780699in, left, base]{\color{textcolor}\rmfamily\fontsize{8.000000}{9.600000}\selectfont dom\_x}%
\end{pgfscope}%
\begin{pgfscope}%
\pgfsetbuttcap%
\pgfsetroundjoin%
\definecolor{currentfill}{rgb}{0.000000,0.000000,0.000000}%
\pgfsetfillcolor{currentfill}%
\pgfsetlinewidth{0.803000pt}%
\definecolor{currentstroke}{rgb}{0.000000,0.000000,0.000000}%
\pgfsetstrokecolor{currentstroke}%
\pgfsetdash{}{0pt}%
\pgfsys@defobject{currentmarker}{\pgfqpoint{-0.048611in}{0.000000in}}{\pgfqpoint{-0.000000in}{0.000000in}}{%
\pgfpathmoveto{\pgfqpoint{-0.000000in}{0.000000in}}%
\pgfpathlineto{\pgfqpoint{-0.048611in}{0.000000in}}%
\pgfusepath{stroke,fill}%
}%
\begin{pgfscope}%
\pgfsys@transformshift{0.935556in}{1.229052in}%
\pgfsys@useobject{currentmarker}{}%
\end{pgfscope}%
\end{pgfscope}%
\begin{pgfscope}%
\definecolor{textcolor}{rgb}{0.000000,0.000000,0.000000}%
\pgfsetstrokecolor{textcolor}%
\pgfsetfillcolor{textcolor}%
\pgftext[x=0.464667in, y=1.190496in, left, base]{\color{textcolor}\rmfamily\fontsize{8.000000}{9.600000}\selectfont dom\_y}%
\end{pgfscope}%
\begin{pgfscope}%
\pgfsetbuttcap%
\pgfsetroundjoin%
\definecolor{currentfill}{rgb}{0.000000,0.000000,0.000000}%
\pgfsetfillcolor{currentfill}%
\pgfsetlinewidth{0.803000pt}%
\definecolor{currentstroke}{rgb}{0.000000,0.000000,0.000000}%
\pgfsetstrokecolor{currentstroke}%
\pgfsetdash{}{0pt}%
\pgfsys@defobject{currentmarker}{\pgfqpoint{-0.048611in}{0.000000in}}{\pgfqpoint{-0.000000in}{0.000000in}}{%
\pgfpathmoveto{\pgfqpoint{-0.000000in}{0.000000in}}%
\pgfpathlineto{\pgfqpoint{-0.048611in}{0.000000in}}%
\pgfusepath{stroke,fill}%
}%
\begin{pgfscope}%
\pgfsys@transformshift{0.935556in}{1.638849in}%
\pgfsys@useobject{currentmarker}{}%
\end{pgfscope}%
\end{pgfscope}%
\begin{pgfscope}%
\definecolor{textcolor}{rgb}{0.000000,0.000000,0.000000}%
\pgfsetstrokecolor{textcolor}%
\pgfsetfillcolor{textcolor}%
\pgftext[x=0.474556in, y=1.600293in, left, base]{\color{textcolor}\rmfamily\fontsize{8.000000}{9.600000}\selectfont dom\_z}%
\end{pgfscope}%
\begin{pgfscope}%
\pgfsetbuttcap%
\pgfsetroundjoin%
\definecolor{currentfill}{rgb}{0.000000,0.000000,0.000000}%
\pgfsetfillcolor{currentfill}%
\pgfsetlinewidth{0.803000pt}%
\definecolor{currentstroke}{rgb}{0.000000,0.000000,0.000000}%
\pgfsetstrokecolor{currentstroke}%
\pgfsetdash{}{0pt}%
\pgfsys@defobject{currentmarker}{\pgfqpoint{-0.048611in}{0.000000in}}{\pgfqpoint{-0.000000in}{0.000000in}}{%
\pgfpathmoveto{\pgfqpoint{-0.000000in}{0.000000in}}%
\pgfpathlineto{\pgfqpoint{-0.048611in}{0.000000in}}%
\pgfusepath{stroke,fill}%
}%
\begin{pgfscope}%
\pgfsys@transformshift{0.935556in}{2.048646in}%
\pgfsys@useobject{currentmarker}{}%
\end{pgfscope}%
\end{pgfscope}%
\begin{pgfscope}%
\definecolor{textcolor}{rgb}{0.000000,0.000000,0.000000}%
\pgfsetstrokecolor{textcolor}%
\pgfsetfillcolor{textcolor}%
\pgftext[x=0.608889in, y=2.010090in, left, base]{\color{textcolor}\rmfamily\fontsize{8.000000}{9.600000}\selectfont time}%
\end{pgfscope}%
\begin{pgfscope}%
\pgfsetbuttcap%
\pgfsetroundjoin%
\definecolor{currentfill}{rgb}{0.000000,0.000000,0.000000}%
\pgfsetfillcolor{currentfill}%
\pgfsetlinewidth{0.803000pt}%
\definecolor{currentstroke}{rgb}{0.000000,0.000000,0.000000}%
\pgfsetstrokecolor{currentstroke}%
\pgfsetdash{}{0pt}%
\pgfsys@defobject{currentmarker}{\pgfqpoint{-0.048611in}{0.000000in}}{\pgfqpoint{-0.000000in}{0.000000in}}{%
\pgfpathmoveto{\pgfqpoint{-0.000000in}{0.000000in}}%
\pgfpathlineto{\pgfqpoint{-0.048611in}{0.000000in}}%
\pgfusepath{stroke,fill}%
}%
\begin{pgfscope}%
\pgfsys@transformshift{0.935556in}{2.458443in}%
\pgfsys@useobject{currentmarker}{}%
\end{pgfscope}%
\end{pgfscope}%
\begin{pgfscope}%
\definecolor{textcolor}{rgb}{0.000000,0.000000,0.000000}%
\pgfsetstrokecolor{textcolor}%
\pgfsetfillcolor{textcolor}%
\pgftext[x=0.150000in, y=2.419887in, left, base]{\color{textcolor}\rmfamily\fontsize{8.000000}{9.600000}\selectfont charge\_log10}%
\end{pgfscope}%
\begin{pgfscope}%
\pgfsetbuttcap%
\pgfsetroundjoin%
\definecolor{currentfill}{rgb}{0.000000,0.000000,0.000000}%
\pgfsetfillcolor{currentfill}%
\pgfsetlinewidth{0.803000pt}%
\definecolor{currentstroke}{rgb}{0.000000,0.000000,0.000000}%
\pgfsetstrokecolor{currentstroke}%
\pgfsetdash{}{0pt}%
\pgfsys@defobject{currentmarker}{\pgfqpoint{-0.048611in}{0.000000in}}{\pgfqpoint{-0.000000in}{0.000000in}}{%
\pgfpathmoveto{\pgfqpoint{-0.000000in}{0.000000in}}%
\pgfpathlineto{\pgfqpoint{-0.048611in}{0.000000in}}%
\pgfusepath{stroke,fill}%
}%
\begin{pgfscope}%
\pgfsys@transformshift{0.935556in}{2.868240in}%
\pgfsys@useobject{currentmarker}{}%
\end{pgfscope}%
\end{pgfscope}%
\begin{pgfscope}%
\definecolor{textcolor}{rgb}{0.000000,0.000000,0.000000}%
\pgfsetstrokecolor{textcolor}%
\pgfsetfillcolor{textcolor}%
\pgftext[x=0.192000in, y=2.829684in, left, base]{\color{textcolor}\rmfamily\fontsize{8.000000}{9.600000}\selectfont pulse\_width}%
\end{pgfscope}%
\begin{pgfscope}%
\pgfpathrectangle{\pgfqpoint{0.935556in}{0.536494in}}{\pgfqpoint{2.430352in}{2.614506in}}%
\pgfusepath{clip}%
\pgfsetbuttcap%
\pgfsetroundjoin%
\pgfsetlinewidth{0.501875pt}%
\definecolor{currentstroke}{rgb}{1.000000,0.000000,0.000000}%
\pgfsetstrokecolor{currentstroke}%
\pgfsetdash{{1.850000pt}{0.800000pt}}{0.000000pt}%
\pgfpathmoveto{\pgfqpoint{1.464442in}{0.536494in}}%
\pgfpathlineto{\pgfqpoint{1.464442in}{3.151000in}}%
\pgfusepath{stroke}%
\end{pgfscope}%
\begin{pgfscope}%
\pgfsetrectcap%
\pgfsetmiterjoin%
\pgfsetlinewidth{0.803000pt}%
\definecolor{currentstroke}{rgb}{0.000000,0.000000,0.000000}%
\pgfsetstrokecolor{currentstroke}%
\pgfsetdash{}{0pt}%
\pgfpathmoveto{\pgfqpoint{0.935556in}{0.536494in}}%
\pgfpathlineto{\pgfqpoint{0.935556in}{3.151000in}}%
\pgfusepath{stroke}%
\end{pgfscope}%
\begin{pgfscope}%
\pgfsetrectcap%
\pgfsetmiterjoin%
\pgfsetlinewidth{0.803000pt}%
\definecolor{currentstroke}{rgb}{0.000000,0.000000,0.000000}%
\pgfsetstrokecolor{currentstroke}%
\pgfsetdash{}{0pt}%
\pgfpathmoveto{\pgfqpoint{3.365908in}{0.536494in}}%
\pgfpathlineto{\pgfqpoint{3.365908in}{3.151000in}}%
\pgfusepath{stroke}%
\end{pgfscope}%
\begin{pgfscope}%
\pgfsetrectcap%
\pgfsetmiterjoin%
\pgfsetlinewidth{0.803000pt}%
\definecolor{currentstroke}{rgb}{0.000000,0.000000,0.000000}%
\pgfsetstrokecolor{currentstroke}%
\pgfsetdash{}{0pt}%
\pgfpathmoveto{\pgfqpoint{0.935556in}{0.536494in}}%
\pgfpathlineto{\pgfqpoint{3.365908in}{0.536494in}}%
\pgfusepath{stroke}%
\end{pgfscope}%
\begin{pgfscope}%
\pgfsetrectcap%
\pgfsetmiterjoin%
\pgfsetlinewidth{0.803000pt}%
\definecolor{currentstroke}{rgb}{0.000000,0.000000,0.000000}%
\pgfsetstrokecolor{currentstroke}%
\pgfsetdash{}{0pt}%
\pgfpathmoveto{\pgfqpoint{0.935556in}{3.151000in}}%
\pgfpathlineto{\pgfqpoint{3.365908in}{3.151000in}}%
\pgfusepath{stroke}%
\end{pgfscope}%
\begin{pgfscope}%
\definecolor{textcolor}{rgb}{0.000000,0.000000,0.000000}%
\pgfsetstrokecolor{textcolor}%
\pgfsetfillcolor{textcolor}%
\pgftext[x=0.935556in,y=3.234333in,left,base]{\color{textcolor}\rmfamily\fontsize{12.000000}{14.400000}\selectfont Run 2012241359, zenith}%
\end{pgfscope}%
\begin{pgfscope}%
\pgfsetbuttcap%
\pgfsetmiterjoin%
\definecolor{currentfill}{rgb}{1.000000,1.000000,1.000000}%
\pgfsetfillcolor{currentfill}%
\pgfsetfillopacity{0.800000}%
\pgfsetlinewidth{1.003750pt}%
\definecolor{currentstroke}{rgb}{0.800000,0.800000,0.800000}%
\pgfsetstrokecolor{currentstroke}%
\pgfsetstrokeopacity{0.800000}%
\pgfsetdash{}{0pt}%
\pgfpathmoveto{\pgfqpoint{2.507463in}{2.907222in}}%
\pgfpathlineto{\pgfqpoint{3.288130in}{2.907222in}}%
\pgfpathquadraticcurveto{\pgfqpoint{3.310352in}{2.907222in}}{\pgfqpoint{3.310352in}{2.929444in}}%
\pgfpathlineto{\pgfqpoint{3.310352in}{3.073222in}}%
\pgfpathquadraticcurveto{\pgfqpoint{3.310352in}{3.095444in}}{\pgfqpoint{3.288130in}{3.095444in}}%
\pgfpathlineto{\pgfqpoint{2.507463in}{3.095444in}}%
\pgfpathquadraticcurveto{\pgfqpoint{2.485241in}{3.095444in}}{\pgfqpoint{2.485241in}{3.073222in}}%
\pgfpathlineto{\pgfqpoint{2.485241in}{2.929444in}}%
\pgfpathquadraticcurveto{\pgfqpoint{2.485241in}{2.907222in}}{\pgfqpoint{2.507463in}{2.907222in}}%
\pgfpathclose%
\pgfusepath{stroke,fill}%
\end{pgfscope}%
\begin{pgfscope}%
\pgfsetbuttcap%
\pgfsetroundjoin%
\pgfsetlinewidth{0.501875pt}%
\definecolor{currentstroke}{rgb}{1.000000,0.000000,0.000000}%
\pgfsetstrokecolor{currentstroke}%
\pgfsetdash{{1.850000pt}{0.800000pt}}{0.000000pt}%
\pgfpathmoveto{\pgfqpoint{2.529685in}{3.012111in}}%
\pgfpathlineto{\pgfqpoint{2.751908in}{3.012111in}}%
\pgfusepath{stroke}%
\end{pgfscope}%
\begin{pgfscope}%
\definecolor{textcolor}{rgb}{0.000000,0.000000,0.000000}%
\pgfsetstrokecolor{textcolor}%
\pgfsetfillcolor{textcolor}%
\pgftext[x=2.840796in,y=2.973222in,left,base]{\color{textcolor}\rmfamily\fontsize{8.000000}{9.600000}\selectfont Baseline}%
\end{pgfscope}%
\begin{pgfscope}%
\pgfsetbuttcap%
\pgfsetmiterjoin%
\definecolor{currentfill}{rgb}{1.000000,1.000000,1.000000}%
\pgfsetfillcolor{currentfill}%
\pgfsetlinewidth{0.000000pt}%
\definecolor{currentstroke}{rgb}{0.000000,0.000000,0.000000}%
\pgfsetstrokecolor{currentstroke}%
\pgfsetstrokeopacity{0.000000}%
\pgfsetdash{}{0pt}%
\pgfpathmoveto{\pgfqpoint{3.620848in}{0.536494in}}%
\pgfpathlineto{\pgfqpoint{6.051200in}{0.536494in}}%
\pgfpathlineto{\pgfqpoint{6.051200in}{3.151000in}}%
\pgfpathlineto{\pgfqpoint{3.620848in}{3.151000in}}%
\pgfpathclose%
\pgfusepath{fill}%
\end{pgfscope}%
\begin{pgfscope}%
\pgfpathrectangle{\pgfqpoint{3.620848in}{0.536494in}}{\pgfqpoint{2.430352in}{2.614506in}}%
\pgfusepath{clip}%
\pgfsetbuttcap%
\pgfsetmiterjoin%
\definecolor{currentfill}{rgb}{0.313725,0.317647,0.309804}%
\pgfsetfillcolor{currentfill}%
\pgfsetlinewidth{0.000000pt}%
\definecolor{currentstroke}{rgb}{0.000000,0.000000,0.000000}%
\pgfsetstrokecolor{currentstroke}%
\pgfsetstrokeopacity{0.000000}%
\pgfsetdash{}{0pt}%
\pgfpathmoveto{\pgfqpoint{3.620848in}{0.655336in}}%
\pgfpathlineto{\pgfqpoint{5.387807in}{0.655336in}}%
\pgfpathlineto{\pgfqpoint{5.387807in}{0.983173in}}%
\pgfpathlineto{\pgfqpoint{3.620848in}{0.983173in}}%
\pgfpathclose%
\pgfusepath{fill}%
\end{pgfscope}%
\begin{pgfscope}%
\pgfpathrectangle{\pgfqpoint{3.620848in}{0.536494in}}{\pgfqpoint{2.430352in}{2.614506in}}%
\pgfusepath{clip}%
\pgfsetbuttcap%
\pgfsetmiterjoin%
\definecolor{currentfill}{rgb}{0.313725,0.317647,0.309804}%
\pgfsetfillcolor{currentfill}%
\pgfsetlinewidth{0.000000pt}%
\definecolor{currentstroke}{rgb}{0.000000,0.000000,0.000000}%
\pgfsetstrokecolor{currentstroke}%
\pgfsetstrokeopacity{0.000000}%
\pgfsetdash{}{0pt}%
\pgfpathmoveto{\pgfqpoint{3.620848in}{1.065133in}}%
\pgfpathlineto{\pgfqpoint{5.665179in}{1.065133in}}%
\pgfpathlineto{\pgfqpoint{5.665179in}{1.392970in}}%
\pgfpathlineto{\pgfqpoint{3.620848in}{1.392970in}}%
\pgfpathclose%
\pgfusepath{fill}%
\end{pgfscope}%
\begin{pgfscope}%
\pgfpathrectangle{\pgfqpoint{3.620848in}{0.536494in}}{\pgfqpoint{2.430352in}{2.614506in}}%
\pgfusepath{clip}%
\pgfsetbuttcap%
\pgfsetmiterjoin%
\definecolor{currentfill}{rgb}{0.313725,0.317647,0.309804}%
\pgfsetfillcolor{currentfill}%
\pgfsetlinewidth{0.000000pt}%
\definecolor{currentstroke}{rgb}{0.000000,0.000000,0.000000}%
\pgfsetstrokecolor{currentstroke}%
\pgfsetstrokeopacity{0.000000}%
\pgfsetdash{}{0pt}%
\pgfpathmoveto{\pgfqpoint{3.620848in}{1.474930in}}%
\pgfpathlineto{\pgfqpoint{5.918116in}{1.474930in}}%
\pgfpathlineto{\pgfqpoint{5.918116in}{1.802768in}}%
\pgfpathlineto{\pgfqpoint{3.620848in}{1.802768in}}%
\pgfpathclose%
\pgfusepath{fill}%
\end{pgfscope}%
\begin{pgfscope}%
\pgfpathrectangle{\pgfqpoint{3.620848in}{0.536494in}}{\pgfqpoint{2.430352in}{2.614506in}}%
\pgfusepath{clip}%
\pgfsetbuttcap%
\pgfsetmiterjoin%
\definecolor{currentfill}{rgb}{0.313725,0.317647,0.309804}%
\pgfsetfillcolor{currentfill}%
\pgfsetlinewidth{0.000000pt}%
\definecolor{currentstroke}{rgb}{0.000000,0.000000,0.000000}%
\pgfsetstrokecolor{currentstroke}%
\pgfsetstrokeopacity{0.000000}%
\pgfsetdash{}{0pt}%
\pgfpathmoveto{\pgfqpoint{3.620848in}{1.884727in}}%
\pgfpathlineto{\pgfqpoint{5.935469in}{1.884727in}}%
\pgfpathlineto{\pgfqpoint{5.935469in}{2.212565in}}%
\pgfpathlineto{\pgfqpoint{3.620848in}{2.212565in}}%
\pgfpathclose%
\pgfusepath{fill}%
\end{pgfscope}%
\begin{pgfscope}%
\pgfpathrectangle{\pgfqpoint{3.620848in}{0.536494in}}{\pgfqpoint{2.430352in}{2.614506in}}%
\pgfusepath{clip}%
\pgfsetbuttcap%
\pgfsetmiterjoin%
\definecolor{currentfill}{rgb}{0.313725,0.317647,0.309804}%
\pgfsetfillcolor{currentfill}%
\pgfsetlinewidth{0.000000pt}%
\definecolor{currentstroke}{rgb}{0.000000,0.000000,0.000000}%
\pgfsetstrokecolor{currentstroke}%
\pgfsetstrokeopacity{0.000000}%
\pgfsetdash{}{0pt}%
\pgfpathmoveto{\pgfqpoint{3.620848in}{2.294524in}}%
\pgfpathlineto{\pgfqpoint{5.426199in}{2.294524in}}%
\pgfpathlineto{\pgfqpoint{5.426199in}{2.622362in}}%
\pgfpathlineto{\pgfqpoint{3.620848in}{2.622362in}}%
\pgfpathclose%
\pgfusepath{fill}%
\end{pgfscope}%
\begin{pgfscope}%
\pgfpathrectangle{\pgfqpoint{3.620848in}{0.536494in}}{\pgfqpoint{2.430352in}{2.614506in}}%
\pgfusepath{clip}%
\pgfsetbuttcap%
\pgfsetmiterjoin%
\definecolor{currentfill}{rgb}{0.313725,0.317647,0.309804}%
\pgfsetfillcolor{currentfill}%
\pgfsetlinewidth{0.000000pt}%
\definecolor{currentstroke}{rgb}{0.000000,0.000000,0.000000}%
\pgfsetstrokecolor{currentstroke}%
\pgfsetstrokeopacity{0.000000}%
\pgfsetdash{}{0pt}%
\pgfpathmoveto{\pgfqpoint{3.620848in}{2.704321in}}%
\pgfpathlineto{\pgfqpoint{5.675633in}{2.704321in}}%
\pgfpathlineto{\pgfqpoint{5.675633in}{3.032159in}}%
\pgfpathlineto{\pgfqpoint{3.620848in}{3.032159in}}%
\pgfpathclose%
\pgfusepath{fill}%
\end{pgfscope}%
\begin{pgfscope}%
\pgfsetbuttcap%
\pgfsetroundjoin%
\definecolor{currentfill}{rgb}{0.000000,0.000000,0.000000}%
\pgfsetfillcolor{currentfill}%
\pgfsetlinewidth{0.803000pt}%
\definecolor{currentstroke}{rgb}{0.000000,0.000000,0.000000}%
\pgfsetstrokecolor{currentstroke}%
\pgfsetdash{}{0pt}%
\pgfsys@defobject{currentmarker}{\pgfqpoint{0.000000in}{-0.048611in}}{\pgfqpoint{0.000000in}{0.000000in}}{%
\pgfpathmoveto{\pgfqpoint{0.000000in}{0.000000in}}%
\pgfpathlineto{\pgfqpoint{0.000000in}{-0.048611in}}%
\pgfusepath{stroke,fill}%
}%
\begin{pgfscope}%
\pgfsys@transformshift{3.620848in}{0.536494in}%
\pgfsys@useobject{currentmarker}{}%
\end{pgfscope}%
\end{pgfscope}%
\begin{pgfscope}%
\definecolor{textcolor}{rgb}{0.000000,0.000000,0.000000}%
\pgfsetstrokecolor{textcolor}%
\pgfsetfillcolor{textcolor}%
\pgftext[x=3.620848in,y=0.439272in,,top]{\color{textcolor}\rmfamily\fontsize{8.000000}{9.600000}\selectfont \(\displaystyle {0.0}\)}%
\end{pgfscope}%
\begin{pgfscope}%
\pgfsetbuttcap%
\pgfsetroundjoin%
\definecolor{currentfill}{rgb}{0.000000,0.000000,0.000000}%
\pgfsetfillcolor{currentfill}%
\pgfsetlinewidth{0.803000pt}%
\definecolor{currentstroke}{rgb}{0.000000,0.000000,0.000000}%
\pgfsetstrokecolor{currentstroke}%
\pgfsetdash{}{0pt}%
\pgfsys@defobject{currentmarker}{\pgfqpoint{0.000000in}{-0.048611in}}{\pgfqpoint{0.000000in}{0.000000in}}{%
\pgfpathmoveto{\pgfqpoint{0.000000in}{0.000000in}}%
\pgfpathlineto{\pgfqpoint{0.000000in}{-0.048611in}}%
\pgfusepath{stroke,fill}%
}%
\begin{pgfscope}%
\pgfsys@transformshift{3.974245in}{0.536494in}%
\pgfsys@useobject{currentmarker}{}%
\end{pgfscope}%
\end{pgfscope}%
\begin{pgfscope}%
\definecolor{textcolor}{rgb}{0.000000,0.000000,0.000000}%
\pgfsetstrokecolor{textcolor}%
\pgfsetfillcolor{textcolor}%
\pgftext[x=3.974245in,y=0.439272in,,top]{\color{textcolor}\rmfamily\fontsize{8.000000}{9.600000}\selectfont \(\displaystyle {0.2}\)}%
\end{pgfscope}%
\begin{pgfscope}%
\pgfsetbuttcap%
\pgfsetroundjoin%
\definecolor{currentfill}{rgb}{0.000000,0.000000,0.000000}%
\pgfsetfillcolor{currentfill}%
\pgfsetlinewidth{0.803000pt}%
\definecolor{currentstroke}{rgb}{0.000000,0.000000,0.000000}%
\pgfsetstrokecolor{currentstroke}%
\pgfsetdash{}{0pt}%
\pgfsys@defobject{currentmarker}{\pgfqpoint{0.000000in}{-0.048611in}}{\pgfqpoint{0.000000in}{0.000000in}}{%
\pgfpathmoveto{\pgfqpoint{0.000000in}{0.000000in}}%
\pgfpathlineto{\pgfqpoint{0.000000in}{-0.048611in}}%
\pgfusepath{stroke,fill}%
}%
\begin{pgfscope}%
\pgfsys@transformshift{4.327643in}{0.536494in}%
\pgfsys@useobject{currentmarker}{}%
\end{pgfscope}%
\end{pgfscope}%
\begin{pgfscope}%
\definecolor{textcolor}{rgb}{0.000000,0.000000,0.000000}%
\pgfsetstrokecolor{textcolor}%
\pgfsetfillcolor{textcolor}%
\pgftext[x=4.327643in,y=0.439272in,,top]{\color{textcolor}\rmfamily\fontsize{8.000000}{9.600000}\selectfont \(\displaystyle {0.4}\)}%
\end{pgfscope}%
\begin{pgfscope}%
\pgfsetbuttcap%
\pgfsetroundjoin%
\definecolor{currentfill}{rgb}{0.000000,0.000000,0.000000}%
\pgfsetfillcolor{currentfill}%
\pgfsetlinewidth{0.803000pt}%
\definecolor{currentstroke}{rgb}{0.000000,0.000000,0.000000}%
\pgfsetstrokecolor{currentstroke}%
\pgfsetdash{}{0pt}%
\pgfsys@defobject{currentmarker}{\pgfqpoint{0.000000in}{-0.048611in}}{\pgfqpoint{0.000000in}{0.000000in}}{%
\pgfpathmoveto{\pgfqpoint{0.000000in}{0.000000in}}%
\pgfpathlineto{\pgfqpoint{0.000000in}{-0.048611in}}%
\pgfusepath{stroke,fill}%
}%
\begin{pgfscope}%
\pgfsys@transformshift{4.681041in}{0.536494in}%
\pgfsys@useobject{currentmarker}{}%
\end{pgfscope}%
\end{pgfscope}%
\begin{pgfscope}%
\definecolor{textcolor}{rgb}{0.000000,0.000000,0.000000}%
\pgfsetstrokecolor{textcolor}%
\pgfsetfillcolor{textcolor}%
\pgftext[x=4.681041in,y=0.439272in,,top]{\color{textcolor}\rmfamily\fontsize{8.000000}{9.600000}\selectfont \(\displaystyle {0.6}\)}%
\end{pgfscope}%
\begin{pgfscope}%
\pgfsetbuttcap%
\pgfsetroundjoin%
\definecolor{currentfill}{rgb}{0.000000,0.000000,0.000000}%
\pgfsetfillcolor{currentfill}%
\pgfsetlinewidth{0.803000pt}%
\definecolor{currentstroke}{rgb}{0.000000,0.000000,0.000000}%
\pgfsetstrokecolor{currentstroke}%
\pgfsetdash{}{0pt}%
\pgfsys@defobject{currentmarker}{\pgfqpoint{0.000000in}{-0.048611in}}{\pgfqpoint{0.000000in}{0.000000in}}{%
\pgfpathmoveto{\pgfqpoint{0.000000in}{0.000000in}}%
\pgfpathlineto{\pgfqpoint{0.000000in}{-0.048611in}}%
\pgfusepath{stroke,fill}%
}%
\begin{pgfscope}%
\pgfsys@transformshift{5.034439in}{0.536494in}%
\pgfsys@useobject{currentmarker}{}%
\end{pgfscope}%
\end{pgfscope}%
\begin{pgfscope}%
\definecolor{textcolor}{rgb}{0.000000,0.000000,0.000000}%
\pgfsetstrokecolor{textcolor}%
\pgfsetfillcolor{textcolor}%
\pgftext[x=5.034439in,y=0.439272in,,top]{\color{textcolor}\rmfamily\fontsize{8.000000}{9.600000}\selectfont \(\displaystyle {0.8}\)}%
\end{pgfscope}%
\begin{pgfscope}%
\pgfsetbuttcap%
\pgfsetroundjoin%
\definecolor{currentfill}{rgb}{0.000000,0.000000,0.000000}%
\pgfsetfillcolor{currentfill}%
\pgfsetlinewidth{0.803000pt}%
\definecolor{currentstroke}{rgb}{0.000000,0.000000,0.000000}%
\pgfsetstrokecolor{currentstroke}%
\pgfsetdash{}{0pt}%
\pgfsys@defobject{currentmarker}{\pgfqpoint{0.000000in}{-0.048611in}}{\pgfqpoint{0.000000in}{0.000000in}}{%
\pgfpathmoveto{\pgfqpoint{0.000000in}{0.000000in}}%
\pgfpathlineto{\pgfqpoint{0.000000in}{-0.048611in}}%
\pgfusepath{stroke,fill}%
}%
\begin{pgfscope}%
\pgfsys@transformshift{5.387837in}{0.536494in}%
\pgfsys@useobject{currentmarker}{}%
\end{pgfscope}%
\end{pgfscope}%
\begin{pgfscope}%
\definecolor{textcolor}{rgb}{0.000000,0.000000,0.000000}%
\pgfsetstrokecolor{textcolor}%
\pgfsetfillcolor{textcolor}%
\pgftext[x=5.387837in,y=0.439272in,,top]{\color{textcolor}\rmfamily\fontsize{8.000000}{9.600000}\selectfont \(\displaystyle {1.0}\)}%
\end{pgfscope}%
\begin{pgfscope}%
\pgfsetbuttcap%
\pgfsetroundjoin%
\definecolor{currentfill}{rgb}{0.000000,0.000000,0.000000}%
\pgfsetfillcolor{currentfill}%
\pgfsetlinewidth{0.803000pt}%
\definecolor{currentstroke}{rgb}{0.000000,0.000000,0.000000}%
\pgfsetstrokecolor{currentstroke}%
\pgfsetdash{}{0pt}%
\pgfsys@defobject{currentmarker}{\pgfqpoint{0.000000in}{-0.048611in}}{\pgfqpoint{0.000000in}{0.000000in}}{%
\pgfpathmoveto{\pgfqpoint{0.000000in}{0.000000in}}%
\pgfpathlineto{\pgfqpoint{0.000000in}{-0.048611in}}%
\pgfusepath{stroke,fill}%
}%
\begin{pgfscope}%
\pgfsys@transformshift{5.741234in}{0.536494in}%
\pgfsys@useobject{currentmarker}{}%
\end{pgfscope}%
\end{pgfscope}%
\begin{pgfscope}%
\definecolor{textcolor}{rgb}{0.000000,0.000000,0.000000}%
\pgfsetstrokecolor{textcolor}%
\pgfsetfillcolor{textcolor}%
\pgftext[x=5.741234in,y=0.439272in,,top]{\color{textcolor}\rmfamily\fontsize{8.000000}{9.600000}\selectfont \(\displaystyle {1.2}\)}%
\end{pgfscope}%
\begin{pgfscope}%
\definecolor{textcolor}{rgb}{0.000000,0.000000,0.000000}%
\pgfsetstrokecolor{textcolor}%
\pgfsetfillcolor{textcolor}%
\pgftext[x=4.836024in,y=0.285050in,,top]{\color{textcolor}\rmfamily\fontsize{10.950000}{13.140000}\selectfont Permutation importance}%
\end{pgfscope}%
\begin{pgfscope}%
\pgfsetbuttcap%
\pgfsetroundjoin%
\definecolor{currentfill}{rgb}{0.000000,0.000000,0.000000}%
\pgfsetfillcolor{currentfill}%
\pgfsetlinewidth{0.803000pt}%
\definecolor{currentstroke}{rgb}{0.000000,0.000000,0.000000}%
\pgfsetstrokecolor{currentstroke}%
\pgfsetdash{}{0pt}%
\pgfsys@defobject{currentmarker}{\pgfqpoint{-0.048611in}{0.000000in}}{\pgfqpoint{-0.000000in}{0.000000in}}{%
\pgfpathmoveto{\pgfqpoint{-0.000000in}{0.000000in}}%
\pgfpathlineto{\pgfqpoint{-0.048611in}{0.000000in}}%
\pgfusepath{stroke,fill}%
}%
\begin{pgfscope}%
\pgfsys@transformshift{3.620848in}{0.819254in}%
\pgfsys@useobject{currentmarker}{}%
\end{pgfscope}%
\end{pgfscope}%
\begin{pgfscope}%
\pgfsetbuttcap%
\pgfsetroundjoin%
\definecolor{currentfill}{rgb}{0.000000,0.000000,0.000000}%
\pgfsetfillcolor{currentfill}%
\pgfsetlinewidth{0.803000pt}%
\definecolor{currentstroke}{rgb}{0.000000,0.000000,0.000000}%
\pgfsetstrokecolor{currentstroke}%
\pgfsetdash{}{0pt}%
\pgfsys@defobject{currentmarker}{\pgfqpoint{-0.048611in}{0.000000in}}{\pgfqpoint{-0.000000in}{0.000000in}}{%
\pgfpathmoveto{\pgfqpoint{-0.000000in}{0.000000in}}%
\pgfpathlineto{\pgfqpoint{-0.048611in}{0.000000in}}%
\pgfusepath{stroke,fill}%
}%
\begin{pgfscope}%
\pgfsys@transformshift{3.620848in}{1.229052in}%
\pgfsys@useobject{currentmarker}{}%
\end{pgfscope}%
\end{pgfscope}%
\begin{pgfscope}%
\pgfsetbuttcap%
\pgfsetroundjoin%
\definecolor{currentfill}{rgb}{0.000000,0.000000,0.000000}%
\pgfsetfillcolor{currentfill}%
\pgfsetlinewidth{0.803000pt}%
\definecolor{currentstroke}{rgb}{0.000000,0.000000,0.000000}%
\pgfsetstrokecolor{currentstroke}%
\pgfsetdash{}{0pt}%
\pgfsys@defobject{currentmarker}{\pgfqpoint{-0.048611in}{0.000000in}}{\pgfqpoint{-0.000000in}{0.000000in}}{%
\pgfpathmoveto{\pgfqpoint{-0.000000in}{0.000000in}}%
\pgfpathlineto{\pgfqpoint{-0.048611in}{0.000000in}}%
\pgfusepath{stroke,fill}%
}%
\begin{pgfscope}%
\pgfsys@transformshift{3.620848in}{1.638849in}%
\pgfsys@useobject{currentmarker}{}%
\end{pgfscope}%
\end{pgfscope}%
\begin{pgfscope}%
\pgfsetbuttcap%
\pgfsetroundjoin%
\definecolor{currentfill}{rgb}{0.000000,0.000000,0.000000}%
\pgfsetfillcolor{currentfill}%
\pgfsetlinewidth{0.803000pt}%
\definecolor{currentstroke}{rgb}{0.000000,0.000000,0.000000}%
\pgfsetstrokecolor{currentstroke}%
\pgfsetdash{}{0pt}%
\pgfsys@defobject{currentmarker}{\pgfqpoint{-0.048611in}{0.000000in}}{\pgfqpoint{-0.000000in}{0.000000in}}{%
\pgfpathmoveto{\pgfqpoint{-0.000000in}{0.000000in}}%
\pgfpathlineto{\pgfqpoint{-0.048611in}{0.000000in}}%
\pgfusepath{stroke,fill}%
}%
\begin{pgfscope}%
\pgfsys@transformshift{3.620848in}{2.048646in}%
\pgfsys@useobject{currentmarker}{}%
\end{pgfscope}%
\end{pgfscope}%
\begin{pgfscope}%
\pgfsetbuttcap%
\pgfsetroundjoin%
\definecolor{currentfill}{rgb}{0.000000,0.000000,0.000000}%
\pgfsetfillcolor{currentfill}%
\pgfsetlinewidth{0.803000pt}%
\definecolor{currentstroke}{rgb}{0.000000,0.000000,0.000000}%
\pgfsetstrokecolor{currentstroke}%
\pgfsetdash{}{0pt}%
\pgfsys@defobject{currentmarker}{\pgfqpoint{-0.048611in}{0.000000in}}{\pgfqpoint{-0.000000in}{0.000000in}}{%
\pgfpathmoveto{\pgfqpoint{-0.000000in}{0.000000in}}%
\pgfpathlineto{\pgfqpoint{-0.048611in}{0.000000in}}%
\pgfusepath{stroke,fill}%
}%
\begin{pgfscope}%
\pgfsys@transformshift{3.620848in}{2.458443in}%
\pgfsys@useobject{currentmarker}{}%
\end{pgfscope}%
\end{pgfscope}%
\begin{pgfscope}%
\pgfsetbuttcap%
\pgfsetroundjoin%
\definecolor{currentfill}{rgb}{0.000000,0.000000,0.000000}%
\pgfsetfillcolor{currentfill}%
\pgfsetlinewidth{0.803000pt}%
\definecolor{currentstroke}{rgb}{0.000000,0.000000,0.000000}%
\pgfsetstrokecolor{currentstroke}%
\pgfsetdash{}{0pt}%
\pgfsys@defobject{currentmarker}{\pgfqpoint{-0.048611in}{0.000000in}}{\pgfqpoint{-0.000000in}{0.000000in}}{%
\pgfpathmoveto{\pgfqpoint{-0.000000in}{0.000000in}}%
\pgfpathlineto{\pgfqpoint{-0.048611in}{0.000000in}}%
\pgfusepath{stroke,fill}%
}%
\begin{pgfscope}%
\pgfsys@transformshift{3.620848in}{2.868240in}%
\pgfsys@useobject{currentmarker}{}%
\end{pgfscope}%
\end{pgfscope}%
\begin{pgfscope}%
\pgfpathrectangle{\pgfqpoint{3.620848in}{0.536494in}}{\pgfqpoint{2.430352in}{2.614506in}}%
\pgfusepath{clip}%
\pgfsetbuttcap%
\pgfsetroundjoin%
\pgfsetlinewidth{0.501875pt}%
\definecolor{currentstroke}{rgb}{1.000000,0.000000,0.000000}%
\pgfsetstrokecolor{currentstroke}%
\pgfsetdash{{1.850000pt}{0.800000pt}}{0.000000pt}%
\pgfpathmoveto{\pgfqpoint{5.387837in}{0.536494in}}%
\pgfpathlineto{\pgfqpoint{5.387837in}{3.151000in}}%
\pgfusepath{stroke}%
\end{pgfscope}%
\begin{pgfscope}%
\pgfsetrectcap%
\pgfsetmiterjoin%
\pgfsetlinewidth{0.803000pt}%
\definecolor{currentstroke}{rgb}{0.000000,0.000000,0.000000}%
\pgfsetstrokecolor{currentstroke}%
\pgfsetdash{}{0pt}%
\pgfpathmoveto{\pgfqpoint{3.620848in}{0.536494in}}%
\pgfpathlineto{\pgfqpoint{3.620848in}{3.151000in}}%
\pgfusepath{stroke}%
\end{pgfscope}%
\begin{pgfscope}%
\pgfsetrectcap%
\pgfsetmiterjoin%
\pgfsetlinewidth{0.803000pt}%
\definecolor{currentstroke}{rgb}{0.000000,0.000000,0.000000}%
\pgfsetstrokecolor{currentstroke}%
\pgfsetdash{}{0pt}%
\pgfpathmoveto{\pgfqpoint{6.051200in}{0.536494in}}%
\pgfpathlineto{\pgfqpoint{6.051200in}{3.151000in}}%
\pgfusepath{stroke}%
\end{pgfscope}%
\begin{pgfscope}%
\pgfsetrectcap%
\pgfsetmiterjoin%
\pgfsetlinewidth{0.803000pt}%
\definecolor{currentstroke}{rgb}{0.000000,0.000000,0.000000}%
\pgfsetstrokecolor{currentstroke}%
\pgfsetdash{}{0pt}%
\pgfpathmoveto{\pgfqpoint{3.620848in}{0.536494in}}%
\pgfpathlineto{\pgfqpoint{6.051200in}{0.536494in}}%
\pgfusepath{stroke}%
\end{pgfscope}%
\begin{pgfscope}%
\pgfsetrectcap%
\pgfsetmiterjoin%
\pgfsetlinewidth{0.803000pt}%
\definecolor{currentstroke}{rgb}{0.000000,0.000000,0.000000}%
\pgfsetstrokecolor{currentstroke}%
\pgfsetdash{}{0pt}%
\pgfpathmoveto{\pgfqpoint{3.620848in}{3.151000in}}%
\pgfpathlineto{\pgfqpoint{6.051200in}{3.151000in}}%
\pgfusepath{stroke}%
\end{pgfscope}%
\begin{pgfscope}%
\definecolor{textcolor}{rgb}{0.000000,0.000000,0.000000}%
\pgfsetstrokecolor{textcolor}%
\pgfsetfillcolor{textcolor}%
\pgftext[x=3.620848in,y=3.234333in,left,base]{\color{textcolor}\rmfamily\fontsize{12.000000}{14.400000}\selectfont Run 2012242224, energy}%
\end{pgfscope}%
\end{pgfpicture}%
\makeatother%
\endgroup%

    \caption{Permutation importance for zenith reconstruction (left) and energy reconstruction (right).
    At no great surprise the zenith reconstruction relies heavily on the \( z \)-coordinate of the activated DOMs, with the relative DOM activation time second-most important.
    Both the \( z \)- and the \( y \)-coordinates are important in the energy case, with time continuing to be important.
    Pulse width is critical to the performance of the network as well.}\label{fig:perm_imp}
\end{figure}

To gauge the contribution of features, permutation importance has been employed.
The straight-forward way of doing this would be to train a model with/without a certain feature, and compare the loss values between the two runs.
Permutation importance works by running \( n + 1 \) inference passes with a trained model, where \( n \) is the number of features.
One run is a normal inference, and represents the baseline while all following inference passes shuffle the values of one feature which becomes random noise for the network.
In this way one may see the value of a feature; if the loss does not change, the feature does not contribute to the networks inference capabilities.
If, however, the loss suffers, the network relies on the feature being present, and the degree to which it does can be quantified as the permuted loss value divided by the baseline---a high permutation importance value thus represents an important feature.

\Vref{fig:perm_imp} shows the application of permutation importance to this problem.
It is unsurprising that the \( z \)-coordinate of the activated DOMs plays a significant role in determining the zenith value of a neutrino, owing to the fact that \( z = \cos{\theta} \).
Relative DOM activation times is similarly important, also not surprising as it gives the order in which the pulses arrive.

The energy reconstruction also uses the \( y \)-coordinate to a larger degree than the zenith reconstruction, because the movement in the plane is probably important for the reconstruction.
In both cases pulse width seems to be important, related to the resolution of the pulse information gathered by the DOM.\@

\Vref{fig:energy_comparison,fig:zenith_comparison} show the overall performance of the best performing network, with the best performing hyperparameters as outlined in~\vref{chap:design}.
Until around \SI{50}{\giga\electronvolt} CubeFlow outperforms Retro Reco, both in energy and zenith reconstruction.
In the zenith case the performance drops off rapidly, and it should be noted that the lower plot---which shows the relative improvement---has been constrained to \( \pm \SI{100}{\percent} \).

\begin{figure}
    \centering
    %% Creator: Matplotlib, PGF backend
%%
%% To include the figure in your LaTeX document, write
%%   \input{<filename>.pgf}
%%
%% Make sure the required packages are loaded in your preamble
%%   \usepackage{pgf}
%%
%% and, on pdftex
%%   \usepackage[utf8]{inputenc}\DeclareUnicodeCharacter{2212}{-}
%%
%% or, on luatex and xetex
%%   \usepackage{unicode-math}
%%
%% Figures using additional raster images can only be included by \input if
%% they are in the same directory as the main LaTeX file. For loading figures
%% from other directories you can use the `import` package
%%   \usepackage{import}
%%
%% and then include the figures with
%%   \import{<path to file>}{<filename>.pgf}
%%
%% Matplotlib used the following preamble
%%   \usepackage{siunitx} \usepackage{amsmath} \usepackage{bm}
%%   \usepackage{fontspec}
%%
\begingroup%
\makeatletter%
\begin{pgfpicture}%
\pgfpathrectangle{\pgfpointorigin}{\pgfqpoint{6.201200in}{3.000000in}}%
\pgfusepath{use as bounding box, clip}%
\begin{pgfscope}%
\pgfsetbuttcap%
\pgfsetmiterjoin%
\definecolor{currentfill}{rgb}{1.000000,1.000000,1.000000}%
\pgfsetfillcolor{currentfill}%
\pgfsetlinewidth{0.000000pt}%
\definecolor{currentstroke}{rgb}{1.000000,1.000000,1.000000}%
\pgfsetstrokecolor{currentstroke}%
\pgfsetdash{}{0pt}%
\pgfpathmoveto{\pgfqpoint{0.000000in}{0.000000in}}%
\pgfpathlineto{\pgfqpoint{6.201200in}{0.000000in}}%
\pgfpathlineto{\pgfqpoint{6.201200in}{3.000000in}}%
\pgfpathlineto{\pgfqpoint{0.000000in}{3.000000in}}%
\pgfpathclose%
\pgfusepath{fill}%
\end{pgfscope}%
\begin{pgfscope}%
\pgfsetbuttcap%
\pgfsetmiterjoin%
\definecolor{currentfill}{rgb}{1.000000,1.000000,1.000000}%
\pgfsetfillcolor{currentfill}%
\pgfsetlinewidth{0.000000pt}%
\definecolor{currentstroke}{rgb}{0.000000,0.000000,0.000000}%
\pgfsetstrokecolor{currentstroke}%
\pgfsetstrokeopacity{0.000000}%
\pgfsetdash{}{0pt}%
\pgfpathmoveto{\pgfqpoint{0.572918in}{0.553781in}}%
\pgfpathlineto{\pgfqpoint{5.587445in}{0.553781in}}%
\pgfpathlineto{\pgfqpoint{5.587445in}{2.649333in}}%
\pgfpathlineto{\pgfqpoint{0.572918in}{2.649333in}}%
\pgfpathclose%
\pgfusepath{fill}%
\end{pgfscope}%
\begin{pgfscope}%
\pgfpathrectangle{\pgfqpoint{0.572918in}{0.553781in}}{\pgfqpoint{5.014527in}{2.095553in}}%
\pgfusepath{clip}%
\pgfsetbuttcap%
\pgfsetroundjoin%
\pgfsetlinewidth{0.501875pt}%
\definecolor{currentstroke}{rgb}{0.690196,0.690196,0.690196}%
\pgfsetstrokecolor{currentstroke}%
\pgfsetstrokeopacity{0.500000}%
\pgfsetdash{{0.500000pt}{0.825000pt}}{0.000000pt}%
\pgfpathmoveto{\pgfqpoint{0.828660in}{0.553781in}}%
\pgfpathlineto{\pgfqpoint{0.828660in}{2.649333in}}%
\pgfusepath{stroke}%
\end{pgfscope}%
\begin{pgfscope}%
\pgfsetbuttcap%
\pgfsetroundjoin%
\definecolor{currentfill}{rgb}{0.000000,0.000000,0.000000}%
\pgfsetfillcolor{currentfill}%
\pgfsetlinewidth{0.803000pt}%
\definecolor{currentstroke}{rgb}{0.000000,0.000000,0.000000}%
\pgfsetstrokecolor{currentstroke}%
\pgfsetdash{}{0pt}%
\pgfsys@defobject{currentmarker}{\pgfqpoint{0.000000in}{-0.048611in}}{\pgfqpoint{0.000000in}{0.000000in}}{%
\pgfpathmoveto{\pgfqpoint{0.000000in}{0.000000in}}%
\pgfpathlineto{\pgfqpoint{0.000000in}{-0.048611in}}%
\pgfusepath{stroke,fill}%
}%
\begin{pgfscope}%
\pgfsys@transformshift{0.828660in}{0.553781in}%
\pgfsys@useobject{currentmarker}{}%
\end{pgfscope}%
\end{pgfscope}%
\begin{pgfscope}%
\definecolor{textcolor}{rgb}{0.000000,0.000000,0.000000}%
\pgfsetstrokecolor{textcolor}%
\pgfsetfillcolor{textcolor}%
\pgftext[x=0.828660in,y=0.456558in,,top]{\color{textcolor}\rmfamily\fontsize{8.000000}{9.600000}\selectfont \(\displaystyle {-0.5}\)}%
\end{pgfscope}%
\begin{pgfscope}%
\pgfpathrectangle{\pgfqpoint{0.572918in}{0.553781in}}{\pgfqpoint{5.014527in}{2.095553in}}%
\pgfusepath{clip}%
\pgfsetbuttcap%
\pgfsetroundjoin%
\pgfsetlinewidth{0.501875pt}%
\definecolor{currentstroke}{rgb}{0.690196,0.690196,0.690196}%
\pgfsetstrokecolor{currentstroke}%
\pgfsetstrokeopacity{0.500000}%
\pgfsetdash{{0.500000pt}{0.825000pt}}{0.000000pt}%
\pgfpathmoveto{\pgfqpoint{1.475924in}{0.553781in}}%
\pgfpathlineto{\pgfqpoint{1.475924in}{2.649333in}}%
\pgfusepath{stroke}%
\end{pgfscope}%
\begin{pgfscope}%
\pgfsetbuttcap%
\pgfsetroundjoin%
\definecolor{currentfill}{rgb}{0.000000,0.000000,0.000000}%
\pgfsetfillcolor{currentfill}%
\pgfsetlinewidth{0.803000pt}%
\definecolor{currentstroke}{rgb}{0.000000,0.000000,0.000000}%
\pgfsetstrokecolor{currentstroke}%
\pgfsetdash{}{0pt}%
\pgfsys@defobject{currentmarker}{\pgfqpoint{0.000000in}{-0.048611in}}{\pgfqpoint{0.000000in}{0.000000in}}{%
\pgfpathmoveto{\pgfqpoint{0.000000in}{0.000000in}}%
\pgfpathlineto{\pgfqpoint{0.000000in}{-0.048611in}}%
\pgfusepath{stroke,fill}%
}%
\begin{pgfscope}%
\pgfsys@transformshift{1.475924in}{0.553781in}%
\pgfsys@useobject{currentmarker}{}%
\end{pgfscope}%
\end{pgfscope}%
\begin{pgfscope}%
\definecolor{textcolor}{rgb}{0.000000,0.000000,0.000000}%
\pgfsetstrokecolor{textcolor}%
\pgfsetfillcolor{textcolor}%
\pgftext[x=1.475924in,y=0.456558in,,top]{\color{textcolor}\rmfamily\fontsize{8.000000}{9.600000}\selectfont \(\displaystyle {0.0}\)}%
\end{pgfscope}%
\begin{pgfscope}%
\pgfpathrectangle{\pgfqpoint{0.572918in}{0.553781in}}{\pgfqpoint{5.014527in}{2.095553in}}%
\pgfusepath{clip}%
\pgfsetbuttcap%
\pgfsetroundjoin%
\pgfsetlinewidth{0.501875pt}%
\definecolor{currentstroke}{rgb}{0.690196,0.690196,0.690196}%
\pgfsetstrokecolor{currentstroke}%
\pgfsetstrokeopacity{0.500000}%
\pgfsetdash{{0.500000pt}{0.825000pt}}{0.000000pt}%
\pgfpathmoveto{\pgfqpoint{2.123189in}{0.553781in}}%
\pgfpathlineto{\pgfqpoint{2.123189in}{2.649333in}}%
\pgfusepath{stroke}%
\end{pgfscope}%
\begin{pgfscope}%
\pgfsetbuttcap%
\pgfsetroundjoin%
\definecolor{currentfill}{rgb}{0.000000,0.000000,0.000000}%
\pgfsetfillcolor{currentfill}%
\pgfsetlinewidth{0.803000pt}%
\definecolor{currentstroke}{rgb}{0.000000,0.000000,0.000000}%
\pgfsetstrokecolor{currentstroke}%
\pgfsetdash{}{0pt}%
\pgfsys@defobject{currentmarker}{\pgfqpoint{0.000000in}{-0.048611in}}{\pgfqpoint{0.000000in}{0.000000in}}{%
\pgfpathmoveto{\pgfqpoint{0.000000in}{0.000000in}}%
\pgfpathlineto{\pgfqpoint{0.000000in}{-0.048611in}}%
\pgfusepath{stroke,fill}%
}%
\begin{pgfscope}%
\pgfsys@transformshift{2.123189in}{0.553781in}%
\pgfsys@useobject{currentmarker}{}%
\end{pgfscope}%
\end{pgfscope}%
\begin{pgfscope}%
\definecolor{textcolor}{rgb}{0.000000,0.000000,0.000000}%
\pgfsetstrokecolor{textcolor}%
\pgfsetfillcolor{textcolor}%
\pgftext[x=2.123189in,y=0.456558in,,top]{\color{textcolor}\rmfamily\fontsize{8.000000}{9.600000}\selectfont \(\displaystyle {0.5}\)}%
\end{pgfscope}%
\begin{pgfscope}%
\pgfpathrectangle{\pgfqpoint{0.572918in}{0.553781in}}{\pgfqpoint{5.014527in}{2.095553in}}%
\pgfusepath{clip}%
\pgfsetbuttcap%
\pgfsetroundjoin%
\pgfsetlinewidth{0.501875pt}%
\definecolor{currentstroke}{rgb}{0.690196,0.690196,0.690196}%
\pgfsetstrokecolor{currentstroke}%
\pgfsetstrokeopacity{0.500000}%
\pgfsetdash{{0.500000pt}{0.825000pt}}{0.000000pt}%
\pgfpathmoveto{\pgfqpoint{2.770454in}{0.553781in}}%
\pgfpathlineto{\pgfqpoint{2.770454in}{2.649333in}}%
\pgfusepath{stroke}%
\end{pgfscope}%
\begin{pgfscope}%
\pgfsetbuttcap%
\pgfsetroundjoin%
\definecolor{currentfill}{rgb}{0.000000,0.000000,0.000000}%
\pgfsetfillcolor{currentfill}%
\pgfsetlinewidth{0.803000pt}%
\definecolor{currentstroke}{rgb}{0.000000,0.000000,0.000000}%
\pgfsetstrokecolor{currentstroke}%
\pgfsetdash{}{0pt}%
\pgfsys@defobject{currentmarker}{\pgfqpoint{0.000000in}{-0.048611in}}{\pgfqpoint{0.000000in}{0.000000in}}{%
\pgfpathmoveto{\pgfqpoint{0.000000in}{0.000000in}}%
\pgfpathlineto{\pgfqpoint{0.000000in}{-0.048611in}}%
\pgfusepath{stroke,fill}%
}%
\begin{pgfscope}%
\pgfsys@transformshift{2.770454in}{0.553781in}%
\pgfsys@useobject{currentmarker}{}%
\end{pgfscope}%
\end{pgfscope}%
\begin{pgfscope}%
\definecolor{textcolor}{rgb}{0.000000,0.000000,0.000000}%
\pgfsetstrokecolor{textcolor}%
\pgfsetfillcolor{textcolor}%
\pgftext[x=2.770454in,y=0.456558in,,top]{\color{textcolor}\rmfamily\fontsize{8.000000}{9.600000}\selectfont \(\displaystyle {1.0}\)}%
\end{pgfscope}%
\begin{pgfscope}%
\pgfpathrectangle{\pgfqpoint{0.572918in}{0.553781in}}{\pgfqpoint{5.014527in}{2.095553in}}%
\pgfusepath{clip}%
\pgfsetbuttcap%
\pgfsetroundjoin%
\pgfsetlinewidth{0.501875pt}%
\definecolor{currentstroke}{rgb}{0.690196,0.690196,0.690196}%
\pgfsetstrokecolor{currentstroke}%
\pgfsetstrokeopacity{0.500000}%
\pgfsetdash{{0.500000pt}{0.825000pt}}{0.000000pt}%
\pgfpathmoveto{\pgfqpoint{3.417718in}{0.553781in}}%
\pgfpathlineto{\pgfqpoint{3.417718in}{2.649333in}}%
\pgfusepath{stroke}%
\end{pgfscope}%
\begin{pgfscope}%
\pgfsetbuttcap%
\pgfsetroundjoin%
\definecolor{currentfill}{rgb}{0.000000,0.000000,0.000000}%
\pgfsetfillcolor{currentfill}%
\pgfsetlinewidth{0.803000pt}%
\definecolor{currentstroke}{rgb}{0.000000,0.000000,0.000000}%
\pgfsetstrokecolor{currentstroke}%
\pgfsetdash{}{0pt}%
\pgfsys@defobject{currentmarker}{\pgfqpoint{0.000000in}{-0.048611in}}{\pgfqpoint{0.000000in}{0.000000in}}{%
\pgfpathmoveto{\pgfqpoint{0.000000in}{0.000000in}}%
\pgfpathlineto{\pgfqpoint{0.000000in}{-0.048611in}}%
\pgfusepath{stroke,fill}%
}%
\begin{pgfscope}%
\pgfsys@transformshift{3.417718in}{0.553781in}%
\pgfsys@useobject{currentmarker}{}%
\end{pgfscope}%
\end{pgfscope}%
\begin{pgfscope}%
\definecolor{textcolor}{rgb}{0.000000,0.000000,0.000000}%
\pgfsetstrokecolor{textcolor}%
\pgfsetfillcolor{textcolor}%
\pgftext[x=3.417718in,y=0.456558in,,top]{\color{textcolor}\rmfamily\fontsize{8.000000}{9.600000}\selectfont \(\displaystyle {1.5}\)}%
\end{pgfscope}%
\begin{pgfscope}%
\pgfpathrectangle{\pgfqpoint{0.572918in}{0.553781in}}{\pgfqpoint{5.014527in}{2.095553in}}%
\pgfusepath{clip}%
\pgfsetbuttcap%
\pgfsetroundjoin%
\pgfsetlinewidth{0.501875pt}%
\definecolor{currentstroke}{rgb}{0.690196,0.690196,0.690196}%
\pgfsetstrokecolor{currentstroke}%
\pgfsetstrokeopacity{0.500000}%
\pgfsetdash{{0.500000pt}{0.825000pt}}{0.000000pt}%
\pgfpathmoveto{\pgfqpoint{4.064983in}{0.553781in}}%
\pgfpathlineto{\pgfqpoint{4.064983in}{2.649333in}}%
\pgfusepath{stroke}%
\end{pgfscope}%
\begin{pgfscope}%
\pgfsetbuttcap%
\pgfsetroundjoin%
\definecolor{currentfill}{rgb}{0.000000,0.000000,0.000000}%
\pgfsetfillcolor{currentfill}%
\pgfsetlinewidth{0.803000pt}%
\definecolor{currentstroke}{rgb}{0.000000,0.000000,0.000000}%
\pgfsetstrokecolor{currentstroke}%
\pgfsetdash{}{0pt}%
\pgfsys@defobject{currentmarker}{\pgfqpoint{0.000000in}{-0.048611in}}{\pgfqpoint{0.000000in}{0.000000in}}{%
\pgfpathmoveto{\pgfqpoint{0.000000in}{0.000000in}}%
\pgfpathlineto{\pgfqpoint{0.000000in}{-0.048611in}}%
\pgfusepath{stroke,fill}%
}%
\begin{pgfscope}%
\pgfsys@transformshift{4.064983in}{0.553781in}%
\pgfsys@useobject{currentmarker}{}%
\end{pgfscope}%
\end{pgfscope}%
\begin{pgfscope}%
\definecolor{textcolor}{rgb}{0.000000,0.000000,0.000000}%
\pgfsetstrokecolor{textcolor}%
\pgfsetfillcolor{textcolor}%
\pgftext[x=4.064983in,y=0.456558in,,top]{\color{textcolor}\rmfamily\fontsize{8.000000}{9.600000}\selectfont \(\displaystyle {2.0}\)}%
\end{pgfscope}%
\begin{pgfscope}%
\pgfpathrectangle{\pgfqpoint{0.572918in}{0.553781in}}{\pgfqpoint{5.014527in}{2.095553in}}%
\pgfusepath{clip}%
\pgfsetbuttcap%
\pgfsetroundjoin%
\pgfsetlinewidth{0.501875pt}%
\definecolor{currentstroke}{rgb}{0.690196,0.690196,0.690196}%
\pgfsetstrokecolor{currentstroke}%
\pgfsetstrokeopacity{0.500000}%
\pgfsetdash{{0.500000pt}{0.825000pt}}{0.000000pt}%
\pgfpathmoveto{\pgfqpoint{4.712248in}{0.553781in}}%
\pgfpathlineto{\pgfqpoint{4.712248in}{2.649333in}}%
\pgfusepath{stroke}%
\end{pgfscope}%
\begin{pgfscope}%
\pgfsetbuttcap%
\pgfsetroundjoin%
\definecolor{currentfill}{rgb}{0.000000,0.000000,0.000000}%
\pgfsetfillcolor{currentfill}%
\pgfsetlinewidth{0.803000pt}%
\definecolor{currentstroke}{rgb}{0.000000,0.000000,0.000000}%
\pgfsetstrokecolor{currentstroke}%
\pgfsetdash{}{0pt}%
\pgfsys@defobject{currentmarker}{\pgfqpoint{0.000000in}{-0.048611in}}{\pgfqpoint{0.000000in}{0.000000in}}{%
\pgfpathmoveto{\pgfqpoint{0.000000in}{0.000000in}}%
\pgfpathlineto{\pgfqpoint{0.000000in}{-0.048611in}}%
\pgfusepath{stroke,fill}%
}%
\begin{pgfscope}%
\pgfsys@transformshift{4.712248in}{0.553781in}%
\pgfsys@useobject{currentmarker}{}%
\end{pgfscope}%
\end{pgfscope}%
\begin{pgfscope}%
\definecolor{textcolor}{rgb}{0.000000,0.000000,0.000000}%
\pgfsetstrokecolor{textcolor}%
\pgfsetfillcolor{textcolor}%
\pgftext[x=4.712248in,y=0.456558in,,top]{\color{textcolor}\rmfamily\fontsize{8.000000}{9.600000}\selectfont \(\displaystyle {2.5}\)}%
\end{pgfscope}%
\begin{pgfscope}%
\pgfpathrectangle{\pgfqpoint{0.572918in}{0.553781in}}{\pgfqpoint{5.014527in}{2.095553in}}%
\pgfusepath{clip}%
\pgfsetbuttcap%
\pgfsetroundjoin%
\pgfsetlinewidth{0.501875pt}%
\definecolor{currentstroke}{rgb}{0.690196,0.690196,0.690196}%
\pgfsetstrokecolor{currentstroke}%
\pgfsetstrokeopacity{0.500000}%
\pgfsetdash{{0.500000pt}{0.825000pt}}{0.000000pt}%
\pgfpathmoveto{\pgfqpoint{5.359512in}{0.553781in}}%
\pgfpathlineto{\pgfqpoint{5.359512in}{2.649333in}}%
\pgfusepath{stroke}%
\end{pgfscope}%
\begin{pgfscope}%
\pgfsetbuttcap%
\pgfsetroundjoin%
\definecolor{currentfill}{rgb}{0.000000,0.000000,0.000000}%
\pgfsetfillcolor{currentfill}%
\pgfsetlinewidth{0.803000pt}%
\definecolor{currentstroke}{rgb}{0.000000,0.000000,0.000000}%
\pgfsetstrokecolor{currentstroke}%
\pgfsetdash{}{0pt}%
\pgfsys@defobject{currentmarker}{\pgfqpoint{0.000000in}{-0.048611in}}{\pgfqpoint{0.000000in}{0.000000in}}{%
\pgfpathmoveto{\pgfqpoint{0.000000in}{0.000000in}}%
\pgfpathlineto{\pgfqpoint{0.000000in}{-0.048611in}}%
\pgfusepath{stroke,fill}%
}%
\begin{pgfscope}%
\pgfsys@transformshift{5.359512in}{0.553781in}%
\pgfsys@useobject{currentmarker}{}%
\end{pgfscope}%
\end{pgfscope}%
\begin{pgfscope}%
\definecolor{textcolor}{rgb}{0.000000,0.000000,0.000000}%
\pgfsetstrokecolor{textcolor}%
\pgfsetfillcolor{textcolor}%
\pgftext[x=5.359512in,y=0.456558in,,top]{\color{textcolor}\rmfamily\fontsize{8.000000}{9.600000}\selectfont \(\displaystyle {3.0}\)}%
\end{pgfscope}%
\begin{pgfscope}%
\definecolor{textcolor}{rgb}{0.000000,0.000000,0.000000}%
\pgfsetstrokecolor{textcolor}%
\pgfsetfillcolor{textcolor}%
\pgftext[x=3.080182in,y=0.302336in,,top]{\color{textcolor}\rmfamily\fontsize{10.950000}{13.140000}\selectfont \(\displaystyle \log_{10}(E_{\textup{true}}) \, \left[ E / \textup{GeV} \right]\)}%
\end{pgfscope}%
\begin{pgfscope}%
\pgfpathrectangle{\pgfqpoint{0.572918in}{0.553781in}}{\pgfqpoint{5.014527in}{2.095553in}}%
\pgfusepath{clip}%
\pgfsetbuttcap%
\pgfsetroundjoin%
\pgfsetlinewidth{0.501875pt}%
\definecolor{currentstroke}{rgb}{0.690196,0.690196,0.690196}%
\pgfsetstrokecolor{currentstroke}%
\pgfsetstrokeopacity{0.500000}%
\pgfsetdash{{0.500000pt}{0.825000pt}}{0.000000pt}%
\pgfpathmoveto{\pgfqpoint{0.572918in}{0.590724in}}%
\pgfpathlineto{\pgfqpoint{5.587445in}{0.590724in}}%
\pgfusepath{stroke}%
\end{pgfscope}%
\begin{pgfscope}%
\pgfsetbuttcap%
\pgfsetroundjoin%
\definecolor{currentfill}{rgb}{0.000000,0.000000,0.000000}%
\pgfsetfillcolor{currentfill}%
\pgfsetlinewidth{0.803000pt}%
\definecolor{currentstroke}{rgb}{0.000000,0.000000,0.000000}%
\pgfsetstrokecolor{currentstroke}%
\pgfsetdash{}{0pt}%
\pgfsys@defobject{currentmarker}{\pgfqpoint{-0.048611in}{0.000000in}}{\pgfqpoint{-0.000000in}{0.000000in}}{%
\pgfpathmoveto{\pgfqpoint{-0.000000in}{0.000000in}}%
\pgfpathlineto{\pgfqpoint{-0.048611in}{0.000000in}}%
\pgfusepath{stroke,fill}%
}%
\begin{pgfscope}%
\pgfsys@transformshift{0.572918in}{0.590724in}%
\pgfsys@useobject{currentmarker}{}%
\end{pgfscope}%
\end{pgfscope}%
\begin{pgfscope}%
\definecolor{textcolor}{rgb}{0.000000,0.000000,0.000000}%
\pgfsetstrokecolor{textcolor}%
\pgfsetfillcolor{textcolor}%
\pgftext[x=0.357639in, y=0.552168in, left, base]{\color{textcolor}\rmfamily\fontsize{8.000000}{9.600000}\selectfont \(\displaystyle {10}\)}%
\end{pgfscope}%
\begin{pgfscope}%
\pgfpathrectangle{\pgfqpoint{0.572918in}{0.553781in}}{\pgfqpoint{5.014527in}{2.095553in}}%
\pgfusepath{clip}%
\pgfsetbuttcap%
\pgfsetroundjoin%
\pgfsetlinewidth{0.501875pt}%
\definecolor{currentstroke}{rgb}{0.690196,0.690196,0.690196}%
\pgfsetstrokecolor{currentstroke}%
\pgfsetstrokeopacity{0.500000}%
\pgfsetdash{{0.500000pt}{0.825000pt}}{0.000000pt}%
\pgfpathmoveto{\pgfqpoint{0.572918in}{0.869587in}}%
\pgfpathlineto{\pgfqpoint{5.587445in}{0.869587in}}%
\pgfusepath{stroke}%
\end{pgfscope}%
\begin{pgfscope}%
\pgfsetbuttcap%
\pgfsetroundjoin%
\definecolor{currentfill}{rgb}{0.000000,0.000000,0.000000}%
\pgfsetfillcolor{currentfill}%
\pgfsetlinewidth{0.803000pt}%
\definecolor{currentstroke}{rgb}{0.000000,0.000000,0.000000}%
\pgfsetstrokecolor{currentstroke}%
\pgfsetdash{}{0pt}%
\pgfsys@defobject{currentmarker}{\pgfqpoint{-0.048611in}{0.000000in}}{\pgfqpoint{-0.000000in}{0.000000in}}{%
\pgfpathmoveto{\pgfqpoint{-0.000000in}{0.000000in}}%
\pgfpathlineto{\pgfqpoint{-0.048611in}{0.000000in}}%
\pgfusepath{stroke,fill}%
}%
\begin{pgfscope}%
\pgfsys@transformshift{0.572918in}{0.869587in}%
\pgfsys@useobject{currentmarker}{}%
\end{pgfscope}%
\end{pgfscope}%
\begin{pgfscope}%
\definecolor{textcolor}{rgb}{0.000000,0.000000,0.000000}%
\pgfsetstrokecolor{textcolor}%
\pgfsetfillcolor{textcolor}%
\pgftext[x=0.357639in, y=0.831032in, left, base]{\color{textcolor}\rmfamily\fontsize{8.000000}{9.600000}\selectfont \(\displaystyle {15}\)}%
\end{pgfscope}%
\begin{pgfscope}%
\pgfpathrectangle{\pgfqpoint{0.572918in}{0.553781in}}{\pgfqpoint{5.014527in}{2.095553in}}%
\pgfusepath{clip}%
\pgfsetbuttcap%
\pgfsetroundjoin%
\pgfsetlinewidth{0.501875pt}%
\definecolor{currentstroke}{rgb}{0.690196,0.690196,0.690196}%
\pgfsetstrokecolor{currentstroke}%
\pgfsetstrokeopacity{0.500000}%
\pgfsetdash{{0.500000pt}{0.825000pt}}{0.000000pt}%
\pgfpathmoveto{\pgfqpoint{0.572918in}{1.148451in}}%
\pgfpathlineto{\pgfqpoint{5.587445in}{1.148451in}}%
\pgfusepath{stroke}%
\end{pgfscope}%
\begin{pgfscope}%
\pgfsetbuttcap%
\pgfsetroundjoin%
\definecolor{currentfill}{rgb}{0.000000,0.000000,0.000000}%
\pgfsetfillcolor{currentfill}%
\pgfsetlinewidth{0.803000pt}%
\definecolor{currentstroke}{rgb}{0.000000,0.000000,0.000000}%
\pgfsetstrokecolor{currentstroke}%
\pgfsetdash{}{0pt}%
\pgfsys@defobject{currentmarker}{\pgfqpoint{-0.048611in}{0.000000in}}{\pgfqpoint{-0.000000in}{0.000000in}}{%
\pgfpathmoveto{\pgfqpoint{-0.000000in}{0.000000in}}%
\pgfpathlineto{\pgfqpoint{-0.048611in}{0.000000in}}%
\pgfusepath{stroke,fill}%
}%
\begin{pgfscope}%
\pgfsys@transformshift{0.572918in}{1.148451in}%
\pgfsys@useobject{currentmarker}{}%
\end{pgfscope}%
\end{pgfscope}%
\begin{pgfscope}%
\definecolor{textcolor}{rgb}{0.000000,0.000000,0.000000}%
\pgfsetstrokecolor{textcolor}%
\pgfsetfillcolor{textcolor}%
\pgftext[x=0.357639in, y=1.109895in, left, base]{\color{textcolor}\rmfamily\fontsize{8.000000}{9.600000}\selectfont \(\displaystyle {20}\)}%
\end{pgfscope}%
\begin{pgfscope}%
\pgfpathrectangle{\pgfqpoint{0.572918in}{0.553781in}}{\pgfqpoint{5.014527in}{2.095553in}}%
\pgfusepath{clip}%
\pgfsetbuttcap%
\pgfsetroundjoin%
\pgfsetlinewidth{0.501875pt}%
\definecolor{currentstroke}{rgb}{0.690196,0.690196,0.690196}%
\pgfsetstrokecolor{currentstroke}%
\pgfsetstrokeopacity{0.500000}%
\pgfsetdash{{0.500000pt}{0.825000pt}}{0.000000pt}%
\pgfpathmoveto{\pgfqpoint{0.572918in}{1.427314in}}%
\pgfpathlineto{\pgfqpoint{5.587445in}{1.427314in}}%
\pgfusepath{stroke}%
\end{pgfscope}%
\begin{pgfscope}%
\pgfsetbuttcap%
\pgfsetroundjoin%
\definecolor{currentfill}{rgb}{0.000000,0.000000,0.000000}%
\pgfsetfillcolor{currentfill}%
\pgfsetlinewidth{0.803000pt}%
\definecolor{currentstroke}{rgb}{0.000000,0.000000,0.000000}%
\pgfsetstrokecolor{currentstroke}%
\pgfsetdash{}{0pt}%
\pgfsys@defobject{currentmarker}{\pgfqpoint{-0.048611in}{0.000000in}}{\pgfqpoint{-0.000000in}{0.000000in}}{%
\pgfpathmoveto{\pgfqpoint{-0.000000in}{0.000000in}}%
\pgfpathlineto{\pgfqpoint{-0.048611in}{0.000000in}}%
\pgfusepath{stroke,fill}%
}%
\begin{pgfscope}%
\pgfsys@transformshift{0.572918in}{1.427314in}%
\pgfsys@useobject{currentmarker}{}%
\end{pgfscope}%
\end{pgfscope}%
\begin{pgfscope}%
\definecolor{textcolor}{rgb}{0.000000,0.000000,0.000000}%
\pgfsetstrokecolor{textcolor}%
\pgfsetfillcolor{textcolor}%
\pgftext[x=0.357639in, y=1.388759in, left, base]{\color{textcolor}\rmfamily\fontsize{8.000000}{9.600000}\selectfont \(\displaystyle {25}\)}%
\end{pgfscope}%
\begin{pgfscope}%
\pgfpathrectangle{\pgfqpoint{0.572918in}{0.553781in}}{\pgfqpoint{5.014527in}{2.095553in}}%
\pgfusepath{clip}%
\pgfsetbuttcap%
\pgfsetroundjoin%
\pgfsetlinewidth{0.501875pt}%
\definecolor{currentstroke}{rgb}{0.690196,0.690196,0.690196}%
\pgfsetstrokecolor{currentstroke}%
\pgfsetstrokeopacity{0.500000}%
\pgfsetdash{{0.500000pt}{0.825000pt}}{0.000000pt}%
\pgfpathmoveto{\pgfqpoint{0.572918in}{1.706178in}}%
\pgfpathlineto{\pgfqpoint{5.587445in}{1.706178in}}%
\pgfusepath{stroke}%
\end{pgfscope}%
\begin{pgfscope}%
\pgfsetbuttcap%
\pgfsetroundjoin%
\definecolor{currentfill}{rgb}{0.000000,0.000000,0.000000}%
\pgfsetfillcolor{currentfill}%
\pgfsetlinewidth{0.803000pt}%
\definecolor{currentstroke}{rgb}{0.000000,0.000000,0.000000}%
\pgfsetstrokecolor{currentstroke}%
\pgfsetdash{}{0pt}%
\pgfsys@defobject{currentmarker}{\pgfqpoint{-0.048611in}{0.000000in}}{\pgfqpoint{-0.000000in}{0.000000in}}{%
\pgfpathmoveto{\pgfqpoint{-0.000000in}{0.000000in}}%
\pgfpathlineto{\pgfqpoint{-0.048611in}{0.000000in}}%
\pgfusepath{stroke,fill}%
}%
\begin{pgfscope}%
\pgfsys@transformshift{0.572918in}{1.706178in}%
\pgfsys@useobject{currentmarker}{}%
\end{pgfscope}%
\end{pgfscope}%
\begin{pgfscope}%
\definecolor{textcolor}{rgb}{0.000000,0.000000,0.000000}%
\pgfsetstrokecolor{textcolor}%
\pgfsetfillcolor{textcolor}%
\pgftext[x=0.357639in, y=1.667622in, left, base]{\color{textcolor}\rmfamily\fontsize{8.000000}{9.600000}\selectfont \(\displaystyle {30}\)}%
\end{pgfscope}%
\begin{pgfscope}%
\pgfpathrectangle{\pgfqpoint{0.572918in}{0.553781in}}{\pgfqpoint{5.014527in}{2.095553in}}%
\pgfusepath{clip}%
\pgfsetbuttcap%
\pgfsetroundjoin%
\pgfsetlinewidth{0.501875pt}%
\definecolor{currentstroke}{rgb}{0.690196,0.690196,0.690196}%
\pgfsetstrokecolor{currentstroke}%
\pgfsetstrokeopacity{0.500000}%
\pgfsetdash{{0.500000pt}{0.825000pt}}{0.000000pt}%
\pgfpathmoveto{\pgfqpoint{0.572918in}{1.985041in}}%
\pgfpathlineto{\pgfqpoint{5.587445in}{1.985041in}}%
\pgfusepath{stroke}%
\end{pgfscope}%
\begin{pgfscope}%
\pgfsetbuttcap%
\pgfsetroundjoin%
\definecolor{currentfill}{rgb}{0.000000,0.000000,0.000000}%
\pgfsetfillcolor{currentfill}%
\pgfsetlinewidth{0.803000pt}%
\definecolor{currentstroke}{rgb}{0.000000,0.000000,0.000000}%
\pgfsetstrokecolor{currentstroke}%
\pgfsetdash{}{0pt}%
\pgfsys@defobject{currentmarker}{\pgfqpoint{-0.048611in}{0.000000in}}{\pgfqpoint{-0.000000in}{0.000000in}}{%
\pgfpathmoveto{\pgfqpoint{-0.000000in}{0.000000in}}%
\pgfpathlineto{\pgfqpoint{-0.048611in}{0.000000in}}%
\pgfusepath{stroke,fill}%
}%
\begin{pgfscope}%
\pgfsys@transformshift{0.572918in}{1.985041in}%
\pgfsys@useobject{currentmarker}{}%
\end{pgfscope}%
\end{pgfscope}%
\begin{pgfscope}%
\definecolor{textcolor}{rgb}{0.000000,0.000000,0.000000}%
\pgfsetstrokecolor{textcolor}%
\pgfsetfillcolor{textcolor}%
\pgftext[x=0.357639in, y=1.946486in, left, base]{\color{textcolor}\rmfamily\fontsize{8.000000}{9.600000}\selectfont \(\displaystyle {35}\)}%
\end{pgfscope}%
\begin{pgfscope}%
\pgfpathrectangle{\pgfqpoint{0.572918in}{0.553781in}}{\pgfqpoint{5.014527in}{2.095553in}}%
\pgfusepath{clip}%
\pgfsetbuttcap%
\pgfsetroundjoin%
\pgfsetlinewidth{0.501875pt}%
\definecolor{currentstroke}{rgb}{0.690196,0.690196,0.690196}%
\pgfsetstrokecolor{currentstroke}%
\pgfsetstrokeopacity{0.500000}%
\pgfsetdash{{0.500000pt}{0.825000pt}}{0.000000pt}%
\pgfpathmoveto{\pgfqpoint{0.572918in}{2.263905in}}%
\pgfpathlineto{\pgfqpoint{5.587445in}{2.263905in}}%
\pgfusepath{stroke}%
\end{pgfscope}%
\begin{pgfscope}%
\pgfsetbuttcap%
\pgfsetroundjoin%
\definecolor{currentfill}{rgb}{0.000000,0.000000,0.000000}%
\pgfsetfillcolor{currentfill}%
\pgfsetlinewidth{0.803000pt}%
\definecolor{currentstroke}{rgb}{0.000000,0.000000,0.000000}%
\pgfsetstrokecolor{currentstroke}%
\pgfsetdash{}{0pt}%
\pgfsys@defobject{currentmarker}{\pgfqpoint{-0.048611in}{0.000000in}}{\pgfqpoint{-0.000000in}{0.000000in}}{%
\pgfpathmoveto{\pgfqpoint{-0.000000in}{0.000000in}}%
\pgfpathlineto{\pgfqpoint{-0.048611in}{0.000000in}}%
\pgfusepath{stroke,fill}%
}%
\begin{pgfscope}%
\pgfsys@transformshift{0.572918in}{2.263905in}%
\pgfsys@useobject{currentmarker}{}%
\end{pgfscope}%
\end{pgfscope}%
\begin{pgfscope}%
\definecolor{textcolor}{rgb}{0.000000,0.000000,0.000000}%
\pgfsetstrokecolor{textcolor}%
\pgfsetfillcolor{textcolor}%
\pgftext[x=0.357639in, y=2.225349in, left, base]{\color{textcolor}\rmfamily\fontsize{8.000000}{9.600000}\selectfont \(\displaystyle {40}\)}%
\end{pgfscope}%
\begin{pgfscope}%
\pgfpathrectangle{\pgfqpoint{0.572918in}{0.553781in}}{\pgfqpoint{5.014527in}{2.095553in}}%
\pgfusepath{clip}%
\pgfsetbuttcap%
\pgfsetroundjoin%
\pgfsetlinewidth{0.501875pt}%
\definecolor{currentstroke}{rgb}{0.690196,0.690196,0.690196}%
\pgfsetstrokecolor{currentstroke}%
\pgfsetstrokeopacity{0.500000}%
\pgfsetdash{{0.500000pt}{0.825000pt}}{0.000000pt}%
\pgfpathmoveto{\pgfqpoint{0.572918in}{2.542768in}}%
\pgfpathlineto{\pgfqpoint{5.587445in}{2.542768in}}%
\pgfusepath{stroke}%
\end{pgfscope}%
\begin{pgfscope}%
\pgfsetbuttcap%
\pgfsetroundjoin%
\definecolor{currentfill}{rgb}{0.000000,0.000000,0.000000}%
\pgfsetfillcolor{currentfill}%
\pgfsetlinewidth{0.803000pt}%
\definecolor{currentstroke}{rgb}{0.000000,0.000000,0.000000}%
\pgfsetstrokecolor{currentstroke}%
\pgfsetdash{}{0pt}%
\pgfsys@defobject{currentmarker}{\pgfqpoint{-0.048611in}{0.000000in}}{\pgfqpoint{-0.000000in}{0.000000in}}{%
\pgfpathmoveto{\pgfqpoint{-0.000000in}{0.000000in}}%
\pgfpathlineto{\pgfqpoint{-0.048611in}{0.000000in}}%
\pgfusepath{stroke,fill}%
}%
\begin{pgfscope}%
\pgfsys@transformshift{0.572918in}{2.542768in}%
\pgfsys@useobject{currentmarker}{}%
\end{pgfscope}%
\end{pgfscope}%
\begin{pgfscope}%
\definecolor{textcolor}{rgb}{0.000000,0.000000,0.000000}%
\pgfsetstrokecolor{textcolor}%
\pgfsetfillcolor{textcolor}%
\pgftext[x=0.357639in, y=2.504213in, left, base]{\color{textcolor}\rmfamily\fontsize{8.000000}{9.600000}\selectfont \(\displaystyle {45}\)}%
\end{pgfscope}%
\begin{pgfscope}%
\definecolor{textcolor}{rgb}{0.000000,0.000000,0.000000}%
\pgfsetstrokecolor{textcolor}%
\pgfsetfillcolor{textcolor}%
\pgftext[x=0.302083in,y=1.601557in,,bottom,rotate=90.000000]{\color{textcolor}\rmfamily\fontsize{10.950000}{13.140000}\selectfont IQR / 1.349 \(\displaystyle \left[ \textup{deg} \right]\)}%
\end{pgfscope}%
\begin{pgfscope}%
\pgfpathrectangle{\pgfqpoint{0.572918in}{0.553781in}}{\pgfqpoint{5.014527in}{2.095553in}}%
\pgfusepath{clip}%
\pgfsetbuttcap%
\pgfsetroundjoin%
\pgfsetlinewidth{1.505625pt}%
\definecolor{currentstroke}{rgb}{0.313725,0.317647,0.309804}%
\pgfsetstrokecolor{currentstroke}%
\pgfsetstrokeopacity{0.900000}%
\pgfsetdash{}{0pt}%
\pgfpathmoveto{\pgfqpoint{0.828660in}{2.194069in}}%
\pgfpathlineto{\pgfqpoint{1.080374in}{2.194069in}}%
\pgfusepath{stroke}%
\end{pgfscope}%
\begin{pgfscope}%
\pgfpathrectangle{\pgfqpoint{0.572918in}{0.553781in}}{\pgfqpoint{5.014527in}{2.095553in}}%
\pgfusepath{clip}%
\pgfsetbuttcap%
\pgfsetroundjoin%
\pgfsetlinewidth{1.505625pt}%
\definecolor{currentstroke}{rgb}{0.313725,0.317647,0.309804}%
\pgfsetstrokecolor{currentstroke}%
\pgfsetstrokeopacity{0.900000}%
\pgfsetdash{}{0pt}%
\pgfpathmoveto{\pgfqpoint{1.080374in}{2.430491in}}%
\pgfpathlineto{\pgfqpoint{1.332088in}{2.430491in}}%
\pgfusepath{stroke}%
\end{pgfscope}%
\begin{pgfscope}%
\pgfpathrectangle{\pgfqpoint{0.572918in}{0.553781in}}{\pgfqpoint{5.014527in}{2.095553in}}%
\pgfusepath{clip}%
\pgfsetbuttcap%
\pgfsetroundjoin%
\pgfsetlinewidth{1.505625pt}%
\definecolor{currentstroke}{rgb}{0.313725,0.317647,0.309804}%
\pgfsetstrokecolor{currentstroke}%
\pgfsetstrokeopacity{0.900000}%
\pgfsetdash{}{0pt}%
\pgfpathmoveto{\pgfqpoint{1.332088in}{2.375301in}}%
\pgfpathlineto{\pgfqpoint{1.583802in}{2.375301in}}%
\pgfusepath{stroke}%
\end{pgfscope}%
\begin{pgfscope}%
\pgfpathrectangle{\pgfqpoint{0.572918in}{0.553781in}}{\pgfqpoint{5.014527in}{2.095553in}}%
\pgfusepath{clip}%
\pgfsetbuttcap%
\pgfsetroundjoin%
\pgfsetlinewidth{1.505625pt}%
\definecolor{currentstroke}{rgb}{0.313725,0.317647,0.309804}%
\pgfsetstrokecolor{currentstroke}%
\pgfsetstrokeopacity{0.900000}%
\pgfsetdash{}{0pt}%
\pgfpathmoveto{\pgfqpoint{1.583802in}{2.414645in}}%
\pgfpathlineto{\pgfqpoint{1.835516in}{2.414645in}}%
\pgfusepath{stroke}%
\end{pgfscope}%
\begin{pgfscope}%
\pgfpathrectangle{\pgfqpoint{0.572918in}{0.553781in}}{\pgfqpoint{5.014527in}{2.095553in}}%
\pgfusepath{clip}%
\pgfsetbuttcap%
\pgfsetroundjoin%
\pgfsetlinewidth{1.505625pt}%
\definecolor{currentstroke}{rgb}{0.313725,0.317647,0.309804}%
\pgfsetstrokecolor{currentstroke}%
\pgfsetstrokeopacity{0.900000}%
\pgfsetdash{}{0pt}%
\pgfpathmoveto{\pgfqpoint{1.835516in}{2.358447in}}%
\pgfpathlineto{\pgfqpoint{2.087230in}{2.358447in}}%
\pgfusepath{stroke}%
\end{pgfscope}%
\begin{pgfscope}%
\pgfpathrectangle{\pgfqpoint{0.572918in}{0.553781in}}{\pgfqpoint{5.014527in}{2.095553in}}%
\pgfusepath{clip}%
\pgfsetbuttcap%
\pgfsetroundjoin%
\pgfsetlinewidth{1.505625pt}%
\definecolor{currentstroke}{rgb}{0.313725,0.317647,0.309804}%
\pgfsetstrokecolor{currentstroke}%
\pgfsetstrokeopacity{0.900000}%
\pgfsetdash{}{0pt}%
\pgfpathmoveto{\pgfqpoint{2.087230in}{2.348881in}}%
\pgfpathlineto{\pgfqpoint{2.338944in}{2.348881in}}%
\pgfusepath{stroke}%
\end{pgfscope}%
\begin{pgfscope}%
\pgfpathrectangle{\pgfqpoint{0.572918in}{0.553781in}}{\pgfqpoint{5.014527in}{2.095553in}}%
\pgfusepath{clip}%
\pgfsetbuttcap%
\pgfsetroundjoin%
\pgfsetlinewidth{1.505625pt}%
\definecolor{currentstroke}{rgb}{0.313725,0.317647,0.309804}%
\pgfsetstrokecolor{currentstroke}%
\pgfsetstrokeopacity{0.900000}%
\pgfsetdash{}{0pt}%
\pgfpathmoveto{\pgfqpoint{2.338944in}{2.207525in}}%
\pgfpathlineto{\pgfqpoint{2.590658in}{2.207525in}}%
\pgfusepath{stroke}%
\end{pgfscope}%
\begin{pgfscope}%
\pgfpathrectangle{\pgfqpoint{0.572918in}{0.553781in}}{\pgfqpoint{5.014527in}{2.095553in}}%
\pgfusepath{clip}%
\pgfsetbuttcap%
\pgfsetroundjoin%
\pgfsetlinewidth{1.505625pt}%
\definecolor{currentstroke}{rgb}{0.313725,0.317647,0.309804}%
\pgfsetstrokecolor{currentstroke}%
\pgfsetstrokeopacity{0.900000}%
\pgfsetdash{}{0pt}%
\pgfpathmoveto{\pgfqpoint{2.590658in}{2.172889in}}%
\pgfpathlineto{\pgfqpoint{2.842372in}{2.172889in}}%
\pgfusepath{stroke}%
\end{pgfscope}%
\begin{pgfscope}%
\pgfpathrectangle{\pgfqpoint{0.572918in}{0.553781in}}{\pgfqpoint{5.014527in}{2.095553in}}%
\pgfusepath{clip}%
\pgfsetbuttcap%
\pgfsetroundjoin%
\pgfsetlinewidth{1.505625pt}%
\definecolor{currentstroke}{rgb}{0.313725,0.317647,0.309804}%
\pgfsetstrokecolor{currentstroke}%
\pgfsetstrokeopacity{0.900000}%
\pgfsetdash{}{0pt}%
\pgfpathmoveto{\pgfqpoint{2.842372in}{2.082129in}}%
\pgfpathlineto{\pgfqpoint{3.094086in}{2.082129in}}%
\pgfusepath{stroke}%
\end{pgfscope}%
\begin{pgfscope}%
\pgfpathrectangle{\pgfqpoint{0.572918in}{0.553781in}}{\pgfqpoint{5.014527in}{2.095553in}}%
\pgfusepath{clip}%
\pgfsetbuttcap%
\pgfsetroundjoin%
\pgfsetlinewidth{1.505625pt}%
\definecolor{currentstroke}{rgb}{0.313725,0.317647,0.309804}%
\pgfsetstrokecolor{currentstroke}%
\pgfsetstrokeopacity{0.900000}%
\pgfsetdash{}{0pt}%
\pgfpathmoveto{\pgfqpoint{3.094086in}{1.955203in}}%
\pgfpathlineto{\pgfqpoint{3.345800in}{1.955203in}}%
\pgfusepath{stroke}%
\end{pgfscope}%
\begin{pgfscope}%
\pgfpathrectangle{\pgfqpoint{0.572918in}{0.553781in}}{\pgfqpoint{5.014527in}{2.095553in}}%
\pgfusepath{clip}%
\pgfsetbuttcap%
\pgfsetroundjoin%
\pgfsetlinewidth{1.505625pt}%
\definecolor{currentstroke}{rgb}{0.313725,0.317647,0.309804}%
\pgfsetstrokecolor{currentstroke}%
\pgfsetstrokeopacity{0.900000}%
\pgfsetdash{}{0pt}%
\pgfpathmoveto{\pgfqpoint{3.345800in}{1.871549in}}%
\pgfpathlineto{\pgfqpoint{3.597514in}{1.871549in}}%
\pgfusepath{stroke}%
\end{pgfscope}%
\begin{pgfscope}%
\pgfpathrectangle{\pgfqpoint{0.572918in}{0.553781in}}{\pgfqpoint{5.014527in}{2.095553in}}%
\pgfusepath{clip}%
\pgfsetbuttcap%
\pgfsetroundjoin%
\pgfsetlinewidth{1.505625pt}%
\definecolor{currentstroke}{rgb}{0.313725,0.317647,0.309804}%
\pgfsetstrokecolor{currentstroke}%
\pgfsetstrokeopacity{0.900000}%
\pgfsetdash{}{0pt}%
\pgfpathmoveto{\pgfqpoint{3.597514in}{1.712260in}}%
\pgfpathlineto{\pgfqpoint{3.849228in}{1.712260in}}%
\pgfusepath{stroke}%
\end{pgfscope}%
\begin{pgfscope}%
\pgfpathrectangle{\pgfqpoint{0.572918in}{0.553781in}}{\pgfqpoint{5.014527in}{2.095553in}}%
\pgfusepath{clip}%
\pgfsetbuttcap%
\pgfsetroundjoin%
\pgfsetlinewidth{1.505625pt}%
\definecolor{currentstroke}{rgb}{0.313725,0.317647,0.309804}%
\pgfsetstrokecolor{currentstroke}%
\pgfsetstrokeopacity{0.900000}%
\pgfsetdash{}{0pt}%
\pgfpathmoveto{\pgfqpoint{3.849228in}{1.497458in}}%
\pgfpathlineto{\pgfqpoint{4.100942in}{1.497458in}}%
\pgfusepath{stroke}%
\end{pgfscope}%
\begin{pgfscope}%
\pgfpathrectangle{\pgfqpoint{0.572918in}{0.553781in}}{\pgfqpoint{5.014527in}{2.095553in}}%
\pgfusepath{clip}%
\pgfsetbuttcap%
\pgfsetroundjoin%
\pgfsetlinewidth{1.505625pt}%
\definecolor{currentstroke}{rgb}{0.313725,0.317647,0.309804}%
\pgfsetstrokecolor{currentstroke}%
\pgfsetstrokeopacity{0.900000}%
\pgfsetdash{}{0pt}%
\pgfpathmoveto{\pgfqpoint{4.100942in}{1.241989in}}%
\pgfpathlineto{\pgfqpoint{4.352656in}{1.241989in}}%
\pgfusepath{stroke}%
\end{pgfscope}%
\begin{pgfscope}%
\pgfpathrectangle{\pgfqpoint{0.572918in}{0.553781in}}{\pgfqpoint{5.014527in}{2.095553in}}%
\pgfusepath{clip}%
\pgfsetbuttcap%
\pgfsetroundjoin%
\pgfsetlinewidth{1.505625pt}%
\definecolor{currentstroke}{rgb}{0.313725,0.317647,0.309804}%
\pgfsetstrokecolor{currentstroke}%
\pgfsetstrokeopacity{0.900000}%
\pgfsetdash{}{0pt}%
\pgfpathmoveto{\pgfqpoint{4.352656in}{1.001450in}}%
\pgfpathlineto{\pgfqpoint{4.604370in}{1.001450in}}%
\pgfusepath{stroke}%
\end{pgfscope}%
\begin{pgfscope}%
\pgfpathrectangle{\pgfqpoint{0.572918in}{0.553781in}}{\pgfqpoint{5.014527in}{2.095553in}}%
\pgfusepath{clip}%
\pgfsetbuttcap%
\pgfsetroundjoin%
\pgfsetlinewidth{1.505625pt}%
\definecolor{currentstroke}{rgb}{0.313725,0.317647,0.309804}%
\pgfsetstrokecolor{currentstroke}%
\pgfsetstrokeopacity{0.900000}%
\pgfsetdash{}{0pt}%
\pgfpathmoveto{\pgfqpoint{4.604370in}{0.800305in}}%
\pgfpathlineto{\pgfqpoint{4.856084in}{0.800305in}}%
\pgfusepath{stroke}%
\end{pgfscope}%
\begin{pgfscope}%
\pgfpathrectangle{\pgfqpoint{0.572918in}{0.553781in}}{\pgfqpoint{5.014527in}{2.095553in}}%
\pgfusepath{clip}%
\pgfsetbuttcap%
\pgfsetroundjoin%
\pgfsetlinewidth{1.505625pt}%
\definecolor{currentstroke}{rgb}{0.313725,0.317647,0.309804}%
\pgfsetstrokecolor{currentstroke}%
\pgfsetstrokeopacity{0.900000}%
\pgfsetdash{}{0pt}%
\pgfpathmoveto{\pgfqpoint{4.856084in}{0.712397in}}%
\pgfpathlineto{\pgfqpoint{5.107798in}{0.712397in}}%
\pgfusepath{stroke}%
\end{pgfscope}%
\begin{pgfscope}%
\pgfpathrectangle{\pgfqpoint{0.572918in}{0.553781in}}{\pgfqpoint{5.014527in}{2.095553in}}%
\pgfusepath{clip}%
\pgfsetbuttcap%
\pgfsetroundjoin%
\pgfsetlinewidth{1.505625pt}%
\definecolor{currentstroke}{rgb}{0.313725,0.317647,0.309804}%
\pgfsetstrokecolor{currentstroke}%
\pgfsetstrokeopacity{0.900000}%
\pgfsetdash{}{0pt}%
\pgfpathmoveto{\pgfqpoint{5.107798in}{0.664962in}}%
\pgfpathlineto{\pgfqpoint{5.359512in}{0.664962in}}%
\pgfusepath{stroke}%
\end{pgfscope}%
\begin{pgfscope}%
\pgfpathrectangle{\pgfqpoint{0.572918in}{0.553781in}}{\pgfqpoint{5.014527in}{2.095553in}}%
\pgfusepath{clip}%
\pgfsetbuttcap%
\pgfsetroundjoin%
\pgfsetlinewidth{1.505625pt}%
\definecolor{currentstroke}{rgb}{0.313725,0.317647,0.309804}%
\pgfsetstrokecolor{currentstroke}%
\pgfsetstrokeopacity{0.900000}%
\pgfsetdash{}{0pt}%
\pgfpathmoveto{\pgfqpoint{0.954517in}{1.946285in}}%
\pgfpathlineto{\pgfqpoint{0.954517in}{2.427244in}}%
\pgfusepath{stroke}%
\end{pgfscope}%
\begin{pgfscope}%
\pgfpathrectangle{\pgfqpoint{0.572918in}{0.553781in}}{\pgfqpoint{5.014527in}{2.095553in}}%
\pgfusepath{clip}%
\pgfsetbuttcap%
\pgfsetroundjoin%
\pgfsetlinewidth{1.505625pt}%
\definecolor{currentstroke}{rgb}{0.313725,0.317647,0.309804}%
\pgfsetstrokecolor{currentstroke}%
\pgfsetstrokeopacity{0.900000}%
\pgfsetdash{}{0pt}%
\pgfpathmoveto{\pgfqpoint{1.206231in}{2.305153in}}%
\pgfpathlineto{\pgfqpoint{1.206231in}{2.554081in}}%
\pgfusepath{stroke}%
\end{pgfscope}%
\begin{pgfscope}%
\pgfpathrectangle{\pgfqpoint{0.572918in}{0.553781in}}{\pgfqpoint{5.014527in}{2.095553in}}%
\pgfusepath{clip}%
\pgfsetbuttcap%
\pgfsetroundjoin%
\pgfsetlinewidth{1.505625pt}%
\definecolor{currentstroke}{rgb}{0.313725,0.317647,0.309804}%
\pgfsetstrokecolor{currentstroke}%
\pgfsetstrokeopacity{0.900000}%
\pgfsetdash{}{0pt}%
\pgfpathmoveto{\pgfqpoint{1.457945in}{2.269811in}}%
\pgfpathlineto{\pgfqpoint{1.457945in}{2.471591in}}%
\pgfusepath{stroke}%
\end{pgfscope}%
\begin{pgfscope}%
\pgfpathrectangle{\pgfqpoint{0.572918in}{0.553781in}}{\pgfqpoint{5.014527in}{2.095553in}}%
\pgfusepath{clip}%
\pgfsetbuttcap%
\pgfsetroundjoin%
\pgfsetlinewidth{1.505625pt}%
\definecolor{currentstroke}{rgb}{0.313725,0.317647,0.309804}%
\pgfsetstrokecolor{currentstroke}%
\pgfsetstrokeopacity{0.900000}%
\pgfsetdash{}{0pt}%
\pgfpathmoveto{\pgfqpoint{1.709659in}{2.354016in}}%
\pgfpathlineto{\pgfqpoint{1.709659in}{2.475906in}}%
\pgfusepath{stroke}%
\end{pgfscope}%
\begin{pgfscope}%
\pgfpathrectangle{\pgfqpoint{0.572918in}{0.553781in}}{\pgfqpoint{5.014527in}{2.095553in}}%
\pgfusepath{clip}%
\pgfsetbuttcap%
\pgfsetroundjoin%
\pgfsetlinewidth{1.505625pt}%
\definecolor{currentstroke}{rgb}{0.313725,0.317647,0.309804}%
\pgfsetstrokecolor{currentstroke}%
\pgfsetstrokeopacity{0.900000}%
\pgfsetdash{}{0pt}%
\pgfpathmoveto{\pgfqpoint{1.961373in}{2.314150in}}%
\pgfpathlineto{\pgfqpoint{1.961373in}{2.400930in}}%
\pgfusepath{stroke}%
\end{pgfscope}%
\begin{pgfscope}%
\pgfpathrectangle{\pgfqpoint{0.572918in}{0.553781in}}{\pgfqpoint{5.014527in}{2.095553in}}%
\pgfusepath{clip}%
\pgfsetbuttcap%
\pgfsetroundjoin%
\pgfsetlinewidth{1.505625pt}%
\definecolor{currentstroke}{rgb}{0.313725,0.317647,0.309804}%
\pgfsetstrokecolor{currentstroke}%
\pgfsetstrokeopacity{0.900000}%
\pgfsetdash{}{0pt}%
\pgfpathmoveto{\pgfqpoint{2.213087in}{2.308052in}}%
\pgfpathlineto{\pgfqpoint{2.213087in}{2.393735in}}%
\pgfusepath{stroke}%
\end{pgfscope}%
\begin{pgfscope}%
\pgfpathrectangle{\pgfqpoint{0.572918in}{0.553781in}}{\pgfqpoint{5.014527in}{2.095553in}}%
\pgfusepath{clip}%
\pgfsetbuttcap%
\pgfsetroundjoin%
\pgfsetlinewidth{1.505625pt}%
\definecolor{currentstroke}{rgb}{0.313725,0.317647,0.309804}%
\pgfsetstrokecolor{currentstroke}%
\pgfsetstrokeopacity{0.900000}%
\pgfsetdash{}{0pt}%
\pgfpathmoveto{\pgfqpoint{2.464801in}{2.175154in}}%
\pgfpathlineto{\pgfqpoint{2.464801in}{2.242290in}}%
\pgfusepath{stroke}%
\end{pgfscope}%
\begin{pgfscope}%
\pgfpathrectangle{\pgfqpoint{0.572918in}{0.553781in}}{\pgfqpoint{5.014527in}{2.095553in}}%
\pgfusepath{clip}%
\pgfsetbuttcap%
\pgfsetroundjoin%
\pgfsetlinewidth{1.505625pt}%
\definecolor{currentstroke}{rgb}{0.313725,0.317647,0.309804}%
\pgfsetstrokecolor{currentstroke}%
\pgfsetstrokeopacity{0.900000}%
\pgfsetdash{}{0pt}%
\pgfpathmoveto{\pgfqpoint{2.716515in}{2.137636in}}%
\pgfpathlineto{\pgfqpoint{2.716515in}{2.210373in}}%
\pgfusepath{stroke}%
\end{pgfscope}%
\begin{pgfscope}%
\pgfpathrectangle{\pgfqpoint{0.572918in}{0.553781in}}{\pgfqpoint{5.014527in}{2.095553in}}%
\pgfusepath{clip}%
\pgfsetbuttcap%
\pgfsetroundjoin%
\pgfsetlinewidth{1.505625pt}%
\definecolor{currentstroke}{rgb}{0.313725,0.317647,0.309804}%
\pgfsetstrokecolor{currentstroke}%
\pgfsetstrokeopacity{0.900000}%
\pgfsetdash{}{0pt}%
\pgfpathmoveto{\pgfqpoint{2.968229in}{2.051705in}}%
\pgfpathlineto{\pgfqpoint{2.968229in}{2.118291in}}%
\pgfusepath{stroke}%
\end{pgfscope}%
\begin{pgfscope}%
\pgfpathrectangle{\pgfqpoint{0.572918in}{0.553781in}}{\pgfqpoint{5.014527in}{2.095553in}}%
\pgfusepath{clip}%
\pgfsetbuttcap%
\pgfsetroundjoin%
\pgfsetlinewidth{1.505625pt}%
\definecolor{currentstroke}{rgb}{0.313725,0.317647,0.309804}%
\pgfsetstrokecolor{currentstroke}%
\pgfsetstrokeopacity{0.900000}%
\pgfsetdash{}{0pt}%
\pgfpathmoveto{\pgfqpoint{3.219943in}{1.922966in}}%
\pgfpathlineto{\pgfqpoint{3.219943in}{1.986759in}}%
\pgfusepath{stroke}%
\end{pgfscope}%
\begin{pgfscope}%
\pgfpathrectangle{\pgfqpoint{0.572918in}{0.553781in}}{\pgfqpoint{5.014527in}{2.095553in}}%
\pgfusepath{clip}%
\pgfsetbuttcap%
\pgfsetroundjoin%
\pgfsetlinewidth{1.505625pt}%
\definecolor{currentstroke}{rgb}{0.313725,0.317647,0.309804}%
\pgfsetstrokecolor{currentstroke}%
\pgfsetstrokeopacity{0.900000}%
\pgfsetdash{}{0pt}%
\pgfpathmoveto{\pgfqpoint{3.471657in}{1.843395in}}%
\pgfpathlineto{\pgfqpoint{3.471657in}{1.897531in}}%
\pgfusepath{stroke}%
\end{pgfscope}%
\begin{pgfscope}%
\pgfpathrectangle{\pgfqpoint{0.572918in}{0.553781in}}{\pgfqpoint{5.014527in}{2.095553in}}%
\pgfusepath{clip}%
\pgfsetbuttcap%
\pgfsetroundjoin%
\pgfsetlinewidth{1.505625pt}%
\definecolor{currentstroke}{rgb}{0.313725,0.317647,0.309804}%
\pgfsetstrokecolor{currentstroke}%
\pgfsetstrokeopacity{0.900000}%
\pgfsetdash{}{0pt}%
\pgfpathmoveto{\pgfqpoint{3.723371in}{1.681123in}}%
\pgfpathlineto{\pgfqpoint{3.723371in}{1.743904in}}%
\pgfusepath{stroke}%
\end{pgfscope}%
\begin{pgfscope}%
\pgfpathrectangle{\pgfqpoint{0.572918in}{0.553781in}}{\pgfqpoint{5.014527in}{2.095553in}}%
\pgfusepath{clip}%
\pgfsetbuttcap%
\pgfsetroundjoin%
\pgfsetlinewidth{1.505625pt}%
\definecolor{currentstroke}{rgb}{0.313725,0.317647,0.309804}%
\pgfsetstrokecolor{currentstroke}%
\pgfsetstrokeopacity{0.900000}%
\pgfsetdash{}{0pt}%
\pgfpathmoveto{\pgfqpoint{3.975085in}{1.471241in}}%
\pgfpathlineto{\pgfqpoint{3.975085in}{1.529112in}}%
\pgfusepath{stroke}%
\end{pgfscope}%
\begin{pgfscope}%
\pgfpathrectangle{\pgfqpoint{0.572918in}{0.553781in}}{\pgfqpoint{5.014527in}{2.095553in}}%
\pgfusepath{clip}%
\pgfsetbuttcap%
\pgfsetroundjoin%
\pgfsetlinewidth{1.505625pt}%
\definecolor{currentstroke}{rgb}{0.313725,0.317647,0.309804}%
\pgfsetstrokecolor{currentstroke}%
\pgfsetstrokeopacity{0.900000}%
\pgfsetdash{}{0pt}%
\pgfpathmoveto{\pgfqpoint{4.226799in}{1.219092in}}%
\pgfpathlineto{\pgfqpoint{4.226799in}{1.268141in}}%
\pgfusepath{stroke}%
\end{pgfscope}%
\begin{pgfscope}%
\pgfpathrectangle{\pgfqpoint{0.572918in}{0.553781in}}{\pgfqpoint{5.014527in}{2.095553in}}%
\pgfusepath{clip}%
\pgfsetbuttcap%
\pgfsetroundjoin%
\pgfsetlinewidth{1.505625pt}%
\definecolor{currentstroke}{rgb}{0.313725,0.317647,0.309804}%
\pgfsetstrokecolor{currentstroke}%
\pgfsetstrokeopacity{0.900000}%
\pgfsetdash{}{0pt}%
\pgfpathmoveto{\pgfqpoint{4.478513in}{0.977939in}}%
\pgfpathlineto{\pgfqpoint{4.478513in}{1.026796in}}%
\pgfusepath{stroke}%
\end{pgfscope}%
\begin{pgfscope}%
\pgfpathrectangle{\pgfqpoint{0.572918in}{0.553781in}}{\pgfqpoint{5.014527in}{2.095553in}}%
\pgfusepath{clip}%
\pgfsetbuttcap%
\pgfsetroundjoin%
\pgfsetlinewidth{1.505625pt}%
\definecolor{currentstroke}{rgb}{0.313725,0.317647,0.309804}%
\pgfsetstrokecolor{currentstroke}%
\pgfsetstrokeopacity{0.900000}%
\pgfsetdash{}{0pt}%
\pgfpathmoveto{\pgfqpoint{4.730227in}{0.780070in}}%
\pgfpathlineto{\pgfqpoint{4.730227in}{0.822928in}}%
\pgfusepath{stroke}%
\end{pgfscope}%
\begin{pgfscope}%
\pgfpathrectangle{\pgfqpoint{0.572918in}{0.553781in}}{\pgfqpoint{5.014527in}{2.095553in}}%
\pgfusepath{clip}%
\pgfsetbuttcap%
\pgfsetroundjoin%
\pgfsetlinewidth{1.505625pt}%
\definecolor{currentstroke}{rgb}{0.313725,0.317647,0.309804}%
\pgfsetstrokecolor{currentstroke}%
\pgfsetstrokeopacity{0.900000}%
\pgfsetdash{}{0pt}%
\pgfpathmoveto{\pgfqpoint{4.981941in}{0.693544in}}%
\pgfpathlineto{\pgfqpoint{4.981941in}{0.733053in}}%
\pgfusepath{stroke}%
\end{pgfscope}%
\begin{pgfscope}%
\pgfpathrectangle{\pgfqpoint{0.572918in}{0.553781in}}{\pgfqpoint{5.014527in}{2.095553in}}%
\pgfusepath{clip}%
\pgfsetbuttcap%
\pgfsetroundjoin%
\pgfsetlinewidth{1.505625pt}%
\definecolor{currentstroke}{rgb}{0.313725,0.317647,0.309804}%
\pgfsetstrokecolor{currentstroke}%
\pgfsetstrokeopacity{0.900000}%
\pgfsetdash{}{0pt}%
\pgfpathmoveto{\pgfqpoint{5.233655in}{0.649033in}}%
\pgfpathlineto{\pgfqpoint{5.233655in}{0.682423in}}%
\pgfusepath{stroke}%
\end{pgfscope}%
\begin{pgfscope}%
\pgfpathrectangle{\pgfqpoint{0.572918in}{0.553781in}}{\pgfqpoint{5.014527in}{2.095553in}}%
\pgfusepath{clip}%
\pgfsetbuttcap%
\pgfsetroundjoin%
\definecolor{currentfill}{rgb}{0.313725,0.317647,0.309804}%
\pgfsetfillcolor{currentfill}%
\pgfsetfillopacity{0.900000}%
\pgfsetlinewidth{1.003750pt}%
\definecolor{currentstroke}{rgb}{0.313725,0.317647,0.309804}%
\pgfsetstrokecolor{currentstroke}%
\pgfsetstrokeopacity{0.900000}%
\pgfsetdash{}{0pt}%
\pgfsys@defobject{currentmarker}{\pgfqpoint{0.000000in}{-0.013889in}}{\pgfqpoint{0.000000in}{0.013889in}}{%
\pgfpathmoveto{\pgfqpoint{0.000000in}{-0.013889in}}%
\pgfpathlineto{\pgfqpoint{0.000000in}{0.013889in}}%
\pgfusepath{stroke,fill}%
}%
\begin{pgfscope}%
\pgfsys@transformshift{0.828660in}{2.194069in}%
\pgfsys@useobject{currentmarker}{}%
\end{pgfscope}%
\begin{pgfscope}%
\pgfsys@transformshift{1.080374in}{2.430491in}%
\pgfsys@useobject{currentmarker}{}%
\end{pgfscope}%
\begin{pgfscope}%
\pgfsys@transformshift{1.332088in}{2.375301in}%
\pgfsys@useobject{currentmarker}{}%
\end{pgfscope}%
\begin{pgfscope}%
\pgfsys@transformshift{1.583802in}{2.414645in}%
\pgfsys@useobject{currentmarker}{}%
\end{pgfscope}%
\begin{pgfscope}%
\pgfsys@transformshift{1.835516in}{2.358447in}%
\pgfsys@useobject{currentmarker}{}%
\end{pgfscope}%
\begin{pgfscope}%
\pgfsys@transformshift{2.087230in}{2.348881in}%
\pgfsys@useobject{currentmarker}{}%
\end{pgfscope}%
\begin{pgfscope}%
\pgfsys@transformshift{2.338944in}{2.207525in}%
\pgfsys@useobject{currentmarker}{}%
\end{pgfscope}%
\begin{pgfscope}%
\pgfsys@transformshift{2.590658in}{2.172889in}%
\pgfsys@useobject{currentmarker}{}%
\end{pgfscope}%
\begin{pgfscope}%
\pgfsys@transformshift{2.842372in}{2.082129in}%
\pgfsys@useobject{currentmarker}{}%
\end{pgfscope}%
\begin{pgfscope}%
\pgfsys@transformshift{3.094086in}{1.955203in}%
\pgfsys@useobject{currentmarker}{}%
\end{pgfscope}%
\begin{pgfscope}%
\pgfsys@transformshift{3.345800in}{1.871549in}%
\pgfsys@useobject{currentmarker}{}%
\end{pgfscope}%
\begin{pgfscope}%
\pgfsys@transformshift{3.597514in}{1.712260in}%
\pgfsys@useobject{currentmarker}{}%
\end{pgfscope}%
\begin{pgfscope}%
\pgfsys@transformshift{3.849228in}{1.497458in}%
\pgfsys@useobject{currentmarker}{}%
\end{pgfscope}%
\begin{pgfscope}%
\pgfsys@transformshift{4.100942in}{1.241989in}%
\pgfsys@useobject{currentmarker}{}%
\end{pgfscope}%
\begin{pgfscope}%
\pgfsys@transformshift{4.352656in}{1.001450in}%
\pgfsys@useobject{currentmarker}{}%
\end{pgfscope}%
\begin{pgfscope}%
\pgfsys@transformshift{4.604370in}{0.800305in}%
\pgfsys@useobject{currentmarker}{}%
\end{pgfscope}%
\begin{pgfscope}%
\pgfsys@transformshift{4.856084in}{0.712397in}%
\pgfsys@useobject{currentmarker}{}%
\end{pgfscope}%
\begin{pgfscope}%
\pgfsys@transformshift{5.107798in}{0.664962in}%
\pgfsys@useobject{currentmarker}{}%
\end{pgfscope}%
\end{pgfscope}%
\begin{pgfscope}%
\pgfpathrectangle{\pgfqpoint{0.572918in}{0.553781in}}{\pgfqpoint{5.014527in}{2.095553in}}%
\pgfusepath{clip}%
\pgfsetbuttcap%
\pgfsetroundjoin%
\definecolor{currentfill}{rgb}{0.313725,0.317647,0.309804}%
\pgfsetfillcolor{currentfill}%
\pgfsetfillopacity{0.900000}%
\pgfsetlinewidth{1.003750pt}%
\definecolor{currentstroke}{rgb}{0.313725,0.317647,0.309804}%
\pgfsetstrokecolor{currentstroke}%
\pgfsetstrokeopacity{0.900000}%
\pgfsetdash{}{0pt}%
\pgfsys@defobject{currentmarker}{\pgfqpoint{0.000000in}{-0.013889in}}{\pgfqpoint{0.000000in}{0.013889in}}{%
\pgfpathmoveto{\pgfqpoint{0.000000in}{-0.013889in}}%
\pgfpathlineto{\pgfqpoint{0.000000in}{0.013889in}}%
\pgfusepath{stroke,fill}%
}%
\begin{pgfscope}%
\pgfsys@transformshift{1.080374in}{2.194069in}%
\pgfsys@useobject{currentmarker}{}%
\end{pgfscope}%
\begin{pgfscope}%
\pgfsys@transformshift{1.332088in}{2.430491in}%
\pgfsys@useobject{currentmarker}{}%
\end{pgfscope}%
\begin{pgfscope}%
\pgfsys@transformshift{1.583802in}{2.375301in}%
\pgfsys@useobject{currentmarker}{}%
\end{pgfscope}%
\begin{pgfscope}%
\pgfsys@transformshift{1.835516in}{2.414645in}%
\pgfsys@useobject{currentmarker}{}%
\end{pgfscope}%
\begin{pgfscope}%
\pgfsys@transformshift{2.087230in}{2.358447in}%
\pgfsys@useobject{currentmarker}{}%
\end{pgfscope}%
\begin{pgfscope}%
\pgfsys@transformshift{2.338944in}{2.348881in}%
\pgfsys@useobject{currentmarker}{}%
\end{pgfscope}%
\begin{pgfscope}%
\pgfsys@transformshift{2.590658in}{2.207525in}%
\pgfsys@useobject{currentmarker}{}%
\end{pgfscope}%
\begin{pgfscope}%
\pgfsys@transformshift{2.842372in}{2.172889in}%
\pgfsys@useobject{currentmarker}{}%
\end{pgfscope}%
\begin{pgfscope}%
\pgfsys@transformshift{3.094086in}{2.082129in}%
\pgfsys@useobject{currentmarker}{}%
\end{pgfscope}%
\begin{pgfscope}%
\pgfsys@transformshift{3.345800in}{1.955203in}%
\pgfsys@useobject{currentmarker}{}%
\end{pgfscope}%
\begin{pgfscope}%
\pgfsys@transformshift{3.597514in}{1.871549in}%
\pgfsys@useobject{currentmarker}{}%
\end{pgfscope}%
\begin{pgfscope}%
\pgfsys@transformshift{3.849228in}{1.712260in}%
\pgfsys@useobject{currentmarker}{}%
\end{pgfscope}%
\begin{pgfscope}%
\pgfsys@transformshift{4.100942in}{1.497458in}%
\pgfsys@useobject{currentmarker}{}%
\end{pgfscope}%
\begin{pgfscope}%
\pgfsys@transformshift{4.352656in}{1.241989in}%
\pgfsys@useobject{currentmarker}{}%
\end{pgfscope}%
\begin{pgfscope}%
\pgfsys@transformshift{4.604370in}{1.001450in}%
\pgfsys@useobject{currentmarker}{}%
\end{pgfscope}%
\begin{pgfscope}%
\pgfsys@transformshift{4.856084in}{0.800305in}%
\pgfsys@useobject{currentmarker}{}%
\end{pgfscope}%
\begin{pgfscope}%
\pgfsys@transformshift{5.107798in}{0.712397in}%
\pgfsys@useobject{currentmarker}{}%
\end{pgfscope}%
\begin{pgfscope}%
\pgfsys@transformshift{5.359512in}{0.664962in}%
\pgfsys@useobject{currentmarker}{}%
\end{pgfscope}%
\end{pgfscope}%
\begin{pgfscope}%
\pgfpathrectangle{\pgfqpoint{0.572918in}{0.553781in}}{\pgfqpoint{5.014527in}{2.095553in}}%
\pgfusepath{clip}%
\pgfsetbuttcap%
\pgfsetroundjoin%
\definecolor{currentfill}{rgb}{0.313725,0.317647,0.309804}%
\pgfsetfillcolor{currentfill}%
\pgfsetfillopacity{0.900000}%
\pgfsetlinewidth{1.003750pt}%
\definecolor{currentstroke}{rgb}{0.313725,0.317647,0.309804}%
\pgfsetstrokecolor{currentstroke}%
\pgfsetstrokeopacity{0.900000}%
\pgfsetdash{}{0pt}%
\pgfsys@defobject{currentmarker}{\pgfqpoint{-0.013889in}{-0.000000in}}{\pgfqpoint{0.013889in}{0.000000in}}{%
\pgfpathmoveto{\pgfqpoint{0.013889in}{-0.000000in}}%
\pgfpathlineto{\pgfqpoint{-0.013889in}{0.000000in}}%
\pgfusepath{stroke,fill}%
}%
\begin{pgfscope}%
\pgfsys@transformshift{0.954517in}{1.946285in}%
\pgfsys@useobject{currentmarker}{}%
\end{pgfscope}%
\begin{pgfscope}%
\pgfsys@transformshift{1.206231in}{2.305153in}%
\pgfsys@useobject{currentmarker}{}%
\end{pgfscope}%
\begin{pgfscope}%
\pgfsys@transformshift{1.457945in}{2.269811in}%
\pgfsys@useobject{currentmarker}{}%
\end{pgfscope}%
\begin{pgfscope}%
\pgfsys@transformshift{1.709659in}{2.354016in}%
\pgfsys@useobject{currentmarker}{}%
\end{pgfscope}%
\begin{pgfscope}%
\pgfsys@transformshift{1.961373in}{2.314150in}%
\pgfsys@useobject{currentmarker}{}%
\end{pgfscope}%
\begin{pgfscope}%
\pgfsys@transformshift{2.213087in}{2.308052in}%
\pgfsys@useobject{currentmarker}{}%
\end{pgfscope}%
\begin{pgfscope}%
\pgfsys@transformshift{2.464801in}{2.175154in}%
\pgfsys@useobject{currentmarker}{}%
\end{pgfscope}%
\begin{pgfscope}%
\pgfsys@transformshift{2.716515in}{2.137636in}%
\pgfsys@useobject{currentmarker}{}%
\end{pgfscope}%
\begin{pgfscope}%
\pgfsys@transformshift{2.968229in}{2.051705in}%
\pgfsys@useobject{currentmarker}{}%
\end{pgfscope}%
\begin{pgfscope}%
\pgfsys@transformshift{3.219943in}{1.922966in}%
\pgfsys@useobject{currentmarker}{}%
\end{pgfscope}%
\begin{pgfscope}%
\pgfsys@transformshift{3.471657in}{1.843395in}%
\pgfsys@useobject{currentmarker}{}%
\end{pgfscope}%
\begin{pgfscope}%
\pgfsys@transformshift{3.723371in}{1.681123in}%
\pgfsys@useobject{currentmarker}{}%
\end{pgfscope}%
\begin{pgfscope}%
\pgfsys@transformshift{3.975085in}{1.471241in}%
\pgfsys@useobject{currentmarker}{}%
\end{pgfscope}%
\begin{pgfscope}%
\pgfsys@transformshift{4.226799in}{1.219092in}%
\pgfsys@useobject{currentmarker}{}%
\end{pgfscope}%
\begin{pgfscope}%
\pgfsys@transformshift{4.478513in}{0.977939in}%
\pgfsys@useobject{currentmarker}{}%
\end{pgfscope}%
\begin{pgfscope}%
\pgfsys@transformshift{4.730227in}{0.780070in}%
\pgfsys@useobject{currentmarker}{}%
\end{pgfscope}%
\begin{pgfscope}%
\pgfsys@transformshift{4.981941in}{0.693544in}%
\pgfsys@useobject{currentmarker}{}%
\end{pgfscope}%
\begin{pgfscope}%
\pgfsys@transformshift{5.233655in}{0.649033in}%
\pgfsys@useobject{currentmarker}{}%
\end{pgfscope}%
\end{pgfscope}%
\begin{pgfscope}%
\pgfpathrectangle{\pgfqpoint{0.572918in}{0.553781in}}{\pgfqpoint{5.014527in}{2.095553in}}%
\pgfusepath{clip}%
\pgfsetbuttcap%
\pgfsetroundjoin%
\definecolor{currentfill}{rgb}{0.313725,0.317647,0.309804}%
\pgfsetfillcolor{currentfill}%
\pgfsetfillopacity{0.900000}%
\pgfsetlinewidth{1.003750pt}%
\definecolor{currentstroke}{rgb}{0.313725,0.317647,0.309804}%
\pgfsetstrokecolor{currentstroke}%
\pgfsetstrokeopacity{0.900000}%
\pgfsetdash{}{0pt}%
\pgfsys@defobject{currentmarker}{\pgfqpoint{-0.013889in}{-0.000000in}}{\pgfqpoint{0.013889in}{0.000000in}}{%
\pgfpathmoveto{\pgfqpoint{0.013889in}{-0.000000in}}%
\pgfpathlineto{\pgfqpoint{-0.013889in}{0.000000in}}%
\pgfusepath{stroke,fill}%
}%
\begin{pgfscope}%
\pgfsys@transformshift{0.954517in}{2.427244in}%
\pgfsys@useobject{currentmarker}{}%
\end{pgfscope}%
\begin{pgfscope}%
\pgfsys@transformshift{1.206231in}{2.554081in}%
\pgfsys@useobject{currentmarker}{}%
\end{pgfscope}%
\begin{pgfscope}%
\pgfsys@transformshift{1.457945in}{2.471591in}%
\pgfsys@useobject{currentmarker}{}%
\end{pgfscope}%
\begin{pgfscope}%
\pgfsys@transformshift{1.709659in}{2.475906in}%
\pgfsys@useobject{currentmarker}{}%
\end{pgfscope}%
\begin{pgfscope}%
\pgfsys@transformshift{1.961373in}{2.400930in}%
\pgfsys@useobject{currentmarker}{}%
\end{pgfscope}%
\begin{pgfscope}%
\pgfsys@transformshift{2.213087in}{2.393735in}%
\pgfsys@useobject{currentmarker}{}%
\end{pgfscope}%
\begin{pgfscope}%
\pgfsys@transformshift{2.464801in}{2.242290in}%
\pgfsys@useobject{currentmarker}{}%
\end{pgfscope}%
\begin{pgfscope}%
\pgfsys@transformshift{2.716515in}{2.210373in}%
\pgfsys@useobject{currentmarker}{}%
\end{pgfscope}%
\begin{pgfscope}%
\pgfsys@transformshift{2.968229in}{2.118291in}%
\pgfsys@useobject{currentmarker}{}%
\end{pgfscope}%
\begin{pgfscope}%
\pgfsys@transformshift{3.219943in}{1.986759in}%
\pgfsys@useobject{currentmarker}{}%
\end{pgfscope}%
\begin{pgfscope}%
\pgfsys@transformshift{3.471657in}{1.897531in}%
\pgfsys@useobject{currentmarker}{}%
\end{pgfscope}%
\begin{pgfscope}%
\pgfsys@transformshift{3.723371in}{1.743904in}%
\pgfsys@useobject{currentmarker}{}%
\end{pgfscope}%
\begin{pgfscope}%
\pgfsys@transformshift{3.975085in}{1.529112in}%
\pgfsys@useobject{currentmarker}{}%
\end{pgfscope}%
\begin{pgfscope}%
\pgfsys@transformshift{4.226799in}{1.268141in}%
\pgfsys@useobject{currentmarker}{}%
\end{pgfscope}%
\begin{pgfscope}%
\pgfsys@transformshift{4.478513in}{1.026796in}%
\pgfsys@useobject{currentmarker}{}%
\end{pgfscope}%
\begin{pgfscope}%
\pgfsys@transformshift{4.730227in}{0.822928in}%
\pgfsys@useobject{currentmarker}{}%
\end{pgfscope}%
\begin{pgfscope}%
\pgfsys@transformshift{4.981941in}{0.733053in}%
\pgfsys@useobject{currentmarker}{}%
\end{pgfscope}%
\begin{pgfscope}%
\pgfsys@transformshift{5.233655in}{0.682423in}%
\pgfsys@useobject{currentmarker}{}%
\end{pgfscope}%
\end{pgfscope}%
\begin{pgfscope}%
\pgfsetrectcap%
\pgfsetmiterjoin%
\pgfsetlinewidth{0.803000pt}%
\definecolor{currentstroke}{rgb}{0.000000,0.000000,0.000000}%
\pgfsetstrokecolor{currentstroke}%
\pgfsetdash{}{0pt}%
\pgfpathmoveto{\pgfqpoint{0.572918in}{0.553781in}}%
\pgfpathlineto{\pgfqpoint{0.572918in}{2.649333in}}%
\pgfusepath{stroke}%
\end{pgfscope}%
\begin{pgfscope}%
\pgfsetrectcap%
\pgfsetmiterjoin%
\pgfsetlinewidth{0.803000pt}%
\definecolor{currentstroke}{rgb}{0.000000,0.000000,0.000000}%
\pgfsetstrokecolor{currentstroke}%
\pgfsetdash{}{0pt}%
\pgfpathmoveto{\pgfqpoint{5.587445in}{0.553781in}}%
\pgfpathlineto{\pgfqpoint{5.587445in}{2.649333in}}%
\pgfusepath{stroke}%
\end{pgfscope}%
\begin{pgfscope}%
\pgfsetrectcap%
\pgfsetmiterjoin%
\pgfsetlinewidth{0.803000pt}%
\definecolor{currentstroke}{rgb}{0.000000,0.000000,0.000000}%
\pgfsetstrokecolor{currentstroke}%
\pgfsetdash{}{0pt}%
\pgfpathmoveto{\pgfqpoint{0.572918in}{0.553781in}}%
\pgfpathlineto{\pgfqpoint{5.587445in}{0.553781in}}%
\pgfusepath{stroke}%
\end{pgfscope}%
\begin{pgfscope}%
\pgfsetrectcap%
\pgfsetmiterjoin%
\pgfsetlinewidth{0.803000pt}%
\definecolor{currentstroke}{rgb}{0.000000,0.000000,0.000000}%
\pgfsetstrokecolor{currentstroke}%
\pgfsetdash{}{0pt}%
\pgfpathmoveto{\pgfqpoint{0.572918in}{2.649333in}}%
\pgfpathlineto{\pgfqpoint{5.587445in}{2.649333in}}%
\pgfusepath{stroke}%
\end{pgfscope}%
\begin{pgfscope}%
\definecolor{textcolor}{rgb}{0.000000,0.000000,0.000000}%
\pgfsetstrokecolor{textcolor}%
\pgfsetfillcolor{textcolor}%
\pgftext[x=0.572918in,y=2.732667in,left,base]{\color{textcolor}\rmfamily\fontsize{12.000000}{14.400000}\selectfont Zenith performance}%
\end{pgfscope}%
\begin{pgfscope}%
\pgfsetbuttcap%
\pgfsetmiterjoin%
\definecolor{currentfill}{rgb}{1.000000,1.000000,1.000000}%
\pgfsetfillcolor{currentfill}%
\pgfsetfillopacity{0.800000}%
\pgfsetlinewidth{1.003750pt}%
\definecolor{currentstroke}{rgb}{0.800000,0.800000,0.800000}%
\pgfsetstrokecolor{currentstroke}%
\pgfsetstrokeopacity{0.800000}%
\pgfsetdash{}{0pt}%
\pgfpathmoveto{\pgfqpoint{4.360223in}{2.249444in}}%
\pgfpathlineto{\pgfqpoint{5.509668in}{2.249444in}}%
\pgfpathquadraticcurveto{\pgfqpoint{5.531890in}{2.249444in}}{\pgfqpoint{5.531890in}{2.271667in}}%
\pgfpathlineto{\pgfqpoint{5.531890in}{2.571556in}}%
\pgfpathquadraticcurveto{\pgfqpoint{5.531890in}{2.593778in}}{\pgfqpoint{5.509668in}{2.593778in}}%
\pgfpathlineto{\pgfqpoint{4.360223in}{2.593778in}}%
\pgfpathquadraticcurveto{\pgfqpoint{4.338001in}{2.593778in}}{\pgfqpoint{4.338001in}{2.571556in}}%
\pgfpathlineto{\pgfqpoint{4.338001in}{2.271667in}}%
\pgfpathquadraticcurveto{\pgfqpoint{4.338001in}{2.249444in}}{\pgfqpoint{4.360223in}{2.249444in}}%
\pgfpathclose%
\pgfusepath{stroke,fill}%
\end{pgfscope}%
\begin{pgfscope}%
\pgfsetbuttcap%
\pgfsetmiterjoin%
\definecolor{currentfill}{rgb}{0.501961,0.501961,0.501961}%
\pgfsetfillcolor{currentfill}%
\pgfsetfillopacity{0.200000}%
\pgfsetlinewidth{0.000000pt}%
\definecolor{currentstroke}{rgb}{0.000000,0.000000,0.000000}%
\pgfsetstrokecolor{currentstroke}%
\pgfsetstrokeopacity{0.200000}%
\pgfsetdash{}{0pt}%
\pgfpathmoveto{\pgfqpoint{4.382445in}{2.471556in}}%
\pgfpathlineto{\pgfqpoint{4.604668in}{2.471556in}}%
\pgfpathlineto{\pgfqpoint{4.604668in}{2.549333in}}%
\pgfpathlineto{\pgfqpoint{4.382445in}{2.549333in}}%
\pgfpathclose%
\pgfusepath{fill}%
\end{pgfscope}%
\begin{pgfscope}%
\definecolor{textcolor}{rgb}{0.000000,0.000000,0.000000}%
\pgfsetstrokecolor{textcolor}%
\pgfsetfillcolor{textcolor}%
\pgftext[x=4.693557in,y=2.471556in,left,base]{\color{textcolor}\rmfamily\fontsize{8.000000}{9.600000}\selectfont Training events}%
\end{pgfscope}%
\begin{pgfscope}%
\pgfsetbuttcap%
\pgfsetroundjoin%
\pgfsetlinewidth{1.505625pt}%
\definecolor{currentstroke}{rgb}{0.313725,0.317647,0.309804}%
\pgfsetstrokecolor{currentstroke}%
\pgfsetstrokeopacity{0.900000}%
\pgfsetdash{}{0pt}%
\pgfpathmoveto{\pgfqpoint{4.438001in}{2.354333in}}%
\pgfpathlineto{\pgfqpoint{4.549112in}{2.354333in}}%
\pgfusepath{stroke}%
\end{pgfscope}%
\begin{pgfscope}%
\pgfsetbuttcap%
\pgfsetroundjoin%
\pgfsetlinewidth{1.505625pt}%
\definecolor{currentstroke}{rgb}{0.313725,0.317647,0.309804}%
\pgfsetstrokecolor{currentstroke}%
\pgfsetstrokeopacity{0.900000}%
\pgfsetdash{}{0pt}%
\pgfpathmoveto{\pgfqpoint{4.493557in}{2.298778in}}%
\pgfpathlineto{\pgfqpoint{4.493557in}{2.409889in}}%
\pgfusepath{stroke}%
\end{pgfscope}%
\begin{pgfscope}%
\pgfsetbuttcap%
\pgfsetroundjoin%
\definecolor{currentfill}{rgb}{0.313725,0.317647,0.309804}%
\pgfsetfillcolor{currentfill}%
\pgfsetfillopacity{0.900000}%
\pgfsetlinewidth{1.003750pt}%
\definecolor{currentstroke}{rgb}{0.313725,0.317647,0.309804}%
\pgfsetstrokecolor{currentstroke}%
\pgfsetstrokeopacity{0.900000}%
\pgfsetdash{}{0pt}%
\pgfsys@defobject{currentmarker}{\pgfqpoint{0.000000in}{-0.013889in}}{\pgfqpoint{0.000000in}{0.013889in}}{%
\pgfpathmoveto{\pgfqpoint{0.000000in}{-0.013889in}}%
\pgfpathlineto{\pgfqpoint{0.000000in}{0.013889in}}%
\pgfusepath{stroke,fill}%
}%
\begin{pgfscope}%
\pgfsys@transformshift{4.438001in}{2.354333in}%
\pgfsys@useobject{currentmarker}{}%
\end{pgfscope}%
\end{pgfscope}%
\begin{pgfscope}%
\pgfsetbuttcap%
\pgfsetroundjoin%
\definecolor{currentfill}{rgb}{0.313725,0.317647,0.309804}%
\pgfsetfillcolor{currentfill}%
\pgfsetfillopacity{0.900000}%
\pgfsetlinewidth{1.003750pt}%
\definecolor{currentstroke}{rgb}{0.313725,0.317647,0.309804}%
\pgfsetstrokecolor{currentstroke}%
\pgfsetstrokeopacity{0.900000}%
\pgfsetdash{}{0pt}%
\pgfsys@defobject{currentmarker}{\pgfqpoint{0.000000in}{-0.013889in}}{\pgfqpoint{0.000000in}{0.013889in}}{%
\pgfpathmoveto{\pgfqpoint{0.000000in}{-0.013889in}}%
\pgfpathlineto{\pgfqpoint{0.000000in}{0.013889in}}%
\pgfusepath{stroke,fill}%
}%
\begin{pgfscope}%
\pgfsys@transformshift{4.549112in}{2.354333in}%
\pgfsys@useobject{currentmarker}{}%
\end{pgfscope}%
\end{pgfscope}%
\begin{pgfscope}%
\pgfsetbuttcap%
\pgfsetroundjoin%
\definecolor{currentfill}{rgb}{0.313725,0.317647,0.309804}%
\pgfsetfillcolor{currentfill}%
\pgfsetfillopacity{0.900000}%
\pgfsetlinewidth{1.003750pt}%
\definecolor{currentstroke}{rgb}{0.313725,0.317647,0.309804}%
\pgfsetstrokecolor{currentstroke}%
\pgfsetstrokeopacity{0.900000}%
\pgfsetdash{}{0pt}%
\pgfsys@defobject{currentmarker}{\pgfqpoint{-0.013889in}{-0.000000in}}{\pgfqpoint{0.013889in}{0.000000in}}{%
\pgfpathmoveto{\pgfqpoint{0.013889in}{-0.000000in}}%
\pgfpathlineto{\pgfqpoint{-0.013889in}{0.000000in}}%
\pgfusepath{stroke,fill}%
}%
\begin{pgfscope}%
\pgfsys@transformshift{4.493557in}{2.298778in}%
\pgfsys@useobject{currentmarker}{}%
\end{pgfscope}%
\end{pgfscope}%
\begin{pgfscope}%
\pgfsetbuttcap%
\pgfsetroundjoin%
\definecolor{currentfill}{rgb}{0.313725,0.317647,0.309804}%
\pgfsetfillcolor{currentfill}%
\pgfsetfillopacity{0.900000}%
\pgfsetlinewidth{1.003750pt}%
\definecolor{currentstroke}{rgb}{0.313725,0.317647,0.309804}%
\pgfsetstrokecolor{currentstroke}%
\pgfsetstrokeopacity{0.900000}%
\pgfsetdash{}{0pt}%
\pgfsys@defobject{currentmarker}{\pgfqpoint{-0.013889in}{-0.000000in}}{\pgfqpoint{0.013889in}{0.000000in}}{%
\pgfpathmoveto{\pgfqpoint{0.013889in}{-0.000000in}}%
\pgfpathlineto{\pgfqpoint{-0.013889in}{0.000000in}}%
\pgfusepath{stroke,fill}%
}%
\begin{pgfscope}%
\pgfsys@transformshift{4.493557in}{2.409889in}%
\pgfsys@useobject{currentmarker}{}%
\end{pgfscope}%
\end{pgfscope}%
\begin{pgfscope}%
\definecolor{textcolor}{rgb}{0.000000,0.000000,0.000000}%
\pgfsetstrokecolor{textcolor}%
\pgfsetfillcolor{textcolor}%
\pgftext[x=4.693557in,y=2.315444in,left,base]{\color{textcolor}\rmfamily\fontsize{8.000000}{9.600000}\selectfont CubeFlow}%
\end{pgfscope}%
\begin{pgfscope}%
\pgfpathrectangle{\pgfqpoint{0.572918in}{0.553781in}}{\pgfqpoint{5.014527in}{2.095553in}}%
\pgfusepath{clip}%
\pgfsetbuttcap%
\pgfsetmiterjoin%
\definecolor{currentfill}{rgb}{0.501961,0.501961,0.501961}%
\pgfsetfillcolor{currentfill}%
\pgfsetfillopacity{0.200000}%
\pgfsetlinewidth{0.000000pt}%
\definecolor{currentstroke}{rgb}{0.000000,0.000000,0.000000}%
\pgfsetstrokecolor{currentstroke}%
\pgfsetstrokeopacity{0.200000}%
\pgfsetdash{}{0pt}%
\pgfpathmoveto{\pgfqpoint{0.800851in}{0.553781in}}%
\pgfpathlineto{\pgfqpoint{1.054103in}{0.553781in}}%
\pgfpathlineto{\pgfqpoint{1.054103in}{1.752692in}}%
\pgfpathlineto{\pgfqpoint{0.800851in}{1.752692in}}%
\pgfpathclose%
\pgfusepath{fill}%
\end{pgfscope}%
\begin{pgfscope}%
\pgfpathrectangle{\pgfqpoint{0.572918in}{0.553781in}}{\pgfqpoint{5.014527in}{2.095553in}}%
\pgfusepath{clip}%
\pgfsetbuttcap%
\pgfsetmiterjoin%
\definecolor{currentfill}{rgb}{0.501961,0.501961,0.501961}%
\pgfsetfillcolor{currentfill}%
\pgfsetfillopacity{0.200000}%
\pgfsetlinewidth{0.000000pt}%
\definecolor{currentstroke}{rgb}{0.000000,0.000000,0.000000}%
\pgfsetstrokecolor{currentstroke}%
\pgfsetstrokeopacity{0.200000}%
\pgfsetdash{}{0pt}%
\pgfpathmoveto{\pgfqpoint{1.054103in}{0.553781in}}%
\pgfpathlineto{\pgfqpoint{1.307355in}{0.553781in}}%
\pgfpathlineto{\pgfqpoint{1.307355in}{1.982800in}}%
\pgfpathlineto{\pgfqpoint{1.054103in}{1.982800in}}%
\pgfpathclose%
\pgfusepath{fill}%
\end{pgfscope}%
\begin{pgfscope}%
\pgfpathrectangle{\pgfqpoint{0.572918in}{0.553781in}}{\pgfqpoint{5.014527in}{2.095553in}}%
\pgfusepath{clip}%
\pgfsetbuttcap%
\pgfsetmiterjoin%
\definecolor{currentfill}{rgb}{0.501961,0.501961,0.501961}%
\pgfsetfillcolor{currentfill}%
\pgfsetfillopacity{0.200000}%
\pgfsetlinewidth{0.000000pt}%
\definecolor{currentstroke}{rgb}{0.000000,0.000000,0.000000}%
\pgfsetstrokecolor{currentstroke}%
\pgfsetstrokeopacity{0.200000}%
\pgfsetdash{}{0pt}%
\pgfpathmoveto{\pgfqpoint{1.307355in}{0.553781in}}%
\pgfpathlineto{\pgfqpoint{1.560606in}{0.553781in}}%
\pgfpathlineto{\pgfqpoint{1.560606in}{2.166736in}}%
\pgfpathlineto{\pgfqpoint{1.307355in}{2.166736in}}%
\pgfpathclose%
\pgfusepath{fill}%
\end{pgfscope}%
\begin{pgfscope}%
\pgfpathrectangle{\pgfqpoint{0.572918in}{0.553781in}}{\pgfqpoint{5.014527in}{2.095553in}}%
\pgfusepath{clip}%
\pgfsetbuttcap%
\pgfsetmiterjoin%
\definecolor{currentfill}{rgb}{0.501961,0.501961,0.501961}%
\pgfsetfillcolor{currentfill}%
\pgfsetfillopacity{0.200000}%
\pgfsetlinewidth{0.000000pt}%
\definecolor{currentstroke}{rgb}{0.000000,0.000000,0.000000}%
\pgfsetstrokecolor{currentstroke}%
\pgfsetstrokeopacity{0.200000}%
\pgfsetdash{}{0pt}%
\pgfpathmoveto{\pgfqpoint{1.560606in}{0.553781in}}%
\pgfpathlineto{\pgfqpoint{1.813858in}{0.553781in}}%
\pgfpathlineto{\pgfqpoint{1.813858in}{2.342955in}}%
\pgfpathlineto{\pgfqpoint{1.560606in}{2.342955in}}%
\pgfpathclose%
\pgfusepath{fill}%
\end{pgfscope}%
\begin{pgfscope}%
\pgfpathrectangle{\pgfqpoint{0.572918in}{0.553781in}}{\pgfqpoint{5.014527in}{2.095553in}}%
\pgfusepath{clip}%
\pgfsetbuttcap%
\pgfsetmiterjoin%
\definecolor{currentfill}{rgb}{0.501961,0.501961,0.501961}%
\pgfsetfillcolor{currentfill}%
\pgfsetfillopacity{0.200000}%
\pgfsetlinewidth{0.000000pt}%
\definecolor{currentstroke}{rgb}{0.000000,0.000000,0.000000}%
\pgfsetstrokecolor{currentstroke}%
\pgfsetstrokeopacity{0.200000}%
\pgfsetdash{}{0pt}%
\pgfpathmoveto{\pgfqpoint{1.813858in}{0.553781in}}%
\pgfpathlineto{\pgfqpoint{2.067110in}{0.553781in}}%
\pgfpathlineto{\pgfqpoint{2.067110in}{2.476225in}}%
\pgfpathlineto{\pgfqpoint{1.813858in}{2.476225in}}%
\pgfpathclose%
\pgfusepath{fill}%
\end{pgfscope}%
\begin{pgfscope}%
\pgfpathrectangle{\pgfqpoint{0.572918in}{0.553781in}}{\pgfqpoint{5.014527in}{2.095553in}}%
\pgfusepath{clip}%
\pgfsetbuttcap%
\pgfsetmiterjoin%
\definecolor{currentfill}{rgb}{0.501961,0.501961,0.501961}%
\pgfsetfillcolor{currentfill}%
\pgfsetfillopacity{0.200000}%
\pgfsetlinewidth{0.000000pt}%
\definecolor{currentstroke}{rgb}{0.000000,0.000000,0.000000}%
\pgfsetstrokecolor{currentstroke}%
\pgfsetstrokeopacity{0.200000}%
\pgfsetdash{}{0pt}%
\pgfpathmoveto{\pgfqpoint{2.067110in}{0.553781in}}%
\pgfpathlineto{\pgfqpoint{2.320362in}{0.553781in}}%
\pgfpathlineto{\pgfqpoint{2.320362in}{2.494221in}}%
\pgfpathlineto{\pgfqpoint{2.067110in}{2.494221in}}%
\pgfpathclose%
\pgfusepath{fill}%
\end{pgfscope}%
\begin{pgfscope}%
\pgfpathrectangle{\pgfqpoint{0.572918in}{0.553781in}}{\pgfqpoint{5.014527in}{2.095553in}}%
\pgfusepath{clip}%
\pgfsetbuttcap%
\pgfsetmiterjoin%
\definecolor{currentfill}{rgb}{0.501961,0.501961,0.501961}%
\pgfsetfillcolor{currentfill}%
\pgfsetfillopacity{0.200000}%
\pgfsetlinewidth{0.000000pt}%
\definecolor{currentstroke}{rgb}{0.000000,0.000000,0.000000}%
\pgfsetstrokecolor{currentstroke}%
\pgfsetstrokeopacity{0.200000}%
\pgfsetdash{}{0pt}%
\pgfpathmoveto{\pgfqpoint{2.320362in}{0.553781in}}%
\pgfpathlineto{\pgfqpoint{2.573613in}{0.553781in}}%
\pgfpathlineto{\pgfqpoint{2.573613in}{2.535113in}}%
\pgfpathlineto{\pgfqpoint{2.320362in}{2.535113in}}%
\pgfpathclose%
\pgfusepath{fill}%
\end{pgfscope}%
\begin{pgfscope}%
\pgfpathrectangle{\pgfqpoint{0.572918in}{0.553781in}}{\pgfqpoint{5.014527in}{2.095553in}}%
\pgfusepath{clip}%
\pgfsetbuttcap%
\pgfsetmiterjoin%
\definecolor{currentfill}{rgb}{0.501961,0.501961,0.501961}%
\pgfsetfillcolor{currentfill}%
\pgfsetfillopacity{0.200000}%
\pgfsetlinewidth{0.000000pt}%
\definecolor{currentstroke}{rgb}{0.000000,0.000000,0.000000}%
\pgfsetstrokecolor{currentstroke}%
\pgfsetstrokeopacity{0.200000}%
\pgfsetdash{}{0pt}%
\pgfpathmoveto{\pgfqpoint{2.573613in}{0.553781in}}%
\pgfpathlineto{\pgfqpoint{2.826865in}{0.553781in}}%
\pgfpathlineto{\pgfqpoint{2.826865in}{2.538274in}}%
\pgfpathlineto{\pgfqpoint{2.573613in}{2.538274in}}%
\pgfpathclose%
\pgfusepath{fill}%
\end{pgfscope}%
\begin{pgfscope}%
\pgfpathrectangle{\pgfqpoint{0.572918in}{0.553781in}}{\pgfqpoint{5.014527in}{2.095553in}}%
\pgfusepath{clip}%
\pgfsetbuttcap%
\pgfsetmiterjoin%
\definecolor{currentfill}{rgb}{0.501961,0.501961,0.501961}%
\pgfsetfillcolor{currentfill}%
\pgfsetfillopacity{0.200000}%
\pgfsetlinewidth{0.000000pt}%
\definecolor{currentstroke}{rgb}{0.000000,0.000000,0.000000}%
\pgfsetstrokecolor{currentstroke}%
\pgfsetstrokeopacity{0.200000}%
\pgfsetdash{}{0pt}%
\pgfpathmoveto{\pgfqpoint{2.826865in}{0.553781in}}%
\pgfpathlineto{\pgfqpoint{3.080117in}{0.553781in}}%
\pgfpathlineto{\pgfqpoint{3.080117in}{2.549545in}}%
\pgfpathlineto{\pgfqpoint{2.826865in}{2.549545in}}%
\pgfpathclose%
\pgfusepath{fill}%
\end{pgfscope}%
\begin{pgfscope}%
\pgfpathrectangle{\pgfqpoint{0.572918in}{0.553781in}}{\pgfqpoint{5.014527in}{2.095553in}}%
\pgfusepath{clip}%
\pgfsetbuttcap%
\pgfsetmiterjoin%
\definecolor{currentfill}{rgb}{0.501961,0.501961,0.501961}%
\pgfsetfillcolor{currentfill}%
\pgfsetfillopacity{0.200000}%
\pgfsetlinewidth{0.000000pt}%
\definecolor{currentstroke}{rgb}{0.000000,0.000000,0.000000}%
\pgfsetstrokecolor{currentstroke}%
\pgfsetstrokeopacity{0.200000}%
\pgfsetdash{}{0pt}%
\pgfpathmoveto{\pgfqpoint{3.080117in}{0.553781in}}%
\pgfpathlineto{\pgfqpoint{3.333368in}{0.553781in}}%
\pgfpathlineto{\pgfqpoint{3.333368in}{2.548075in}}%
\pgfpathlineto{\pgfqpoint{3.080117in}{2.548075in}}%
\pgfpathclose%
\pgfusepath{fill}%
\end{pgfscope}%
\begin{pgfscope}%
\pgfpathrectangle{\pgfqpoint{0.572918in}{0.553781in}}{\pgfqpoint{5.014527in}{2.095553in}}%
\pgfusepath{clip}%
\pgfsetbuttcap%
\pgfsetmiterjoin%
\definecolor{currentfill}{rgb}{0.501961,0.501961,0.501961}%
\pgfsetfillcolor{currentfill}%
\pgfsetfillopacity{0.200000}%
\pgfsetlinewidth{0.000000pt}%
\definecolor{currentstroke}{rgb}{0.000000,0.000000,0.000000}%
\pgfsetstrokecolor{currentstroke}%
\pgfsetstrokeopacity{0.200000}%
\pgfsetdash{}{0pt}%
\pgfpathmoveto{\pgfqpoint{3.333368in}{0.553781in}}%
\pgfpathlineto{\pgfqpoint{3.586620in}{0.553781in}}%
\pgfpathlineto{\pgfqpoint{3.586620in}{2.539261in}}%
\pgfpathlineto{\pgfqpoint{3.333368in}{2.539261in}}%
\pgfpathclose%
\pgfusepath{fill}%
\end{pgfscope}%
\begin{pgfscope}%
\pgfpathrectangle{\pgfqpoint{0.572918in}{0.553781in}}{\pgfqpoint{5.014527in}{2.095553in}}%
\pgfusepath{clip}%
\pgfsetbuttcap%
\pgfsetmiterjoin%
\definecolor{currentfill}{rgb}{0.501961,0.501961,0.501961}%
\pgfsetfillcolor{currentfill}%
\pgfsetfillopacity{0.200000}%
\pgfsetlinewidth{0.000000pt}%
\definecolor{currentstroke}{rgb}{0.000000,0.000000,0.000000}%
\pgfsetstrokecolor{currentstroke}%
\pgfsetstrokeopacity{0.200000}%
\pgfsetdash{}{0pt}%
\pgfpathmoveto{\pgfqpoint{3.586620in}{0.553781in}}%
\pgfpathlineto{\pgfqpoint{3.839872in}{0.553781in}}%
\pgfpathlineto{\pgfqpoint{3.839872in}{2.525123in}}%
\pgfpathlineto{\pgfqpoint{3.586620in}{2.525123in}}%
\pgfpathclose%
\pgfusepath{fill}%
\end{pgfscope}%
\begin{pgfscope}%
\pgfpathrectangle{\pgfqpoint{0.572918in}{0.553781in}}{\pgfqpoint{5.014527in}{2.095553in}}%
\pgfusepath{clip}%
\pgfsetbuttcap%
\pgfsetmiterjoin%
\definecolor{currentfill}{rgb}{0.501961,0.501961,0.501961}%
\pgfsetfillcolor{currentfill}%
\pgfsetfillopacity{0.200000}%
\pgfsetlinewidth{0.000000pt}%
\definecolor{currentstroke}{rgb}{0.000000,0.000000,0.000000}%
\pgfsetstrokecolor{currentstroke}%
\pgfsetstrokeopacity{0.200000}%
\pgfsetdash{}{0pt}%
\pgfpathmoveto{\pgfqpoint{3.839872in}{0.553781in}}%
\pgfpathlineto{\pgfqpoint{4.093123in}{0.553781in}}%
\pgfpathlineto{\pgfqpoint{4.093123in}{2.501543in}}%
\pgfpathlineto{\pgfqpoint{3.839872in}{2.501543in}}%
\pgfpathclose%
\pgfusepath{fill}%
\end{pgfscope}%
\begin{pgfscope}%
\pgfpathrectangle{\pgfqpoint{0.572918in}{0.553781in}}{\pgfqpoint{5.014527in}{2.095553in}}%
\pgfusepath{clip}%
\pgfsetbuttcap%
\pgfsetmiterjoin%
\definecolor{currentfill}{rgb}{0.501961,0.501961,0.501961}%
\pgfsetfillcolor{currentfill}%
\pgfsetfillopacity{0.200000}%
\pgfsetlinewidth{0.000000pt}%
\definecolor{currentstroke}{rgb}{0.000000,0.000000,0.000000}%
\pgfsetstrokecolor{currentstroke}%
\pgfsetstrokeopacity{0.200000}%
\pgfsetdash{}{0pt}%
\pgfpathmoveto{\pgfqpoint{4.093123in}{0.553781in}}%
\pgfpathlineto{\pgfqpoint{4.346375in}{0.553781in}}%
\pgfpathlineto{\pgfqpoint{4.346375in}{2.467164in}}%
\pgfpathlineto{\pgfqpoint{4.093123in}{2.467164in}}%
\pgfpathclose%
\pgfusepath{fill}%
\end{pgfscope}%
\begin{pgfscope}%
\pgfpathrectangle{\pgfqpoint{0.572918in}{0.553781in}}{\pgfqpoint{5.014527in}{2.095553in}}%
\pgfusepath{clip}%
\pgfsetbuttcap%
\pgfsetmiterjoin%
\definecolor{currentfill}{rgb}{0.501961,0.501961,0.501961}%
\pgfsetfillcolor{currentfill}%
\pgfsetfillopacity{0.200000}%
\pgfsetlinewidth{0.000000pt}%
\definecolor{currentstroke}{rgb}{0.000000,0.000000,0.000000}%
\pgfsetstrokecolor{currentstroke}%
\pgfsetstrokeopacity{0.200000}%
\pgfsetdash{}{0pt}%
\pgfpathmoveto{\pgfqpoint{4.346375in}{0.553781in}}%
\pgfpathlineto{\pgfqpoint{4.599627in}{0.553781in}}%
\pgfpathlineto{\pgfqpoint{4.599627in}{2.423484in}}%
\pgfpathlineto{\pgfqpoint{4.346375in}{2.423484in}}%
\pgfpathclose%
\pgfusepath{fill}%
\end{pgfscope}%
\begin{pgfscope}%
\pgfpathrectangle{\pgfqpoint{0.572918in}{0.553781in}}{\pgfqpoint{5.014527in}{2.095553in}}%
\pgfusepath{clip}%
\pgfsetbuttcap%
\pgfsetmiterjoin%
\definecolor{currentfill}{rgb}{0.501961,0.501961,0.501961}%
\pgfsetfillcolor{currentfill}%
\pgfsetfillopacity{0.200000}%
\pgfsetlinewidth{0.000000pt}%
\definecolor{currentstroke}{rgb}{0.000000,0.000000,0.000000}%
\pgfsetstrokecolor{currentstroke}%
\pgfsetstrokeopacity{0.200000}%
\pgfsetdash{}{0pt}%
\pgfpathmoveto{\pgfqpoint{4.599627in}{0.553781in}}%
\pgfpathlineto{\pgfqpoint{4.852878in}{0.553781in}}%
\pgfpathlineto{\pgfqpoint{4.852878in}{2.365628in}}%
\pgfpathlineto{\pgfqpoint{4.599627in}{2.365628in}}%
\pgfpathclose%
\pgfusepath{fill}%
\end{pgfscope}%
\begin{pgfscope}%
\pgfpathrectangle{\pgfqpoint{0.572918in}{0.553781in}}{\pgfqpoint{5.014527in}{2.095553in}}%
\pgfusepath{clip}%
\pgfsetbuttcap%
\pgfsetmiterjoin%
\definecolor{currentfill}{rgb}{0.501961,0.501961,0.501961}%
\pgfsetfillcolor{currentfill}%
\pgfsetfillopacity{0.200000}%
\pgfsetlinewidth{0.000000pt}%
\definecolor{currentstroke}{rgb}{0.000000,0.000000,0.000000}%
\pgfsetstrokecolor{currentstroke}%
\pgfsetstrokeopacity{0.200000}%
\pgfsetdash{}{0pt}%
\pgfpathmoveto{\pgfqpoint{4.852878in}{0.553781in}}%
\pgfpathlineto{\pgfqpoint{5.106130in}{0.553781in}}%
\pgfpathlineto{\pgfqpoint{5.106130in}{2.330040in}}%
\pgfpathlineto{\pgfqpoint{4.852878in}{2.330040in}}%
\pgfpathclose%
\pgfusepath{fill}%
\end{pgfscope}%
\begin{pgfscope}%
\pgfpathrectangle{\pgfqpoint{0.572918in}{0.553781in}}{\pgfqpoint{5.014527in}{2.095553in}}%
\pgfusepath{clip}%
\pgfsetbuttcap%
\pgfsetmiterjoin%
\definecolor{currentfill}{rgb}{0.501961,0.501961,0.501961}%
\pgfsetfillcolor{currentfill}%
\pgfsetfillopacity{0.200000}%
\pgfsetlinewidth{0.000000pt}%
\definecolor{currentstroke}{rgb}{0.000000,0.000000,0.000000}%
\pgfsetstrokecolor{currentstroke}%
\pgfsetstrokeopacity{0.200000}%
\pgfsetdash{}{0pt}%
\pgfpathmoveto{\pgfqpoint{5.106130in}{0.553781in}}%
\pgfpathlineto{\pgfqpoint{5.359382in}{0.553781in}}%
\pgfpathlineto{\pgfqpoint{5.359382in}{2.319544in}}%
\pgfpathlineto{\pgfqpoint{5.106130in}{2.319544in}}%
\pgfpathclose%
\pgfusepath{fill}%
\end{pgfscope}%
\begin{pgfscope}%
\pgfsetbuttcap%
\pgfsetroundjoin%
\definecolor{currentfill}{rgb}{0.000000,0.000000,0.000000}%
\pgfsetfillcolor{currentfill}%
\pgfsetlinewidth{0.803000pt}%
\definecolor{currentstroke}{rgb}{0.000000,0.000000,0.000000}%
\pgfsetstrokecolor{currentstroke}%
\pgfsetdash{}{0pt}%
\pgfsys@defobject{currentmarker}{\pgfqpoint{0.000000in}{0.000000in}}{\pgfqpoint{0.048611in}{0.000000in}}{%
\pgfpathmoveto{\pgfqpoint{0.000000in}{0.000000in}}%
\pgfpathlineto{\pgfqpoint{0.048611in}{0.000000in}}%
\pgfusepath{stroke,fill}%
}%
\begin{pgfscope}%
\pgfsys@transformshift{5.587445in}{0.553781in}%
\pgfsys@useobject{currentmarker}{}%
\end{pgfscope}%
\end{pgfscope}%
\begin{pgfscope}%
\definecolor{textcolor}{rgb}{0.000000,0.000000,0.000000}%
\pgfsetstrokecolor{textcolor}%
\pgfsetfillcolor{textcolor}%
\pgftext[x=5.684668in, y=0.514628in, left, base]{\color{textcolor}\rmfamily\fontsize{8.000000}{9.600000}\selectfont \(\displaystyle {10^{0}}\)}%
\end{pgfscope}%
\begin{pgfscope}%
\pgfsetbuttcap%
\pgfsetroundjoin%
\definecolor{currentfill}{rgb}{0.000000,0.000000,0.000000}%
\pgfsetfillcolor{currentfill}%
\pgfsetlinewidth{0.803000pt}%
\definecolor{currentstroke}{rgb}{0.000000,0.000000,0.000000}%
\pgfsetstrokecolor{currentstroke}%
\pgfsetdash{}{0pt}%
\pgfsys@defobject{currentmarker}{\pgfqpoint{0.000000in}{0.000000in}}{\pgfqpoint{0.048611in}{0.000000in}}{%
\pgfpathmoveto{\pgfqpoint{0.000000in}{0.000000in}}%
\pgfpathlineto{\pgfqpoint{0.048611in}{0.000000in}}%
\pgfusepath{stroke,fill}%
}%
\begin{pgfscope}%
\pgfsys@transformshift{5.587445in}{1.029571in}%
\pgfsys@useobject{currentmarker}{}%
\end{pgfscope}%
\end{pgfscope}%
\begin{pgfscope}%
\definecolor{textcolor}{rgb}{0.000000,0.000000,0.000000}%
\pgfsetstrokecolor{textcolor}%
\pgfsetfillcolor{textcolor}%
\pgftext[x=5.684668in, y=0.990419in, left, base]{\color{textcolor}\rmfamily\fontsize{8.000000}{9.600000}\selectfont \(\displaystyle {10^{1}}\)}%
\end{pgfscope}%
\begin{pgfscope}%
\pgfsetbuttcap%
\pgfsetroundjoin%
\definecolor{currentfill}{rgb}{0.000000,0.000000,0.000000}%
\pgfsetfillcolor{currentfill}%
\pgfsetlinewidth{0.803000pt}%
\definecolor{currentstroke}{rgb}{0.000000,0.000000,0.000000}%
\pgfsetstrokecolor{currentstroke}%
\pgfsetdash{}{0pt}%
\pgfsys@defobject{currentmarker}{\pgfqpoint{0.000000in}{0.000000in}}{\pgfqpoint{0.048611in}{0.000000in}}{%
\pgfpathmoveto{\pgfqpoint{0.000000in}{0.000000in}}%
\pgfpathlineto{\pgfqpoint{0.048611in}{0.000000in}}%
\pgfusepath{stroke,fill}%
}%
\begin{pgfscope}%
\pgfsys@transformshift{5.587445in}{1.505362in}%
\pgfsys@useobject{currentmarker}{}%
\end{pgfscope}%
\end{pgfscope}%
\begin{pgfscope}%
\definecolor{textcolor}{rgb}{0.000000,0.000000,0.000000}%
\pgfsetstrokecolor{textcolor}%
\pgfsetfillcolor{textcolor}%
\pgftext[x=5.684668in, y=1.466210in, left, base]{\color{textcolor}\rmfamily\fontsize{8.000000}{9.600000}\selectfont \(\displaystyle {10^{2}}\)}%
\end{pgfscope}%
\begin{pgfscope}%
\pgfsetbuttcap%
\pgfsetroundjoin%
\definecolor{currentfill}{rgb}{0.000000,0.000000,0.000000}%
\pgfsetfillcolor{currentfill}%
\pgfsetlinewidth{0.803000pt}%
\definecolor{currentstroke}{rgb}{0.000000,0.000000,0.000000}%
\pgfsetstrokecolor{currentstroke}%
\pgfsetdash{}{0pt}%
\pgfsys@defobject{currentmarker}{\pgfqpoint{0.000000in}{0.000000in}}{\pgfqpoint{0.048611in}{0.000000in}}{%
\pgfpathmoveto{\pgfqpoint{0.000000in}{0.000000in}}%
\pgfpathlineto{\pgfqpoint{0.048611in}{0.000000in}}%
\pgfusepath{stroke,fill}%
}%
\begin{pgfscope}%
\pgfsys@transformshift{5.587445in}{1.981153in}%
\pgfsys@useobject{currentmarker}{}%
\end{pgfscope}%
\end{pgfscope}%
\begin{pgfscope}%
\definecolor{textcolor}{rgb}{0.000000,0.000000,0.000000}%
\pgfsetstrokecolor{textcolor}%
\pgfsetfillcolor{textcolor}%
\pgftext[x=5.684668in, y=1.942001in, left, base]{\color{textcolor}\rmfamily\fontsize{8.000000}{9.600000}\selectfont \(\displaystyle {10^{3}}\)}%
\end{pgfscope}%
\begin{pgfscope}%
\pgfsetbuttcap%
\pgfsetroundjoin%
\definecolor{currentfill}{rgb}{0.000000,0.000000,0.000000}%
\pgfsetfillcolor{currentfill}%
\pgfsetlinewidth{0.803000pt}%
\definecolor{currentstroke}{rgb}{0.000000,0.000000,0.000000}%
\pgfsetstrokecolor{currentstroke}%
\pgfsetdash{}{0pt}%
\pgfsys@defobject{currentmarker}{\pgfqpoint{0.000000in}{0.000000in}}{\pgfqpoint{0.048611in}{0.000000in}}{%
\pgfpathmoveto{\pgfqpoint{0.000000in}{0.000000in}}%
\pgfpathlineto{\pgfqpoint{0.048611in}{0.000000in}}%
\pgfusepath{stroke,fill}%
}%
\begin{pgfscope}%
\pgfsys@transformshift{5.587445in}{2.456944in}%
\pgfsys@useobject{currentmarker}{}%
\end{pgfscope}%
\end{pgfscope}%
\begin{pgfscope}%
\definecolor{textcolor}{rgb}{0.000000,0.000000,0.000000}%
\pgfsetstrokecolor{textcolor}%
\pgfsetfillcolor{textcolor}%
\pgftext[x=5.684668in, y=2.417791in, left, base]{\color{textcolor}\rmfamily\fontsize{8.000000}{9.600000}\selectfont \(\displaystyle {10^{4}}\)}%
\end{pgfscope}%
\begin{pgfscope}%
\pgfsetbuttcap%
\pgfsetroundjoin%
\definecolor{currentfill}{rgb}{0.000000,0.000000,0.000000}%
\pgfsetfillcolor{currentfill}%
\pgfsetlinewidth{0.602250pt}%
\definecolor{currentstroke}{rgb}{0.000000,0.000000,0.000000}%
\pgfsetstrokecolor{currentstroke}%
\pgfsetdash{}{0pt}%
\pgfsys@defobject{currentmarker}{\pgfqpoint{0.000000in}{0.000000in}}{\pgfqpoint{0.027778in}{0.000000in}}{%
\pgfpathmoveto{\pgfqpoint{0.000000in}{0.000000in}}%
\pgfpathlineto{\pgfqpoint{0.027778in}{0.000000in}}%
\pgfusepath{stroke,fill}%
}%
\begin{pgfscope}%
\pgfsys@transformshift{5.587445in}{0.697008in}%
\pgfsys@useobject{currentmarker}{}%
\end{pgfscope}%
\end{pgfscope}%
\begin{pgfscope}%
\pgfsetbuttcap%
\pgfsetroundjoin%
\definecolor{currentfill}{rgb}{0.000000,0.000000,0.000000}%
\pgfsetfillcolor{currentfill}%
\pgfsetlinewidth{0.602250pt}%
\definecolor{currentstroke}{rgb}{0.000000,0.000000,0.000000}%
\pgfsetstrokecolor{currentstroke}%
\pgfsetdash{}{0pt}%
\pgfsys@defobject{currentmarker}{\pgfqpoint{0.000000in}{0.000000in}}{\pgfqpoint{0.027778in}{0.000000in}}{%
\pgfpathmoveto{\pgfqpoint{0.000000in}{0.000000in}}%
\pgfpathlineto{\pgfqpoint{0.027778in}{0.000000in}}%
\pgfusepath{stroke,fill}%
}%
\begin{pgfscope}%
\pgfsys@transformshift{5.587445in}{0.780791in}%
\pgfsys@useobject{currentmarker}{}%
\end{pgfscope}%
\end{pgfscope}%
\begin{pgfscope}%
\pgfsetbuttcap%
\pgfsetroundjoin%
\definecolor{currentfill}{rgb}{0.000000,0.000000,0.000000}%
\pgfsetfillcolor{currentfill}%
\pgfsetlinewidth{0.602250pt}%
\definecolor{currentstroke}{rgb}{0.000000,0.000000,0.000000}%
\pgfsetstrokecolor{currentstroke}%
\pgfsetdash{}{0pt}%
\pgfsys@defobject{currentmarker}{\pgfqpoint{0.000000in}{0.000000in}}{\pgfqpoint{0.027778in}{0.000000in}}{%
\pgfpathmoveto{\pgfqpoint{0.000000in}{0.000000in}}%
\pgfpathlineto{\pgfqpoint{0.027778in}{0.000000in}}%
\pgfusepath{stroke,fill}%
}%
\begin{pgfscope}%
\pgfsys@transformshift{5.587445in}{0.840235in}%
\pgfsys@useobject{currentmarker}{}%
\end{pgfscope}%
\end{pgfscope}%
\begin{pgfscope}%
\pgfsetbuttcap%
\pgfsetroundjoin%
\definecolor{currentfill}{rgb}{0.000000,0.000000,0.000000}%
\pgfsetfillcolor{currentfill}%
\pgfsetlinewidth{0.602250pt}%
\definecolor{currentstroke}{rgb}{0.000000,0.000000,0.000000}%
\pgfsetstrokecolor{currentstroke}%
\pgfsetdash{}{0pt}%
\pgfsys@defobject{currentmarker}{\pgfqpoint{0.000000in}{0.000000in}}{\pgfqpoint{0.027778in}{0.000000in}}{%
\pgfpathmoveto{\pgfqpoint{0.000000in}{0.000000in}}%
\pgfpathlineto{\pgfqpoint{0.027778in}{0.000000in}}%
\pgfusepath{stroke,fill}%
}%
\begin{pgfscope}%
\pgfsys@transformshift{5.587445in}{0.886344in}%
\pgfsys@useobject{currentmarker}{}%
\end{pgfscope}%
\end{pgfscope}%
\begin{pgfscope}%
\pgfsetbuttcap%
\pgfsetroundjoin%
\definecolor{currentfill}{rgb}{0.000000,0.000000,0.000000}%
\pgfsetfillcolor{currentfill}%
\pgfsetlinewidth{0.602250pt}%
\definecolor{currentstroke}{rgb}{0.000000,0.000000,0.000000}%
\pgfsetstrokecolor{currentstroke}%
\pgfsetdash{}{0pt}%
\pgfsys@defobject{currentmarker}{\pgfqpoint{0.000000in}{0.000000in}}{\pgfqpoint{0.027778in}{0.000000in}}{%
\pgfpathmoveto{\pgfqpoint{0.000000in}{0.000000in}}%
\pgfpathlineto{\pgfqpoint{0.027778in}{0.000000in}}%
\pgfusepath{stroke,fill}%
}%
\begin{pgfscope}%
\pgfsys@transformshift{5.587445in}{0.924018in}%
\pgfsys@useobject{currentmarker}{}%
\end{pgfscope}%
\end{pgfscope}%
\begin{pgfscope}%
\pgfsetbuttcap%
\pgfsetroundjoin%
\definecolor{currentfill}{rgb}{0.000000,0.000000,0.000000}%
\pgfsetfillcolor{currentfill}%
\pgfsetlinewidth{0.602250pt}%
\definecolor{currentstroke}{rgb}{0.000000,0.000000,0.000000}%
\pgfsetstrokecolor{currentstroke}%
\pgfsetdash{}{0pt}%
\pgfsys@defobject{currentmarker}{\pgfqpoint{0.000000in}{0.000000in}}{\pgfqpoint{0.027778in}{0.000000in}}{%
\pgfpathmoveto{\pgfqpoint{0.000000in}{0.000000in}}%
\pgfpathlineto{\pgfqpoint{0.027778in}{0.000000in}}%
\pgfusepath{stroke,fill}%
}%
\begin{pgfscope}%
\pgfsys@transformshift{5.587445in}{0.955871in}%
\pgfsys@useobject{currentmarker}{}%
\end{pgfscope}%
\end{pgfscope}%
\begin{pgfscope}%
\pgfsetbuttcap%
\pgfsetroundjoin%
\definecolor{currentfill}{rgb}{0.000000,0.000000,0.000000}%
\pgfsetfillcolor{currentfill}%
\pgfsetlinewidth{0.602250pt}%
\definecolor{currentstroke}{rgb}{0.000000,0.000000,0.000000}%
\pgfsetstrokecolor{currentstroke}%
\pgfsetdash{}{0pt}%
\pgfsys@defobject{currentmarker}{\pgfqpoint{0.000000in}{0.000000in}}{\pgfqpoint{0.027778in}{0.000000in}}{%
\pgfpathmoveto{\pgfqpoint{0.000000in}{0.000000in}}%
\pgfpathlineto{\pgfqpoint{0.027778in}{0.000000in}}%
\pgfusepath{stroke,fill}%
}%
\begin{pgfscope}%
\pgfsys@transformshift{5.587445in}{0.983463in}%
\pgfsys@useobject{currentmarker}{}%
\end{pgfscope}%
\end{pgfscope}%
\begin{pgfscope}%
\pgfsetbuttcap%
\pgfsetroundjoin%
\definecolor{currentfill}{rgb}{0.000000,0.000000,0.000000}%
\pgfsetfillcolor{currentfill}%
\pgfsetlinewidth{0.602250pt}%
\definecolor{currentstroke}{rgb}{0.000000,0.000000,0.000000}%
\pgfsetstrokecolor{currentstroke}%
\pgfsetdash{}{0pt}%
\pgfsys@defobject{currentmarker}{\pgfqpoint{0.000000in}{0.000000in}}{\pgfqpoint{0.027778in}{0.000000in}}{%
\pgfpathmoveto{\pgfqpoint{0.000000in}{0.000000in}}%
\pgfpathlineto{\pgfqpoint{0.027778in}{0.000000in}}%
\pgfusepath{stroke,fill}%
}%
\begin{pgfscope}%
\pgfsys@transformshift{5.587445in}{1.007800in}%
\pgfsys@useobject{currentmarker}{}%
\end{pgfscope}%
\end{pgfscope}%
\begin{pgfscope}%
\pgfsetbuttcap%
\pgfsetroundjoin%
\definecolor{currentfill}{rgb}{0.000000,0.000000,0.000000}%
\pgfsetfillcolor{currentfill}%
\pgfsetlinewidth{0.602250pt}%
\definecolor{currentstroke}{rgb}{0.000000,0.000000,0.000000}%
\pgfsetstrokecolor{currentstroke}%
\pgfsetdash{}{0pt}%
\pgfsys@defobject{currentmarker}{\pgfqpoint{0.000000in}{0.000000in}}{\pgfqpoint{0.027778in}{0.000000in}}{%
\pgfpathmoveto{\pgfqpoint{0.000000in}{0.000000in}}%
\pgfpathlineto{\pgfqpoint{0.027778in}{0.000000in}}%
\pgfusepath{stroke,fill}%
}%
\begin{pgfscope}%
\pgfsys@transformshift{5.587445in}{1.172799in}%
\pgfsys@useobject{currentmarker}{}%
\end{pgfscope}%
\end{pgfscope}%
\begin{pgfscope}%
\pgfsetbuttcap%
\pgfsetroundjoin%
\definecolor{currentfill}{rgb}{0.000000,0.000000,0.000000}%
\pgfsetfillcolor{currentfill}%
\pgfsetlinewidth{0.602250pt}%
\definecolor{currentstroke}{rgb}{0.000000,0.000000,0.000000}%
\pgfsetstrokecolor{currentstroke}%
\pgfsetdash{}{0pt}%
\pgfsys@defobject{currentmarker}{\pgfqpoint{0.000000in}{0.000000in}}{\pgfqpoint{0.027778in}{0.000000in}}{%
\pgfpathmoveto{\pgfqpoint{0.000000in}{0.000000in}}%
\pgfpathlineto{\pgfqpoint{0.027778in}{0.000000in}}%
\pgfusepath{stroke,fill}%
}%
\begin{pgfscope}%
\pgfsys@transformshift{5.587445in}{1.256581in}%
\pgfsys@useobject{currentmarker}{}%
\end{pgfscope}%
\end{pgfscope}%
\begin{pgfscope}%
\pgfsetbuttcap%
\pgfsetroundjoin%
\definecolor{currentfill}{rgb}{0.000000,0.000000,0.000000}%
\pgfsetfillcolor{currentfill}%
\pgfsetlinewidth{0.602250pt}%
\definecolor{currentstroke}{rgb}{0.000000,0.000000,0.000000}%
\pgfsetstrokecolor{currentstroke}%
\pgfsetdash{}{0pt}%
\pgfsys@defobject{currentmarker}{\pgfqpoint{0.000000in}{0.000000in}}{\pgfqpoint{0.027778in}{0.000000in}}{%
\pgfpathmoveto{\pgfqpoint{0.000000in}{0.000000in}}%
\pgfpathlineto{\pgfqpoint{0.027778in}{0.000000in}}%
\pgfusepath{stroke,fill}%
}%
\begin{pgfscope}%
\pgfsys@transformshift{5.587445in}{1.316026in}%
\pgfsys@useobject{currentmarker}{}%
\end{pgfscope}%
\end{pgfscope}%
\begin{pgfscope}%
\pgfsetbuttcap%
\pgfsetroundjoin%
\definecolor{currentfill}{rgb}{0.000000,0.000000,0.000000}%
\pgfsetfillcolor{currentfill}%
\pgfsetlinewidth{0.602250pt}%
\definecolor{currentstroke}{rgb}{0.000000,0.000000,0.000000}%
\pgfsetstrokecolor{currentstroke}%
\pgfsetdash{}{0pt}%
\pgfsys@defobject{currentmarker}{\pgfqpoint{0.000000in}{0.000000in}}{\pgfqpoint{0.027778in}{0.000000in}}{%
\pgfpathmoveto{\pgfqpoint{0.000000in}{0.000000in}}%
\pgfpathlineto{\pgfqpoint{0.027778in}{0.000000in}}%
\pgfusepath{stroke,fill}%
}%
\begin{pgfscope}%
\pgfsys@transformshift{5.587445in}{1.362135in}%
\pgfsys@useobject{currentmarker}{}%
\end{pgfscope}%
\end{pgfscope}%
\begin{pgfscope}%
\pgfsetbuttcap%
\pgfsetroundjoin%
\definecolor{currentfill}{rgb}{0.000000,0.000000,0.000000}%
\pgfsetfillcolor{currentfill}%
\pgfsetlinewidth{0.602250pt}%
\definecolor{currentstroke}{rgb}{0.000000,0.000000,0.000000}%
\pgfsetstrokecolor{currentstroke}%
\pgfsetdash{}{0pt}%
\pgfsys@defobject{currentmarker}{\pgfqpoint{0.000000in}{0.000000in}}{\pgfqpoint{0.027778in}{0.000000in}}{%
\pgfpathmoveto{\pgfqpoint{0.000000in}{0.000000in}}%
\pgfpathlineto{\pgfqpoint{0.027778in}{0.000000in}}%
\pgfusepath{stroke,fill}%
}%
\begin{pgfscope}%
\pgfsys@transformshift{5.587445in}{1.399809in}%
\pgfsys@useobject{currentmarker}{}%
\end{pgfscope}%
\end{pgfscope}%
\begin{pgfscope}%
\pgfsetbuttcap%
\pgfsetroundjoin%
\definecolor{currentfill}{rgb}{0.000000,0.000000,0.000000}%
\pgfsetfillcolor{currentfill}%
\pgfsetlinewidth{0.602250pt}%
\definecolor{currentstroke}{rgb}{0.000000,0.000000,0.000000}%
\pgfsetstrokecolor{currentstroke}%
\pgfsetdash{}{0pt}%
\pgfsys@defobject{currentmarker}{\pgfqpoint{0.000000in}{0.000000in}}{\pgfqpoint{0.027778in}{0.000000in}}{%
\pgfpathmoveto{\pgfqpoint{0.000000in}{0.000000in}}%
\pgfpathlineto{\pgfqpoint{0.027778in}{0.000000in}}%
\pgfusepath{stroke,fill}%
}%
\begin{pgfscope}%
\pgfsys@transformshift{5.587445in}{1.431661in}%
\pgfsys@useobject{currentmarker}{}%
\end{pgfscope}%
\end{pgfscope}%
\begin{pgfscope}%
\pgfsetbuttcap%
\pgfsetroundjoin%
\definecolor{currentfill}{rgb}{0.000000,0.000000,0.000000}%
\pgfsetfillcolor{currentfill}%
\pgfsetlinewidth{0.602250pt}%
\definecolor{currentstroke}{rgb}{0.000000,0.000000,0.000000}%
\pgfsetstrokecolor{currentstroke}%
\pgfsetdash{}{0pt}%
\pgfsys@defobject{currentmarker}{\pgfqpoint{0.000000in}{0.000000in}}{\pgfqpoint{0.027778in}{0.000000in}}{%
\pgfpathmoveto{\pgfqpoint{0.000000in}{0.000000in}}%
\pgfpathlineto{\pgfqpoint{0.027778in}{0.000000in}}%
\pgfusepath{stroke,fill}%
}%
\begin{pgfscope}%
\pgfsys@transformshift{5.587445in}{1.459253in}%
\pgfsys@useobject{currentmarker}{}%
\end{pgfscope}%
\end{pgfscope}%
\begin{pgfscope}%
\pgfsetbuttcap%
\pgfsetroundjoin%
\definecolor{currentfill}{rgb}{0.000000,0.000000,0.000000}%
\pgfsetfillcolor{currentfill}%
\pgfsetlinewidth{0.602250pt}%
\definecolor{currentstroke}{rgb}{0.000000,0.000000,0.000000}%
\pgfsetstrokecolor{currentstroke}%
\pgfsetdash{}{0pt}%
\pgfsys@defobject{currentmarker}{\pgfqpoint{0.000000in}{0.000000in}}{\pgfqpoint{0.027778in}{0.000000in}}{%
\pgfpathmoveto{\pgfqpoint{0.000000in}{0.000000in}}%
\pgfpathlineto{\pgfqpoint{0.027778in}{0.000000in}}%
\pgfusepath{stroke,fill}%
}%
\begin{pgfscope}%
\pgfsys@transformshift{5.587445in}{1.483591in}%
\pgfsys@useobject{currentmarker}{}%
\end{pgfscope}%
\end{pgfscope}%
\begin{pgfscope}%
\pgfsetbuttcap%
\pgfsetroundjoin%
\definecolor{currentfill}{rgb}{0.000000,0.000000,0.000000}%
\pgfsetfillcolor{currentfill}%
\pgfsetlinewidth{0.602250pt}%
\definecolor{currentstroke}{rgb}{0.000000,0.000000,0.000000}%
\pgfsetstrokecolor{currentstroke}%
\pgfsetdash{}{0pt}%
\pgfsys@defobject{currentmarker}{\pgfqpoint{0.000000in}{0.000000in}}{\pgfqpoint{0.027778in}{0.000000in}}{%
\pgfpathmoveto{\pgfqpoint{0.000000in}{0.000000in}}%
\pgfpathlineto{\pgfqpoint{0.027778in}{0.000000in}}%
\pgfusepath{stroke,fill}%
}%
\begin{pgfscope}%
\pgfsys@transformshift{5.587445in}{1.648590in}%
\pgfsys@useobject{currentmarker}{}%
\end{pgfscope}%
\end{pgfscope}%
\begin{pgfscope}%
\pgfsetbuttcap%
\pgfsetroundjoin%
\definecolor{currentfill}{rgb}{0.000000,0.000000,0.000000}%
\pgfsetfillcolor{currentfill}%
\pgfsetlinewidth{0.602250pt}%
\definecolor{currentstroke}{rgb}{0.000000,0.000000,0.000000}%
\pgfsetstrokecolor{currentstroke}%
\pgfsetdash{}{0pt}%
\pgfsys@defobject{currentmarker}{\pgfqpoint{0.000000in}{0.000000in}}{\pgfqpoint{0.027778in}{0.000000in}}{%
\pgfpathmoveto{\pgfqpoint{0.000000in}{0.000000in}}%
\pgfpathlineto{\pgfqpoint{0.027778in}{0.000000in}}%
\pgfusepath{stroke,fill}%
}%
\begin{pgfscope}%
\pgfsys@transformshift{5.587445in}{1.732372in}%
\pgfsys@useobject{currentmarker}{}%
\end{pgfscope}%
\end{pgfscope}%
\begin{pgfscope}%
\pgfsetbuttcap%
\pgfsetroundjoin%
\definecolor{currentfill}{rgb}{0.000000,0.000000,0.000000}%
\pgfsetfillcolor{currentfill}%
\pgfsetlinewidth{0.602250pt}%
\definecolor{currentstroke}{rgb}{0.000000,0.000000,0.000000}%
\pgfsetstrokecolor{currentstroke}%
\pgfsetdash{}{0pt}%
\pgfsys@defobject{currentmarker}{\pgfqpoint{0.000000in}{0.000000in}}{\pgfqpoint{0.027778in}{0.000000in}}{%
\pgfpathmoveto{\pgfqpoint{0.000000in}{0.000000in}}%
\pgfpathlineto{\pgfqpoint{0.027778in}{0.000000in}}%
\pgfusepath{stroke,fill}%
}%
\begin{pgfscope}%
\pgfsys@transformshift{5.587445in}{1.791817in}%
\pgfsys@useobject{currentmarker}{}%
\end{pgfscope}%
\end{pgfscope}%
\begin{pgfscope}%
\pgfsetbuttcap%
\pgfsetroundjoin%
\definecolor{currentfill}{rgb}{0.000000,0.000000,0.000000}%
\pgfsetfillcolor{currentfill}%
\pgfsetlinewidth{0.602250pt}%
\definecolor{currentstroke}{rgb}{0.000000,0.000000,0.000000}%
\pgfsetstrokecolor{currentstroke}%
\pgfsetdash{}{0pt}%
\pgfsys@defobject{currentmarker}{\pgfqpoint{0.000000in}{0.000000in}}{\pgfqpoint{0.027778in}{0.000000in}}{%
\pgfpathmoveto{\pgfqpoint{0.000000in}{0.000000in}}%
\pgfpathlineto{\pgfqpoint{0.027778in}{0.000000in}}%
\pgfusepath{stroke,fill}%
}%
\begin{pgfscope}%
\pgfsys@transformshift{5.587445in}{1.837926in}%
\pgfsys@useobject{currentmarker}{}%
\end{pgfscope}%
\end{pgfscope}%
\begin{pgfscope}%
\pgfsetbuttcap%
\pgfsetroundjoin%
\definecolor{currentfill}{rgb}{0.000000,0.000000,0.000000}%
\pgfsetfillcolor{currentfill}%
\pgfsetlinewidth{0.602250pt}%
\definecolor{currentstroke}{rgb}{0.000000,0.000000,0.000000}%
\pgfsetstrokecolor{currentstroke}%
\pgfsetdash{}{0pt}%
\pgfsys@defobject{currentmarker}{\pgfqpoint{0.000000in}{0.000000in}}{\pgfqpoint{0.027778in}{0.000000in}}{%
\pgfpathmoveto{\pgfqpoint{0.000000in}{0.000000in}}%
\pgfpathlineto{\pgfqpoint{0.027778in}{0.000000in}}%
\pgfusepath{stroke,fill}%
}%
\begin{pgfscope}%
\pgfsys@transformshift{5.587445in}{1.875600in}%
\pgfsys@useobject{currentmarker}{}%
\end{pgfscope}%
\end{pgfscope}%
\begin{pgfscope}%
\pgfsetbuttcap%
\pgfsetroundjoin%
\definecolor{currentfill}{rgb}{0.000000,0.000000,0.000000}%
\pgfsetfillcolor{currentfill}%
\pgfsetlinewidth{0.602250pt}%
\definecolor{currentstroke}{rgb}{0.000000,0.000000,0.000000}%
\pgfsetstrokecolor{currentstroke}%
\pgfsetdash{}{0pt}%
\pgfsys@defobject{currentmarker}{\pgfqpoint{0.000000in}{0.000000in}}{\pgfqpoint{0.027778in}{0.000000in}}{%
\pgfpathmoveto{\pgfqpoint{0.000000in}{0.000000in}}%
\pgfpathlineto{\pgfqpoint{0.027778in}{0.000000in}}%
\pgfusepath{stroke,fill}%
}%
\begin{pgfscope}%
\pgfsys@transformshift{5.587445in}{1.907452in}%
\pgfsys@useobject{currentmarker}{}%
\end{pgfscope}%
\end{pgfscope}%
\begin{pgfscope}%
\pgfsetbuttcap%
\pgfsetroundjoin%
\definecolor{currentfill}{rgb}{0.000000,0.000000,0.000000}%
\pgfsetfillcolor{currentfill}%
\pgfsetlinewidth{0.602250pt}%
\definecolor{currentstroke}{rgb}{0.000000,0.000000,0.000000}%
\pgfsetstrokecolor{currentstroke}%
\pgfsetdash{}{0pt}%
\pgfsys@defobject{currentmarker}{\pgfqpoint{0.000000in}{0.000000in}}{\pgfqpoint{0.027778in}{0.000000in}}{%
\pgfpathmoveto{\pgfqpoint{0.000000in}{0.000000in}}%
\pgfpathlineto{\pgfqpoint{0.027778in}{0.000000in}}%
\pgfusepath{stroke,fill}%
}%
\begin{pgfscope}%
\pgfsys@transformshift{5.587445in}{1.935044in}%
\pgfsys@useobject{currentmarker}{}%
\end{pgfscope}%
\end{pgfscope}%
\begin{pgfscope}%
\pgfsetbuttcap%
\pgfsetroundjoin%
\definecolor{currentfill}{rgb}{0.000000,0.000000,0.000000}%
\pgfsetfillcolor{currentfill}%
\pgfsetlinewidth{0.602250pt}%
\definecolor{currentstroke}{rgb}{0.000000,0.000000,0.000000}%
\pgfsetstrokecolor{currentstroke}%
\pgfsetdash{}{0pt}%
\pgfsys@defobject{currentmarker}{\pgfqpoint{0.000000in}{0.000000in}}{\pgfqpoint{0.027778in}{0.000000in}}{%
\pgfpathmoveto{\pgfqpoint{0.000000in}{0.000000in}}%
\pgfpathlineto{\pgfqpoint{0.027778in}{0.000000in}}%
\pgfusepath{stroke,fill}%
}%
\begin{pgfscope}%
\pgfsys@transformshift{5.587445in}{1.959382in}%
\pgfsys@useobject{currentmarker}{}%
\end{pgfscope}%
\end{pgfscope}%
\begin{pgfscope}%
\pgfsetbuttcap%
\pgfsetroundjoin%
\definecolor{currentfill}{rgb}{0.000000,0.000000,0.000000}%
\pgfsetfillcolor{currentfill}%
\pgfsetlinewidth{0.602250pt}%
\definecolor{currentstroke}{rgb}{0.000000,0.000000,0.000000}%
\pgfsetstrokecolor{currentstroke}%
\pgfsetdash{}{0pt}%
\pgfsys@defobject{currentmarker}{\pgfqpoint{0.000000in}{0.000000in}}{\pgfqpoint{0.027778in}{0.000000in}}{%
\pgfpathmoveto{\pgfqpoint{0.000000in}{0.000000in}}%
\pgfpathlineto{\pgfqpoint{0.027778in}{0.000000in}}%
\pgfusepath{stroke,fill}%
}%
\begin{pgfscope}%
\pgfsys@transformshift{5.587445in}{2.124381in}%
\pgfsys@useobject{currentmarker}{}%
\end{pgfscope}%
\end{pgfscope}%
\begin{pgfscope}%
\pgfsetbuttcap%
\pgfsetroundjoin%
\definecolor{currentfill}{rgb}{0.000000,0.000000,0.000000}%
\pgfsetfillcolor{currentfill}%
\pgfsetlinewidth{0.602250pt}%
\definecolor{currentstroke}{rgb}{0.000000,0.000000,0.000000}%
\pgfsetstrokecolor{currentstroke}%
\pgfsetdash{}{0pt}%
\pgfsys@defobject{currentmarker}{\pgfqpoint{0.000000in}{0.000000in}}{\pgfqpoint{0.027778in}{0.000000in}}{%
\pgfpathmoveto{\pgfqpoint{0.000000in}{0.000000in}}%
\pgfpathlineto{\pgfqpoint{0.027778in}{0.000000in}}%
\pgfusepath{stroke,fill}%
}%
\begin{pgfscope}%
\pgfsys@transformshift{5.587445in}{2.208163in}%
\pgfsys@useobject{currentmarker}{}%
\end{pgfscope}%
\end{pgfscope}%
\begin{pgfscope}%
\pgfsetbuttcap%
\pgfsetroundjoin%
\definecolor{currentfill}{rgb}{0.000000,0.000000,0.000000}%
\pgfsetfillcolor{currentfill}%
\pgfsetlinewidth{0.602250pt}%
\definecolor{currentstroke}{rgb}{0.000000,0.000000,0.000000}%
\pgfsetstrokecolor{currentstroke}%
\pgfsetdash{}{0pt}%
\pgfsys@defobject{currentmarker}{\pgfqpoint{0.000000in}{0.000000in}}{\pgfqpoint{0.027778in}{0.000000in}}{%
\pgfpathmoveto{\pgfqpoint{0.000000in}{0.000000in}}%
\pgfpathlineto{\pgfqpoint{0.027778in}{0.000000in}}%
\pgfusepath{stroke,fill}%
}%
\begin{pgfscope}%
\pgfsys@transformshift{5.587445in}{2.267608in}%
\pgfsys@useobject{currentmarker}{}%
\end{pgfscope}%
\end{pgfscope}%
\begin{pgfscope}%
\pgfsetbuttcap%
\pgfsetroundjoin%
\definecolor{currentfill}{rgb}{0.000000,0.000000,0.000000}%
\pgfsetfillcolor{currentfill}%
\pgfsetlinewidth{0.602250pt}%
\definecolor{currentstroke}{rgb}{0.000000,0.000000,0.000000}%
\pgfsetstrokecolor{currentstroke}%
\pgfsetdash{}{0pt}%
\pgfsys@defobject{currentmarker}{\pgfqpoint{0.000000in}{0.000000in}}{\pgfqpoint{0.027778in}{0.000000in}}{%
\pgfpathmoveto{\pgfqpoint{0.000000in}{0.000000in}}%
\pgfpathlineto{\pgfqpoint{0.027778in}{0.000000in}}%
\pgfusepath{stroke,fill}%
}%
\begin{pgfscope}%
\pgfsys@transformshift{5.587445in}{2.313717in}%
\pgfsys@useobject{currentmarker}{}%
\end{pgfscope}%
\end{pgfscope}%
\begin{pgfscope}%
\pgfsetbuttcap%
\pgfsetroundjoin%
\definecolor{currentfill}{rgb}{0.000000,0.000000,0.000000}%
\pgfsetfillcolor{currentfill}%
\pgfsetlinewidth{0.602250pt}%
\definecolor{currentstroke}{rgb}{0.000000,0.000000,0.000000}%
\pgfsetstrokecolor{currentstroke}%
\pgfsetdash{}{0pt}%
\pgfsys@defobject{currentmarker}{\pgfqpoint{0.000000in}{0.000000in}}{\pgfqpoint{0.027778in}{0.000000in}}{%
\pgfpathmoveto{\pgfqpoint{0.000000in}{0.000000in}}%
\pgfpathlineto{\pgfqpoint{0.027778in}{0.000000in}}%
\pgfusepath{stroke,fill}%
}%
\begin{pgfscope}%
\pgfsys@transformshift{5.587445in}{2.351391in}%
\pgfsys@useobject{currentmarker}{}%
\end{pgfscope}%
\end{pgfscope}%
\begin{pgfscope}%
\pgfsetbuttcap%
\pgfsetroundjoin%
\definecolor{currentfill}{rgb}{0.000000,0.000000,0.000000}%
\pgfsetfillcolor{currentfill}%
\pgfsetlinewidth{0.602250pt}%
\definecolor{currentstroke}{rgb}{0.000000,0.000000,0.000000}%
\pgfsetstrokecolor{currentstroke}%
\pgfsetdash{}{0pt}%
\pgfsys@defobject{currentmarker}{\pgfqpoint{0.000000in}{0.000000in}}{\pgfqpoint{0.027778in}{0.000000in}}{%
\pgfpathmoveto{\pgfqpoint{0.000000in}{0.000000in}}%
\pgfpathlineto{\pgfqpoint{0.027778in}{0.000000in}}%
\pgfusepath{stroke,fill}%
}%
\begin{pgfscope}%
\pgfsys@transformshift{5.587445in}{2.383243in}%
\pgfsys@useobject{currentmarker}{}%
\end{pgfscope}%
\end{pgfscope}%
\begin{pgfscope}%
\pgfsetbuttcap%
\pgfsetroundjoin%
\definecolor{currentfill}{rgb}{0.000000,0.000000,0.000000}%
\pgfsetfillcolor{currentfill}%
\pgfsetlinewidth{0.602250pt}%
\definecolor{currentstroke}{rgb}{0.000000,0.000000,0.000000}%
\pgfsetstrokecolor{currentstroke}%
\pgfsetdash{}{0pt}%
\pgfsys@defobject{currentmarker}{\pgfqpoint{0.000000in}{0.000000in}}{\pgfqpoint{0.027778in}{0.000000in}}{%
\pgfpathmoveto{\pgfqpoint{0.000000in}{0.000000in}}%
\pgfpathlineto{\pgfqpoint{0.027778in}{0.000000in}}%
\pgfusepath{stroke,fill}%
}%
\begin{pgfscope}%
\pgfsys@transformshift{5.587445in}{2.410835in}%
\pgfsys@useobject{currentmarker}{}%
\end{pgfscope}%
\end{pgfscope}%
\begin{pgfscope}%
\pgfsetbuttcap%
\pgfsetroundjoin%
\definecolor{currentfill}{rgb}{0.000000,0.000000,0.000000}%
\pgfsetfillcolor{currentfill}%
\pgfsetlinewidth{0.602250pt}%
\definecolor{currentstroke}{rgb}{0.000000,0.000000,0.000000}%
\pgfsetstrokecolor{currentstroke}%
\pgfsetdash{}{0pt}%
\pgfsys@defobject{currentmarker}{\pgfqpoint{0.000000in}{0.000000in}}{\pgfqpoint{0.027778in}{0.000000in}}{%
\pgfpathmoveto{\pgfqpoint{0.000000in}{0.000000in}}%
\pgfpathlineto{\pgfqpoint{0.027778in}{0.000000in}}%
\pgfusepath{stroke,fill}%
}%
\begin{pgfscope}%
\pgfsys@transformshift{5.587445in}{2.435173in}%
\pgfsys@useobject{currentmarker}{}%
\end{pgfscope}%
\end{pgfscope}%
\begin{pgfscope}%
\pgfsetbuttcap%
\pgfsetroundjoin%
\definecolor{currentfill}{rgb}{0.000000,0.000000,0.000000}%
\pgfsetfillcolor{currentfill}%
\pgfsetlinewidth{0.602250pt}%
\definecolor{currentstroke}{rgb}{0.000000,0.000000,0.000000}%
\pgfsetstrokecolor{currentstroke}%
\pgfsetdash{}{0pt}%
\pgfsys@defobject{currentmarker}{\pgfqpoint{0.000000in}{0.000000in}}{\pgfqpoint{0.027778in}{0.000000in}}{%
\pgfpathmoveto{\pgfqpoint{0.000000in}{0.000000in}}%
\pgfpathlineto{\pgfqpoint{0.027778in}{0.000000in}}%
\pgfusepath{stroke,fill}%
}%
\begin{pgfscope}%
\pgfsys@transformshift{5.587445in}{2.600171in}%
\pgfsys@useobject{currentmarker}{}%
\end{pgfscope}%
\end{pgfscope}%
\begin{pgfscope}%
\definecolor{textcolor}{rgb}{0.000000,0.000000,0.000000}%
\pgfsetstrokecolor{textcolor}%
\pgfsetfillcolor{textcolor}%
\pgftext[x=5.916150in,y=1.601557in,,top,rotate=90.000000]{\color{textcolor}\rmfamily\fontsize{10.950000}{13.140000}\selectfont Events}%
\end{pgfscope}%
\begin{pgfscope}%
\pgfsetrectcap%
\pgfsetmiterjoin%
\pgfsetlinewidth{0.803000pt}%
\definecolor{currentstroke}{rgb}{0.000000,0.000000,0.000000}%
\pgfsetstrokecolor{currentstroke}%
\pgfsetdash{}{0pt}%
\pgfpathmoveto{\pgfqpoint{0.572918in}{0.553781in}}%
\pgfpathlineto{\pgfqpoint{0.572918in}{2.649333in}}%
\pgfusepath{stroke}%
\end{pgfscope}%
\begin{pgfscope}%
\pgfsetrectcap%
\pgfsetmiterjoin%
\pgfsetlinewidth{0.803000pt}%
\definecolor{currentstroke}{rgb}{0.000000,0.000000,0.000000}%
\pgfsetstrokecolor{currentstroke}%
\pgfsetdash{}{0pt}%
\pgfpathmoveto{\pgfqpoint{5.587445in}{0.553781in}}%
\pgfpathlineto{\pgfqpoint{5.587445in}{2.649333in}}%
\pgfusepath{stroke}%
\end{pgfscope}%
\begin{pgfscope}%
\pgfsetrectcap%
\pgfsetmiterjoin%
\pgfsetlinewidth{0.803000pt}%
\definecolor{currentstroke}{rgb}{0.000000,0.000000,0.000000}%
\pgfsetstrokecolor{currentstroke}%
\pgfsetdash{}{0pt}%
\pgfpathmoveto{\pgfqpoint{0.572918in}{0.553781in}}%
\pgfpathlineto{\pgfqpoint{5.587445in}{0.553781in}}%
\pgfusepath{stroke}%
\end{pgfscope}%
\begin{pgfscope}%
\pgfsetrectcap%
\pgfsetmiterjoin%
\pgfsetlinewidth{0.803000pt}%
\definecolor{currentstroke}{rgb}{0.000000,0.000000,0.000000}%
\pgfsetstrokecolor{currentstroke}%
\pgfsetdash{}{0pt}%
\pgfpathmoveto{\pgfqpoint{0.572918in}{2.649333in}}%
\pgfpathlineto{\pgfqpoint{5.587445in}{2.649333in}}%
\pgfusepath{stroke}%
\end{pgfscope}%
\end{pgfpicture}%
\makeatother%
\endgroup%

    \caption{Zenith resolution performance of CNN 2.0 on the upgrade dataset.
    The performance is worse than on DeepCore data, even with new hardware features fed to the network, probably due to excessive noise.
    The network \textit{does} have some predictive power, though, starting to improve performance around \SI{10}{\giga\electronvolt}.
    This is comforting for the future, as a noise removal pipeline akin to oscNext should drastically improve the results.}\label{fig:upgrade_resolution}
\end{figure}

The CNN 2.0 algorithm turned out to work best on the Upgrade dataset.
Here, only zenith reconstruction (see~\vref{fig:upgrade_resolution}) gave some usable results, while energy construction showed poor predictive power as seen in~\vref{fig:upgrade_2d_hist}.
The network tends to predict one value around \SI{30}{\giga\electronvolt} for events where it cannot do a meaningful reconstruction, thereby minimizing the loss.
This is probably caused by the excessive amount of noise in the dataset, as no cuts akin to oscNext have been performed on the set (no similar pipeline exists).
However, as expected the methodology transferred with no issues, amounting to simply adding new features (the directional pointing vectors of the PMTs) as inputs to the network; this bodes well for the future of deep learning as applied to the Upgrade set, pending work on cuts to bring the set in line with real data and the removal of noise.

\begin{figure}
    \centering
    \includegraphics[width=1.0\textwidth]{./images/results/energy_log10_prediction_2d_histogram.pdf}
    \caption{An example from an energy reconstruction run on the upgrade dataset.
    The network does not perform well, probably because of noise, defaulting to a loss-minimizing value for events where no clear reconstruction can be performed.
    This should be remedied by noise removal.}\label{fig:upgrade_2d_hist}
\end{figure}



\end{document}
