%!TEX root = ../main.tex
\documentclass[../main.tex]{subfiles}

\begin{document}

\section{Defining the problem}\label{sec:defining_the_problems}

With the given dataset, it is possible to train a neural network to predict the following properties of an incoming neutrino: energy, interaction vertex, interaction time and direction.
As the oscillation studies---as described in~\vref{sec:neutrino_oscillations}---depend on the energy and the polar angle (as a proxy for the length traveled through the Earth), these two variables are the ones examined in this work.

The algorithm performance will be evaluated based on \( \log_{10}(E) \) and \( \theta \) resolution in 18 energy bins in the range \SIrange{1}{1000}{\giga\electronvolt} of simulated charged current muon neutrinos \enquote{detected} in DeepCore, 6 bins per order of magnitude.
To gauge the performance the algorithm will be tested against the current best DeepCore reconstruction algorithm, Retro Reco (\vref{sec:l6}).

In the following sections first the custom made tooling used to solve the problem is described, whereafter hyperparameter- and model experiments are performed.

The full DeepCore, charged current muon neutrino dataset (\vref{sec:distributions_and_selections}) where Retro Reco reconstruction is available contains \num{7280939} events in total, where \num{2050125} are blinded, set aside as a test set, leaving \num{5230814} for training.
Of these \num{1046163} are used for validation, which means \num{4184651} are trainable.
To save time, most experiments were run on a set containing around \num{400000} events, and all experiments employ early stopping.

\begin{figure}
     \centering
     \begin{subfigure}[b]{0.49\textwidth}
         \centering
         %% Creator: Matplotlib, PGF backend
%%
%% To include the figure in your LaTeX document, write
%%   \input{<filename>.pgf}
%%
%% Make sure the required packages are loaded in your preamble
%%   \usepackage{pgf}
%%
%% and, on pdftex
%%   \usepackage[utf8]{inputenc}\DeclareUnicodeCharacter{2212}{-}
%%
%% or, on luatex and xetex
%%   \usepackage{unicode-math}
%%
%% Figures using additional raster images can only be included by \input if
%% they are in the same directory as the main LaTeX file. For loading figures
%% from other directories you can use the `import` package
%%   \usepackage{import}
%%
%% and then include the figures with
%%   \import{<path to file>}{<filename>.pgf}
%%
%% Matplotlib used the following preamble
%%   \usepackage{siunitx} \usepackage{amsmath} \usepackage{bm}
%%   \usepackage{fontspec}
%%
\begingroup%
\makeatletter%
\begin{pgfpicture}%
\pgfpathrectangle{\pgfpointorigin}{\pgfqpoint{3.038588in}{2.500000in}}%
\pgfusepath{use as bounding box, clip}%
\begin{pgfscope}%
\pgfsetbuttcap%
\pgfsetmiterjoin%
\definecolor{currentfill}{rgb}{1.000000,1.000000,1.000000}%
\pgfsetfillcolor{currentfill}%
\pgfsetlinewidth{0.000000pt}%
\definecolor{currentstroke}{rgb}{1.000000,1.000000,1.000000}%
\pgfsetstrokecolor{currentstroke}%
\pgfsetdash{}{0pt}%
\pgfpathmoveto{\pgfqpoint{0.000000in}{0.000000in}}%
\pgfpathlineto{\pgfqpoint{3.038588in}{0.000000in}}%
\pgfpathlineto{\pgfqpoint{3.038588in}{2.500000in}}%
\pgfpathlineto{\pgfqpoint{0.000000in}{2.500000in}}%
\pgfpathclose%
\pgfusepath{fill}%
\end{pgfscope}%
\begin{pgfscope}%
\pgfsetbuttcap%
\pgfsetmiterjoin%
\definecolor{currentfill}{rgb}{1.000000,1.000000,1.000000}%
\pgfsetfillcolor{currentfill}%
\pgfsetlinewidth{0.000000pt}%
\definecolor{currentstroke}{rgb}{0.000000,0.000000,0.000000}%
\pgfsetstrokecolor{currentstroke}%
\pgfsetstrokeopacity{0.000000}%
\pgfsetdash{}{0pt}%
\pgfpathmoveto{\pgfqpoint{0.647707in}{0.536494in}}%
\pgfpathlineto{\pgfqpoint{2.888588in}{0.536494in}}%
\pgfpathlineto{\pgfqpoint{2.888588in}{2.151000in}}%
\pgfpathlineto{\pgfqpoint{0.647707in}{2.151000in}}%
\pgfpathclose%
\pgfusepath{fill}%
\end{pgfscope}%
\begin{pgfscope}%
\pgfsetbuttcap%
\pgfsetroundjoin%
\definecolor{currentfill}{rgb}{0.000000,0.000000,0.000000}%
\pgfsetfillcolor{currentfill}%
\pgfsetlinewidth{0.803000pt}%
\definecolor{currentstroke}{rgb}{0.000000,0.000000,0.000000}%
\pgfsetstrokecolor{currentstroke}%
\pgfsetdash{}{0pt}%
\pgfsys@defobject{currentmarker}{\pgfqpoint{0.000000in}{-0.048611in}}{\pgfqpoint{0.000000in}{0.000000in}}{%
\pgfpathmoveto{\pgfqpoint{0.000000in}{0.000000in}}%
\pgfpathlineto{\pgfqpoint{0.000000in}{-0.048611in}}%
\pgfusepath{stroke,fill}%
}%
\begin{pgfscope}%
\pgfsys@transformshift{0.749566in}{0.536494in}%
\pgfsys@useobject{currentmarker}{}%
\end{pgfscope}%
\end{pgfscope}%
\begin{pgfscope}%
\definecolor{textcolor}{rgb}{0.000000,0.000000,0.000000}%
\pgfsetstrokecolor{textcolor}%
\pgfsetfillcolor{textcolor}%
\pgftext[x=0.749566in,y=0.439272in,,top]{\color{textcolor}\rmfamily\fontsize{8.000000}{9.600000}\selectfont \(\displaystyle {0}\)}%
\end{pgfscope}%
\begin{pgfscope}%
\pgfsetbuttcap%
\pgfsetroundjoin%
\definecolor{currentfill}{rgb}{0.000000,0.000000,0.000000}%
\pgfsetfillcolor{currentfill}%
\pgfsetlinewidth{0.803000pt}%
\definecolor{currentstroke}{rgb}{0.000000,0.000000,0.000000}%
\pgfsetstrokecolor{currentstroke}%
\pgfsetdash{}{0pt}%
\pgfsys@defobject{currentmarker}{\pgfqpoint{0.000000in}{-0.048611in}}{\pgfqpoint{0.000000in}{0.000000in}}{%
\pgfpathmoveto{\pgfqpoint{0.000000in}{0.000000in}}%
\pgfpathlineto{\pgfqpoint{0.000000in}{-0.048611in}}%
\pgfusepath{stroke,fill}%
}%
\begin{pgfscope}%
\pgfsys@transformshift{1.386179in}{0.536494in}%
\pgfsys@useobject{currentmarker}{}%
\end{pgfscope}%
\end{pgfscope}%
\begin{pgfscope}%
\definecolor{textcolor}{rgb}{0.000000,0.000000,0.000000}%
\pgfsetstrokecolor{textcolor}%
\pgfsetfillcolor{textcolor}%
\pgftext[x=1.386179in,y=0.439272in,,top]{\color{textcolor}\rmfamily\fontsize{8.000000}{9.600000}\selectfont \(\displaystyle {5}\)}%
\end{pgfscope}%
\begin{pgfscope}%
\pgfsetbuttcap%
\pgfsetroundjoin%
\definecolor{currentfill}{rgb}{0.000000,0.000000,0.000000}%
\pgfsetfillcolor{currentfill}%
\pgfsetlinewidth{0.803000pt}%
\definecolor{currentstroke}{rgb}{0.000000,0.000000,0.000000}%
\pgfsetstrokecolor{currentstroke}%
\pgfsetdash{}{0pt}%
\pgfsys@defobject{currentmarker}{\pgfqpoint{0.000000in}{-0.048611in}}{\pgfqpoint{0.000000in}{0.000000in}}{%
\pgfpathmoveto{\pgfqpoint{0.000000in}{0.000000in}}%
\pgfpathlineto{\pgfqpoint{0.000000in}{-0.048611in}}%
\pgfusepath{stroke,fill}%
}%
\begin{pgfscope}%
\pgfsys@transformshift{2.022793in}{0.536494in}%
\pgfsys@useobject{currentmarker}{}%
\end{pgfscope}%
\end{pgfscope}%
\begin{pgfscope}%
\definecolor{textcolor}{rgb}{0.000000,0.000000,0.000000}%
\pgfsetstrokecolor{textcolor}%
\pgfsetfillcolor{textcolor}%
\pgftext[x=2.022793in,y=0.439272in,,top]{\color{textcolor}\rmfamily\fontsize{8.000000}{9.600000}\selectfont \(\displaystyle {10}\)}%
\end{pgfscope}%
\begin{pgfscope}%
\pgfsetbuttcap%
\pgfsetroundjoin%
\definecolor{currentfill}{rgb}{0.000000,0.000000,0.000000}%
\pgfsetfillcolor{currentfill}%
\pgfsetlinewidth{0.803000pt}%
\definecolor{currentstroke}{rgb}{0.000000,0.000000,0.000000}%
\pgfsetstrokecolor{currentstroke}%
\pgfsetdash{}{0pt}%
\pgfsys@defobject{currentmarker}{\pgfqpoint{0.000000in}{-0.048611in}}{\pgfqpoint{0.000000in}{0.000000in}}{%
\pgfpathmoveto{\pgfqpoint{0.000000in}{0.000000in}}%
\pgfpathlineto{\pgfqpoint{0.000000in}{-0.048611in}}%
\pgfusepath{stroke,fill}%
}%
\begin{pgfscope}%
\pgfsys@transformshift{2.659407in}{0.536494in}%
\pgfsys@useobject{currentmarker}{}%
\end{pgfscope}%
\end{pgfscope}%
\begin{pgfscope}%
\definecolor{textcolor}{rgb}{0.000000,0.000000,0.000000}%
\pgfsetstrokecolor{textcolor}%
\pgfsetfillcolor{textcolor}%
\pgftext[x=2.659407in,y=0.439272in,,top]{\color{textcolor}\rmfamily\fontsize{8.000000}{9.600000}\selectfont \(\displaystyle {15}\)}%
\end{pgfscope}%
\begin{pgfscope}%
\definecolor{textcolor}{rgb}{0.000000,0.000000,0.000000}%
\pgfsetstrokecolor{textcolor}%
\pgfsetfillcolor{textcolor}%
\pgftext[x=1.768148in,y=0.285050in,,top]{\color{textcolor}\rmfamily\fontsize{10.950000}{13.140000}\selectfont Epoch}%
\end{pgfscope}%
\begin{pgfscope}%
\pgfsetbuttcap%
\pgfsetroundjoin%
\definecolor{currentfill}{rgb}{0.000000,0.000000,0.000000}%
\pgfsetfillcolor{currentfill}%
\pgfsetlinewidth{0.803000pt}%
\definecolor{currentstroke}{rgb}{0.000000,0.000000,0.000000}%
\pgfsetstrokecolor{currentstroke}%
\pgfsetdash{}{0pt}%
\pgfsys@defobject{currentmarker}{\pgfqpoint{-0.048611in}{0.000000in}}{\pgfqpoint{-0.000000in}{0.000000in}}{%
\pgfpathmoveto{\pgfqpoint{-0.000000in}{0.000000in}}%
\pgfpathlineto{\pgfqpoint{-0.048611in}{0.000000in}}%
\pgfusepath{stroke,fill}%
}%
\begin{pgfscope}%
\pgfsys@transformshift{0.647707in}{0.536680in}%
\pgfsys@useobject{currentmarker}{}%
\end{pgfscope}%
\end{pgfscope}%
\begin{pgfscope}%
\definecolor{textcolor}{rgb}{0.000000,0.000000,0.000000}%
\pgfsetstrokecolor{textcolor}%
\pgfsetfillcolor{textcolor}%
\pgftext[x=0.340606in, y=0.498125in, left, base]{\color{textcolor}\rmfamily\fontsize{8.000000}{9.600000}\selectfont \(\displaystyle {0.06}\)}%
\end{pgfscope}%
\begin{pgfscope}%
\pgfsetbuttcap%
\pgfsetroundjoin%
\definecolor{currentfill}{rgb}{0.000000,0.000000,0.000000}%
\pgfsetfillcolor{currentfill}%
\pgfsetlinewidth{0.803000pt}%
\definecolor{currentstroke}{rgb}{0.000000,0.000000,0.000000}%
\pgfsetstrokecolor{currentstroke}%
\pgfsetdash{}{0pt}%
\pgfsys@defobject{currentmarker}{\pgfqpoint{-0.048611in}{0.000000in}}{\pgfqpoint{-0.000000in}{0.000000in}}{%
\pgfpathmoveto{\pgfqpoint{-0.000000in}{0.000000in}}%
\pgfpathlineto{\pgfqpoint{-0.048611in}{0.000000in}}%
\pgfusepath{stroke,fill}%
}%
\begin{pgfscope}%
\pgfsys@transformshift{0.647707in}{0.879110in}%
\pgfsys@useobject{currentmarker}{}%
\end{pgfscope}%
\end{pgfscope}%
\begin{pgfscope}%
\definecolor{textcolor}{rgb}{0.000000,0.000000,0.000000}%
\pgfsetstrokecolor{textcolor}%
\pgfsetfillcolor{textcolor}%
\pgftext[x=0.340606in, y=0.840554in, left, base]{\color{textcolor}\rmfamily\fontsize{8.000000}{9.600000}\selectfont \(\displaystyle {0.08}\)}%
\end{pgfscope}%
\begin{pgfscope}%
\pgfsetbuttcap%
\pgfsetroundjoin%
\definecolor{currentfill}{rgb}{0.000000,0.000000,0.000000}%
\pgfsetfillcolor{currentfill}%
\pgfsetlinewidth{0.803000pt}%
\definecolor{currentstroke}{rgb}{0.000000,0.000000,0.000000}%
\pgfsetstrokecolor{currentstroke}%
\pgfsetdash{}{0pt}%
\pgfsys@defobject{currentmarker}{\pgfqpoint{-0.048611in}{0.000000in}}{\pgfqpoint{-0.000000in}{0.000000in}}{%
\pgfpathmoveto{\pgfqpoint{-0.000000in}{0.000000in}}%
\pgfpathlineto{\pgfqpoint{-0.048611in}{0.000000in}}%
\pgfusepath{stroke,fill}%
}%
\begin{pgfscope}%
\pgfsys@transformshift{0.647707in}{1.221539in}%
\pgfsys@useobject{currentmarker}{}%
\end{pgfscope}%
\end{pgfscope}%
\begin{pgfscope}%
\definecolor{textcolor}{rgb}{0.000000,0.000000,0.000000}%
\pgfsetstrokecolor{textcolor}%
\pgfsetfillcolor{textcolor}%
\pgftext[x=0.340606in, y=1.182984in, left, base]{\color{textcolor}\rmfamily\fontsize{8.000000}{9.600000}\selectfont \(\displaystyle {0.10}\)}%
\end{pgfscope}%
\begin{pgfscope}%
\pgfsetbuttcap%
\pgfsetroundjoin%
\definecolor{currentfill}{rgb}{0.000000,0.000000,0.000000}%
\pgfsetfillcolor{currentfill}%
\pgfsetlinewidth{0.803000pt}%
\definecolor{currentstroke}{rgb}{0.000000,0.000000,0.000000}%
\pgfsetstrokecolor{currentstroke}%
\pgfsetdash{}{0pt}%
\pgfsys@defobject{currentmarker}{\pgfqpoint{-0.048611in}{0.000000in}}{\pgfqpoint{-0.000000in}{0.000000in}}{%
\pgfpathmoveto{\pgfqpoint{-0.000000in}{0.000000in}}%
\pgfpathlineto{\pgfqpoint{-0.048611in}{0.000000in}}%
\pgfusepath{stroke,fill}%
}%
\begin{pgfscope}%
\pgfsys@transformshift{0.647707in}{1.563969in}%
\pgfsys@useobject{currentmarker}{}%
\end{pgfscope}%
\end{pgfscope}%
\begin{pgfscope}%
\definecolor{textcolor}{rgb}{0.000000,0.000000,0.000000}%
\pgfsetstrokecolor{textcolor}%
\pgfsetfillcolor{textcolor}%
\pgftext[x=0.340606in, y=1.525413in, left, base]{\color{textcolor}\rmfamily\fontsize{8.000000}{9.600000}\selectfont \(\displaystyle {0.12}\)}%
\end{pgfscope}%
\begin{pgfscope}%
\pgfsetbuttcap%
\pgfsetroundjoin%
\definecolor{currentfill}{rgb}{0.000000,0.000000,0.000000}%
\pgfsetfillcolor{currentfill}%
\pgfsetlinewidth{0.803000pt}%
\definecolor{currentstroke}{rgb}{0.000000,0.000000,0.000000}%
\pgfsetstrokecolor{currentstroke}%
\pgfsetdash{}{0pt}%
\pgfsys@defobject{currentmarker}{\pgfqpoint{-0.048611in}{0.000000in}}{\pgfqpoint{-0.000000in}{0.000000in}}{%
\pgfpathmoveto{\pgfqpoint{-0.000000in}{0.000000in}}%
\pgfpathlineto{\pgfqpoint{-0.048611in}{0.000000in}}%
\pgfusepath{stroke,fill}%
}%
\begin{pgfscope}%
\pgfsys@transformshift{0.647707in}{1.906399in}%
\pgfsys@useobject{currentmarker}{}%
\end{pgfscope}%
\end{pgfscope}%
\begin{pgfscope}%
\definecolor{textcolor}{rgb}{0.000000,0.000000,0.000000}%
\pgfsetstrokecolor{textcolor}%
\pgfsetfillcolor{textcolor}%
\pgftext[x=0.340606in, y=1.867843in, left, base]{\color{textcolor}\rmfamily\fontsize{8.000000}{9.600000}\selectfont \(\displaystyle {0.14}\)}%
\end{pgfscope}%
\begin{pgfscope}%
\definecolor{textcolor}{rgb}{0.000000,0.000000,0.000000}%
\pgfsetstrokecolor{textcolor}%
\pgfsetfillcolor{textcolor}%
\pgftext[x=0.285050in,y=1.343747in,,bottom,rotate=90.000000]{\color{textcolor}\rmfamily\fontsize{10.950000}{13.140000}\selectfont Loss}%
\end{pgfscope}%
\begin{pgfscope}%
\pgfpathrectangle{\pgfqpoint{0.647707in}{0.536494in}}{\pgfqpoint{2.240881in}{1.614506in}}%
\pgfusepath{clip}%
\pgfsetrectcap%
\pgfsetroundjoin%
\pgfsetlinewidth{0.501875pt}%
\definecolor{currentstroke}{rgb}{0.313725,0.317647,0.309804}%
\pgfsetstrokecolor{currentstroke}%
\pgfsetdash{}{0pt}%
\pgfpathmoveto{\pgfqpoint{0.749566in}{1.827910in}}%
\pgfpathlineto{\pgfqpoint{0.876888in}{0.926452in}}%
\pgfpathlineto{\pgfqpoint{1.004211in}{0.808677in}}%
\pgfpathlineto{\pgfqpoint{1.131534in}{0.761975in}}%
\pgfpathlineto{\pgfqpoint{1.258857in}{0.733856in}}%
\pgfpathlineto{\pgfqpoint{1.386179in}{0.714401in}}%
\pgfpathlineto{\pgfqpoint{1.513502in}{0.698917in}}%
\pgfpathlineto{\pgfqpoint{1.640825in}{0.686248in}}%
\pgfpathlineto{\pgfqpoint{1.768148in}{0.675448in}}%
\pgfpathlineto{\pgfqpoint{1.895470in}{0.665863in}}%
\pgfpathlineto{\pgfqpoint{2.022793in}{0.656715in}}%
\pgfpathlineto{\pgfqpoint{2.150116in}{0.648596in}}%
\pgfpathlineto{\pgfqpoint{2.277439in}{0.640453in}}%
\pgfpathlineto{\pgfqpoint{2.404761in}{0.632759in}}%
\pgfpathlineto{\pgfqpoint{2.532084in}{0.625154in}}%
\pgfpathlineto{\pgfqpoint{2.659407in}{0.617448in}}%
\pgfpathlineto{\pgfqpoint{2.786730in}{0.609881in}}%
\pgfusepath{stroke}%
\end{pgfscope}%
\begin{pgfscope}%
\pgfpathrectangle{\pgfqpoint{0.647707in}{0.536494in}}{\pgfqpoint{2.240881in}{1.614506in}}%
\pgfusepath{clip}%
\pgfsetbuttcap%
\pgfsetroundjoin%
\pgfsetlinewidth{0.501875pt}%
\definecolor{currentstroke}{rgb}{0.949020,0.372549,0.360784}%
\pgfsetstrokecolor{currentstroke}%
\pgfsetdash{{0.500000pt}{0.825000pt}}{0.000000pt}%
\pgfpathmoveto{\pgfqpoint{0.749566in}{2.077613in}}%
\pgfpathlineto{\pgfqpoint{0.876888in}{1.372849in}}%
\pgfpathlineto{\pgfqpoint{1.004211in}{0.944227in}}%
\pgfpathlineto{\pgfqpoint{1.131534in}{0.787863in}}%
\pgfpathlineto{\pgfqpoint{1.258857in}{0.754856in}}%
\pgfpathlineto{\pgfqpoint{1.386179in}{0.744827in}}%
\pgfpathlineto{\pgfqpoint{1.513502in}{0.741577in}}%
\pgfpathlineto{\pgfqpoint{1.640825in}{0.715994in}}%
\pgfpathlineto{\pgfqpoint{1.768148in}{0.708267in}}%
\pgfpathlineto{\pgfqpoint{1.895470in}{0.703392in}}%
\pgfpathlineto{\pgfqpoint{2.022793in}{0.707947in}}%
\pgfpathlineto{\pgfqpoint{2.150116in}{0.698633in}}%
\pgfpathlineto{\pgfqpoint{2.277439in}{0.710104in}}%
\pgfpathlineto{\pgfqpoint{2.404761in}{0.701703in}}%
\pgfpathlineto{\pgfqpoint{2.532084in}{0.701896in}}%
\pgfpathlineto{\pgfqpoint{2.659407in}{0.710752in}}%
\pgfpathlineto{\pgfqpoint{2.786730in}{0.708584in}}%
\pgfusepath{stroke}%
\end{pgfscope}%
\begin{pgfscope}%
\pgfsetrectcap%
\pgfsetmiterjoin%
\pgfsetlinewidth{0.803000pt}%
\definecolor{currentstroke}{rgb}{0.000000,0.000000,0.000000}%
\pgfsetstrokecolor{currentstroke}%
\pgfsetdash{}{0pt}%
\pgfpathmoveto{\pgfqpoint{0.647707in}{0.536494in}}%
\pgfpathlineto{\pgfqpoint{0.647707in}{2.151000in}}%
\pgfusepath{stroke}%
\end{pgfscope}%
\begin{pgfscope}%
\pgfsetrectcap%
\pgfsetmiterjoin%
\pgfsetlinewidth{0.803000pt}%
\definecolor{currentstroke}{rgb}{0.000000,0.000000,0.000000}%
\pgfsetstrokecolor{currentstroke}%
\pgfsetdash{}{0pt}%
\pgfpathmoveto{\pgfqpoint{2.888588in}{0.536494in}}%
\pgfpathlineto{\pgfqpoint{2.888588in}{2.151000in}}%
\pgfusepath{stroke}%
\end{pgfscope}%
\begin{pgfscope}%
\pgfsetrectcap%
\pgfsetmiterjoin%
\pgfsetlinewidth{0.803000pt}%
\definecolor{currentstroke}{rgb}{0.000000,0.000000,0.000000}%
\pgfsetstrokecolor{currentstroke}%
\pgfsetdash{}{0pt}%
\pgfpathmoveto{\pgfqpoint{0.647707in}{0.536494in}}%
\pgfpathlineto{\pgfqpoint{2.888588in}{0.536494in}}%
\pgfusepath{stroke}%
\end{pgfscope}%
\begin{pgfscope}%
\pgfsetrectcap%
\pgfsetmiterjoin%
\pgfsetlinewidth{0.803000pt}%
\definecolor{currentstroke}{rgb}{0.000000,0.000000,0.000000}%
\pgfsetstrokecolor{currentstroke}%
\pgfsetdash{}{0pt}%
\pgfpathmoveto{\pgfqpoint{0.647707in}{2.151000in}}%
\pgfpathlineto{\pgfqpoint{2.888588in}{2.151000in}}%
\pgfusepath{stroke}%
\end{pgfscope}%
\begin{pgfscope}%
\definecolor{textcolor}{rgb}{0.000000,0.000000,0.000000}%
\pgfsetstrokecolor{textcolor}%
\pgfsetfillcolor{textcolor}%
\pgftext[x=0.647707in,y=2.234333in,left,base]{\color{textcolor}\rmfamily\fontsize{12.000000}{14.400000}\selectfont Run 2011221753 loss}%
\end{pgfscope}%
\begin{pgfscope}%
\pgfsetbuttcap%
\pgfsetmiterjoin%
\definecolor{currentfill}{rgb}{1.000000,1.000000,1.000000}%
\pgfsetfillcolor{currentfill}%
\pgfsetfillopacity{0.800000}%
\pgfsetlinewidth{1.003750pt}%
\definecolor{currentstroke}{rgb}{0.800000,0.800000,0.800000}%
\pgfsetstrokecolor{currentstroke}%
\pgfsetstrokeopacity{0.800000}%
\pgfsetdash{}{0pt}%
\pgfpathmoveto{\pgfqpoint{1.700143in}{1.751111in}}%
\pgfpathlineto{\pgfqpoint{2.810810in}{1.751111in}}%
\pgfpathquadraticcurveto{\pgfqpoint{2.833032in}{1.751111in}}{\pgfqpoint{2.833032in}{1.773333in}}%
\pgfpathlineto{\pgfqpoint{2.833032in}{2.073222in}}%
\pgfpathquadraticcurveto{\pgfqpoint{2.833032in}{2.095444in}}{\pgfqpoint{2.810810in}{2.095444in}}%
\pgfpathlineto{\pgfqpoint{1.700143in}{2.095444in}}%
\pgfpathquadraticcurveto{\pgfqpoint{1.677921in}{2.095444in}}{\pgfqpoint{1.677921in}{2.073222in}}%
\pgfpathlineto{\pgfqpoint{1.677921in}{1.773333in}}%
\pgfpathquadraticcurveto{\pgfqpoint{1.677921in}{1.751111in}}{\pgfqpoint{1.700143in}{1.751111in}}%
\pgfpathclose%
\pgfusepath{stroke,fill}%
\end{pgfscope}%
\begin{pgfscope}%
\pgfsetrectcap%
\pgfsetroundjoin%
\pgfsetlinewidth{0.501875pt}%
\definecolor{currentstroke}{rgb}{0.313725,0.317647,0.309804}%
\pgfsetstrokecolor{currentstroke}%
\pgfsetdash{}{0pt}%
\pgfpathmoveto{\pgfqpoint{1.722366in}{2.012111in}}%
\pgfpathlineto{\pgfqpoint{1.944588in}{2.012111in}}%
\pgfusepath{stroke}%
\end{pgfscope}%
\begin{pgfscope}%
\definecolor{textcolor}{rgb}{0.000000,0.000000,0.000000}%
\pgfsetstrokecolor{textcolor}%
\pgfsetfillcolor{textcolor}%
\pgftext[x=2.033477in,y=1.973222in,left,base]{\color{textcolor}\rmfamily\fontsize{8.000000}{9.600000}\selectfont Training loss}%
\end{pgfscope}%
\begin{pgfscope}%
\pgfsetbuttcap%
\pgfsetroundjoin%
\pgfsetlinewidth{0.501875pt}%
\definecolor{currentstroke}{rgb}{0.949020,0.372549,0.360784}%
\pgfsetstrokecolor{currentstroke}%
\pgfsetdash{{0.500000pt}{0.825000pt}}{0.000000pt}%
\pgfpathmoveto{\pgfqpoint{1.722366in}{1.856000in}}%
\pgfpathlineto{\pgfqpoint{1.944588in}{1.856000in}}%
\pgfusepath{stroke}%
\end{pgfscope}%
\begin{pgfscope}%
\definecolor{textcolor}{rgb}{0.000000,0.000000,0.000000}%
\pgfsetstrokecolor{textcolor}%
\pgfsetfillcolor{textcolor}%
\pgftext[x=2.033477in,y=1.817111in,left,base]{\color{textcolor}\rmfamily\fontsize{8.000000}{9.600000}\selectfont Validation loss}%
\end{pgfscope}%
\end{pgfpicture}%
\makeatother%
\endgroup%

         \caption{}\label{fig:run_loss}
     \end{subfigure}
     \hfill
     \begin{subfigure}[b]{0.49\textwidth}
         \centering
         \includegraphics[width=0.85\textwidth]{./images/design/1200px-Overfitting.svg.png}
         \caption{}\label{fig:overfitting}
     \end{subfigure}
        \caption{(a) shows an actual example of the training and validation loss values during an experiment.
        The two diverge before the 10th epoch with the validation loss staying somewhat flat, and the training loss decreasing.
        This indicates that the network has begun overfitting to the training data, an example of which is shown by the green line in (b) (from~\cite{WikiOverfitting}).
        This will lead to a decreased inference capacity of the network because it is unable to generalize to unseen data.}\label{fig:loss_and_overfitting}
\end{figure}

Early stopping runs an inference pass on the validation step at the end of each epoch, and records the loss function value.
When the loss is seen to improve on the training set, but not the validation set, it is a sign of overfitting to the training data, because the network is no longer able to infer well on unseen data.

\section{Tooling}\label{sec:tooling}

Having defined the problem, solving it requires finding an appropriate architecture.
Before that is possible, one needs the pipeline up and running, because training cannot proceed without the data structures defined and built, and evaluation cannot proceed without calculating the metrics and comparing one run to another.

While CubeDB was built for the creation of datasets, CubeFlow (\url{https://gitlab.com/ehrhorn/cubeflow}) served as the code for training and metric calculation.
Early on in the process, it was discovered that IceCube's event viewer, Steamshovel, while extremely powerful was difficult (and in fact, impossible on the author's available machines) to compile and run, it was decided to build an event viewer by using modern frameworks which supported data in the SQLite format from CubeDB.\@
The software was named Powershovel, a play on IceCube's native Steamshovel event viewer\footnote{A steam shovel is a power shovel; steam shovels were replaced by diesel-powered shovels in the 1930s}, and the source code is found at \url{https://gitlab.com/ehrhorn/powershovel}.
It was soon realized that more visualization tasks were needed, in addition to event viewing, and Powershovel was thus amended to contain two more sections, Distributions and Runs.

Distributions shows dataset variable histograms, calculated when a new set is created.
This is important for visual inspection of the variable distributions, to ensure nothing went wrong in the creation, to test if training and validation samples are similar, and to confirm that distributions are not changed by the transformations described in~\vref{sec:data_transformation}; a screenshot of the distribution screen, showing the variable \verb|energy_log10|, can be seen in~\vref{fig:powershovel_1}.

\begin{figure}
    \centering
    \includegraphics[width=1.0\textwidth]{./images/design/powershovel_1.png}
    \caption{The true energy distribution of a DeepCore dataset shown in Powershovel.
    Note the ability to chose transformation (raw or scaled), table (features or truth) and variable.}\label{fig:powershovel_1}
\end{figure}

The histograms are calculated and saved in \verb|pickle| files during dataset creation such that the tool is actually responsive and useful, and can load a new dataset---and a new variable---in milliseconds.
Powershovel is made using Python, Streamlit and Plotly to ensure wide support, modern web technologies and ease of use.

\begin{figure}
    \centering
    \includegraphics[width=1.0\textwidth]{./images/design/powershovel_2.png}
    \caption{The Powershovel event viewer page.
    Each light pulse is indicated by the colored sphere, superimposed on the black DOMs.
    The size of the pulse is proportional to its charge, while the color is decided by the relative time offset of the pulse.
    The user can see relevant reconstructions, such as direction or vertex position, if any are available for the dataset, while the events are binned in energy.
    The user may also filter on reconstruction metrics; as shown in the screenshot, this view has been filtered on showing events where the absolute value of the zenith reconstruction error is \SIrange{18.97}{40.6}{\degree}.}\label{fig:powershovel_2}
\end{figure}

\Vref{fig:powershovel_2} shows the event viewer page where a user may inspect events from different SQLite datasets.
This uses a subset of events from the true dataset database, in order to load quickly, an issue because all events need to be loaded up front, such that the user may filter on e.g.\ true energy of the event.
The event view is static, with the time dimension instead indicated by color, but the 3D plot is interactive such that the user may pan and zoom to view the event from different angles.
This page was also adapted so that it is possible to upload I3 files for viewing; it was spurred on by the IceCube group at The Niels Bohr Institute having trouble with Steamshovel, and so Powershovel was amended and sent to them for evaluation.

\begin{figure}
    \centering
    \includegraphics[width=1.0\textwidth]{./images/design/powershovel_3.png}
    \caption{The Runs page showing two different runs, side by side.
    This shows the zenith resolution, as detailed in~\vref{sec:metrics}.
    Notice the \enquote{Rel.\ imp.} in the bottom of each figure, showing the relative change in performance between the two runs.}\label{fig:powershovel_3}
\end{figure}

The Runs page shows the result of a training run.
After training, CubeFlow runs reconstructions on a given dataset, saves them to a SQLite database, calculates errors and metrics (see~\vref{sec:metrics}), makes histograms, and stores them in \verb|pickle| files for fast retrieval.
The runs have a common naming convention, which includes certain key indicators of the algorithm (e.g.\ maximum event length, loss function, etc.) which is then used for filtering purposes such that the user can quickly see all algorithms using for example an MSE loss function.
Multiple filters can be applied simultaneously.

\begin{figure}
    \centering
    \includegraphics[width=1.0\textwidth]{./images/design/powershovel_4.png}
    \caption{The Runs screen, further down the page than shown in~\vref{fig:powershovel_3}.
    Two 2D histogram plots are shown, true zenith vs.\ the reconstructed zenith.
    The energy bin selection is shown in the bottom left corner, and below the 2D histograms 1D histograms showing the distribution of zenith predictions in the selected energy bin can be seen.
    Furthermore, to highlight the interactive nature of Plotly (the framework drawing all plots in Powershovel) extra info is visible the left 2D histogram, a consequence of the cursor hovering over it.
    This allows the user to see values for every point in the histogram.
    Zooming and panning is also possible.}\label{fig:powershovel_4}
\end{figure}

The page has a two-column layout, which accommodates side-by-side comparison of two runs.
This is crucial, as there is no single figure of merit\footnote{Resolution is used for evaluating different algorithms, but it is not a scalar. See~\vref{sec:metrics}} summarizing the performance in one number.
Most figures have a subfigure below it, showing the data basis for calculating that figure;
and because figures are binned in energy, a slider allows the user to choose what energy bin this subfigure should show data from.
This can be seen in~\vref{fig:powershovel_4}.

The creation of this tool and CubeDB represents a large part of the man-hours of this project, but the tool was used extensively during development of the machine learning algorithm, detailed in~\vref{sec:algorithm}.
Although not written by a computer science graduate, nor a professional programmer, the code is shared freely in the hope that any idea is taken up and implemented by IceCube.

\section{Metrics}\label{sec:metrics}

\begin{figure}
    \centering
    %% Creator: Matplotlib, PGF backend
%%
%% To include the figure in your LaTeX document, write
%%   \input{<filename>.pgf}
%%
%% Make sure the required packages are loaded in your preamble
%%   \usepackage{pgf}
%%
%% and, on pdftex
%%   \usepackage[utf8]{inputenc}\DeclareUnicodeCharacter{2212}{-}
%%
%% or, on luatex and xetex
%%   \usepackage{unicode-math}
%%
%% Figures using additional raster images can only be included by \input if
%% they are in the same directory as the main LaTeX file. For loading figures
%% from other directories you can use the `import` package
%%   \usepackage{import}
%%
%% and then include the figures with
%%   \import{<path to file>}{<filename>.pgf}
%%
%% Matplotlib used the following preamble
%%   \usepackage{siunitx} \usepackage{amsmath} \usepackage{bm}
%%   \usepackage{fontspec}
%%
\begingroup%
\makeatletter%
\begin{pgfpicture}%
\pgfpathrectangle{\pgfpointorigin}{\pgfqpoint{6.201200in}{5.000000in}}%
\pgfusepath{use as bounding box, clip}%
\begin{pgfscope}%
\pgfsetbuttcap%
\pgfsetmiterjoin%
\definecolor{currentfill}{rgb}{1.000000,1.000000,1.000000}%
\pgfsetfillcolor{currentfill}%
\pgfsetlinewidth{0.000000pt}%
\definecolor{currentstroke}{rgb}{1.000000,1.000000,1.000000}%
\pgfsetstrokecolor{currentstroke}%
\pgfsetdash{}{0pt}%
\pgfpathmoveto{\pgfqpoint{0.000000in}{0.000000in}}%
\pgfpathlineto{\pgfqpoint{6.201200in}{0.000000in}}%
\pgfpathlineto{\pgfqpoint{6.201200in}{5.000000in}}%
\pgfpathlineto{\pgfqpoint{0.000000in}{5.000000in}}%
\pgfpathclose%
\pgfusepath{fill}%
\end{pgfscope}%
\begin{pgfscope}%
\pgfsetbuttcap%
\pgfsetmiterjoin%
\definecolor{currentfill}{rgb}{1.000000,1.000000,1.000000}%
\pgfsetfillcolor{currentfill}%
\pgfsetlinewidth{0.000000pt}%
\definecolor{currentstroke}{rgb}{0.000000,0.000000,0.000000}%
\pgfsetstrokecolor{currentstroke}%
\pgfsetstrokeopacity{0.000000}%
\pgfsetdash{}{0pt}%
\pgfpathmoveto{\pgfqpoint{0.706736in}{2.828611in}}%
\pgfpathlineto{\pgfqpoint{3.249656in}{2.828611in}}%
\pgfpathlineto{\pgfqpoint{3.249656in}{4.381972in}}%
\pgfpathlineto{\pgfqpoint{0.706736in}{4.381972in}}%
\pgfpathclose%
\pgfusepath{fill}%
\end{pgfscope}%
\begin{pgfscope}%
\pgfpathrectangle{\pgfqpoint{0.706736in}{2.828611in}}{\pgfqpoint{2.542920in}{1.553361in}}%
\pgfusepath{clip}%
\pgfsetbuttcap%
\pgfsetmiterjoin%
\definecolor{currentfill}{rgb}{0.501961,0.501961,0.501961}%
\pgfsetfillcolor{currentfill}%
\pgfsetfillopacity{0.200000}%
\pgfsetlinewidth{0.000000pt}%
\definecolor{currentstroke}{rgb}{0.000000,0.000000,0.000000}%
\pgfsetstrokecolor{currentstroke}%
\pgfsetstrokeopacity{0.200000}%
\pgfsetdash{}{0pt}%
\pgfpathmoveto{\pgfqpoint{1.854867in}{2.828611in}}%
\pgfpathlineto{\pgfqpoint{1.854867in}{4.381972in}}%
\pgfpathlineto{\pgfqpoint{2.093103in}{4.381972in}}%
\pgfpathlineto{\pgfqpoint{2.093103in}{2.828611in}}%
\pgfpathclose%
\pgfusepath{fill}%
\end{pgfscope}%
\begin{pgfscope}%
\pgfsetbuttcap%
\pgfsetroundjoin%
\definecolor{currentfill}{rgb}{0.000000,0.000000,0.000000}%
\pgfsetfillcolor{currentfill}%
\pgfsetlinewidth{0.803000pt}%
\definecolor{currentstroke}{rgb}{0.000000,0.000000,0.000000}%
\pgfsetstrokecolor{currentstroke}%
\pgfsetdash{}{0pt}%
\pgfsys@defobject{currentmarker}{\pgfqpoint{0.000000in}{-0.048611in}}{\pgfqpoint{0.000000in}{0.000000in}}{%
\pgfpathmoveto{\pgfqpoint{0.000000in}{0.000000in}}%
\pgfpathlineto{\pgfqpoint{0.000000in}{-0.048611in}}%
\pgfusepath{stroke,fill}%
}%
\begin{pgfscope}%
\pgfsys@transformshift{1.095837in}{2.828611in}%
\pgfsys@useobject{currentmarker}{}%
\end{pgfscope}%
\end{pgfscope}%
\begin{pgfscope}%
\definecolor{textcolor}{rgb}{0.000000,0.000000,0.000000}%
\pgfsetstrokecolor{textcolor}%
\pgfsetfillcolor{textcolor}%
\pgftext[x=1.095837in,y=2.731389in,,top]{\color{textcolor}\rmfamily\fontsize{8.000000}{9.600000}\selectfont \(\displaystyle {−100}\)}%
\end{pgfscope}%
\begin{pgfscope}%
\pgfsetbuttcap%
\pgfsetroundjoin%
\definecolor{currentfill}{rgb}{0.000000,0.000000,0.000000}%
\pgfsetfillcolor{currentfill}%
\pgfsetlinewidth{0.803000pt}%
\definecolor{currentstroke}{rgb}{0.000000,0.000000,0.000000}%
\pgfsetstrokecolor{currentstroke}%
\pgfsetdash{}{0pt}%
\pgfsys@defobject{currentmarker}{\pgfqpoint{0.000000in}{-0.048611in}}{\pgfqpoint{0.000000in}{0.000000in}}{%
\pgfpathmoveto{\pgfqpoint{0.000000in}{0.000000in}}%
\pgfpathlineto{\pgfqpoint{0.000000in}{-0.048611in}}%
\pgfusepath{stroke,fill}%
}%
\begin{pgfscope}%
\pgfsys@transformshift{1.534546in}{2.828611in}%
\pgfsys@useobject{currentmarker}{}%
\end{pgfscope}%
\end{pgfscope}%
\begin{pgfscope}%
\definecolor{textcolor}{rgb}{0.000000,0.000000,0.000000}%
\pgfsetstrokecolor{textcolor}%
\pgfsetfillcolor{textcolor}%
\pgftext[x=1.534546in,y=2.731389in,,top]{\color{textcolor}\rmfamily\fontsize{8.000000}{9.600000}\selectfont \(\displaystyle {−50}\)}%
\end{pgfscope}%
\begin{pgfscope}%
\pgfsetbuttcap%
\pgfsetroundjoin%
\definecolor{currentfill}{rgb}{0.000000,0.000000,0.000000}%
\pgfsetfillcolor{currentfill}%
\pgfsetlinewidth{0.803000pt}%
\definecolor{currentstroke}{rgb}{0.000000,0.000000,0.000000}%
\pgfsetstrokecolor{currentstroke}%
\pgfsetdash{}{0pt}%
\pgfsys@defobject{currentmarker}{\pgfqpoint{0.000000in}{-0.048611in}}{\pgfqpoint{0.000000in}{0.000000in}}{%
\pgfpathmoveto{\pgfqpoint{0.000000in}{0.000000in}}%
\pgfpathlineto{\pgfqpoint{0.000000in}{-0.048611in}}%
\pgfusepath{stroke,fill}%
}%
\begin{pgfscope}%
\pgfsys@transformshift{1.973255in}{2.828611in}%
\pgfsys@useobject{currentmarker}{}%
\end{pgfscope}%
\end{pgfscope}%
\begin{pgfscope}%
\definecolor{textcolor}{rgb}{0.000000,0.000000,0.000000}%
\pgfsetstrokecolor{textcolor}%
\pgfsetfillcolor{textcolor}%
\pgftext[x=1.973255in,y=2.731389in,,top]{\color{textcolor}\rmfamily\fontsize{8.000000}{9.600000}\selectfont \(\displaystyle {0}\)}%
\end{pgfscope}%
\begin{pgfscope}%
\pgfsetbuttcap%
\pgfsetroundjoin%
\definecolor{currentfill}{rgb}{0.000000,0.000000,0.000000}%
\pgfsetfillcolor{currentfill}%
\pgfsetlinewidth{0.803000pt}%
\definecolor{currentstroke}{rgb}{0.000000,0.000000,0.000000}%
\pgfsetstrokecolor{currentstroke}%
\pgfsetdash{}{0pt}%
\pgfsys@defobject{currentmarker}{\pgfqpoint{0.000000in}{-0.048611in}}{\pgfqpoint{0.000000in}{0.000000in}}{%
\pgfpathmoveto{\pgfqpoint{0.000000in}{0.000000in}}%
\pgfpathlineto{\pgfqpoint{0.000000in}{-0.048611in}}%
\pgfusepath{stroke,fill}%
}%
\begin{pgfscope}%
\pgfsys@transformshift{2.411965in}{2.828611in}%
\pgfsys@useobject{currentmarker}{}%
\end{pgfscope}%
\end{pgfscope}%
\begin{pgfscope}%
\definecolor{textcolor}{rgb}{0.000000,0.000000,0.000000}%
\pgfsetstrokecolor{textcolor}%
\pgfsetfillcolor{textcolor}%
\pgftext[x=2.411965in,y=2.731389in,,top]{\color{textcolor}\rmfamily\fontsize{8.000000}{9.600000}\selectfont \(\displaystyle {50}\)}%
\end{pgfscope}%
\begin{pgfscope}%
\pgfsetbuttcap%
\pgfsetroundjoin%
\definecolor{currentfill}{rgb}{0.000000,0.000000,0.000000}%
\pgfsetfillcolor{currentfill}%
\pgfsetlinewidth{0.803000pt}%
\definecolor{currentstroke}{rgb}{0.000000,0.000000,0.000000}%
\pgfsetstrokecolor{currentstroke}%
\pgfsetdash{}{0pt}%
\pgfsys@defobject{currentmarker}{\pgfqpoint{0.000000in}{-0.048611in}}{\pgfqpoint{0.000000in}{0.000000in}}{%
\pgfpathmoveto{\pgfqpoint{0.000000in}{0.000000in}}%
\pgfpathlineto{\pgfqpoint{0.000000in}{-0.048611in}}%
\pgfusepath{stroke,fill}%
}%
\begin{pgfscope}%
\pgfsys@transformshift{2.850674in}{2.828611in}%
\pgfsys@useobject{currentmarker}{}%
\end{pgfscope}%
\end{pgfscope}%
\begin{pgfscope}%
\definecolor{textcolor}{rgb}{0.000000,0.000000,0.000000}%
\pgfsetstrokecolor{textcolor}%
\pgfsetfillcolor{textcolor}%
\pgftext[x=2.850674in,y=2.731389in,,top]{\color{textcolor}\rmfamily\fontsize{8.000000}{9.600000}\selectfont \(\displaystyle {100}\)}%
\end{pgfscope}%
\begin{pgfscope}%
\definecolor{textcolor}{rgb}{0.000000,0.000000,0.000000}%
\pgfsetstrokecolor{textcolor}%
\pgfsetfillcolor{textcolor}%
\pgftext[x=1.978196in,y=2.577167in,,top]{\color{textcolor}\rmfamily\fontsize{10.950000}{13.140000}\selectfont \(\displaystyle  \theta_{\textup{true}} - \theta_{\textup{reco}} \, [\textup{rad}] \)}%
\end{pgfscope}%
\begin{pgfscope}%
\pgfsetbuttcap%
\pgfsetroundjoin%
\definecolor{currentfill}{rgb}{0.000000,0.000000,0.000000}%
\pgfsetfillcolor{currentfill}%
\pgfsetlinewidth{0.803000pt}%
\definecolor{currentstroke}{rgb}{0.000000,0.000000,0.000000}%
\pgfsetstrokecolor{currentstroke}%
\pgfsetdash{}{0pt}%
\pgfsys@defobject{currentmarker}{\pgfqpoint{-0.048611in}{0.000000in}}{\pgfqpoint{-0.000000in}{0.000000in}}{%
\pgfpathmoveto{\pgfqpoint{-0.000000in}{0.000000in}}%
\pgfpathlineto{\pgfqpoint{-0.048611in}{0.000000in}}%
\pgfusepath{stroke,fill}%
}%
\begin{pgfscope}%
\pgfsys@transformshift{0.706736in}{2.828611in}%
\pgfsys@useobject{currentmarker}{}%
\end{pgfscope}%
\end{pgfscope}%
\begin{pgfscope}%
\definecolor{textcolor}{rgb}{0.000000,0.000000,0.000000}%
\pgfsetstrokecolor{textcolor}%
\pgfsetfillcolor{textcolor}%
\pgftext[x=0.399634in, y=2.790055in, left, base]{\color{textcolor}\rmfamily\fontsize{8.000000}{9.600000}\selectfont \(\displaystyle {0.00}\)}%
\end{pgfscope}%
\begin{pgfscope}%
\pgfsetbuttcap%
\pgfsetroundjoin%
\definecolor{currentfill}{rgb}{0.000000,0.000000,0.000000}%
\pgfsetfillcolor{currentfill}%
\pgfsetlinewidth{0.803000pt}%
\definecolor{currentstroke}{rgb}{0.000000,0.000000,0.000000}%
\pgfsetstrokecolor{currentstroke}%
\pgfsetdash{}{0pt}%
\pgfsys@defobject{currentmarker}{\pgfqpoint{-0.048611in}{0.000000in}}{\pgfqpoint{-0.000000in}{0.000000in}}{%
\pgfpathmoveto{\pgfqpoint{-0.000000in}{0.000000in}}%
\pgfpathlineto{\pgfqpoint{-0.048611in}{0.000000in}}%
\pgfusepath{stroke,fill}%
}%
\begin{pgfscope}%
\pgfsys@transformshift{0.706736in}{3.232596in}%
\pgfsys@useobject{currentmarker}{}%
\end{pgfscope}%
\end{pgfscope}%
\begin{pgfscope}%
\definecolor{textcolor}{rgb}{0.000000,0.000000,0.000000}%
\pgfsetstrokecolor{textcolor}%
\pgfsetfillcolor{textcolor}%
\pgftext[x=0.399634in, y=3.194041in, left, base]{\color{textcolor}\rmfamily\fontsize{8.000000}{9.600000}\selectfont \(\displaystyle {0.01}\)}%
\end{pgfscope}%
\begin{pgfscope}%
\pgfsetbuttcap%
\pgfsetroundjoin%
\definecolor{currentfill}{rgb}{0.000000,0.000000,0.000000}%
\pgfsetfillcolor{currentfill}%
\pgfsetlinewidth{0.803000pt}%
\definecolor{currentstroke}{rgb}{0.000000,0.000000,0.000000}%
\pgfsetstrokecolor{currentstroke}%
\pgfsetdash{}{0pt}%
\pgfsys@defobject{currentmarker}{\pgfqpoint{-0.048611in}{0.000000in}}{\pgfqpoint{-0.000000in}{0.000000in}}{%
\pgfpathmoveto{\pgfqpoint{-0.000000in}{0.000000in}}%
\pgfpathlineto{\pgfqpoint{-0.048611in}{0.000000in}}%
\pgfusepath{stroke,fill}%
}%
\begin{pgfscope}%
\pgfsys@transformshift{0.706736in}{3.636581in}%
\pgfsys@useobject{currentmarker}{}%
\end{pgfscope}%
\end{pgfscope}%
\begin{pgfscope}%
\definecolor{textcolor}{rgb}{0.000000,0.000000,0.000000}%
\pgfsetstrokecolor{textcolor}%
\pgfsetfillcolor{textcolor}%
\pgftext[x=0.399634in, y=3.598026in, left, base]{\color{textcolor}\rmfamily\fontsize{8.000000}{9.600000}\selectfont \(\displaystyle {0.02}\)}%
\end{pgfscope}%
\begin{pgfscope}%
\pgfsetbuttcap%
\pgfsetroundjoin%
\definecolor{currentfill}{rgb}{0.000000,0.000000,0.000000}%
\pgfsetfillcolor{currentfill}%
\pgfsetlinewidth{0.803000pt}%
\definecolor{currentstroke}{rgb}{0.000000,0.000000,0.000000}%
\pgfsetstrokecolor{currentstroke}%
\pgfsetdash{}{0pt}%
\pgfsys@defobject{currentmarker}{\pgfqpoint{-0.048611in}{0.000000in}}{\pgfqpoint{-0.000000in}{0.000000in}}{%
\pgfpathmoveto{\pgfqpoint{-0.000000in}{0.000000in}}%
\pgfpathlineto{\pgfqpoint{-0.048611in}{0.000000in}}%
\pgfusepath{stroke,fill}%
}%
\begin{pgfscope}%
\pgfsys@transformshift{0.706736in}{4.040567in}%
\pgfsys@useobject{currentmarker}{}%
\end{pgfscope}%
\end{pgfscope}%
\begin{pgfscope}%
\definecolor{textcolor}{rgb}{0.000000,0.000000,0.000000}%
\pgfsetstrokecolor{textcolor}%
\pgfsetfillcolor{textcolor}%
\pgftext[x=0.399634in, y=4.002011in, left, base]{\color{textcolor}\rmfamily\fontsize{8.000000}{9.600000}\selectfont \(\displaystyle {0.03}\)}%
\end{pgfscope}%
\begin{pgfscope}%
\definecolor{textcolor}{rgb}{0.000000,0.000000,0.000000}%
\pgfsetstrokecolor{textcolor}%
\pgfsetfillcolor{textcolor}%
\pgftext[x=0.344079in,y=3.605292in,,bottom,rotate=90.000000]{\color{textcolor}\rmfamily\fontsize{10.950000}{13.140000}\selectfont Density}%
\end{pgfscope}%
\begin{pgfscope}%
\pgfpathrectangle{\pgfqpoint{0.706736in}{2.828611in}}{\pgfqpoint{2.542920in}{1.553361in}}%
\pgfusepath{clip}%
\pgfsetbuttcap%
\pgfsetmiterjoin%
\pgfsetlinewidth{1.003750pt}%
\definecolor{currentstroke}{rgb}{0.313725,0.317647,0.309804}%
\pgfsetstrokecolor{currentstroke}%
\pgfsetdash{}{0pt}%
\pgfpathmoveto{\pgfqpoint{0.822323in}{2.828611in}}%
\pgfpathlineto{\pgfqpoint{0.822323in}{2.828698in}}%
\pgfpathlineto{\pgfqpoint{1.085491in}{2.829660in}}%
\pgfpathlineto{\pgfqpoint{1.085491in}{2.829922in}}%
\pgfpathlineto{\pgfqpoint{1.152573in}{2.830795in}}%
\pgfpathlineto{\pgfqpoint{1.152573in}{2.831058in}}%
\pgfpathlineto{\pgfqpoint{1.199014in}{2.831757in}}%
\pgfpathlineto{\pgfqpoint{1.199014in}{2.832892in}}%
\pgfpathlineto{\pgfqpoint{1.250615in}{2.833504in}}%
\pgfpathlineto{\pgfqpoint{1.250615in}{2.835426in}}%
\pgfpathlineto{\pgfqpoint{1.266096in}{2.834465in}}%
\pgfpathlineto{\pgfqpoint{1.266096in}{2.834028in}}%
\pgfpathlineto{\pgfqpoint{1.271256in}{2.834028in}}%
\pgfpathlineto{\pgfqpoint{1.271256in}{2.836213in}}%
\pgfpathlineto{\pgfqpoint{1.276416in}{2.836213in}}%
\pgfpathlineto{\pgfqpoint{1.276416in}{2.837523in}}%
\pgfpathlineto{\pgfqpoint{1.281576in}{2.837523in}}%
\pgfpathlineto{\pgfqpoint{1.281576in}{2.836038in}}%
\pgfpathlineto{\pgfqpoint{1.291896in}{2.836125in}}%
\pgfpathlineto{\pgfqpoint{1.291896in}{2.837261in}}%
\pgfpathlineto{\pgfqpoint{1.307377in}{2.837349in}}%
\pgfpathlineto{\pgfqpoint{1.307377in}{2.838485in}}%
\pgfpathlineto{\pgfqpoint{1.317697in}{2.837523in}}%
\pgfpathlineto{\pgfqpoint{1.317697in}{2.839621in}}%
\pgfpathlineto{\pgfqpoint{1.338338in}{2.839795in}}%
\pgfpathlineto{\pgfqpoint{1.338338in}{2.843290in}}%
\pgfpathlineto{\pgfqpoint{1.343498in}{2.843290in}}%
\pgfpathlineto{\pgfqpoint{1.343498in}{2.841106in}}%
\pgfpathlineto{\pgfqpoint{1.348658in}{2.841106in}}%
\pgfpathlineto{\pgfqpoint{1.348658in}{2.843727in}}%
\pgfpathlineto{\pgfqpoint{1.369299in}{2.844339in}}%
\pgfpathlineto{\pgfqpoint{1.369299in}{2.842591in}}%
\pgfpathlineto{\pgfqpoint{1.374459in}{2.842591in}}%
\pgfpathlineto{\pgfqpoint{1.374459in}{2.846261in}}%
\pgfpathlineto{\pgfqpoint{1.379619in}{2.846261in}}%
\pgfpathlineto{\pgfqpoint{1.379619in}{2.849669in}}%
\pgfpathlineto{\pgfqpoint{1.389939in}{2.849232in}}%
\pgfpathlineto{\pgfqpoint{1.389939in}{2.847747in}}%
\pgfpathlineto{\pgfqpoint{1.400259in}{2.847572in}}%
\pgfpathlineto{\pgfqpoint{1.400259in}{2.849844in}}%
\pgfpathlineto{\pgfqpoint{1.405420in}{2.849844in}}%
\pgfpathlineto{\pgfqpoint{1.405420in}{2.850980in}}%
\pgfpathlineto{\pgfqpoint{1.415740in}{2.850018in}}%
\pgfpathlineto{\pgfqpoint{1.415740in}{2.852290in}}%
\pgfpathlineto{\pgfqpoint{1.420900in}{2.852290in}}%
\pgfpathlineto{\pgfqpoint{1.420900in}{2.853776in}}%
\pgfpathlineto{\pgfqpoint{1.426060in}{2.853776in}}%
\pgfpathlineto{\pgfqpoint{1.426060in}{2.854999in}}%
\pgfpathlineto{\pgfqpoint{1.436381in}{2.854387in}}%
\pgfpathlineto{\pgfqpoint{1.436381in}{2.858407in}}%
\pgfpathlineto{\pgfqpoint{1.446701in}{2.858669in}}%
\pgfpathlineto{\pgfqpoint{1.446701in}{2.861115in}}%
\pgfpathlineto{\pgfqpoint{1.451861in}{2.861115in}}%
\pgfpathlineto{\pgfqpoint{1.451861in}{2.865572in}}%
\pgfpathlineto{\pgfqpoint{1.462181in}{2.864698in}}%
\pgfpathlineto{\pgfqpoint{1.462181in}{2.862688in}}%
\pgfpathlineto{\pgfqpoint{1.472502in}{2.863038in}}%
\pgfpathlineto{\pgfqpoint{1.472502in}{2.868280in}}%
\pgfpathlineto{\pgfqpoint{1.487982in}{2.867581in}}%
\pgfpathlineto{\pgfqpoint{1.487982in}{2.871775in}}%
\pgfpathlineto{\pgfqpoint{1.493142in}{2.871775in}}%
\pgfpathlineto{\pgfqpoint{1.493142in}{2.877280in}}%
\pgfpathlineto{\pgfqpoint{1.503462in}{2.877368in}}%
\pgfpathlineto{\pgfqpoint{1.503462in}{2.880513in}}%
\pgfpathlineto{\pgfqpoint{1.513783in}{2.880863in}}%
\pgfpathlineto{\pgfqpoint{1.513783in}{2.888552in}}%
\pgfpathlineto{\pgfqpoint{1.518943in}{2.888552in}}%
\pgfpathlineto{\pgfqpoint{1.518943in}{2.891697in}}%
\pgfpathlineto{\pgfqpoint{1.524103in}{2.891697in}}%
\pgfpathlineto{\pgfqpoint{1.524103in}{2.888202in}}%
\pgfpathlineto{\pgfqpoint{1.529263in}{2.888202in}}%
\pgfpathlineto{\pgfqpoint{1.529263in}{2.890649in}}%
\pgfpathlineto{\pgfqpoint{1.534423in}{2.890649in}}%
\pgfpathlineto{\pgfqpoint{1.534423in}{2.892047in}}%
\pgfpathlineto{\pgfqpoint{1.539583in}{2.892047in}}%
\pgfpathlineto{\pgfqpoint{1.539583in}{2.897814in}}%
\pgfpathlineto{\pgfqpoint{1.549904in}{2.897901in}}%
\pgfpathlineto{\pgfqpoint{1.549904in}{2.899649in}}%
\pgfpathlineto{\pgfqpoint{1.555064in}{2.899649in}}%
\pgfpathlineto{\pgfqpoint{1.555064in}{2.905416in}}%
\pgfpathlineto{\pgfqpoint{1.560224in}{2.905416in}}%
\pgfpathlineto{\pgfqpoint{1.560224in}{2.908474in}}%
\pgfpathlineto{\pgfqpoint{1.565384in}{2.908474in}}%
\pgfpathlineto{\pgfqpoint{1.565384in}{2.913367in}}%
\pgfpathlineto{\pgfqpoint{1.575704in}{2.914328in}}%
\pgfpathlineto{\pgfqpoint{1.575704in}{2.918348in}}%
\pgfpathlineto{\pgfqpoint{1.580865in}{2.918348in}}%
\pgfpathlineto{\pgfqpoint{1.580865in}{2.922979in}}%
\pgfpathlineto{\pgfqpoint{1.586025in}{2.922979in}}%
\pgfpathlineto{\pgfqpoint{1.586025in}{2.924377in}}%
\pgfpathlineto{\pgfqpoint{1.591185in}{2.924377in}}%
\pgfpathlineto{\pgfqpoint{1.591185in}{2.925862in}}%
\pgfpathlineto{\pgfqpoint{1.596345in}{2.925862in}}%
\pgfpathlineto{\pgfqpoint{1.596345in}{2.936435in}}%
\pgfpathlineto{\pgfqpoint{1.601505in}{2.936435in}}%
\pgfpathlineto{\pgfqpoint{1.601505in}{2.945172in}}%
\pgfpathlineto{\pgfqpoint{1.606665in}{2.945172in}}%
\pgfpathlineto{\pgfqpoint{1.606665in}{2.941328in}}%
\pgfpathlineto{\pgfqpoint{1.611825in}{2.941328in}}%
\pgfpathlineto{\pgfqpoint{1.611825in}{2.946134in}}%
\pgfpathlineto{\pgfqpoint{1.616986in}{2.946134in}}%
\pgfpathlineto{\pgfqpoint{1.616986in}{2.954522in}}%
\pgfpathlineto{\pgfqpoint{1.622146in}{2.954522in}}%
\pgfpathlineto{\pgfqpoint{1.622146in}{2.961949in}}%
\pgfpathlineto{\pgfqpoint{1.627306in}{2.961949in}}%
\pgfpathlineto{\pgfqpoint{1.627306in}{2.955832in}}%
\pgfpathlineto{\pgfqpoint{1.632466in}{2.955832in}}%
\pgfpathlineto{\pgfqpoint{1.632466in}{2.972696in}}%
\pgfpathlineto{\pgfqpoint{1.637626in}{2.972696in}}%
\pgfpathlineto{\pgfqpoint{1.637626in}{2.970949in}}%
\pgfpathlineto{\pgfqpoint{1.642786in}{2.970949in}}%
\pgfpathlineto{\pgfqpoint{1.642786in}{2.978813in}}%
\pgfpathlineto{\pgfqpoint{1.647946in}{2.978813in}}%
\pgfpathlineto{\pgfqpoint{1.647946in}{2.985628in}}%
\pgfpathlineto{\pgfqpoint{1.658267in}{2.986414in}}%
\pgfpathlineto{\pgfqpoint{1.658267in}{2.994541in}}%
\pgfpathlineto{\pgfqpoint{1.663427in}{2.994541in}}%
\pgfpathlineto{\pgfqpoint{1.663427in}{3.002230in}}%
\pgfpathlineto{\pgfqpoint{1.668587in}{3.002230in}}%
\pgfpathlineto{\pgfqpoint{1.668587in}{3.012802in}}%
\pgfpathlineto{\pgfqpoint{1.678907in}{3.012366in}}%
\pgfpathlineto{\pgfqpoint{1.678907in}{3.025035in}}%
\pgfpathlineto{\pgfqpoint{1.684067in}{3.025035in}}%
\pgfpathlineto{\pgfqpoint{1.684067in}{3.034035in}}%
\pgfpathlineto{\pgfqpoint{1.689228in}{3.034035in}}%
\pgfpathlineto{\pgfqpoint{1.689228in}{3.043734in}}%
\pgfpathlineto{\pgfqpoint{1.699548in}{3.043210in}}%
\pgfpathlineto{\pgfqpoint{1.699548in}{3.058501in}}%
\pgfpathlineto{\pgfqpoint{1.704708in}{3.058501in}}%
\pgfpathlineto{\pgfqpoint{1.704708in}{3.061996in}}%
\pgfpathlineto{\pgfqpoint{1.709868in}{3.061996in}}%
\pgfpathlineto{\pgfqpoint{1.709868in}{3.071170in}}%
\pgfpathlineto{\pgfqpoint{1.715028in}{3.071170in}}%
\pgfpathlineto{\pgfqpoint{1.715028in}{3.081044in}}%
\pgfpathlineto{\pgfqpoint{1.720189in}{3.081044in}}%
\pgfpathlineto{\pgfqpoint{1.720189in}{3.093539in}}%
\pgfpathlineto{\pgfqpoint{1.725349in}{3.093539in}}%
\pgfpathlineto{\pgfqpoint{1.725349in}{3.095461in}}%
\pgfpathlineto{\pgfqpoint{1.730509in}{3.095461in}}%
\pgfpathlineto{\pgfqpoint{1.730509in}{3.110403in}}%
\pgfpathlineto{\pgfqpoint{1.735669in}{3.110403in}}%
\pgfpathlineto{\pgfqpoint{1.735669in}{3.115296in}}%
\pgfpathlineto{\pgfqpoint{1.740829in}{3.115296in}}%
\pgfpathlineto{\pgfqpoint{1.740829in}{3.126830in}}%
\pgfpathlineto{\pgfqpoint{1.745989in}{3.126830in}}%
\pgfpathlineto{\pgfqpoint{1.745989in}{3.150422in}}%
\pgfpathlineto{\pgfqpoint{1.751149in}{3.150422in}}%
\pgfpathlineto{\pgfqpoint{1.751149in}{3.152169in}}%
\pgfpathlineto{\pgfqpoint{1.756310in}{3.152169in}}%
\pgfpathlineto{\pgfqpoint{1.756310in}{3.164402in}}%
\pgfpathlineto{\pgfqpoint{1.761470in}{3.164402in}}%
\pgfpathlineto{\pgfqpoint{1.761470in}{3.178120in}}%
\pgfpathlineto{\pgfqpoint{1.766630in}{3.178120in}}%
\pgfpathlineto{\pgfqpoint{1.766630in}{3.193062in}}%
\pgfpathlineto{\pgfqpoint{1.776950in}{3.192975in}}%
\pgfpathlineto{\pgfqpoint{1.776950in}{3.211761in}}%
\pgfpathlineto{\pgfqpoint{1.782110in}{3.211761in}}%
\pgfpathlineto{\pgfqpoint{1.782110in}{3.216392in}}%
\pgfpathlineto{\pgfqpoint{1.787270in}{3.216392in}}%
\pgfpathlineto{\pgfqpoint{1.787270in}{3.227489in}}%
\pgfpathlineto{\pgfqpoint{1.792431in}{3.227489in}}%
\pgfpathlineto{\pgfqpoint{1.792431in}{3.238149in}}%
\pgfpathlineto{\pgfqpoint{1.797591in}{3.238149in}}%
\pgfpathlineto{\pgfqpoint{1.797591in}{3.255187in}}%
\pgfpathlineto{\pgfqpoint{1.802751in}{3.255187in}}%
\pgfpathlineto{\pgfqpoint{1.802751in}{3.272837in}}%
\pgfpathlineto{\pgfqpoint{1.807911in}{3.272837in}}%
\pgfpathlineto{\pgfqpoint{1.807911in}{3.271614in}}%
\pgfpathlineto{\pgfqpoint{1.813071in}{3.271614in}}%
\pgfpathlineto{\pgfqpoint{1.813071in}{3.306216in}}%
\pgfpathlineto{\pgfqpoint{1.818231in}{3.306216in}}%
\pgfpathlineto{\pgfqpoint{1.818231in}{3.311196in}}%
\pgfpathlineto{\pgfqpoint{1.823391in}{3.311196in}}%
\pgfpathlineto{\pgfqpoint{1.823391in}{3.327710in}}%
\pgfpathlineto{\pgfqpoint{1.828552in}{3.327710in}}%
\pgfpathlineto{\pgfqpoint{1.828552in}{3.356720in}}%
\pgfpathlineto{\pgfqpoint{1.833712in}{3.356720in}}%
\pgfpathlineto{\pgfqpoint{1.833712in}{3.369389in}}%
\pgfpathlineto{\pgfqpoint{1.838872in}{3.369389in}}%
\pgfpathlineto{\pgfqpoint{1.838872in}{3.377690in}}%
\pgfpathlineto{\pgfqpoint{1.844032in}{3.377690in}}%
\pgfpathlineto{\pgfqpoint{1.844032in}{3.388350in}}%
\pgfpathlineto{\pgfqpoint{1.849192in}{3.388350in}}%
\pgfpathlineto{\pgfqpoint{1.849192in}{3.431689in}}%
\pgfpathlineto{\pgfqpoint{1.854352in}{3.431689in}}%
\pgfpathlineto{\pgfqpoint{1.854352in}{3.434922in}}%
\pgfpathlineto{\pgfqpoint{1.864673in}{3.434136in}}%
\pgfpathlineto{\pgfqpoint{1.864673in}{3.448379in}}%
\pgfpathlineto{\pgfqpoint{1.869833in}{3.448379in}}%
\pgfpathlineto{\pgfqpoint{1.869833in}{3.475029in}}%
\pgfpathlineto{\pgfqpoint{1.874993in}{3.475029in}}%
\pgfpathlineto{\pgfqpoint{1.874993in}{3.491543in}}%
\pgfpathlineto{\pgfqpoint{1.880153in}{3.491543in}}%
\pgfpathlineto{\pgfqpoint{1.880153in}{3.504300in}}%
\pgfpathlineto{\pgfqpoint{1.885313in}{3.504300in}}%
\pgfpathlineto{\pgfqpoint{1.885313in}{3.519416in}}%
\pgfpathlineto{\pgfqpoint{1.890473in}{3.519416in}}%
\pgfpathlineto{\pgfqpoint{1.890473in}{3.530950in}}%
\pgfpathlineto{\pgfqpoint{1.895633in}{3.530950in}}%
\pgfpathlineto{\pgfqpoint{1.895633in}{3.546678in}}%
\pgfpathlineto{\pgfqpoint{1.900794in}{3.546678in}}%
\pgfpathlineto{\pgfqpoint{1.900794in}{3.563979in}}%
\pgfpathlineto{\pgfqpoint{1.905954in}{3.563979in}}%
\pgfpathlineto{\pgfqpoint{1.905954in}{3.575862in}}%
\pgfpathlineto{\pgfqpoint{1.911114in}{3.575862in}}%
\pgfpathlineto{\pgfqpoint{1.911114in}{3.601114in}}%
\pgfpathlineto{\pgfqpoint{1.916274in}{3.601114in}}%
\pgfpathlineto{\pgfqpoint{1.916274in}{3.605221in}}%
\pgfpathlineto{\pgfqpoint{1.921434in}{3.605221in}}%
\pgfpathlineto{\pgfqpoint{1.921434in}{3.622784in}}%
\pgfpathlineto{\pgfqpoint{1.926594in}{3.622784in}}%
\pgfpathlineto{\pgfqpoint{1.926594in}{3.615357in}}%
\pgfpathlineto{\pgfqpoint{1.931754in}{3.615357in}}%
\pgfpathlineto{\pgfqpoint{1.931754in}{3.638162in}}%
\pgfpathlineto{\pgfqpoint{1.936915in}{3.638162in}}%
\pgfpathlineto{\pgfqpoint{1.936915in}{3.645764in}}%
\pgfpathlineto{\pgfqpoint{1.942075in}{3.645764in}}%
\pgfpathlineto{\pgfqpoint{1.942075in}{3.653977in}}%
\pgfpathlineto{\pgfqpoint{1.947235in}{3.653977in}}%
\pgfpathlineto{\pgfqpoint{1.947235in}{3.674336in}}%
\pgfpathlineto{\pgfqpoint{1.952395in}{3.674336in}}%
\pgfpathlineto{\pgfqpoint{1.952395in}{3.665948in}}%
\pgfpathlineto{\pgfqpoint{1.957555in}{3.665948in}}%
\pgfpathlineto{\pgfqpoint{1.957555in}{3.686045in}}%
\pgfpathlineto{\pgfqpoint{1.962715in}{3.686045in}}%
\pgfpathlineto{\pgfqpoint{1.962715in}{3.674511in}}%
\pgfpathlineto{\pgfqpoint{1.967875in}{3.674511in}}%
\pgfpathlineto{\pgfqpoint{1.967875in}{3.673288in}}%
\pgfpathlineto{\pgfqpoint{1.973036in}{3.673288in}}%
\pgfpathlineto{\pgfqpoint{1.973036in}{3.674598in}}%
\pgfpathlineto{\pgfqpoint{1.978196in}{3.674598in}}%
\pgfpathlineto{\pgfqpoint{1.978196in}{3.687792in}}%
\pgfpathlineto{\pgfqpoint{1.983356in}{3.687792in}}%
\pgfpathlineto{\pgfqpoint{1.983356in}{3.667958in}}%
\pgfpathlineto{\pgfqpoint{1.988516in}{3.667958in}}%
\pgfpathlineto{\pgfqpoint{1.988516in}{3.671802in}}%
\pgfpathlineto{\pgfqpoint{1.993676in}{3.671802in}}%
\pgfpathlineto{\pgfqpoint{1.993676in}{3.655113in}}%
\pgfpathlineto{\pgfqpoint{1.998836in}{3.655113in}}%
\pgfpathlineto{\pgfqpoint{1.998836in}{3.652317in}}%
\pgfpathlineto{\pgfqpoint{2.003996in}{3.652317in}}%
\pgfpathlineto{\pgfqpoint{2.003996in}{3.642269in}}%
\pgfpathlineto{\pgfqpoint{2.009157in}{3.642269in}}%
\pgfpathlineto{\pgfqpoint{2.009157in}{3.621910in}}%
\pgfpathlineto{\pgfqpoint{2.014317in}{3.621910in}}%
\pgfpathlineto{\pgfqpoint{2.014317in}{3.629861in}}%
\pgfpathlineto{\pgfqpoint{2.019477in}{3.629861in}}%
\pgfpathlineto{\pgfqpoint{2.019477in}{3.621560in}}%
\pgfpathlineto{\pgfqpoint{2.024637in}{3.621560in}}%
\pgfpathlineto{\pgfqpoint{2.024637in}{3.598318in}}%
\pgfpathlineto{\pgfqpoint{2.029797in}{3.598318in}}%
\pgfpathlineto{\pgfqpoint{2.029797in}{3.594473in}}%
\pgfpathlineto{\pgfqpoint{2.034957in}{3.594473in}}%
\pgfpathlineto{\pgfqpoint{2.034957in}{3.564154in}}%
\pgfpathlineto{\pgfqpoint{2.040118in}{3.564154in}}%
\pgfpathlineto{\pgfqpoint{2.040118in}{3.539076in}}%
\pgfpathlineto{\pgfqpoint{2.045278in}{3.539076in}}%
\pgfpathlineto{\pgfqpoint{2.045278in}{3.544494in}}%
\pgfpathlineto{\pgfqpoint{2.050438in}{3.544494in}}%
\pgfpathlineto{\pgfqpoint{2.050438in}{3.536368in}}%
\pgfpathlineto{\pgfqpoint{2.055598in}{3.536368in}}%
\pgfpathlineto{\pgfqpoint{2.055598in}{3.503951in}}%
\pgfpathlineto{\pgfqpoint{2.060758in}{3.503951in}}%
\pgfpathlineto{\pgfqpoint{2.060758in}{3.491456in}}%
\pgfpathlineto{\pgfqpoint{2.065918in}{3.491456in}}%
\pgfpathlineto{\pgfqpoint{2.065918in}{3.464281in}}%
\pgfpathlineto{\pgfqpoint{2.071078in}{3.464281in}}%
\pgfpathlineto{\pgfqpoint{2.071078in}{3.455019in}}%
\pgfpathlineto{\pgfqpoint{2.076239in}{3.455019in}}%
\pgfpathlineto{\pgfqpoint{2.076239in}{3.432651in}}%
\pgfpathlineto{\pgfqpoint{2.081399in}{3.432651in}}%
\pgfpathlineto{\pgfqpoint{2.081399in}{3.429156in}}%
\pgfpathlineto{\pgfqpoint{2.086559in}{3.429156in}}%
\pgfpathlineto{\pgfqpoint{2.086559in}{3.407486in}}%
\pgfpathlineto{\pgfqpoint{2.091719in}{3.407486in}}%
\pgfpathlineto{\pgfqpoint{2.091719in}{3.400321in}}%
\pgfpathlineto{\pgfqpoint{2.096879in}{3.400321in}}%
\pgfpathlineto{\pgfqpoint{2.096879in}{3.383457in}}%
\pgfpathlineto{\pgfqpoint{2.102039in}{3.383457in}}%
\pgfpathlineto{\pgfqpoint{2.102039in}{3.364322in}}%
\pgfpathlineto{\pgfqpoint{2.107199in}{3.364322in}}%
\pgfpathlineto{\pgfqpoint{2.107199in}{3.347895in}}%
\pgfpathlineto{\pgfqpoint{2.112360in}{3.347895in}}%
\pgfpathlineto{\pgfqpoint{2.112360in}{3.334875in}}%
\pgfpathlineto{\pgfqpoint{2.117520in}{3.334875in}}%
\pgfpathlineto{\pgfqpoint{2.117520in}{3.327448in}}%
\pgfpathlineto{\pgfqpoint{2.122680in}{3.327448in}}%
\pgfpathlineto{\pgfqpoint{2.122680in}{3.312594in}}%
\pgfpathlineto{\pgfqpoint{2.127840in}{3.312594in}}%
\pgfpathlineto{\pgfqpoint{2.127840in}{3.295031in}}%
\pgfpathlineto{\pgfqpoint{2.133000in}{3.295031in}}%
\pgfpathlineto{\pgfqpoint{2.133000in}{3.276769in}}%
\pgfpathlineto{\pgfqpoint{2.138160in}{3.276769in}}%
\pgfpathlineto{\pgfqpoint{2.138160in}{3.274498in}}%
\pgfpathlineto{\pgfqpoint{2.143320in}{3.274498in}}%
\pgfpathlineto{\pgfqpoint{2.143320in}{3.250731in}}%
\pgfpathlineto{\pgfqpoint{2.148481in}{3.250731in}}%
\pgfpathlineto{\pgfqpoint{2.148481in}{3.242517in}}%
\pgfpathlineto{\pgfqpoint{2.153641in}{3.242517in}}%
\pgfpathlineto{\pgfqpoint{2.153641in}{3.225566in}}%
\pgfpathlineto{\pgfqpoint{2.158801in}{3.225566in}}%
\pgfpathlineto{\pgfqpoint{2.158801in}{3.222159in}}%
\pgfpathlineto{\pgfqpoint{2.163961in}{3.222159in}}%
\pgfpathlineto{\pgfqpoint{2.163961in}{3.204334in}}%
\pgfpathlineto{\pgfqpoint{2.169121in}{3.204334in}}%
\pgfpathlineto{\pgfqpoint{2.169121in}{3.185635in}}%
\pgfpathlineto{\pgfqpoint{2.174281in}{3.185635in}}%
\pgfpathlineto{\pgfqpoint{2.174281in}{3.177334in}}%
\pgfpathlineto{\pgfqpoint{2.179441in}{3.177334in}}%
\pgfpathlineto{\pgfqpoint{2.179441in}{3.171130in}}%
\pgfpathlineto{\pgfqpoint{2.184602in}{3.171130in}}%
\pgfpathlineto{\pgfqpoint{2.184602in}{3.166237in}}%
\pgfpathlineto{\pgfqpoint{2.189762in}{3.166237in}}%
\pgfpathlineto{\pgfqpoint{2.189762in}{3.149461in}}%
\pgfpathlineto{\pgfqpoint{2.194922in}{3.149461in}}%
\pgfpathlineto{\pgfqpoint{2.194922in}{3.131898in}}%
\pgfpathlineto{\pgfqpoint{2.200082in}{3.131898in}}%
\pgfpathlineto{\pgfqpoint{2.200082in}{3.138189in}}%
\pgfpathlineto{\pgfqpoint{2.205242in}{3.138189in}}%
\pgfpathlineto{\pgfqpoint{2.205242in}{3.122461in}}%
\pgfpathlineto{\pgfqpoint{2.210402in}{3.122461in}}%
\pgfpathlineto{\pgfqpoint{2.210402in}{3.116257in}}%
\pgfpathlineto{\pgfqpoint{2.215562in}{3.116257in}}%
\pgfpathlineto{\pgfqpoint{2.215562in}{3.103500in}}%
\pgfpathlineto{\pgfqpoint{2.220723in}{3.103500in}}%
\pgfpathlineto{\pgfqpoint{2.220723in}{3.089170in}}%
\pgfpathlineto{\pgfqpoint{2.225883in}{3.089170in}}%
\pgfpathlineto{\pgfqpoint{2.225883in}{3.086374in}}%
\pgfpathlineto{\pgfqpoint{2.231043in}{3.086374in}}%
\pgfpathlineto{\pgfqpoint{2.231043in}{3.083403in}}%
\pgfpathlineto{\pgfqpoint{2.236203in}{3.083403in}}%
\pgfpathlineto{\pgfqpoint{2.236203in}{3.063394in}}%
\pgfpathlineto{\pgfqpoint{2.241363in}{3.063394in}}%
\pgfpathlineto{\pgfqpoint{2.241363in}{3.060248in}}%
\pgfpathlineto{\pgfqpoint{2.246523in}{3.060248in}}%
\pgfpathlineto{\pgfqpoint{2.246523in}{3.056404in}}%
\pgfpathlineto{\pgfqpoint{2.251683in}{3.056404in}}%
\pgfpathlineto{\pgfqpoint{2.251683in}{3.040152in}}%
\pgfpathlineto{\pgfqpoint{2.256844in}{3.040152in}}%
\pgfpathlineto{\pgfqpoint{2.256844in}{3.031938in}}%
\pgfpathlineto{\pgfqpoint{2.262004in}{3.031938in}}%
\pgfpathlineto{\pgfqpoint{2.262004in}{3.029142in}}%
\pgfpathlineto{\pgfqpoint{2.267164in}{3.029142in}}%
\pgfpathlineto{\pgfqpoint{2.267164in}{3.020929in}}%
\pgfpathlineto{\pgfqpoint{2.272324in}{3.020929in}}%
\pgfpathlineto{\pgfqpoint{2.272324in}{3.014375in}}%
\pgfpathlineto{\pgfqpoint{2.277484in}{3.014375in}}%
\pgfpathlineto{\pgfqpoint{2.277484in}{3.009569in}}%
\pgfpathlineto{\pgfqpoint{2.282644in}{3.009569in}}%
\pgfpathlineto{\pgfqpoint{2.282644in}{3.004589in}}%
\pgfpathlineto{\pgfqpoint{2.287804in}{3.004589in}}%
\pgfpathlineto{\pgfqpoint{2.287804in}{2.997861in}}%
\pgfpathlineto{\pgfqpoint{2.292965in}{2.997861in}}%
\pgfpathlineto{\pgfqpoint{2.292965in}{2.987113in}}%
\pgfpathlineto{\pgfqpoint{2.298125in}{2.987113in}}%
\pgfpathlineto{\pgfqpoint{2.298125in}{2.982395in}}%
\pgfpathlineto{\pgfqpoint{2.303285in}{2.982395in}}%
\pgfpathlineto{\pgfqpoint{2.303285in}{2.975492in}}%
\pgfpathlineto{\pgfqpoint{2.308445in}{2.975492in}}%
\pgfpathlineto{\pgfqpoint{2.308445in}{2.969114in}}%
\pgfpathlineto{\pgfqpoint{2.313605in}{2.969114in}}%
\pgfpathlineto{\pgfqpoint{2.313605in}{2.974094in}}%
\pgfpathlineto{\pgfqpoint{2.318765in}{2.974094in}}%
\pgfpathlineto{\pgfqpoint{2.318765in}{2.966405in}}%
\pgfpathlineto{\pgfqpoint{2.323925in}{2.966405in}}%
\pgfpathlineto{\pgfqpoint{2.323925in}{2.956968in}}%
\pgfpathlineto{\pgfqpoint{2.329086in}{2.956968in}}%
\pgfpathlineto{\pgfqpoint{2.329086in}{2.958891in}}%
\pgfpathlineto{\pgfqpoint{2.334246in}{2.958891in}}%
\pgfpathlineto{\pgfqpoint{2.334246in}{2.951726in}}%
\pgfpathlineto{\pgfqpoint{2.339406in}{2.951726in}}%
\pgfpathlineto{\pgfqpoint{2.339406in}{2.948231in}}%
\pgfpathlineto{\pgfqpoint{2.344566in}{2.948231in}}%
\pgfpathlineto{\pgfqpoint{2.344566in}{2.939143in}}%
\pgfpathlineto{\pgfqpoint{2.360047in}{2.939755in}}%
\pgfpathlineto{\pgfqpoint{2.360047in}{2.929794in}}%
\pgfpathlineto{\pgfqpoint{2.365207in}{2.929794in}}%
\pgfpathlineto{\pgfqpoint{2.365207in}{2.924202in}}%
\pgfpathlineto{\pgfqpoint{2.370367in}{2.924202in}}%
\pgfpathlineto{\pgfqpoint{2.370367in}{2.922279in}}%
\pgfpathlineto{\pgfqpoint{2.375527in}{2.922279in}}%
\pgfpathlineto{\pgfqpoint{2.375527in}{2.918872in}}%
\pgfpathlineto{\pgfqpoint{2.380687in}{2.918872in}}%
\pgfpathlineto{\pgfqpoint{2.380687in}{2.914503in}}%
\pgfpathlineto{\pgfqpoint{2.391007in}{2.913542in}}%
\pgfpathlineto{\pgfqpoint{2.391007in}{2.909435in}}%
\pgfpathlineto{\pgfqpoint{2.396168in}{2.909435in}}%
\pgfpathlineto{\pgfqpoint{2.396168in}{2.905590in}}%
\pgfpathlineto{\pgfqpoint{2.401328in}{2.905590in}}%
\pgfpathlineto{\pgfqpoint{2.401328in}{2.909435in}}%
\pgfpathlineto{\pgfqpoint{2.406488in}{2.909435in}}%
\pgfpathlineto{\pgfqpoint{2.406488in}{2.898600in}}%
\pgfpathlineto{\pgfqpoint{2.411648in}{2.898600in}}%
\pgfpathlineto{\pgfqpoint{2.411648in}{2.892047in}}%
\pgfpathlineto{\pgfqpoint{2.416808in}{2.892047in}}%
\pgfpathlineto{\pgfqpoint{2.416808in}{2.893969in}}%
\pgfpathlineto{\pgfqpoint{2.421968in}{2.893969in}}%
\pgfpathlineto{\pgfqpoint{2.421968in}{2.895717in}}%
\pgfpathlineto{\pgfqpoint{2.427128in}{2.895717in}}%
\pgfpathlineto{\pgfqpoint{2.427128in}{2.887154in}}%
\pgfpathlineto{\pgfqpoint{2.432289in}{2.887154in}}%
\pgfpathlineto{\pgfqpoint{2.432289in}{2.888727in}}%
\pgfpathlineto{\pgfqpoint{2.442609in}{2.888377in}}%
\pgfpathlineto{\pgfqpoint{2.442609in}{2.879989in}}%
\pgfpathlineto{\pgfqpoint{2.447769in}{2.879989in}}%
\pgfpathlineto{\pgfqpoint{2.447769in}{2.882261in}}%
\pgfpathlineto{\pgfqpoint{2.452929in}{2.882261in}}%
\pgfpathlineto{\pgfqpoint{2.452929in}{2.878329in}}%
\pgfpathlineto{\pgfqpoint{2.458089in}{2.878329in}}%
\pgfpathlineto{\pgfqpoint{2.458089in}{2.876756in}}%
\pgfpathlineto{\pgfqpoint{2.468410in}{2.877455in}}%
\pgfpathlineto{\pgfqpoint{2.468410in}{2.873785in}}%
\pgfpathlineto{\pgfqpoint{2.473570in}{2.873785in}}%
\pgfpathlineto{\pgfqpoint{2.473570in}{2.869067in}}%
\pgfpathlineto{\pgfqpoint{2.478730in}{2.869067in}}%
\pgfpathlineto{\pgfqpoint{2.478730in}{2.866620in}}%
\pgfpathlineto{\pgfqpoint{2.489050in}{2.867057in}}%
\pgfpathlineto{\pgfqpoint{2.489050in}{2.863824in}}%
\pgfpathlineto{\pgfqpoint{2.499370in}{2.863649in}}%
\pgfpathlineto{\pgfqpoint{2.499370in}{2.861028in}}%
\pgfpathlineto{\pgfqpoint{2.509691in}{2.860067in}}%
\pgfpathlineto{\pgfqpoint{2.509691in}{2.858232in}}%
\pgfpathlineto{\pgfqpoint{2.514851in}{2.858232in}}%
\pgfpathlineto{\pgfqpoint{2.514851in}{2.855698in}}%
\pgfpathlineto{\pgfqpoint{2.530331in}{2.855785in}}%
\pgfpathlineto{\pgfqpoint{2.530331in}{2.853164in}}%
\pgfpathlineto{\pgfqpoint{2.540652in}{2.853339in}}%
\pgfpathlineto{\pgfqpoint{2.540652in}{2.850455in}}%
\pgfpathlineto{\pgfqpoint{2.545812in}{2.850455in}}%
\pgfpathlineto{\pgfqpoint{2.545812in}{2.852989in}}%
\pgfpathlineto{\pgfqpoint{2.550972in}{2.852989in}}%
\pgfpathlineto{\pgfqpoint{2.550972in}{2.847747in}}%
\pgfpathlineto{\pgfqpoint{2.556132in}{2.847747in}}%
\pgfpathlineto{\pgfqpoint{2.556132in}{2.848883in}}%
\pgfpathlineto{\pgfqpoint{2.561292in}{2.848883in}}%
\pgfpathlineto{\pgfqpoint{2.561292in}{2.846086in}}%
\pgfpathlineto{\pgfqpoint{2.566452in}{2.846086in}}%
\pgfpathlineto{\pgfqpoint{2.566452in}{2.848009in}}%
\pgfpathlineto{\pgfqpoint{2.571612in}{2.848009in}}%
\pgfpathlineto{\pgfqpoint{2.571612in}{2.845213in}}%
\pgfpathlineto{\pgfqpoint{2.581933in}{2.845912in}}%
\pgfpathlineto{\pgfqpoint{2.581933in}{2.843727in}}%
\pgfpathlineto{\pgfqpoint{2.597413in}{2.842941in}}%
\pgfpathlineto{\pgfqpoint{2.597413in}{2.841892in}}%
\pgfpathlineto{\pgfqpoint{2.602573in}{2.841892in}}%
\pgfpathlineto{\pgfqpoint{2.602573in}{2.839446in}}%
\pgfpathlineto{\pgfqpoint{2.607733in}{2.839446in}}%
\pgfpathlineto{\pgfqpoint{2.607733in}{2.841805in}}%
\pgfpathlineto{\pgfqpoint{2.612894in}{2.841805in}}%
\pgfpathlineto{\pgfqpoint{2.612894in}{2.839708in}}%
\pgfpathlineto{\pgfqpoint{2.633534in}{2.840407in}}%
\pgfpathlineto{\pgfqpoint{2.633534in}{2.838135in}}%
\pgfpathlineto{\pgfqpoint{2.643855in}{2.837349in}}%
\pgfpathlineto{\pgfqpoint{2.643855in}{2.836999in}}%
\pgfpathlineto{\pgfqpoint{2.674815in}{2.835951in}}%
\pgfpathlineto{\pgfqpoint{2.674815in}{2.835776in}}%
\pgfpathlineto{\pgfqpoint{2.690296in}{2.835863in}}%
\pgfpathlineto{\pgfqpoint{2.690296in}{2.833067in}}%
\pgfpathlineto{\pgfqpoint{2.695456in}{2.833067in}}%
\pgfpathlineto{\pgfqpoint{2.695456in}{2.835077in}}%
\pgfpathlineto{\pgfqpoint{2.700616in}{2.835077in}}%
\pgfpathlineto{\pgfqpoint{2.700616in}{2.833941in}}%
\pgfpathlineto{\pgfqpoint{2.716097in}{2.834028in}}%
\pgfpathlineto{\pgfqpoint{2.716097in}{2.832193in}}%
\pgfpathlineto{\pgfqpoint{2.721257in}{2.832193in}}%
\pgfpathlineto{\pgfqpoint{2.721257in}{2.833329in}}%
\pgfpathlineto{\pgfqpoint{2.731577in}{2.833417in}}%
\pgfpathlineto{\pgfqpoint{2.731577in}{2.831232in}}%
\pgfpathlineto{\pgfqpoint{2.747057in}{2.831582in}}%
\pgfpathlineto{\pgfqpoint{2.747057in}{2.832718in}}%
\pgfpathlineto{\pgfqpoint{2.752218in}{2.832718in}}%
\pgfpathlineto{\pgfqpoint{2.752218in}{2.831320in}}%
\pgfpathlineto{\pgfqpoint{2.778018in}{2.830970in}}%
\pgfpathlineto{\pgfqpoint{2.778018in}{2.830184in}}%
\pgfpathlineto{\pgfqpoint{2.907022in}{2.829135in}}%
\pgfpathlineto{\pgfqpoint{2.907022in}{2.829048in}}%
\pgfpathlineto{\pgfqpoint{3.134068in}{2.828786in}}%
\pgfpathlineto{\pgfqpoint{3.134068in}{2.828611in}}%
\pgfusepath{stroke}%
\end{pgfscope}%
\begin{pgfscope}%
\pgfsetrectcap%
\pgfsetmiterjoin%
\pgfsetlinewidth{0.803000pt}%
\definecolor{currentstroke}{rgb}{0.000000,0.000000,0.000000}%
\pgfsetstrokecolor{currentstroke}%
\pgfsetdash{}{0pt}%
\pgfpathmoveto{\pgfqpoint{0.706736in}{2.828611in}}%
\pgfpathlineto{\pgfqpoint{0.706736in}{4.381972in}}%
\pgfusepath{stroke}%
\end{pgfscope}%
\begin{pgfscope}%
\pgfsetrectcap%
\pgfsetmiterjoin%
\pgfsetlinewidth{0.803000pt}%
\definecolor{currentstroke}{rgb}{0.000000,0.000000,0.000000}%
\pgfsetstrokecolor{currentstroke}%
\pgfsetdash{}{0pt}%
\pgfpathmoveto{\pgfqpoint{3.249656in}{2.828611in}}%
\pgfpathlineto{\pgfqpoint{3.249656in}{4.381972in}}%
\pgfusepath{stroke}%
\end{pgfscope}%
\begin{pgfscope}%
\pgfsetrectcap%
\pgfsetmiterjoin%
\pgfsetlinewidth{0.803000pt}%
\definecolor{currentstroke}{rgb}{0.000000,0.000000,0.000000}%
\pgfsetstrokecolor{currentstroke}%
\pgfsetdash{}{0pt}%
\pgfpathmoveto{\pgfqpoint{0.706736in}{2.828611in}}%
\pgfpathlineto{\pgfqpoint{3.249656in}{2.828611in}}%
\pgfusepath{stroke}%
\end{pgfscope}%
\begin{pgfscope}%
\pgfsetrectcap%
\pgfsetmiterjoin%
\pgfsetlinewidth{0.803000pt}%
\definecolor{currentstroke}{rgb}{0.000000,0.000000,0.000000}%
\pgfsetstrokecolor{currentstroke}%
\pgfsetdash{}{0pt}%
\pgfpathmoveto{\pgfqpoint{0.706736in}{4.381972in}}%
\pgfpathlineto{\pgfqpoint{3.249656in}{4.381972in}}%
\pgfusepath{stroke}%
\end{pgfscope}%
\begin{pgfscope}%
\definecolor{textcolor}{rgb}{0.000000,0.000000,0.000000}%
\pgfsetstrokecolor{textcolor}%
\pgfsetfillcolor{textcolor}%
\pgftext[x=1.973255in,y=3.030604in,,]{\color{textcolor}\rmfamily\fontsize{10.000000}{12.000000}\selectfont IQR}%
\end{pgfscope}%
\begin{pgfscope}%
\definecolor{textcolor}{rgb}{0.000000,0.000000,0.000000}%
\pgfsetstrokecolor{textcolor}%
\pgfsetfillcolor{textcolor}%
\pgftext[x=0.706736in,y=4.465306in,left,base]{\color{textcolor}\ttfamily\fontsize{10.000000}{12.000000}\selectfont Bin [1.0, 1.2), 786,161 events}%
\end{pgfscope}%
\begin{pgfscope}%
\pgfsetbuttcap%
\pgfsetmiterjoin%
\definecolor{currentfill}{rgb}{1.000000,1.000000,1.000000}%
\pgfsetfillcolor{currentfill}%
\pgfsetlinewidth{0.000000pt}%
\definecolor{currentstroke}{rgb}{0.000000,0.000000,0.000000}%
\pgfsetstrokecolor{currentstroke}%
\pgfsetstrokeopacity{0.000000}%
\pgfsetdash{}{0pt}%
\pgfpathmoveto{\pgfqpoint{3.448267in}{2.828611in}}%
\pgfpathlineto{\pgfqpoint{5.991186in}{2.828611in}}%
\pgfpathlineto{\pgfqpoint{5.991186in}{4.381972in}}%
\pgfpathlineto{\pgfqpoint{3.448267in}{4.381972in}}%
\pgfpathclose%
\pgfusepath{fill}%
\end{pgfscope}%
\begin{pgfscope}%
\pgfpathrectangle{\pgfqpoint{3.448267in}{2.828611in}}{\pgfqpoint{2.542920in}{1.553361in}}%
\pgfusepath{clip}%
\pgfsetbuttcap%
\pgfsetmiterjoin%
\definecolor{currentfill}{rgb}{0.501961,0.501961,0.501961}%
\pgfsetfillcolor{currentfill}%
\pgfsetfillopacity{0.200000}%
\pgfsetlinewidth{0.000000pt}%
\definecolor{currentstroke}{rgb}{0.000000,0.000000,0.000000}%
\pgfsetstrokecolor{currentstroke}%
\pgfsetstrokeopacity{0.200000}%
\pgfsetdash{}{0pt}%
\pgfpathmoveto{\pgfqpoint{4.584837in}{2.828611in}}%
\pgfpathlineto{\pgfqpoint{4.584837in}{4.381972in}}%
\pgfpathlineto{\pgfqpoint{4.730233in}{4.381972in}}%
\pgfpathlineto{\pgfqpoint{4.730233in}{2.828611in}}%
\pgfpathclose%
\pgfusepath{fill}%
\end{pgfscope}%
\begin{pgfscope}%
\pgfsetbuttcap%
\pgfsetroundjoin%
\definecolor{currentfill}{rgb}{0.000000,0.000000,0.000000}%
\pgfsetfillcolor{currentfill}%
\pgfsetlinewidth{0.803000pt}%
\definecolor{currentstroke}{rgb}{0.000000,0.000000,0.000000}%
\pgfsetstrokecolor{currentstroke}%
\pgfsetdash{}{0pt}%
\pgfsys@defobject{currentmarker}{\pgfqpoint{0.000000in}{-0.048611in}}{\pgfqpoint{0.000000in}{0.000000in}}{%
\pgfpathmoveto{\pgfqpoint{0.000000in}{0.000000in}}%
\pgfpathlineto{\pgfqpoint{0.000000in}{-0.048611in}}%
\pgfusepath{stroke,fill}%
}%
\begin{pgfscope}%
\pgfsys@transformshift{3.794511in}{2.828611in}%
\pgfsys@useobject{currentmarker}{}%
\end{pgfscope}%
\end{pgfscope}%
\begin{pgfscope}%
\definecolor{textcolor}{rgb}{0.000000,0.000000,0.000000}%
\pgfsetstrokecolor{textcolor}%
\pgfsetfillcolor{textcolor}%
\pgftext[x=3.794511in,y=2.731389in,,top]{\color{textcolor}\rmfamily\fontsize{8.000000}{9.600000}\selectfont \(\displaystyle {−100}\)}%
\end{pgfscope}%
\begin{pgfscope}%
\pgfsetbuttcap%
\pgfsetroundjoin%
\definecolor{currentfill}{rgb}{0.000000,0.000000,0.000000}%
\pgfsetfillcolor{currentfill}%
\pgfsetlinewidth{0.803000pt}%
\definecolor{currentstroke}{rgb}{0.000000,0.000000,0.000000}%
\pgfsetstrokecolor{currentstroke}%
\pgfsetdash{}{0pt}%
\pgfsys@defobject{currentmarker}{\pgfqpoint{0.000000in}{-0.048611in}}{\pgfqpoint{0.000000in}{0.000000in}}{%
\pgfpathmoveto{\pgfqpoint{0.000000in}{0.000000in}}%
\pgfpathlineto{\pgfqpoint{0.000000in}{-0.048611in}}%
\pgfusepath{stroke,fill}%
}%
\begin{pgfscope}%
\pgfsys@transformshift{4.227204in}{2.828611in}%
\pgfsys@useobject{currentmarker}{}%
\end{pgfscope}%
\end{pgfscope}%
\begin{pgfscope}%
\definecolor{textcolor}{rgb}{0.000000,0.000000,0.000000}%
\pgfsetstrokecolor{textcolor}%
\pgfsetfillcolor{textcolor}%
\pgftext[x=4.227204in,y=2.731389in,,top]{\color{textcolor}\rmfamily\fontsize{8.000000}{9.600000}\selectfont \(\displaystyle {−50}\)}%
\end{pgfscope}%
\begin{pgfscope}%
\pgfsetbuttcap%
\pgfsetroundjoin%
\definecolor{currentfill}{rgb}{0.000000,0.000000,0.000000}%
\pgfsetfillcolor{currentfill}%
\pgfsetlinewidth{0.803000pt}%
\definecolor{currentstroke}{rgb}{0.000000,0.000000,0.000000}%
\pgfsetstrokecolor{currentstroke}%
\pgfsetdash{}{0pt}%
\pgfsys@defobject{currentmarker}{\pgfqpoint{0.000000in}{-0.048611in}}{\pgfqpoint{0.000000in}{0.000000in}}{%
\pgfpathmoveto{\pgfqpoint{0.000000in}{0.000000in}}%
\pgfpathlineto{\pgfqpoint{0.000000in}{-0.048611in}}%
\pgfusepath{stroke,fill}%
}%
\begin{pgfscope}%
\pgfsys@transformshift{4.659897in}{2.828611in}%
\pgfsys@useobject{currentmarker}{}%
\end{pgfscope}%
\end{pgfscope}%
\begin{pgfscope}%
\definecolor{textcolor}{rgb}{0.000000,0.000000,0.000000}%
\pgfsetstrokecolor{textcolor}%
\pgfsetfillcolor{textcolor}%
\pgftext[x=4.659897in,y=2.731389in,,top]{\color{textcolor}\rmfamily\fontsize{8.000000}{9.600000}\selectfont \(\displaystyle {0}\)}%
\end{pgfscope}%
\begin{pgfscope}%
\pgfsetbuttcap%
\pgfsetroundjoin%
\definecolor{currentfill}{rgb}{0.000000,0.000000,0.000000}%
\pgfsetfillcolor{currentfill}%
\pgfsetlinewidth{0.803000pt}%
\definecolor{currentstroke}{rgb}{0.000000,0.000000,0.000000}%
\pgfsetstrokecolor{currentstroke}%
\pgfsetdash{}{0pt}%
\pgfsys@defobject{currentmarker}{\pgfqpoint{0.000000in}{-0.048611in}}{\pgfqpoint{0.000000in}{0.000000in}}{%
\pgfpathmoveto{\pgfqpoint{0.000000in}{0.000000in}}%
\pgfpathlineto{\pgfqpoint{0.000000in}{-0.048611in}}%
\pgfusepath{stroke,fill}%
}%
\begin{pgfscope}%
\pgfsys@transformshift{5.092590in}{2.828611in}%
\pgfsys@useobject{currentmarker}{}%
\end{pgfscope}%
\end{pgfscope}%
\begin{pgfscope}%
\definecolor{textcolor}{rgb}{0.000000,0.000000,0.000000}%
\pgfsetstrokecolor{textcolor}%
\pgfsetfillcolor{textcolor}%
\pgftext[x=5.092590in,y=2.731389in,,top]{\color{textcolor}\rmfamily\fontsize{8.000000}{9.600000}\selectfont \(\displaystyle {50}\)}%
\end{pgfscope}%
\begin{pgfscope}%
\pgfsetbuttcap%
\pgfsetroundjoin%
\definecolor{currentfill}{rgb}{0.000000,0.000000,0.000000}%
\pgfsetfillcolor{currentfill}%
\pgfsetlinewidth{0.803000pt}%
\definecolor{currentstroke}{rgb}{0.000000,0.000000,0.000000}%
\pgfsetstrokecolor{currentstroke}%
\pgfsetdash{}{0pt}%
\pgfsys@defobject{currentmarker}{\pgfqpoint{0.000000in}{-0.048611in}}{\pgfqpoint{0.000000in}{0.000000in}}{%
\pgfpathmoveto{\pgfqpoint{0.000000in}{0.000000in}}%
\pgfpathlineto{\pgfqpoint{0.000000in}{-0.048611in}}%
\pgfusepath{stroke,fill}%
}%
\begin{pgfscope}%
\pgfsys@transformshift{5.525283in}{2.828611in}%
\pgfsys@useobject{currentmarker}{}%
\end{pgfscope}%
\end{pgfscope}%
\begin{pgfscope}%
\definecolor{textcolor}{rgb}{0.000000,0.000000,0.000000}%
\pgfsetstrokecolor{textcolor}%
\pgfsetfillcolor{textcolor}%
\pgftext[x=5.525283in,y=2.731389in,,top]{\color{textcolor}\rmfamily\fontsize{8.000000}{9.600000}\selectfont \(\displaystyle {100}\)}%
\end{pgfscope}%
\begin{pgfscope}%
\pgfsetbuttcap%
\pgfsetroundjoin%
\definecolor{currentfill}{rgb}{0.000000,0.000000,0.000000}%
\pgfsetfillcolor{currentfill}%
\pgfsetlinewidth{0.803000pt}%
\definecolor{currentstroke}{rgb}{0.000000,0.000000,0.000000}%
\pgfsetstrokecolor{currentstroke}%
\pgfsetdash{}{0pt}%
\pgfsys@defobject{currentmarker}{\pgfqpoint{0.000000in}{-0.048611in}}{\pgfqpoint{0.000000in}{0.000000in}}{%
\pgfpathmoveto{\pgfqpoint{0.000000in}{0.000000in}}%
\pgfpathlineto{\pgfqpoint{0.000000in}{-0.048611in}}%
\pgfusepath{stroke,fill}%
}%
\begin{pgfscope}%
\pgfsys@transformshift{5.957976in}{2.828611in}%
\pgfsys@useobject{currentmarker}{}%
\end{pgfscope}%
\end{pgfscope}%
\begin{pgfscope}%
\definecolor{textcolor}{rgb}{0.000000,0.000000,0.000000}%
\pgfsetstrokecolor{textcolor}%
\pgfsetfillcolor{textcolor}%
\pgftext[x=5.957976in,y=2.731389in,,top]{\color{textcolor}\rmfamily\fontsize{8.000000}{9.600000}\selectfont \(\displaystyle {150}\)}%
\end{pgfscope}%
\begin{pgfscope}%
\definecolor{textcolor}{rgb}{0.000000,0.000000,0.000000}%
\pgfsetstrokecolor{textcolor}%
\pgfsetfillcolor{textcolor}%
\pgftext[x=4.719726in,y=2.577167in,,top]{\color{textcolor}\rmfamily\fontsize{10.950000}{13.140000}\selectfont \(\displaystyle  \theta_{\textup{true}} - \theta_{\textup{reco}} \, [\textup{rad}] \)}%
\end{pgfscope}%
\begin{pgfscope}%
\pgfsetbuttcap%
\pgfsetroundjoin%
\definecolor{currentfill}{rgb}{0.000000,0.000000,0.000000}%
\pgfsetfillcolor{currentfill}%
\pgfsetlinewidth{0.803000pt}%
\definecolor{currentstroke}{rgb}{0.000000,0.000000,0.000000}%
\pgfsetstrokecolor{currentstroke}%
\pgfsetdash{}{0pt}%
\pgfsys@defobject{currentmarker}{\pgfqpoint{-0.048611in}{0.000000in}}{\pgfqpoint{-0.000000in}{0.000000in}}{%
\pgfpathmoveto{\pgfqpoint{-0.000000in}{0.000000in}}%
\pgfpathlineto{\pgfqpoint{-0.048611in}{0.000000in}}%
\pgfusepath{stroke,fill}%
}%
\begin{pgfscope}%
\pgfsys@transformshift{3.448267in}{2.828611in}%
\pgfsys@useobject{currentmarker}{}%
\end{pgfscope}%
\end{pgfscope}%
\begin{pgfscope}%
\pgfsetbuttcap%
\pgfsetroundjoin%
\definecolor{currentfill}{rgb}{0.000000,0.000000,0.000000}%
\pgfsetfillcolor{currentfill}%
\pgfsetlinewidth{0.803000pt}%
\definecolor{currentstroke}{rgb}{0.000000,0.000000,0.000000}%
\pgfsetstrokecolor{currentstroke}%
\pgfsetdash{}{0pt}%
\pgfsys@defobject{currentmarker}{\pgfqpoint{-0.048611in}{0.000000in}}{\pgfqpoint{-0.000000in}{0.000000in}}{%
\pgfpathmoveto{\pgfqpoint{-0.000000in}{0.000000in}}%
\pgfpathlineto{\pgfqpoint{-0.048611in}{0.000000in}}%
\pgfusepath{stroke,fill}%
}%
\begin{pgfscope}%
\pgfsys@transformshift{3.448267in}{3.232596in}%
\pgfsys@useobject{currentmarker}{}%
\end{pgfscope}%
\end{pgfscope}%
\begin{pgfscope}%
\pgfsetbuttcap%
\pgfsetroundjoin%
\definecolor{currentfill}{rgb}{0.000000,0.000000,0.000000}%
\pgfsetfillcolor{currentfill}%
\pgfsetlinewidth{0.803000pt}%
\definecolor{currentstroke}{rgb}{0.000000,0.000000,0.000000}%
\pgfsetstrokecolor{currentstroke}%
\pgfsetdash{}{0pt}%
\pgfsys@defobject{currentmarker}{\pgfqpoint{-0.048611in}{0.000000in}}{\pgfqpoint{-0.000000in}{0.000000in}}{%
\pgfpathmoveto{\pgfqpoint{-0.000000in}{0.000000in}}%
\pgfpathlineto{\pgfqpoint{-0.048611in}{0.000000in}}%
\pgfusepath{stroke,fill}%
}%
\begin{pgfscope}%
\pgfsys@transformshift{3.448267in}{3.636581in}%
\pgfsys@useobject{currentmarker}{}%
\end{pgfscope}%
\end{pgfscope}%
\begin{pgfscope}%
\pgfsetbuttcap%
\pgfsetroundjoin%
\definecolor{currentfill}{rgb}{0.000000,0.000000,0.000000}%
\pgfsetfillcolor{currentfill}%
\pgfsetlinewidth{0.803000pt}%
\definecolor{currentstroke}{rgb}{0.000000,0.000000,0.000000}%
\pgfsetstrokecolor{currentstroke}%
\pgfsetdash{}{0pt}%
\pgfsys@defobject{currentmarker}{\pgfqpoint{-0.048611in}{0.000000in}}{\pgfqpoint{-0.000000in}{0.000000in}}{%
\pgfpathmoveto{\pgfqpoint{-0.000000in}{0.000000in}}%
\pgfpathlineto{\pgfqpoint{-0.048611in}{0.000000in}}%
\pgfusepath{stroke,fill}%
}%
\begin{pgfscope}%
\pgfsys@transformshift{3.448267in}{4.040567in}%
\pgfsys@useobject{currentmarker}{}%
\end{pgfscope}%
\end{pgfscope}%
\begin{pgfscope}%
\pgfpathrectangle{\pgfqpoint{3.448267in}{2.828611in}}{\pgfqpoint{2.542920in}{1.553361in}}%
\pgfusepath{clip}%
\pgfsetbuttcap%
\pgfsetmiterjoin%
\pgfsetlinewidth{1.003750pt}%
\definecolor{currentstroke}{rgb}{0.313725,0.317647,0.309804}%
\pgfsetstrokecolor{currentstroke}%
\pgfsetdash{}{0pt}%
\pgfpathmoveto{\pgfqpoint{3.563854in}{2.828611in}}%
\pgfpathlineto{\pgfqpoint{3.563854in}{2.828904in}}%
\pgfpathlineto{\pgfqpoint{3.753860in}{2.828904in}}%
\pgfpathlineto{\pgfqpoint{3.753860in}{2.830074in}}%
\pgfpathlineto{\pgfqpoint{3.767432in}{2.829782in}}%
\pgfpathlineto{\pgfqpoint{3.767432in}{2.828904in}}%
\pgfpathlineto{\pgfqpoint{3.785528in}{2.829782in}}%
\pgfpathlineto{\pgfqpoint{3.785528in}{2.830074in}}%
\pgfpathlineto{\pgfqpoint{3.794576in}{2.829489in}}%
\pgfpathlineto{\pgfqpoint{3.794576in}{2.828904in}}%
\pgfpathlineto{\pgfqpoint{3.808148in}{2.829196in}}%
\pgfpathlineto{\pgfqpoint{3.808148in}{2.830074in}}%
\pgfpathlineto{\pgfqpoint{3.812672in}{2.830074in}}%
\pgfpathlineto{\pgfqpoint{3.812672in}{2.828904in}}%
\pgfpathlineto{\pgfqpoint{3.821720in}{2.829782in}}%
\pgfpathlineto{\pgfqpoint{3.821720in}{2.830367in}}%
\pgfpathlineto{\pgfqpoint{3.880531in}{2.830367in}}%
\pgfpathlineto{\pgfqpoint{3.880531in}{2.831538in}}%
\pgfpathlineto{\pgfqpoint{3.903151in}{2.831245in}}%
\pgfpathlineto{\pgfqpoint{3.903151in}{2.830074in}}%
\pgfpathlineto{\pgfqpoint{3.907675in}{2.830074in}}%
\pgfpathlineto{\pgfqpoint{3.907675in}{2.832123in}}%
\pgfpathlineto{\pgfqpoint{3.925771in}{2.832123in}}%
\pgfpathlineto{\pgfqpoint{3.925771in}{2.833294in}}%
\pgfpathlineto{\pgfqpoint{3.939343in}{2.832708in}}%
\pgfpathlineto{\pgfqpoint{3.939343in}{2.831830in}}%
\pgfpathlineto{\pgfqpoint{3.943867in}{2.831830in}}%
\pgfpathlineto{\pgfqpoint{3.943867in}{2.834171in}}%
\pgfpathlineto{\pgfqpoint{3.952915in}{2.834464in}}%
\pgfpathlineto{\pgfqpoint{3.952915in}{2.835927in}}%
\pgfpathlineto{\pgfqpoint{3.957439in}{2.835927in}}%
\pgfpathlineto{\pgfqpoint{3.957439in}{2.833586in}}%
\pgfpathlineto{\pgfqpoint{3.971011in}{2.834171in}}%
\pgfpathlineto{\pgfqpoint{3.971011in}{2.832416in}}%
\pgfpathlineto{\pgfqpoint{3.975535in}{2.832416in}}%
\pgfpathlineto{\pgfqpoint{3.975535in}{2.834757in}}%
\pgfpathlineto{\pgfqpoint{3.980059in}{2.834757in}}%
\pgfpathlineto{\pgfqpoint{3.980059in}{2.835927in}}%
\pgfpathlineto{\pgfqpoint{3.984582in}{2.835927in}}%
\pgfpathlineto{\pgfqpoint{3.984582in}{2.833586in}}%
\pgfpathlineto{\pgfqpoint{3.989106in}{2.833586in}}%
\pgfpathlineto{\pgfqpoint{3.989106in}{2.836805in}}%
\pgfpathlineto{\pgfqpoint{3.993630in}{2.836805in}}%
\pgfpathlineto{\pgfqpoint{3.993630in}{2.835635in}}%
\pgfpathlineto{\pgfqpoint{4.011726in}{2.835635in}}%
\pgfpathlineto{\pgfqpoint{4.011726in}{2.838561in}}%
\pgfpathlineto{\pgfqpoint{4.016250in}{2.838561in}}%
\pgfpathlineto{\pgfqpoint{4.016250in}{2.835635in}}%
\pgfpathlineto{\pgfqpoint{4.020774in}{2.835635in}}%
\pgfpathlineto{\pgfqpoint{4.020774in}{2.837098in}}%
\pgfpathlineto{\pgfqpoint{4.038870in}{2.836513in}}%
\pgfpathlineto{\pgfqpoint{4.038870in}{2.838269in}}%
\pgfpathlineto{\pgfqpoint{4.043394in}{2.838269in}}%
\pgfpathlineto{\pgfqpoint{4.043394in}{2.835635in}}%
\pgfpathlineto{\pgfqpoint{4.052442in}{2.836513in}}%
\pgfpathlineto{\pgfqpoint{4.052442in}{2.837683in}}%
\pgfpathlineto{\pgfqpoint{4.056966in}{2.837683in}}%
\pgfpathlineto{\pgfqpoint{4.056966in}{2.840610in}}%
\pgfpathlineto{\pgfqpoint{4.061490in}{2.840610in}}%
\pgfpathlineto{\pgfqpoint{4.061490in}{2.837976in}}%
\pgfpathlineto{\pgfqpoint{4.066014in}{2.837976in}}%
\pgfpathlineto{\pgfqpoint{4.066014in}{2.840903in}}%
\pgfpathlineto{\pgfqpoint{4.079586in}{2.840610in}}%
\pgfpathlineto{\pgfqpoint{4.079586in}{2.842073in}}%
\pgfpathlineto{\pgfqpoint{4.093158in}{2.841488in}}%
\pgfpathlineto{\pgfqpoint{4.093158in}{2.845000in}}%
\pgfpathlineto{\pgfqpoint{4.097682in}{2.845000in}}%
\pgfpathlineto{\pgfqpoint{4.097682in}{2.843829in}}%
\pgfpathlineto{\pgfqpoint{4.106729in}{2.843829in}}%
\pgfpathlineto{\pgfqpoint{4.106729in}{2.846756in}}%
\pgfpathlineto{\pgfqpoint{4.111253in}{2.846756in}}%
\pgfpathlineto{\pgfqpoint{4.111253in}{2.845000in}}%
\pgfpathlineto{\pgfqpoint{4.115777in}{2.845000in}}%
\pgfpathlineto{\pgfqpoint{4.115777in}{2.849975in}}%
\pgfpathlineto{\pgfqpoint{4.120301in}{2.849975in}}%
\pgfpathlineto{\pgfqpoint{4.120301in}{2.847341in}}%
\pgfpathlineto{\pgfqpoint{4.124825in}{2.847341in}}%
\pgfpathlineto{\pgfqpoint{4.124825in}{2.842951in}}%
\pgfpathlineto{\pgfqpoint{4.129349in}{2.842951in}}%
\pgfpathlineto{\pgfqpoint{4.129349in}{2.844707in}}%
\pgfpathlineto{\pgfqpoint{4.133873in}{2.844707in}}%
\pgfpathlineto{\pgfqpoint{4.133873in}{2.845878in}}%
\pgfpathlineto{\pgfqpoint{4.138397in}{2.845878in}}%
\pgfpathlineto{\pgfqpoint{4.138397in}{2.850560in}}%
\pgfpathlineto{\pgfqpoint{4.142921in}{2.850560in}}%
\pgfpathlineto{\pgfqpoint{4.142921in}{2.847341in}}%
\pgfpathlineto{\pgfqpoint{4.147445in}{2.847341in}}%
\pgfpathlineto{\pgfqpoint{4.147445in}{2.849682in}}%
\pgfpathlineto{\pgfqpoint{4.156493in}{2.848804in}}%
\pgfpathlineto{\pgfqpoint{4.156493in}{2.851146in}}%
\pgfpathlineto{\pgfqpoint{4.161017in}{2.851146in}}%
\pgfpathlineto{\pgfqpoint{4.161017in}{2.857292in}}%
\pgfpathlineto{\pgfqpoint{4.174589in}{2.857292in}}%
\pgfpathlineto{\pgfqpoint{4.174589in}{2.851731in}}%
\pgfpathlineto{\pgfqpoint{4.179113in}{2.851731in}}%
\pgfpathlineto{\pgfqpoint{4.179113in}{2.857877in}}%
\pgfpathlineto{\pgfqpoint{4.183637in}{2.857877in}}%
\pgfpathlineto{\pgfqpoint{4.183637in}{2.860511in}}%
\pgfpathlineto{\pgfqpoint{4.188161in}{2.860511in}}%
\pgfpathlineto{\pgfqpoint{4.188161in}{2.858755in}}%
\pgfpathlineto{\pgfqpoint{4.192685in}{2.858755in}}%
\pgfpathlineto{\pgfqpoint{4.192685in}{2.863145in}}%
\pgfpathlineto{\pgfqpoint{4.201733in}{2.862559in}}%
\pgfpathlineto{\pgfqpoint{4.201733in}{2.860218in}}%
\pgfpathlineto{\pgfqpoint{4.210781in}{2.860511in}}%
\pgfpathlineto{\pgfqpoint{4.210781in}{2.864608in}}%
\pgfpathlineto{\pgfqpoint{4.215305in}{2.864608in}}%
\pgfpathlineto{\pgfqpoint{4.215305in}{2.869876in}}%
\pgfpathlineto{\pgfqpoint{4.219829in}{2.869876in}}%
\pgfpathlineto{\pgfqpoint{4.219829in}{2.865193in}}%
\pgfpathlineto{\pgfqpoint{4.224353in}{2.865193in}}%
\pgfpathlineto{\pgfqpoint{4.224353in}{2.868413in}}%
\pgfpathlineto{\pgfqpoint{4.228876in}{2.868413in}}%
\pgfpathlineto{\pgfqpoint{4.228876in}{2.864315in}}%
\pgfpathlineto{\pgfqpoint{4.233400in}{2.864315in}}%
\pgfpathlineto{\pgfqpoint{4.233400in}{2.874558in}}%
\pgfpathlineto{\pgfqpoint{4.237924in}{2.874558in}}%
\pgfpathlineto{\pgfqpoint{4.237924in}{2.878948in}}%
\pgfpathlineto{\pgfqpoint{4.242448in}{2.878948in}}%
\pgfpathlineto{\pgfqpoint{4.242448in}{2.873388in}}%
\pgfpathlineto{\pgfqpoint{4.246972in}{2.873388in}}%
\pgfpathlineto{\pgfqpoint{4.246972in}{2.876022in}}%
\pgfpathlineto{\pgfqpoint{4.251496in}{2.876022in}}%
\pgfpathlineto{\pgfqpoint{4.251496in}{2.881290in}}%
\pgfpathlineto{\pgfqpoint{4.256020in}{2.881290in}}%
\pgfpathlineto{\pgfqpoint{4.256020in}{2.878070in}}%
\pgfpathlineto{\pgfqpoint{4.260544in}{2.878070in}}%
\pgfpathlineto{\pgfqpoint{4.260544in}{2.880412in}}%
\pgfpathlineto{\pgfqpoint{4.265068in}{2.880412in}}%
\pgfpathlineto{\pgfqpoint{4.265068in}{2.879241in}}%
\pgfpathlineto{\pgfqpoint{4.269592in}{2.879241in}}%
\pgfpathlineto{\pgfqpoint{4.269592in}{2.883046in}}%
\pgfpathlineto{\pgfqpoint{4.278640in}{2.882753in}}%
\pgfpathlineto{\pgfqpoint{4.278640in}{2.888606in}}%
\pgfpathlineto{\pgfqpoint{4.283164in}{2.888606in}}%
\pgfpathlineto{\pgfqpoint{4.283164in}{2.896508in}}%
\pgfpathlineto{\pgfqpoint{4.287688in}{2.896508in}}%
\pgfpathlineto{\pgfqpoint{4.287688in}{2.891825in}}%
\pgfpathlineto{\pgfqpoint{4.292212in}{2.891825in}}%
\pgfpathlineto{\pgfqpoint{4.292212in}{2.894167in}}%
\pgfpathlineto{\pgfqpoint{4.296736in}{2.894167in}}%
\pgfpathlineto{\pgfqpoint{4.296736in}{2.889191in}}%
\pgfpathlineto{\pgfqpoint{4.301260in}{2.889191in}}%
\pgfpathlineto{\pgfqpoint{4.301260in}{2.907922in}}%
\pgfpathlineto{\pgfqpoint{4.305784in}{2.907922in}}%
\pgfpathlineto{\pgfqpoint{4.305784in}{2.898556in}}%
\pgfpathlineto{\pgfqpoint{4.310308in}{2.898556in}}%
\pgfpathlineto{\pgfqpoint{4.310308in}{2.901190in}}%
\pgfpathlineto{\pgfqpoint{4.314832in}{2.901190in}}%
\pgfpathlineto{\pgfqpoint{4.314832in}{2.907044in}}%
\pgfpathlineto{\pgfqpoint{4.319356in}{2.907044in}}%
\pgfpathlineto{\pgfqpoint{4.319356in}{2.913482in}}%
\pgfpathlineto{\pgfqpoint{4.328404in}{2.912897in}}%
\pgfpathlineto{\pgfqpoint{4.328404in}{2.920506in}}%
\pgfpathlineto{\pgfqpoint{4.341976in}{2.921091in}}%
\pgfpathlineto{\pgfqpoint{4.341976in}{2.911433in}}%
\pgfpathlineto{\pgfqpoint{4.346500in}{2.911433in}}%
\pgfpathlineto{\pgfqpoint{4.346500in}{2.927237in}}%
\pgfpathlineto{\pgfqpoint{4.351023in}{2.927237in}}%
\pgfpathlineto{\pgfqpoint{4.351023in}{2.928700in}}%
\pgfpathlineto{\pgfqpoint{4.355547in}{2.928700in}}%
\pgfpathlineto{\pgfqpoint{4.355547in}{2.938358in}}%
\pgfpathlineto{\pgfqpoint{4.360071in}{2.938358in}}%
\pgfpathlineto{\pgfqpoint{4.360071in}{2.943041in}}%
\pgfpathlineto{\pgfqpoint{4.369119in}{2.942748in}}%
\pgfpathlineto{\pgfqpoint{4.369119in}{2.940114in}}%
\pgfpathlineto{\pgfqpoint{4.373643in}{2.940114in}}%
\pgfpathlineto{\pgfqpoint{4.373643in}{2.948601in}}%
\pgfpathlineto{\pgfqpoint{4.378167in}{2.948601in}}%
\pgfpathlineto{\pgfqpoint{4.378167in}{2.944504in}}%
\pgfpathlineto{\pgfqpoint{4.382691in}{2.944504in}}%
\pgfpathlineto{\pgfqpoint{4.382691in}{2.956503in}}%
\pgfpathlineto{\pgfqpoint{4.387215in}{2.956503in}}%
\pgfpathlineto{\pgfqpoint{4.387215in}{2.960600in}}%
\pgfpathlineto{\pgfqpoint{4.391739in}{2.960600in}}%
\pgfpathlineto{\pgfqpoint{4.391739in}{2.957088in}}%
\pgfpathlineto{\pgfqpoint{4.396263in}{2.957088in}}%
\pgfpathlineto{\pgfqpoint{4.396263in}{2.974648in}}%
\pgfpathlineto{\pgfqpoint{4.400787in}{2.974648in}}%
\pgfpathlineto{\pgfqpoint{4.400787in}{2.979916in}}%
\pgfpathlineto{\pgfqpoint{4.409835in}{2.979623in}}%
\pgfpathlineto{\pgfqpoint{4.409835in}{2.973185in}}%
\pgfpathlineto{\pgfqpoint{4.414359in}{2.973185in}}%
\pgfpathlineto{\pgfqpoint{4.414359in}{3.000109in}}%
\pgfpathlineto{\pgfqpoint{4.418883in}{3.000109in}}%
\pgfpathlineto{\pgfqpoint{4.418883in}{3.021181in}}%
\pgfpathlineto{\pgfqpoint{4.423407in}{3.021181in}}%
\pgfpathlineto{\pgfqpoint{4.423407in}{3.003036in}}%
\pgfpathlineto{\pgfqpoint{4.427931in}{3.003036in}}%
\pgfpathlineto{\pgfqpoint{4.427931in}{2.999817in}}%
\pgfpathlineto{\pgfqpoint{4.432455in}{2.999817in}}%
\pgfpathlineto{\pgfqpoint{4.432455in}{3.027619in}}%
\pgfpathlineto{\pgfqpoint{4.436979in}{3.027619in}}%
\pgfpathlineto{\pgfqpoint{4.436979in}{3.039911in}}%
\pgfpathlineto{\pgfqpoint{4.441503in}{3.039911in}}%
\pgfpathlineto{\pgfqpoint{4.441503in}{3.036692in}}%
\pgfpathlineto{\pgfqpoint{4.446027in}{3.036692in}}%
\pgfpathlineto{\pgfqpoint{4.446027in}{3.039033in}}%
\pgfpathlineto{\pgfqpoint{4.450551in}{3.039033in}}%
\pgfpathlineto{\pgfqpoint{4.450551in}{3.067128in}}%
\pgfpathlineto{\pgfqpoint{4.455075in}{3.067128in}}%
\pgfpathlineto{\pgfqpoint{4.455075in}{3.055422in}}%
\pgfpathlineto{\pgfqpoint{4.459599in}{3.055422in}}%
\pgfpathlineto{\pgfqpoint{4.459599in}{3.066835in}}%
\pgfpathlineto{\pgfqpoint{4.464123in}{3.066835in}}%
\pgfpathlineto{\pgfqpoint{4.464123in}{3.086736in}}%
\pgfpathlineto{\pgfqpoint{4.468647in}{3.086736in}}%
\pgfpathlineto{\pgfqpoint{4.468647in}{3.084688in}}%
\pgfpathlineto{\pgfqpoint{4.473170in}{3.084688in}}%
\pgfpathlineto{\pgfqpoint{4.473170in}{3.104003in}}%
\pgfpathlineto{\pgfqpoint{4.477694in}{3.104003in}}%
\pgfpathlineto{\pgfqpoint{4.477694in}{3.125075in}}%
\pgfpathlineto{\pgfqpoint{4.482218in}{3.125075in}}%
\pgfpathlineto{\pgfqpoint{4.482218in}{3.122148in}}%
\pgfpathlineto{\pgfqpoint{4.486742in}{3.122148in}}%
\pgfpathlineto{\pgfqpoint{4.486742in}{3.148780in}}%
\pgfpathlineto{\pgfqpoint{4.491266in}{3.148780in}}%
\pgfpathlineto{\pgfqpoint{4.491266in}{3.163413in}}%
\pgfpathlineto{\pgfqpoint{4.500314in}{3.163413in}}%
\pgfpathlineto{\pgfqpoint{4.500314in}{3.185655in}}%
\pgfpathlineto{\pgfqpoint{4.504838in}{3.185655in}}%
\pgfpathlineto{\pgfqpoint{4.504838in}{3.191801in}}%
\pgfpathlineto{\pgfqpoint{4.513886in}{3.192386in}}%
\pgfpathlineto{\pgfqpoint{4.513886in}{3.226042in}}%
\pgfpathlineto{\pgfqpoint{4.518410in}{3.226042in}}%
\pgfpathlineto{\pgfqpoint{4.518410in}{3.237456in}}%
\pgfpathlineto{\pgfqpoint{4.522934in}{3.237456in}}%
\pgfpathlineto{\pgfqpoint{4.522934in}{3.245943in}}%
\pgfpathlineto{\pgfqpoint{4.527458in}{3.245943in}}%
\pgfpathlineto{\pgfqpoint{4.527458in}{3.264088in}}%
\pgfpathlineto{\pgfqpoint{4.531982in}{3.264088in}}%
\pgfpathlineto{\pgfqpoint{4.531982in}{3.335204in}}%
\pgfpathlineto{\pgfqpoint{4.536506in}{3.335204in}}%
\pgfpathlineto{\pgfqpoint{4.536506in}{3.325839in}}%
\pgfpathlineto{\pgfqpoint{4.541030in}{3.325839in}}%
\pgfpathlineto{\pgfqpoint{4.541030in}{3.348374in}}%
\pgfpathlineto{\pgfqpoint{4.545554in}{3.348374in}}%
\pgfpathlineto{\pgfqpoint{4.545554in}{3.366518in}}%
\pgfpathlineto{\pgfqpoint{4.550078in}{3.366518in}}%
\pgfpathlineto{\pgfqpoint{4.550078in}{3.387590in}}%
\pgfpathlineto{\pgfqpoint{4.554602in}{3.387590in}}%
\pgfpathlineto{\pgfqpoint{4.554602in}{3.448170in}}%
\pgfpathlineto{\pgfqpoint{4.559126in}{3.448170in}}%
\pgfpathlineto{\pgfqpoint{4.559126in}{3.444366in}}%
\pgfpathlineto{\pgfqpoint{4.563650in}{3.444366in}}%
\pgfpathlineto{\pgfqpoint{4.563650in}{3.491776in}}%
\pgfpathlineto{\pgfqpoint{4.568174in}{3.491776in}}%
\pgfpathlineto{\pgfqpoint{4.568174in}{3.542114in}}%
\pgfpathlineto{\pgfqpoint{4.577222in}{3.542114in}}%
\pgfpathlineto{\pgfqpoint{4.577222in}{3.624058in}}%
\pgfpathlineto{\pgfqpoint{4.581746in}{3.624058in}}%
\pgfpathlineto{\pgfqpoint{4.581746in}{3.645130in}}%
\pgfpathlineto{\pgfqpoint{4.586270in}{3.645130in}}%
\pgfpathlineto{\pgfqpoint{4.586270in}{3.667372in}}%
\pgfpathlineto{\pgfqpoint{4.590794in}{3.667372in}}%
\pgfpathlineto{\pgfqpoint{4.590794in}{3.758682in}}%
\pgfpathlineto{\pgfqpoint{4.595317in}{3.758682in}}%
\pgfpathlineto{\pgfqpoint{4.595317in}{3.749024in}}%
\pgfpathlineto{\pgfqpoint{4.599841in}{3.749024in}}%
\pgfpathlineto{\pgfqpoint{4.599841in}{3.856722in}}%
\pgfpathlineto{\pgfqpoint{4.604365in}{3.856722in}}%
\pgfpathlineto{\pgfqpoint{4.604365in}{3.836236in}}%
\pgfpathlineto{\pgfqpoint{4.608889in}{3.836236in}}%
\pgfpathlineto{\pgfqpoint{4.608889in}{3.920815in}}%
\pgfpathlineto{\pgfqpoint{4.613413in}{3.920815in}}%
\pgfpathlineto{\pgfqpoint{4.613413in}{3.996321in}}%
\pgfpathlineto{\pgfqpoint{4.617937in}{3.996321in}}%
\pgfpathlineto{\pgfqpoint{4.617937in}{4.046658in}}%
\pgfpathlineto{\pgfqpoint{4.622461in}{4.046658in}}%
\pgfpathlineto{\pgfqpoint{4.622461in}{4.116604in}}%
\pgfpathlineto{\pgfqpoint{4.626985in}{4.116604in}}%
\pgfpathlineto{\pgfqpoint{4.626985in}{4.154357in}}%
\pgfpathlineto{\pgfqpoint{4.631509in}{4.154357in}}%
\pgfpathlineto{\pgfqpoint{4.631509in}{4.184501in}}%
\pgfpathlineto{\pgfqpoint{4.636033in}{4.184501in}}%
\pgfpathlineto{\pgfqpoint{4.636033in}{4.227814in}}%
\pgfpathlineto{\pgfqpoint{4.640557in}{4.227814in}}%
\pgfpathlineto{\pgfqpoint{4.640557in}{4.255617in}}%
\pgfpathlineto{\pgfqpoint{4.645081in}{4.255617in}}%
\pgfpathlineto{\pgfqpoint{4.645081in}{4.308003in}}%
\pgfpathlineto{\pgfqpoint{4.649605in}{4.308003in}}%
\pgfpathlineto{\pgfqpoint{4.649605in}{4.297760in}}%
\pgfpathlineto{\pgfqpoint{4.654129in}{4.297760in}}%
\pgfpathlineto{\pgfqpoint{4.654129in}{4.292199in}}%
\pgfpathlineto{\pgfqpoint{4.658653in}{4.292199in}}%
\pgfpathlineto{\pgfqpoint{4.658653in}{4.270835in}}%
\pgfpathlineto{\pgfqpoint{4.663177in}{4.270835in}}%
\pgfpathlineto{\pgfqpoint{4.663177in}{4.272884in}}%
\pgfpathlineto{\pgfqpoint{4.667701in}{4.272884in}}%
\pgfpathlineto{\pgfqpoint{4.667701in}{4.253568in}}%
\pgfpathlineto{\pgfqpoint{4.672225in}{4.253568in}}%
\pgfpathlineto{\pgfqpoint{4.672225in}{4.283419in}}%
\pgfpathlineto{\pgfqpoint{4.676749in}{4.283419in}}%
\pgfpathlineto{\pgfqpoint{4.676749in}{4.186842in}}%
\pgfpathlineto{\pgfqpoint{4.681273in}{4.186842in}}%
\pgfpathlineto{\pgfqpoint{4.681273in}{4.213474in}}%
\pgfpathlineto{\pgfqpoint{4.685797in}{4.213474in}}%
\pgfpathlineto{\pgfqpoint{4.685797in}{4.140309in}}%
\pgfpathlineto{\pgfqpoint{4.690321in}{4.140309in}}%
\pgfpathlineto{\pgfqpoint{4.690321in}{4.049585in}}%
\pgfpathlineto{\pgfqpoint{4.694845in}{4.049585in}}%
\pgfpathlineto{\pgfqpoint{4.694845in}{4.035830in}}%
\pgfpathlineto{\pgfqpoint{4.699369in}{4.035830in}}%
\pgfpathlineto{\pgfqpoint{4.699369in}{3.955348in}}%
\pgfpathlineto{\pgfqpoint{4.703893in}{3.955348in}}%
\pgfpathlineto{\pgfqpoint{4.703893in}{3.889500in}}%
\pgfpathlineto{\pgfqpoint{4.708417in}{3.889500in}}%
\pgfpathlineto{\pgfqpoint{4.708417in}{3.858771in}}%
\pgfpathlineto{\pgfqpoint{4.712941in}{3.858771in}}%
\pgfpathlineto{\pgfqpoint{4.712941in}{3.788825in}}%
\pgfpathlineto{\pgfqpoint{4.717464in}{3.788825in}}%
\pgfpathlineto{\pgfqpoint{4.717464in}{3.767461in}}%
\pgfpathlineto{\pgfqpoint{4.721988in}{3.767461in}}%
\pgfpathlineto{\pgfqpoint{4.721988in}{3.727660in}}%
\pgfpathlineto{\pgfqpoint{4.726512in}{3.727660in}}%
\pgfpathlineto{\pgfqpoint{4.726512in}{3.680542in}}%
\pgfpathlineto{\pgfqpoint{4.731036in}{3.680542in}}%
\pgfpathlineto{\pgfqpoint{4.731036in}{3.619668in}}%
\pgfpathlineto{\pgfqpoint{4.735560in}{3.619668in}}%
\pgfpathlineto{\pgfqpoint{4.735560in}{3.579574in}}%
\pgfpathlineto{\pgfqpoint{4.740084in}{3.579574in}}%
\pgfpathlineto{\pgfqpoint{4.740084in}{3.538017in}}%
\pgfpathlineto{\pgfqpoint{4.744608in}{3.538017in}}%
\pgfpathlineto{\pgfqpoint{4.744608in}{3.509629in}}%
\pgfpathlineto{\pgfqpoint{4.749132in}{3.509629in}}%
\pgfpathlineto{\pgfqpoint{4.749132in}{3.483875in}}%
\pgfpathlineto{\pgfqpoint{4.753656in}{3.483875in}}%
\pgfpathlineto{\pgfqpoint{4.753656in}{3.442902in}}%
\pgfpathlineto{\pgfqpoint{4.758180in}{3.442902in}}%
\pgfpathlineto{\pgfqpoint{4.758180in}{3.401052in}}%
\pgfpathlineto{\pgfqpoint{4.762704in}{3.401052in}}%
\pgfpathlineto{\pgfqpoint{4.762704in}{3.379981in}}%
\pgfpathlineto{\pgfqpoint{4.767228in}{3.379981in}}%
\pgfpathlineto{\pgfqpoint{4.767228in}{3.365348in}}%
\pgfpathlineto{\pgfqpoint{4.771752in}{3.365348in}}%
\pgfpathlineto{\pgfqpoint{4.771752in}{3.319986in}}%
\pgfpathlineto{\pgfqpoint{4.776276in}{3.319986in}}%
\pgfpathlineto{\pgfqpoint{4.776276in}{3.315010in}}%
\pgfpathlineto{\pgfqpoint{4.780800in}{3.315010in}}%
\pgfpathlineto{\pgfqpoint{4.780800in}{3.294524in}}%
\pgfpathlineto{\pgfqpoint{4.785324in}{3.294524in}}%
\pgfpathlineto{\pgfqpoint{4.785324in}{3.275209in}}%
\pgfpathlineto{\pgfqpoint{4.789848in}{3.275209in}}%
\pgfpathlineto{\pgfqpoint{4.789848in}{3.237748in}}%
\pgfpathlineto{\pgfqpoint{4.794372in}{3.237748in}}%
\pgfpathlineto{\pgfqpoint{4.794372in}{3.221652in}}%
\pgfpathlineto{\pgfqpoint{4.798896in}{3.221652in}}%
\pgfpathlineto{\pgfqpoint{4.798896in}{3.205263in}}%
\pgfpathlineto{\pgfqpoint{4.803420in}{3.205263in}}%
\pgfpathlineto{\pgfqpoint{4.803420in}{3.196483in}}%
\pgfpathlineto{\pgfqpoint{4.807944in}{3.196483in}}%
\pgfpathlineto{\pgfqpoint{4.807944in}{3.165169in}}%
\pgfpathlineto{\pgfqpoint{4.812468in}{3.165169in}}%
\pgfpathlineto{\pgfqpoint{4.812468in}{3.166632in}}%
\pgfpathlineto{\pgfqpoint{4.816992in}{3.166632in}}%
\pgfpathlineto{\pgfqpoint{4.816992in}{3.142342in}}%
\pgfpathlineto{\pgfqpoint{4.821516in}{3.142342in}}%
\pgfpathlineto{\pgfqpoint{4.821516in}{3.137074in}}%
\pgfpathlineto{\pgfqpoint{4.826040in}{3.137074in}}%
\pgfpathlineto{\pgfqpoint{4.826040in}{3.113076in}}%
\pgfpathlineto{\pgfqpoint{4.835088in}{3.113954in}}%
\pgfpathlineto{\pgfqpoint{4.835088in}{3.100784in}}%
\pgfpathlineto{\pgfqpoint{4.839611in}{3.100784in}}%
\pgfpathlineto{\pgfqpoint{4.839611in}{3.093175in}}%
\pgfpathlineto{\pgfqpoint{4.844135in}{3.093175in}}%
\pgfpathlineto{\pgfqpoint{4.844135in}{3.064787in}}%
\pgfpathlineto{\pgfqpoint{4.853183in}{3.064787in}}%
\pgfpathlineto{\pgfqpoint{4.853183in}{3.063616in}}%
\pgfpathlineto{\pgfqpoint{4.857707in}{3.063616in}}%
\pgfpathlineto{\pgfqpoint{4.857707in}{3.070347in}}%
\pgfpathlineto{\pgfqpoint{4.862231in}{3.070347in}}%
\pgfpathlineto{\pgfqpoint{4.862231in}{3.055714in}}%
\pgfpathlineto{\pgfqpoint{4.866755in}{3.055714in}}%
\pgfpathlineto{\pgfqpoint{4.866755in}{3.036106in}}%
\pgfpathlineto{\pgfqpoint{4.871279in}{3.036106in}}%
\pgfpathlineto{\pgfqpoint{4.871279in}{3.038740in}}%
\pgfpathlineto{\pgfqpoint{4.875803in}{3.038740in}}%
\pgfpathlineto{\pgfqpoint{4.875803in}{3.016791in}}%
\pgfpathlineto{\pgfqpoint{4.880327in}{3.016791in}}%
\pgfpathlineto{\pgfqpoint{4.880327in}{3.009182in}}%
\pgfpathlineto{\pgfqpoint{4.884851in}{3.009182in}}%
\pgfpathlineto{\pgfqpoint{4.884851in}{3.006548in}}%
\pgfpathlineto{\pgfqpoint{4.889375in}{3.006548in}}%
\pgfpathlineto{\pgfqpoint{4.889375in}{3.003914in}}%
\pgfpathlineto{\pgfqpoint{4.893899in}{3.003914in}}%
\pgfpathlineto{\pgfqpoint{4.893899in}{2.992793in}}%
\pgfpathlineto{\pgfqpoint{4.898423in}{2.992793in}}%
\pgfpathlineto{\pgfqpoint{4.898423in}{2.978745in}}%
\pgfpathlineto{\pgfqpoint{4.902947in}{2.978745in}}%
\pgfpathlineto{\pgfqpoint{4.902947in}{2.987818in}}%
\pgfpathlineto{\pgfqpoint{4.907471in}{2.987818in}}%
\pgfpathlineto{\pgfqpoint{4.907471in}{2.978160in}}%
\pgfpathlineto{\pgfqpoint{4.911995in}{2.978160in}}%
\pgfpathlineto{\pgfqpoint{4.911995in}{2.976989in}}%
\pgfpathlineto{\pgfqpoint{4.916519in}{2.976989in}}%
\pgfpathlineto{\pgfqpoint{4.916519in}{2.969380in}}%
\pgfpathlineto{\pgfqpoint{4.921043in}{2.969380in}}%
\pgfpathlineto{\pgfqpoint{4.921043in}{2.960600in}}%
\pgfpathlineto{\pgfqpoint{4.925567in}{2.960600in}}%
\pgfpathlineto{\pgfqpoint{4.925567in}{2.954454in}}%
\pgfpathlineto{\pgfqpoint{4.934615in}{2.953869in}}%
\pgfpathlineto{\pgfqpoint{4.934615in}{2.951235in}}%
\pgfpathlineto{\pgfqpoint{4.943663in}{2.952113in}}%
\pgfpathlineto{\pgfqpoint{4.943663in}{2.948016in}}%
\pgfpathlineto{\pgfqpoint{4.948187in}{2.948016in}}%
\pgfpathlineto{\pgfqpoint{4.948187in}{2.933090in}}%
\pgfpathlineto{\pgfqpoint{4.952711in}{2.933090in}}%
\pgfpathlineto{\pgfqpoint{4.952711in}{2.943333in}}%
\pgfpathlineto{\pgfqpoint{4.957235in}{2.943333in}}%
\pgfpathlineto{\pgfqpoint{4.957235in}{2.929286in}}%
\pgfpathlineto{\pgfqpoint{4.961758in}{2.929286in}}%
\pgfpathlineto{\pgfqpoint{4.961758in}{2.931627in}}%
\pgfpathlineto{\pgfqpoint{4.966282in}{2.931627in}}%
\pgfpathlineto{\pgfqpoint{4.966282in}{2.924310in}}%
\pgfpathlineto{\pgfqpoint{4.975330in}{2.924896in}}%
\pgfpathlineto{\pgfqpoint{4.975330in}{2.917872in}}%
\pgfpathlineto{\pgfqpoint{4.979854in}{2.917872in}}%
\pgfpathlineto{\pgfqpoint{4.979854in}{2.914067in}}%
\pgfpathlineto{\pgfqpoint{4.984378in}{2.914067in}}%
\pgfpathlineto{\pgfqpoint{4.984378in}{2.928115in}}%
\pgfpathlineto{\pgfqpoint{4.988902in}{2.928115in}}%
\pgfpathlineto{\pgfqpoint{4.988902in}{2.907044in}}%
\pgfpathlineto{\pgfqpoint{4.997950in}{2.906166in}}%
\pgfpathlineto{\pgfqpoint{4.997950in}{2.902946in}}%
\pgfpathlineto{\pgfqpoint{5.002474in}{2.902946in}}%
\pgfpathlineto{\pgfqpoint{5.002474in}{2.911141in}}%
\pgfpathlineto{\pgfqpoint{5.006998in}{2.911141in}}%
\pgfpathlineto{\pgfqpoint{5.006998in}{2.906458in}}%
\pgfpathlineto{\pgfqpoint{5.011522in}{2.906458in}}%
\pgfpathlineto{\pgfqpoint{5.011522in}{2.898556in}}%
\pgfpathlineto{\pgfqpoint{5.020570in}{2.899142in}}%
\pgfpathlineto{\pgfqpoint{5.020570in}{2.890947in}}%
\pgfpathlineto{\pgfqpoint{5.025094in}{2.890947in}}%
\pgfpathlineto{\pgfqpoint{5.025094in}{2.893874in}}%
\pgfpathlineto{\pgfqpoint{5.034142in}{2.893581in}}%
\pgfpathlineto{\pgfqpoint{5.034142in}{2.885387in}}%
\pgfpathlineto{\pgfqpoint{5.038666in}{2.885387in}}%
\pgfpathlineto{\pgfqpoint{5.038666in}{2.883924in}}%
\pgfpathlineto{\pgfqpoint{5.043190in}{2.883924in}}%
\pgfpathlineto{\pgfqpoint{5.043190in}{2.886265in}}%
\pgfpathlineto{\pgfqpoint{5.047714in}{2.886265in}}%
\pgfpathlineto{\pgfqpoint{5.047714in}{2.883631in}}%
\pgfpathlineto{\pgfqpoint{5.052238in}{2.883631in}}%
\pgfpathlineto{\pgfqpoint{5.052238in}{2.889484in}}%
\pgfpathlineto{\pgfqpoint{5.056762in}{2.889484in}}%
\pgfpathlineto{\pgfqpoint{5.056762in}{2.887143in}}%
\pgfpathlineto{\pgfqpoint{5.061286in}{2.887143in}}%
\pgfpathlineto{\pgfqpoint{5.061286in}{2.881582in}}%
\pgfpathlineto{\pgfqpoint{5.065810in}{2.881582in}}%
\pgfpathlineto{\pgfqpoint{5.065810in}{2.876314in}}%
\pgfpathlineto{\pgfqpoint{5.070334in}{2.876314in}}%
\pgfpathlineto{\pgfqpoint{5.070334in}{2.877778in}}%
\pgfpathlineto{\pgfqpoint{5.074858in}{2.877778in}}%
\pgfpathlineto{\pgfqpoint{5.074858in}{2.869876in}}%
\pgfpathlineto{\pgfqpoint{5.083905in}{2.869291in}}%
\pgfpathlineto{\pgfqpoint{5.083905in}{2.866071in}}%
\pgfpathlineto{\pgfqpoint{5.088429in}{2.866071in}}%
\pgfpathlineto{\pgfqpoint{5.088429in}{2.870461in}}%
\pgfpathlineto{\pgfqpoint{5.092953in}{2.870461in}}%
\pgfpathlineto{\pgfqpoint{5.092953in}{2.865193in}}%
\pgfpathlineto{\pgfqpoint{5.097477in}{2.865193in}}%
\pgfpathlineto{\pgfqpoint{5.097477in}{2.868705in}}%
\pgfpathlineto{\pgfqpoint{5.102001in}{2.868705in}}%
\pgfpathlineto{\pgfqpoint{5.102001in}{2.870754in}}%
\pgfpathlineto{\pgfqpoint{5.106525in}{2.870754in}}%
\pgfpathlineto{\pgfqpoint{5.106525in}{2.865486in}}%
\pgfpathlineto{\pgfqpoint{5.111049in}{2.865486in}}%
\pgfpathlineto{\pgfqpoint{5.111049in}{2.864023in}}%
\pgfpathlineto{\pgfqpoint{5.120097in}{2.864023in}}%
\pgfpathlineto{\pgfqpoint{5.120097in}{2.861974in}}%
\pgfpathlineto{\pgfqpoint{5.124621in}{2.861974in}}%
\pgfpathlineto{\pgfqpoint{5.124621in}{2.863437in}}%
\pgfpathlineto{\pgfqpoint{5.129145in}{2.863437in}}%
\pgfpathlineto{\pgfqpoint{5.129145in}{2.856121in}}%
\pgfpathlineto{\pgfqpoint{5.133669in}{2.856121in}}%
\pgfpathlineto{\pgfqpoint{5.133669in}{2.859925in}}%
\pgfpathlineto{\pgfqpoint{5.138193in}{2.859925in}}%
\pgfpathlineto{\pgfqpoint{5.138193in}{2.862559in}}%
\pgfpathlineto{\pgfqpoint{5.142717in}{2.862559in}}%
\pgfpathlineto{\pgfqpoint{5.142717in}{2.859633in}}%
\pgfpathlineto{\pgfqpoint{5.147241in}{2.859633in}}%
\pgfpathlineto{\pgfqpoint{5.147241in}{2.851438in}}%
\pgfpathlineto{\pgfqpoint{5.151765in}{2.851438in}}%
\pgfpathlineto{\pgfqpoint{5.151765in}{2.856999in}}%
\pgfpathlineto{\pgfqpoint{5.156289in}{2.856999in}}%
\pgfpathlineto{\pgfqpoint{5.156289in}{2.852609in}}%
\pgfpathlineto{\pgfqpoint{5.160813in}{2.852609in}}%
\pgfpathlineto{\pgfqpoint{5.160813in}{2.855243in}}%
\pgfpathlineto{\pgfqpoint{5.165337in}{2.855243in}}%
\pgfpathlineto{\pgfqpoint{5.165337in}{2.850268in}}%
\pgfpathlineto{\pgfqpoint{5.174385in}{2.851146in}}%
\pgfpathlineto{\pgfqpoint{5.174385in}{2.854950in}}%
\pgfpathlineto{\pgfqpoint{5.183433in}{2.854072in}}%
\pgfpathlineto{\pgfqpoint{5.183433in}{2.850268in}}%
\pgfpathlineto{\pgfqpoint{5.192481in}{2.850560in}}%
\pgfpathlineto{\pgfqpoint{5.192481in}{2.853780in}}%
\pgfpathlineto{\pgfqpoint{5.197005in}{2.853780in}}%
\pgfpathlineto{\pgfqpoint{5.197005in}{2.848804in}}%
\pgfpathlineto{\pgfqpoint{5.201529in}{2.848804in}}%
\pgfpathlineto{\pgfqpoint{5.201529in}{2.845878in}}%
\pgfpathlineto{\pgfqpoint{5.206052in}{2.845878in}}%
\pgfpathlineto{\pgfqpoint{5.206052in}{2.847926in}}%
\pgfpathlineto{\pgfqpoint{5.219624in}{2.847341in}}%
\pgfpathlineto{\pgfqpoint{5.219624in}{2.846756in}}%
\pgfpathlineto{\pgfqpoint{5.224148in}{2.846756in}}%
\pgfpathlineto{\pgfqpoint{5.224148in}{2.844415in}}%
\pgfpathlineto{\pgfqpoint{5.237720in}{2.845293in}}%
\pgfpathlineto{\pgfqpoint{5.237720in}{2.846171in}}%
\pgfpathlineto{\pgfqpoint{5.242244in}{2.846171in}}%
\pgfpathlineto{\pgfqpoint{5.242244in}{2.841488in}}%
\pgfpathlineto{\pgfqpoint{5.246768in}{2.841488in}}%
\pgfpathlineto{\pgfqpoint{5.246768in}{2.839439in}}%
\pgfpathlineto{\pgfqpoint{5.251292in}{2.839439in}}%
\pgfpathlineto{\pgfqpoint{5.251292in}{2.841488in}}%
\pgfpathlineto{\pgfqpoint{5.260340in}{2.840610in}}%
\pgfpathlineto{\pgfqpoint{5.260340in}{2.844415in}}%
\pgfpathlineto{\pgfqpoint{5.264864in}{2.844415in}}%
\pgfpathlineto{\pgfqpoint{5.264864in}{2.837683in}}%
\pgfpathlineto{\pgfqpoint{5.273912in}{2.837683in}}%
\pgfpathlineto{\pgfqpoint{5.273912in}{2.838854in}}%
\pgfpathlineto{\pgfqpoint{5.305580in}{2.839147in}}%
\pgfpathlineto{\pgfqpoint{5.305580in}{2.836220in}}%
\pgfpathlineto{\pgfqpoint{5.328199in}{2.835927in}}%
\pgfpathlineto{\pgfqpoint{5.328199in}{2.833294in}}%
\pgfpathlineto{\pgfqpoint{5.332723in}{2.833294in}}%
\pgfpathlineto{\pgfqpoint{5.332723in}{2.835049in}}%
\pgfpathlineto{\pgfqpoint{5.341771in}{2.835635in}}%
\pgfpathlineto{\pgfqpoint{5.341771in}{2.833294in}}%
\pgfpathlineto{\pgfqpoint{5.346295in}{2.833294in}}%
\pgfpathlineto{\pgfqpoint{5.346295in}{2.834757in}}%
\pgfpathlineto{\pgfqpoint{5.350819in}{2.834757in}}%
\pgfpathlineto{\pgfqpoint{5.350819in}{2.832123in}}%
\pgfpathlineto{\pgfqpoint{5.355343in}{2.832123in}}%
\pgfpathlineto{\pgfqpoint{5.355343in}{2.833294in}}%
\pgfpathlineto{\pgfqpoint{5.382487in}{2.833586in}}%
\pgfpathlineto{\pgfqpoint{5.382487in}{2.835342in}}%
\pgfpathlineto{\pgfqpoint{5.387011in}{2.835342in}}%
\pgfpathlineto{\pgfqpoint{5.387011in}{2.831245in}}%
\pgfpathlineto{\pgfqpoint{5.391535in}{2.831245in}}%
\pgfpathlineto{\pgfqpoint{5.391535in}{2.833294in}}%
\pgfpathlineto{\pgfqpoint{5.396059in}{2.833294in}}%
\pgfpathlineto{\pgfqpoint{5.396059in}{2.831830in}}%
\pgfpathlineto{\pgfqpoint{5.405107in}{2.832416in}}%
\pgfpathlineto{\pgfqpoint{5.405107in}{2.834171in}}%
\pgfpathlineto{\pgfqpoint{5.409631in}{2.834171in}}%
\pgfpathlineto{\pgfqpoint{5.409631in}{2.831245in}}%
\pgfpathlineto{\pgfqpoint{5.418679in}{2.830367in}}%
\pgfpathlineto{\pgfqpoint{5.418679in}{2.832708in}}%
\pgfpathlineto{\pgfqpoint{5.423203in}{2.832708in}}%
\pgfpathlineto{\pgfqpoint{5.423203in}{2.831245in}}%
\pgfpathlineto{\pgfqpoint{5.454870in}{2.831245in}}%
\pgfpathlineto{\pgfqpoint{5.454870in}{2.830074in}}%
\pgfpathlineto{\pgfqpoint{5.459394in}{2.830074in}}%
\pgfpathlineto{\pgfqpoint{5.459394in}{2.831538in}}%
\pgfpathlineto{\pgfqpoint{5.463918in}{2.831538in}}%
\pgfpathlineto{\pgfqpoint{5.463918in}{2.830074in}}%
\pgfpathlineto{\pgfqpoint{5.495586in}{2.830367in}}%
\pgfpathlineto{\pgfqpoint{5.495586in}{2.828904in}}%
\pgfpathlineto{\pgfqpoint{5.522730in}{2.828904in}}%
\pgfpathlineto{\pgfqpoint{5.522730in}{2.830074in}}%
\pgfpathlineto{\pgfqpoint{5.540826in}{2.829489in}}%
\pgfpathlineto{\pgfqpoint{5.540826in}{2.828904in}}%
\pgfpathlineto{\pgfqpoint{5.549874in}{2.829489in}}%
\pgfpathlineto{\pgfqpoint{5.549874in}{2.830074in}}%
\pgfpathlineto{\pgfqpoint{5.586065in}{2.829196in}}%
\pgfpathlineto{\pgfqpoint{5.586065in}{2.828904in}}%
\pgfpathlineto{\pgfqpoint{5.875599in}{2.828904in}}%
\pgfpathlineto{\pgfqpoint{5.875599in}{2.828611in}}%
\pgfusepath{stroke}%
\end{pgfscope}%
\begin{pgfscope}%
\pgfsetrectcap%
\pgfsetmiterjoin%
\pgfsetlinewidth{0.803000pt}%
\definecolor{currentstroke}{rgb}{0.000000,0.000000,0.000000}%
\pgfsetstrokecolor{currentstroke}%
\pgfsetdash{}{0pt}%
\pgfpathmoveto{\pgfqpoint{3.448267in}{2.828611in}}%
\pgfpathlineto{\pgfqpoint{3.448267in}{4.381972in}}%
\pgfusepath{stroke}%
\end{pgfscope}%
\begin{pgfscope}%
\pgfsetrectcap%
\pgfsetmiterjoin%
\pgfsetlinewidth{0.803000pt}%
\definecolor{currentstroke}{rgb}{0.000000,0.000000,0.000000}%
\pgfsetstrokecolor{currentstroke}%
\pgfsetdash{}{0pt}%
\pgfpathmoveto{\pgfqpoint{5.991186in}{2.828611in}}%
\pgfpathlineto{\pgfqpoint{5.991186in}{4.381972in}}%
\pgfusepath{stroke}%
\end{pgfscope}%
\begin{pgfscope}%
\pgfsetrectcap%
\pgfsetmiterjoin%
\pgfsetlinewidth{0.803000pt}%
\definecolor{currentstroke}{rgb}{0.000000,0.000000,0.000000}%
\pgfsetstrokecolor{currentstroke}%
\pgfsetdash{}{0pt}%
\pgfpathmoveto{\pgfqpoint{3.448267in}{2.828611in}}%
\pgfpathlineto{\pgfqpoint{5.991186in}{2.828611in}}%
\pgfusepath{stroke}%
\end{pgfscope}%
\begin{pgfscope}%
\pgfsetrectcap%
\pgfsetmiterjoin%
\pgfsetlinewidth{0.803000pt}%
\definecolor{currentstroke}{rgb}{0.000000,0.000000,0.000000}%
\pgfsetstrokecolor{currentstroke}%
\pgfsetdash{}{0pt}%
\pgfpathmoveto{\pgfqpoint{3.448267in}{4.381972in}}%
\pgfpathlineto{\pgfqpoint{5.991186in}{4.381972in}}%
\pgfusepath{stroke}%
\end{pgfscope}%
\begin{pgfscope}%
\definecolor{textcolor}{rgb}{0.000000,0.000000,0.000000}%
\pgfsetstrokecolor{textcolor}%
\pgfsetfillcolor{textcolor}%
\pgftext[x=4.659897in,y=3.030604in,,]{\color{textcolor}\rmfamily\fontsize{10.000000}{12.000000}\selectfont IQR}%
\end{pgfscope}%
\begin{pgfscope}%
\definecolor{textcolor}{rgb}{0.000000,0.000000,0.000000}%
\pgfsetstrokecolor{textcolor}%
\pgfsetfillcolor{textcolor}%
\pgftext[x=3.448267in,y=4.465306in,left,base]{\color{textcolor}\ttfamily\fontsize{10.000000}{12.000000}\selectfont Bin [1.7, 1.8), 264,055 events}%
\end{pgfscope}%
\begin{pgfscope}%
\pgfsetbuttcap%
\pgfsetmiterjoin%
\definecolor{currentfill}{rgb}{1.000000,1.000000,1.000000}%
\pgfsetfillcolor{currentfill}%
\pgfsetlinewidth{0.000000pt}%
\definecolor{currentstroke}{rgb}{0.000000,0.000000,0.000000}%
\pgfsetstrokecolor{currentstroke}%
\pgfsetstrokeopacity{0.000000}%
\pgfsetdash{}{0pt}%
\pgfpathmoveto{\pgfqpoint{0.706736in}{0.553528in}}%
\pgfpathlineto{\pgfqpoint{3.249656in}{0.553528in}}%
\pgfpathlineto{\pgfqpoint{3.249656in}{2.106889in}}%
\pgfpathlineto{\pgfqpoint{0.706736in}{2.106889in}}%
\pgfpathclose%
\pgfusepath{fill}%
\end{pgfscope}%
\begin{pgfscope}%
\pgfpathrectangle{\pgfqpoint{0.706736in}{0.553528in}}{\pgfqpoint{2.542920in}{1.553361in}}%
\pgfusepath{clip}%
\pgfsetbuttcap%
\pgfsetmiterjoin%
\definecolor{currentfill}{rgb}{0.501961,0.501961,0.501961}%
\pgfsetfillcolor{currentfill}%
\pgfsetfillopacity{0.200000}%
\pgfsetlinewidth{0.000000pt}%
\definecolor{currentstroke}{rgb}{0.000000,0.000000,0.000000}%
\pgfsetstrokecolor{currentstroke}%
\pgfsetstrokeopacity{0.200000}%
\pgfsetdash{}{0pt}%
\pgfpathmoveto{\pgfqpoint{0.822323in}{0.553528in}}%
\pgfpathlineto{\pgfqpoint{0.822323in}{2.106889in}}%
\pgfpathlineto{\pgfqpoint{1.894068in}{2.106889in}}%
\pgfpathlineto{\pgfqpoint{1.894068in}{0.553528in}}%
\pgfpathclose%
\pgfusepath{fill}%
\end{pgfscope}%
\begin{pgfscope}%
\pgfsetbuttcap%
\pgfsetroundjoin%
\definecolor{currentfill}{rgb}{0.000000,0.000000,0.000000}%
\pgfsetfillcolor{currentfill}%
\pgfsetlinewidth{0.803000pt}%
\definecolor{currentstroke}{rgb}{0.000000,0.000000,0.000000}%
\pgfsetstrokecolor{currentstroke}%
\pgfsetdash{}{0pt}%
\pgfsys@defobject{currentmarker}{\pgfqpoint{0.000000in}{-0.048611in}}{\pgfqpoint{0.000000in}{0.000000in}}{%
\pgfpathmoveto{\pgfqpoint{0.000000in}{0.000000in}}%
\pgfpathlineto{\pgfqpoint{0.000000in}{-0.048611in}}%
\pgfusepath{stroke,fill}%
}%
\begin{pgfscope}%
\pgfsys@transformshift{0.822323in}{0.553528in}%
\pgfsys@useobject{currentmarker}{}%
\end{pgfscope}%
\end{pgfscope}%
\begin{pgfscope}%
\definecolor{textcolor}{rgb}{0.000000,0.000000,0.000000}%
\pgfsetstrokecolor{textcolor}%
\pgfsetfillcolor{textcolor}%
\pgftext[x=0.822323in,y=0.456305in,,top]{\color{textcolor}\rmfamily\fontsize{8.000000}{9.600000}\selectfont \(\displaystyle {0}\)}%
\end{pgfscope}%
\begin{pgfscope}%
\pgfsetbuttcap%
\pgfsetroundjoin%
\definecolor{currentfill}{rgb}{0.000000,0.000000,0.000000}%
\pgfsetfillcolor{currentfill}%
\pgfsetlinewidth{0.803000pt}%
\definecolor{currentstroke}{rgb}{0.000000,0.000000,0.000000}%
\pgfsetstrokecolor{currentstroke}%
\pgfsetdash{}{0pt}%
\pgfsys@defobject{currentmarker}{\pgfqpoint{0.000000in}{-0.048611in}}{\pgfqpoint{0.000000in}{0.000000in}}{%
\pgfpathmoveto{\pgfqpoint{0.000000in}{0.000000in}}%
\pgfpathlineto{\pgfqpoint{0.000000in}{-0.048611in}}%
\pgfusepath{stroke,fill}%
}%
\begin{pgfscope}%
\pgfsys@transformshift{1.464477in}{0.553528in}%
\pgfsys@useobject{currentmarker}{}%
\end{pgfscope}%
\end{pgfscope}%
\begin{pgfscope}%
\definecolor{textcolor}{rgb}{0.000000,0.000000,0.000000}%
\pgfsetstrokecolor{textcolor}%
\pgfsetfillcolor{textcolor}%
\pgftext[x=1.464477in,y=0.456305in,,top]{\color{textcolor}\rmfamily\fontsize{8.000000}{9.600000}\selectfont \(\displaystyle {50}\)}%
\end{pgfscope}%
\begin{pgfscope}%
\pgfsetbuttcap%
\pgfsetroundjoin%
\definecolor{currentfill}{rgb}{0.000000,0.000000,0.000000}%
\pgfsetfillcolor{currentfill}%
\pgfsetlinewidth{0.803000pt}%
\definecolor{currentstroke}{rgb}{0.000000,0.000000,0.000000}%
\pgfsetstrokecolor{currentstroke}%
\pgfsetdash{}{0pt}%
\pgfsys@defobject{currentmarker}{\pgfqpoint{0.000000in}{-0.048611in}}{\pgfqpoint{0.000000in}{0.000000in}}{%
\pgfpathmoveto{\pgfqpoint{0.000000in}{0.000000in}}%
\pgfpathlineto{\pgfqpoint{0.000000in}{-0.048611in}}%
\pgfusepath{stroke,fill}%
}%
\begin{pgfscope}%
\pgfsys@transformshift{2.106630in}{0.553528in}%
\pgfsys@useobject{currentmarker}{}%
\end{pgfscope}%
\end{pgfscope}%
\begin{pgfscope}%
\definecolor{textcolor}{rgb}{0.000000,0.000000,0.000000}%
\pgfsetstrokecolor{textcolor}%
\pgfsetfillcolor{textcolor}%
\pgftext[x=2.106630in,y=0.456305in,,top]{\color{textcolor}\rmfamily\fontsize{8.000000}{9.600000}\selectfont \(\displaystyle {100}\)}%
\end{pgfscope}%
\begin{pgfscope}%
\pgfsetbuttcap%
\pgfsetroundjoin%
\definecolor{currentfill}{rgb}{0.000000,0.000000,0.000000}%
\pgfsetfillcolor{currentfill}%
\pgfsetlinewidth{0.803000pt}%
\definecolor{currentstroke}{rgb}{0.000000,0.000000,0.000000}%
\pgfsetstrokecolor{currentstroke}%
\pgfsetdash{}{0pt}%
\pgfsys@defobject{currentmarker}{\pgfqpoint{0.000000in}{-0.048611in}}{\pgfqpoint{0.000000in}{0.000000in}}{%
\pgfpathmoveto{\pgfqpoint{0.000000in}{0.000000in}}%
\pgfpathlineto{\pgfqpoint{0.000000in}{-0.048611in}}%
\pgfusepath{stroke,fill}%
}%
\begin{pgfscope}%
\pgfsys@transformshift{2.748784in}{0.553528in}%
\pgfsys@useobject{currentmarker}{}%
\end{pgfscope}%
\end{pgfscope}%
\begin{pgfscope}%
\definecolor{textcolor}{rgb}{0.000000,0.000000,0.000000}%
\pgfsetstrokecolor{textcolor}%
\pgfsetfillcolor{textcolor}%
\pgftext[x=2.748784in,y=0.456305in,,top]{\color{textcolor}\rmfamily\fontsize{8.000000}{9.600000}\selectfont \(\displaystyle {150}\)}%
\end{pgfscope}%
\begin{pgfscope}%
\definecolor{textcolor}{rgb}{0.000000,0.000000,0.000000}%
\pgfsetstrokecolor{textcolor}%
\pgfsetfillcolor{textcolor}%
\pgftext[x=1.978196in,y=0.302083in,,top]{\color{textcolor}\rmfamily\fontsize{10.950000}{13.140000}\selectfont \(\displaystyle  \phi_{\textup{true}} - \phi_{\textup{reco}} \, [\textup{rad}] \)}%
\end{pgfscope}%
\begin{pgfscope}%
\pgfsetbuttcap%
\pgfsetroundjoin%
\definecolor{currentfill}{rgb}{0.000000,0.000000,0.000000}%
\pgfsetfillcolor{currentfill}%
\pgfsetlinewidth{0.803000pt}%
\definecolor{currentstroke}{rgb}{0.000000,0.000000,0.000000}%
\pgfsetstrokecolor{currentstroke}%
\pgfsetdash{}{0pt}%
\pgfsys@defobject{currentmarker}{\pgfqpoint{-0.048611in}{0.000000in}}{\pgfqpoint{-0.000000in}{0.000000in}}{%
\pgfpathmoveto{\pgfqpoint{-0.000000in}{0.000000in}}%
\pgfpathlineto{\pgfqpoint{-0.048611in}{0.000000in}}%
\pgfusepath{stroke,fill}%
}%
\begin{pgfscope}%
\pgfsys@transformshift{0.706736in}{0.553528in}%
\pgfsys@useobject{currentmarker}{}%
\end{pgfscope}%
\end{pgfscope}%
\begin{pgfscope}%
\definecolor{textcolor}{rgb}{0.000000,0.000000,0.000000}%
\pgfsetstrokecolor{textcolor}%
\pgfsetfillcolor{textcolor}%
\pgftext[x=0.340606in, y=0.514972in, left, base]{\color{textcolor}\rmfamily\fontsize{8.000000}{9.600000}\selectfont \(\displaystyle {0.000}\)}%
\end{pgfscope}%
\begin{pgfscope}%
\pgfsetbuttcap%
\pgfsetroundjoin%
\definecolor{currentfill}{rgb}{0.000000,0.000000,0.000000}%
\pgfsetfillcolor{currentfill}%
\pgfsetlinewidth{0.803000pt}%
\definecolor{currentstroke}{rgb}{0.000000,0.000000,0.000000}%
\pgfsetstrokecolor{currentstroke}%
\pgfsetdash{}{0pt}%
\pgfsys@defobject{currentmarker}{\pgfqpoint{-0.048611in}{0.000000in}}{\pgfqpoint{-0.000000in}{0.000000in}}{%
\pgfpathmoveto{\pgfqpoint{-0.000000in}{0.000000in}}%
\pgfpathlineto{\pgfqpoint{-0.048611in}{0.000000in}}%
\pgfusepath{stroke,fill}%
}%
\begin{pgfscope}%
\pgfsys@transformshift{0.706736in}{0.815849in}%
\pgfsys@useobject{currentmarker}{}%
\end{pgfscope}%
\end{pgfscope}%
\begin{pgfscope}%
\definecolor{textcolor}{rgb}{0.000000,0.000000,0.000000}%
\pgfsetstrokecolor{textcolor}%
\pgfsetfillcolor{textcolor}%
\pgftext[x=0.340606in, y=0.777294in, left, base]{\color{textcolor}\rmfamily\fontsize{8.000000}{9.600000}\selectfont \(\displaystyle {0.005}\)}%
\end{pgfscope}%
\begin{pgfscope}%
\pgfsetbuttcap%
\pgfsetroundjoin%
\definecolor{currentfill}{rgb}{0.000000,0.000000,0.000000}%
\pgfsetfillcolor{currentfill}%
\pgfsetlinewidth{0.803000pt}%
\definecolor{currentstroke}{rgb}{0.000000,0.000000,0.000000}%
\pgfsetstrokecolor{currentstroke}%
\pgfsetdash{}{0pt}%
\pgfsys@defobject{currentmarker}{\pgfqpoint{-0.048611in}{0.000000in}}{\pgfqpoint{-0.000000in}{0.000000in}}{%
\pgfpathmoveto{\pgfqpoint{-0.000000in}{0.000000in}}%
\pgfpathlineto{\pgfqpoint{-0.048611in}{0.000000in}}%
\pgfusepath{stroke,fill}%
}%
\begin{pgfscope}%
\pgfsys@transformshift{0.706736in}{1.078171in}%
\pgfsys@useobject{currentmarker}{}%
\end{pgfscope}%
\end{pgfscope}%
\begin{pgfscope}%
\definecolor{textcolor}{rgb}{0.000000,0.000000,0.000000}%
\pgfsetstrokecolor{textcolor}%
\pgfsetfillcolor{textcolor}%
\pgftext[x=0.340606in, y=1.039615in, left, base]{\color{textcolor}\rmfamily\fontsize{8.000000}{9.600000}\selectfont \(\displaystyle {0.010}\)}%
\end{pgfscope}%
\begin{pgfscope}%
\pgfsetbuttcap%
\pgfsetroundjoin%
\definecolor{currentfill}{rgb}{0.000000,0.000000,0.000000}%
\pgfsetfillcolor{currentfill}%
\pgfsetlinewidth{0.803000pt}%
\definecolor{currentstroke}{rgb}{0.000000,0.000000,0.000000}%
\pgfsetstrokecolor{currentstroke}%
\pgfsetdash{}{0pt}%
\pgfsys@defobject{currentmarker}{\pgfqpoint{-0.048611in}{0.000000in}}{\pgfqpoint{-0.000000in}{0.000000in}}{%
\pgfpathmoveto{\pgfqpoint{-0.000000in}{0.000000in}}%
\pgfpathlineto{\pgfqpoint{-0.048611in}{0.000000in}}%
\pgfusepath{stroke,fill}%
}%
\begin{pgfscope}%
\pgfsys@transformshift{0.706736in}{1.340493in}%
\pgfsys@useobject{currentmarker}{}%
\end{pgfscope}%
\end{pgfscope}%
\begin{pgfscope}%
\definecolor{textcolor}{rgb}{0.000000,0.000000,0.000000}%
\pgfsetstrokecolor{textcolor}%
\pgfsetfillcolor{textcolor}%
\pgftext[x=0.340606in, y=1.301937in, left, base]{\color{textcolor}\rmfamily\fontsize{8.000000}{9.600000}\selectfont \(\displaystyle {0.015}\)}%
\end{pgfscope}%
\begin{pgfscope}%
\pgfsetbuttcap%
\pgfsetroundjoin%
\definecolor{currentfill}{rgb}{0.000000,0.000000,0.000000}%
\pgfsetfillcolor{currentfill}%
\pgfsetlinewidth{0.803000pt}%
\definecolor{currentstroke}{rgb}{0.000000,0.000000,0.000000}%
\pgfsetstrokecolor{currentstroke}%
\pgfsetdash{}{0pt}%
\pgfsys@defobject{currentmarker}{\pgfqpoint{-0.048611in}{0.000000in}}{\pgfqpoint{-0.000000in}{0.000000in}}{%
\pgfpathmoveto{\pgfqpoint{-0.000000in}{0.000000in}}%
\pgfpathlineto{\pgfqpoint{-0.048611in}{0.000000in}}%
\pgfusepath{stroke,fill}%
}%
\begin{pgfscope}%
\pgfsys@transformshift{0.706736in}{1.602814in}%
\pgfsys@useobject{currentmarker}{}%
\end{pgfscope}%
\end{pgfscope}%
\begin{pgfscope}%
\definecolor{textcolor}{rgb}{0.000000,0.000000,0.000000}%
\pgfsetstrokecolor{textcolor}%
\pgfsetfillcolor{textcolor}%
\pgftext[x=0.340606in, y=1.564259in, left, base]{\color{textcolor}\rmfamily\fontsize{8.000000}{9.600000}\selectfont \(\displaystyle {0.020}\)}%
\end{pgfscope}%
\begin{pgfscope}%
\pgfsetbuttcap%
\pgfsetroundjoin%
\definecolor{currentfill}{rgb}{0.000000,0.000000,0.000000}%
\pgfsetfillcolor{currentfill}%
\pgfsetlinewidth{0.803000pt}%
\definecolor{currentstroke}{rgb}{0.000000,0.000000,0.000000}%
\pgfsetstrokecolor{currentstroke}%
\pgfsetdash{}{0pt}%
\pgfsys@defobject{currentmarker}{\pgfqpoint{-0.048611in}{0.000000in}}{\pgfqpoint{-0.000000in}{0.000000in}}{%
\pgfpathmoveto{\pgfqpoint{-0.000000in}{0.000000in}}%
\pgfpathlineto{\pgfqpoint{-0.048611in}{0.000000in}}%
\pgfusepath{stroke,fill}%
}%
\begin{pgfscope}%
\pgfsys@transformshift{0.706736in}{1.865136in}%
\pgfsys@useobject{currentmarker}{}%
\end{pgfscope}%
\end{pgfscope}%
\begin{pgfscope}%
\definecolor{textcolor}{rgb}{0.000000,0.000000,0.000000}%
\pgfsetstrokecolor{textcolor}%
\pgfsetfillcolor{textcolor}%
\pgftext[x=0.340606in, y=1.826580in, left, base]{\color{textcolor}\rmfamily\fontsize{8.000000}{9.600000}\selectfont \(\displaystyle {0.025}\)}%
\end{pgfscope}%
\begin{pgfscope}%
\definecolor{textcolor}{rgb}{0.000000,0.000000,0.000000}%
\pgfsetstrokecolor{textcolor}%
\pgfsetfillcolor{textcolor}%
\pgftext[x=0.285050in,y=1.330208in,,bottom,rotate=90.000000]{\color{textcolor}\rmfamily\fontsize{10.950000}{13.140000}\selectfont Density}%
\end{pgfscope}%
\begin{pgfscope}%
\pgfpathrectangle{\pgfqpoint{0.706736in}{0.553528in}}{\pgfqpoint{2.542920in}{1.553361in}}%
\pgfusepath{clip}%
\pgfsetbuttcap%
\pgfsetmiterjoin%
\pgfsetlinewidth{1.003750pt}%
\definecolor{currentstroke}{rgb}{0.313725,0.317647,0.309804}%
\pgfsetstrokecolor{currentstroke}%
\pgfsetdash{}{0pt}%
\pgfpathmoveto{\pgfqpoint{0.822324in}{0.553528in}}%
\pgfpathlineto{\pgfqpoint{0.822324in}{1.157867in}}%
\pgfpathlineto{\pgfqpoint{0.843729in}{1.157867in}}%
\pgfpathlineto{\pgfqpoint{0.843729in}{1.166396in}}%
\pgfpathlineto{\pgfqpoint{0.865134in}{1.166396in}}%
\pgfpathlineto{\pgfqpoint{0.865134in}{1.159349in}}%
\pgfpathlineto{\pgfqpoint{0.907944in}{1.158588in}}%
\pgfpathlineto{\pgfqpoint{0.907944in}{1.154504in}}%
\pgfpathlineto{\pgfqpoint{0.929349in}{1.154504in}}%
\pgfpathlineto{\pgfqpoint{0.929349in}{1.142532in}}%
\pgfpathlineto{\pgfqpoint{0.950754in}{1.142532in}}%
\pgfpathlineto{\pgfqpoint{0.950754in}{1.141090in}}%
\pgfpathlineto{\pgfqpoint{0.972159in}{1.141090in}}%
\pgfpathlineto{\pgfqpoint{0.972159in}{1.146015in}}%
\pgfpathlineto{\pgfqpoint{0.993564in}{1.146015in}}%
\pgfpathlineto{\pgfqpoint{0.993564in}{1.133362in}}%
\pgfpathlineto{\pgfqpoint{1.014969in}{1.133362in}}%
\pgfpathlineto{\pgfqpoint{1.014969in}{1.124033in}}%
\pgfpathlineto{\pgfqpoint{1.036374in}{1.124033in}}%
\pgfpathlineto{\pgfqpoint{1.036374in}{1.113942in}}%
\pgfpathlineto{\pgfqpoint{1.057780in}{1.113942in}}%
\pgfpathlineto{\pgfqpoint{1.057780in}{1.102330in}}%
\pgfpathlineto{\pgfqpoint{1.079185in}{1.102330in}}%
\pgfpathlineto{\pgfqpoint{1.079185in}{1.094002in}}%
\pgfpathlineto{\pgfqpoint{1.100590in}{1.094002in}}%
\pgfpathlineto{\pgfqpoint{1.100590in}{1.085113in}}%
\pgfpathlineto{\pgfqpoint{1.121995in}{1.085113in}}%
\pgfpathlineto{\pgfqpoint{1.121995in}{1.076464in}}%
\pgfpathlineto{\pgfqpoint{1.143400in}{1.076464in}}%
\pgfpathlineto{\pgfqpoint{1.143400in}{1.062449in}}%
\pgfpathlineto{\pgfqpoint{1.164805in}{1.062449in}}%
\pgfpathlineto{\pgfqpoint{1.164805in}{1.053120in}}%
\pgfpathlineto{\pgfqpoint{1.186210in}{1.053120in}}%
\pgfpathlineto{\pgfqpoint{1.186210in}{1.045352in}}%
\pgfpathlineto{\pgfqpoint{1.207615in}{1.045352in}}%
\pgfpathlineto{\pgfqpoint{1.207615in}{1.026412in}}%
\pgfpathlineto{\pgfqpoint{1.229020in}{1.026412in}}%
\pgfpathlineto{\pgfqpoint{1.229020in}{1.031858in}}%
\pgfpathlineto{\pgfqpoint{1.250425in}{1.031858in}}%
\pgfpathlineto{\pgfqpoint{1.250425in}{1.016202in}}%
\pgfpathlineto{\pgfqpoint{1.271830in}{1.016202in}}%
\pgfpathlineto{\pgfqpoint{1.271830in}{1.003629in}}%
\pgfpathlineto{\pgfqpoint{1.293235in}{1.003629in}}%
\pgfpathlineto{\pgfqpoint{1.293235in}{0.996141in}}%
\pgfpathlineto{\pgfqpoint{1.314640in}{0.996141in}}%
\pgfpathlineto{\pgfqpoint{1.314640in}{0.987773in}}%
\pgfpathlineto{\pgfqpoint{1.336045in}{0.987773in}}%
\pgfpathlineto{\pgfqpoint{1.336045in}{0.984209in}}%
\pgfpathlineto{\pgfqpoint{1.357450in}{0.984209in}}%
\pgfpathlineto{\pgfqpoint{1.357450in}{0.972597in}}%
\pgfpathlineto{\pgfqpoint{1.378855in}{0.972597in}}%
\pgfpathlineto{\pgfqpoint{1.378855in}{0.963468in}}%
\pgfpathlineto{\pgfqpoint{1.400260in}{0.963468in}}%
\pgfpathlineto{\pgfqpoint{1.400260in}{0.946170in}}%
\pgfpathlineto{\pgfqpoint{1.443070in}{0.945289in}}%
\pgfpathlineto{\pgfqpoint{1.443070in}{0.932476in}}%
\pgfpathlineto{\pgfqpoint{1.464475in}{0.932476in}}%
\pgfpathlineto{\pgfqpoint{1.464475in}{0.922906in}}%
\pgfpathlineto{\pgfqpoint{1.485880in}{0.922906in}}%
\pgfpathlineto{\pgfqpoint{1.485880in}{0.918502in}}%
\pgfpathlineto{\pgfqpoint{1.507285in}{0.918502in}}%
\pgfpathlineto{\pgfqpoint{1.507285in}{0.906770in}}%
\pgfpathlineto{\pgfqpoint{1.528690in}{0.906770in}}%
\pgfpathlineto{\pgfqpoint{1.528690in}{0.891194in}}%
\pgfpathlineto{\pgfqpoint{1.571500in}{0.891114in}}%
\pgfpathlineto{\pgfqpoint{1.571500in}{0.882665in}}%
\pgfpathlineto{\pgfqpoint{1.592906in}{0.882665in}}%
\pgfpathlineto{\pgfqpoint{1.592906in}{0.876379in}}%
\pgfpathlineto{\pgfqpoint{1.614311in}{0.876379in}}%
\pgfpathlineto{\pgfqpoint{1.614311in}{0.872815in}}%
\pgfpathlineto{\pgfqpoint{1.635716in}{0.872815in}}%
\pgfpathlineto{\pgfqpoint{1.635716in}{0.865608in}}%
\pgfpathlineto{\pgfqpoint{1.657121in}{0.865608in}}%
\pgfpathlineto{\pgfqpoint{1.657121in}{0.861604in}}%
\pgfpathlineto{\pgfqpoint{1.678526in}{0.861604in}}%
\pgfpathlineto{\pgfqpoint{1.678526in}{0.851273in}}%
\pgfpathlineto{\pgfqpoint{1.699931in}{0.851273in}}%
\pgfpathlineto{\pgfqpoint{1.699931in}{0.843705in}}%
\pgfpathlineto{\pgfqpoint{1.721336in}{0.843705in}}%
\pgfpathlineto{\pgfqpoint{1.721336in}{0.833375in}}%
\pgfpathlineto{\pgfqpoint{1.742741in}{0.833375in}}%
\pgfpathlineto{\pgfqpoint{1.742741in}{0.830932in}}%
\pgfpathlineto{\pgfqpoint{1.764146in}{0.830932in}}%
\pgfpathlineto{\pgfqpoint{1.764146in}{0.829250in}}%
\pgfpathlineto{\pgfqpoint{1.785551in}{0.829250in}}%
\pgfpathlineto{\pgfqpoint{1.785551in}{0.820682in}}%
\pgfpathlineto{\pgfqpoint{1.806956in}{0.820682in}}%
\pgfpathlineto{\pgfqpoint{1.806956in}{0.816878in}}%
\pgfpathlineto{\pgfqpoint{1.828361in}{0.816878in}}%
\pgfpathlineto{\pgfqpoint{1.828361in}{0.810751in}}%
\pgfpathlineto{\pgfqpoint{1.849766in}{0.810751in}}%
\pgfpathlineto{\pgfqpoint{1.849766in}{0.808349in}}%
\pgfpathlineto{\pgfqpoint{1.871171in}{0.808349in}}%
\pgfpathlineto{\pgfqpoint{1.871171in}{0.806227in}}%
\pgfpathlineto{\pgfqpoint{1.913981in}{0.805146in}}%
\pgfpathlineto{\pgfqpoint{1.913981in}{0.794575in}}%
\pgfpathlineto{\pgfqpoint{1.935386in}{0.794575in}}%
\pgfpathlineto{\pgfqpoint{1.935386in}{0.786847in}}%
\pgfpathlineto{\pgfqpoint{1.956791in}{0.786847in}}%
\pgfpathlineto{\pgfqpoint{1.956791in}{0.791892in}}%
\pgfpathlineto{\pgfqpoint{1.978196in}{0.791892in}}%
\pgfpathlineto{\pgfqpoint{1.978196in}{0.785485in}}%
\pgfpathlineto{\pgfqpoint{1.999601in}{0.785485in}}%
\pgfpathlineto{\pgfqpoint{1.999601in}{0.777197in}}%
\pgfpathlineto{\pgfqpoint{2.021006in}{0.777197in}}%
\pgfpathlineto{\pgfqpoint{2.021006in}{0.770670in}}%
\pgfpathlineto{\pgfqpoint{2.042411in}{0.770670in}}%
\pgfpathlineto{\pgfqpoint{2.042411in}{0.771912in}}%
\pgfpathlineto{\pgfqpoint{2.063816in}{0.771912in}}%
\pgfpathlineto{\pgfqpoint{2.063816in}{0.767147in}}%
\pgfpathlineto{\pgfqpoint{2.085221in}{0.767147in}}%
\pgfpathlineto{\pgfqpoint{2.085221in}{0.759539in}}%
\pgfpathlineto{\pgfqpoint{2.106626in}{0.759539in}}%
\pgfpathlineto{\pgfqpoint{2.106626in}{0.756816in}}%
\pgfpathlineto{\pgfqpoint{2.128031in}{0.756816in}}%
\pgfpathlineto{\pgfqpoint{2.128031in}{0.760700in}}%
\pgfpathlineto{\pgfqpoint{2.149437in}{0.760700in}}%
\pgfpathlineto{\pgfqpoint{2.149437in}{0.753092in}}%
\pgfpathlineto{\pgfqpoint{2.170842in}{0.753092in}}%
\pgfpathlineto{\pgfqpoint{2.170842in}{0.751010in}}%
\pgfpathlineto{\pgfqpoint{2.213652in}{0.750690in}}%
\pgfpathlineto{\pgfqpoint{2.213652in}{0.739678in}}%
\pgfpathlineto{\pgfqpoint{2.235057in}{0.739678in}}%
\pgfpathlineto{\pgfqpoint{2.235057in}{0.744163in}}%
\pgfpathlineto{\pgfqpoint{2.256462in}{0.744163in}}%
\pgfpathlineto{\pgfqpoint{2.256462in}{0.745444in}}%
\pgfpathlineto{\pgfqpoint{2.277867in}{0.745444in}}%
\pgfpathlineto{\pgfqpoint{2.277867in}{0.739358in}}%
\pgfpathlineto{\pgfqpoint{2.299272in}{0.739358in}}%
\pgfpathlineto{\pgfqpoint{2.299272in}{0.733712in}}%
\pgfpathlineto{\pgfqpoint{2.342082in}{0.733352in}}%
\pgfpathlineto{\pgfqpoint{2.342082in}{0.731590in}}%
\pgfpathlineto{\pgfqpoint{2.384892in}{0.731110in}}%
\pgfpathlineto{\pgfqpoint{2.384892in}{0.726145in}}%
\pgfpathlineto{\pgfqpoint{2.406297in}{0.726145in}}%
\pgfpathlineto{\pgfqpoint{2.406297in}{0.722221in}}%
\pgfpathlineto{\pgfqpoint{2.427702in}{0.722221in}}%
\pgfpathlineto{\pgfqpoint{2.427702in}{0.718857in}}%
\pgfpathlineto{\pgfqpoint{2.449107in}{0.718857in}}%
\pgfpathlineto{\pgfqpoint{2.449107in}{0.722020in}}%
\pgfpathlineto{\pgfqpoint{2.470512in}{0.722020in}}%
\pgfpathlineto{\pgfqpoint{2.470512in}{0.720098in}}%
\pgfpathlineto{\pgfqpoint{2.491917in}{0.720098in}}%
\pgfpathlineto{\pgfqpoint{2.491917in}{0.712210in}}%
\pgfpathlineto{\pgfqpoint{2.513322in}{0.712210in}}%
\pgfpathlineto{\pgfqpoint{2.513322in}{0.715734in}}%
\pgfpathlineto{\pgfqpoint{2.577537in}{0.715734in}}%
\pgfpathlineto{\pgfqpoint{2.577537in}{0.717175in}}%
\pgfpathlineto{\pgfqpoint{2.598942in}{0.717175in}}%
\pgfpathlineto{\pgfqpoint{2.598942in}{0.714413in}}%
\pgfpathlineto{\pgfqpoint{2.620347in}{0.714413in}}%
\pgfpathlineto{\pgfqpoint{2.620347in}{0.708687in}}%
\pgfpathlineto{\pgfqpoint{2.663157in}{0.709247in}}%
\pgfpathlineto{\pgfqpoint{2.663157in}{0.706004in}}%
\pgfpathlineto{\pgfqpoint{2.684562in}{0.706004in}}%
\pgfpathlineto{\pgfqpoint{2.684562in}{0.707245in}}%
\pgfpathlineto{\pgfqpoint{2.705968in}{0.707245in}}%
\pgfpathlineto{\pgfqpoint{2.705968in}{0.705243in}}%
\pgfpathlineto{\pgfqpoint{2.748778in}{0.706044in}}%
\pgfpathlineto{\pgfqpoint{2.748778in}{0.703481in}}%
\pgfpathlineto{\pgfqpoint{2.770183in}{0.703481in}}%
\pgfpathlineto{\pgfqpoint{2.770183in}{0.705363in}}%
\pgfpathlineto{\pgfqpoint{2.791588in}{0.705363in}}%
\pgfpathlineto{\pgfqpoint{2.791588in}{0.700839in}}%
\pgfpathlineto{\pgfqpoint{2.812993in}{0.700839in}}%
\pgfpathlineto{\pgfqpoint{2.812993in}{0.704162in}}%
\pgfpathlineto{\pgfqpoint{2.834398in}{0.704162in}}%
\pgfpathlineto{\pgfqpoint{2.834398in}{0.698837in}}%
\pgfpathlineto{\pgfqpoint{2.855803in}{0.698837in}}%
\pgfpathlineto{\pgfqpoint{2.855803in}{0.697355in}}%
\pgfpathlineto{\pgfqpoint{2.898613in}{0.696594in}}%
\pgfpathlineto{\pgfqpoint{2.898613in}{0.700358in}}%
\pgfpathlineto{\pgfqpoint{2.920018in}{0.700358in}}%
\pgfpathlineto{\pgfqpoint{2.920018in}{0.697996in}}%
\pgfpathlineto{\pgfqpoint{2.941423in}{0.697996in}}%
\pgfpathlineto{\pgfqpoint{2.941423in}{0.699477in}}%
\pgfpathlineto{\pgfqpoint{2.962828in}{0.699477in}}%
\pgfpathlineto{\pgfqpoint{2.962828in}{0.696714in}}%
\pgfpathlineto{\pgfqpoint{3.005638in}{0.697675in}}%
\pgfpathlineto{\pgfqpoint{3.005638in}{0.692470in}}%
\pgfpathlineto{\pgfqpoint{3.027043in}{0.692470in}}%
\pgfpathlineto{\pgfqpoint{3.027043in}{0.696234in}}%
\pgfpathlineto{\pgfqpoint{3.048448in}{0.696234in}}%
\pgfpathlineto{\pgfqpoint{3.048448in}{0.693751in}}%
\pgfpathlineto{\pgfqpoint{3.069853in}{0.693751in}}%
\pgfpathlineto{\pgfqpoint{3.069853in}{0.696795in}}%
\pgfpathlineto{\pgfqpoint{3.091258in}{0.696795in}}%
\pgfpathlineto{\pgfqpoint{3.091258in}{0.695513in}}%
\pgfpathlineto{\pgfqpoint{3.112663in}{0.695513in}}%
\pgfpathlineto{\pgfqpoint{3.112663in}{0.698596in}}%
\pgfpathlineto{\pgfqpoint{3.134068in}{0.698596in}}%
\pgfpathlineto{\pgfqpoint{3.134068in}{0.553528in}}%
\pgfpathlineto{\pgfqpoint{3.134068in}{0.553528in}}%
\pgfusepath{stroke}%
\end{pgfscope}%
\begin{pgfscope}%
\pgfsetrectcap%
\pgfsetmiterjoin%
\pgfsetlinewidth{0.803000pt}%
\definecolor{currentstroke}{rgb}{0.000000,0.000000,0.000000}%
\pgfsetstrokecolor{currentstroke}%
\pgfsetdash{}{0pt}%
\pgfpathmoveto{\pgfqpoint{0.706736in}{0.553528in}}%
\pgfpathlineto{\pgfqpoint{0.706736in}{2.106889in}}%
\pgfusepath{stroke}%
\end{pgfscope}%
\begin{pgfscope}%
\pgfsetrectcap%
\pgfsetmiterjoin%
\pgfsetlinewidth{0.803000pt}%
\definecolor{currentstroke}{rgb}{0.000000,0.000000,0.000000}%
\pgfsetstrokecolor{currentstroke}%
\pgfsetdash{}{0pt}%
\pgfpathmoveto{\pgfqpoint{3.249656in}{0.553528in}}%
\pgfpathlineto{\pgfqpoint{3.249656in}{2.106889in}}%
\pgfusepath{stroke}%
\end{pgfscope}%
\begin{pgfscope}%
\pgfsetrectcap%
\pgfsetmiterjoin%
\pgfsetlinewidth{0.803000pt}%
\definecolor{currentstroke}{rgb}{0.000000,0.000000,0.000000}%
\pgfsetstrokecolor{currentstroke}%
\pgfsetdash{}{0pt}%
\pgfpathmoveto{\pgfqpoint{0.706736in}{0.553528in}}%
\pgfpathlineto{\pgfqpoint{3.249656in}{0.553528in}}%
\pgfusepath{stroke}%
\end{pgfscope}%
\begin{pgfscope}%
\pgfsetrectcap%
\pgfsetmiterjoin%
\pgfsetlinewidth{0.803000pt}%
\definecolor{currentstroke}{rgb}{0.000000,0.000000,0.000000}%
\pgfsetstrokecolor{currentstroke}%
\pgfsetdash{}{0pt}%
\pgfpathmoveto{\pgfqpoint{0.706736in}{2.106889in}}%
\pgfpathlineto{\pgfqpoint{3.249656in}{2.106889in}}%
\pgfusepath{stroke}%
\end{pgfscope}%
\begin{pgfscope}%
\definecolor{textcolor}{rgb}{0.000000,0.000000,0.000000}%
\pgfsetstrokecolor{textcolor}%
\pgfsetfillcolor{textcolor}%
\pgftext[x=0.950754in,y=0.658456in,left,]{\color{textcolor}\rmfamily\fontsize{10.000000}{12.000000}\selectfont 68th pctl}%
\end{pgfscope}%
\begin{pgfscope}%
\definecolor{textcolor}{rgb}{0.000000,0.000000,0.000000}%
\pgfsetstrokecolor{textcolor}%
\pgfsetfillcolor{textcolor}%
\pgftext[x=0.706736in,y=2.190222in,left,base]{\color{textcolor}\ttfamily\fontsize{10.000000}{12.000000}\selectfont Bin [1.0, 1.2), 786,161 events}%
\end{pgfscope}%
\begin{pgfscope}%
\pgfsetbuttcap%
\pgfsetmiterjoin%
\definecolor{currentfill}{rgb}{1.000000,1.000000,1.000000}%
\pgfsetfillcolor{currentfill}%
\pgfsetlinewidth{0.000000pt}%
\definecolor{currentstroke}{rgb}{0.000000,0.000000,0.000000}%
\pgfsetstrokecolor{currentstroke}%
\pgfsetstrokeopacity{0.000000}%
\pgfsetdash{}{0pt}%
\pgfpathmoveto{\pgfqpoint{3.448267in}{0.553528in}}%
\pgfpathlineto{\pgfqpoint{5.991186in}{0.553528in}}%
\pgfpathlineto{\pgfqpoint{5.991186in}{2.106889in}}%
\pgfpathlineto{\pgfqpoint{3.448267in}{2.106889in}}%
\pgfpathclose%
\pgfusepath{fill}%
\end{pgfscope}%
\begin{pgfscope}%
\pgfpathrectangle{\pgfqpoint{3.448267in}{0.553528in}}{\pgfqpoint{2.542920in}{1.553361in}}%
\pgfusepath{clip}%
\pgfsetbuttcap%
\pgfsetmiterjoin%
\definecolor{currentfill}{rgb}{0.501961,0.501961,0.501961}%
\pgfsetfillcolor{currentfill}%
\pgfsetfillopacity{0.200000}%
\pgfsetlinewidth{0.000000pt}%
\definecolor{currentstroke}{rgb}{0.000000,0.000000,0.000000}%
\pgfsetstrokecolor{currentstroke}%
\pgfsetstrokeopacity{0.200000}%
\pgfsetdash{}{0pt}%
\pgfpathmoveto{\pgfqpoint{3.563854in}{0.553528in}}%
\pgfpathlineto{\pgfqpoint{3.563854in}{2.106889in}}%
\pgfpathlineto{\pgfqpoint{4.140732in}{2.106889in}}%
\pgfpathlineto{\pgfqpoint{4.140732in}{0.553528in}}%
\pgfpathclose%
\pgfusepath{fill}%
\end{pgfscope}%
\begin{pgfscope}%
\pgfsetbuttcap%
\pgfsetroundjoin%
\definecolor{currentfill}{rgb}{0.000000,0.000000,0.000000}%
\pgfsetfillcolor{currentfill}%
\pgfsetlinewidth{0.803000pt}%
\definecolor{currentstroke}{rgb}{0.000000,0.000000,0.000000}%
\pgfsetstrokecolor{currentstroke}%
\pgfsetdash{}{0pt}%
\pgfsys@defobject{currentmarker}{\pgfqpoint{0.000000in}{-0.048611in}}{\pgfqpoint{0.000000in}{0.000000in}}{%
\pgfpathmoveto{\pgfqpoint{0.000000in}{0.000000in}}%
\pgfpathlineto{\pgfqpoint{0.000000in}{-0.048611in}}%
\pgfusepath{stroke,fill}%
}%
\begin{pgfscope}%
\pgfsys@transformshift{3.563854in}{0.553528in}%
\pgfsys@useobject{currentmarker}{}%
\end{pgfscope}%
\end{pgfscope}%
\begin{pgfscope}%
\definecolor{textcolor}{rgb}{0.000000,0.000000,0.000000}%
\pgfsetstrokecolor{textcolor}%
\pgfsetfillcolor{textcolor}%
\pgftext[x=3.563854in,y=0.456305in,,top]{\color{textcolor}\rmfamily\fontsize{8.000000}{9.600000}\selectfont \(\displaystyle {0}\)}%
\end{pgfscope}%
\begin{pgfscope}%
\pgfsetbuttcap%
\pgfsetroundjoin%
\definecolor{currentfill}{rgb}{0.000000,0.000000,0.000000}%
\pgfsetfillcolor{currentfill}%
\pgfsetlinewidth{0.803000pt}%
\definecolor{currentstroke}{rgb}{0.000000,0.000000,0.000000}%
\pgfsetstrokecolor{currentstroke}%
\pgfsetdash{}{0pt}%
\pgfsys@defobject{currentmarker}{\pgfqpoint{0.000000in}{-0.048611in}}{\pgfqpoint{0.000000in}{0.000000in}}{%
\pgfpathmoveto{\pgfqpoint{0.000000in}{0.000000in}}%
\pgfpathlineto{\pgfqpoint{0.000000in}{-0.048611in}}%
\pgfusepath{stroke,fill}%
}%
\begin{pgfscope}%
\pgfsys@transformshift{4.206008in}{0.553528in}%
\pgfsys@useobject{currentmarker}{}%
\end{pgfscope}%
\end{pgfscope}%
\begin{pgfscope}%
\definecolor{textcolor}{rgb}{0.000000,0.000000,0.000000}%
\pgfsetstrokecolor{textcolor}%
\pgfsetfillcolor{textcolor}%
\pgftext[x=4.206008in,y=0.456305in,,top]{\color{textcolor}\rmfamily\fontsize{8.000000}{9.600000}\selectfont \(\displaystyle {50}\)}%
\end{pgfscope}%
\begin{pgfscope}%
\pgfsetbuttcap%
\pgfsetroundjoin%
\definecolor{currentfill}{rgb}{0.000000,0.000000,0.000000}%
\pgfsetfillcolor{currentfill}%
\pgfsetlinewidth{0.803000pt}%
\definecolor{currentstroke}{rgb}{0.000000,0.000000,0.000000}%
\pgfsetstrokecolor{currentstroke}%
\pgfsetdash{}{0pt}%
\pgfsys@defobject{currentmarker}{\pgfqpoint{0.000000in}{-0.048611in}}{\pgfqpoint{0.000000in}{0.000000in}}{%
\pgfpathmoveto{\pgfqpoint{0.000000in}{0.000000in}}%
\pgfpathlineto{\pgfqpoint{0.000000in}{-0.048611in}}%
\pgfusepath{stroke,fill}%
}%
\begin{pgfscope}%
\pgfsys@transformshift{4.848162in}{0.553528in}%
\pgfsys@useobject{currentmarker}{}%
\end{pgfscope}%
\end{pgfscope}%
\begin{pgfscope}%
\definecolor{textcolor}{rgb}{0.000000,0.000000,0.000000}%
\pgfsetstrokecolor{textcolor}%
\pgfsetfillcolor{textcolor}%
\pgftext[x=4.848162in,y=0.456305in,,top]{\color{textcolor}\rmfamily\fontsize{8.000000}{9.600000}\selectfont \(\displaystyle {100}\)}%
\end{pgfscope}%
\begin{pgfscope}%
\pgfsetbuttcap%
\pgfsetroundjoin%
\definecolor{currentfill}{rgb}{0.000000,0.000000,0.000000}%
\pgfsetfillcolor{currentfill}%
\pgfsetlinewidth{0.803000pt}%
\definecolor{currentstroke}{rgb}{0.000000,0.000000,0.000000}%
\pgfsetstrokecolor{currentstroke}%
\pgfsetdash{}{0pt}%
\pgfsys@defobject{currentmarker}{\pgfqpoint{0.000000in}{-0.048611in}}{\pgfqpoint{0.000000in}{0.000000in}}{%
\pgfpathmoveto{\pgfqpoint{0.000000in}{0.000000in}}%
\pgfpathlineto{\pgfqpoint{0.000000in}{-0.048611in}}%
\pgfusepath{stroke,fill}%
}%
\begin{pgfscope}%
\pgfsys@transformshift{5.490317in}{0.553528in}%
\pgfsys@useobject{currentmarker}{}%
\end{pgfscope}%
\end{pgfscope}%
\begin{pgfscope}%
\definecolor{textcolor}{rgb}{0.000000,0.000000,0.000000}%
\pgfsetstrokecolor{textcolor}%
\pgfsetfillcolor{textcolor}%
\pgftext[x=5.490317in,y=0.456305in,,top]{\color{textcolor}\rmfamily\fontsize{8.000000}{9.600000}\selectfont \(\displaystyle {150}\)}%
\end{pgfscope}%
\begin{pgfscope}%
\definecolor{textcolor}{rgb}{0.000000,0.000000,0.000000}%
\pgfsetstrokecolor{textcolor}%
\pgfsetfillcolor{textcolor}%
\pgftext[x=4.719726in,y=0.302083in,,top]{\color{textcolor}\rmfamily\fontsize{10.950000}{13.140000}\selectfont \(\displaystyle  \phi_{\textup{true}} - \phi_{\textup{reco}} \, [\textup{rad}] \)}%
\end{pgfscope}%
\begin{pgfscope}%
\pgfsetbuttcap%
\pgfsetroundjoin%
\definecolor{currentfill}{rgb}{0.000000,0.000000,0.000000}%
\pgfsetfillcolor{currentfill}%
\pgfsetlinewidth{0.803000pt}%
\definecolor{currentstroke}{rgb}{0.000000,0.000000,0.000000}%
\pgfsetstrokecolor{currentstroke}%
\pgfsetdash{}{0pt}%
\pgfsys@defobject{currentmarker}{\pgfqpoint{-0.048611in}{0.000000in}}{\pgfqpoint{-0.000000in}{0.000000in}}{%
\pgfpathmoveto{\pgfqpoint{-0.000000in}{0.000000in}}%
\pgfpathlineto{\pgfqpoint{-0.048611in}{0.000000in}}%
\pgfusepath{stroke,fill}%
}%
\begin{pgfscope}%
\pgfsys@transformshift{3.448267in}{0.553528in}%
\pgfsys@useobject{currentmarker}{}%
\end{pgfscope}%
\end{pgfscope}%
\begin{pgfscope}%
\pgfsetbuttcap%
\pgfsetroundjoin%
\definecolor{currentfill}{rgb}{0.000000,0.000000,0.000000}%
\pgfsetfillcolor{currentfill}%
\pgfsetlinewidth{0.803000pt}%
\definecolor{currentstroke}{rgb}{0.000000,0.000000,0.000000}%
\pgfsetstrokecolor{currentstroke}%
\pgfsetdash{}{0pt}%
\pgfsys@defobject{currentmarker}{\pgfqpoint{-0.048611in}{0.000000in}}{\pgfqpoint{-0.000000in}{0.000000in}}{%
\pgfpathmoveto{\pgfqpoint{-0.000000in}{0.000000in}}%
\pgfpathlineto{\pgfqpoint{-0.048611in}{0.000000in}}%
\pgfusepath{stroke,fill}%
}%
\begin{pgfscope}%
\pgfsys@transformshift{3.448267in}{0.815849in}%
\pgfsys@useobject{currentmarker}{}%
\end{pgfscope}%
\end{pgfscope}%
\begin{pgfscope}%
\pgfsetbuttcap%
\pgfsetroundjoin%
\definecolor{currentfill}{rgb}{0.000000,0.000000,0.000000}%
\pgfsetfillcolor{currentfill}%
\pgfsetlinewidth{0.803000pt}%
\definecolor{currentstroke}{rgb}{0.000000,0.000000,0.000000}%
\pgfsetstrokecolor{currentstroke}%
\pgfsetdash{}{0pt}%
\pgfsys@defobject{currentmarker}{\pgfqpoint{-0.048611in}{0.000000in}}{\pgfqpoint{-0.000000in}{0.000000in}}{%
\pgfpathmoveto{\pgfqpoint{-0.000000in}{0.000000in}}%
\pgfpathlineto{\pgfqpoint{-0.048611in}{0.000000in}}%
\pgfusepath{stroke,fill}%
}%
\begin{pgfscope}%
\pgfsys@transformshift{3.448267in}{1.078171in}%
\pgfsys@useobject{currentmarker}{}%
\end{pgfscope}%
\end{pgfscope}%
\begin{pgfscope}%
\pgfsetbuttcap%
\pgfsetroundjoin%
\definecolor{currentfill}{rgb}{0.000000,0.000000,0.000000}%
\pgfsetfillcolor{currentfill}%
\pgfsetlinewidth{0.803000pt}%
\definecolor{currentstroke}{rgb}{0.000000,0.000000,0.000000}%
\pgfsetstrokecolor{currentstroke}%
\pgfsetdash{}{0pt}%
\pgfsys@defobject{currentmarker}{\pgfqpoint{-0.048611in}{0.000000in}}{\pgfqpoint{-0.000000in}{0.000000in}}{%
\pgfpathmoveto{\pgfqpoint{-0.000000in}{0.000000in}}%
\pgfpathlineto{\pgfqpoint{-0.048611in}{0.000000in}}%
\pgfusepath{stroke,fill}%
}%
\begin{pgfscope}%
\pgfsys@transformshift{3.448267in}{1.340493in}%
\pgfsys@useobject{currentmarker}{}%
\end{pgfscope}%
\end{pgfscope}%
\begin{pgfscope}%
\pgfsetbuttcap%
\pgfsetroundjoin%
\definecolor{currentfill}{rgb}{0.000000,0.000000,0.000000}%
\pgfsetfillcolor{currentfill}%
\pgfsetlinewidth{0.803000pt}%
\definecolor{currentstroke}{rgb}{0.000000,0.000000,0.000000}%
\pgfsetstrokecolor{currentstroke}%
\pgfsetdash{}{0pt}%
\pgfsys@defobject{currentmarker}{\pgfqpoint{-0.048611in}{0.000000in}}{\pgfqpoint{-0.000000in}{0.000000in}}{%
\pgfpathmoveto{\pgfqpoint{-0.000000in}{0.000000in}}%
\pgfpathlineto{\pgfqpoint{-0.048611in}{0.000000in}}%
\pgfusepath{stroke,fill}%
}%
\begin{pgfscope}%
\pgfsys@transformshift{3.448267in}{1.602814in}%
\pgfsys@useobject{currentmarker}{}%
\end{pgfscope}%
\end{pgfscope}%
\begin{pgfscope}%
\pgfsetbuttcap%
\pgfsetroundjoin%
\definecolor{currentfill}{rgb}{0.000000,0.000000,0.000000}%
\pgfsetfillcolor{currentfill}%
\pgfsetlinewidth{0.803000pt}%
\definecolor{currentstroke}{rgb}{0.000000,0.000000,0.000000}%
\pgfsetstrokecolor{currentstroke}%
\pgfsetdash{}{0pt}%
\pgfsys@defobject{currentmarker}{\pgfqpoint{-0.048611in}{0.000000in}}{\pgfqpoint{-0.000000in}{0.000000in}}{%
\pgfpathmoveto{\pgfqpoint{-0.000000in}{0.000000in}}%
\pgfpathlineto{\pgfqpoint{-0.048611in}{0.000000in}}%
\pgfusepath{stroke,fill}%
}%
\begin{pgfscope}%
\pgfsys@transformshift{3.448267in}{1.865136in}%
\pgfsys@useobject{currentmarker}{}%
\end{pgfscope}%
\end{pgfscope}%
\begin{pgfscope}%
\pgfpathrectangle{\pgfqpoint{3.448267in}{0.553528in}}{\pgfqpoint{2.542920in}{1.553361in}}%
\pgfusepath{clip}%
\pgfsetbuttcap%
\pgfsetmiterjoin%
\pgfsetlinewidth{1.003750pt}%
\definecolor{currentstroke}{rgb}{0.313725,0.317647,0.309804}%
\pgfsetstrokecolor{currentstroke}%
\pgfsetdash{}{0pt}%
\pgfpathmoveto{\pgfqpoint{3.563855in}{0.553528in}}%
\pgfpathlineto{\pgfqpoint{3.563855in}{2.032919in}}%
\pgfpathlineto{\pgfqpoint{3.583446in}{2.032919in}}%
\pgfpathlineto{\pgfqpoint{3.583446in}{2.031226in}}%
\pgfpathlineto{\pgfqpoint{3.603037in}{2.031226in}}%
\pgfpathlineto{\pgfqpoint{3.603037in}{1.980298in}}%
\pgfpathlineto{\pgfqpoint{3.622628in}{1.980298in}}%
\pgfpathlineto{\pgfqpoint{3.622628in}{1.931193in}}%
\pgfpathlineto{\pgfqpoint{3.642219in}{1.931193in}}%
\pgfpathlineto{\pgfqpoint{3.642219in}{1.866589in}}%
\pgfpathlineto{\pgfqpoint{3.661810in}{1.866589in}}%
\pgfpathlineto{\pgfqpoint{3.661810in}{1.846921in}}%
\pgfpathlineto{\pgfqpoint{3.681401in}{1.846921in}}%
\pgfpathlineto{\pgfqpoint{3.681401in}{1.750535in}}%
\pgfpathlineto{\pgfqpoint{3.700992in}{1.750535in}}%
\pgfpathlineto{\pgfqpoint{3.700992in}{1.672905in}}%
\pgfpathlineto{\pgfqpoint{3.720583in}{1.672905in}}%
\pgfpathlineto{\pgfqpoint{3.720583in}{1.627187in}}%
\pgfpathlineto{\pgfqpoint{3.740174in}{1.627187in}}%
\pgfpathlineto{\pgfqpoint{3.740174in}{1.542003in}}%
\pgfpathlineto{\pgfqpoint{3.759765in}{1.542003in}}%
\pgfpathlineto{\pgfqpoint{3.759765in}{1.489512in}}%
\pgfpathlineto{\pgfqpoint{3.779356in}{1.489512in}}%
\pgfpathlineto{\pgfqpoint{3.779356in}{1.413445in}}%
\pgfpathlineto{\pgfqpoint{3.798947in}{1.413445in}}%
\pgfpathlineto{\pgfqpoint{3.798947in}{1.350404in}}%
\pgfpathlineto{\pgfqpoint{3.818538in}{1.350404in}}%
\pgfpathlineto{\pgfqpoint{3.818538in}{1.315496in}}%
\pgfpathlineto{\pgfqpoint{3.838129in}{1.315496in}}%
\pgfpathlineto{\pgfqpoint{3.838129in}{1.250762in}}%
\pgfpathlineto{\pgfqpoint{3.857720in}{1.250762in}}%
\pgfpathlineto{\pgfqpoint{3.857720in}{1.211165in}}%
\pgfpathlineto{\pgfqpoint{3.877311in}{1.211165in}}%
\pgfpathlineto{\pgfqpoint{3.877311in}{1.165968in}}%
\pgfpathlineto{\pgfqpoint{3.896903in}{1.165968in}}%
\pgfpathlineto{\pgfqpoint{3.896903in}{1.131191in}}%
\pgfpathlineto{\pgfqpoint{3.916494in}{1.131191in}}%
\pgfpathlineto{\pgfqpoint{3.916494in}{1.088599in}}%
\pgfpathlineto{\pgfqpoint{3.936085in}{1.088599in}}%
\pgfpathlineto{\pgfqpoint{3.936085in}{1.069061in}}%
\pgfpathlineto{\pgfqpoint{3.955676in}{1.069061in}}%
\pgfpathlineto{\pgfqpoint{3.955676in}{1.038062in}}%
\pgfpathlineto{\pgfqpoint{3.975267in}{1.038062in}}%
\pgfpathlineto{\pgfqpoint{3.975267in}{1.011881in}}%
\pgfpathlineto{\pgfqpoint{3.994858in}{1.011881in}}%
\pgfpathlineto{\pgfqpoint{3.994858in}{0.981793in}}%
\pgfpathlineto{\pgfqpoint{4.014449in}{0.981793in}}%
\pgfpathlineto{\pgfqpoint{4.014449in}{0.973066in}}%
\pgfpathlineto{\pgfqpoint{4.034040in}{0.973066in}}%
\pgfpathlineto{\pgfqpoint{4.034040in}{0.933991in}}%
\pgfpathlineto{\pgfqpoint{4.053631in}{0.933991in}}%
\pgfpathlineto{\pgfqpoint{4.053631in}{0.916147in}}%
\pgfpathlineto{\pgfqpoint{4.073222in}{0.916147in}}%
\pgfpathlineto{\pgfqpoint{4.073222in}{0.911458in}}%
\pgfpathlineto{\pgfqpoint{4.092813in}{0.911458in}}%
\pgfpathlineto{\pgfqpoint{4.092813in}{0.891008in}}%
\pgfpathlineto{\pgfqpoint{4.112404in}{0.891008in}}%
\pgfpathlineto{\pgfqpoint{4.112404in}{0.875378in}}%
\pgfpathlineto{\pgfqpoint{4.131995in}{0.875378in}}%
\pgfpathlineto{\pgfqpoint{4.131995in}{0.862744in}}%
\pgfpathlineto{\pgfqpoint{4.151586in}{0.862744in}}%
\pgfpathlineto{\pgfqpoint{4.151586in}{0.856492in}}%
\pgfpathlineto{\pgfqpoint{4.171177in}{0.856492in}}%
\pgfpathlineto{\pgfqpoint{4.171177in}{0.845941in}}%
\pgfpathlineto{\pgfqpoint{4.190768in}{0.845941in}}%
\pgfpathlineto{\pgfqpoint{4.190768in}{0.833307in}}%
\pgfpathlineto{\pgfqpoint{4.210359in}{0.833307in}}%
\pgfpathlineto{\pgfqpoint{4.210359in}{0.817156in}}%
\pgfpathlineto{\pgfqpoint{4.229950in}{0.817156in}}%
\pgfpathlineto{\pgfqpoint{4.229950in}{0.813248in}}%
\pgfpathlineto{\pgfqpoint{4.249542in}{0.813248in}}%
\pgfpathlineto{\pgfqpoint{4.249542in}{0.800874in}}%
\pgfpathlineto{\pgfqpoint{4.269133in}{0.800874in}}%
\pgfpathlineto{\pgfqpoint{4.269133in}{0.789673in}}%
\pgfpathlineto{\pgfqpoint{4.308315in}{0.790064in}}%
\pgfpathlineto{\pgfqpoint{4.308315in}{0.778471in}}%
\pgfpathlineto{\pgfqpoint{4.327906in}{0.778471in}}%
\pgfpathlineto{\pgfqpoint{4.327906in}{0.764665in}}%
\pgfpathlineto{\pgfqpoint{4.347497in}{0.764665in}}%
\pgfpathlineto{\pgfqpoint{4.347497in}{0.761539in}}%
\pgfpathlineto{\pgfqpoint{4.367088in}{0.761539in}}%
\pgfpathlineto{\pgfqpoint{4.367088in}{0.756068in}}%
\pgfpathlineto{\pgfqpoint{4.386679in}{0.756068in}}%
\pgfpathlineto{\pgfqpoint{4.386679in}{0.748123in}}%
\pgfpathlineto{\pgfqpoint{4.406270in}{0.748123in}}%
\pgfpathlineto{\pgfqpoint{4.406270in}{0.745518in}}%
\pgfpathlineto{\pgfqpoint{4.425861in}{0.745518in}}%
\pgfpathlineto{\pgfqpoint{4.425861in}{0.732232in}}%
\pgfpathlineto{\pgfqpoint{4.445452in}{0.732232in}}%
\pgfpathlineto{\pgfqpoint{4.445452in}{0.735749in}}%
\pgfpathlineto{\pgfqpoint{4.465043in}{0.735749in}}%
\pgfpathlineto{\pgfqpoint{4.465043in}{0.732623in}}%
\pgfpathlineto{\pgfqpoint{4.504225in}{0.731972in}}%
\pgfpathlineto{\pgfqpoint{4.504225in}{0.722594in}}%
\pgfpathlineto{\pgfqpoint{4.543407in}{0.722463in}}%
\pgfpathlineto{\pgfqpoint{4.543407in}{0.706703in}}%
\pgfpathlineto{\pgfqpoint{4.562998in}{0.706703in}}%
\pgfpathlineto{\pgfqpoint{4.562998in}{0.702405in}}%
\pgfpathlineto{\pgfqpoint{4.602180in}{0.702014in}}%
\pgfpathlineto{\pgfqpoint{4.602180in}{0.709047in}}%
\pgfpathlineto{\pgfqpoint{4.621772in}{0.709047in}}%
\pgfpathlineto{\pgfqpoint{4.621772in}{0.697325in}}%
\pgfpathlineto{\pgfqpoint{4.641363in}{0.697325in}}%
\pgfpathlineto{\pgfqpoint{4.641363in}{0.688728in}}%
\pgfpathlineto{\pgfqpoint{4.660954in}{0.688728in}}%
\pgfpathlineto{\pgfqpoint{4.660954in}{0.696934in}}%
\pgfpathlineto{\pgfqpoint{4.680545in}{0.696934in}}%
\pgfpathlineto{\pgfqpoint{4.680545in}{0.688077in}}%
\pgfpathlineto{\pgfqpoint{4.700136in}{0.688077in}}%
\pgfpathlineto{\pgfqpoint{4.700136in}{0.678829in}}%
\pgfpathlineto{\pgfqpoint{4.719727in}{0.678829in}}%
\pgfpathlineto{\pgfqpoint{4.719727in}{0.684951in}}%
\pgfpathlineto{\pgfqpoint{4.739318in}{0.684951in}}%
\pgfpathlineto{\pgfqpoint{4.739318in}{0.680653in}}%
\pgfpathlineto{\pgfqpoint{4.758909in}{0.680653in}}%
\pgfpathlineto{\pgfqpoint{4.758909in}{0.675703in}}%
\pgfpathlineto{\pgfqpoint{4.778500in}{0.675703in}}%
\pgfpathlineto{\pgfqpoint{4.778500in}{0.670623in}}%
\pgfpathlineto{\pgfqpoint{4.798091in}{0.670623in}}%
\pgfpathlineto{\pgfqpoint{4.798091in}{0.666976in}}%
\pgfpathlineto{\pgfqpoint{4.817682in}{0.666976in}}%
\pgfpathlineto{\pgfqpoint{4.817682in}{0.671014in}}%
\pgfpathlineto{\pgfqpoint{4.837273in}{0.671014in}}%
\pgfpathlineto{\pgfqpoint{4.837273in}{0.666586in}}%
\pgfpathlineto{\pgfqpoint{4.876455in}{0.666716in}}%
\pgfpathlineto{\pgfqpoint{4.876455in}{0.659031in}}%
\pgfpathlineto{\pgfqpoint{4.896046in}{0.659031in}}%
\pgfpathlineto{\pgfqpoint{4.896046in}{0.660334in}}%
\pgfpathlineto{\pgfqpoint{4.915637in}{0.660334in}}%
\pgfpathlineto{\pgfqpoint{4.915637in}{0.656296in}}%
\pgfpathlineto{\pgfqpoint{4.954819in}{0.656296in}}%
\pgfpathlineto{\pgfqpoint{4.954819in}{0.654733in}}%
\pgfpathlineto{\pgfqpoint{4.974411in}{0.654733in}}%
\pgfpathlineto{\pgfqpoint{4.974411in}{0.649262in}}%
\pgfpathlineto{\pgfqpoint{4.994002in}{0.649262in}}%
\pgfpathlineto{\pgfqpoint{4.994002in}{0.656426in}}%
\pgfpathlineto{\pgfqpoint{5.013593in}{0.656426in}}%
\pgfpathlineto{\pgfqpoint{5.013593in}{0.645876in}}%
\pgfpathlineto{\pgfqpoint{5.052775in}{0.645876in}}%
\pgfpathlineto{\pgfqpoint{5.052775in}{0.649913in}}%
\pgfpathlineto{\pgfqpoint{5.072366in}{0.649913in}}%
\pgfpathlineto{\pgfqpoint{5.072366in}{0.640666in}}%
\pgfpathlineto{\pgfqpoint{5.091957in}{0.640666in}}%
\pgfpathlineto{\pgfqpoint{5.091957in}{0.644703in}}%
\pgfpathlineto{\pgfqpoint{5.111548in}{0.644703in}}%
\pgfpathlineto{\pgfqpoint{5.111548in}{0.641577in}}%
\pgfpathlineto{\pgfqpoint{5.131139in}{0.641577in}}%
\pgfpathlineto{\pgfqpoint{5.131139in}{0.647048in}}%
\pgfpathlineto{\pgfqpoint{5.150730in}{0.647048in}}%
\pgfpathlineto{\pgfqpoint{5.150730in}{0.648220in}}%
\pgfpathlineto{\pgfqpoint{5.170321in}{0.648220in}}%
\pgfpathlineto{\pgfqpoint{5.170321in}{0.644703in}}%
\pgfpathlineto{\pgfqpoint{5.189912in}{0.644703in}}%
\pgfpathlineto{\pgfqpoint{5.189912in}{0.637019in}}%
\pgfpathlineto{\pgfqpoint{5.229094in}{0.635977in}}%
\pgfpathlineto{\pgfqpoint{5.229094in}{0.634153in}}%
\pgfpathlineto{\pgfqpoint{5.248685in}{0.634153in}}%
\pgfpathlineto{\pgfqpoint{5.248685in}{0.632590in}}%
\pgfpathlineto{\pgfqpoint{5.268276in}{0.632590in}}%
\pgfpathlineto{\pgfqpoint{5.268276in}{0.635846in}}%
\pgfpathlineto{\pgfqpoint{5.287867in}{0.635846in}}%
\pgfpathlineto{\pgfqpoint{5.287867in}{0.639754in}}%
\pgfpathlineto{\pgfqpoint{5.307458in}{0.639754in}}%
\pgfpathlineto{\pgfqpoint{5.307458in}{0.638321in}}%
\pgfpathlineto{\pgfqpoint{5.327049in}{0.638321in}}%
\pgfpathlineto{\pgfqpoint{5.327049in}{0.631548in}}%
\pgfpathlineto{\pgfqpoint{5.346641in}{0.631548in}}%
\pgfpathlineto{\pgfqpoint{5.346641in}{0.629073in}}%
\pgfpathlineto{\pgfqpoint{5.366232in}{0.629073in}}%
\pgfpathlineto{\pgfqpoint{5.366232in}{0.635325in}}%
\pgfpathlineto{\pgfqpoint{5.385823in}{0.635325in}}%
\pgfpathlineto{\pgfqpoint{5.385823in}{0.633241in}}%
\pgfpathlineto{\pgfqpoint{5.405414in}{0.633241in}}%
\pgfpathlineto{\pgfqpoint{5.405414in}{0.625947in}}%
\pgfpathlineto{\pgfqpoint{5.425005in}{0.625947in}}%
\pgfpathlineto{\pgfqpoint{5.425005in}{0.633241in}}%
\pgfpathlineto{\pgfqpoint{5.444596in}{0.633241in}}%
\pgfpathlineto{\pgfqpoint{5.444596in}{0.622951in}}%
\pgfpathlineto{\pgfqpoint{5.464187in}{0.622951in}}%
\pgfpathlineto{\pgfqpoint{5.464187in}{0.633502in}}%
\pgfpathlineto{\pgfqpoint{5.483778in}{0.633502in}}%
\pgfpathlineto{\pgfqpoint{5.483778in}{0.631027in}}%
\pgfpathlineto{\pgfqpoint{5.503369in}{0.631027in}}%
\pgfpathlineto{\pgfqpoint{5.503369in}{0.633893in}}%
\pgfpathlineto{\pgfqpoint{5.522960in}{0.633893in}}%
\pgfpathlineto{\pgfqpoint{5.522960in}{0.627119in}}%
\pgfpathlineto{\pgfqpoint{5.542551in}{0.627119in}}%
\pgfpathlineto{\pgfqpoint{5.542551in}{0.625296in}}%
\pgfpathlineto{\pgfqpoint{5.562142in}{0.625296in}}%
\pgfpathlineto{\pgfqpoint{5.562142in}{0.621128in}}%
\pgfpathlineto{\pgfqpoint{5.581733in}{0.621128in}}%
\pgfpathlineto{\pgfqpoint{5.581733in}{0.624905in}}%
\pgfpathlineto{\pgfqpoint{5.601324in}{0.624905in}}%
\pgfpathlineto{\pgfqpoint{5.601324in}{0.620607in}}%
\pgfpathlineto{\pgfqpoint{5.620915in}{0.620607in}}%
\pgfpathlineto{\pgfqpoint{5.620915in}{0.632199in}}%
\pgfpathlineto{\pgfqpoint{5.640506in}{0.632199in}}%
\pgfpathlineto{\pgfqpoint{5.640506in}{0.625687in}}%
\pgfpathlineto{\pgfqpoint{5.660097in}{0.625687in}}%
\pgfpathlineto{\pgfqpoint{5.660097in}{0.631939in}}%
\pgfpathlineto{\pgfqpoint{5.679688in}{0.631939in}}%
\pgfpathlineto{\pgfqpoint{5.679688in}{0.627901in}}%
\pgfpathlineto{\pgfqpoint{5.699280in}{0.627901in}}%
\pgfpathlineto{\pgfqpoint{5.699280in}{0.620607in}}%
\pgfpathlineto{\pgfqpoint{5.718871in}{0.620607in}}%
\pgfpathlineto{\pgfqpoint{5.718871in}{0.623342in}}%
\pgfpathlineto{\pgfqpoint{5.738462in}{0.623342in}}%
\pgfpathlineto{\pgfqpoint{5.738462in}{0.628422in}}%
\pgfpathlineto{\pgfqpoint{5.758053in}{0.628422in}}%
\pgfpathlineto{\pgfqpoint{5.758053in}{0.621128in}}%
\pgfpathlineto{\pgfqpoint{5.777644in}{0.621128in}}%
\pgfpathlineto{\pgfqpoint{5.777644in}{0.625947in}}%
\pgfpathlineto{\pgfqpoint{5.797235in}{0.625947in}}%
\pgfpathlineto{\pgfqpoint{5.797235in}{0.622040in}}%
\pgfpathlineto{\pgfqpoint{5.816826in}{0.622040in}}%
\pgfpathlineto{\pgfqpoint{5.816826in}{0.620346in}}%
\pgfpathlineto{\pgfqpoint{5.836417in}{0.620346in}}%
\pgfpathlineto{\pgfqpoint{5.836417in}{0.626338in}}%
\pgfpathlineto{\pgfqpoint{5.875599in}{0.625426in}}%
\pgfpathlineto{\pgfqpoint{5.875599in}{0.553528in}}%
\pgfpathlineto{\pgfqpoint{5.875599in}{0.553528in}}%
\pgfusepath{stroke}%
\end{pgfscope}%
\begin{pgfscope}%
\pgfsetrectcap%
\pgfsetmiterjoin%
\pgfsetlinewidth{0.803000pt}%
\definecolor{currentstroke}{rgb}{0.000000,0.000000,0.000000}%
\pgfsetstrokecolor{currentstroke}%
\pgfsetdash{}{0pt}%
\pgfpathmoveto{\pgfqpoint{3.448267in}{0.553528in}}%
\pgfpathlineto{\pgfqpoint{3.448267in}{2.106889in}}%
\pgfusepath{stroke}%
\end{pgfscope}%
\begin{pgfscope}%
\pgfsetrectcap%
\pgfsetmiterjoin%
\pgfsetlinewidth{0.803000pt}%
\definecolor{currentstroke}{rgb}{0.000000,0.000000,0.000000}%
\pgfsetstrokecolor{currentstroke}%
\pgfsetdash{}{0pt}%
\pgfpathmoveto{\pgfqpoint{5.991186in}{0.553528in}}%
\pgfpathlineto{\pgfqpoint{5.991186in}{2.106889in}}%
\pgfusepath{stroke}%
\end{pgfscope}%
\begin{pgfscope}%
\pgfsetrectcap%
\pgfsetmiterjoin%
\pgfsetlinewidth{0.803000pt}%
\definecolor{currentstroke}{rgb}{0.000000,0.000000,0.000000}%
\pgfsetstrokecolor{currentstroke}%
\pgfsetdash{}{0pt}%
\pgfpathmoveto{\pgfqpoint{3.448267in}{0.553528in}}%
\pgfpathlineto{\pgfqpoint{5.991186in}{0.553528in}}%
\pgfusepath{stroke}%
\end{pgfscope}%
\begin{pgfscope}%
\pgfsetrectcap%
\pgfsetmiterjoin%
\pgfsetlinewidth{0.803000pt}%
\definecolor{currentstroke}{rgb}{0.000000,0.000000,0.000000}%
\pgfsetstrokecolor{currentstroke}%
\pgfsetdash{}{0pt}%
\pgfpathmoveto{\pgfqpoint{3.448267in}{2.106889in}}%
\pgfpathlineto{\pgfqpoint{5.991186in}{2.106889in}}%
\pgfusepath{stroke}%
\end{pgfscope}%
\begin{pgfscope}%
\definecolor{textcolor}{rgb}{0.000000,0.000000,0.000000}%
\pgfsetstrokecolor{textcolor}%
\pgfsetfillcolor{textcolor}%
\pgftext[x=3.692285in,y=0.658456in,left,]{\color{textcolor}\rmfamily\fontsize{10.000000}{12.000000}\selectfont 68th pctl}%
\end{pgfscope}%
\begin{pgfscope}%
\definecolor{textcolor}{rgb}{0.000000,0.000000,0.000000}%
\pgfsetstrokecolor{textcolor}%
\pgfsetfillcolor{textcolor}%
\pgftext[x=3.448267in,y=2.190222in,left,base]{\color{textcolor}\ttfamily\fontsize{10.000000}{12.000000}\selectfont Bin [1.7, 1.8), 264,055 events}%
\end{pgfscope}%
\begin{pgfscope}%
\definecolor{textcolor}{rgb}{0.000000,0.000000,0.000000}%
\pgfsetstrokecolor{textcolor}%
\pgfsetfillcolor{textcolor}%
\pgftext[x=0.620120in,y=4.900000in,left,top]{\color{textcolor}\rmfamily\fontsize{12.000000}{14.400000}\selectfont Run 2011300821 angular error distribution in selected energy bins}%
\end{pgfscope}%
\end{pgfpicture}%
\makeatother%
\endgroup%

    \caption{Error resolution in energy bins.
    Top row shows the error distribution in energy bins [1.0, 1.2) (left) and [1.7, 1.8) }\label{fig:error_resolution_1}
\end{figure}

To define the performance of an algorithm the resolution is chosen.
The resolution is here defined as the normalized interquartile range of the error distribution in a certain energy bin, i.e.
\begin{equation}
    w \approx \frac{\text{IQR}}{1.349},
\end{equation}
an example of which is shown in the top row of~\vref{fig:error_resolution_1}.
The normalization is chosen because for a normal distribution
\begin{equation}
    \text{IQR} \approx 1.349 \sigma,
\end{equation}
meaning that for a normal distribution \( w \) would be \( \sigma \).

The error distribution is different for different metrics, but represents how well the algorithm reconstructs a neutrino; it is zero for perfect reconstruction.

For error distributions that are bounded, the 68th percentile is used instead, also shown in~\vref{fig:error_resolution_1}.

To give some sort of error estimate on \( w \) bootstrapping is employed with the BCa method~\cite{Efron1987} to give a \SI{90}{\percent} confidence interval.
The IQR reported by energy bin and its confidence interval can be seen in e.g.~\vref{fig:powershovel_4}.

\section{Algorithm}\label{sec:algorithm}

The best algorithm is chosen by essentially running experiments of different neural network designs.
Early experiments used a simple 1D CNN, which slides a kernel across the data in one direction (hence the name), followed by fully connected layers.
Later experiments employed TCNs, as these were seen to much improve the performance with a similar amount of parameters and similar performance with much fewer parameters.

Because the inner workings of a neural network is somewhat opaque, it can be difficult to reason about the design with supreme confidence, and so for new tasks\footnote{Staples of machine learning, such as the MNIST task, have years of experimentation behind it; in that sense it is an old task, contrary to the one in this thesis} experimenting is the only way to reason about the most effective design.

\subsection{Hyperparameters}

There is also the case of hyperparameters.
Not only must one choose an appropriate loss function, general values such as the learning rate need also be tuned, and even the shape of the scheduler that controls the change of learning rate over time.
Furthermore problem specific parameters may arise, such as the maximum length of an event in the case of IceCube reconstruction, necessary because tensors going into the neural network must have the same shape, meaning zero-padding to some common value is required\footnote{Even the nature of the zero-padding must be decided; do you zero-pad the end of the arrays, or maybe both ends (with the detector data in the middle)?}.

The data itself also comes in different flavors, in a sense; one may choose the collection of pulses in an event where SRT cleaning (used in level 2 cleaning, but can be implemented at any stage; see~\vref{sec:l2}) has been applied, which removes light that is not causally connected to the event, and must thus be considered noise.

Experiments were run on smaller sets (around \num{200000} events) to facilitate quick turnaround under the assumption that adding more data will improve performance incrementally.
The choice of learning rate was done via a learning rate finder, or \enquote{the Learning Rate Range Test} (LRRT)~\cite{Smith2017}, an attempt at automating the process.

\begin{figure}
    \centering
    \includegraphics[width=1.0\textwidth]{./images/design/lr_finder.png}
    \caption{An example of an LRRT run.
    At first, the learning rate is too low, and the loss is plateauing.
    In the optimal range, the loss descends, until the loss becomes unstable and eventually explodes.
    Figure from~\cite{LearningRateScan}.}\label{fig:lrrt}
\end{figure}

LRRT runs a learning rate scan---between two values chosen manually by the user---over one epoch, increasing the learning rate between each mini-batch, the training loss is recorded at each step.
When the rate is too low, the training loss will plateau, then descent and finally---when the learning rate is too high---explode; the user should then choose the learning rate to be where the training loss is descendant, as that represents the most optimal range of learning rates.
An example is shown in~\vref{fig:lrrt}.

\subsection{Loss function}

\begin{figure}
    \centering
    %% Creator: Matplotlib, PGF backend
%%
%% To include the figure in your LaTeX document, write
%%   \input{<filename>.pgf}
%%
%% Make sure the required packages are loaded in your preamble
%%   \usepackage{pgf}
%%
%% and, on pdftex
%%   \usepackage[utf8]{inputenc}\DeclareUnicodeCharacter{2212}{-}
%%
%% or, on luatex and xetex
%%   \usepackage{unicode-math}
%%
%% Figures using additional raster images can only be included by \input if
%% they are in the same directory as the main LaTeX file. For loading figures
%% from other directories you can use the `import` package
%%   \usepackage{import}
%%
%% and then include the figures with
%%   \import{<path to file>}{<filename>.pgf}
%%
%% Matplotlib used the following preamble
%%   \usepackage{siunitx} \usepackage{amsmath} \usepackage{bm}
%%   \usepackage{fontspec}
%%
\begingroup%
\makeatletter%
\begin{pgfpicture}%
\pgfpathrectangle{\pgfpointorigin}{\pgfqpoint{6.201200in}{3.000000in}}%
\pgfusepath{use as bounding box, clip}%
\begin{pgfscope}%
\pgfsetbuttcap%
\pgfsetmiterjoin%
\definecolor{currentfill}{rgb}{1.000000,1.000000,1.000000}%
\pgfsetfillcolor{currentfill}%
\pgfsetlinewidth{0.000000pt}%
\definecolor{currentstroke}{rgb}{1.000000,1.000000,1.000000}%
\pgfsetstrokecolor{currentstroke}%
\pgfsetdash{}{0pt}%
\pgfpathmoveto{\pgfqpoint{0.000000in}{0.000000in}}%
\pgfpathlineto{\pgfqpoint{6.201200in}{0.000000in}}%
\pgfpathlineto{\pgfqpoint{6.201200in}{3.000000in}}%
\pgfpathlineto{\pgfqpoint{0.000000in}{3.000000in}}%
\pgfpathclose%
\pgfusepath{fill}%
\end{pgfscope}%
\begin{pgfscope}%
\pgfsetbuttcap%
\pgfsetmiterjoin%
\definecolor{currentfill}{rgb}{1.000000,1.000000,1.000000}%
\pgfsetfillcolor{currentfill}%
\pgfsetlinewidth{0.000000pt}%
\definecolor{currentstroke}{rgb}{0.000000,0.000000,0.000000}%
\pgfsetstrokecolor{currentstroke}%
\pgfsetstrokeopacity{0.000000}%
\pgfsetdash{}{0pt}%
\pgfpathmoveto{\pgfqpoint{0.572918in}{0.553781in}}%
\pgfpathlineto{\pgfqpoint{6.051200in}{0.553781in}}%
\pgfpathlineto{\pgfqpoint{6.051200in}{2.649333in}}%
\pgfpathlineto{\pgfqpoint{0.572918in}{2.649333in}}%
\pgfpathclose%
\pgfusepath{fill}%
\end{pgfscope}%
\begin{pgfscope}%
\pgfpathrectangle{\pgfqpoint{0.572918in}{0.553781in}}{\pgfqpoint{5.478282in}{2.095553in}}%
\pgfusepath{clip}%
\pgfsetbuttcap%
\pgfsetroundjoin%
\pgfsetlinewidth{0.501875pt}%
\definecolor{currentstroke}{rgb}{0.690196,0.690196,0.690196}%
\pgfsetstrokecolor{currentstroke}%
\pgfsetstrokeopacity{0.500000}%
\pgfsetdash{{0.500000pt}{0.825000pt}}{0.000000pt}%
\pgfpathmoveto{\pgfqpoint{0.675453in}{0.553781in}}%
\pgfpathlineto{\pgfqpoint{0.675453in}{2.649333in}}%
\pgfusepath{stroke}%
\end{pgfscope}%
\begin{pgfscope}%
\pgfsetbuttcap%
\pgfsetroundjoin%
\definecolor{currentfill}{rgb}{0.000000,0.000000,0.000000}%
\pgfsetfillcolor{currentfill}%
\pgfsetlinewidth{0.803000pt}%
\definecolor{currentstroke}{rgb}{0.000000,0.000000,0.000000}%
\pgfsetstrokecolor{currentstroke}%
\pgfsetdash{}{0pt}%
\pgfsys@defobject{currentmarker}{\pgfqpoint{0.000000in}{-0.048611in}}{\pgfqpoint{0.000000in}{0.000000in}}{%
\pgfpathmoveto{\pgfqpoint{0.000000in}{0.000000in}}%
\pgfpathlineto{\pgfqpoint{0.000000in}{-0.048611in}}%
\pgfusepath{stroke,fill}%
}%
\begin{pgfscope}%
\pgfsys@transformshift{0.675453in}{0.553781in}%
\pgfsys@useobject{currentmarker}{}%
\end{pgfscope}%
\end{pgfscope}%
\begin{pgfscope}%
\definecolor{textcolor}{rgb}{0.000000,0.000000,0.000000}%
\pgfsetstrokecolor{textcolor}%
\pgfsetfillcolor{textcolor}%
\pgftext[x=0.675453in,y=0.456558in,,top]{\color{textcolor}\rmfamily\fontsize{8.000000}{9.600000}\selectfont \(\displaystyle {0.0}\)}%
\end{pgfscope}%
\begin{pgfscope}%
\pgfpathrectangle{\pgfqpoint{0.572918in}{0.553781in}}{\pgfqpoint{5.478282in}{2.095553in}}%
\pgfusepath{clip}%
\pgfsetbuttcap%
\pgfsetroundjoin%
\pgfsetlinewidth{0.501875pt}%
\definecolor{currentstroke}{rgb}{0.690196,0.690196,0.690196}%
\pgfsetstrokecolor{currentstroke}%
\pgfsetstrokeopacity{0.500000}%
\pgfsetdash{{0.500000pt}{0.825000pt}}{0.000000pt}%
\pgfpathmoveto{\pgfqpoint{1.554322in}{0.553781in}}%
\pgfpathlineto{\pgfqpoint{1.554322in}{2.649333in}}%
\pgfusepath{stroke}%
\end{pgfscope}%
\begin{pgfscope}%
\pgfsetbuttcap%
\pgfsetroundjoin%
\definecolor{currentfill}{rgb}{0.000000,0.000000,0.000000}%
\pgfsetfillcolor{currentfill}%
\pgfsetlinewidth{0.803000pt}%
\definecolor{currentstroke}{rgb}{0.000000,0.000000,0.000000}%
\pgfsetstrokecolor{currentstroke}%
\pgfsetdash{}{0pt}%
\pgfsys@defobject{currentmarker}{\pgfqpoint{0.000000in}{-0.048611in}}{\pgfqpoint{0.000000in}{0.000000in}}{%
\pgfpathmoveto{\pgfqpoint{0.000000in}{0.000000in}}%
\pgfpathlineto{\pgfqpoint{0.000000in}{-0.048611in}}%
\pgfusepath{stroke,fill}%
}%
\begin{pgfscope}%
\pgfsys@transformshift{1.554322in}{0.553781in}%
\pgfsys@useobject{currentmarker}{}%
\end{pgfscope}%
\end{pgfscope}%
\begin{pgfscope}%
\definecolor{textcolor}{rgb}{0.000000,0.000000,0.000000}%
\pgfsetstrokecolor{textcolor}%
\pgfsetfillcolor{textcolor}%
\pgftext[x=1.554322in,y=0.456558in,,top]{\color{textcolor}\rmfamily\fontsize{8.000000}{9.600000}\selectfont \(\displaystyle {0.5}\)}%
\end{pgfscope}%
\begin{pgfscope}%
\pgfpathrectangle{\pgfqpoint{0.572918in}{0.553781in}}{\pgfqpoint{5.478282in}{2.095553in}}%
\pgfusepath{clip}%
\pgfsetbuttcap%
\pgfsetroundjoin%
\pgfsetlinewidth{0.501875pt}%
\definecolor{currentstroke}{rgb}{0.690196,0.690196,0.690196}%
\pgfsetstrokecolor{currentstroke}%
\pgfsetstrokeopacity{0.500000}%
\pgfsetdash{{0.500000pt}{0.825000pt}}{0.000000pt}%
\pgfpathmoveto{\pgfqpoint{2.433190in}{0.553781in}}%
\pgfpathlineto{\pgfqpoint{2.433190in}{2.649333in}}%
\pgfusepath{stroke}%
\end{pgfscope}%
\begin{pgfscope}%
\pgfsetbuttcap%
\pgfsetroundjoin%
\definecolor{currentfill}{rgb}{0.000000,0.000000,0.000000}%
\pgfsetfillcolor{currentfill}%
\pgfsetlinewidth{0.803000pt}%
\definecolor{currentstroke}{rgb}{0.000000,0.000000,0.000000}%
\pgfsetstrokecolor{currentstroke}%
\pgfsetdash{}{0pt}%
\pgfsys@defobject{currentmarker}{\pgfqpoint{0.000000in}{-0.048611in}}{\pgfqpoint{0.000000in}{0.000000in}}{%
\pgfpathmoveto{\pgfqpoint{0.000000in}{0.000000in}}%
\pgfpathlineto{\pgfqpoint{0.000000in}{-0.048611in}}%
\pgfusepath{stroke,fill}%
}%
\begin{pgfscope}%
\pgfsys@transformshift{2.433190in}{0.553781in}%
\pgfsys@useobject{currentmarker}{}%
\end{pgfscope}%
\end{pgfscope}%
\begin{pgfscope}%
\definecolor{textcolor}{rgb}{0.000000,0.000000,0.000000}%
\pgfsetstrokecolor{textcolor}%
\pgfsetfillcolor{textcolor}%
\pgftext[x=2.433190in,y=0.456558in,,top]{\color{textcolor}\rmfamily\fontsize{8.000000}{9.600000}\selectfont \(\displaystyle {1.0}\)}%
\end{pgfscope}%
\begin{pgfscope}%
\pgfpathrectangle{\pgfqpoint{0.572918in}{0.553781in}}{\pgfqpoint{5.478282in}{2.095553in}}%
\pgfusepath{clip}%
\pgfsetbuttcap%
\pgfsetroundjoin%
\pgfsetlinewidth{0.501875pt}%
\definecolor{currentstroke}{rgb}{0.690196,0.690196,0.690196}%
\pgfsetstrokecolor{currentstroke}%
\pgfsetstrokeopacity{0.500000}%
\pgfsetdash{{0.500000pt}{0.825000pt}}{0.000000pt}%
\pgfpathmoveto{\pgfqpoint{3.312059in}{0.553781in}}%
\pgfpathlineto{\pgfqpoint{3.312059in}{2.649333in}}%
\pgfusepath{stroke}%
\end{pgfscope}%
\begin{pgfscope}%
\pgfsetbuttcap%
\pgfsetroundjoin%
\definecolor{currentfill}{rgb}{0.000000,0.000000,0.000000}%
\pgfsetfillcolor{currentfill}%
\pgfsetlinewidth{0.803000pt}%
\definecolor{currentstroke}{rgb}{0.000000,0.000000,0.000000}%
\pgfsetstrokecolor{currentstroke}%
\pgfsetdash{}{0pt}%
\pgfsys@defobject{currentmarker}{\pgfqpoint{0.000000in}{-0.048611in}}{\pgfqpoint{0.000000in}{0.000000in}}{%
\pgfpathmoveto{\pgfqpoint{0.000000in}{0.000000in}}%
\pgfpathlineto{\pgfqpoint{0.000000in}{-0.048611in}}%
\pgfusepath{stroke,fill}%
}%
\begin{pgfscope}%
\pgfsys@transformshift{3.312059in}{0.553781in}%
\pgfsys@useobject{currentmarker}{}%
\end{pgfscope}%
\end{pgfscope}%
\begin{pgfscope}%
\definecolor{textcolor}{rgb}{0.000000,0.000000,0.000000}%
\pgfsetstrokecolor{textcolor}%
\pgfsetfillcolor{textcolor}%
\pgftext[x=3.312059in,y=0.456558in,,top]{\color{textcolor}\rmfamily\fontsize{8.000000}{9.600000}\selectfont \(\displaystyle {1.5}\)}%
\end{pgfscope}%
\begin{pgfscope}%
\pgfpathrectangle{\pgfqpoint{0.572918in}{0.553781in}}{\pgfqpoint{5.478282in}{2.095553in}}%
\pgfusepath{clip}%
\pgfsetbuttcap%
\pgfsetroundjoin%
\pgfsetlinewidth{0.501875pt}%
\definecolor{currentstroke}{rgb}{0.690196,0.690196,0.690196}%
\pgfsetstrokecolor{currentstroke}%
\pgfsetstrokeopacity{0.500000}%
\pgfsetdash{{0.500000pt}{0.825000pt}}{0.000000pt}%
\pgfpathmoveto{\pgfqpoint{4.190928in}{0.553781in}}%
\pgfpathlineto{\pgfqpoint{4.190928in}{2.649333in}}%
\pgfusepath{stroke}%
\end{pgfscope}%
\begin{pgfscope}%
\pgfsetbuttcap%
\pgfsetroundjoin%
\definecolor{currentfill}{rgb}{0.000000,0.000000,0.000000}%
\pgfsetfillcolor{currentfill}%
\pgfsetlinewidth{0.803000pt}%
\definecolor{currentstroke}{rgb}{0.000000,0.000000,0.000000}%
\pgfsetstrokecolor{currentstroke}%
\pgfsetdash{}{0pt}%
\pgfsys@defobject{currentmarker}{\pgfqpoint{0.000000in}{-0.048611in}}{\pgfqpoint{0.000000in}{0.000000in}}{%
\pgfpathmoveto{\pgfqpoint{0.000000in}{0.000000in}}%
\pgfpathlineto{\pgfqpoint{0.000000in}{-0.048611in}}%
\pgfusepath{stroke,fill}%
}%
\begin{pgfscope}%
\pgfsys@transformshift{4.190928in}{0.553781in}%
\pgfsys@useobject{currentmarker}{}%
\end{pgfscope}%
\end{pgfscope}%
\begin{pgfscope}%
\definecolor{textcolor}{rgb}{0.000000,0.000000,0.000000}%
\pgfsetstrokecolor{textcolor}%
\pgfsetfillcolor{textcolor}%
\pgftext[x=4.190928in,y=0.456558in,,top]{\color{textcolor}\rmfamily\fontsize{8.000000}{9.600000}\selectfont \(\displaystyle {2.0}\)}%
\end{pgfscope}%
\begin{pgfscope}%
\pgfpathrectangle{\pgfqpoint{0.572918in}{0.553781in}}{\pgfqpoint{5.478282in}{2.095553in}}%
\pgfusepath{clip}%
\pgfsetbuttcap%
\pgfsetroundjoin%
\pgfsetlinewidth{0.501875pt}%
\definecolor{currentstroke}{rgb}{0.690196,0.690196,0.690196}%
\pgfsetstrokecolor{currentstroke}%
\pgfsetstrokeopacity{0.500000}%
\pgfsetdash{{0.500000pt}{0.825000pt}}{0.000000pt}%
\pgfpathmoveto{\pgfqpoint{5.069797in}{0.553781in}}%
\pgfpathlineto{\pgfqpoint{5.069797in}{2.649333in}}%
\pgfusepath{stroke}%
\end{pgfscope}%
\begin{pgfscope}%
\pgfsetbuttcap%
\pgfsetroundjoin%
\definecolor{currentfill}{rgb}{0.000000,0.000000,0.000000}%
\pgfsetfillcolor{currentfill}%
\pgfsetlinewidth{0.803000pt}%
\definecolor{currentstroke}{rgb}{0.000000,0.000000,0.000000}%
\pgfsetstrokecolor{currentstroke}%
\pgfsetdash{}{0pt}%
\pgfsys@defobject{currentmarker}{\pgfqpoint{0.000000in}{-0.048611in}}{\pgfqpoint{0.000000in}{0.000000in}}{%
\pgfpathmoveto{\pgfqpoint{0.000000in}{0.000000in}}%
\pgfpathlineto{\pgfqpoint{0.000000in}{-0.048611in}}%
\pgfusepath{stroke,fill}%
}%
\begin{pgfscope}%
\pgfsys@transformshift{5.069797in}{0.553781in}%
\pgfsys@useobject{currentmarker}{}%
\end{pgfscope}%
\end{pgfscope}%
\begin{pgfscope}%
\definecolor{textcolor}{rgb}{0.000000,0.000000,0.000000}%
\pgfsetstrokecolor{textcolor}%
\pgfsetfillcolor{textcolor}%
\pgftext[x=5.069797in,y=0.456558in,,top]{\color{textcolor}\rmfamily\fontsize{8.000000}{9.600000}\selectfont \(\displaystyle {2.5}\)}%
\end{pgfscope}%
\begin{pgfscope}%
\pgfpathrectangle{\pgfqpoint{0.572918in}{0.553781in}}{\pgfqpoint{5.478282in}{2.095553in}}%
\pgfusepath{clip}%
\pgfsetbuttcap%
\pgfsetroundjoin%
\pgfsetlinewidth{0.501875pt}%
\definecolor{currentstroke}{rgb}{0.690196,0.690196,0.690196}%
\pgfsetstrokecolor{currentstroke}%
\pgfsetstrokeopacity{0.500000}%
\pgfsetdash{{0.500000pt}{0.825000pt}}{0.000000pt}%
\pgfpathmoveto{\pgfqpoint{5.948665in}{0.553781in}}%
\pgfpathlineto{\pgfqpoint{5.948665in}{2.649333in}}%
\pgfusepath{stroke}%
\end{pgfscope}%
\begin{pgfscope}%
\pgfsetbuttcap%
\pgfsetroundjoin%
\definecolor{currentfill}{rgb}{0.000000,0.000000,0.000000}%
\pgfsetfillcolor{currentfill}%
\pgfsetlinewidth{0.803000pt}%
\definecolor{currentstroke}{rgb}{0.000000,0.000000,0.000000}%
\pgfsetstrokecolor{currentstroke}%
\pgfsetdash{}{0pt}%
\pgfsys@defobject{currentmarker}{\pgfqpoint{0.000000in}{-0.048611in}}{\pgfqpoint{0.000000in}{0.000000in}}{%
\pgfpathmoveto{\pgfqpoint{0.000000in}{0.000000in}}%
\pgfpathlineto{\pgfqpoint{0.000000in}{-0.048611in}}%
\pgfusepath{stroke,fill}%
}%
\begin{pgfscope}%
\pgfsys@transformshift{5.948665in}{0.553781in}%
\pgfsys@useobject{currentmarker}{}%
\end{pgfscope}%
\end{pgfscope}%
\begin{pgfscope}%
\definecolor{textcolor}{rgb}{0.000000,0.000000,0.000000}%
\pgfsetstrokecolor{textcolor}%
\pgfsetfillcolor{textcolor}%
\pgftext[x=5.948665in,y=0.456558in,,top]{\color{textcolor}\rmfamily\fontsize{8.000000}{9.600000}\selectfont \(\displaystyle {3.0}\)}%
\end{pgfscope}%
\begin{pgfscope}%
\definecolor{textcolor}{rgb}{0.000000,0.000000,0.000000}%
\pgfsetstrokecolor{textcolor}%
\pgfsetfillcolor{textcolor}%
\pgftext[x=3.312059in,y=0.302336in,,top]{\color{textcolor}\rmfamily\fontsize{10.950000}{13.140000}\selectfont \(\displaystyle \log_{10}(E_{\textup{true}}) \, \left[ E / \textup{GeV} \right]\)}%
\end{pgfscope}%
\begin{pgfscope}%
\pgfpathrectangle{\pgfqpoint{0.572918in}{0.553781in}}{\pgfqpoint{5.478282in}{2.095553in}}%
\pgfusepath{clip}%
\pgfsetbuttcap%
\pgfsetroundjoin%
\pgfsetlinewidth{0.501875pt}%
\definecolor{currentstroke}{rgb}{0.690196,0.690196,0.690196}%
\pgfsetstrokecolor{currentstroke}%
\pgfsetstrokeopacity{0.500000}%
\pgfsetdash{{0.500000pt}{0.825000pt}}{0.000000pt}%
\pgfpathmoveto{\pgfqpoint{0.572918in}{0.580751in}}%
\pgfpathlineto{\pgfqpoint{6.051200in}{0.580751in}}%
\pgfusepath{stroke}%
\end{pgfscope}%
\begin{pgfscope}%
\pgfsetbuttcap%
\pgfsetroundjoin%
\definecolor{currentfill}{rgb}{0.000000,0.000000,0.000000}%
\pgfsetfillcolor{currentfill}%
\pgfsetlinewidth{0.803000pt}%
\definecolor{currentstroke}{rgb}{0.000000,0.000000,0.000000}%
\pgfsetstrokecolor{currentstroke}%
\pgfsetdash{}{0pt}%
\pgfsys@defobject{currentmarker}{\pgfqpoint{-0.048611in}{0.000000in}}{\pgfqpoint{-0.000000in}{0.000000in}}{%
\pgfpathmoveto{\pgfqpoint{-0.000000in}{0.000000in}}%
\pgfpathlineto{\pgfqpoint{-0.048611in}{0.000000in}}%
\pgfusepath{stroke,fill}%
}%
\begin{pgfscope}%
\pgfsys@transformshift{0.572918in}{0.580751in}%
\pgfsys@useobject{currentmarker}{}%
\end{pgfscope}%
\end{pgfscope}%
\begin{pgfscope}%
\definecolor{textcolor}{rgb}{0.000000,0.000000,0.000000}%
\pgfsetstrokecolor{textcolor}%
\pgfsetfillcolor{textcolor}%
\pgftext[x=0.357639in, y=0.542195in, left, base]{\color{textcolor}\rmfamily\fontsize{8.000000}{9.600000}\selectfont \(\displaystyle {10}\)}%
\end{pgfscope}%
\begin{pgfscope}%
\pgfpathrectangle{\pgfqpoint{0.572918in}{0.553781in}}{\pgfqpoint{5.478282in}{2.095553in}}%
\pgfusepath{clip}%
\pgfsetbuttcap%
\pgfsetroundjoin%
\pgfsetlinewidth{0.501875pt}%
\definecolor{currentstroke}{rgb}{0.690196,0.690196,0.690196}%
\pgfsetstrokecolor{currentstroke}%
\pgfsetstrokeopacity{0.500000}%
\pgfsetdash{{0.500000pt}{0.825000pt}}{0.000000pt}%
\pgfpathmoveto{\pgfqpoint{0.572918in}{0.841801in}}%
\pgfpathlineto{\pgfqpoint{6.051200in}{0.841801in}}%
\pgfusepath{stroke}%
\end{pgfscope}%
\begin{pgfscope}%
\pgfsetbuttcap%
\pgfsetroundjoin%
\definecolor{currentfill}{rgb}{0.000000,0.000000,0.000000}%
\pgfsetfillcolor{currentfill}%
\pgfsetlinewidth{0.803000pt}%
\definecolor{currentstroke}{rgb}{0.000000,0.000000,0.000000}%
\pgfsetstrokecolor{currentstroke}%
\pgfsetdash{}{0pt}%
\pgfsys@defobject{currentmarker}{\pgfqpoint{-0.048611in}{0.000000in}}{\pgfqpoint{-0.000000in}{0.000000in}}{%
\pgfpathmoveto{\pgfqpoint{-0.000000in}{0.000000in}}%
\pgfpathlineto{\pgfqpoint{-0.048611in}{0.000000in}}%
\pgfusepath{stroke,fill}%
}%
\begin{pgfscope}%
\pgfsys@transformshift{0.572918in}{0.841801in}%
\pgfsys@useobject{currentmarker}{}%
\end{pgfscope}%
\end{pgfscope}%
\begin{pgfscope}%
\definecolor{textcolor}{rgb}{0.000000,0.000000,0.000000}%
\pgfsetstrokecolor{textcolor}%
\pgfsetfillcolor{textcolor}%
\pgftext[x=0.357639in, y=0.803245in, left, base]{\color{textcolor}\rmfamily\fontsize{8.000000}{9.600000}\selectfont \(\displaystyle {15}\)}%
\end{pgfscope}%
\begin{pgfscope}%
\pgfpathrectangle{\pgfqpoint{0.572918in}{0.553781in}}{\pgfqpoint{5.478282in}{2.095553in}}%
\pgfusepath{clip}%
\pgfsetbuttcap%
\pgfsetroundjoin%
\pgfsetlinewidth{0.501875pt}%
\definecolor{currentstroke}{rgb}{0.690196,0.690196,0.690196}%
\pgfsetstrokecolor{currentstroke}%
\pgfsetstrokeopacity{0.500000}%
\pgfsetdash{{0.500000pt}{0.825000pt}}{0.000000pt}%
\pgfpathmoveto{\pgfqpoint{0.572918in}{1.102851in}}%
\pgfpathlineto{\pgfqpoint{6.051200in}{1.102851in}}%
\pgfusepath{stroke}%
\end{pgfscope}%
\begin{pgfscope}%
\pgfsetbuttcap%
\pgfsetroundjoin%
\definecolor{currentfill}{rgb}{0.000000,0.000000,0.000000}%
\pgfsetfillcolor{currentfill}%
\pgfsetlinewidth{0.803000pt}%
\definecolor{currentstroke}{rgb}{0.000000,0.000000,0.000000}%
\pgfsetstrokecolor{currentstroke}%
\pgfsetdash{}{0pt}%
\pgfsys@defobject{currentmarker}{\pgfqpoint{-0.048611in}{0.000000in}}{\pgfqpoint{-0.000000in}{0.000000in}}{%
\pgfpathmoveto{\pgfqpoint{-0.000000in}{0.000000in}}%
\pgfpathlineto{\pgfqpoint{-0.048611in}{0.000000in}}%
\pgfusepath{stroke,fill}%
}%
\begin{pgfscope}%
\pgfsys@transformshift{0.572918in}{1.102851in}%
\pgfsys@useobject{currentmarker}{}%
\end{pgfscope}%
\end{pgfscope}%
\begin{pgfscope}%
\definecolor{textcolor}{rgb}{0.000000,0.000000,0.000000}%
\pgfsetstrokecolor{textcolor}%
\pgfsetfillcolor{textcolor}%
\pgftext[x=0.357639in, y=1.064295in, left, base]{\color{textcolor}\rmfamily\fontsize{8.000000}{9.600000}\selectfont \(\displaystyle {20}\)}%
\end{pgfscope}%
\begin{pgfscope}%
\pgfpathrectangle{\pgfqpoint{0.572918in}{0.553781in}}{\pgfqpoint{5.478282in}{2.095553in}}%
\pgfusepath{clip}%
\pgfsetbuttcap%
\pgfsetroundjoin%
\pgfsetlinewidth{0.501875pt}%
\definecolor{currentstroke}{rgb}{0.690196,0.690196,0.690196}%
\pgfsetstrokecolor{currentstroke}%
\pgfsetstrokeopacity{0.500000}%
\pgfsetdash{{0.500000pt}{0.825000pt}}{0.000000pt}%
\pgfpathmoveto{\pgfqpoint{0.572918in}{1.363901in}}%
\pgfpathlineto{\pgfqpoint{6.051200in}{1.363901in}}%
\pgfusepath{stroke}%
\end{pgfscope}%
\begin{pgfscope}%
\pgfsetbuttcap%
\pgfsetroundjoin%
\definecolor{currentfill}{rgb}{0.000000,0.000000,0.000000}%
\pgfsetfillcolor{currentfill}%
\pgfsetlinewidth{0.803000pt}%
\definecolor{currentstroke}{rgb}{0.000000,0.000000,0.000000}%
\pgfsetstrokecolor{currentstroke}%
\pgfsetdash{}{0pt}%
\pgfsys@defobject{currentmarker}{\pgfqpoint{-0.048611in}{0.000000in}}{\pgfqpoint{-0.000000in}{0.000000in}}{%
\pgfpathmoveto{\pgfqpoint{-0.000000in}{0.000000in}}%
\pgfpathlineto{\pgfqpoint{-0.048611in}{0.000000in}}%
\pgfusepath{stroke,fill}%
}%
\begin{pgfscope}%
\pgfsys@transformshift{0.572918in}{1.363901in}%
\pgfsys@useobject{currentmarker}{}%
\end{pgfscope}%
\end{pgfscope}%
\begin{pgfscope}%
\definecolor{textcolor}{rgb}{0.000000,0.000000,0.000000}%
\pgfsetstrokecolor{textcolor}%
\pgfsetfillcolor{textcolor}%
\pgftext[x=0.357639in, y=1.325345in, left, base]{\color{textcolor}\rmfamily\fontsize{8.000000}{9.600000}\selectfont \(\displaystyle {25}\)}%
\end{pgfscope}%
\begin{pgfscope}%
\pgfpathrectangle{\pgfqpoint{0.572918in}{0.553781in}}{\pgfqpoint{5.478282in}{2.095553in}}%
\pgfusepath{clip}%
\pgfsetbuttcap%
\pgfsetroundjoin%
\pgfsetlinewidth{0.501875pt}%
\definecolor{currentstroke}{rgb}{0.690196,0.690196,0.690196}%
\pgfsetstrokecolor{currentstroke}%
\pgfsetstrokeopacity{0.500000}%
\pgfsetdash{{0.500000pt}{0.825000pt}}{0.000000pt}%
\pgfpathmoveto{\pgfqpoint{0.572918in}{1.624951in}}%
\pgfpathlineto{\pgfqpoint{6.051200in}{1.624951in}}%
\pgfusepath{stroke}%
\end{pgfscope}%
\begin{pgfscope}%
\pgfsetbuttcap%
\pgfsetroundjoin%
\definecolor{currentfill}{rgb}{0.000000,0.000000,0.000000}%
\pgfsetfillcolor{currentfill}%
\pgfsetlinewidth{0.803000pt}%
\definecolor{currentstroke}{rgb}{0.000000,0.000000,0.000000}%
\pgfsetstrokecolor{currentstroke}%
\pgfsetdash{}{0pt}%
\pgfsys@defobject{currentmarker}{\pgfqpoint{-0.048611in}{0.000000in}}{\pgfqpoint{-0.000000in}{0.000000in}}{%
\pgfpathmoveto{\pgfqpoint{-0.000000in}{0.000000in}}%
\pgfpathlineto{\pgfqpoint{-0.048611in}{0.000000in}}%
\pgfusepath{stroke,fill}%
}%
\begin{pgfscope}%
\pgfsys@transformshift{0.572918in}{1.624951in}%
\pgfsys@useobject{currentmarker}{}%
\end{pgfscope}%
\end{pgfscope}%
\begin{pgfscope}%
\definecolor{textcolor}{rgb}{0.000000,0.000000,0.000000}%
\pgfsetstrokecolor{textcolor}%
\pgfsetfillcolor{textcolor}%
\pgftext[x=0.357639in, y=1.586395in, left, base]{\color{textcolor}\rmfamily\fontsize{8.000000}{9.600000}\selectfont \(\displaystyle {30}\)}%
\end{pgfscope}%
\begin{pgfscope}%
\pgfpathrectangle{\pgfqpoint{0.572918in}{0.553781in}}{\pgfqpoint{5.478282in}{2.095553in}}%
\pgfusepath{clip}%
\pgfsetbuttcap%
\pgfsetroundjoin%
\pgfsetlinewidth{0.501875pt}%
\definecolor{currentstroke}{rgb}{0.690196,0.690196,0.690196}%
\pgfsetstrokecolor{currentstroke}%
\pgfsetstrokeopacity{0.500000}%
\pgfsetdash{{0.500000pt}{0.825000pt}}{0.000000pt}%
\pgfpathmoveto{\pgfqpoint{0.572918in}{1.886001in}}%
\pgfpathlineto{\pgfqpoint{6.051200in}{1.886001in}}%
\pgfusepath{stroke}%
\end{pgfscope}%
\begin{pgfscope}%
\pgfsetbuttcap%
\pgfsetroundjoin%
\definecolor{currentfill}{rgb}{0.000000,0.000000,0.000000}%
\pgfsetfillcolor{currentfill}%
\pgfsetlinewidth{0.803000pt}%
\definecolor{currentstroke}{rgb}{0.000000,0.000000,0.000000}%
\pgfsetstrokecolor{currentstroke}%
\pgfsetdash{}{0pt}%
\pgfsys@defobject{currentmarker}{\pgfqpoint{-0.048611in}{0.000000in}}{\pgfqpoint{-0.000000in}{0.000000in}}{%
\pgfpathmoveto{\pgfqpoint{-0.000000in}{0.000000in}}%
\pgfpathlineto{\pgfqpoint{-0.048611in}{0.000000in}}%
\pgfusepath{stroke,fill}%
}%
\begin{pgfscope}%
\pgfsys@transformshift{0.572918in}{1.886001in}%
\pgfsys@useobject{currentmarker}{}%
\end{pgfscope}%
\end{pgfscope}%
\begin{pgfscope}%
\definecolor{textcolor}{rgb}{0.000000,0.000000,0.000000}%
\pgfsetstrokecolor{textcolor}%
\pgfsetfillcolor{textcolor}%
\pgftext[x=0.357639in, y=1.847445in, left, base]{\color{textcolor}\rmfamily\fontsize{8.000000}{9.600000}\selectfont \(\displaystyle {35}\)}%
\end{pgfscope}%
\begin{pgfscope}%
\pgfpathrectangle{\pgfqpoint{0.572918in}{0.553781in}}{\pgfqpoint{5.478282in}{2.095553in}}%
\pgfusepath{clip}%
\pgfsetbuttcap%
\pgfsetroundjoin%
\pgfsetlinewidth{0.501875pt}%
\definecolor{currentstroke}{rgb}{0.690196,0.690196,0.690196}%
\pgfsetstrokecolor{currentstroke}%
\pgfsetstrokeopacity{0.500000}%
\pgfsetdash{{0.500000pt}{0.825000pt}}{0.000000pt}%
\pgfpathmoveto{\pgfqpoint{0.572918in}{2.147051in}}%
\pgfpathlineto{\pgfqpoint{6.051200in}{2.147051in}}%
\pgfusepath{stroke}%
\end{pgfscope}%
\begin{pgfscope}%
\pgfsetbuttcap%
\pgfsetroundjoin%
\definecolor{currentfill}{rgb}{0.000000,0.000000,0.000000}%
\pgfsetfillcolor{currentfill}%
\pgfsetlinewidth{0.803000pt}%
\definecolor{currentstroke}{rgb}{0.000000,0.000000,0.000000}%
\pgfsetstrokecolor{currentstroke}%
\pgfsetdash{}{0pt}%
\pgfsys@defobject{currentmarker}{\pgfqpoint{-0.048611in}{0.000000in}}{\pgfqpoint{-0.000000in}{0.000000in}}{%
\pgfpathmoveto{\pgfqpoint{-0.000000in}{0.000000in}}%
\pgfpathlineto{\pgfqpoint{-0.048611in}{0.000000in}}%
\pgfusepath{stroke,fill}%
}%
\begin{pgfscope}%
\pgfsys@transformshift{0.572918in}{2.147051in}%
\pgfsys@useobject{currentmarker}{}%
\end{pgfscope}%
\end{pgfscope}%
\begin{pgfscope}%
\definecolor{textcolor}{rgb}{0.000000,0.000000,0.000000}%
\pgfsetstrokecolor{textcolor}%
\pgfsetfillcolor{textcolor}%
\pgftext[x=0.357639in, y=2.108495in, left, base]{\color{textcolor}\rmfamily\fontsize{8.000000}{9.600000}\selectfont \(\displaystyle {40}\)}%
\end{pgfscope}%
\begin{pgfscope}%
\pgfpathrectangle{\pgfqpoint{0.572918in}{0.553781in}}{\pgfqpoint{5.478282in}{2.095553in}}%
\pgfusepath{clip}%
\pgfsetbuttcap%
\pgfsetroundjoin%
\pgfsetlinewidth{0.501875pt}%
\definecolor{currentstroke}{rgb}{0.690196,0.690196,0.690196}%
\pgfsetstrokecolor{currentstroke}%
\pgfsetstrokeopacity{0.500000}%
\pgfsetdash{{0.500000pt}{0.825000pt}}{0.000000pt}%
\pgfpathmoveto{\pgfqpoint{0.572918in}{2.408101in}}%
\pgfpathlineto{\pgfqpoint{6.051200in}{2.408101in}}%
\pgfusepath{stroke}%
\end{pgfscope}%
\begin{pgfscope}%
\pgfsetbuttcap%
\pgfsetroundjoin%
\definecolor{currentfill}{rgb}{0.000000,0.000000,0.000000}%
\pgfsetfillcolor{currentfill}%
\pgfsetlinewidth{0.803000pt}%
\definecolor{currentstroke}{rgb}{0.000000,0.000000,0.000000}%
\pgfsetstrokecolor{currentstroke}%
\pgfsetdash{}{0pt}%
\pgfsys@defobject{currentmarker}{\pgfqpoint{-0.048611in}{0.000000in}}{\pgfqpoint{-0.000000in}{0.000000in}}{%
\pgfpathmoveto{\pgfqpoint{-0.000000in}{0.000000in}}%
\pgfpathlineto{\pgfqpoint{-0.048611in}{0.000000in}}%
\pgfusepath{stroke,fill}%
}%
\begin{pgfscope}%
\pgfsys@transformshift{0.572918in}{2.408101in}%
\pgfsys@useobject{currentmarker}{}%
\end{pgfscope}%
\end{pgfscope}%
\begin{pgfscope}%
\definecolor{textcolor}{rgb}{0.000000,0.000000,0.000000}%
\pgfsetstrokecolor{textcolor}%
\pgfsetfillcolor{textcolor}%
\pgftext[x=0.357639in, y=2.369545in, left, base]{\color{textcolor}\rmfamily\fontsize{8.000000}{9.600000}\selectfont \(\displaystyle {45}\)}%
\end{pgfscope}%
\begin{pgfscope}%
\definecolor{textcolor}{rgb}{0.000000,0.000000,0.000000}%
\pgfsetstrokecolor{textcolor}%
\pgfsetfillcolor{textcolor}%
\pgftext[x=0.302083in,y=1.601557in,,bottom,rotate=90.000000]{\color{textcolor}\rmfamily\fontsize{10.950000}{13.140000}\selectfont IQR / 1.349 \(\displaystyle \left[ \textup{deg} \right]\)}%
\end{pgfscope}%
\begin{pgfscope}%
\pgfpathrectangle{\pgfqpoint{0.572918in}{0.553781in}}{\pgfqpoint{5.478282in}{2.095553in}}%
\pgfusepath{clip}%
\pgfsetbuttcap%
\pgfsetroundjoin%
\pgfsetlinewidth{1.505625pt}%
\definecolor{currentstroke}{rgb}{0.313725,0.317647,0.309804}%
\pgfsetstrokecolor{currentstroke}%
\pgfsetstrokeopacity{0.900000}%
\pgfsetdash{}{0pt}%
\pgfpathmoveto{\pgfqpoint{0.821931in}{1.853223in}}%
\pgfpathlineto{\pgfqpoint{0.821931in}{2.266559in}}%
\pgfusepath{stroke}%
\end{pgfscope}%
\begin{pgfscope}%
\pgfpathrectangle{\pgfqpoint{0.572918in}{0.553781in}}{\pgfqpoint{5.478282in}{2.095553in}}%
\pgfusepath{clip}%
\pgfsetbuttcap%
\pgfsetroundjoin%
\pgfsetlinewidth{1.505625pt}%
\definecolor{currentstroke}{rgb}{0.313725,0.317647,0.309804}%
\pgfsetstrokecolor{currentstroke}%
\pgfsetstrokeopacity{0.900000}%
\pgfsetdash{}{0pt}%
\pgfpathmoveto{\pgfqpoint{1.114887in}{1.863076in}}%
\pgfpathlineto{\pgfqpoint{1.114887in}{2.062997in}}%
\pgfusepath{stroke}%
\end{pgfscope}%
\begin{pgfscope}%
\pgfpathrectangle{\pgfqpoint{0.572918in}{0.553781in}}{\pgfqpoint{5.478282in}{2.095553in}}%
\pgfusepath{clip}%
\pgfsetbuttcap%
\pgfsetroundjoin%
\pgfsetlinewidth{1.505625pt}%
\definecolor{currentstroke}{rgb}{0.313725,0.317647,0.309804}%
\pgfsetstrokecolor{currentstroke}%
\pgfsetstrokeopacity{0.900000}%
\pgfsetdash{}{0pt}%
\pgfpathmoveto{\pgfqpoint{1.407844in}{1.870788in}}%
\pgfpathlineto{\pgfqpoint{1.407844in}{2.001867in}}%
\pgfusepath{stroke}%
\end{pgfscope}%
\begin{pgfscope}%
\pgfpathrectangle{\pgfqpoint{0.572918in}{0.553781in}}{\pgfqpoint{5.478282in}{2.095553in}}%
\pgfusepath{clip}%
\pgfsetbuttcap%
\pgfsetroundjoin%
\pgfsetlinewidth{1.505625pt}%
\definecolor{currentstroke}{rgb}{0.313725,0.317647,0.309804}%
\pgfsetstrokecolor{currentstroke}%
\pgfsetstrokeopacity{0.900000}%
\pgfsetdash{}{0pt}%
\pgfpathmoveto{\pgfqpoint{1.700800in}{1.955754in}}%
\pgfpathlineto{\pgfqpoint{1.700800in}{2.056646in}}%
\pgfusepath{stroke}%
\end{pgfscope}%
\begin{pgfscope}%
\pgfpathrectangle{\pgfqpoint{0.572918in}{0.553781in}}{\pgfqpoint{5.478282in}{2.095553in}}%
\pgfusepath{clip}%
\pgfsetbuttcap%
\pgfsetroundjoin%
\pgfsetlinewidth{1.505625pt}%
\definecolor{currentstroke}{rgb}{0.313725,0.317647,0.309804}%
\pgfsetstrokecolor{currentstroke}%
\pgfsetstrokeopacity{0.900000}%
\pgfsetdash{}{0pt}%
\pgfpathmoveto{\pgfqpoint{1.993756in}{1.977838in}}%
\pgfpathlineto{\pgfqpoint{1.993756in}{2.056402in}}%
\pgfusepath{stroke}%
\end{pgfscope}%
\begin{pgfscope}%
\pgfpathrectangle{\pgfqpoint{0.572918in}{0.553781in}}{\pgfqpoint{5.478282in}{2.095553in}}%
\pgfusepath{clip}%
\pgfsetbuttcap%
\pgfsetroundjoin%
\pgfsetlinewidth{1.505625pt}%
\definecolor{currentstroke}{rgb}{0.313725,0.317647,0.309804}%
\pgfsetstrokecolor{currentstroke}%
\pgfsetstrokeopacity{0.900000}%
\pgfsetdash{}{0pt}%
\pgfpathmoveto{\pgfqpoint{2.286712in}{1.963330in}}%
\pgfpathlineto{\pgfqpoint{2.286712in}{2.032372in}}%
\pgfusepath{stroke}%
\end{pgfscope}%
\begin{pgfscope}%
\pgfpathrectangle{\pgfqpoint{0.572918in}{0.553781in}}{\pgfqpoint{5.478282in}{2.095553in}}%
\pgfusepath{clip}%
\pgfsetbuttcap%
\pgfsetroundjoin%
\pgfsetlinewidth{1.505625pt}%
\definecolor{currentstroke}{rgb}{0.313725,0.317647,0.309804}%
\pgfsetstrokecolor{currentstroke}%
\pgfsetstrokeopacity{0.900000}%
\pgfsetdash{}{0pt}%
\pgfpathmoveto{\pgfqpoint{2.579669in}{1.786495in}}%
\pgfpathlineto{\pgfqpoint{2.579669in}{1.838736in}}%
\pgfusepath{stroke}%
\end{pgfscope}%
\begin{pgfscope}%
\pgfpathrectangle{\pgfqpoint{0.572918in}{0.553781in}}{\pgfqpoint{5.478282in}{2.095553in}}%
\pgfusepath{clip}%
\pgfsetbuttcap%
\pgfsetroundjoin%
\pgfsetlinewidth{1.505625pt}%
\definecolor{currentstroke}{rgb}{0.313725,0.317647,0.309804}%
\pgfsetstrokecolor{currentstroke}%
\pgfsetstrokeopacity{0.900000}%
\pgfsetdash{}{0pt}%
\pgfpathmoveto{\pgfqpoint{2.872625in}{1.653168in}}%
\pgfpathlineto{\pgfqpoint{2.872625in}{1.704447in}}%
\pgfusepath{stroke}%
\end{pgfscope}%
\begin{pgfscope}%
\pgfpathrectangle{\pgfqpoint{0.572918in}{0.553781in}}{\pgfqpoint{5.478282in}{2.095553in}}%
\pgfusepath{clip}%
\pgfsetbuttcap%
\pgfsetroundjoin%
\pgfsetlinewidth{1.505625pt}%
\definecolor{currentstroke}{rgb}{0.313725,0.317647,0.309804}%
\pgfsetstrokecolor{currentstroke}%
\pgfsetstrokeopacity{0.900000}%
\pgfsetdash{}{0pt}%
\pgfpathmoveto{\pgfqpoint{3.165581in}{1.447782in}}%
\pgfpathlineto{\pgfqpoint{3.165581in}{1.488545in}}%
\pgfusepath{stroke}%
\end{pgfscope}%
\begin{pgfscope}%
\pgfpathrectangle{\pgfqpoint{0.572918in}{0.553781in}}{\pgfqpoint{5.478282in}{2.095553in}}%
\pgfusepath{clip}%
\pgfsetbuttcap%
\pgfsetroundjoin%
\pgfsetlinewidth{1.505625pt}%
\definecolor{currentstroke}{rgb}{0.313725,0.317647,0.309804}%
\pgfsetstrokecolor{currentstroke}%
\pgfsetstrokeopacity{0.900000}%
\pgfsetdash{}{0pt}%
\pgfpathmoveto{\pgfqpoint{3.458537in}{1.281419in}}%
\pgfpathlineto{\pgfqpoint{3.458537in}{1.320106in}}%
\pgfusepath{stroke}%
\end{pgfscope}%
\begin{pgfscope}%
\pgfpathrectangle{\pgfqpoint{0.572918in}{0.553781in}}{\pgfqpoint{5.478282in}{2.095553in}}%
\pgfusepath{clip}%
\pgfsetbuttcap%
\pgfsetroundjoin%
\pgfsetlinewidth{1.505625pt}%
\definecolor{currentstroke}{rgb}{0.313725,0.317647,0.309804}%
\pgfsetstrokecolor{currentstroke}%
\pgfsetstrokeopacity{0.900000}%
\pgfsetdash{}{0pt}%
\pgfpathmoveto{\pgfqpoint{3.751494in}{1.084302in}}%
\pgfpathlineto{\pgfqpoint{3.751494in}{1.120020in}}%
\pgfusepath{stroke}%
\end{pgfscope}%
\begin{pgfscope}%
\pgfpathrectangle{\pgfqpoint{0.572918in}{0.553781in}}{\pgfqpoint{5.478282in}{2.095553in}}%
\pgfusepath{clip}%
\pgfsetbuttcap%
\pgfsetroundjoin%
\pgfsetlinewidth{1.505625pt}%
\definecolor{currentstroke}{rgb}{0.313725,0.317647,0.309804}%
\pgfsetstrokecolor{currentstroke}%
\pgfsetstrokeopacity{0.900000}%
\pgfsetdash{}{0pt}%
\pgfpathmoveto{\pgfqpoint{4.044450in}{0.969110in}}%
\pgfpathlineto{\pgfqpoint{4.044450in}{1.008389in}}%
\pgfusepath{stroke}%
\end{pgfscope}%
\begin{pgfscope}%
\pgfpathrectangle{\pgfqpoint{0.572918in}{0.553781in}}{\pgfqpoint{5.478282in}{2.095553in}}%
\pgfusepath{clip}%
\pgfsetbuttcap%
\pgfsetroundjoin%
\pgfsetlinewidth{1.505625pt}%
\definecolor{currentstroke}{rgb}{0.313725,0.317647,0.309804}%
\pgfsetstrokecolor{currentstroke}%
\pgfsetstrokeopacity{0.900000}%
\pgfsetdash{}{0pt}%
\pgfpathmoveto{\pgfqpoint{4.337406in}{0.843489in}}%
\pgfpathlineto{\pgfqpoint{4.337406in}{0.881250in}}%
\pgfusepath{stroke}%
\end{pgfscope}%
\begin{pgfscope}%
\pgfpathrectangle{\pgfqpoint{0.572918in}{0.553781in}}{\pgfqpoint{5.478282in}{2.095553in}}%
\pgfusepath{clip}%
\pgfsetbuttcap%
\pgfsetroundjoin%
\pgfsetlinewidth{1.505625pt}%
\definecolor{currentstroke}{rgb}{0.313725,0.317647,0.309804}%
\pgfsetstrokecolor{currentstroke}%
\pgfsetstrokeopacity{0.900000}%
\pgfsetdash{}{0pt}%
\pgfpathmoveto{\pgfqpoint{4.630362in}{0.764754in}}%
\pgfpathlineto{\pgfqpoint{4.630362in}{0.804011in}}%
\pgfusepath{stroke}%
\end{pgfscope}%
\begin{pgfscope}%
\pgfpathrectangle{\pgfqpoint{0.572918in}{0.553781in}}{\pgfqpoint{5.478282in}{2.095553in}}%
\pgfusepath{clip}%
\pgfsetbuttcap%
\pgfsetroundjoin%
\pgfsetlinewidth{1.505625pt}%
\definecolor{currentstroke}{rgb}{0.313725,0.317647,0.309804}%
\pgfsetstrokecolor{currentstroke}%
\pgfsetstrokeopacity{0.900000}%
\pgfsetdash{}{0pt}%
\pgfpathmoveto{\pgfqpoint{4.923318in}{0.698590in}}%
\pgfpathlineto{\pgfqpoint{4.923318in}{0.739377in}}%
\pgfusepath{stroke}%
\end{pgfscope}%
\begin{pgfscope}%
\pgfpathrectangle{\pgfqpoint{0.572918in}{0.553781in}}{\pgfqpoint{5.478282in}{2.095553in}}%
\pgfusepath{clip}%
\pgfsetbuttcap%
\pgfsetroundjoin%
\pgfsetlinewidth{1.505625pt}%
\definecolor{currentstroke}{rgb}{0.313725,0.317647,0.309804}%
\pgfsetstrokecolor{currentstroke}%
\pgfsetstrokeopacity{0.900000}%
\pgfsetdash{}{0pt}%
\pgfpathmoveto{\pgfqpoint{5.216275in}{0.690584in}}%
\pgfpathlineto{\pgfqpoint{5.216275in}{0.744443in}}%
\pgfusepath{stroke}%
\end{pgfscope}%
\begin{pgfscope}%
\pgfpathrectangle{\pgfqpoint{0.572918in}{0.553781in}}{\pgfqpoint{5.478282in}{2.095553in}}%
\pgfusepath{clip}%
\pgfsetbuttcap%
\pgfsetroundjoin%
\pgfsetlinewidth{1.505625pt}%
\definecolor{currentstroke}{rgb}{0.313725,0.317647,0.309804}%
\pgfsetstrokecolor{currentstroke}%
\pgfsetstrokeopacity{0.900000}%
\pgfsetdash{}{0pt}%
\pgfpathmoveto{\pgfqpoint{5.509231in}{0.649033in}}%
\pgfpathlineto{\pgfqpoint{5.509231in}{0.704159in}}%
\pgfusepath{stroke}%
\end{pgfscope}%
\begin{pgfscope}%
\pgfpathrectangle{\pgfqpoint{0.572918in}{0.553781in}}{\pgfqpoint{5.478282in}{2.095553in}}%
\pgfusepath{clip}%
\pgfsetbuttcap%
\pgfsetroundjoin%
\pgfsetlinewidth{1.505625pt}%
\definecolor{currentstroke}{rgb}{0.313725,0.317647,0.309804}%
\pgfsetstrokecolor{currentstroke}%
\pgfsetstrokeopacity{0.900000}%
\pgfsetdash{}{0pt}%
\pgfpathmoveto{\pgfqpoint{5.802187in}{0.654496in}}%
\pgfpathlineto{\pgfqpoint{5.802187in}{0.735595in}}%
\pgfusepath{stroke}%
\end{pgfscope}%
\begin{pgfscope}%
\pgfpathrectangle{\pgfqpoint{0.572918in}{0.553781in}}{\pgfqpoint{5.478282in}{2.095553in}}%
\pgfusepath{clip}%
\pgfsetbuttcap%
\pgfsetroundjoin%
\pgfsetlinewidth{1.505625pt}%
\definecolor{currentstroke}{rgb}{0.949020,0.372549,0.360784}%
\pgfsetstrokecolor{currentstroke}%
\pgfsetstrokeopacity{0.900000}%
\pgfsetdash{}{0pt}%
\pgfpathmoveto{\pgfqpoint{0.821931in}{2.009052in}}%
\pgfpathlineto{\pgfqpoint{0.821931in}{2.554081in}}%
\pgfusepath{stroke}%
\end{pgfscope}%
\begin{pgfscope}%
\pgfpathrectangle{\pgfqpoint{0.572918in}{0.553781in}}{\pgfqpoint{5.478282in}{2.095553in}}%
\pgfusepath{clip}%
\pgfsetbuttcap%
\pgfsetroundjoin%
\pgfsetlinewidth{1.505625pt}%
\definecolor{currentstroke}{rgb}{0.949020,0.372549,0.360784}%
\pgfsetstrokecolor{currentstroke}%
\pgfsetstrokeopacity{0.900000}%
\pgfsetdash{}{0pt}%
\pgfpathmoveto{\pgfqpoint{1.114887in}{2.065386in}}%
\pgfpathlineto{\pgfqpoint{1.114887in}{2.323778in}}%
\pgfusepath{stroke}%
\end{pgfscope}%
\begin{pgfscope}%
\pgfpathrectangle{\pgfqpoint{0.572918in}{0.553781in}}{\pgfqpoint{5.478282in}{2.095553in}}%
\pgfusepath{clip}%
\pgfsetbuttcap%
\pgfsetroundjoin%
\pgfsetlinewidth{1.505625pt}%
\definecolor{currentstroke}{rgb}{0.949020,0.372549,0.360784}%
\pgfsetstrokecolor{currentstroke}%
\pgfsetstrokeopacity{0.900000}%
\pgfsetdash{}{0pt}%
\pgfpathmoveto{\pgfqpoint{1.407844in}{1.964859in}}%
\pgfpathlineto{\pgfqpoint{1.407844in}{2.137199in}}%
\pgfusepath{stroke}%
\end{pgfscope}%
\begin{pgfscope}%
\pgfpathrectangle{\pgfqpoint{0.572918in}{0.553781in}}{\pgfqpoint{5.478282in}{2.095553in}}%
\pgfusepath{clip}%
\pgfsetbuttcap%
\pgfsetroundjoin%
\pgfsetlinewidth{1.505625pt}%
\definecolor{currentstroke}{rgb}{0.949020,0.372549,0.360784}%
\pgfsetstrokecolor{currentstroke}%
\pgfsetstrokeopacity{0.900000}%
\pgfsetdash{}{0pt}%
\pgfpathmoveto{\pgfqpoint{1.700800in}{2.068923in}}%
\pgfpathlineto{\pgfqpoint{1.700800in}{2.171082in}}%
\pgfusepath{stroke}%
\end{pgfscope}%
\begin{pgfscope}%
\pgfpathrectangle{\pgfqpoint{0.572918in}{0.553781in}}{\pgfqpoint{5.478282in}{2.095553in}}%
\pgfusepath{clip}%
\pgfsetbuttcap%
\pgfsetroundjoin%
\pgfsetlinewidth{1.505625pt}%
\definecolor{currentstroke}{rgb}{0.949020,0.372549,0.360784}%
\pgfsetstrokecolor{currentstroke}%
\pgfsetstrokeopacity{0.900000}%
\pgfsetdash{}{0pt}%
\pgfpathmoveto{\pgfqpoint{1.993756in}{2.036912in}}%
\pgfpathlineto{\pgfqpoint{1.993756in}{2.112253in}}%
\pgfusepath{stroke}%
\end{pgfscope}%
\begin{pgfscope}%
\pgfpathrectangle{\pgfqpoint{0.572918in}{0.553781in}}{\pgfqpoint{5.478282in}{2.095553in}}%
\pgfusepath{clip}%
\pgfsetbuttcap%
\pgfsetroundjoin%
\pgfsetlinewidth{1.505625pt}%
\definecolor{currentstroke}{rgb}{0.949020,0.372549,0.360784}%
\pgfsetstrokecolor{currentstroke}%
\pgfsetstrokeopacity{0.900000}%
\pgfsetdash{}{0pt}%
\pgfpathmoveto{\pgfqpoint{2.286712in}{1.963741in}}%
\pgfpathlineto{\pgfqpoint{2.286712in}{2.032198in}}%
\pgfusepath{stroke}%
\end{pgfscope}%
\begin{pgfscope}%
\pgfpathrectangle{\pgfqpoint{0.572918in}{0.553781in}}{\pgfqpoint{5.478282in}{2.095553in}}%
\pgfusepath{clip}%
\pgfsetbuttcap%
\pgfsetroundjoin%
\pgfsetlinewidth{1.505625pt}%
\definecolor{currentstroke}{rgb}{0.949020,0.372549,0.360784}%
\pgfsetstrokecolor{currentstroke}%
\pgfsetstrokeopacity{0.900000}%
\pgfsetdash{}{0pt}%
\pgfpathmoveto{\pgfqpoint{2.579669in}{1.849130in}}%
\pgfpathlineto{\pgfqpoint{2.579669in}{1.900667in}}%
\pgfusepath{stroke}%
\end{pgfscope}%
\begin{pgfscope}%
\pgfpathrectangle{\pgfqpoint{0.572918in}{0.553781in}}{\pgfqpoint{5.478282in}{2.095553in}}%
\pgfusepath{clip}%
\pgfsetbuttcap%
\pgfsetroundjoin%
\pgfsetlinewidth{1.505625pt}%
\definecolor{currentstroke}{rgb}{0.949020,0.372549,0.360784}%
\pgfsetstrokecolor{currentstroke}%
\pgfsetstrokeopacity{0.900000}%
\pgfsetdash{}{0pt}%
\pgfpathmoveto{\pgfqpoint{2.872625in}{1.688203in}}%
\pgfpathlineto{\pgfqpoint{2.872625in}{1.735750in}}%
\pgfusepath{stroke}%
\end{pgfscope}%
\begin{pgfscope}%
\pgfpathrectangle{\pgfqpoint{0.572918in}{0.553781in}}{\pgfqpoint{5.478282in}{2.095553in}}%
\pgfusepath{clip}%
\pgfsetbuttcap%
\pgfsetroundjoin%
\pgfsetlinewidth{1.505625pt}%
\definecolor{currentstroke}{rgb}{0.949020,0.372549,0.360784}%
\pgfsetstrokecolor{currentstroke}%
\pgfsetstrokeopacity{0.900000}%
\pgfsetdash{}{0pt}%
\pgfpathmoveto{\pgfqpoint{3.165581in}{1.470520in}}%
\pgfpathlineto{\pgfqpoint{3.165581in}{1.518505in}}%
\pgfusepath{stroke}%
\end{pgfscope}%
\begin{pgfscope}%
\pgfpathrectangle{\pgfqpoint{0.572918in}{0.553781in}}{\pgfqpoint{5.478282in}{2.095553in}}%
\pgfusepath{clip}%
\pgfsetbuttcap%
\pgfsetroundjoin%
\pgfsetlinewidth{1.505625pt}%
\definecolor{currentstroke}{rgb}{0.949020,0.372549,0.360784}%
\pgfsetstrokecolor{currentstroke}%
\pgfsetstrokeopacity{0.900000}%
\pgfsetdash{}{0pt}%
\pgfpathmoveto{\pgfqpoint{3.458537in}{1.317661in}}%
\pgfpathlineto{\pgfqpoint{3.458537in}{1.361766in}}%
\pgfusepath{stroke}%
\end{pgfscope}%
\begin{pgfscope}%
\pgfpathrectangle{\pgfqpoint{0.572918in}{0.553781in}}{\pgfqpoint{5.478282in}{2.095553in}}%
\pgfusepath{clip}%
\pgfsetbuttcap%
\pgfsetroundjoin%
\pgfsetlinewidth{1.505625pt}%
\definecolor{currentstroke}{rgb}{0.949020,0.372549,0.360784}%
\pgfsetstrokecolor{currentstroke}%
\pgfsetstrokeopacity{0.900000}%
\pgfsetdash{}{0pt}%
\pgfpathmoveto{\pgfqpoint{3.751494in}{1.100790in}}%
\pgfpathlineto{\pgfqpoint{3.751494in}{1.143692in}}%
\pgfusepath{stroke}%
\end{pgfscope}%
\begin{pgfscope}%
\pgfpathrectangle{\pgfqpoint{0.572918in}{0.553781in}}{\pgfqpoint{5.478282in}{2.095553in}}%
\pgfusepath{clip}%
\pgfsetbuttcap%
\pgfsetroundjoin%
\pgfsetlinewidth{1.505625pt}%
\definecolor{currentstroke}{rgb}{0.949020,0.372549,0.360784}%
\pgfsetstrokecolor{currentstroke}%
\pgfsetstrokeopacity{0.900000}%
\pgfsetdash{}{0pt}%
\pgfpathmoveto{\pgfqpoint{4.044450in}{0.950593in}}%
\pgfpathlineto{\pgfqpoint{4.044450in}{0.988447in}}%
\pgfusepath{stroke}%
\end{pgfscope}%
\begin{pgfscope}%
\pgfpathrectangle{\pgfqpoint{0.572918in}{0.553781in}}{\pgfqpoint{5.478282in}{2.095553in}}%
\pgfusepath{clip}%
\pgfsetbuttcap%
\pgfsetroundjoin%
\pgfsetlinewidth{1.505625pt}%
\definecolor{currentstroke}{rgb}{0.949020,0.372549,0.360784}%
\pgfsetstrokecolor{currentstroke}%
\pgfsetstrokeopacity{0.900000}%
\pgfsetdash{}{0pt}%
\pgfpathmoveto{\pgfqpoint{4.337406in}{0.840002in}}%
\pgfpathlineto{\pgfqpoint{4.337406in}{0.883678in}}%
\pgfusepath{stroke}%
\end{pgfscope}%
\begin{pgfscope}%
\pgfpathrectangle{\pgfqpoint{0.572918in}{0.553781in}}{\pgfqpoint{5.478282in}{2.095553in}}%
\pgfusepath{clip}%
\pgfsetbuttcap%
\pgfsetroundjoin%
\pgfsetlinewidth{1.505625pt}%
\definecolor{currentstroke}{rgb}{0.949020,0.372549,0.360784}%
\pgfsetstrokecolor{currentstroke}%
\pgfsetstrokeopacity{0.900000}%
\pgfsetdash{}{0pt}%
\pgfpathmoveto{\pgfqpoint{4.630362in}{0.755124in}}%
\pgfpathlineto{\pgfqpoint{4.630362in}{0.801728in}}%
\pgfusepath{stroke}%
\end{pgfscope}%
\begin{pgfscope}%
\pgfpathrectangle{\pgfqpoint{0.572918in}{0.553781in}}{\pgfqpoint{5.478282in}{2.095553in}}%
\pgfusepath{clip}%
\pgfsetbuttcap%
\pgfsetroundjoin%
\pgfsetlinewidth{1.505625pt}%
\definecolor{currentstroke}{rgb}{0.949020,0.372549,0.360784}%
\pgfsetstrokecolor{currentstroke}%
\pgfsetstrokeopacity{0.900000}%
\pgfsetdash{}{0pt}%
\pgfpathmoveto{\pgfqpoint{4.923318in}{0.700563in}}%
\pgfpathlineto{\pgfqpoint{4.923318in}{0.758738in}}%
\pgfusepath{stroke}%
\end{pgfscope}%
\begin{pgfscope}%
\pgfpathrectangle{\pgfqpoint{0.572918in}{0.553781in}}{\pgfqpoint{5.478282in}{2.095553in}}%
\pgfusepath{clip}%
\pgfsetbuttcap%
\pgfsetroundjoin%
\pgfsetlinewidth{1.505625pt}%
\definecolor{currentstroke}{rgb}{0.949020,0.372549,0.360784}%
\pgfsetstrokecolor{currentstroke}%
\pgfsetstrokeopacity{0.900000}%
\pgfsetdash{}{0pt}%
\pgfpathmoveto{\pgfqpoint{5.216275in}{0.675814in}}%
\pgfpathlineto{\pgfqpoint{5.216275in}{0.743578in}}%
\pgfusepath{stroke}%
\end{pgfscope}%
\begin{pgfscope}%
\pgfpathrectangle{\pgfqpoint{0.572918in}{0.553781in}}{\pgfqpoint{5.478282in}{2.095553in}}%
\pgfusepath{clip}%
\pgfsetbuttcap%
\pgfsetroundjoin%
\pgfsetlinewidth{1.505625pt}%
\definecolor{currentstroke}{rgb}{0.949020,0.372549,0.360784}%
\pgfsetstrokecolor{currentstroke}%
\pgfsetstrokeopacity{0.900000}%
\pgfsetdash{}{0pt}%
\pgfpathmoveto{\pgfqpoint{5.509231in}{0.688296in}}%
\pgfpathlineto{\pgfqpoint{5.509231in}{0.761210in}}%
\pgfusepath{stroke}%
\end{pgfscope}%
\begin{pgfscope}%
\pgfpathrectangle{\pgfqpoint{0.572918in}{0.553781in}}{\pgfqpoint{5.478282in}{2.095553in}}%
\pgfusepath{clip}%
\pgfsetbuttcap%
\pgfsetroundjoin%
\pgfsetlinewidth{1.505625pt}%
\definecolor{currentstroke}{rgb}{0.949020,0.372549,0.360784}%
\pgfsetstrokecolor{currentstroke}%
\pgfsetstrokeopacity{0.900000}%
\pgfsetdash{}{0pt}%
\pgfpathmoveto{\pgfqpoint{5.802187in}{0.720752in}}%
\pgfpathlineto{\pgfqpoint{5.802187in}{0.870775in}}%
\pgfusepath{stroke}%
\end{pgfscope}%
\begin{pgfscope}%
\pgfpathrectangle{\pgfqpoint{0.572918in}{0.553781in}}{\pgfqpoint{5.478282in}{2.095553in}}%
\pgfusepath{clip}%
\pgfsetbuttcap%
\pgfsetroundjoin%
\pgfsetlinewidth{1.505625pt}%
\definecolor{currentstroke}{rgb}{1.000000,0.819608,0.101961}%
\pgfsetstrokecolor{currentstroke}%
\pgfsetstrokeopacity{0.900000}%
\pgfsetdash{}{0pt}%
\pgfpathmoveto{\pgfqpoint{0.821931in}{1.877923in}}%
\pgfpathlineto{\pgfqpoint{0.821931in}{2.311417in}}%
\pgfusepath{stroke}%
\end{pgfscope}%
\begin{pgfscope}%
\pgfpathrectangle{\pgfqpoint{0.572918in}{0.553781in}}{\pgfqpoint{5.478282in}{2.095553in}}%
\pgfusepath{clip}%
\pgfsetbuttcap%
\pgfsetroundjoin%
\pgfsetlinewidth{1.505625pt}%
\definecolor{currentstroke}{rgb}{1.000000,0.819608,0.101961}%
\pgfsetstrokecolor{currentstroke}%
\pgfsetstrokeopacity{0.900000}%
\pgfsetdash{}{0pt}%
\pgfpathmoveto{\pgfqpoint{1.114887in}{1.906293in}}%
\pgfpathlineto{\pgfqpoint{1.114887in}{2.173612in}}%
\pgfusepath{stroke}%
\end{pgfscope}%
\begin{pgfscope}%
\pgfpathrectangle{\pgfqpoint{0.572918in}{0.553781in}}{\pgfqpoint{5.478282in}{2.095553in}}%
\pgfusepath{clip}%
\pgfsetbuttcap%
\pgfsetroundjoin%
\pgfsetlinewidth{1.505625pt}%
\definecolor{currentstroke}{rgb}{1.000000,0.819608,0.101961}%
\pgfsetstrokecolor{currentstroke}%
\pgfsetstrokeopacity{0.900000}%
\pgfsetdash{}{0pt}%
\pgfpathmoveto{\pgfqpoint{1.407844in}{2.004423in}}%
\pgfpathlineto{\pgfqpoint{1.407844in}{2.161887in}}%
\pgfusepath{stroke}%
\end{pgfscope}%
\begin{pgfscope}%
\pgfpathrectangle{\pgfqpoint{0.572918in}{0.553781in}}{\pgfqpoint{5.478282in}{2.095553in}}%
\pgfusepath{clip}%
\pgfsetbuttcap%
\pgfsetroundjoin%
\pgfsetlinewidth{1.505625pt}%
\definecolor{currentstroke}{rgb}{1.000000,0.819608,0.101961}%
\pgfsetstrokecolor{currentstroke}%
\pgfsetstrokeopacity{0.900000}%
\pgfsetdash{}{0pt}%
\pgfpathmoveto{\pgfqpoint{1.700800in}{2.120581in}}%
\pgfpathlineto{\pgfqpoint{1.700800in}{2.219266in}}%
\pgfusepath{stroke}%
\end{pgfscope}%
\begin{pgfscope}%
\pgfpathrectangle{\pgfqpoint{0.572918in}{0.553781in}}{\pgfqpoint{5.478282in}{2.095553in}}%
\pgfusepath{clip}%
\pgfsetbuttcap%
\pgfsetroundjoin%
\pgfsetlinewidth{1.505625pt}%
\definecolor{currentstroke}{rgb}{1.000000,0.819608,0.101961}%
\pgfsetstrokecolor{currentstroke}%
\pgfsetstrokeopacity{0.900000}%
\pgfsetdash{}{0pt}%
\pgfpathmoveto{\pgfqpoint{1.993756in}{2.186343in}}%
\pgfpathlineto{\pgfqpoint{1.993756in}{2.269515in}}%
\pgfusepath{stroke}%
\end{pgfscope}%
\begin{pgfscope}%
\pgfpathrectangle{\pgfqpoint{0.572918in}{0.553781in}}{\pgfqpoint{5.478282in}{2.095553in}}%
\pgfusepath{clip}%
\pgfsetbuttcap%
\pgfsetroundjoin%
\pgfsetlinewidth{1.505625pt}%
\definecolor{currentstroke}{rgb}{1.000000,0.819608,0.101961}%
\pgfsetstrokecolor{currentstroke}%
\pgfsetstrokeopacity{0.900000}%
\pgfsetdash{}{0pt}%
\pgfpathmoveto{\pgfqpoint{2.286712in}{2.115089in}}%
\pgfpathlineto{\pgfqpoint{2.286712in}{2.180838in}}%
\pgfusepath{stroke}%
\end{pgfscope}%
\begin{pgfscope}%
\pgfpathrectangle{\pgfqpoint{0.572918in}{0.553781in}}{\pgfqpoint{5.478282in}{2.095553in}}%
\pgfusepath{clip}%
\pgfsetbuttcap%
\pgfsetroundjoin%
\pgfsetlinewidth{1.505625pt}%
\definecolor{currentstroke}{rgb}{1.000000,0.819608,0.101961}%
\pgfsetstrokecolor{currentstroke}%
\pgfsetstrokeopacity{0.900000}%
\pgfsetdash{}{0pt}%
\pgfpathmoveto{\pgfqpoint{2.579669in}{1.981400in}}%
\pgfpathlineto{\pgfqpoint{2.579669in}{2.034027in}}%
\pgfusepath{stroke}%
\end{pgfscope}%
\begin{pgfscope}%
\pgfpathrectangle{\pgfqpoint{0.572918in}{0.553781in}}{\pgfqpoint{5.478282in}{2.095553in}}%
\pgfusepath{clip}%
\pgfsetbuttcap%
\pgfsetroundjoin%
\pgfsetlinewidth{1.505625pt}%
\definecolor{currentstroke}{rgb}{1.000000,0.819608,0.101961}%
\pgfsetstrokecolor{currentstroke}%
\pgfsetstrokeopacity{0.900000}%
\pgfsetdash{}{0pt}%
\pgfpathmoveto{\pgfqpoint{2.872625in}{1.772145in}}%
\pgfpathlineto{\pgfqpoint{2.872625in}{1.818632in}}%
\pgfusepath{stroke}%
\end{pgfscope}%
\begin{pgfscope}%
\pgfpathrectangle{\pgfqpoint{0.572918in}{0.553781in}}{\pgfqpoint{5.478282in}{2.095553in}}%
\pgfusepath{clip}%
\pgfsetbuttcap%
\pgfsetroundjoin%
\pgfsetlinewidth{1.505625pt}%
\definecolor{currentstroke}{rgb}{1.000000,0.819608,0.101961}%
\pgfsetstrokecolor{currentstroke}%
\pgfsetstrokeopacity{0.900000}%
\pgfsetdash{}{0pt}%
\pgfpathmoveto{\pgfqpoint{3.165581in}{1.578334in}}%
\pgfpathlineto{\pgfqpoint{3.165581in}{1.625665in}}%
\pgfusepath{stroke}%
\end{pgfscope}%
\begin{pgfscope}%
\pgfpathrectangle{\pgfqpoint{0.572918in}{0.553781in}}{\pgfqpoint{5.478282in}{2.095553in}}%
\pgfusepath{clip}%
\pgfsetbuttcap%
\pgfsetroundjoin%
\pgfsetlinewidth{1.505625pt}%
\definecolor{currentstroke}{rgb}{1.000000,0.819608,0.101961}%
\pgfsetstrokecolor{currentstroke}%
\pgfsetstrokeopacity{0.900000}%
\pgfsetdash{}{0pt}%
\pgfpathmoveto{\pgfqpoint{3.458537in}{1.398995in}}%
\pgfpathlineto{\pgfqpoint{3.458537in}{1.444708in}}%
\pgfusepath{stroke}%
\end{pgfscope}%
\begin{pgfscope}%
\pgfpathrectangle{\pgfqpoint{0.572918in}{0.553781in}}{\pgfqpoint{5.478282in}{2.095553in}}%
\pgfusepath{clip}%
\pgfsetbuttcap%
\pgfsetroundjoin%
\pgfsetlinewidth{1.505625pt}%
\definecolor{currentstroke}{rgb}{1.000000,0.819608,0.101961}%
\pgfsetstrokecolor{currentstroke}%
\pgfsetstrokeopacity{0.900000}%
\pgfsetdash{}{0pt}%
\pgfpathmoveto{\pgfqpoint{3.751494in}{1.203698in}}%
\pgfpathlineto{\pgfqpoint{3.751494in}{1.244319in}}%
\pgfusepath{stroke}%
\end{pgfscope}%
\begin{pgfscope}%
\pgfpathrectangle{\pgfqpoint{0.572918in}{0.553781in}}{\pgfqpoint{5.478282in}{2.095553in}}%
\pgfusepath{clip}%
\pgfsetbuttcap%
\pgfsetroundjoin%
\pgfsetlinewidth{1.505625pt}%
\definecolor{currentstroke}{rgb}{1.000000,0.819608,0.101961}%
\pgfsetstrokecolor{currentstroke}%
\pgfsetstrokeopacity{0.900000}%
\pgfsetdash{}{0pt}%
\pgfpathmoveto{\pgfqpoint{4.044450in}{1.060931in}}%
\pgfpathlineto{\pgfqpoint{4.044450in}{1.105462in}}%
\pgfusepath{stroke}%
\end{pgfscope}%
\begin{pgfscope}%
\pgfpathrectangle{\pgfqpoint{0.572918in}{0.553781in}}{\pgfqpoint{5.478282in}{2.095553in}}%
\pgfusepath{clip}%
\pgfsetbuttcap%
\pgfsetroundjoin%
\pgfsetlinewidth{1.505625pt}%
\definecolor{currentstroke}{rgb}{1.000000,0.819608,0.101961}%
\pgfsetstrokecolor{currentstroke}%
\pgfsetstrokeopacity{0.900000}%
\pgfsetdash{}{0pt}%
\pgfpathmoveto{\pgfqpoint{4.337406in}{0.929979in}}%
\pgfpathlineto{\pgfqpoint{4.337406in}{0.970460in}}%
\pgfusepath{stroke}%
\end{pgfscope}%
\begin{pgfscope}%
\pgfpathrectangle{\pgfqpoint{0.572918in}{0.553781in}}{\pgfqpoint{5.478282in}{2.095553in}}%
\pgfusepath{clip}%
\pgfsetbuttcap%
\pgfsetroundjoin%
\pgfsetlinewidth{1.505625pt}%
\definecolor{currentstroke}{rgb}{1.000000,0.819608,0.101961}%
\pgfsetstrokecolor{currentstroke}%
\pgfsetstrokeopacity{0.900000}%
\pgfsetdash{}{0pt}%
\pgfpathmoveto{\pgfqpoint{4.630362in}{0.869302in}}%
\pgfpathlineto{\pgfqpoint{4.630362in}{0.909914in}}%
\pgfusepath{stroke}%
\end{pgfscope}%
\begin{pgfscope}%
\pgfpathrectangle{\pgfqpoint{0.572918in}{0.553781in}}{\pgfqpoint{5.478282in}{2.095553in}}%
\pgfusepath{clip}%
\pgfsetbuttcap%
\pgfsetroundjoin%
\pgfsetlinewidth{1.505625pt}%
\definecolor{currentstroke}{rgb}{1.000000,0.819608,0.101961}%
\pgfsetstrokecolor{currentstroke}%
\pgfsetstrokeopacity{0.900000}%
\pgfsetdash{}{0pt}%
\pgfpathmoveto{\pgfqpoint{4.923318in}{0.801982in}}%
\pgfpathlineto{\pgfqpoint{4.923318in}{0.858479in}}%
\pgfusepath{stroke}%
\end{pgfscope}%
\begin{pgfscope}%
\pgfpathrectangle{\pgfqpoint{0.572918in}{0.553781in}}{\pgfqpoint{5.478282in}{2.095553in}}%
\pgfusepath{clip}%
\pgfsetbuttcap%
\pgfsetroundjoin%
\pgfsetlinewidth{1.505625pt}%
\definecolor{currentstroke}{rgb}{1.000000,0.819608,0.101961}%
\pgfsetstrokecolor{currentstroke}%
\pgfsetstrokeopacity{0.900000}%
\pgfsetdash{}{0pt}%
\pgfpathmoveto{\pgfqpoint{5.216275in}{0.803049in}}%
\pgfpathlineto{\pgfqpoint{5.216275in}{0.881126in}}%
\pgfusepath{stroke}%
\end{pgfscope}%
\begin{pgfscope}%
\pgfpathrectangle{\pgfqpoint{0.572918in}{0.553781in}}{\pgfqpoint{5.478282in}{2.095553in}}%
\pgfusepath{clip}%
\pgfsetbuttcap%
\pgfsetroundjoin%
\pgfsetlinewidth{1.505625pt}%
\definecolor{currentstroke}{rgb}{1.000000,0.819608,0.101961}%
\pgfsetstrokecolor{currentstroke}%
\pgfsetstrokeopacity{0.900000}%
\pgfsetdash{}{0pt}%
\pgfpathmoveto{\pgfqpoint{5.509231in}{0.771955in}}%
\pgfpathlineto{\pgfqpoint{5.509231in}{0.867379in}}%
\pgfusepath{stroke}%
\end{pgfscope}%
\begin{pgfscope}%
\pgfpathrectangle{\pgfqpoint{0.572918in}{0.553781in}}{\pgfqpoint{5.478282in}{2.095553in}}%
\pgfusepath{clip}%
\pgfsetbuttcap%
\pgfsetroundjoin%
\pgfsetlinewidth{1.505625pt}%
\definecolor{currentstroke}{rgb}{1.000000,0.819608,0.101961}%
\pgfsetstrokecolor{currentstroke}%
\pgfsetstrokeopacity{0.900000}%
\pgfsetdash{}{0pt}%
\pgfpathmoveto{\pgfqpoint{5.802187in}{0.822892in}}%
\pgfpathlineto{\pgfqpoint{5.802187in}{0.945042in}}%
\pgfusepath{stroke}%
\end{pgfscope}%
\begin{pgfscope}%
\pgfpathrectangle{\pgfqpoint{0.572918in}{0.553781in}}{\pgfqpoint{5.478282in}{2.095553in}}%
\pgfusepath{clip}%
\pgfsetbuttcap%
\pgfsetroundjoin%
\definecolor{currentfill}{rgb}{0.313725,0.317647,0.309804}%
\pgfsetfillcolor{currentfill}%
\pgfsetfillopacity{0.900000}%
\pgfsetlinewidth{1.003750pt}%
\definecolor{currentstroke}{rgb}{0.313725,0.317647,0.309804}%
\pgfsetstrokecolor{currentstroke}%
\pgfsetstrokeopacity{0.900000}%
\pgfsetdash{}{0pt}%
\pgfsys@defobject{currentmarker}{\pgfqpoint{-0.013889in}{-0.000000in}}{\pgfqpoint{0.013889in}{0.000000in}}{%
\pgfpathmoveto{\pgfqpoint{0.013889in}{-0.000000in}}%
\pgfpathlineto{\pgfqpoint{-0.013889in}{0.000000in}}%
\pgfusepath{stroke,fill}%
}%
\begin{pgfscope}%
\pgfsys@transformshift{0.821931in}{1.853223in}%
\pgfsys@useobject{currentmarker}{}%
\end{pgfscope}%
\begin{pgfscope}%
\pgfsys@transformshift{1.114887in}{1.863076in}%
\pgfsys@useobject{currentmarker}{}%
\end{pgfscope}%
\begin{pgfscope}%
\pgfsys@transformshift{1.407844in}{1.870788in}%
\pgfsys@useobject{currentmarker}{}%
\end{pgfscope}%
\begin{pgfscope}%
\pgfsys@transformshift{1.700800in}{1.955754in}%
\pgfsys@useobject{currentmarker}{}%
\end{pgfscope}%
\begin{pgfscope}%
\pgfsys@transformshift{1.993756in}{1.977838in}%
\pgfsys@useobject{currentmarker}{}%
\end{pgfscope}%
\begin{pgfscope}%
\pgfsys@transformshift{2.286712in}{1.963330in}%
\pgfsys@useobject{currentmarker}{}%
\end{pgfscope}%
\begin{pgfscope}%
\pgfsys@transformshift{2.579669in}{1.786495in}%
\pgfsys@useobject{currentmarker}{}%
\end{pgfscope}%
\begin{pgfscope}%
\pgfsys@transformshift{2.872625in}{1.653168in}%
\pgfsys@useobject{currentmarker}{}%
\end{pgfscope}%
\begin{pgfscope}%
\pgfsys@transformshift{3.165581in}{1.447782in}%
\pgfsys@useobject{currentmarker}{}%
\end{pgfscope}%
\begin{pgfscope}%
\pgfsys@transformshift{3.458537in}{1.281419in}%
\pgfsys@useobject{currentmarker}{}%
\end{pgfscope}%
\begin{pgfscope}%
\pgfsys@transformshift{3.751494in}{1.084302in}%
\pgfsys@useobject{currentmarker}{}%
\end{pgfscope}%
\begin{pgfscope}%
\pgfsys@transformshift{4.044450in}{0.969110in}%
\pgfsys@useobject{currentmarker}{}%
\end{pgfscope}%
\begin{pgfscope}%
\pgfsys@transformshift{4.337406in}{0.843489in}%
\pgfsys@useobject{currentmarker}{}%
\end{pgfscope}%
\begin{pgfscope}%
\pgfsys@transformshift{4.630362in}{0.764754in}%
\pgfsys@useobject{currentmarker}{}%
\end{pgfscope}%
\begin{pgfscope}%
\pgfsys@transformshift{4.923318in}{0.698590in}%
\pgfsys@useobject{currentmarker}{}%
\end{pgfscope}%
\begin{pgfscope}%
\pgfsys@transformshift{5.216275in}{0.690584in}%
\pgfsys@useobject{currentmarker}{}%
\end{pgfscope}%
\begin{pgfscope}%
\pgfsys@transformshift{5.509231in}{0.649033in}%
\pgfsys@useobject{currentmarker}{}%
\end{pgfscope}%
\begin{pgfscope}%
\pgfsys@transformshift{5.802187in}{0.654496in}%
\pgfsys@useobject{currentmarker}{}%
\end{pgfscope}%
\end{pgfscope}%
\begin{pgfscope}%
\pgfpathrectangle{\pgfqpoint{0.572918in}{0.553781in}}{\pgfqpoint{5.478282in}{2.095553in}}%
\pgfusepath{clip}%
\pgfsetbuttcap%
\pgfsetroundjoin%
\definecolor{currentfill}{rgb}{0.313725,0.317647,0.309804}%
\pgfsetfillcolor{currentfill}%
\pgfsetfillopacity{0.900000}%
\pgfsetlinewidth{1.003750pt}%
\definecolor{currentstroke}{rgb}{0.313725,0.317647,0.309804}%
\pgfsetstrokecolor{currentstroke}%
\pgfsetstrokeopacity{0.900000}%
\pgfsetdash{}{0pt}%
\pgfsys@defobject{currentmarker}{\pgfqpoint{-0.013889in}{-0.000000in}}{\pgfqpoint{0.013889in}{0.000000in}}{%
\pgfpathmoveto{\pgfqpoint{0.013889in}{-0.000000in}}%
\pgfpathlineto{\pgfqpoint{-0.013889in}{0.000000in}}%
\pgfusepath{stroke,fill}%
}%
\begin{pgfscope}%
\pgfsys@transformshift{0.821931in}{2.266559in}%
\pgfsys@useobject{currentmarker}{}%
\end{pgfscope}%
\begin{pgfscope}%
\pgfsys@transformshift{1.114887in}{2.062997in}%
\pgfsys@useobject{currentmarker}{}%
\end{pgfscope}%
\begin{pgfscope}%
\pgfsys@transformshift{1.407844in}{2.001867in}%
\pgfsys@useobject{currentmarker}{}%
\end{pgfscope}%
\begin{pgfscope}%
\pgfsys@transformshift{1.700800in}{2.056646in}%
\pgfsys@useobject{currentmarker}{}%
\end{pgfscope}%
\begin{pgfscope}%
\pgfsys@transformshift{1.993756in}{2.056402in}%
\pgfsys@useobject{currentmarker}{}%
\end{pgfscope}%
\begin{pgfscope}%
\pgfsys@transformshift{2.286712in}{2.032372in}%
\pgfsys@useobject{currentmarker}{}%
\end{pgfscope}%
\begin{pgfscope}%
\pgfsys@transformshift{2.579669in}{1.838736in}%
\pgfsys@useobject{currentmarker}{}%
\end{pgfscope}%
\begin{pgfscope}%
\pgfsys@transformshift{2.872625in}{1.704447in}%
\pgfsys@useobject{currentmarker}{}%
\end{pgfscope}%
\begin{pgfscope}%
\pgfsys@transformshift{3.165581in}{1.488545in}%
\pgfsys@useobject{currentmarker}{}%
\end{pgfscope}%
\begin{pgfscope}%
\pgfsys@transformshift{3.458537in}{1.320106in}%
\pgfsys@useobject{currentmarker}{}%
\end{pgfscope}%
\begin{pgfscope}%
\pgfsys@transformshift{3.751494in}{1.120020in}%
\pgfsys@useobject{currentmarker}{}%
\end{pgfscope}%
\begin{pgfscope}%
\pgfsys@transformshift{4.044450in}{1.008389in}%
\pgfsys@useobject{currentmarker}{}%
\end{pgfscope}%
\begin{pgfscope}%
\pgfsys@transformshift{4.337406in}{0.881250in}%
\pgfsys@useobject{currentmarker}{}%
\end{pgfscope}%
\begin{pgfscope}%
\pgfsys@transformshift{4.630362in}{0.804011in}%
\pgfsys@useobject{currentmarker}{}%
\end{pgfscope}%
\begin{pgfscope}%
\pgfsys@transformshift{4.923318in}{0.739377in}%
\pgfsys@useobject{currentmarker}{}%
\end{pgfscope}%
\begin{pgfscope}%
\pgfsys@transformshift{5.216275in}{0.744443in}%
\pgfsys@useobject{currentmarker}{}%
\end{pgfscope}%
\begin{pgfscope}%
\pgfsys@transformshift{5.509231in}{0.704159in}%
\pgfsys@useobject{currentmarker}{}%
\end{pgfscope}%
\begin{pgfscope}%
\pgfsys@transformshift{5.802187in}{0.735595in}%
\pgfsys@useobject{currentmarker}{}%
\end{pgfscope}%
\end{pgfscope}%
\begin{pgfscope}%
\pgfpathrectangle{\pgfqpoint{0.572918in}{0.553781in}}{\pgfqpoint{5.478282in}{2.095553in}}%
\pgfusepath{clip}%
\pgfsetbuttcap%
\pgfsetroundjoin%
\definecolor{currentfill}{rgb}{0.949020,0.372549,0.360784}%
\pgfsetfillcolor{currentfill}%
\pgfsetfillopacity{0.900000}%
\pgfsetlinewidth{1.003750pt}%
\definecolor{currentstroke}{rgb}{0.949020,0.372549,0.360784}%
\pgfsetstrokecolor{currentstroke}%
\pgfsetstrokeopacity{0.900000}%
\pgfsetdash{}{0pt}%
\pgfsys@defobject{currentmarker}{\pgfqpoint{-0.013889in}{-0.000000in}}{\pgfqpoint{0.013889in}{0.000000in}}{%
\pgfpathmoveto{\pgfqpoint{0.013889in}{-0.000000in}}%
\pgfpathlineto{\pgfqpoint{-0.013889in}{0.000000in}}%
\pgfusepath{stroke,fill}%
}%
\begin{pgfscope}%
\pgfsys@transformshift{0.821931in}{2.009052in}%
\pgfsys@useobject{currentmarker}{}%
\end{pgfscope}%
\begin{pgfscope}%
\pgfsys@transformshift{1.114887in}{2.065386in}%
\pgfsys@useobject{currentmarker}{}%
\end{pgfscope}%
\begin{pgfscope}%
\pgfsys@transformshift{1.407844in}{1.964859in}%
\pgfsys@useobject{currentmarker}{}%
\end{pgfscope}%
\begin{pgfscope}%
\pgfsys@transformshift{1.700800in}{2.068923in}%
\pgfsys@useobject{currentmarker}{}%
\end{pgfscope}%
\begin{pgfscope}%
\pgfsys@transformshift{1.993756in}{2.036912in}%
\pgfsys@useobject{currentmarker}{}%
\end{pgfscope}%
\begin{pgfscope}%
\pgfsys@transformshift{2.286712in}{1.963741in}%
\pgfsys@useobject{currentmarker}{}%
\end{pgfscope}%
\begin{pgfscope}%
\pgfsys@transformshift{2.579669in}{1.849130in}%
\pgfsys@useobject{currentmarker}{}%
\end{pgfscope}%
\begin{pgfscope}%
\pgfsys@transformshift{2.872625in}{1.688203in}%
\pgfsys@useobject{currentmarker}{}%
\end{pgfscope}%
\begin{pgfscope}%
\pgfsys@transformshift{3.165581in}{1.470520in}%
\pgfsys@useobject{currentmarker}{}%
\end{pgfscope}%
\begin{pgfscope}%
\pgfsys@transformshift{3.458537in}{1.317661in}%
\pgfsys@useobject{currentmarker}{}%
\end{pgfscope}%
\begin{pgfscope}%
\pgfsys@transformshift{3.751494in}{1.100790in}%
\pgfsys@useobject{currentmarker}{}%
\end{pgfscope}%
\begin{pgfscope}%
\pgfsys@transformshift{4.044450in}{0.950593in}%
\pgfsys@useobject{currentmarker}{}%
\end{pgfscope}%
\begin{pgfscope}%
\pgfsys@transformshift{4.337406in}{0.840002in}%
\pgfsys@useobject{currentmarker}{}%
\end{pgfscope}%
\begin{pgfscope}%
\pgfsys@transformshift{4.630362in}{0.755124in}%
\pgfsys@useobject{currentmarker}{}%
\end{pgfscope}%
\begin{pgfscope}%
\pgfsys@transformshift{4.923318in}{0.700563in}%
\pgfsys@useobject{currentmarker}{}%
\end{pgfscope}%
\begin{pgfscope}%
\pgfsys@transformshift{5.216275in}{0.675814in}%
\pgfsys@useobject{currentmarker}{}%
\end{pgfscope}%
\begin{pgfscope}%
\pgfsys@transformshift{5.509231in}{0.688296in}%
\pgfsys@useobject{currentmarker}{}%
\end{pgfscope}%
\begin{pgfscope}%
\pgfsys@transformshift{5.802187in}{0.720752in}%
\pgfsys@useobject{currentmarker}{}%
\end{pgfscope}%
\end{pgfscope}%
\begin{pgfscope}%
\pgfpathrectangle{\pgfqpoint{0.572918in}{0.553781in}}{\pgfqpoint{5.478282in}{2.095553in}}%
\pgfusepath{clip}%
\pgfsetbuttcap%
\pgfsetroundjoin%
\definecolor{currentfill}{rgb}{0.949020,0.372549,0.360784}%
\pgfsetfillcolor{currentfill}%
\pgfsetfillopacity{0.900000}%
\pgfsetlinewidth{1.003750pt}%
\definecolor{currentstroke}{rgb}{0.949020,0.372549,0.360784}%
\pgfsetstrokecolor{currentstroke}%
\pgfsetstrokeopacity{0.900000}%
\pgfsetdash{}{0pt}%
\pgfsys@defobject{currentmarker}{\pgfqpoint{-0.013889in}{-0.000000in}}{\pgfqpoint{0.013889in}{0.000000in}}{%
\pgfpathmoveto{\pgfqpoint{0.013889in}{-0.000000in}}%
\pgfpathlineto{\pgfqpoint{-0.013889in}{0.000000in}}%
\pgfusepath{stroke,fill}%
}%
\begin{pgfscope}%
\pgfsys@transformshift{0.821931in}{2.554081in}%
\pgfsys@useobject{currentmarker}{}%
\end{pgfscope}%
\begin{pgfscope}%
\pgfsys@transformshift{1.114887in}{2.323778in}%
\pgfsys@useobject{currentmarker}{}%
\end{pgfscope}%
\begin{pgfscope}%
\pgfsys@transformshift{1.407844in}{2.137199in}%
\pgfsys@useobject{currentmarker}{}%
\end{pgfscope}%
\begin{pgfscope}%
\pgfsys@transformshift{1.700800in}{2.171082in}%
\pgfsys@useobject{currentmarker}{}%
\end{pgfscope}%
\begin{pgfscope}%
\pgfsys@transformshift{1.993756in}{2.112253in}%
\pgfsys@useobject{currentmarker}{}%
\end{pgfscope}%
\begin{pgfscope}%
\pgfsys@transformshift{2.286712in}{2.032198in}%
\pgfsys@useobject{currentmarker}{}%
\end{pgfscope}%
\begin{pgfscope}%
\pgfsys@transformshift{2.579669in}{1.900667in}%
\pgfsys@useobject{currentmarker}{}%
\end{pgfscope}%
\begin{pgfscope}%
\pgfsys@transformshift{2.872625in}{1.735750in}%
\pgfsys@useobject{currentmarker}{}%
\end{pgfscope}%
\begin{pgfscope}%
\pgfsys@transformshift{3.165581in}{1.518505in}%
\pgfsys@useobject{currentmarker}{}%
\end{pgfscope}%
\begin{pgfscope}%
\pgfsys@transformshift{3.458537in}{1.361766in}%
\pgfsys@useobject{currentmarker}{}%
\end{pgfscope}%
\begin{pgfscope}%
\pgfsys@transformshift{3.751494in}{1.143692in}%
\pgfsys@useobject{currentmarker}{}%
\end{pgfscope}%
\begin{pgfscope}%
\pgfsys@transformshift{4.044450in}{0.988447in}%
\pgfsys@useobject{currentmarker}{}%
\end{pgfscope}%
\begin{pgfscope}%
\pgfsys@transformshift{4.337406in}{0.883678in}%
\pgfsys@useobject{currentmarker}{}%
\end{pgfscope}%
\begin{pgfscope}%
\pgfsys@transformshift{4.630362in}{0.801728in}%
\pgfsys@useobject{currentmarker}{}%
\end{pgfscope}%
\begin{pgfscope}%
\pgfsys@transformshift{4.923318in}{0.758738in}%
\pgfsys@useobject{currentmarker}{}%
\end{pgfscope}%
\begin{pgfscope}%
\pgfsys@transformshift{5.216275in}{0.743578in}%
\pgfsys@useobject{currentmarker}{}%
\end{pgfscope}%
\begin{pgfscope}%
\pgfsys@transformshift{5.509231in}{0.761210in}%
\pgfsys@useobject{currentmarker}{}%
\end{pgfscope}%
\begin{pgfscope}%
\pgfsys@transformshift{5.802187in}{0.870775in}%
\pgfsys@useobject{currentmarker}{}%
\end{pgfscope}%
\end{pgfscope}%
\begin{pgfscope}%
\pgfpathrectangle{\pgfqpoint{0.572918in}{0.553781in}}{\pgfqpoint{5.478282in}{2.095553in}}%
\pgfusepath{clip}%
\pgfsetbuttcap%
\pgfsetroundjoin%
\definecolor{currentfill}{rgb}{1.000000,0.819608,0.101961}%
\pgfsetfillcolor{currentfill}%
\pgfsetfillopacity{0.900000}%
\pgfsetlinewidth{1.003750pt}%
\definecolor{currentstroke}{rgb}{1.000000,0.819608,0.101961}%
\pgfsetstrokecolor{currentstroke}%
\pgfsetstrokeopacity{0.900000}%
\pgfsetdash{}{0pt}%
\pgfsys@defobject{currentmarker}{\pgfqpoint{-0.013889in}{-0.000000in}}{\pgfqpoint{0.013889in}{0.000000in}}{%
\pgfpathmoveto{\pgfqpoint{0.013889in}{-0.000000in}}%
\pgfpathlineto{\pgfqpoint{-0.013889in}{0.000000in}}%
\pgfusepath{stroke,fill}%
}%
\begin{pgfscope}%
\pgfsys@transformshift{0.821931in}{1.877923in}%
\pgfsys@useobject{currentmarker}{}%
\end{pgfscope}%
\begin{pgfscope}%
\pgfsys@transformshift{1.114887in}{1.906293in}%
\pgfsys@useobject{currentmarker}{}%
\end{pgfscope}%
\begin{pgfscope}%
\pgfsys@transformshift{1.407844in}{2.004423in}%
\pgfsys@useobject{currentmarker}{}%
\end{pgfscope}%
\begin{pgfscope}%
\pgfsys@transformshift{1.700800in}{2.120581in}%
\pgfsys@useobject{currentmarker}{}%
\end{pgfscope}%
\begin{pgfscope}%
\pgfsys@transformshift{1.993756in}{2.186343in}%
\pgfsys@useobject{currentmarker}{}%
\end{pgfscope}%
\begin{pgfscope}%
\pgfsys@transformshift{2.286712in}{2.115089in}%
\pgfsys@useobject{currentmarker}{}%
\end{pgfscope}%
\begin{pgfscope}%
\pgfsys@transformshift{2.579669in}{1.981400in}%
\pgfsys@useobject{currentmarker}{}%
\end{pgfscope}%
\begin{pgfscope}%
\pgfsys@transformshift{2.872625in}{1.772145in}%
\pgfsys@useobject{currentmarker}{}%
\end{pgfscope}%
\begin{pgfscope}%
\pgfsys@transformshift{3.165581in}{1.578334in}%
\pgfsys@useobject{currentmarker}{}%
\end{pgfscope}%
\begin{pgfscope}%
\pgfsys@transformshift{3.458537in}{1.398995in}%
\pgfsys@useobject{currentmarker}{}%
\end{pgfscope}%
\begin{pgfscope}%
\pgfsys@transformshift{3.751494in}{1.203698in}%
\pgfsys@useobject{currentmarker}{}%
\end{pgfscope}%
\begin{pgfscope}%
\pgfsys@transformshift{4.044450in}{1.060931in}%
\pgfsys@useobject{currentmarker}{}%
\end{pgfscope}%
\begin{pgfscope}%
\pgfsys@transformshift{4.337406in}{0.929979in}%
\pgfsys@useobject{currentmarker}{}%
\end{pgfscope}%
\begin{pgfscope}%
\pgfsys@transformshift{4.630362in}{0.869302in}%
\pgfsys@useobject{currentmarker}{}%
\end{pgfscope}%
\begin{pgfscope}%
\pgfsys@transformshift{4.923318in}{0.801982in}%
\pgfsys@useobject{currentmarker}{}%
\end{pgfscope}%
\begin{pgfscope}%
\pgfsys@transformshift{5.216275in}{0.803049in}%
\pgfsys@useobject{currentmarker}{}%
\end{pgfscope}%
\begin{pgfscope}%
\pgfsys@transformshift{5.509231in}{0.771955in}%
\pgfsys@useobject{currentmarker}{}%
\end{pgfscope}%
\begin{pgfscope}%
\pgfsys@transformshift{5.802187in}{0.822892in}%
\pgfsys@useobject{currentmarker}{}%
\end{pgfscope}%
\end{pgfscope}%
\begin{pgfscope}%
\pgfpathrectangle{\pgfqpoint{0.572918in}{0.553781in}}{\pgfqpoint{5.478282in}{2.095553in}}%
\pgfusepath{clip}%
\pgfsetbuttcap%
\pgfsetroundjoin%
\definecolor{currentfill}{rgb}{1.000000,0.819608,0.101961}%
\pgfsetfillcolor{currentfill}%
\pgfsetfillopacity{0.900000}%
\pgfsetlinewidth{1.003750pt}%
\definecolor{currentstroke}{rgb}{1.000000,0.819608,0.101961}%
\pgfsetstrokecolor{currentstroke}%
\pgfsetstrokeopacity{0.900000}%
\pgfsetdash{}{0pt}%
\pgfsys@defobject{currentmarker}{\pgfqpoint{-0.013889in}{-0.000000in}}{\pgfqpoint{0.013889in}{0.000000in}}{%
\pgfpathmoveto{\pgfqpoint{0.013889in}{-0.000000in}}%
\pgfpathlineto{\pgfqpoint{-0.013889in}{0.000000in}}%
\pgfusepath{stroke,fill}%
}%
\begin{pgfscope}%
\pgfsys@transformshift{0.821931in}{2.311417in}%
\pgfsys@useobject{currentmarker}{}%
\end{pgfscope}%
\begin{pgfscope}%
\pgfsys@transformshift{1.114887in}{2.173612in}%
\pgfsys@useobject{currentmarker}{}%
\end{pgfscope}%
\begin{pgfscope}%
\pgfsys@transformshift{1.407844in}{2.161887in}%
\pgfsys@useobject{currentmarker}{}%
\end{pgfscope}%
\begin{pgfscope}%
\pgfsys@transformshift{1.700800in}{2.219266in}%
\pgfsys@useobject{currentmarker}{}%
\end{pgfscope}%
\begin{pgfscope}%
\pgfsys@transformshift{1.993756in}{2.269515in}%
\pgfsys@useobject{currentmarker}{}%
\end{pgfscope}%
\begin{pgfscope}%
\pgfsys@transformshift{2.286712in}{2.180838in}%
\pgfsys@useobject{currentmarker}{}%
\end{pgfscope}%
\begin{pgfscope}%
\pgfsys@transformshift{2.579669in}{2.034027in}%
\pgfsys@useobject{currentmarker}{}%
\end{pgfscope}%
\begin{pgfscope}%
\pgfsys@transformshift{2.872625in}{1.818632in}%
\pgfsys@useobject{currentmarker}{}%
\end{pgfscope}%
\begin{pgfscope}%
\pgfsys@transformshift{3.165581in}{1.625665in}%
\pgfsys@useobject{currentmarker}{}%
\end{pgfscope}%
\begin{pgfscope}%
\pgfsys@transformshift{3.458537in}{1.444708in}%
\pgfsys@useobject{currentmarker}{}%
\end{pgfscope}%
\begin{pgfscope}%
\pgfsys@transformshift{3.751494in}{1.244319in}%
\pgfsys@useobject{currentmarker}{}%
\end{pgfscope}%
\begin{pgfscope}%
\pgfsys@transformshift{4.044450in}{1.105462in}%
\pgfsys@useobject{currentmarker}{}%
\end{pgfscope}%
\begin{pgfscope}%
\pgfsys@transformshift{4.337406in}{0.970460in}%
\pgfsys@useobject{currentmarker}{}%
\end{pgfscope}%
\begin{pgfscope}%
\pgfsys@transformshift{4.630362in}{0.909914in}%
\pgfsys@useobject{currentmarker}{}%
\end{pgfscope}%
\begin{pgfscope}%
\pgfsys@transformshift{4.923318in}{0.858479in}%
\pgfsys@useobject{currentmarker}{}%
\end{pgfscope}%
\begin{pgfscope}%
\pgfsys@transformshift{5.216275in}{0.881126in}%
\pgfsys@useobject{currentmarker}{}%
\end{pgfscope}%
\begin{pgfscope}%
\pgfsys@transformshift{5.509231in}{0.867379in}%
\pgfsys@useobject{currentmarker}{}%
\end{pgfscope}%
\begin{pgfscope}%
\pgfsys@transformshift{5.802187in}{0.945042in}%
\pgfsys@useobject{currentmarker}{}%
\end{pgfscope}%
\end{pgfscope}%
\begin{pgfscope}%
\pgfpathrectangle{\pgfqpoint{0.572918in}{0.553781in}}{\pgfqpoint{5.478282in}{2.095553in}}%
\pgfusepath{clip}%
\pgfsetrectcap%
\pgfsetroundjoin%
\pgfsetlinewidth{1.505625pt}%
\definecolor{currentstroke}{rgb}{0.313725,0.317647,0.309804}%
\pgfsetstrokecolor{currentstroke}%
\pgfsetstrokeopacity{0.900000}%
\pgfsetdash{}{0pt}%
\pgfpathmoveto{\pgfqpoint{0.821931in}{2.027005in}}%
\pgfpathlineto{\pgfqpoint{1.114887in}{1.947499in}}%
\pgfpathlineto{\pgfqpoint{1.407844in}{1.934946in}}%
\pgfpathlineto{\pgfqpoint{1.700800in}{1.999906in}}%
\pgfpathlineto{\pgfqpoint{1.993756in}{2.016765in}}%
\pgfpathlineto{\pgfqpoint{2.286712in}{1.998388in}}%
\pgfpathlineto{\pgfqpoint{2.579669in}{1.812827in}}%
\pgfpathlineto{\pgfqpoint{2.872625in}{1.679371in}}%
\pgfpathlineto{\pgfqpoint{3.165581in}{1.468380in}}%
\pgfpathlineto{\pgfqpoint{3.458537in}{1.301231in}}%
\pgfpathlineto{\pgfqpoint{3.751494in}{1.102807in}}%
\pgfpathlineto{\pgfqpoint{4.044450in}{0.989662in}}%
\pgfpathlineto{\pgfqpoint{4.337406in}{0.862097in}}%
\pgfpathlineto{\pgfqpoint{4.630362in}{0.782898in}}%
\pgfpathlineto{\pgfqpoint{4.923318in}{0.720083in}}%
\pgfpathlineto{\pgfqpoint{5.216275in}{0.713719in}}%
\pgfpathlineto{\pgfqpoint{5.509231in}{0.670456in}}%
\pgfpathlineto{\pgfqpoint{5.802187in}{0.687726in}}%
\pgfusepath{stroke}%
\end{pgfscope}%
\begin{pgfscope}%
\pgfpathrectangle{\pgfqpoint{0.572918in}{0.553781in}}{\pgfqpoint{5.478282in}{2.095553in}}%
\pgfusepath{clip}%
\pgfsetbuttcap%
\pgfsetroundjoin%
\pgfsetlinewidth{1.505625pt}%
\definecolor{currentstroke}{rgb}{0.949020,0.372549,0.360784}%
\pgfsetstrokecolor{currentstroke}%
\pgfsetstrokeopacity{0.900000}%
\pgfsetdash{{1.500000pt}{2.475000pt}}{0.000000pt}%
\pgfpathmoveto{\pgfqpoint{0.821931in}{2.256279in}}%
\pgfpathlineto{\pgfqpoint{1.114887in}{2.189072in}}%
\pgfpathlineto{\pgfqpoint{1.407844in}{2.050396in}}%
\pgfpathlineto{\pgfqpoint{1.700800in}{2.122516in}}%
\pgfpathlineto{\pgfqpoint{1.993756in}{2.074969in}}%
\pgfpathlineto{\pgfqpoint{2.286712in}{2.000335in}}%
\pgfpathlineto{\pgfqpoint{2.579669in}{1.875558in}}%
\pgfpathlineto{\pgfqpoint{2.872625in}{1.713271in}}%
\pgfpathlineto{\pgfqpoint{3.165581in}{1.493833in}}%
\pgfpathlineto{\pgfqpoint{3.458537in}{1.337763in}}%
\pgfpathlineto{\pgfqpoint{3.751494in}{1.122286in}}%
\pgfpathlineto{\pgfqpoint{4.044450in}{0.967565in}}%
\pgfpathlineto{\pgfqpoint{4.337406in}{0.861941in}}%
\pgfpathlineto{\pgfqpoint{4.630362in}{0.776763in}}%
\pgfpathlineto{\pgfqpoint{4.923318in}{0.727196in}}%
\pgfpathlineto{\pgfqpoint{5.216275in}{0.707485in}}%
\pgfpathlineto{\pgfqpoint{5.509231in}{0.718591in}}%
\pgfpathlineto{\pgfqpoint{5.802187in}{0.781947in}}%
\pgfusepath{stroke}%
\end{pgfscope}%
\begin{pgfscope}%
\pgfpathrectangle{\pgfqpoint{0.572918in}{0.553781in}}{\pgfqpoint{5.478282in}{2.095553in}}%
\pgfusepath{clip}%
\pgfsetbuttcap%
\pgfsetroundjoin%
\pgfsetlinewidth{1.505625pt}%
\definecolor{currentstroke}{rgb}{1.000000,0.819608,0.101961}%
\pgfsetstrokecolor{currentstroke}%
\pgfsetstrokeopacity{0.900000}%
\pgfsetdash{{5.550000pt}{2.400000pt}}{0.000000pt}%
\pgfpathmoveto{\pgfqpoint{0.821931in}{2.103848in}}%
\pgfpathlineto{\pgfqpoint{1.114887in}{2.044044in}}%
\pgfpathlineto{\pgfqpoint{1.407844in}{2.088033in}}%
\pgfpathlineto{\pgfqpoint{1.700800in}{2.174506in}}%
\pgfpathlineto{\pgfqpoint{1.993756in}{2.224334in}}%
\pgfpathlineto{\pgfqpoint{2.286712in}{2.144933in}}%
\pgfpathlineto{\pgfqpoint{2.579669in}{2.004244in}}%
\pgfpathlineto{\pgfqpoint{2.872625in}{1.795690in}}%
\pgfpathlineto{\pgfqpoint{3.165581in}{1.601041in}}%
\pgfpathlineto{\pgfqpoint{3.458537in}{1.421003in}}%
\pgfpathlineto{\pgfqpoint{3.751494in}{1.224447in}}%
\pgfpathlineto{\pgfqpoint{4.044450in}{1.082847in}}%
\pgfpathlineto{\pgfqpoint{4.337406in}{0.950037in}}%
\pgfpathlineto{\pgfqpoint{4.630362in}{0.886699in}}%
\pgfpathlineto{\pgfqpoint{4.923318in}{0.829434in}}%
\pgfpathlineto{\pgfqpoint{5.216275in}{0.843003in}}%
\pgfpathlineto{\pgfqpoint{5.509231in}{0.819968in}}%
\pgfpathlineto{\pgfqpoint{5.802187in}{0.886377in}}%
\pgfusepath{stroke}%
\end{pgfscope}%
\begin{pgfscope}%
\pgfsetrectcap%
\pgfsetmiterjoin%
\pgfsetlinewidth{0.803000pt}%
\definecolor{currentstroke}{rgb}{0.000000,0.000000,0.000000}%
\pgfsetstrokecolor{currentstroke}%
\pgfsetdash{}{0pt}%
\pgfpathmoveto{\pgfqpoint{0.572918in}{0.553781in}}%
\pgfpathlineto{\pgfqpoint{0.572918in}{2.649333in}}%
\pgfusepath{stroke}%
\end{pgfscope}%
\begin{pgfscope}%
\pgfsetrectcap%
\pgfsetmiterjoin%
\pgfsetlinewidth{0.803000pt}%
\definecolor{currentstroke}{rgb}{0.000000,0.000000,0.000000}%
\pgfsetstrokecolor{currentstroke}%
\pgfsetdash{}{0pt}%
\pgfpathmoveto{\pgfqpoint{6.051200in}{0.553781in}}%
\pgfpathlineto{\pgfqpoint{6.051200in}{2.649333in}}%
\pgfusepath{stroke}%
\end{pgfscope}%
\begin{pgfscope}%
\pgfsetrectcap%
\pgfsetmiterjoin%
\pgfsetlinewidth{0.803000pt}%
\definecolor{currentstroke}{rgb}{0.000000,0.000000,0.000000}%
\pgfsetstrokecolor{currentstroke}%
\pgfsetdash{}{0pt}%
\pgfpathmoveto{\pgfqpoint{0.572918in}{0.553781in}}%
\pgfpathlineto{\pgfqpoint{6.051200in}{0.553781in}}%
\pgfusepath{stroke}%
\end{pgfscope}%
\begin{pgfscope}%
\pgfsetrectcap%
\pgfsetmiterjoin%
\pgfsetlinewidth{0.803000pt}%
\definecolor{currentstroke}{rgb}{0.000000,0.000000,0.000000}%
\pgfsetstrokecolor{currentstroke}%
\pgfsetdash{}{0pt}%
\pgfpathmoveto{\pgfqpoint{0.572918in}{2.649333in}}%
\pgfpathlineto{\pgfqpoint{6.051200in}{2.649333in}}%
\pgfusepath{stroke}%
\end{pgfscope}%
\begin{pgfscope}%
\definecolor{textcolor}{rgb}{0.000000,0.000000,0.000000}%
\pgfsetstrokecolor{textcolor}%
\pgfsetfillcolor{textcolor}%
\pgftext[x=0.572918in,y=2.732667in,left,base]{\color{textcolor}\rmfamily\fontsize{12.000000}{14.400000}\selectfont Zenith performance, different loss functions}%
\end{pgfscope}%
\begin{pgfscope}%
\pgfsetbuttcap%
\pgfsetmiterjoin%
\definecolor{currentfill}{rgb}{1.000000,1.000000,1.000000}%
\pgfsetfillcolor{currentfill}%
\pgfsetfillopacity{0.800000}%
\pgfsetlinewidth{1.003750pt}%
\definecolor{currentstroke}{rgb}{0.800000,0.800000,0.800000}%
\pgfsetstrokecolor{currentstroke}%
\pgfsetstrokeopacity{0.800000}%
\pgfsetdash{}{0pt}%
\pgfpathmoveto{\pgfqpoint{5.243533in}{2.094556in}}%
\pgfpathlineto{\pgfqpoint{5.973422in}{2.094556in}}%
\pgfpathquadraticcurveto{\pgfqpoint{5.995644in}{2.094556in}}{\pgfqpoint{5.995644in}{2.116778in}}%
\pgfpathlineto{\pgfqpoint{5.995644in}{2.571556in}}%
\pgfpathquadraticcurveto{\pgfqpoint{5.995644in}{2.593778in}}{\pgfqpoint{5.973422in}{2.593778in}}%
\pgfpathlineto{\pgfqpoint{5.243533in}{2.593778in}}%
\pgfpathquadraticcurveto{\pgfqpoint{5.221311in}{2.593778in}}{\pgfqpoint{5.221311in}{2.571556in}}%
\pgfpathlineto{\pgfqpoint{5.221311in}{2.116778in}}%
\pgfpathquadraticcurveto{\pgfqpoint{5.221311in}{2.094556in}}{\pgfqpoint{5.243533in}{2.094556in}}%
\pgfpathclose%
\pgfusepath{stroke,fill}%
\end{pgfscope}%
\begin{pgfscope}%
\pgfsetbuttcap%
\pgfsetroundjoin%
\pgfsetlinewidth{1.505625pt}%
\definecolor{currentstroke}{rgb}{0.313725,0.317647,0.309804}%
\pgfsetstrokecolor{currentstroke}%
\pgfsetstrokeopacity{0.900000}%
\pgfsetdash{}{0pt}%
\pgfpathmoveto{\pgfqpoint{5.376867in}{2.454889in}}%
\pgfpathlineto{\pgfqpoint{5.376867in}{2.566000in}}%
\pgfusepath{stroke}%
\end{pgfscope}%
\begin{pgfscope}%
\pgfsetbuttcap%
\pgfsetroundjoin%
\definecolor{currentfill}{rgb}{0.313725,0.317647,0.309804}%
\pgfsetfillcolor{currentfill}%
\pgfsetfillopacity{0.900000}%
\pgfsetlinewidth{1.003750pt}%
\definecolor{currentstroke}{rgb}{0.313725,0.317647,0.309804}%
\pgfsetstrokecolor{currentstroke}%
\pgfsetstrokeopacity{0.900000}%
\pgfsetdash{}{0pt}%
\pgfsys@defobject{currentmarker}{\pgfqpoint{-0.013889in}{-0.000000in}}{\pgfqpoint{0.013889in}{0.000000in}}{%
\pgfpathmoveto{\pgfqpoint{0.013889in}{-0.000000in}}%
\pgfpathlineto{\pgfqpoint{-0.013889in}{0.000000in}}%
\pgfusepath{stroke,fill}%
}%
\begin{pgfscope}%
\pgfsys@transformshift{5.376867in}{2.454889in}%
\pgfsys@useobject{currentmarker}{}%
\end{pgfscope}%
\end{pgfscope}%
\begin{pgfscope}%
\pgfsetbuttcap%
\pgfsetroundjoin%
\definecolor{currentfill}{rgb}{0.313725,0.317647,0.309804}%
\pgfsetfillcolor{currentfill}%
\pgfsetfillopacity{0.900000}%
\pgfsetlinewidth{1.003750pt}%
\definecolor{currentstroke}{rgb}{0.313725,0.317647,0.309804}%
\pgfsetstrokecolor{currentstroke}%
\pgfsetstrokeopacity{0.900000}%
\pgfsetdash{}{0pt}%
\pgfsys@defobject{currentmarker}{\pgfqpoint{-0.013889in}{-0.000000in}}{\pgfqpoint{0.013889in}{0.000000in}}{%
\pgfpathmoveto{\pgfqpoint{0.013889in}{-0.000000in}}%
\pgfpathlineto{\pgfqpoint{-0.013889in}{0.000000in}}%
\pgfusepath{stroke,fill}%
}%
\begin{pgfscope}%
\pgfsys@transformshift{5.376867in}{2.566000in}%
\pgfsys@useobject{currentmarker}{}%
\end{pgfscope}%
\end{pgfscope}%
\begin{pgfscope}%
\pgfsetrectcap%
\pgfsetroundjoin%
\pgfsetlinewidth{1.505625pt}%
\definecolor{currentstroke}{rgb}{0.313725,0.317647,0.309804}%
\pgfsetstrokecolor{currentstroke}%
\pgfsetstrokeopacity{0.900000}%
\pgfsetdash{}{0pt}%
\pgfpathmoveto{\pgfqpoint{5.265755in}{2.510444in}}%
\pgfpathlineto{\pgfqpoint{5.487978in}{2.510444in}}%
\pgfusepath{stroke}%
\end{pgfscope}%
\begin{pgfscope}%
\definecolor{textcolor}{rgb}{0.000000,0.000000,0.000000}%
\pgfsetstrokecolor{textcolor}%
\pgfsetfillcolor{textcolor}%
\pgftext[x=5.576867in,y=2.471556in,left,base]{\color{textcolor}\rmfamily\fontsize{8.000000}{9.600000}\selectfont MSE}%
\end{pgfscope}%
\begin{pgfscope}%
\pgfsetbuttcap%
\pgfsetroundjoin%
\pgfsetlinewidth{1.505625pt}%
\definecolor{currentstroke}{rgb}{0.949020,0.372549,0.360784}%
\pgfsetstrokecolor{currentstroke}%
\pgfsetstrokeopacity{0.900000}%
\pgfsetdash{}{0pt}%
\pgfpathmoveto{\pgfqpoint{5.376867in}{2.300000in}}%
\pgfpathlineto{\pgfqpoint{5.376867in}{2.411111in}}%
\pgfusepath{stroke}%
\end{pgfscope}%
\begin{pgfscope}%
\pgfsetbuttcap%
\pgfsetroundjoin%
\definecolor{currentfill}{rgb}{0.949020,0.372549,0.360784}%
\pgfsetfillcolor{currentfill}%
\pgfsetfillopacity{0.900000}%
\pgfsetlinewidth{1.003750pt}%
\definecolor{currentstroke}{rgb}{0.949020,0.372549,0.360784}%
\pgfsetstrokecolor{currentstroke}%
\pgfsetstrokeopacity{0.900000}%
\pgfsetdash{}{0pt}%
\pgfsys@defobject{currentmarker}{\pgfqpoint{-0.013889in}{-0.000000in}}{\pgfqpoint{0.013889in}{0.000000in}}{%
\pgfpathmoveto{\pgfqpoint{0.013889in}{-0.000000in}}%
\pgfpathlineto{\pgfqpoint{-0.013889in}{0.000000in}}%
\pgfusepath{stroke,fill}%
}%
\begin{pgfscope}%
\pgfsys@transformshift{5.376867in}{2.300000in}%
\pgfsys@useobject{currentmarker}{}%
\end{pgfscope}%
\end{pgfscope}%
\begin{pgfscope}%
\pgfsetbuttcap%
\pgfsetroundjoin%
\definecolor{currentfill}{rgb}{0.949020,0.372549,0.360784}%
\pgfsetfillcolor{currentfill}%
\pgfsetfillopacity{0.900000}%
\pgfsetlinewidth{1.003750pt}%
\definecolor{currentstroke}{rgb}{0.949020,0.372549,0.360784}%
\pgfsetstrokecolor{currentstroke}%
\pgfsetstrokeopacity{0.900000}%
\pgfsetdash{}{0pt}%
\pgfsys@defobject{currentmarker}{\pgfqpoint{-0.013889in}{-0.000000in}}{\pgfqpoint{0.013889in}{0.000000in}}{%
\pgfpathmoveto{\pgfqpoint{0.013889in}{-0.000000in}}%
\pgfpathlineto{\pgfqpoint{-0.013889in}{0.000000in}}%
\pgfusepath{stroke,fill}%
}%
\begin{pgfscope}%
\pgfsys@transformshift{5.376867in}{2.411111in}%
\pgfsys@useobject{currentmarker}{}%
\end{pgfscope}%
\end{pgfscope}%
\begin{pgfscope}%
\pgfsetbuttcap%
\pgfsetroundjoin%
\pgfsetlinewidth{1.505625pt}%
\definecolor{currentstroke}{rgb}{0.949020,0.372549,0.360784}%
\pgfsetstrokecolor{currentstroke}%
\pgfsetstrokeopacity{0.900000}%
\pgfsetdash{{1.500000pt}{2.475000pt}}{0.000000pt}%
\pgfpathmoveto{\pgfqpoint{5.265755in}{2.355556in}}%
\pgfpathlineto{\pgfqpoint{5.487978in}{2.355556in}}%
\pgfusepath{stroke}%
\end{pgfscope}%
\begin{pgfscope}%
\definecolor{textcolor}{rgb}{0.000000,0.000000,0.000000}%
\pgfsetstrokecolor{textcolor}%
\pgfsetfillcolor{textcolor}%
\pgftext[x=5.576867in,y=2.316667in,left,base]{\color{textcolor}\rmfamily\fontsize{8.000000}{9.600000}\selectfont MAE}%
\end{pgfscope}%
\begin{pgfscope}%
\pgfsetbuttcap%
\pgfsetroundjoin%
\pgfsetlinewidth{1.505625pt}%
\definecolor{currentstroke}{rgb}{1.000000,0.819608,0.101961}%
\pgfsetstrokecolor{currentstroke}%
\pgfsetstrokeopacity{0.900000}%
\pgfsetdash{}{0pt}%
\pgfpathmoveto{\pgfqpoint{5.376867in}{2.145111in}}%
\pgfpathlineto{\pgfqpoint{5.376867in}{2.256222in}}%
\pgfusepath{stroke}%
\end{pgfscope}%
\begin{pgfscope}%
\pgfsetbuttcap%
\pgfsetroundjoin%
\definecolor{currentfill}{rgb}{1.000000,0.819608,0.101961}%
\pgfsetfillcolor{currentfill}%
\pgfsetfillopacity{0.900000}%
\pgfsetlinewidth{1.003750pt}%
\definecolor{currentstroke}{rgb}{1.000000,0.819608,0.101961}%
\pgfsetstrokecolor{currentstroke}%
\pgfsetstrokeopacity{0.900000}%
\pgfsetdash{}{0pt}%
\pgfsys@defobject{currentmarker}{\pgfqpoint{-0.013889in}{-0.000000in}}{\pgfqpoint{0.013889in}{0.000000in}}{%
\pgfpathmoveto{\pgfqpoint{0.013889in}{-0.000000in}}%
\pgfpathlineto{\pgfqpoint{-0.013889in}{0.000000in}}%
\pgfusepath{stroke,fill}%
}%
\begin{pgfscope}%
\pgfsys@transformshift{5.376867in}{2.145111in}%
\pgfsys@useobject{currentmarker}{}%
\end{pgfscope}%
\end{pgfscope}%
\begin{pgfscope}%
\pgfsetbuttcap%
\pgfsetroundjoin%
\definecolor{currentfill}{rgb}{1.000000,0.819608,0.101961}%
\pgfsetfillcolor{currentfill}%
\pgfsetfillopacity{0.900000}%
\pgfsetlinewidth{1.003750pt}%
\definecolor{currentstroke}{rgb}{1.000000,0.819608,0.101961}%
\pgfsetstrokecolor{currentstroke}%
\pgfsetstrokeopacity{0.900000}%
\pgfsetdash{}{0pt}%
\pgfsys@defobject{currentmarker}{\pgfqpoint{-0.013889in}{-0.000000in}}{\pgfqpoint{0.013889in}{0.000000in}}{%
\pgfpathmoveto{\pgfqpoint{0.013889in}{-0.000000in}}%
\pgfpathlineto{\pgfqpoint{-0.013889in}{0.000000in}}%
\pgfusepath{stroke,fill}%
}%
\begin{pgfscope}%
\pgfsys@transformshift{5.376867in}{2.256222in}%
\pgfsys@useobject{currentmarker}{}%
\end{pgfscope}%
\end{pgfscope}%
\begin{pgfscope}%
\pgfsetbuttcap%
\pgfsetroundjoin%
\pgfsetlinewidth{1.505625pt}%
\definecolor{currentstroke}{rgb}{1.000000,0.819608,0.101961}%
\pgfsetstrokecolor{currentstroke}%
\pgfsetstrokeopacity{0.900000}%
\pgfsetdash{{5.550000pt}{2.400000pt}}{0.000000pt}%
\pgfpathmoveto{\pgfqpoint{5.265755in}{2.200667in}}%
\pgfpathlineto{\pgfqpoint{5.487978in}{2.200667in}}%
\pgfusepath{stroke}%
\end{pgfscope}%
\begin{pgfscope}%
\definecolor{textcolor}{rgb}{0.000000,0.000000,0.000000}%
\pgfsetstrokecolor{textcolor}%
\pgfsetfillcolor{textcolor}%
\pgftext[x=5.576867in,y=2.161778in,left,base]{\color{textcolor}\rmfamily\fontsize{8.000000}{9.600000}\selectfont logcosh}%
\end{pgfscope}%
\end{pgfpicture}%
\makeatother%
\endgroup%

    \caption{The zenith resolution over \num{18} energy bins from \SIrange{1}{1000}{\giga\electronvolt}.
    The lines serve to guide the eye, as the plot becomes too busy as a scatter plot with more than two groups.
    Zenith performance on small DeepCore dataset with three different loss functions, MSE, MAE and logcosh.
    Logcosh consistently performs worse than MSE, while MSE and MAE are similar at energies above \SI{10}{\giga\electronvolt}.
    However, MSE seems to perform best at very low energy.}\label{fig:zenith_performance_loss}
\end{figure}

\begin{figure}
    \centering
    %% Creator: Matplotlib, PGF backend
%%
%% To include the figure in your LaTeX document, write
%%   \input{<filename>.pgf}
%%
%% Make sure the required packages are loaded in your preamble
%%   \usepackage{pgf}
%%
%% and, on pdftex
%%   \usepackage[utf8]{inputenc}\DeclareUnicodeCharacter{2212}{-}
%%
%% or, on luatex and xetex
%%   \usepackage{unicode-math}
%%
%% Figures using additional raster images can only be included by \input if
%% they are in the same directory as the main LaTeX file. For loading figures
%% from other directories you can use the `import` package
%%   \usepackage{import}
%%
%% and then include the figures with
%%   \import{<path to file>}{<filename>.pgf}
%%
%% Matplotlib used the following preamble
%%   \usepackage{siunitx} \usepackage{amsmath} \usepackage{bm}
%%   \usepackage{fontspec}
%%
\begingroup%
\makeatletter%
\begin{pgfpicture}%
\pgfpathrectangle{\pgfpointorigin}{\pgfqpoint{6.201200in}{2.500000in}}%
\pgfusepath{use as bounding box, clip}%
\begin{pgfscope}%
\pgfsetbuttcap%
\pgfsetmiterjoin%
\definecolor{currentfill}{rgb}{1.000000,1.000000,1.000000}%
\pgfsetfillcolor{currentfill}%
\pgfsetlinewidth{0.000000pt}%
\definecolor{currentstroke}{rgb}{1.000000,1.000000,1.000000}%
\pgfsetstrokecolor{currentstroke}%
\pgfsetdash{}{0pt}%
\pgfpathmoveto{\pgfqpoint{0.000000in}{0.000000in}}%
\pgfpathlineto{\pgfqpoint{6.201200in}{0.000000in}}%
\pgfpathlineto{\pgfqpoint{6.201200in}{2.500000in}}%
\pgfpathlineto{\pgfqpoint{0.000000in}{2.500000in}}%
\pgfpathclose%
\pgfusepath{fill}%
\end{pgfscope}%
\begin{pgfscope}%
\pgfsetbuttcap%
\pgfsetmiterjoin%
\definecolor{currentfill}{rgb}{1.000000,1.000000,1.000000}%
\pgfsetfillcolor{currentfill}%
\pgfsetlinewidth{0.000000pt}%
\definecolor{currentstroke}{rgb}{0.000000,0.000000,0.000000}%
\pgfsetstrokecolor{currentstroke}%
\pgfsetstrokeopacity{0.000000}%
\pgfsetdash{}{0pt}%
\pgfpathmoveto{\pgfqpoint{0.513890in}{0.536494in}}%
\pgfpathlineto{\pgfqpoint{6.051200in}{0.536494in}}%
\pgfpathlineto{\pgfqpoint{6.051200in}{2.161121in}}%
\pgfpathlineto{\pgfqpoint{0.513890in}{2.161121in}}%
\pgfpathclose%
\pgfusepath{fill}%
\end{pgfscope}%
\begin{pgfscope}%
\pgfpathrectangle{\pgfqpoint{0.513890in}{0.536494in}}{\pgfqpoint{5.537310in}{1.624626in}}%
\pgfusepath{clip}%
\pgfsetrectcap%
\pgfsetroundjoin%
\pgfsetlinewidth{0.803000pt}%
\definecolor{currentstroke}{rgb}{0.690196,0.690196,0.690196}%
\pgfsetstrokecolor{currentstroke}%
\pgfsetstrokeopacity{0.200000}%
\pgfsetdash{}{0pt}%
\pgfpathmoveto{\pgfqpoint{1.604572in}{0.536494in}}%
\pgfpathlineto{\pgfqpoint{1.604572in}{2.161121in}}%
\pgfusepath{stroke}%
\end{pgfscope}%
\begin{pgfscope}%
\pgfsetbuttcap%
\pgfsetroundjoin%
\definecolor{currentfill}{rgb}{0.000000,0.000000,0.000000}%
\pgfsetfillcolor{currentfill}%
\pgfsetlinewidth{0.803000pt}%
\definecolor{currentstroke}{rgb}{0.000000,0.000000,0.000000}%
\pgfsetstrokecolor{currentstroke}%
\pgfsetdash{}{0pt}%
\pgfsys@defobject{currentmarker}{\pgfqpoint{0.000000in}{-0.048611in}}{\pgfqpoint{0.000000in}{0.000000in}}{%
\pgfpathmoveto{\pgfqpoint{0.000000in}{0.000000in}}%
\pgfpathlineto{\pgfqpoint{0.000000in}{-0.048611in}}%
\pgfusepath{stroke,fill}%
}%
\begin{pgfscope}%
\pgfsys@transformshift{1.604572in}{0.536494in}%
\pgfsys@useobject{currentmarker}{}%
\end{pgfscope}%
\end{pgfscope}%
\begin{pgfscope}%
\definecolor{textcolor}{rgb}{0.000000,0.000000,0.000000}%
\pgfsetstrokecolor{textcolor}%
\pgfsetfillcolor{textcolor}%
\pgftext[x=1.604572in,y=0.439272in,,top]{\color{textcolor}\rmfamily\fontsize{8.000000}{9.600000}\selectfont \(\displaystyle {-1}\)}%
\end{pgfscope}%
\begin{pgfscope}%
\pgfpathrectangle{\pgfqpoint{0.513890in}{0.536494in}}{\pgfqpoint{5.537310in}{1.624626in}}%
\pgfusepath{clip}%
\pgfsetrectcap%
\pgfsetroundjoin%
\pgfsetlinewidth{0.803000pt}%
\definecolor{currentstroke}{rgb}{0.690196,0.690196,0.690196}%
\pgfsetstrokecolor{currentstroke}%
\pgfsetstrokeopacity{0.200000}%
\pgfsetdash{}{0pt}%
\pgfpathmoveto{\pgfqpoint{3.282545in}{0.536494in}}%
\pgfpathlineto{\pgfqpoint{3.282545in}{2.161121in}}%
\pgfusepath{stroke}%
\end{pgfscope}%
\begin{pgfscope}%
\pgfsetbuttcap%
\pgfsetroundjoin%
\definecolor{currentfill}{rgb}{0.000000,0.000000,0.000000}%
\pgfsetfillcolor{currentfill}%
\pgfsetlinewidth{0.803000pt}%
\definecolor{currentstroke}{rgb}{0.000000,0.000000,0.000000}%
\pgfsetstrokecolor{currentstroke}%
\pgfsetdash{}{0pt}%
\pgfsys@defobject{currentmarker}{\pgfqpoint{0.000000in}{-0.048611in}}{\pgfqpoint{0.000000in}{0.000000in}}{%
\pgfpathmoveto{\pgfqpoint{0.000000in}{0.000000in}}%
\pgfpathlineto{\pgfqpoint{0.000000in}{-0.048611in}}%
\pgfusepath{stroke,fill}%
}%
\begin{pgfscope}%
\pgfsys@transformshift{3.282545in}{0.536494in}%
\pgfsys@useobject{currentmarker}{}%
\end{pgfscope}%
\end{pgfscope}%
\begin{pgfscope}%
\definecolor{textcolor}{rgb}{0.000000,0.000000,0.000000}%
\pgfsetstrokecolor{textcolor}%
\pgfsetfillcolor{textcolor}%
\pgftext[x=3.282545in,y=0.439272in,,top]{\color{textcolor}\rmfamily\fontsize{8.000000}{9.600000}\selectfont \(\displaystyle {0}\)}%
\end{pgfscope}%
\begin{pgfscope}%
\pgfpathrectangle{\pgfqpoint{0.513890in}{0.536494in}}{\pgfqpoint{5.537310in}{1.624626in}}%
\pgfusepath{clip}%
\pgfsetrectcap%
\pgfsetroundjoin%
\pgfsetlinewidth{0.803000pt}%
\definecolor{currentstroke}{rgb}{0.690196,0.690196,0.690196}%
\pgfsetstrokecolor{currentstroke}%
\pgfsetstrokeopacity{0.200000}%
\pgfsetdash{}{0pt}%
\pgfpathmoveto{\pgfqpoint{4.960518in}{0.536494in}}%
\pgfpathlineto{\pgfqpoint{4.960518in}{2.161121in}}%
\pgfusepath{stroke}%
\end{pgfscope}%
\begin{pgfscope}%
\pgfsetbuttcap%
\pgfsetroundjoin%
\definecolor{currentfill}{rgb}{0.000000,0.000000,0.000000}%
\pgfsetfillcolor{currentfill}%
\pgfsetlinewidth{0.803000pt}%
\definecolor{currentstroke}{rgb}{0.000000,0.000000,0.000000}%
\pgfsetstrokecolor{currentstroke}%
\pgfsetdash{}{0pt}%
\pgfsys@defobject{currentmarker}{\pgfqpoint{0.000000in}{-0.048611in}}{\pgfqpoint{0.000000in}{0.000000in}}{%
\pgfpathmoveto{\pgfqpoint{0.000000in}{0.000000in}}%
\pgfpathlineto{\pgfqpoint{0.000000in}{-0.048611in}}%
\pgfusepath{stroke,fill}%
}%
\begin{pgfscope}%
\pgfsys@transformshift{4.960518in}{0.536494in}%
\pgfsys@useobject{currentmarker}{}%
\end{pgfscope}%
\end{pgfscope}%
\begin{pgfscope}%
\definecolor{textcolor}{rgb}{0.000000,0.000000,0.000000}%
\pgfsetstrokecolor{textcolor}%
\pgfsetfillcolor{textcolor}%
\pgftext[x=4.960518in,y=0.439272in,,top]{\color{textcolor}\rmfamily\fontsize{8.000000}{9.600000}\selectfont \(\displaystyle {1}\)}%
\end{pgfscope}%
\begin{pgfscope}%
\definecolor{textcolor}{rgb}{0.000000,0.000000,0.000000}%
\pgfsetstrokecolor{textcolor}%
\pgfsetfillcolor{textcolor}%
\pgftext[x=3.282545in,y=0.285050in,,top]{\color{textcolor}\rmfamily\fontsize{10.950000}{13.140000}\selectfont \(\displaystyle x\)}%
\end{pgfscope}%
\begin{pgfscope}%
\pgfpathrectangle{\pgfqpoint{0.513890in}{0.536494in}}{\pgfqpoint{5.537310in}{1.624626in}}%
\pgfusepath{clip}%
\pgfsetrectcap%
\pgfsetroundjoin%
\pgfsetlinewidth{0.803000pt}%
\definecolor{currentstroke}{rgb}{0.690196,0.690196,0.690196}%
\pgfsetstrokecolor{currentstroke}%
\pgfsetstrokeopacity{0.200000}%
\pgfsetdash{}{0pt}%
\pgfpathmoveto{\pgfqpoint{0.513890in}{0.610341in}}%
\pgfpathlineto{\pgfqpoint{6.051200in}{0.610341in}}%
\pgfusepath{stroke}%
\end{pgfscope}%
\begin{pgfscope}%
\pgfsetbuttcap%
\pgfsetroundjoin%
\definecolor{currentfill}{rgb}{0.000000,0.000000,0.000000}%
\pgfsetfillcolor{currentfill}%
\pgfsetlinewidth{0.803000pt}%
\definecolor{currentstroke}{rgb}{0.000000,0.000000,0.000000}%
\pgfsetstrokecolor{currentstroke}%
\pgfsetdash{}{0pt}%
\pgfsys@defobject{currentmarker}{\pgfqpoint{-0.048611in}{0.000000in}}{\pgfqpoint{-0.000000in}{0.000000in}}{%
\pgfpathmoveto{\pgfqpoint{-0.000000in}{0.000000in}}%
\pgfpathlineto{\pgfqpoint{-0.048611in}{0.000000in}}%
\pgfusepath{stroke,fill}%
}%
\begin{pgfscope}%
\pgfsys@transformshift{0.513890in}{0.610341in}%
\pgfsys@useobject{currentmarker}{}%
\end{pgfscope}%
\end{pgfscope}%
\begin{pgfscope}%
\definecolor{textcolor}{rgb}{0.000000,0.000000,0.000000}%
\pgfsetstrokecolor{textcolor}%
\pgfsetfillcolor{textcolor}%
\pgftext[x=0.357639in, y=0.571785in, left, base]{\color{textcolor}\rmfamily\fontsize{8.000000}{9.600000}\selectfont \(\displaystyle {0}\)}%
\end{pgfscope}%
\begin{pgfscope}%
\pgfpathrectangle{\pgfqpoint{0.513890in}{0.536494in}}{\pgfqpoint{5.537310in}{1.624626in}}%
\pgfusepath{clip}%
\pgfsetrectcap%
\pgfsetroundjoin%
\pgfsetlinewidth{0.803000pt}%
\definecolor{currentstroke}{rgb}{0.690196,0.690196,0.690196}%
\pgfsetstrokecolor{currentstroke}%
\pgfsetstrokeopacity{0.200000}%
\pgfsetdash{}{0pt}%
\pgfpathmoveto{\pgfqpoint{0.513890in}{1.266756in}}%
\pgfpathlineto{\pgfqpoint{6.051200in}{1.266756in}}%
\pgfusepath{stroke}%
\end{pgfscope}%
\begin{pgfscope}%
\pgfsetbuttcap%
\pgfsetroundjoin%
\definecolor{currentfill}{rgb}{0.000000,0.000000,0.000000}%
\pgfsetfillcolor{currentfill}%
\pgfsetlinewidth{0.803000pt}%
\definecolor{currentstroke}{rgb}{0.000000,0.000000,0.000000}%
\pgfsetstrokecolor{currentstroke}%
\pgfsetdash{}{0pt}%
\pgfsys@defobject{currentmarker}{\pgfqpoint{-0.048611in}{0.000000in}}{\pgfqpoint{-0.000000in}{0.000000in}}{%
\pgfpathmoveto{\pgfqpoint{-0.000000in}{0.000000in}}%
\pgfpathlineto{\pgfqpoint{-0.048611in}{0.000000in}}%
\pgfusepath{stroke,fill}%
}%
\begin{pgfscope}%
\pgfsys@transformshift{0.513890in}{1.266756in}%
\pgfsys@useobject{currentmarker}{}%
\end{pgfscope}%
\end{pgfscope}%
\begin{pgfscope}%
\definecolor{textcolor}{rgb}{0.000000,0.000000,0.000000}%
\pgfsetstrokecolor{textcolor}%
\pgfsetfillcolor{textcolor}%
\pgftext[x=0.357639in, y=1.228200in, left, base]{\color{textcolor}\rmfamily\fontsize{8.000000}{9.600000}\selectfont \(\displaystyle {1}\)}%
\end{pgfscope}%
\begin{pgfscope}%
\pgfpathrectangle{\pgfqpoint{0.513890in}{0.536494in}}{\pgfqpoint{5.537310in}{1.624626in}}%
\pgfusepath{clip}%
\pgfsetrectcap%
\pgfsetroundjoin%
\pgfsetlinewidth{0.803000pt}%
\definecolor{currentstroke}{rgb}{0.690196,0.690196,0.690196}%
\pgfsetstrokecolor{currentstroke}%
\pgfsetstrokeopacity{0.200000}%
\pgfsetdash{}{0pt}%
\pgfpathmoveto{\pgfqpoint{0.513890in}{1.923170in}}%
\pgfpathlineto{\pgfqpoint{6.051200in}{1.923170in}}%
\pgfusepath{stroke}%
\end{pgfscope}%
\begin{pgfscope}%
\pgfsetbuttcap%
\pgfsetroundjoin%
\definecolor{currentfill}{rgb}{0.000000,0.000000,0.000000}%
\pgfsetfillcolor{currentfill}%
\pgfsetlinewidth{0.803000pt}%
\definecolor{currentstroke}{rgb}{0.000000,0.000000,0.000000}%
\pgfsetstrokecolor{currentstroke}%
\pgfsetdash{}{0pt}%
\pgfsys@defobject{currentmarker}{\pgfqpoint{-0.048611in}{0.000000in}}{\pgfqpoint{-0.000000in}{0.000000in}}{%
\pgfpathmoveto{\pgfqpoint{-0.000000in}{0.000000in}}%
\pgfpathlineto{\pgfqpoint{-0.048611in}{0.000000in}}%
\pgfusepath{stroke,fill}%
}%
\begin{pgfscope}%
\pgfsys@transformshift{0.513890in}{1.923170in}%
\pgfsys@useobject{currentmarker}{}%
\end{pgfscope}%
\end{pgfscope}%
\begin{pgfscope}%
\definecolor{textcolor}{rgb}{0.000000,0.000000,0.000000}%
\pgfsetstrokecolor{textcolor}%
\pgfsetfillcolor{textcolor}%
\pgftext[x=0.357639in, y=1.884615in, left, base]{\color{textcolor}\rmfamily\fontsize{8.000000}{9.600000}\selectfont \(\displaystyle {2}\)}%
\end{pgfscope}%
\begin{pgfscope}%
\definecolor{textcolor}{rgb}{0.000000,0.000000,0.000000}%
\pgfsetstrokecolor{textcolor}%
\pgfsetfillcolor{textcolor}%
\pgftext[x=0.302083in,y=1.348808in,,bottom,rotate=90.000000]{\color{textcolor}\rmfamily\fontsize{10.950000}{13.140000}\selectfont \(\displaystyle f(x)\)}%
\end{pgfscope}%
\begin{pgfscope}%
\pgfpathrectangle{\pgfqpoint{0.513890in}{0.536494in}}{\pgfqpoint{5.537310in}{1.624626in}}%
\pgfusepath{clip}%
\pgfsetrectcap%
\pgfsetroundjoin%
\pgfsetlinewidth{0.501875pt}%
\definecolor{currentstroke}{rgb}{0.000000,0.000000,0.000000}%
\pgfsetstrokecolor{currentstroke}%
\pgfsetdash{}{0pt}%
\pgfpathmoveto{\pgfqpoint{0.765586in}{2.087274in}}%
\pgfpathlineto{\pgfqpoint{0.851248in}{1.988453in}}%
\pgfpathlineto{\pgfqpoint{0.936910in}{1.893053in}}%
\pgfpathlineto{\pgfqpoint{1.022572in}{1.801075in}}%
\pgfpathlineto{\pgfqpoint{1.108235in}{1.712518in}}%
\pgfpathlineto{\pgfqpoint{1.188858in}{1.632296in}}%
\pgfpathlineto{\pgfqpoint{1.269481in}{1.555105in}}%
\pgfpathlineto{\pgfqpoint{1.350105in}{1.480945in}}%
\pgfpathlineto{\pgfqpoint{1.430728in}{1.409815in}}%
\pgfpathlineto{\pgfqpoint{1.511351in}{1.341717in}}%
\pgfpathlineto{\pgfqpoint{1.591975in}{1.276649in}}%
\pgfpathlineto{\pgfqpoint{1.667559in}{1.218400in}}%
\pgfpathlineto{\pgfqpoint{1.743143in}{1.162815in}}%
\pgfpathlineto{\pgfqpoint{1.818728in}{1.109895in}}%
\pgfpathlineto{\pgfqpoint{1.894312in}{1.059637in}}%
\pgfpathlineto{\pgfqpoint{1.969896in}{1.012044in}}%
\pgfpathlineto{\pgfqpoint{2.045481in}{0.967115in}}%
\pgfpathlineto{\pgfqpoint{2.121065in}{0.924849in}}%
\pgfpathlineto{\pgfqpoint{2.196649in}{0.885247in}}%
\pgfpathlineto{\pgfqpoint{2.272234in}{0.848309in}}%
\pgfpathlineto{\pgfqpoint{2.342779in}{0.816237in}}%
\pgfpathlineto{\pgfqpoint{2.413325in}{0.786485in}}%
\pgfpathlineto{\pgfqpoint{2.483870in}{0.759054in}}%
\pgfpathlineto{\pgfqpoint{2.554415in}{0.733943in}}%
\pgfpathlineto{\pgfqpoint{2.624961in}{0.711152in}}%
\pgfpathlineto{\pgfqpoint{2.695506in}{0.690683in}}%
\pgfpathlineto{\pgfqpoint{2.766052in}{0.672533in}}%
\pgfpathlineto{\pgfqpoint{2.836597in}{0.656704in}}%
\pgfpathlineto{\pgfqpoint{2.907142in}{0.643196in}}%
\pgfpathlineto{\pgfqpoint{2.977688in}{0.632008in}}%
\pgfpathlineto{\pgfqpoint{3.048233in}{0.623140in}}%
\pgfpathlineto{\pgfqpoint{3.118779in}{0.616593in}}%
\pgfpathlineto{\pgfqpoint{3.189324in}{0.612367in}}%
\pgfpathlineto{\pgfqpoint{3.259869in}{0.610461in}}%
\pgfpathlineto{\pgfqpoint{3.330415in}{0.610875in}}%
\pgfpathlineto{\pgfqpoint{3.400960in}{0.613610in}}%
\pgfpathlineto{\pgfqpoint{3.471506in}{0.618665in}}%
\pgfpathlineto{\pgfqpoint{3.542051in}{0.626041in}}%
\pgfpathlineto{\pgfqpoint{3.612597in}{0.635737in}}%
\pgfpathlineto{\pgfqpoint{3.683142in}{0.647754in}}%
\pgfpathlineto{\pgfqpoint{3.753687in}{0.662091in}}%
\pgfpathlineto{\pgfqpoint{3.824233in}{0.678749in}}%
\pgfpathlineto{\pgfqpoint{3.894778in}{0.697727in}}%
\pgfpathlineto{\pgfqpoint{3.965324in}{0.719026in}}%
\pgfpathlineto{\pgfqpoint{4.035869in}{0.742645in}}%
\pgfpathlineto{\pgfqpoint{4.106414in}{0.768584in}}%
\pgfpathlineto{\pgfqpoint{4.176960in}{0.796844in}}%
\pgfpathlineto{\pgfqpoint{4.247505in}{0.827425in}}%
\pgfpathlineto{\pgfqpoint{4.318051in}{0.860326in}}%
\pgfpathlineto{\pgfqpoint{4.388596in}{0.895547in}}%
\pgfpathlineto{\pgfqpoint{4.464180in}{0.935859in}}%
\pgfpathlineto{\pgfqpoint{4.539765in}{0.978835in}}%
\pgfpathlineto{\pgfqpoint{4.615349in}{1.024475in}}%
\pgfpathlineto{\pgfqpoint{4.690933in}{1.072779in}}%
\pgfpathlineto{\pgfqpoint{4.766518in}{1.123746in}}%
\pgfpathlineto{\pgfqpoint{4.842102in}{1.177378in}}%
\pgfpathlineto{\pgfqpoint{4.917687in}{1.233673in}}%
\pgfpathlineto{\pgfqpoint{4.993271in}{1.292632in}}%
\pgfpathlineto{\pgfqpoint{5.073894in}{1.358457in}}%
\pgfpathlineto{\pgfqpoint{5.154518in}{1.427314in}}%
\pgfpathlineto{\pgfqpoint{5.235141in}{1.499201in}}%
\pgfpathlineto{\pgfqpoint{5.315764in}{1.574119in}}%
\pgfpathlineto{\pgfqpoint{5.396387in}{1.652068in}}%
\pgfpathlineto{\pgfqpoint{5.477011in}{1.733047in}}%
\pgfpathlineto{\pgfqpoint{5.557634in}{1.817058in}}%
\pgfpathlineto{\pgfqpoint{5.643296in}{1.909640in}}%
\pgfpathlineto{\pgfqpoint{5.728959in}{2.005643in}}%
\pgfpathlineto{\pgfqpoint{5.799504in}{2.087274in}}%
\pgfpathlineto{\pgfqpoint{5.799504in}{2.087274in}}%
\pgfusepath{stroke}%
\end{pgfscope}%
\begin{pgfscope}%
\pgfpathrectangle{\pgfqpoint{0.513890in}{0.536494in}}{\pgfqpoint{5.537310in}{1.624626in}}%
\pgfusepath{clip}%
\pgfsetbuttcap%
\pgfsetroundjoin%
\pgfsetlinewidth{0.501875pt}%
\definecolor{currentstroke}{rgb}{0.000000,0.000000,0.000000}%
\pgfsetstrokecolor{currentstroke}%
\pgfsetdash{{0.500000pt}{0.825000pt}}{0.000000pt}%
\pgfpathmoveto{\pgfqpoint{0.765586in}{1.594963in}}%
\pgfpathlineto{\pgfqpoint{3.280025in}{0.611326in}}%
\pgfpathlineto{\pgfqpoint{3.285064in}{0.611326in}}%
\pgfpathlineto{\pgfqpoint{5.799504in}{1.594963in}}%
\pgfpathlineto{\pgfqpoint{5.799504in}{1.594963in}}%
\pgfusepath{stroke}%
\end{pgfscope}%
\begin{pgfscope}%
\pgfpathrectangle{\pgfqpoint{0.513890in}{0.536494in}}{\pgfqpoint{5.537310in}{1.624626in}}%
\pgfusepath{clip}%
\pgfsetbuttcap%
\pgfsetroundjoin%
\pgfsetlinewidth{0.501875pt}%
\definecolor{currentstroke}{rgb}{0.000000,0.000000,0.000000}%
\pgfsetstrokecolor{currentstroke}%
\pgfsetdash{{1.850000pt}{0.800000pt}}{0.000000pt}%
\pgfpathmoveto{\pgfqpoint{0.765586in}{0.854207in}}%
\pgfpathlineto{\pgfqpoint{1.093118in}{0.804946in}}%
\pgfpathlineto{\pgfqpoint{1.375299in}{0.764710in}}%
\pgfpathlineto{\pgfqpoint{1.622208in}{0.731727in}}%
\pgfpathlineto{\pgfqpoint{1.843922in}{0.704341in}}%
\pgfpathlineto{\pgfqpoint{2.045481in}{0.681670in}}%
\pgfpathlineto{\pgfqpoint{2.231922in}{0.662911in}}%
\pgfpathlineto{\pgfqpoint{2.408286in}{0.647401in}}%
\pgfpathlineto{\pgfqpoint{2.574571in}{0.634996in}}%
\pgfpathlineto{\pgfqpoint{2.735818in}{0.625213in}}%
\pgfpathlineto{\pgfqpoint{2.892026in}{0.617993in}}%
\pgfpathlineto{\pgfqpoint{3.043194in}{0.613231in}}%
\pgfpathlineto{\pgfqpoint{3.194363in}{0.610734in}}%
\pgfpathlineto{\pgfqpoint{3.340493in}{0.610511in}}%
\pgfpathlineto{\pgfqpoint{3.486623in}{0.612444in}}%
\pgfpathlineto{\pgfqpoint{3.637791in}{0.616682in}}%
\pgfpathlineto{\pgfqpoint{3.788960in}{0.623131in}}%
\pgfpathlineto{\pgfqpoint{3.945168in}{0.632014in}}%
\pgfpathlineto{\pgfqpoint{4.106414in}{0.643405in}}%
\pgfpathlineto{\pgfqpoint{4.277739in}{0.657790in}}%
\pgfpathlineto{\pgfqpoint{4.459141in}{0.675337in}}%
\pgfpathlineto{\pgfqpoint{4.650622in}{0.696146in}}%
\pgfpathlineto{\pgfqpoint{4.857219in}{0.720870in}}%
\pgfpathlineto{\pgfqpoint{5.083972in}{0.750288in}}%
\pgfpathlineto{\pgfqpoint{5.335920in}{0.785250in}}%
\pgfpathlineto{\pgfqpoint{5.623141in}{0.827390in}}%
\pgfpathlineto{\pgfqpoint{5.799504in}{0.854207in}}%
\pgfpathlineto{\pgfqpoint{5.799504in}{0.854207in}}%
\pgfusepath{stroke}%
\end{pgfscope}%
\begin{pgfscope}%
\pgfsetrectcap%
\pgfsetmiterjoin%
\pgfsetlinewidth{0.803000pt}%
\definecolor{currentstroke}{rgb}{0.000000,0.000000,0.000000}%
\pgfsetstrokecolor{currentstroke}%
\pgfsetdash{}{0pt}%
\pgfpathmoveto{\pgfqpoint{0.513890in}{0.536494in}}%
\pgfpathlineto{\pgfqpoint{0.513890in}{2.161121in}}%
\pgfusepath{stroke}%
\end{pgfscope}%
\begin{pgfscope}%
\pgfsetrectcap%
\pgfsetmiterjoin%
\pgfsetlinewidth{0.803000pt}%
\definecolor{currentstroke}{rgb}{0.000000,0.000000,0.000000}%
\pgfsetstrokecolor{currentstroke}%
\pgfsetdash{}{0pt}%
\pgfpathmoveto{\pgfqpoint{6.051200in}{0.536494in}}%
\pgfpathlineto{\pgfqpoint{6.051200in}{2.161121in}}%
\pgfusepath{stroke}%
\end{pgfscope}%
\begin{pgfscope}%
\pgfsetrectcap%
\pgfsetmiterjoin%
\pgfsetlinewidth{0.803000pt}%
\definecolor{currentstroke}{rgb}{0.000000,0.000000,0.000000}%
\pgfsetstrokecolor{currentstroke}%
\pgfsetdash{}{0pt}%
\pgfpathmoveto{\pgfqpoint{0.513890in}{0.536494in}}%
\pgfpathlineto{\pgfqpoint{6.051200in}{0.536494in}}%
\pgfusepath{stroke}%
\end{pgfscope}%
\begin{pgfscope}%
\pgfsetrectcap%
\pgfsetmiterjoin%
\pgfsetlinewidth{0.803000pt}%
\definecolor{currentstroke}{rgb}{0.000000,0.000000,0.000000}%
\pgfsetstrokecolor{currentstroke}%
\pgfsetdash{}{0pt}%
\pgfpathmoveto{\pgfqpoint{0.513890in}{2.161121in}}%
\pgfpathlineto{\pgfqpoint{6.051200in}{2.161121in}}%
\pgfusepath{stroke}%
\end{pgfscope}%
\begin{pgfscope}%
\definecolor{textcolor}{rgb}{0.000000,0.000000,0.000000}%
\pgfsetstrokecolor{textcolor}%
\pgfsetfillcolor{textcolor}%
\pgftext[x=0.513890in,y=2.244454in,left,base]{\color{textcolor}\rmfamily\fontsize{10.950000}{13.140000}\selectfont Loss functions}%
\end{pgfscope}%
\begin{pgfscope}%
\pgfsetbuttcap%
\pgfsetmiterjoin%
\definecolor{currentfill}{rgb}{1.000000,1.000000,1.000000}%
\pgfsetfillcolor{currentfill}%
\pgfsetfillopacity{0.800000}%
\pgfsetlinewidth{1.003750pt}%
\definecolor{currentstroke}{rgb}{0.800000,0.800000,0.800000}%
\pgfsetstrokecolor{currentstroke}%
\pgfsetstrokeopacity{0.800000}%
\pgfsetdash{}{0pt}%
\pgfpathmoveto{\pgfqpoint{2.917600in}{1.606343in}}%
\pgfpathlineto{\pgfqpoint{3.647489in}{1.606343in}}%
\pgfpathquadraticcurveto{\pgfqpoint{3.669712in}{1.606343in}}{\pgfqpoint{3.669712in}{1.628565in}}%
\pgfpathlineto{\pgfqpoint{3.669712in}{2.083343in}}%
\pgfpathquadraticcurveto{\pgfqpoint{3.669712in}{2.105565in}}{\pgfqpoint{3.647489in}{2.105565in}}%
\pgfpathlineto{\pgfqpoint{2.917600in}{2.105565in}}%
\pgfpathquadraticcurveto{\pgfqpoint{2.895378in}{2.105565in}}{\pgfqpoint{2.895378in}{2.083343in}}%
\pgfpathlineto{\pgfqpoint{2.895378in}{1.628565in}}%
\pgfpathquadraticcurveto{\pgfqpoint{2.895378in}{1.606343in}}{\pgfqpoint{2.917600in}{1.606343in}}%
\pgfpathclose%
\pgfusepath{stroke,fill}%
\end{pgfscope}%
\begin{pgfscope}%
\pgfsetrectcap%
\pgfsetroundjoin%
\pgfsetlinewidth{0.501875pt}%
\definecolor{currentstroke}{rgb}{0.000000,0.000000,0.000000}%
\pgfsetstrokecolor{currentstroke}%
\pgfsetdash{}{0pt}%
\pgfpathmoveto{\pgfqpoint{2.939823in}{2.022232in}}%
\pgfpathlineto{\pgfqpoint{3.162045in}{2.022232in}}%
\pgfusepath{stroke}%
\end{pgfscope}%
\begin{pgfscope}%
\definecolor{textcolor}{rgb}{0.000000,0.000000,0.000000}%
\pgfsetstrokecolor{textcolor}%
\pgfsetfillcolor{textcolor}%
\pgftext[x=3.250934in,y=1.983343in,left,base]{\color{textcolor}\rmfamily\fontsize{8.000000}{9.600000}\selectfont MSE}%
\end{pgfscope}%
\begin{pgfscope}%
\pgfsetbuttcap%
\pgfsetroundjoin%
\pgfsetlinewidth{0.501875pt}%
\definecolor{currentstroke}{rgb}{0.000000,0.000000,0.000000}%
\pgfsetstrokecolor{currentstroke}%
\pgfsetdash{{0.500000pt}{0.825000pt}}{0.000000pt}%
\pgfpathmoveto{\pgfqpoint{2.939823in}{1.867343in}}%
\pgfpathlineto{\pgfqpoint{3.162045in}{1.867343in}}%
\pgfusepath{stroke}%
\end{pgfscope}%
\begin{pgfscope}%
\definecolor{textcolor}{rgb}{0.000000,0.000000,0.000000}%
\pgfsetstrokecolor{textcolor}%
\pgfsetfillcolor{textcolor}%
\pgftext[x=3.250934in,y=1.828454in,left,base]{\color{textcolor}\rmfamily\fontsize{8.000000}{9.600000}\selectfont MAE}%
\end{pgfscope}%
\begin{pgfscope}%
\pgfsetbuttcap%
\pgfsetroundjoin%
\pgfsetlinewidth{0.501875pt}%
\definecolor{currentstroke}{rgb}{0.000000,0.000000,0.000000}%
\pgfsetstrokecolor{currentstroke}%
\pgfsetdash{{1.850000pt}{0.800000pt}}{0.000000pt}%
\pgfpathmoveto{\pgfqpoint{2.939823in}{1.712454in}}%
\pgfpathlineto{\pgfqpoint{3.162045in}{1.712454in}}%
\pgfusepath{stroke}%
\end{pgfscope}%
\begin{pgfscope}%
\definecolor{textcolor}{rgb}{0.000000,0.000000,0.000000}%
\pgfsetstrokecolor{textcolor}%
\pgfsetfillcolor{textcolor}%
\pgftext[x=3.250934in,y=1.673565in,left,base]{\color{textcolor}\rmfamily\fontsize{8.000000}{9.600000}\selectfont logcosh}%
\end{pgfscope}%
\end{pgfpicture}%
\makeatother%
\endgroup%

    \caption{Loss functions MSE, MAE and logcosh.
    The outsize weight assignment of MSE is seen at values over \( x = 1 \), while the linear fashion and kink at \( x = \) is visible for MAE.\@
    Logcosh is smooth and does not assign outsize weight to outliers.}\label{fig:loss_functions}
\end{figure}

The choice of loss function has an impact on the performance of a model.
A handful were tested, with the usual suspects Mean Squared Error (MSE), Mean Absolute Error (MAE) and the logarithm of the hyperbolic cosine (\( \log[\cosh(x)] \), termed simply logcosh) performing the best, as shown in~\vref{fig:zenith_performance_loss} using the zenith regression task as an example.

The three loss functions---MSE, MAE and logcosh---are staples of machine learning, but perform their duties differently.
MSE,
\begin{equation}
    \text{MSE} = \frac{1}{n} \sum_{i = 1}^{n} {\left( y_{\text{true}} - y_{\text{reco}} \right)}^2,
\end{equation}
assigns loss quadratically, which means that outliers are assigned an outsize weight.
Here, \( n \) is the number of samples in a mini-batch, \( y_{\text{true}} \) the true value and \( y_{\text{reco}} \) the reconstructed value.
MAE,
\begin{equation}
    \text{MAE} = \frac{1}{n} \sum_{i = 1}^{n} \lvert y_{\text{true}} - y_{\text{reco}} \rvert,
\end{equation}
is an attempt to remedy this by assigning a linear weight.
It has a constant derivative almost everywhere, but is non-differentiable at \( y_{\text{true}} = y_{\text{reco}} \), so must be approximated by a smooth function at that point.

Logcosh loss,
\begin{equation}
    \text{logcosh} = \sum_{i = 1}^{n} \log[\cosh(y_{\text{true}} - y_{\text{reco}})],
\end{equation}
is differentiable everywhere, and not as sensitive to outliers, because \( \text{logcosh} \approx x^2 / 2 \) for small \( x \) and \( \text{logcosh} \approx \lvert x \rvert - \log(2) \) for large \( x \), avoiding the problems of MSE.\@

\subsection{Prediction type}

The prediction of angles\footnote{Or, rather, zenith, which is the quantity of importance for oscillation studies} can proceed in several ways.

Because the azimuthal angle wraps around at \( 2 \pi \), it introduces problems for the neural network because it cannot understand that e.g.\ \SI{6.27}{\radian} and \SI{0.01}{\radian} are points close to each other on the circle.
This may be remedied by predicting the \( x \), \( y \) and \( z \) components of the directional vector associated with an event, which has the benefit of not relying on circular statistics.
The loss function may even contain a penalty term ensuring that the network attempts to predict unit vectors.
This was implemented, but was not found to make a significant difference compared to no penalty.

\begin{figure}
    \centering
    %% Creator: Matplotlib, PGF backend
%%
%% To include the figure in your LaTeX document, write
%%   \input{<filename>.pgf}
%%
%% Make sure the required packages are loaded in your preamble
%%   \usepackage{pgf}
%%
%% and, on pdftex
%%   \usepackage[utf8]{inputenc}\DeclareUnicodeCharacter{2212}{-}
%%
%% or, on luatex and xetex
%%   \usepackage{unicode-math}
%%
%% Figures using additional raster images can only be included by \input if
%% they are in the same directory as the main LaTeX file. For loading figures
%% from other directories you can use the `import` package
%%   \usepackage{import}
%%
%% and then include the figures with
%%   \import{<path to file>}{<filename>.pgf}
%%
%% Matplotlib used the following preamble
%%   \usepackage{siunitx} \usepackage{amsmath} \usepackage{bm}
%%   \usepackage{fontspec}
%%
\begingroup%
\makeatletter%
\begin{pgfpicture}%
\pgfpathrectangle{\pgfpointorigin}{\pgfqpoint{6.201200in}{3.000000in}}%
\pgfusepath{use as bounding box, clip}%
\begin{pgfscope}%
\pgfsetbuttcap%
\pgfsetmiterjoin%
\definecolor{currentfill}{rgb}{1.000000,1.000000,1.000000}%
\pgfsetfillcolor{currentfill}%
\pgfsetlinewidth{0.000000pt}%
\definecolor{currentstroke}{rgb}{1.000000,1.000000,1.000000}%
\pgfsetstrokecolor{currentstroke}%
\pgfsetdash{}{0pt}%
\pgfpathmoveto{\pgfqpoint{0.000000in}{0.000000in}}%
\pgfpathlineto{\pgfqpoint{6.201200in}{0.000000in}}%
\pgfpathlineto{\pgfqpoint{6.201200in}{3.000000in}}%
\pgfpathlineto{\pgfqpoint{0.000000in}{3.000000in}}%
\pgfpathclose%
\pgfusepath{fill}%
\end{pgfscope}%
\begin{pgfscope}%
\pgfsetbuttcap%
\pgfsetmiterjoin%
\definecolor{currentfill}{rgb}{1.000000,1.000000,1.000000}%
\pgfsetfillcolor{currentfill}%
\pgfsetlinewidth{0.000000pt}%
\definecolor{currentstroke}{rgb}{0.000000,0.000000,0.000000}%
\pgfsetstrokecolor{currentstroke}%
\pgfsetstrokeopacity{0.000000}%
\pgfsetdash{}{0pt}%
\pgfpathmoveto{\pgfqpoint{0.572918in}{0.553781in}}%
\pgfpathlineto{\pgfqpoint{6.051200in}{0.553781in}}%
\pgfpathlineto{\pgfqpoint{6.051200in}{2.649333in}}%
\pgfpathlineto{\pgfqpoint{0.572918in}{2.649333in}}%
\pgfpathclose%
\pgfusepath{fill}%
\end{pgfscope}%
\begin{pgfscope}%
\pgfpathrectangle{\pgfqpoint{0.572918in}{0.553781in}}{\pgfqpoint{5.478282in}{2.095553in}}%
\pgfusepath{clip}%
\pgfsetbuttcap%
\pgfsetroundjoin%
\pgfsetlinewidth{0.501875pt}%
\definecolor{currentstroke}{rgb}{0.690196,0.690196,0.690196}%
\pgfsetstrokecolor{currentstroke}%
\pgfsetstrokeopacity{0.500000}%
\pgfsetdash{{0.500000pt}{0.825000pt}}{0.000000pt}%
\pgfpathmoveto{\pgfqpoint{0.675453in}{0.553781in}}%
\pgfpathlineto{\pgfqpoint{0.675453in}{2.649333in}}%
\pgfusepath{stroke}%
\end{pgfscope}%
\begin{pgfscope}%
\pgfsetbuttcap%
\pgfsetroundjoin%
\definecolor{currentfill}{rgb}{0.000000,0.000000,0.000000}%
\pgfsetfillcolor{currentfill}%
\pgfsetlinewidth{0.803000pt}%
\definecolor{currentstroke}{rgb}{0.000000,0.000000,0.000000}%
\pgfsetstrokecolor{currentstroke}%
\pgfsetdash{}{0pt}%
\pgfsys@defobject{currentmarker}{\pgfqpoint{0.000000in}{-0.048611in}}{\pgfqpoint{0.000000in}{0.000000in}}{%
\pgfpathmoveto{\pgfqpoint{0.000000in}{0.000000in}}%
\pgfpathlineto{\pgfqpoint{0.000000in}{-0.048611in}}%
\pgfusepath{stroke,fill}%
}%
\begin{pgfscope}%
\pgfsys@transformshift{0.675453in}{0.553781in}%
\pgfsys@useobject{currentmarker}{}%
\end{pgfscope}%
\end{pgfscope}%
\begin{pgfscope}%
\definecolor{textcolor}{rgb}{0.000000,0.000000,0.000000}%
\pgfsetstrokecolor{textcolor}%
\pgfsetfillcolor{textcolor}%
\pgftext[x=0.675453in,y=0.456558in,,top]{\color{textcolor}\rmfamily\fontsize{8.000000}{9.600000}\selectfont \(\displaystyle {0.0}\)}%
\end{pgfscope}%
\begin{pgfscope}%
\pgfpathrectangle{\pgfqpoint{0.572918in}{0.553781in}}{\pgfqpoint{5.478282in}{2.095553in}}%
\pgfusepath{clip}%
\pgfsetbuttcap%
\pgfsetroundjoin%
\pgfsetlinewidth{0.501875pt}%
\definecolor{currentstroke}{rgb}{0.690196,0.690196,0.690196}%
\pgfsetstrokecolor{currentstroke}%
\pgfsetstrokeopacity{0.500000}%
\pgfsetdash{{0.500000pt}{0.825000pt}}{0.000000pt}%
\pgfpathmoveto{\pgfqpoint{1.554322in}{0.553781in}}%
\pgfpathlineto{\pgfqpoint{1.554322in}{2.649333in}}%
\pgfusepath{stroke}%
\end{pgfscope}%
\begin{pgfscope}%
\pgfsetbuttcap%
\pgfsetroundjoin%
\definecolor{currentfill}{rgb}{0.000000,0.000000,0.000000}%
\pgfsetfillcolor{currentfill}%
\pgfsetlinewidth{0.803000pt}%
\definecolor{currentstroke}{rgb}{0.000000,0.000000,0.000000}%
\pgfsetstrokecolor{currentstroke}%
\pgfsetdash{}{0pt}%
\pgfsys@defobject{currentmarker}{\pgfqpoint{0.000000in}{-0.048611in}}{\pgfqpoint{0.000000in}{0.000000in}}{%
\pgfpathmoveto{\pgfqpoint{0.000000in}{0.000000in}}%
\pgfpathlineto{\pgfqpoint{0.000000in}{-0.048611in}}%
\pgfusepath{stroke,fill}%
}%
\begin{pgfscope}%
\pgfsys@transformshift{1.554322in}{0.553781in}%
\pgfsys@useobject{currentmarker}{}%
\end{pgfscope}%
\end{pgfscope}%
\begin{pgfscope}%
\definecolor{textcolor}{rgb}{0.000000,0.000000,0.000000}%
\pgfsetstrokecolor{textcolor}%
\pgfsetfillcolor{textcolor}%
\pgftext[x=1.554322in,y=0.456558in,,top]{\color{textcolor}\rmfamily\fontsize{8.000000}{9.600000}\selectfont \(\displaystyle {0.5}\)}%
\end{pgfscope}%
\begin{pgfscope}%
\pgfpathrectangle{\pgfqpoint{0.572918in}{0.553781in}}{\pgfqpoint{5.478282in}{2.095553in}}%
\pgfusepath{clip}%
\pgfsetbuttcap%
\pgfsetroundjoin%
\pgfsetlinewidth{0.501875pt}%
\definecolor{currentstroke}{rgb}{0.690196,0.690196,0.690196}%
\pgfsetstrokecolor{currentstroke}%
\pgfsetstrokeopacity{0.500000}%
\pgfsetdash{{0.500000pt}{0.825000pt}}{0.000000pt}%
\pgfpathmoveto{\pgfqpoint{2.433190in}{0.553781in}}%
\pgfpathlineto{\pgfqpoint{2.433190in}{2.649333in}}%
\pgfusepath{stroke}%
\end{pgfscope}%
\begin{pgfscope}%
\pgfsetbuttcap%
\pgfsetroundjoin%
\definecolor{currentfill}{rgb}{0.000000,0.000000,0.000000}%
\pgfsetfillcolor{currentfill}%
\pgfsetlinewidth{0.803000pt}%
\definecolor{currentstroke}{rgb}{0.000000,0.000000,0.000000}%
\pgfsetstrokecolor{currentstroke}%
\pgfsetdash{}{0pt}%
\pgfsys@defobject{currentmarker}{\pgfqpoint{0.000000in}{-0.048611in}}{\pgfqpoint{0.000000in}{0.000000in}}{%
\pgfpathmoveto{\pgfqpoint{0.000000in}{0.000000in}}%
\pgfpathlineto{\pgfqpoint{0.000000in}{-0.048611in}}%
\pgfusepath{stroke,fill}%
}%
\begin{pgfscope}%
\pgfsys@transformshift{2.433190in}{0.553781in}%
\pgfsys@useobject{currentmarker}{}%
\end{pgfscope}%
\end{pgfscope}%
\begin{pgfscope}%
\definecolor{textcolor}{rgb}{0.000000,0.000000,0.000000}%
\pgfsetstrokecolor{textcolor}%
\pgfsetfillcolor{textcolor}%
\pgftext[x=2.433190in,y=0.456558in,,top]{\color{textcolor}\rmfamily\fontsize{8.000000}{9.600000}\selectfont \(\displaystyle {1.0}\)}%
\end{pgfscope}%
\begin{pgfscope}%
\pgfpathrectangle{\pgfqpoint{0.572918in}{0.553781in}}{\pgfqpoint{5.478282in}{2.095553in}}%
\pgfusepath{clip}%
\pgfsetbuttcap%
\pgfsetroundjoin%
\pgfsetlinewidth{0.501875pt}%
\definecolor{currentstroke}{rgb}{0.690196,0.690196,0.690196}%
\pgfsetstrokecolor{currentstroke}%
\pgfsetstrokeopacity{0.500000}%
\pgfsetdash{{0.500000pt}{0.825000pt}}{0.000000pt}%
\pgfpathmoveto{\pgfqpoint{3.312059in}{0.553781in}}%
\pgfpathlineto{\pgfqpoint{3.312059in}{2.649333in}}%
\pgfusepath{stroke}%
\end{pgfscope}%
\begin{pgfscope}%
\pgfsetbuttcap%
\pgfsetroundjoin%
\definecolor{currentfill}{rgb}{0.000000,0.000000,0.000000}%
\pgfsetfillcolor{currentfill}%
\pgfsetlinewidth{0.803000pt}%
\definecolor{currentstroke}{rgb}{0.000000,0.000000,0.000000}%
\pgfsetstrokecolor{currentstroke}%
\pgfsetdash{}{0pt}%
\pgfsys@defobject{currentmarker}{\pgfqpoint{0.000000in}{-0.048611in}}{\pgfqpoint{0.000000in}{0.000000in}}{%
\pgfpathmoveto{\pgfqpoint{0.000000in}{0.000000in}}%
\pgfpathlineto{\pgfqpoint{0.000000in}{-0.048611in}}%
\pgfusepath{stroke,fill}%
}%
\begin{pgfscope}%
\pgfsys@transformshift{3.312059in}{0.553781in}%
\pgfsys@useobject{currentmarker}{}%
\end{pgfscope}%
\end{pgfscope}%
\begin{pgfscope}%
\definecolor{textcolor}{rgb}{0.000000,0.000000,0.000000}%
\pgfsetstrokecolor{textcolor}%
\pgfsetfillcolor{textcolor}%
\pgftext[x=3.312059in,y=0.456558in,,top]{\color{textcolor}\rmfamily\fontsize{8.000000}{9.600000}\selectfont \(\displaystyle {1.5}\)}%
\end{pgfscope}%
\begin{pgfscope}%
\pgfpathrectangle{\pgfqpoint{0.572918in}{0.553781in}}{\pgfqpoint{5.478282in}{2.095553in}}%
\pgfusepath{clip}%
\pgfsetbuttcap%
\pgfsetroundjoin%
\pgfsetlinewidth{0.501875pt}%
\definecolor{currentstroke}{rgb}{0.690196,0.690196,0.690196}%
\pgfsetstrokecolor{currentstroke}%
\pgfsetstrokeopacity{0.500000}%
\pgfsetdash{{0.500000pt}{0.825000pt}}{0.000000pt}%
\pgfpathmoveto{\pgfqpoint{4.190928in}{0.553781in}}%
\pgfpathlineto{\pgfqpoint{4.190928in}{2.649333in}}%
\pgfusepath{stroke}%
\end{pgfscope}%
\begin{pgfscope}%
\pgfsetbuttcap%
\pgfsetroundjoin%
\definecolor{currentfill}{rgb}{0.000000,0.000000,0.000000}%
\pgfsetfillcolor{currentfill}%
\pgfsetlinewidth{0.803000pt}%
\definecolor{currentstroke}{rgb}{0.000000,0.000000,0.000000}%
\pgfsetstrokecolor{currentstroke}%
\pgfsetdash{}{0pt}%
\pgfsys@defobject{currentmarker}{\pgfqpoint{0.000000in}{-0.048611in}}{\pgfqpoint{0.000000in}{0.000000in}}{%
\pgfpathmoveto{\pgfqpoint{0.000000in}{0.000000in}}%
\pgfpathlineto{\pgfqpoint{0.000000in}{-0.048611in}}%
\pgfusepath{stroke,fill}%
}%
\begin{pgfscope}%
\pgfsys@transformshift{4.190928in}{0.553781in}%
\pgfsys@useobject{currentmarker}{}%
\end{pgfscope}%
\end{pgfscope}%
\begin{pgfscope}%
\definecolor{textcolor}{rgb}{0.000000,0.000000,0.000000}%
\pgfsetstrokecolor{textcolor}%
\pgfsetfillcolor{textcolor}%
\pgftext[x=4.190928in,y=0.456558in,,top]{\color{textcolor}\rmfamily\fontsize{8.000000}{9.600000}\selectfont \(\displaystyle {2.0}\)}%
\end{pgfscope}%
\begin{pgfscope}%
\pgfpathrectangle{\pgfqpoint{0.572918in}{0.553781in}}{\pgfqpoint{5.478282in}{2.095553in}}%
\pgfusepath{clip}%
\pgfsetbuttcap%
\pgfsetroundjoin%
\pgfsetlinewidth{0.501875pt}%
\definecolor{currentstroke}{rgb}{0.690196,0.690196,0.690196}%
\pgfsetstrokecolor{currentstroke}%
\pgfsetstrokeopacity{0.500000}%
\pgfsetdash{{0.500000pt}{0.825000pt}}{0.000000pt}%
\pgfpathmoveto{\pgfqpoint{5.069797in}{0.553781in}}%
\pgfpathlineto{\pgfqpoint{5.069797in}{2.649333in}}%
\pgfusepath{stroke}%
\end{pgfscope}%
\begin{pgfscope}%
\pgfsetbuttcap%
\pgfsetroundjoin%
\definecolor{currentfill}{rgb}{0.000000,0.000000,0.000000}%
\pgfsetfillcolor{currentfill}%
\pgfsetlinewidth{0.803000pt}%
\definecolor{currentstroke}{rgb}{0.000000,0.000000,0.000000}%
\pgfsetstrokecolor{currentstroke}%
\pgfsetdash{}{0pt}%
\pgfsys@defobject{currentmarker}{\pgfqpoint{0.000000in}{-0.048611in}}{\pgfqpoint{0.000000in}{0.000000in}}{%
\pgfpathmoveto{\pgfqpoint{0.000000in}{0.000000in}}%
\pgfpathlineto{\pgfqpoint{0.000000in}{-0.048611in}}%
\pgfusepath{stroke,fill}%
}%
\begin{pgfscope}%
\pgfsys@transformshift{5.069797in}{0.553781in}%
\pgfsys@useobject{currentmarker}{}%
\end{pgfscope}%
\end{pgfscope}%
\begin{pgfscope}%
\definecolor{textcolor}{rgb}{0.000000,0.000000,0.000000}%
\pgfsetstrokecolor{textcolor}%
\pgfsetfillcolor{textcolor}%
\pgftext[x=5.069797in,y=0.456558in,,top]{\color{textcolor}\rmfamily\fontsize{8.000000}{9.600000}\selectfont \(\displaystyle {2.5}\)}%
\end{pgfscope}%
\begin{pgfscope}%
\pgfpathrectangle{\pgfqpoint{0.572918in}{0.553781in}}{\pgfqpoint{5.478282in}{2.095553in}}%
\pgfusepath{clip}%
\pgfsetbuttcap%
\pgfsetroundjoin%
\pgfsetlinewidth{0.501875pt}%
\definecolor{currentstroke}{rgb}{0.690196,0.690196,0.690196}%
\pgfsetstrokecolor{currentstroke}%
\pgfsetstrokeopacity{0.500000}%
\pgfsetdash{{0.500000pt}{0.825000pt}}{0.000000pt}%
\pgfpathmoveto{\pgfqpoint{5.948665in}{0.553781in}}%
\pgfpathlineto{\pgfqpoint{5.948665in}{2.649333in}}%
\pgfusepath{stroke}%
\end{pgfscope}%
\begin{pgfscope}%
\pgfsetbuttcap%
\pgfsetroundjoin%
\definecolor{currentfill}{rgb}{0.000000,0.000000,0.000000}%
\pgfsetfillcolor{currentfill}%
\pgfsetlinewidth{0.803000pt}%
\definecolor{currentstroke}{rgb}{0.000000,0.000000,0.000000}%
\pgfsetstrokecolor{currentstroke}%
\pgfsetdash{}{0pt}%
\pgfsys@defobject{currentmarker}{\pgfqpoint{0.000000in}{-0.048611in}}{\pgfqpoint{0.000000in}{0.000000in}}{%
\pgfpathmoveto{\pgfqpoint{0.000000in}{0.000000in}}%
\pgfpathlineto{\pgfqpoint{0.000000in}{-0.048611in}}%
\pgfusepath{stroke,fill}%
}%
\begin{pgfscope}%
\pgfsys@transformshift{5.948665in}{0.553781in}%
\pgfsys@useobject{currentmarker}{}%
\end{pgfscope}%
\end{pgfscope}%
\begin{pgfscope}%
\definecolor{textcolor}{rgb}{0.000000,0.000000,0.000000}%
\pgfsetstrokecolor{textcolor}%
\pgfsetfillcolor{textcolor}%
\pgftext[x=5.948665in,y=0.456558in,,top]{\color{textcolor}\rmfamily\fontsize{8.000000}{9.600000}\selectfont \(\displaystyle {3.0}\)}%
\end{pgfscope}%
\begin{pgfscope}%
\definecolor{textcolor}{rgb}{0.000000,0.000000,0.000000}%
\pgfsetstrokecolor{textcolor}%
\pgfsetfillcolor{textcolor}%
\pgftext[x=3.312059in,y=0.302336in,,top]{\color{textcolor}\rmfamily\fontsize{10.950000}{13.140000}\selectfont \(\displaystyle \log_{10}(E_{\textup{true}}) \, \left[ E / \textup{GeV} \right]\)}%
\end{pgfscope}%
\begin{pgfscope}%
\pgfpathrectangle{\pgfqpoint{0.572918in}{0.553781in}}{\pgfqpoint{5.478282in}{2.095553in}}%
\pgfusepath{clip}%
\pgfsetbuttcap%
\pgfsetroundjoin%
\pgfsetlinewidth{0.501875pt}%
\definecolor{currentstroke}{rgb}{0.690196,0.690196,0.690196}%
\pgfsetstrokecolor{currentstroke}%
\pgfsetstrokeopacity{0.500000}%
\pgfsetdash{{0.500000pt}{0.825000pt}}{0.000000pt}%
\pgfpathmoveto{\pgfqpoint{0.572918in}{0.568613in}}%
\pgfpathlineto{\pgfqpoint{6.051200in}{0.568613in}}%
\pgfusepath{stroke}%
\end{pgfscope}%
\begin{pgfscope}%
\pgfsetbuttcap%
\pgfsetroundjoin%
\definecolor{currentfill}{rgb}{0.000000,0.000000,0.000000}%
\pgfsetfillcolor{currentfill}%
\pgfsetlinewidth{0.803000pt}%
\definecolor{currentstroke}{rgb}{0.000000,0.000000,0.000000}%
\pgfsetstrokecolor{currentstroke}%
\pgfsetdash{}{0pt}%
\pgfsys@defobject{currentmarker}{\pgfqpoint{-0.048611in}{0.000000in}}{\pgfqpoint{-0.000000in}{0.000000in}}{%
\pgfpathmoveto{\pgfqpoint{-0.000000in}{0.000000in}}%
\pgfpathlineto{\pgfqpoint{-0.048611in}{0.000000in}}%
\pgfusepath{stroke,fill}%
}%
\begin{pgfscope}%
\pgfsys@transformshift{0.572918in}{0.568613in}%
\pgfsys@useobject{currentmarker}{}%
\end{pgfscope}%
\end{pgfscope}%
\begin{pgfscope}%
\definecolor{textcolor}{rgb}{0.000000,0.000000,0.000000}%
\pgfsetstrokecolor{textcolor}%
\pgfsetfillcolor{textcolor}%
\pgftext[x=0.357639in, y=0.530058in, left, base]{\color{textcolor}\rmfamily\fontsize{8.000000}{9.600000}\selectfont \(\displaystyle {10}\)}%
\end{pgfscope}%
\begin{pgfscope}%
\pgfpathrectangle{\pgfqpoint{0.572918in}{0.553781in}}{\pgfqpoint{5.478282in}{2.095553in}}%
\pgfusepath{clip}%
\pgfsetbuttcap%
\pgfsetroundjoin%
\pgfsetlinewidth{0.501875pt}%
\definecolor{currentstroke}{rgb}{0.690196,0.690196,0.690196}%
\pgfsetstrokecolor{currentstroke}%
\pgfsetstrokeopacity{0.500000}%
\pgfsetdash{{0.500000pt}{0.825000pt}}{0.000000pt}%
\pgfpathmoveto{\pgfqpoint{0.572918in}{0.876066in}}%
\pgfpathlineto{\pgfqpoint{6.051200in}{0.876066in}}%
\pgfusepath{stroke}%
\end{pgfscope}%
\begin{pgfscope}%
\pgfsetbuttcap%
\pgfsetroundjoin%
\definecolor{currentfill}{rgb}{0.000000,0.000000,0.000000}%
\pgfsetfillcolor{currentfill}%
\pgfsetlinewidth{0.803000pt}%
\definecolor{currentstroke}{rgb}{0.000000,0.000000,0.000000}%
\pgfsetstrokecolor{currentstroke}%
\pgfsetdash{}{0pt}%
\pgfsys@defobject{currentmarker}{\pgfqpoint{-0.048611in}{0.000000in}}{\pgfqpoint{-0.000000in}{0.000000in}}{%
\pgfpathmoveto{\pgfqpoint{-0.000000in}{0.000000in}}%
\pgfpathlineto{\pgfqpoint{-0.048611in}{0.000000in}}%
\pgfusepath{stroke,fill}%
}%
\begin{pgfscope}%
\pgfsys@transformshift{0.572918in}{0.876066in}%
\pgfsys@useobject{currentmarker}{}%
\end{pgfscope}%
\end{pgfscope}%
\begin{pgfscope}%
\definecolor{textcolor}{rgb}{0.000000,0.000000,0.000000}%
\pgfsetstrokecolor{textcolor}%
\pgfsetfillcolor{textcolor}%
\pgftext[x=0.357639in, y=0.837511in, left, base]{\color{textcolor}\rmfamily\fontsize{8.000000}{9.600000}\selectfont \(\displaystyle {15}\)}%
\end{pgfscope}%
\begin{pgfscope}%
\pgfpathrectangle{\pgfqpoint{0.572918in}{0.553781in}}{\pgfqpoint{5.478282in}{2.095553in}}%
\pgfusepath{clip}%
\pgfsetbuttcap%
\pgfsetroundjoin%
\pgfsetlinewidth{0.501875pt}%
\definecolor{currentstroke}{rgb}{0.690196,0.690196,0.690196}%
\pgfsetstrokecolor{currentstroke}%
\pgfsetstrokeopacity{0.500000}%
\pgfsetdash{{0.500000pt}{0.825000pt}}{0.000000pt}%
\pgfpathmoveto{\pgfqpoint{0.572918in}{1.183519in}}%
\pgfpathlineto{\pgfqpoint{6.051200in}{1.183519in}}%
\pgfusepath{stroke}%
\end{pgfscope}%
\begin{pgfscope}%
\pgfsetbuttcap%
\pgfsetroundjoin%
\definecolor{currentfill}{rgb}{0.000000,0.000000,0.000000}%
\pgfsetfillcolor{currentfill}%
\pgfsetlinewidth{0.803000pt}%
\definecolor{currentstroke}{rgb}{0.000000,0.000000,0.000000}%
\pgfsetstrokecolor{currentstroke}%
\pgfsetdash{}{0pt}%
\pgfsys@defobject{currentmarker}{\pgfqpoint{-0.048611in}{0.000000in}}{\pgfqpoint{-0.000000in}{0.000000in}}{%
\pgfpathmoveto{\pgfqpoint{-0.000000in}{0.000000in}}%
\pgfpathlineto{\pgfqpoint{-0.048611in}{0.000000in}}%
\pgfusepath{stroke,fill}%
}%
\begin{pgfscope}%
\pgfsys@transformshift{0.572918in}{1.183519in}%
\pgfsys@useobject{currentmarker}{}%
\end{pgfscope}%
\end{pgfscope}%
\begin{pgfscope}%
\definecolor{textcolor}{rgb}{0.000000,0.000000,0.000000}%
\pgfsetstrokecolor{textcolor}%
\pgfsetfillcolor{textcolor}%
\pgftext[x=0.357639in, y=1.144963in, left, base]{\color{textcolor}\rmfamily\fontsize{8.000000}{9.600000}\selectfont \(\displaystyle {20}\)}%
\end{pgfscope}%
\begin{pgfscope}%
\pgfpathrectangle{\pgfqpoint{0.572918in}{0.553781in}}{\pgfqpoint{5.478282in}{2.095553in}}%
\pgfusepath{clip}%
\pgfsetbuttcap%
\pgfsetroundjoin%
\pgfsetlinewidth{0.501875pt}%
\definecolor{currentstroke}{rgb}{0.690196,0.690196,0.690196}%
\pgfsetstrokecolor{currentstroke}%
\pgfsetstrokeopacity{0.500000}%
\pgfsetdash{{0.500000pt}{0.825000pt}}{0.000000pt}%
\pgfpathmoveto{\pgfqpoint{0.572918in}{1.490972in}}%
\pgfpathlineto{\pgfqpoint{6.051200in}{1.490972in}}%
\pgfusepath{stroke}%
\end{pgfscope}%
\begin{pgfscope}%
\pgfsetbuttcap%
\pgfsetroundjoin%
\definecolor{currentfill}{rgb}{0.000000,0.000000,0.000000}%
\pgfsetfillcolor{currentfill}%
\pgfsetlinewidth{0.803000pt}%
\definecolor{currentstroke}{rgb}{0.000000,0.000000,0.000000}%
\pgfsetstrokecolor{currentstroke}%
\pgfsetdash{}{0pt}%
\pgfsys@defobject{currentmarker}{\pgfqpoint{-0.048611in}{0.000000in}}{\pgfqpoint{-0.000000in}{0.000000in}}{%
\pgfpathmoveto{\pgfqpoint{-0.000000in}{0.000000in}}%
\pgfpathlineto{\pgfqpoint{-0.048611in}{0.000000in}}%
\pgfusepath{stroke,fill}%
}%
\begin{pgfscope}%
\pgfsys@transformshift{0.572918in}{1.490972in}%
\pgfsys@useobject{currentmarker}{}%
\end{pgfscope}%
\end{pgfscope}%
\begin{pgfscope}%
\definecolor{textcolor}{rgb}{0.000000,0.000000,0.000000}%
\pgfsetstrokecolor{textcolor}%
\pgfsetfillcolor{textcolor}%
\pgftext[x=0.357639in, y=1.452416in, left, base]{\color{textcolor}\rmfamily\fontsize{8.000000}{9.600000}\selectfont \(\displaystyle {25}\)}%
\end{pgfscope}%
\begin{pgfscope}%
\pgfpathrectangle{\pgfqpoint{0.572918in}{0.553781in}}{\pgfqpoint{5.478282in}{2.095553in}}%
\pgfusepath{clip}%
\pgfsetbuttcap%
\pgfsetroundjoin%
\pgfsetlinewidth{0.501875pt}%
\definecolor{currentstroke}{rgb}{0.690196,0.690196,0.690196}%
\pgfsetstrokecolor{currentstroke}%
\pgfsetstrokeopacity{0.500000}%
\pgfsetdash{{0.500000pt}{0.825000pt}}{0.000000pt}%
\pgfpathmoveto{\pgfqpoint{0.572918in}{1.798424in}}%
\pgfpathlineto{\pgfqpoint{6.051200in}{1.798424in}}%
\pgfusepath{stroke}%
\end{pgfscope}%
\begin{pgfscope}%
\pgfsetbuttcap%
\pgfsetroundjoin%
\definecolor{currentfill}{rgb}{0.000000,0.000000,0.000000}%
\pgfsetfillcolor{currentfill}%
\pgfsetlinewidth{0.803000pt}%
\definecolor{currentstroke}{rgb}{0.000000,0.000000,0.000000}%
\pgfsetstrokecolor{currentstroke}%
\pgfsetdash{}{0pt}%
\pgfsys@defobject{currentmarker}{\pgfqpoint{-0.048611in}{0.000000in}}{\pgfqpoint{-0.000000in}{0.000000in}}{%
\pgfpathmoveto{\pgfqpoint{-0.000000in}{0.000000in}}%
\pgfpathlineto{\pgfqpoint{-0.048611in}{0.000000in}}%
\pgfusepath{stroke,fill}%
}%
\begin{pgfscope}%
\pgfsys@transformshift{0.572918in}{1.798424in}%
\pgfsys@useobject{currentmarker}{}%
\end{pgfscope}%
\end{pgfscope}%
\begin{pgfscope}%
\definecolor{textcolor}{rgb}{0.000000,0.000000,0.000000}%
\pgfsetstrokecolor{textcolor}%
\pgfsetfillcolor{textcolor}%
\pgftext[x=0.357639in, y=1.759869in, left, base]{\color{textcolor}\rmfamily\fontsize{8.000000}{9.600000}\selectfont \(\displaystyle {30}\)}%
\end{pgfscope}%
\begin{pgfscope}%
\pgfpathrectangle{\pgfqpoint{0.572918in}{0.553781in}}{\pgfqpoint{5.478282in}{2.095553in}}%
\pgfusepath{clip}%
\pgfsetbuttcap%
\pgfsetroundjoin%
\pgfsetlinewidth{0.501875pt}%
\definecolor{currentstroke}{rgb}{0.690196,0.690196,0.690196}%
\pgfsetstrokecolor{currentstroke}%
\pgfsetstrokeopacity{0.500000}%
\pgfsetdash{{0.500000pt}{0.825000pt}}{0.000000pt}%
\pgfpathmoveto{\pgfqpoint{0.572918in}{2.105877in}}%
\pgfpathlineto{\pgfqpoint{6.051200in}{2.105877in}}%
\pgfusepath{stroke}%
\end{pgfscope}%
\begin{pgfscope}%
\pgfsetbuttcap%
\pgfsetroundjoin%
\definecolor{currentfill}{rgb}{0.000000,0.000000,0.000000}%
\pgfsetfillcolor{currentfill}%
\pgfsetlinewidth{0.803000pt}%
\definecolor{currentstroke}{rgb}{0.000000,0.000000,0.000000}%
\pgfsetstrokecolor{currentstroke}%
\pgfsetdash{}{0pt}%
\pgfsys@defobject{currentmarker}{\pgfqpoint{-0.048611in}{0.000000in}}{\pgfqpoint{-0.000000in}{0.000000in}}{%
\pgfpathmoveto{\pgfqpoint{-0.000000in}{0.000000in}}%
\pgfpathlineto{\pgfqpoint{-0.048611in}{0.000000in}}%
\pgfusepath{stroke,fill}%
}%
\begin{pgfscope}%
\pgfsys@transformshift{0.572918in}{2.105877in}%
\pgfsys@useobject{currentmarker}{}%
\end{pgfscope}%
\end{pgfscope}%
\begin{pgfscope}%
\definecolor{textcolor}{rgb}{0.000000,0.000000,0.000000}%
\pgfsetstrokecolor{textcolor}%
\pgfsetfillcolor{textcolor}%
\pgftext[x=0.357639in, y=2.067321in, left, base]{\color{textcolor}\rmfamily\fontsize{8.000000}{9.600000}\selectfont \(\displaystyle {35}\)}%
\end{pgfscope}%
\begin{pgfscope}%
\pgfpathrectangle{\pgfqpoint{0.572918in}{0.553781in}}{\pgfqpoint{5.478282in}{2.095553in}}%
\pgfusepath{clip}%
\pgfsetbuttcap%
\pgfsetroundjoin%
\pgfsetlinewidth{0.501875pt}%
\definecolor{currentstroke}{rgb}{0.690196,0.690196,0.690196}%
\pgfsetstrokecolor{currentstroke}%
\pgfsetstrokeopacity{0.500000}%
\pgfsetdash{{0.500000pt}{0.825000pt}}{0.000000pt}%
\pgfpathmoveto{\pgfqpoint{0.572918in}{2.413330in}}%
\pgfpathlineto{\pgfqpoint{6.051200in}{2.413330in}}%
\pgfusepath{stroke}%
\end{pgfscope}%
\begin{pgfscope}%
\pgfsetbuttcap%
\pgfsetroundjoin%
\definecolor{currentfill}{rgb}{0.000000,0.000000,0.000000}%
\pgfsetfillcolor{currentfill}%
\pgfsetlinewidth{0.803000pt}%
\definecolor{currentstroke}{rgb}{0.000000,0.000000,0.000000}%
\pgfsetstrokecolor{currentstroke}%
\pgfsetdash{}{0pt}%
\pgfsys@defobject{currentmarker}{\pgfqpoint{-0.048611in}{0.000000in}}{\pgfqpoint{-0.000000in}{0.000000in}}{%
\pgfpathmoveto{\pgfqpoint{-0.000000in}{0.000000in}}%
\pgfpathlineto{\pgfqpoint{-0.048611in}{0.000000in}}%
\pgfusepath{stroke,fill}%
}%
\begin{pgfscope}%
\pgfsys@transformshift{0.572918in}{2.413330in}%
\pgfsys@useobject{currentmarker}{}%
\end{pgfscope}%
\end{pgfscope}%
\begin{pgfscope}%
\definecolor{textcolor}{rgb}{0.000000,0.000000,0.000000}%
\pgfsetstrokecolor{textcolor}%
\pgfsetfillcolor{textcolor}%
\pgftext[x=0.357639in, y=2.374774in, left, base]{\color{textcolor}\rmfamily\fontsize{8.000000}{9.600000}\selectfont \(\displaystyle {40}\)}%
\end{pgfscope}%
\begin{pgfscope}%
\definecolor{textcolor}{rgb}{0.000000,0.000000,0.000000}%
\pgfsetstrokecolor{textcolor}%
\pgfsetfillcolor{textcolor}%
\pgftext[x=0.302083in,y=1.601557in,,bottom,rotate=90.000000]{\color{textcolor}\rmfamily\fontsize{10.950000}{13.140000}\selectfont IQR / 1.349 \(\displaystyle \left[ \textup{deg} \right]\)}%
\end{pgfscope}%
\begin{pgfscope}%
\pgfpathrectangle{\pgfqpoint{0.572918in}{0.553781in}}{\pgfqpoint{5.478282in}{2.095553in}}%
\pgfusepath{clip}%
\pgfsetbuttcap%
\pgfsetroundjoin%
\pgfsetlinewidth{1.505625pt}%
\definecolor{currentstroke}{rgb}{0.313725,0.317647,0.309804}%
\pgfsetstrokecolor{currentstroke}%
\pgfsetstrokeopacity{0.900000}%
\pgfsetdash{}{0pt}%
\pgfpathmoveto{\pgfqpoint{0.821931in}{2.067272in}}%
\pgfpathlineto{\pgfqpoint{0.821931in}{2.554081in}}%
\pgfusepath{stroke}%
\end{pgfscope}%
\begin{pgfscope}%
\pgfpathrectangle{\pgfqpoint{0.572918in}{0.553781in}}{\pgfqpoint{5.478282in}{2.095553in}}%
\pgfusepath{clip}%
\pgfsetbuttcap%
\pgfsetroundjoin%
\pgfsetlinewidth{1.505625pt}%
\definecolor{currentstroke}{rgb}{0.313725,0.317647,0.309804}%
\pgfsetstrokecolor{currentstroke}%
\pgfsetstrokeopacity{0.900000}%
\pgfsetdash{}{0pt}%
\pgfpathmoveto{\pgfqpoint{1.114887in}{2.078877in}}%
\pgfpathlineto{\pgfqpoint{1.114887in}{2.314335in}}%
\pgfusepath{stroke}%
\end{pgfscope}%
\begin{pgfscope}%
\pgfpathrectangle{\pgfqpoint{0.572918in}{0.553781in}}{\pgfqpoint{5.478282in}{2.095553in}}%
\pgfusepath{clip}%
\pgfsetbuttcap%
\pgfsetroundjoin%
\pgfsetlinewidth{1.505625pt}%
\definecolor{currentstroke}{rgb}{0.313725,0.317647,0.309804}%
\pgfsetstrokecolor{currentstroke}%
\pgfsetstrokeopacity{0.900000}%
\pgfsetdash{}{0pt}%
\pgfpathmoveto{\pgfqpoint{1.407844in}{2.087960in}}%
\pgfpathlineto{\pgfqpoint{1.407844in}{2.242339in}}%
\pgfusepath{stroke}%
\end{pgfscope}%
\begin{pgfscope}%
\pgfpathrectangle{\pgfqpoint{0.572918in}{0.553781in}}{\pgfqpoint{5.478282in}{2.095553in}}%
\pgfusepath{clip}%
\pgfsetbuttcap%
\pgfsetroundjoin%
\pgfsetlinewidth{1.505625pt}%
\definecolor{currentstroke}{rgb}{0.313725,0.317647,0.309804}%
\pgfsetstrokecolor{currentstroke}%
\pgfsetstrokeopacity{0.900000}%
\pgfsetdash{}{0pt}%
\pgfpathmoveto{\pgfqpoint{1.700800in}{2.188029in}}%
\pgfpathlineto{\pgfqpoint{1.700800in}{2.306855in}}%
\pgfusepath{stroke}%
\end{pgfscope}%
\begin{pgfscope}%
\pgfpathrectangle{\pgfqpoint{0.572918in}{0.553781in}}{\pgfqpoint{5.478282in}{2.095553in}}%
\pgfusepath{clip}%
\pgfsetbuttcap%
\pgfsetroundjoin%
\pgfsetlinewidth{1.505625pt}%
\definecolor{currentstroke}{rgb}{0.313725,0.317647,0.309804}%
\pgfsetstrokecolor{currentstroke}%
\pgfsetstrokeopacity{0.900000}%
\pgfsetdash{}{0pt}%
\pgfpathmoveto{\pgfqpoint{1.993756in}{2.214038in}}%
\pgfpathlineto{\pgfqpoint{1.993756in}{2.306567in}}%
\pgfusepath{stroke}%
\end{pgfscope}%
\begin{pgfscope}%
\pgfpathrectangle{\pgfqpoint{0.572918in}{0.553781in}}{\pgfqpoint{5.478282in}{2.095553in}}%
\pgfusepath{clip}%
\pgfsetbuttcap%
\pgfsetroundjoin%
\pgfsetlinewidth{1.505625pt}%
\definecolor{currentstroke}{rgb}{0.313725,0.317647,0.309804}%
\pgfsetstrokecolor{currentstroke}%
\pgfsetstrokeopacity{0.900000}%
\pgfsetdash{}{0pt}%
\pgfpathmoveto{\pgfqpoint{2.286712in}{2.196951in}}%
\pgfpathlineto{\pgfqpoint{2.286712in}{2.278266in}}%
\pgfusepath{stroke}%
\end{pgfscope}%
\begin{pgfscope}%
\pgfpathrectangle{\pgfqpoint{0.572918in}{0.553781in}}{\pgfqpoint{5.478282in}{2.095553in}}%
\pgfusepath{clip}%
\pgfsetbuttcap%
\pgfsetroundjoin%
\pgfsetlinewidth{1.505625pt}%
\definecolor{currentstroke}{rgb}{0.313725,0.317647,0.309804}%
\pgfsetstrokecolor{currentstroke}%
\pgfsetstrokeopacity{0.900000}%
\pgfsetdash{}{0pt}%
\pgfpathmoveto{\pgfqpoint{2.579669in}{1.988684in}}%
\pgfpathlineto{\pgfqpoint{2.579669in}{2.050211in}}%
\pgfusepath{stroke}%
\end{pgfscope}%
\begin{pgfscope}%
\pgfpathrectangle{\pgfqpoint{0.572918in}{0.553781in}}{\pgfqpoint{5.478282in}{2.095553in}}%
\pgfusepath{clip}%
\pgfsetbuttcap%
\pgfsetroundjoin%
\pgfsetlinewidth{1.505625pt}%
\definecolor{currentstroke}{rgb}{0.313725,0.317647,0.309804}%
\pgfsetstrokecolor{currentstroke}%
\pgfsetstrokeopacity{0.900000}%
\pgfsetdash{}{0pt}%
\pgfpathmoveto{\pgfqpoint{2.872625in}{1.831657in}}%
\pgfpathlineto{\pgfqpoint{2.872625in}{1.892051in}}%
\pgfusepath{stroke}%
\end{pgfscope}%
\begin{pgfscope}%
\pgfpathrectangle{\pgfqpoint{0.572918in}{0.553781in}}{\pgfqpoint{5.478282in}{2.095553in}}%
\pgfusepath{clip}%
\pgfsetbuttcap%
\pgfsetroundjoin%
\pgfsetlinewidth{1.505625pt}%
\definecolor{currentstroke}{rgb}{0.313725,0.317647,0.309804}%
\pgfsetstrokecolor{currentstroke}%
\pgfsetstrokeopacity{0.900000}%
\pgfsetdash{}{0pt}%
\pgfpathmoveto{\pgfqpoint{3.165581in}{1.589762in}}%
\pgfpathlineto{\pgfqpoint{3.165581in}{1.637771in}}%
\pgfusepath{stroke}%
\end{pgfscope}%
\begin{pgfscope}%
\pgfpathrectangle{\pgfqpoint{0.572918in}{0.553781in}}{\pgfqpoint{5.478282in}{2.095553in}}%
\pgfusepath{clip}%
\pgfsetbuttcap%
\pgfsetroundjoin%
\pgfsetlinewidth{1.505625pt}%
\definecolor{currentstroke}{rgb}{0.313725,0.317647,0.309804}%
\pgfsetstrokecolor{currentstroke}%
\pgfsetstrokeopacity{0.900000}%
\pgfsetdash{}{0pt}%
\pgfpathmoveto{\pgfqpoint{3.458537in}{1.393828in}}%
\pgfpathlineto{\pgfqpoint{3.458537in}{1.439392in}}%
\pgfusepath{stroke}%
\end{pgfscope}%
\begin{pgfscope}%
\pgfpathrectangle{\pgfqpoint{0.572918in}{0.553781in}}{\pgfqpoint{5.478282in}{2.095553in}}%
\pgfusepath{clip}%
\pgfsetbuttcap%
\pgfsetroundjoin%
\pgfsetlinewidth{1.505625pt}%
\definecolor{currentstroke}{rgb}{0.313725,0.317647,0.309804}%
\pgfsetstrokecolor{currentstroke}%
\pgfsetstrokeopacity{0.900000}%
\pgfsetdash{}{0pt}%
\pgfpathmoveto{\pgfqpoint{3.751494in}{1.161673in}}%
\pgfpathlineto{\pgfqpoint{3.751494in}{1.203739in}}%
\pgfusepath{stroke}%
\end{pgfscope}%
\begin{pgfscope}%
\pgfpathrectangle{\pgfqpoint{0.572918in}{0.553781in}}{\pgfqpoint{5.478282in}{2.095553in}}%
\pgfusepath{clip}%
\pgfsetbuttcap%
\pgfsetroundjoin%
\pgfsetlinewidth{1.505625pt}%
\definecolor{currentstroke}{rgb}{0.313725,0.317647,0.309804}%
\pgfsetstrokecolor{currentstroke}%
\pgfsetstrokeopacity{0.900000}%
\pgfsetdash{}{0pt}%
\pgfpathmoveto{\pgfqpoint{4.044450in}{1.026004in}}%
\pgfpathlineto{\pgfqpoint{4.044450in}{1.072266in}}%
\pgfusepath{stroke}%
\end{pgfscope}%
\begin{pgfscope}%
\pgfpathrectangle{\pgfqpoint{0.572918in}{0.553781in}}{\pgfqpoint{5.478282in}{2.095553in}}%
\pgfusepath{clip}%
\pgfsetbuttcap%
\pgfsetroundjoin%
\pgfsetlinewidth{1.505625pt}%
\definecolor{currentstroke}{rgb}{0.313725,0.317647,0.309804}%
\pgfsetstrokecolor{currentstroke}%
\pgfsetstrokeopacity{0.900000}%
\pgfsetdash{}{0pt}%
\pgfpathmoveto{\pgfqpoint{4.337406in}{0.878054in}}%
\pgfpathlineto{\pgfqpoint{4.337406in}{0.922528in}}%
\pgfusepath{stroke}%
\end{pgfscope}%
\begin{pgfscope}%
\pgfpathrectangle{\pgfqpoint{0.572918in}{0.553781in}}{\pgfqpoint{5.478282in}{2.095553in}}%
\pgfusepath{clip}%
\pgfsetbuttcap%
\pgfsetroundjoin%
\pgfsetlinewidth{1.505625pt}%
\definecolor{currentstroke}{rgb}{0.313725,0.317647,0.309804}%
\pgfsetstrokecolor{currentstroke}%
\pgfsetstrokeopacity{0.900000}%
\pgfsetdash{}{0pt}%
\pgfpathmoveto{\pgfqpoint{4.630362in}{0.785324in}}%
\pgfpathlineto{\pgfqpoint{4.630362in}{0.831559in}}%
\pgfusepath{stroke}%
\end{pgfscope}%
\begin{pgfscope}%
\pgfpathrectangle{\pgfqpoint{0.572918in}{0.553781in}}{\pgfqpoint{5.478282in}{2.095553in}}%
\pgfusepath{clip}%
\pgfsetbuttcap%
\pgfsetroundjoin%
\pgfsetlinewidth{1.505625pt}%
\definecolor{currentstroke}{rgb}{0.313725,0.317647,0.309804}%
\pgfsetstrokecolor{currentstroke}%
\pgfsetstrokeopacity{0.900000}%
\pgfsetdash{}{0pt}%
\pgfpathmoveto{\pgfqpoint{4.923318in}{0.707399in}}%
\pgfpathlineto{\pgfqpoint{4.923318in}{0.755435in}}%
\pgfusepath{stroke}%
\end{pgfscope}%
\begin{pgfscope}%
\pgfpathrectangle{\pgfqpoint{0.572918in}{0.553781in}}{\pgfqpoint{5.478282in}{2.095553in}}%
\pgfusepath{clip}%
\pgfsetbuttcap%
\pgfsetroundjoin%
\pgfsetlinewidth{1.505625pt}%
\definecolor{currentstroke}{rgb}{0.313725,0.317647,0.309804}%
\pgfsetstrokecolor{currentstroke}%
\pgfsetstrokeopacity{0.900000}%
\pgfsetdash{}{0pt}%
\pgfpathmoveto{\pgfqpoint{5.216275in}{0.697970in}}%
\pgfpathlineto{\pgfqpoint{5.216275in}{0.761403in}}%
\pgfusepath{stroke}%
\end{pgfscope}%
\begin{pgfscope}%
\pgfpathrectangle{\pgfqpoint{0.572918in}{0.553781in}}{\pgfqpoint{5.478282in}{2.095553in}}%
\pgfusepath{clip}%
\pgfsetbuttcap%
\pgfsetroundjoin%
\pgfsetlinewidth{1.505625pt}%
\definecolor{currentstroke}{rgb}{0.313725,0.317647,0.309804}%
\pgfsetstrokecolor{currentstroke}%
\pgfsetstrokeopacity{0.900000}%
\pgfsetdash{}{0pt}%
\pgfpathmoveto{\pgfqpoint{5.509231in}{0.649033in}}%
\pgfpathlineto{\pgfqpoint{5.509231in}{0.713958in}}%
\pgfusepath{stroke}%
\end{pgfscope}%
\begin{pgfscope}%
\pgfpathrectangle{\pgfqpoint{0.572918in}{0.553781in}}{\pgfqpoint{5.478282in}{2.095553in}}%
\pgfusepath{clip}%
\pgfsetbuttcap%
\pgfsetroundjoin%
\pgfsetlinewidth{1.505625pt}%
\definecolor{currentstroke}{rgb}{0.313725,0.317647,0.309804}%
\pgfsetstrokecolor{currentstroke}%
\pgfsetstrokeopacity{0.900000}%
\pgfsetdash{}{0pt}%
\pgfpathmoveto{\pgfqpoint{5.802187in}{0.655468in}}%
\pgfpathlineto{\pgfqpoint{5.802187in}{0.750982in}}%
\pgfusepath{stroke}%
\end{pgfscope}%
\begin{pgfscope}%
\pgfpathrectangle{\pgfqpoint{0.572918in}{0.553781in}}{\pgfqpoint{5.478282in}{2.095553in}}%
\pgfusepath{clip}%
\pgfsetbuttcap%
\pgfsetroundjoin%
\pgfsetlinewidth{1.505625pt}%
\definecolor{currentstroke}{rgb}{0.949020,0.372549,0.360784}%
\pgfsetstrokecolor{currentstroke}%
\pgfsetstrokeopacity{0.900000}%
\pgfsetdash{}{0pt}%
\pgfpathmoveto{\pgfqpoint{0.821931in}{1.610412in}}%
\pgfpathlineto{\pgfqpoint{0.821931in}{2.081453in}}%
\pgfusepath{stroke}%
\end{pgfscope}%
\begin{pgfscope}%
\pgfpathrectangle{\pgfqpoint{0.572918in}{0.553781in}}{\pgfqpoint{5.478282in}{2.095553in}}%
\pgfusepath{clip}%
\pgfsetbuttcap%
\pgfsetroundjoin%
\pgfsetlinewidth{1.505625pt}%
\definecolor{currentstroke}{rgb}{0.949020,0.372549,0.360784}%
\pgfsetstrokecolor{currentstroke}%
\pgfsetstrokeopacity{0.900000}%
\pgfsetdash{}{0pt}%
\pgfpathmoveto{\pgfqpoint{1.114887in}{1.590032in}}%
\pgfpathlineto{\pgfqpoint{1.114887in}{1.805012in}}%
\pgfusepath{stroke}%
\end{pgfscope}%
\begin{pgfscope}%
\pgfpathrectangle{\pgfqpoint{0.572918in}{0.553781in}}{\pgfqpoint{5.478282in}{2.095553in}}%
\pgfusepath{clip}%
\pgfsetbuttcap%
\pgfsetroundjoin%
\pgfsetlinewidth{1.505625pt}%
\definecolor{currentstroke}{rgb}{0.949020,0.372549,0.360784}%
\pgfsetstrokecolor{currentstroke}%
\pgfsetstrokeopacity{0.900000}%
\pgfsetdash{}{0pt}%
\pgfpathmoveto{\pgfqpoint{1.407844in}{1.639922in}}%
\pgfpathlineto{\pgfqpoint{1.407844in}{1.746536in}}%
\pgfusepath{stroke}%
\end{pgfscope}%
\begin{pgfscope}%
\pgfpathrectangle{\pgfqpoint{0.572918in}{0.553781in}}{\pgfqpoint{5.478282in}{2.095553in}}%
\pgfusepath{clip}%
\pgfsetbuttcap%
\pgfsetroundjoin%
\pgfsetlinewidth{1.505625pt}%
\definecolor{currentstroke}{rgb}{0.949020,0.372549,0.360784}%
\pgfsetstrokecolor{currentstroke}%
\pgfsetstrokeopacity{0.900000}%
\pgfsetdash{}{0pt}%
\pgfpathmoveto{\pgfqpoint{1.700800in}{1.710074in}}%
\pgfpathlineto{\pgfqpoint{1.700800in}{1.793898in}}%
\pgfusepath{stroke}%
\end{pgfscope}%
\begin{pgfscope}%
\pgfpathrectangle{\pgfqpoint{0.572918in}{0.553781in}}{\pgfqpoint{5.478282in}{2.095553in}}%
\pgfusepath{clip}%
\pgfsetbuttcap%
\pgfsetroundjoin%
\pgfsetlinewidth{1.505625pt}%
\definecolor{currentstroke}{rgb}{0.949020,0.372549,0.360784}%
\pgfsetstrokecolor{currentstroke}%
\pgfsetstrokeopacity{0.900000}%
\pgfsetdash{}{0pt}%
\pgfpathmoveto{\pgfqpoint{1.993756in}{1.789830in}}%
\pgfpathlineto{\pgfqpoint{1.993756in}{1.855864in}}%
\pgfusepath{stroke}%
\end{pgfscope}%
\begin{pgfscope}%
\pgfpathrectangle{\pgfqpoint{0.572918in}{0.553781in}}{\pgfqpoint{5.478282in}{2.095553in}}%
\pgfusepath{clip}%
\pgfsetbuttcap%
\pgfsetroundjoin%
\pgfsetlinewidth{1.505625pt}%
\definecolor{currentstroke}{rgb}{0.949020,0.372549,0.360784}%
\pgfsetstrokecolor{currentstroke}%
\pgfsetstrokeopacity{0.900000}%
\pgfsetdash{}{0pt}%
\pgfpathmoveto{\pgfqpoint{2.286712in}{1.791563in}}%
\pgfpathlineto{\pgfqpoint{2.286712in}{1.844556in}}%
\pgfusepath{stroke}%
\end{pgfscope}%
\begin{pgfscope}%
\pgfpathrectangle{\pgfqpoint{0.572918in}{0.553781in}}{\pgfqpoint{5.478282in}{2.095553in}}%
\pgfusepath{clip}%
\pgfsetbuttcap%
\pgfsetroundjoin%
\pgfsetlinewidth{1.505625pt}%
\definecolor{currentstroke}{rgb}{0.949020,0.372549,0.360784}%
\pgfsetstrokecolor{currentstroke}%
\pgfsetstrokeopacity{0.900000}%
\pgfsetdash{}{0pt}%
\pgfpathmoveto{\pgfqpoint{2.579669in}{1.700949in}}%
\pgfpathlineto{\pgfqpoint{2.579669in}{1.749923in}}%
\pgfusepath{stroke}%
\end{pgfscope}%
\begin{pgfscope}%
\pgfpathrectangle{\pgfqpoint{0.572918in}{0.553781in}}{\pgfqpoint{5.478282in}{2.095553in}}%
\pgfusepath{clip}%
\pgfsetbuttcap%
\pgfsetroundjoin%
\pgfsetlinewidth{1.505625pt}%
\definecolor{currentstroke}{rgb}{0.949020,0.372549,0.360784}%
\pgfsetstrokecolor{currentstroke}%
\pgfsetstrokeopacity{0.900000}%
\pgfsetdash{}{0pt}%
\pgfpathmoveto{\pgfqpoint{2.872625in}{1.660783in}}%
\pgfpathlineto{\pgfqpoint{2.872625in}{1.712713in}}%
\pgfusepath{stroke}%
\end{pgfscope}%
\begin{pgfscope}%
\pgfpathrectangle{\pgfqpoint{0.572918in}{0.553781in}}{\pgfqpoint{5.478282in}{2.095553in}}%
\pgfusepath{clip}%
\pgfsetbuttcap%
\pgfsetroundjoin%
\pgfsetlinewidth{1.505625pt}%
\definecolor{currentstroke}{rgb}{0.949020,0.372549,0.360784}%
\pgfsetstrokecolor{currentstroke}%
\pgfsetstrokeopacity{0.900000}%
\pgfsetdash{}{0pt}%
\pgfpathmoveto{\pgfqpoint{3.165581in}{1.543644in}}%
\pgfpathlineto{\pgfqpoint{3.165581in}{1.588539in}}%
\pgfusepath{stroke}%
\end{pgfscope}%
\begin{pgfscope}%
\pgfpathrectangle{\pgfqpoint{0.572918in}{0.553781in}}{\pgfqpoint{5.478282in}{2.095553in}}%
\pgfusepath{clip}%
\pgfsetbuttcap%
\pgfsetroundjoin%
\pgfsetlinewidth{1.505625pt}%
\definecolor{currentstroke}{rgb}{0.949020,0.372549,0.360784}%
\pgfsetstrokecolor{currentstroke}%
\pgfsetstrokeopacity{0.900000}%
\pgfsetdash{}{0pt}%
\pgfpathmoveto{\pgfqpoint{3.458537in}{1.456182in}}%
\pgfpathlineto{\pgfqpoint{3.458537in}{1.500934in}}%
\pgfusepath{stroke}%
\end{pgfscope}%
\begin{pgfscope}%
\pgfpathrectangle{\pgfqpoint{0.572918in}{0.553781in}}{\pgfqpoint{5.478282in}{2.095553in}}%
\pgfusepath{clip}%
\pgfsetbuttcap%
\pgfsetroundjoin%
\pgfsetlinewidth{1.505625pt}%
\definecolor{currentstroke}{rgb}{0.949020,0.372549,0.360784}%
\pgfsetstrokecolor{currentstroke}%
\pgfsetstrokeopacity{0.900000}%
\pgfsetdash{}{0pt}%
\pgfpathmoveto{\pgfqpoint{3.751494in}{1.340109in}}%
\pgfpathlineto{\pgfqpoint{3.751494in}{1.382285in}}%
\pgfusepath{stroke}%
\end{pgfscope}%
\begin{pgfscope}%
\pgfpathrectangle{\pgfqpoint{0.572918in}{0.553781in}}{\pgfqpoint{5.478282in}{2.095553in}}%
\pgfusepath{clip}%
\pgfsetbuttcap%
\pgfsetroundjoin%
\pgfsetlinewidth{1.505625pt}%
\definecolor{currentstroke}{rgb}{0.949020,0.372549,0.360784}%
\pgfsetstrokecolor{currentstroke}%
\pgfsetstrokeopacity{0.900000}%
\pgfsetdash{}{0pt}%
\pgfpathmoveto{\pgfqpoint{4.044450in}{1.298949in}}%
\pgfpathlineto{\pgfqpoint{4.044450in}{1.348791in}}%
\pgfusepath{stroke}%
\end{pgfscope}%
\begin{pgfscope}%
\pgfpathrectangle{\pgfqpoint{0.572918in}{0.553781in}}{\pgfqpoint{5.478282in}{2.095553in}}%
\pgfusepath{clip}%
\pgfsetbuttcap%
\pgfsetroundjoin%
\pgfsetlinewidth{1.505625pt}%
\definecolor{currentstroke}{rgb}{0.949020,0.372549,0.360784}%
\pgfsetstrokecolor{currentstroke}%
\pgfsetstrokeopacity{0.900000}%
\pgfsetdash{}{0pt}%
\pgfpathmoveto{\pgfqpoint{4.337406in}{1.177427in}}%
\pgfpathlineto{\pgfqpoint{4.337406in}{1.238783in}}%
\pgfusepath{stroke}%
\end{pgfscope}%
\begin{pgfscope}%
\pgfpathrectangle{\pgfqpoint{0.572918in}{0.553781in}}{\pgfqpoint{5.478282in}{2.095553in}}%
\pgfusepath{clip}%
\pgfsetbuttcap%
\pgfsetroundjoin%
\pgfsetlinewidth{1.505625pt}%
\definecolor{currentstroke}{rgb}{0.949020,0.372549,0.360784}%
\pgfsetstrokecolor{currentstroke}%
\pgfsetstrokeopacity{0.900000}%
\pgfsetdash{}{0pt}%
\pgfpathmoveto{\pgfqpoint{4.630362in}{1.149333in}}%
\pgfpathlineto{\pgfqpoint{4.630362in}{1.216842in}}%
\pgfusepath{stroke}%
\end{pgfscope}%
\begin{pgfscope}%
\pgfpathrectangle{\pgfqpoint{0.572918in}{0.553781in}}{\pgfqpoint{5.478282in}{2.095553in}}%
\pgfusepath{clip}%
\pgfsetbuttcap%
\pgfsetroundjoin%
\pgfsetlinewidth{1.505625pt}%
\definecolor{currentstroke}{rgb}{0.949020,0.372549,0.360784}%
\pgfsetstrokecolor{currentstroke}%
\pgfsetstrokeopacity{0.900000}%
\pgfsetdash{}{0pt}%
\pgfpathmoveto{\pgfqpoint{4.923318in}{1.062965in}}%
\pgfpathlineto{\pgfqpoint{4.923318in}{1.135601in}}%
\pgfusepath{stroke}%
\end{pgfscope}%
\begin{pgfscope}%
\pgfpathrectangle{\pgfqpoint{0.572918in}{0.553781in}}{\pgfqpoint{5.478282in}{2.095553in}}%
\pgfusepath{clip}%
\pgfsetbuttcap%
\pgfsetroundjoin%
\pgfsetlinewidth{1.505625pt}%
\definecolor{currentstroke}{rgb}{0.949020,0.372549,0.360784}%
\pgfsetstrokecolor{currentstroke}%
\pgfsetstrokeopacity{0.900000}%
\pgfsetdash{}{0pt}%
\pgfpathmoveto{\pgfqpoint{5.216275in}{1.102604in}}%
\pgfpathlineto{\pgfqpoint{5.216275in}{1.202850in}}%
\pgfusepath{stroke}%
\end{pgfscope}%
\begin{pgfscope}%
\pgfpathrectangle{\pgfqpoint{0.572918in}{0.553781in}}{\pgfqpoint{5.478282in}{2.095553in}}%
\pgfusepath{clip}%
\pgfsetbuttcap%
\pgfsetroundjoin%
\pgfsetlinewidth{1.505625pt}%
\definecolor{currentstroke}{rgb}{0.949020,0.372549,0.360784}%
\pgfsetstrokecolor{currentstroke}%
\pgfsetstrokeopacity{0.900000}%
\pgfsetdash{}{0pt}%
\pgfpathmoveto{\pgfqpoint{5.509231in}{1.071548in}}%
\pgfpathlineto{\pgfqpoint{5.509231in}{1.198276in}}%
\pgfusepath{stroke}%
\end{pgfscope}%
\begin{pgfscope}%
\pgfpathrectangle{\pgfqpoint{0.572918in}{0.553781in}}{\pgfqpoint{5.478282in}{2.095553in}}%
\pgfusepath{clip}%
\pgfsetbuttcap%
\pgfsetroundjoin%
\pgfsetlinewidth{1.505625pt}%
\definecolor{currentstroke}{rgb}{0.949020,0.372549,0.360784}%
\pgfsetstrokecolor{currentstroke}%
\pgfsetstrokeopacity{0.900000}%
\pgfsetdash{}{0pt}%
\pgfpathmoveto{\pgfqpoint{5.802187in}{1.088207in}}%
\pgfpathlineto{\pgfqpoint{5.802187in}{1.262489in}}%
\pgfusepath{stroke}%
\end{pgfscope}%
\begin{pgfscope}%
\pgfpathrectangle{\pgfqpoint{0.572918in}{0.553781in}}{\pgfqpoint{5.478282in}{2.095553in}}%
\pgfusepath{clip}%
\pgfsetbuttcap%
\pgfsetroundjoin%
\pgfsetlinewidth{1.505625pt}%
\definecolor{currentstroke}{rgb}{1.000000,0.819608,0.101961}%
\pgfsetstrokecolor{currentstroke}%
\pgfsetstrokeopacity{0.900000}%
\pgfsetdash{}{0pt}%
\pgfpathmoveto{\pgfqpoint{0.821931in}{1.500982in}}%
\pgfpathlineto{\pgfqpoint{0.821931in}{1.891902in}}%
\pgfusepath{stroke}%
\end{pgfscope}%
\begin{pgfscope}%
\pgfpathrectangle{\pgfqpoint{0.572918in}{0.553781in}}{\pgfqpoint{5.478282in}{2.095553in}}%
\pgfusepath{clip}%
\pgfsetbuttcap%
\pgfsetroundjoin%
\pgfsetlinewidth{1.505625pt}%
\definecolor{currentstroke}{rgb}{1.000000,0.819608,0.101961}%
\pgfsetstrokecolor{currentstroke}%
\pgfsetstrokeopacity{0.900000}%
\pgfsetdash{}{0pt}%
\pgfpathmoveto{\pgfqpoint{1.114887in}{1.534942in}}%
\pgfpathlineto{\pgfqpoint{1.114887in}{1.701759in}}%
\pgfusepath{stroke}%
\end{pgfscope}%
\begin{pgfscope}%
\pgfpathrectangle{\pgfqpoint{0.572918in}{0.553781in}}{\pgfqpoint{5.478282in}{2.095553in}}%
\pgfusepath{clip}%
\pgfsetbuttcap%
\pgfsetroundjoin%
\pgfsetlinewidth{1.505625pt}%
\definecolor{currentstroke}{rgb}{1.000000,0.819608,0.101961}%
\pgfsetstrokecolor{currentstroke}%
\pgfsetstrokeopacity{0.900000}%
\pgfsetdash{}{0pt}%
\pgfpathmoveto{\pgfqpoint{1.407844in}{1.567679in}}%
\pgfpathlineto{\pgfqpoint{1.407844in}{1.668585in}}%
\pgfusepath{stroke}%
\end{pgfscope}%
\begin{pgfscope}%
\pgfpathrectangle{\pgfqpoint{0.572918in}{0.553781in}}{\pgfqpoint{5.478282in}{2.095553in}}%
\pgfusepath{clip}%
\pgfsetbuttcap%
\pgfsetroundjoin%
\pgfsetlinewidth{1.505625pt}%
\definecolor{currentstroke}{rgb}{1.000000,0.819608,0.101961}%
\pgfsetstrokecolor{currentstroke}%
\pgfsetstrokeopacity{0.900000}%
\pgfsetdash{}{0pt}%
\pgfpathmoveto{\pgfqpoint{1.700800in}{1.596455in}}%
\pgfpathlineto{\pgfqpoint{1.700800in}{1.671557in}}%
\pgfusepath{stroke}%
\end{pgfscope}%
\begin{pgfscope}%
\pgfpathrectangle{\pgfqpoint{0.572918in}{0.553781in}}{\pgfqpoint{5.478282in}{2.095553in}}%
\pgfusepath{clip}%
\pgfsetbuttcap%
\pgfsetroundjoin%
\pgfsetlinewidth{1.505625pt}%
\definecolor{currentstroke}{rgb}{1.000000,0.819608,0.101961}%
\pgfsetstrokecolor{currentstroke}%
\pgfsetstrokeopacity{0.900000}%
\pgfsetdash{}{0pt}%
\pgfpathmoveto{\pgfqpoint{1.993756in}{1.599015in}}%
\pgfpathlineto{\pgfqpoint{1.993756in}{1.658482in}}%
\pgfusepath{stroke}%
\end{pgfscope}%
\begin{pgfscope}%
\pgfpathrectangle{\pgfqpoint{0.572918in}{0.553781in}}{\pgfqpoint{5.478282in}{2.095553in}}%
\pgfusepath{clip}%
\pgfsetbuttcap%
\pgfsetroundjoin%
\pgfsetlinewidth{1.505625pt}%
\definecolor{currentstroke}{rgb}{1.000000,0.819608,0.101961}%
\pgfsetstrokecolor{currentstroke}%
\pgfsetstrokeopacity{0.900000}%
\pgfsetdash{}{0pt}%
\pgfpathmoveto{\pgfqpoint{2.286712in}{1.607213in}}%
\pgfpathlineto{\pgfqpoint{2.286712in}{1.658702in}}%
\pgfusepath{stroke}%
\end{pgfscope}%
\begin{pgfscope}%
\pgfpathrectangle{\pgfqpoint{0.572918in}{0.553781in}}{\pgfqpoint{5.478282in}{2.095553in}}%
\pgfusepath{clip}%
\pgfsetbuttcap%
\pgfsetroundjoin%
\pgfsetlinewidth{1.505625pt}%
\definecolor{currentstroke}{rgb}{1.000000,0.819608,0.101961}%
\pgfsetstrokecolor{currentstroke}%
\pgfsetstrokeopacity{0.900000}%
\pgfsetdash{}{0pt}%
\pgfpathmoveto{\pgfqpoint{2.579669in}{1.503533in}}%
\pgfpathlineto{\pgfqpoint{2.579669in}{1.546489in}}%
\pgfusepath{stroke}%
\end{pgfscope}%
\begin{pgfscope}%
\pgfpathrectangle{\pgfqpoint{0.572918in}{0.553781in}}{\pgfqpoint{5.478282in}{2.095553in}}%
\pgfusepath{clip}%
\pgfsetbuttcap%
\pgfsetroundjoin%
\pgfsetlinewidth{1.505625pt}%
\definecolor{currentstroke}{rgb}{1.000000,0.819608,0.101961}%
\pgfsetstrokecolor{currentstroke}%
\pgfsetstrokeopacity{0.900000}%
\pgfsetdash{}{0pt}%
\pgfpathmoveto{\pgfqpoint{2.872625in}{1.465870in}}%
\pgfpathlineto{\pgfqpoint{2.872625in}{1.509008in}}%
\pgfusepath{stroke}%
\end{pgfscope}%
\begin{pgfscope}%
\pgfpathrectangle{\pgfqpoint{0.572918in}{0.553781in}}{\pgfqpoint{5.478282in}{2.095553in}}%
\pgfusepath{clip}%
\pgfsetbuttcap%
\pgfsetroundjoin%
\pgfsetlinewidth{1.505625pt}%
\definecolor{currentstroke}{rgb}{1.000000,0.819608,0.101961}%
\pgfsetstrokecolor{currentstroke}%
\pgfsetstrokeopacity{0.900000}%
\pgfsetdash{}{0pt}%
\pgfpathmoveto{\pgfqpoint{3.165581in}{1.344240in}}%
\pgfpathlineto{\pgfqpoint{3.165581in}{1.389412in}}%
\pgfusepath{stroke}%
\end{pgfscope}%
\begin{pgfscope}%
\pgfpathrectangle{\pgfqpoint{0.572918in}{0.553781in}}{\pgfqpoint{5.478282in}{2.095553in}}%
\pgfusepath{clip}%
\pgfsetbuttcap%
\pgfsetroundjoin%
\pgfsetlinewidth{1.505625pt}%
\definecolor{currentstroke}{rgb}{1.000000,0.819608,0.101961}%
\pgfsetstrokecolor{currentstroke}%
\pgfsetstrokeopacity{0.900000}%
\pgfsetdash{}{0pt}%
\pgfpathmoveto{\pgfqpoint{3.458537in}{1.222776in}}%
\pgfpathlineto{\pgfqpoint{3.458537in}{1.261861in}}%
\pgfusepath{stroke}%
\end{pgfscope}%
\begin{pgfscope}%
\pgfpathrectangle{\pgfqpoint{0.572918in}{0.553781in}}{\pgfqpoint{5.478282in}{2.095553in}}%
\pgfusepath{clip}%
\pgfsetbuttcap%
\pgfsetroundjoin%
\pgfsetlinewidth{1.505625pt}%
\definecolor{currentstroke}{rgb}{1.000000,0.819608,0.101961}%
\pgfsetstrokecolor{currentstroke}%
\pgfsetstrokeopacity{0.900000}%
\pgfsetdash{}{0pt}%
\pgfpathmoveto{\pgfqpoint{3.751494in}{1.144554in}}%
\pgfpathlineto{\pgfqpoint{3.751494in}{1.184613in}}%
\pgfusepath{stroke}%
\end{pgfscope}%
\begin{pgfscope}%
\pgfpathrectangle{\pgfqpoint{0.572918in}{0.553781in}}{\pgfqpoint{5.478282in}{2.095553in}}%
\pgfusepath{clip}%
\pgfsetbuttcap%
\pgfsetroundjoin%
\pgfsetlinewidth{1.505625pt}%
\definecolor{currentstroke}{rgb}{1.000000,0.819608,0.101961}%
\pgfsetstrokecolor{currentstroke}%
\pgfsetstrokeopacity{0.900000}%
\pgfsetdash{}{0pt}%
\pgfpathmoveto{\pgfqpoint{4.044450in}{1.076765in}}%
\pgfpathlineto{\pgfqpoint{4.044450in}{1.115083in}}%
\pgfusepath{stroke}%
\end{pgfscope}%
\begin{pgfscope}%
\pgfpathrectangle{\pgfqpoint{0.572918in}{0.553781in}}{\pgfqpoint{5.478282in}{2.095553in}}%
\pgfusepath{clip}%
\pgfsetbuttcap%
\pgfsetroundjoin%
\pgfsetlinewidth{1.505625pt}%
\definecolor{currentstroke}{rgb}{1.000000,0.819608,0.101961}%
\pgfsetstrokecolor{currentstroke}%
\pgfsetstrokeopacity{0.900000}%
\pgfsetdash{}{0pt}%
\pgfpathmoveto{\pgfqpoint{4.337406in}{0.969496in}}%
\pgfpathlineto{\pgfqpoint{4.337406in}{1.016660in}}%
\pgfusepath{stroke}%
\end{pgfscope}%
\begin{pgfscope}%
\pgfpathrectangle{\pgfqpoint{0.572918in}{0.553781in}}{\pgfqpoint{5.478282in}{2.095553in}}%
\pgfusepath{clip}%
\pgfsetbuttcap%
\pgfsetroundjoin%
\pgfsetlinewidth{1.505625pt}%
\definecolor{currentstroke}{rgb}{1.000000,0.819608,0.101961}%
\pgfsetstrokecolor{currentstroke}%
\pgfsetstrokeopacity{0.900000}%
\pgfsetdash{}{0pt}%
\pgfpathmoveto{\pgfqpoint{4.630362in}{0.893002in}}%
\pgfpathlineto{\pgfqpoint{4.630362in}{0.941623in}}%
\pgfusepath{stroke}%
\end{pgfscope}%
\begin{pgfscope}%
\pgfpathrectangle{\pgfqpoint{0.572918in}{0.553781in}}{\pgfqpoint{5.478282in}{2.095553in}}%
\pgfusepath{clip}%
\pgfsetbuttcap%
\pgfsetroundjoin%
\pgfsetlinewidth{1.505625pt}%
\definecolor{currentstroke}{rgb}{1.000000,0.819608,0.101961}%
\pgfsetstrokecolor{currentstroke}%
\pgfsetstrokeopacity{0.900000}%
\pgfsetdash{}{0pt}%
\pgfpathmoveto{\pgfqpoint{4.923318in}{0.824362in}}%
\pgfpathlineto{\pgfqpoint{4.923318in}{0.878685in}}%
\pgfusepath{stroke}%
\end{pgfscope}%
\begin{pgfscope}%
\pgfpathrectangle{\pgfqpoint{0.572918in}{0.553781in}}{\pgfqpoint{5.478282in}{2.095553in}}%
\pgfusepath{clip}%
\pgfsetbuttcap%
\pgfsetroundjoin%
\pgfsetlinewidth{1.505625pt}%
\definecolor{currentstroke}{rgb}{1.000000,0.819608,0.101961}%
\pgfsetstrokecolor{currentstroke}%
\pgfsetstrokeopacity{0.900000}%
\pgfsetdash{}{0pt}%
\pgfpathmoveto{\pgfqpoint{5.216275in}{0.847095in}}%
\pgfpathlineto{\pgfqpoint{5.216275in}{0.932694in}}%
\pgfusepath{stroke}%
\end{pgfscope}%
\begin{pgfscope}%
\pgfpathrectangle{\pgfqpoint{0.572918in}{0.553781in}}{\pgfqpoint{5.478282in}{2.095553in}}%
\pgfusepath{clip}%
\pgfsetbuttcap%
\pgfsetroundjoin%
\pgfsetlinewidth{1.505625pt}%
\definecolor{currentstroke}{rgb}{1.000000,0.819608,0.101961}%
\pgfsetstrokecolor{currentstroke}%
\pgfsetstrokeopacity{0.900000}%
\pgfsetdash{}{0pt}%
\pgfpathmoveto{\pgfqpoint{5.509231in}{0.779743in}}%
\pgfpathlineto{\pgfqpoint{5.509231in}{0.873128in}}%
\pgfusepath{stroke}%
\end{pgfscope}%
\begin{pgfscope}%
\pgfpathrectangle{\pgfqpoint{0.572918in}{0.553781in}}{\pgfqpoint{5.478282in}{2.095553in}}%
\pgfusepath{clip}%
\pgfsetbuttcap%
\pgfsetroundjoin%
\pgfsetlinewidth{1.505625pt}%
\definecolor{currentstroke}{rgb}{1.000000,0.819608,0.101961}%
\pgfsetstrokecolor{currentstroke}%
\pgfsetstrokeopacity{0.900000}%
\pgfsetdash{}{0pt}%
\pgfpathmoveto{\pgfqpoint{5.802187in}{0.853235in}}%
\pgfpathlineto{\pgfqpoint{5.802187in}{0.977556in}}%
\pgfusepath{stroke}%
\end{pgfscope}%
\begin{pgfscope}%
\pgfpathrectangle{\pgfqpoint{0.572918in}{0.553781in}}{\pgfqpoint{5.478282in}{2.095553in}}%
\pgfusepath{clip}%
\pgfsetbuttcap%
\pgfsetroundjoin%
\definecolor{currentfill}{rgb}{0.313725,0.317647,0.309804}%
\pgfsetfillcolor{currentfill}%
\pgfsetfillopacity{0.900000}%
\pgfsetlinewidth{1.003750pt}%
\definecolor{currentstroke}{rgb}{0.313725,0.317647,0.309804}%
\pgfsetstrokecolor{currentstroke}%
\pgfsetstrokeopacity{0.900000}%
\pgfsetdash{}{0pt}%
\pgfsys@defobject{currentmarker}{\pgfqpoint{-0.013889in}{-0.000000in}}{\pgfqpoint{0.013889in}{0.000000in}}{%
\pgfpathmoveto{\pgfqpoint{0.013889in}{-0.000000in}}%
\pgfpathlineto{\pgfqpoint{-0.013889in}{0.000000in}}%
\pgfusepath{stroke,fill}%
}%
\begin{pgfscope}%
\pgfsys@transformshift{0.821931in}{2.067272in}%
\pgfsys@useobject{currentmarker}{}%
\end{pgfscope}%
\begin{pgfscope}%
\pgfsys@transformshift{1.114887in}{2.078877in}%
\pgfsys@useobject{currentmarker}{}%
\end{pgfscope}%
\begin{pgfscope}%
\pgfsys@transformshift{1.407844in}{2.087960in}%
\pgfsys@useobject{currentmarker}{}%
\end{pgfscope}%
\begin{pgfscope}%
\pgfsys@transformshift{1.700800in}{2.188029in}%
\pgfsys@useobject{currentmarker}{}%
\end{pgfscope}%
\begin{pgfscope}%
\pgfsys@transformshift{1.993756in}{2.214038in}%
\pgfsys@useobject{currentmarker}{}%
\end{pgfscope}%
\begin{pgfscope}%
\pgfsys@transformshift{2.286712in}{2.196951in}%
\pgfsys@useobject{currentmarker}{}%
\end{pgfscope}%
\begin{pgfscope}%
\pgfsys@transformshift{2.579669in}{1.988684in}%
\pgfsys@useobject{currentmarker}{}%
\end{pgfscope}%
\begin{pgfscope}%
\pgfsys@transformshift{2.872625in}{1.831657in}%
\pgfsys@useobject{currentmarker}{}%
\end{pgfscope}%
\begin{pgfscope}%
\pgfsys@transformshift{3.165581in}{1.589762in}%
\pgfsys@useobject{currentmarker}{}%
\end{pgfscope}%
\begin{pgfscope}%
\pgfsys@transformshift{3.458537in}{1.393828in}%
\pgfsys@useobject{currentmarker}{}%
\end{pgfscope}%
\begin{pgfscope}%
\pgfsys@transformshift{3.751494in}{1.161673in}%
\pgfsys@useobject{currentmarker}{}%
\end{pgfscope}%
\begin{pgfscope}%
\pgfsys@transformshift{4.044450in}{1.026004in}%
\pgfsys@useobject{currentmarker}{}%
\end{pgfscope}%
\begin{pgfscope}%
\pgfsys@transformshift{4.337406in}{0.878054in}%
\pgfsys@useobject{currentmarker}{}%
\end{pgfscope}%
\begin{pgfscope}%
\pgfsys@transformshift{4.630362in}{0.785324in}%
\pgfsys@useobject{currentmarker}{}%
\end{pgfscope}%
\begin{pgfscope}%
\pgfsys@transformshift{4.923318in}{0.707399in}%
\pgfsys@useobject{currentmarker}{}%
\end{pgfscope}%
\begin{pgfscope}%
\pgfsys@transformshift{5.216275in}{0.697970in}%
\pgfsys@useobject{currentmarker}{}%
\end{pgfscope}%
\begin{pgfscope}%
\pgfsys@transformshift{5.509231in}{0.649033in}%
\pgfsys@useobject{currentmarker}{}%
\end{pgfscope}%
\begin{pgfscope}%
\pgfsys@transformshift{5.802187in}{0.655468in}%
\pgfsys@useobject{currentmarker}{}%
\end{pgfscope}%
\end{pgfscope}%
\begin{pgfscope}%
\pgfpathrectangle{\pgfqpoint{0.572918in}{0.553781in}}{\pgfqpoint{5.478282in}{2.095553in}}%
\pgfusepath{clip}%
\pgfsetbuttcap%
\pgfsetroundjoin%
\definecolor{currentfill}{rgb}{0.313725,0.317647,0.309804}%
\pgfsetfillcolor{currentfill}%
\pgfsetfillopacity{0.900000}%
\pgfsetlinewidth{1.003750pt}%
\definecolor{currentstroke}{rgb}{0.313725,0.317647,0.309804}%
\pgfsetstrokecolor{currentstroke}%
\pgfsetstrokeopacity{0.900000}%
\pgfsetdash{}{0pt}%
\pgfsys@defobject{currentmarker}{\pgfqpoint{-0.013889in}{-0.000000in}}{\pgfqpoint{0.013889in}{0.000000in}}{%
\pgfpathmoveto{\pgfqpoint{0.013889in}{-0.000000in}}%
\pgfpathlineto{\pgfqpoint{-0.013889in}{0.000000in}}%
\pgfusepath{stroke,fill}%
}%
\begin{pgfscope}%
\pgfsys@transformshift{0.821931in}{2.554081in}%
\pgfsys@useobject{currentmarker}{}%
\end{pgfscope}%
\begin{pgfscope}%
\pgfsys@transformshift{1.114887in}{2.314335in}%
\pgfsys@useobject{currentmarker}{}%
\end{pgfscope}%
\begin{pgfscope}%
\pgfsys@transformshift{1.407844in}{2.242339in}%
\pgfsys@useobject{currentmarker}{}%
\end{pgfscope}%
\begin{pgfscope}%
\pgfsys@transformshift{1.700800in}{2.306855in}%
\pgfsys@useobject{currentmarker}{}%
\end{pgfscope}%
\begin{pgfscope}%
\pgfsys@transformshift{1.993756in}{2.306567in}%
\pgfsys@useobject{currentmarker}{}%
\end{pgfscope}%
\begin{pgfscope}%
\pgfsys@transformshift{2.286712in}{2.278266in}%
\pgfsys@useobject{currentmarker}{}%
\end{pgfscope}%
\begin{pgfscope}%
\pgfsys@transformshift{2.579669in}{2.050211in}%
\pgfsys@useobject{currentmarker}{}%
\end{pgfscope}%
\begin{pgfscope}%
\pgfsys@transformshift{2.872625in}{1.892051in}%
\pgfsys@useobject{currentmarker}{}%
\end{pgfscope}%
\begin{pgfscope}%
\pgfsys@transformshift{3.165581in}{1.637771in}%
\pgfsys@useobject{currentmarker}{}%
\end{pgfscope}%
\begin{pgfscope}%
\pgfsys@transformshift{3.458537in}{1.439392in}%
\pgfsys@useobject{currentmarker}{}%
\end{pgfscope}%
\begin{pgfscope}%
\pgfsys@transformshift{3.751494in}{1.203739in}%
\pgfsys@useobject{currentmarker}{}%
\end{pgfscope}%
\begin{pgfscope}%
\pgfsys@transformshift{4.044450in}{1.072266in}%
\pgfsys@useobject{currentmarker}{}%
\end{pgfscope}%
\begin{pgfscope}%
\pgfsys@transformshift{4.337406in}{0.922528in}%
\pgfsys@useobject{currentmarker}{}%
\end{pgfscope}%
\begin{pgfscope}%
\pgfsys@transformshift{4.630362in}{0.831559in}%
\pgfsys@useobject{currentmarker}{}%
\end{pgfscope}%
\begin{pgfscope}%
\pgfsys@transformshift{4.923318in}{0.755435in}%
\pgfsys@useobject{currentmarker}{}%
\end{pgfscope}%
\begin{pgfscope}%
\pgfsys@transformshift{5.216275in}{0.761403in}%
\pgfsys@useobject{currentmarker}{}%
\end{pgfscope}%
\begin{pgfscope}%
\pgfsys@transformshift{5.509231in}{0.713958in}%
\pgfsys@useobject{currentmarker}{}%
\end{pgfscope}%
\begin{pgfscope}%
\pgfsys@transformshift{5.802187in}{0.750982in}%
\pgfsys@useobject{currentmarker}{}%
\end{pgfscope}%
\end{pgfscope}%
\begin{pgfscope}%
\pgfpathrectangle{\pgfqpoint{0.572918in}{0.553781in}}{\pgfqpoint{5.478282in}{2.095553in}}%
\pgfusepath{clip}%
\pgfsetbuttcap%
\pgfsetroundjoin%
\definecolor{currentfill}{rgb}{0.949020,0.372549,0.360784}%
\pgfsetfillcolor{currentfill}%
\pgfsetfillopacity{0.900000}%
\pgfsetlinewidth{1.003750pt}%
\definecolor{currentstroke}{rgb}{0.949020,0.372549,0.360784}%
\pgfsetstrokecolor{currentstroke}%
\pgfsetstrokeopacity{0.900000}%
\pgfsetdash{}{0pt}%
\pgfsys@defobject{currentmarker}{\pgfqpoint{-0.013889in}{-0.000000in}}{\pgfqpoint{0.013889in}{0.000000in}}{%
\pgfpathmoveto{\pgfqpoint{0.013889in}{-0.000000in}}%
\pgfpathlineto{\pgfqpoint{-0.013889in}{0.000000in}}%
\pgfusepath{stroke,fill}%
}%
\begin{pgfscope}%
\pgfsys@transformshift{0.821931in}{1.610412in}%
\pgfsys@useobject{currentmarker}{}%
\end{pgfscope}%
\begin{pgfscope}%
\pgfsys@transformshift{1.114887in}{1.590032in}%
\pgfsys@useobject{currentmarker}{}%
\end{pgfscope}%
\begin{pgfscope}%
\pgfsys@transformshift{1.407844in}{1.639922in}%
\pgfsys@useobject{currentmarker}{}%
\end{pgfscope}%
\begin{pgfscope}%
\pgfsys@transformshift{1.700800in}{1.710074in}%
\pgfsys@useobject{currentmarker}{}%
\end{pgfscope}%
\begin{pgfscope}%
\pgfsys@transformshift{1.993756in}{1.789830in}%
\pgfsys@useobject{currentmarker}{}%
\end{pgfscope}%
\begin{pgfscope}%
\pgfsys@transformshift{2.286712in}{1.791563in}%
\pgfsys@useobject{currentmarker}{}%
\end{pgfscope}%
\begin{pgfscope}%
\pgfsys@transformshift{2.579669in}{1.700949in}%
\pgfsys@useobject{currentmarker}{}%
\end{pgfscope}%
\begin{pgfscope}%
\pgfsys@transformshift{2.872625in}{1.660783in}%
\pgfsys@useobject{currentmarker}{}%
\end{pgfscope}%
\begin{pgfscope}%
\pgfsys@transformshift{3.165581in}{1.543644in}%
\pgfsys@useobject{currentmarker}{}%
\end{pgfscope}%
\begin{pgfscope}%
\pgfsys@transformshift{3.458537in}{1.456182in}%
\pgfsys@useobject{currentmarker}{}%
\end{pgfscope}%
\begin{pgfscope}%
\pgfsys@transformshift{3.751494in}{1.340109in}%
\pgfsys@useobject{currentmarker}{}%
\end{pgfscope}%
\begin{pgfscope}%
\pgfsys@transformshift{4.044450in}{1.298949in}%
\pgfsys@useobject{currentmarker}{}%
\end{pgfscope}%
\begin{pgfscope}%
\pgfsys@transformshift{4.337406in}{1.177427in}%
\pgfsys@useobject{currentmarker}{}%
\end{pgfscope}%
\begin{pgfscope}%
\pgfsys@transformshift{4.630362in}{1.149333in}%
\pgfsys@useobject{currentmarker}{}%
\end{pgfscope}%
\begin{pgfscope}%
\pgfsys@transformshift{4.923318in}{1.062965in}%
\pgfsys@useobject{currentmarker}{}%
\end{pgfscope}%
\begin{pgfscope}%
\pgfsys@transformshift{5.216275in}{1.102604in}%
\pgfsys@useobject{currentmarker}{}%
\end{pgfscope}%
\begin{pgfscope}%
\pgfsys@transformshift{5.509231in}{1.071548in}%
\pgfsys@useobject{currentmarker}{}%
\end{pgfscope}%
\begin{pgfscope}%
\pgfsys@transformshift{5.802187in}{1.088207in}%
\pgfsys@useobject{currentmarker}{}%
\end{pgfscope}%
\end{pgfscope}%
\begin{pgfscope}%
\pgfpathrectangle{\pgfqpoint{0.572918in}{0.553781in}}{\pgfqpoint{5.478282in}{2.095553in}}%
\pgfusepath{clip}%
\pgfsetbuttcap%
\pgfsetroundjoin%
\definecolor{currentfill}{rgb}{0.949020,0.372549,0.360784}%
\pgfsetfillcolor{currentfill}%
\pgfsetfillopacity{0.900000}%
\pgfsetlinewidth{1.003750pt}%
\definecolor{currentstroke}{rgb}{0.949020,0.372549,0.360784}%
\pgfsetstrokecolor{currentstroke}%
\pgfsetstrokeopacity{0.900000}%
\pgfsetdash{}{0pt}%
\pgfsys@defobject{currentmarker}{\pgfqpoint{-0.013889in}{-0.000000in}}{\pgfqpoint{0.013889in}{0.000000in}}{%
\pgfpathmoveto{\pgfqpoint{0.013889in}{-0.000000in}}%
\pgfpathlineto{\pgfqpoint{-0.013889in}{0.000000in}}%
\pgfusepath{stroke,fill}%
}%
\begin{pgfscope}%
\pgfsys@transformshift{0.821931in}{2.081453in}%
\pgfsys@useobject{currentmarker}{}%
\end{pgfscope}%
\begin{pgfscope}%
\pgfsys@transformshift{1.114887in}{1.805012in}%
\pgfsys@useobject{currentmarker}{}%
\end{pgfscope}%
\begin{pgfscope}%
\pgfsys@transformshift{1.407844in}{1.746536in}%
\pgfsys@useobject{currentmarker}{}%
\end{pgfscope}%
\begin{pgfscope}%
\pgfsys@transformshift{1.700800in}{1.793898in}%
\pgfsys@useobject{currentmarker}{}%
\end{pgfscope}%
\begin{pgfscope}%
\pgfsys@transformshift{1.993756in}{1.855864in}%
\pgfsys@useobject{currentmarker}{}%
\end{pgfscope}%
\begin{pgfscope}%
\pgfsys@transformshift{2.286712in}{1.844556in}%
\pgfsys@useobject{currentmarker}{}%
\end{pgfscope}%
\begin{pgfscope}%
\pgfsys@transformshift{2.579669in}{1.749923in}%
\pgfsys@useobject{currentmarker}{}%
\end{pgfscope}%
\begin{pgfscope}%
\pgfsys@transformshift{2.872625in}{1.712713in}%
\pgfsys@useobject{currentmarker}{}%
\end{pgfscope}%
\begin{pgfscope}%
\pgfsys@transformshift{3.165581in}{1.588539in}%
\pgfsys@useobject{currentmarker}{}%
\end{pgfscope}%
\begin{pgfscope}%
\pgfsys@transformshift{3.458537in}{1.500934in}%
\pgfsys@useobject{currentmarker}{}%
\end{pgfscope}%
\begin{pgfscope}%
\pgfsys@transformshift{3.751494in}{1.382285in}%
\pgfsys@useobject{currentmarker}{}%
\end{pgfscope}%
\begin{pgfscope}%
\pgfsys@transformshift{4.044450in}{1.348791in}%
\pgfsys@useobject{currentmarker}{}%
\end{pgfscope}%
\begin{pgfscope}%
\pgfsys@transformshift{4.337406in}{1.238783in}%
\pgfsys@useobject{currentmarker}{}%
\end{pgfscope}%
\begin{pgfscope}%
\pgfsys@transformshift{4.630362in}{1.216842in}%
\pgfsys@useobject{currentmarker}{}%
\end{pgfscope}%
\begin{pgfscope}%
\pgfsys@transformshift{4.923318in}{1.135601in}%
\pgfsys@useobject{currentmarker}{}%
\end{pgfscope}%
\begin{pgfscope}%
\pgfsys@transformshift{5.216275in}{1.202850in}%
\pgfsys@useobject{currentmarker}{}%
\end{pgfscope}%
\begin{pgfscope}%
\pgfsys@transformshift{5.509231in}{1.198276in}%
\pgfsys@useobject{currentmarker}{}%
\end{pgfscope}%
\begin{pgfscope}%
\pgfsys@transformshift{5.802187in}{1.262489in}%
\pgfsys@useobject{currentmarker}{}%
\end{pgfscope}%
\end{pgfscope}%
\begin{pgfscope}%
\pgfpathrectangle{\pgfqpoint{0.572918in}{0.553781in}}{\pgfqpoint{5.478282in}{2.095553in}}%
\pgfusepath{clip}%
\pgfsetbuttcap%
\pgfsetroundjoin%
\definecolor{currentfill}{rgb}{1.000000,0.819608,0.101961}%
\pgfsetfillcolor{currentfill}%
\pgfsetfillopacity{0.900000}%
\pgfsetlinewidth{1.003750pt}%
\definecolor{currentstroke}{rgb}{1.000000,0.819608,0.101961}%
\pgfsetstrokecolor{currentstroke}%
\pgfsetstrokeopacity{0.900000}%
\pgfsetdash{}{0pt}%
\pgfsys@defobject{currentmarker}{\pgfqpoint{-0.013889in}{-0.000000in}}{\pgfqpoint{0.013889in}{0.000000in}}{%
\pgfpathmoveto{\pgfqpoint{0.013889in}{-0.000000in}}%
\pgfpathlineto{\pgfqpoint{-0.013889in}{0.000000in}}%
\pgfusepath{stroke,fill}%
}%
\begin{pgfscope}%
\pgfsys@transformshift{0.821931in}{1.500982in}%
\pgfsys@useobject{currentmarker}{}%
\end{pgfscope}%
\begin{pgfscope}%
\pgfsys@transformshift{1.114887in}{1.534942in}%
\pgfsys@useobject{currentmarker}{}%
\end{pgfscope}%
\begin{pgfscope}%
\pgfsys@transformshift{1.407844in}{1.567679in}%
\pgfsys@useobject{currentmarker}{}%
\end{pgfscope}%
\begin{pgfscope}%
\pgfsys@transformshift{1.700800in}{1.596455in}%
\pgfsys@useobject{currentmarker}{}%
\end{pgfscope}%
\begin{pgfscope}%
\pgfsys@transformshift{1.993756in}{1.599015in}%
\pgfsys@useobject{currentmarker}{}%
\end{pgfscope}%
\begin{pgfscope}%
\pgfsys@transformshift{2.286712in}{1.607213in}%
\pgfsys@useobject{currentmarker}{}%
\end{pgfscope}%
\begin{pgfscope}%
\pgfsys@transformshift{2.579669in}{1.503533in}%
\pgfsys@useobject{currentmarker}{}%
\end{pgfscope}%
\begin{pgfscope}%
\pgfsys@transformshift{2.872625in}{1.465870in}%
\pgfsys@useobject{currentmarker}{}%
\end{pgfscope}%
\begin{pgfscope}%
\pgfsys@transformshift{3.165581in}{1.344240in}%
\pgfsys@useobject{currentmarker}{}%
\end{pgfscope}%
\begin{pgfscope}%
\pgfsys@transformshift{3.458537in}{1.222776in}%
\pgfsys@useobject{currentmarker}{}%
\end{pgfscope}%
\begin{pgfscope}%
\pgfsys@transformshift{3.751494in}{1.144554in}%
\pgfsys@useobject{currentmarker}{}%
\end{pgfscope}%
\begin{pgfscope}%
\pgfsys@transformshift{4.044450in}{1.076765in}%
\pgfsys@useobject{currentmarker}{}%
\end{pgfscope}%
\begin{pgfscope}%
\pgfsys@transformshift{4.337406in}{0.969496in}%
\pgfsys@useobject{currentmarker}{}%
\end{pgfscope}%
\begin{pgfscope}%
\pgfsys@transformshift{4.630362in}{0.893002in}%
\pgfsys@useobject{currentmarker}{}%
\end{pgfscope}%
\begin{pgfscope}%
\pgfsys@transformshift{4.923318in}{0.824362in}%
\pgfsys@useobject{currentmarker}{}%
\end{pgfscope}%
\begin{pgfscope}%
\pgfsys@transformshift{5.216275in}{0.847095in}%
\pgfsys@useobject{currentmarker}{}%
\end{pgfscope}%
\begin{pgfscope}%
\pgfsys@transformshift{5.509231in}{0.779743in}%
\pgfsys@useobject{currentmarker}{}%
\end{pgfscope}%
\begin{pgfscope}%
\pgfsys@transformshift{5.802187in}{0.853235in}%
\pgfsys@useobject{currentmarker}{}%
\end{pgfscope}%
\end{pgfscope}%
\begin{pgfscope}%
\pgfpathrectangle{\pgfqpoint{0.572918in}{0.553781in}}{\pgfqpoint{5.478282in}{2.095553in}}%
\pgfusepath{clip}%
\pgfsetbuttcap%
\pgfsetroundjoin%
\definecolor{currentfill}{rgb}{1.000000,0.819608,0.101961}%
\pgfsetfillcolor{currentfill}%
\pgfsetfillopacity{0.900000}%
\pgfsetlinewidth{1.003750pt}%
\definecolor{currentstroke}{rgb}{1.000000,0.819608,0.101961}%
\pgfsetstrokecolor{currentstroke}%
\pgfsetstrokeopacity{0.900000}%
\pgfsetdash{}{0pt}%
\pgfsys@defobject{currentmarker}{\pgfqpoint{-0.013889in}{-0.000000in}}{\pgfqpoint{0.013889in}{0.000000in}}{%
\pgfpathmoveto{\pgfqpoint{0.013889in}{-0.000000in}}%
\pgfpathlineto{\pgfqpoint{-0.013889in}{0.000000in}}%
\pgfusepath{stroke,fill}%
}%
\begin{pgfscope}%
\pgfsys@transformshift{0.821931in}{1.891902in}%
\pgfsys@useobject{currentmarker}{}%
\end{pgfscope}%
\begin{pgfscope}%
\pgfsys@transformshift{1.114887in}{1.701759in}%
\pgfsys@useobject{currentmarker}{}%
\end{pgfscope}%
\begin{pgfscope}%
\pgfsys@transformshift{1.407844in}{1.668585in}%
\pgfsys@useobject{currentmarker}{}%
\end{pgfscope}%
\begin{pgfscope}%
\pgfsys@transformshift{1.700800in}{1.671557in}%
\pgfsys@useobject{currentmarker}{}%
\end{pgfscope}%
\begin{pgfscope}%
\pgfsys@transformshift{1.993756in}{1.658482in}%
\pgfsys@useobject{currentmarker}{}%
\end{pgfscope}%
\begin{pgfscope}%
\pgfsys@transformshift{2.286712in}{1.658702in}%
\pgfsys@useobject{currentmarker}{}%
\end{pgfscope}%
\begin{pgfscope}%
\pgfsys@transformshift{2.579669in}{1.546489in}%
\pgfsys@useobject{currentmarker}{}%
\end{pgfscope}%
\begin{pgfscope}%
\pgfsys@transformshift{2.872625in}{1.509008in}%
\pgfsys@useobject{currentmarker}{}%
\end{pgfscope}%
\begin{pgfscope}%
\pgfsys@transformshift{3.165581in}{1.389412in}%
\pgfsys@useobject{currentmarker}{}%
\end{pgfscope}%
\begin{pgfscope}%
\pgfsys@transformshift{3.458537in}{1.261861in}%
\pgfsys@useobject{currentmarker}{}%
\end{pgfscope}%
\begin{pgfscope}%
\pgfsys@transformshift{3.751494in}{1.184613in}%
\pgfsys@useobject{currentmarker}{}%
\end{pgfscope}%
\begin{pgfscope}%
\pgfsys@transformshift{4.044450in}{1.115083in}%
\pgfsys@useobject{currentmarker}{}%
\end{pgfscope}%
\begin{pgfscope}%
\pgfsys@transformshift{4.337406in}{1.016660in}%
\pgfsys@useobject{currentmarker}{}%
\end{pgfscope}%
\begin{pgfscope}%
\pgfsys@transformshift{4.630362in}{0.941623in}%
\pgfsys@useobject{currentmarker}{}%
\end{pgfscope}%
\begin{pgfscope}%
\pgfsys@transformshift{4.923318in}{0.878685in}%
\pgfsys@useobject{currentmarker}{}%
\end{pgfscope}%
\begin{pgfscope}%
\pgfsys@transformshift{5.216275in}{0.932694in}%
\pgfsys@useobject{currentmarker}{}%
\end{pgfscope}%
\begin{pgfscope}%
\pgfsys@transformshift{5.509231in}{0.873128in}%
\pgfsys@useobject{currentmarker}{}%
\end{pgfscope}%
\begin{pgfscope}%
\pgfsys@transformshift{5.802187in}{0.977556in}%
\pgfsys@useobject{currentmarker}{}%
\end{pgfscope}%
\end{pgfscope}%
\begin{pgfscope}%
\pgfpathrectangle{\pgfqpoint{0.572918in}{0.553781in}}{\pgfqpoint{5.478282in}{2.095553in}}%
\pgfusepath{clip}%
\pgfsetrectcap%
\pgfsetroundjoin%
\pgfsetlinewidth{1.505625pt}%
\definecolor{currentstroke}{rgb}{0.313725,0.317647,0.309804}%
\pgfsetstrokecolor{currentstroke}%
\pgfsetstrokeopacity{0.900000}%
\pgfsetdash{}{0pt}%
\pgfpathmoveto{\pgfqpoint{0.821931in}{2.271945in}}%
\pgfpathlineto{\pgfqpoint{1.114887in}{2.178306in}}%
\pgfpathlineto{\pgfqpoint{1.407844in}{2.163522in}}%
\pgfpathlineto{\pgfqpoint{1.700800in}{2.240029in}}%
\pgfpathlineto{\pgfqpoint{1.993756in}{2.259884in}}%
\pgfpathlineto{\pgfqpoint{2.286712in}{2.238241in}}%
\pgfpathlineto{\pgfqpoint{2.579669in}{2.019696in}}%
\pgfpathlineto{\pgfqpoint{2.872625in}{1.862518in}}%
\pgfpathlineto{\pgfqpoint{3.165581in}{1.614022in}}%
\pgfpathlineto{\pgfqpoint{3.458537in}{1.417162in}}%
\pgfpathlineto{\pgfqpoint{3.751494in}{1.183467in}}%
\pgfpathlineto{\pgfqpoint{4.044450in}{1.050210in}}%
\pgfpathlineto{\pgfqpoint{4.337406in}{0.899970in}}%
\pgfpathlineto{\pgfqpoint{4.630362in}{0.806693in}}%
\pgfpathlineto{\pgfqpoint{4.923318in}{0.732712in}}%
\pgfpathlineto{\pgfqpoint{5.216275in}{0.725217in}}%
\pgfpathlineto{\pgfqpoint{5.509231in}{0.674264in}}%
\pgfpathlineto{\pgfqpoint{5.802187in}{0.694604in}}%
\pgfusepath{stroke}%
\end{pgfscope}%
\begin{pgfscope}%
\pgfpathrectangle{\pgfqpoint{0.572918in}{0.553781in}}{\pgfqpoint{5.478282in}{2.095553in}}%
\pgfusepath{clip}%
\pgfsetbuttcap%
\pgfsetroundjoin%
\pgfsetlinewidth{1.505625pt}%
\definecolor{currentstroke}{rgb}{0.949020,0.372549,0.360784}%
\pgfsetstrokecolor{currentstroke}%
\pgfsetstrokeopacity{0.900000}%
\pgfsetdash{{1.500000pt}{2.475000pt}}{0.000000pt}%
\pgfpathmoveto{\pgfqpoint{0.821931in}{1.828400in}}%
\pgfpathlineto{\pgfqpoint{1.114887in}{1.699885in}}%
\pgfpathlineto{\pgfqpoint{1.407844in}{1.694346in}}%
\pgfpathlineto{\pgfqpoint{1.700800in}{1.756022in}}%
\pgfpathlineto{\pgfqpoint{1.993756in}{1.824234in}}%
\pgfpathlineto{\pgfqpoint{2.286712in}{1.820898in}}%
\pgfpathlineto{\pgfqpoint{2.579669in}{1.724939in}}%
\pgfpathlineto{\pgfqpoint{2.872625in}{1.685646in}}%
\pgfpathlineto{\pgfqpoint{3.165581in}{1.566964in}}%
\pgfpathlineto{\pgfqpoint{3.458537in}{1.481762in}}%
\pgfpathlineto{\pgfqpoint{3.751494in}{1.361493in}}%
\pgfpathlineto{\pgfqpoint{4.044450in}{1.323560in}}%
\pgfpathlineto{\pgfqpoint{4.337406in}{1.208495in}}%
\pgfpathlineto{\pgfqpoint{4.630362in}{1.183253in}}%
\pgfpathlineto{\pgfqpoint{4.923318in}{1.101012in}}%
\pgfpathlineto{\pgfqpoint{5.216275in}{1.149876in}}%
\pgfpathlineto{\pgfqpoint{5.509231in}{1.132163in}}%
\pgfpathlineto{\pgfqpoint{5.802187in}{1.182261in}}%
\pgfusepath{stroke}%
\end{pgfscope}%
\begin{pgfscope}%
\pgfpathrectangle{\pgfqpoint{0.572918in}{0.553781in}}{\pgfqpoint{5.478282in}{2.095553in}}%
\pgfusepath{clip}%
\pgfsetbuttcap%
\pgfsetroundjoin%
\pgfsetlinewidth{1.505625pt}%
\definecolor{currentstroke}{rgb}{1.000000,0.819608,0.101961}%
\pgfsetstrokecolor{currentstroke}%
\pgfsetstrokeopacity{0.900000}%
\pgfsetdash{{5.550000pt}{2.400000pt}}{0.000000pt}%
\pgfpathmoveto{\pgfqpoint{0.821931in}{1.702409in}}%
\pgfpathlineto{\pgfqpoint{1.114887in}{1.626148in}}%
\pgfpathlineto{\pgfqpoint{1.407844in}{1.616799in}}%
\pgfpathlineto{\pgfqpoint{1.700800in}{1.631932in}}%
\pgfpathlineto{\pgfqpoint{1.993756in}{1.630680in}}%
\pgfpathlineto{\pgfqpoint{2.286712in}{1.631009in}}%
\pgfpathlineto{\pgfqpoint{2.579669in}{1.525074in}}%
\pgfpathlineto{\pgfqpoint{2.872625in}{1.486814in}}%
\pgfpathlineto{\pgfqpoint{3.165581in}{1.365530in}}%
\pgfpathlineto{\pgfqpoint{3.458537in}{1.242462in}}%
\pgfpathlineto{\pgfqpoint{3.751494in}{1.163563in}}%
\pgfpathlineto{\pgfqpoint{4.044450in}{1.097254in}}%
\pgfpathlineto{\pgfqpoint{4.337406in}{0.994190in}}%
\pgfpathlineto{\pgfqpoint{4.630362in}{0.917906in}}%
\pgfpathlineto{\pgfqpoint{4.923318in}{0.850164in}}%
\pgfpathlineto{\pgfqpoint{5.216275in}{0.894645in}}%
\pgfpathlineto{\pgfqpoint{5.509231in}{0.830041in}}%
\pgfpathlineto{\pgfqpoint{5.802187in}{0.921459in}}%
\pgfusepath{stroke}%
\end{pgfscope}%
\begin{pgfscope}%
\pgfsetrectcap%
\pgfsetmiterjoin%
\pgfsetlinewidth{0.803000pt}%
\definecolor{currentstroke}{rgb}{0.000000,0.000000,0.000000}%
\pgfsetstrokecolor{currentstroke}%
\pgfsetdash{}{0pt}%
\pgfpathmoveto{\pgfqpoint{0.572918in}{0.553781in}}%
\pgfpathlineto{\pgfqpoint{0.572918in}{2.649333in}}%
\pgfusepath{stroke}%
\end{pgfscope}%
\begin{pgfscope}%
\pgfsetrectcap%
\pgfsetmiterjoin%
\pgfsetlinewidth{0.803000pt}%
\definecolor{currentstroke}{rgb}{0.000000,0.000000,0.000000}%
\pgfsetstrokecolor{currentstroke}%
\pgfsetdash{}{0pt}%
\pgfpathmoveto{\pgfqpoint{6.051200in}{0.553781in}}%
\pgfpathlineto{\pgfqpoint{6.051200in}{2.649333in}}%
\pgfusepath{stroke}%
\end{pgfscope}%
\begin{pgfscope}%
\pgfsetrectcap%
\pgfsetmiterjoin%
\pgfsetlinewidth{0.803000pt}%
\definecolor{currentstroke}{rgb}{0.000000,0.000000,0.000000}%
\pgfsetstrokecolor{currentstroke}%
\pgfsetdash{}{0pt}%
\pgfpathmoveto{\pgfqpoint{0.572918in}{0.553781in}}%
\pgfpathlineto{\pgfqpoint{6.051200in}{0.553781in}}%
\pgfusepath{stroke}%
\end{pgfscope}%
\begin{pgfscope}%
\pgfsetrectcap%
\pgfsetmiterjoin%
\pgfsetlinewidth{0.803000pt}%
\definecolor{currentstroke}{rgb}{0.000000,0.000000,0.000000}%
\pgfsetstrokecolor{currentstroke}%
\pgfsetdash{}{0pt}%
\pgfpathmoveto{\pgfqpoint{0.572918in}{2.649333in}}%
\pgfpathlineto{\pgfqpoint{6.051200in}{2.649333in}}%
\pgfusepath{stroke}%
\end{pgfscope}%
\begin{pgfscope}%
\definecolor{textcolor}{rgb}{0.000000,0.000000,0.000000}%
\pgfsetstrokecolor{textcolor}%
\pgfsetfillcolor{textcolor}%
\pgftext[x=0.572918in,y=2.732667in,left,base]{\color{textcolor}\rmfamily\fontsize{12.000000}{14.400000}\selectfont Zenith performance}%
\end{pgfscope}%
\begin{pgfscope}%
\pgfsetbuttcap%
\pgfsetmiterjoin%
\definecolor{currentfill}{rgb}{1.000000,1.000000,1.000000}%
\pgfsetfillcolor{currentfill}%
\pgfsetfillopacity{0.800000}%
\pgfsetlinewidth{1.003750pt}%
\definecolor{currentstroke}{rgb}{0.800000,0.800000,0.800000}%
\pgfsetstrokecolor{currentstroke}%
\pgfsetstrokeopacity{0.800000}%
\pgfsetdash{}{0pt}%
\pgfpathmoveto{\pgfqpoint{4.687089in}{2.094667in}}%
\pgfpathlineto{\pgfqpoint{5.973422in}{2.094667in}}%
\pgfpathquadraticcurveto{\pgfqpoint{5.995644in}{2.094667in}}{\pgfqpoint{5.995644in}{2.116889in}}%
\pgfpathlineto{\pgfqpoint{5.995644in}{2.571556in}}%
\pgfpathquadraticcurveto{\pgfqpoint{5.995644in}{2.593778in}}{\pgfqpoint{5.973422in}{2.593778in}}%
\pgfpathlineto{\pgfqpoint{4.687089in}{2.593778in}}%
\pgfpathquadraticcurveto{\pgfqpoint{4.664867in}{2.593778in}}{\pgfqpoint{4.664867in}{2.571556in}}%
\pgfpathlineto{\pgfqpoint{4.664867in}{2.116889in}}%
\pgfpathquadraticcurveto{\pgfqpoint{4.664867in}{2.094667in}}{\pgfqpoint{4.687089in}{2.094667in}}%
\pgfpathclose%
\pgfusepath{stroke,fill}%
\end{pgfscope}%
\begin{pgfscope}%
\pgfsetbuttcap%
\pgfsetroundjoin%
\pgfsetlinewidth{1.505625pt}%
\definecolor{currentstroke}{rgb}{0.313725,0.317647,0.309804}%
\pgfsetstrokecolor{currentstroke}%
\pgfsetstrokeopacity{0.900000}%
\pgfsetdash{}{0pt}%
\pgfpathmoveto{\pgfqpoint{4.820422in}{2.454889in}}%
\pgfpathlineto{\pgfqpoint{4.820422in}{2.566000in}}%
\pgfusepath{stroke}%
\end{pgfscope}%
\begin{pgfscope}%
\pgfsetbuttcap%
\pgfsetroundjoin%
\definecolor{currentfill}{rgb}{0.313725,0.317647,0.309804}%
\pgfsetfillcolor{currentfill}%
\pgfsetfillopacity{0.900000}%
\pgfsetlinewidth{1.003750pt}%
\definecolor{currentstroke}{rgb}{0.313725,0.317647,0.309804}%
\pgfsetstrokecolor{currentstroke}%
\pgfsetstrokeopacity{0.900000}%
\pgfsetdash{}{0pt}%
\pgfsys@defobject{currentmarker}{\pgfqpoint{-0.013889in}{-0.000000in}}{\pgfqpoint{0.013889in}{0.000000in}}{%
\pgfpathmoveto{\pgfqpoint{0.013889in}{-0.000000in}}%
\pgfpathlineto{\pgfqpoint{-0.013889in}{0.000000in}}%
\pgfusepath{stroke,fill}%
}%
\begin{pgfscope}%
\pgfsys@transformshift{4.820422in}{2.454889in}%
\pgfsys@useobject{currentmarker}{}%
\end{pgfscope}%
\end{pgfscope}%
\begin{pgfscope}%
\pgfsetbuttcap%
\pgfsetroundjoin%
\definecolor{currentfill}{rgb}{0.313725,0.317647,0.309804}%
\pgfsetfillcolor{currentfill}%
\pgfsetfillopacity{0.900000}%
\pgfsetlinewidth{1.003750pt}%
\definecolor{currentstroke}{rgb}{0.313725,0.317647,0.309804}%
\pgfsetstrokecolor{currentstroke}%
\pgfsetstrokeopacity{0.900000}%
\pgfsetdash{}{0pt}%
\pgfsys@defobject{currentmarker}{\pgfqpoint{-0.013889in}{-0.000000in}}{\pgfqpoint{0.013889in}{0.000000in}}{%
\pgfpathmoveto{\pgfqpoint{0.013889in}{-0.000000in}}%
\pgfpathlineto{\pgfqpoint{-0.013889in}{0.000000in}}%
\pgfusepath{stroke,fill}%
}%
\begin{pgfscope}%
\pgfsys@transformshift{4.820422in}{2.566000in}%
\pgfsys@useobject{currentmarker}{}%
\end{pgfscope}%
\end{pgfscope}%
\begin{pgfscope}%
\pgfsetrectcap%
\pgfsetroundjoin%
\pgfsetlinewidth{1.505625pt}%
\definecolor{currentstroke}{rgb}{0.313725,0.317647,0.309804}%
\pgfsetstrokecolor{currentstroke}%
\pgfsetstrokeopacity{0.900000}%
\pgfsetdash{}{0pt}%
\pgfpathmoveto{\pgfqpoint{4.709311in}{2.510444in}}%
\pgfpathlineto{\pgfqpoint{4.931533in}{2.510444in}}%
\pgfusepath{stroke}%
\end{pgfscope}%
\begin{pgfscope}%
\definecolor{textcolor}{rgb}{0.000000,0.000000,0.000000}%
\pgfsetstrokecolor{textcolor}%
\pgfsetfillcolor{textcolor}%
\pgftext[x=5.020422in,y=2.471556in,left,base]{\color{textcolor}\rmfamily\fontsize{8.000000}{9.600000}\selectfont Unit vectors}%
\end{pgfscope}%
\begin{pgfscope}%
\pgfsetbuttcap%
\pgfsetroundjoin%
\pgfsetlinewidth{1.505625pt}%
\definecolor{currentstroke}{rgb}{0.949020,0.372549,0.360784}%
\pgfsetstrokecolor{currentstroke}%
\pgfsetstrokeopacity{0.900000}%
\pgfsetdash{}{0pt}%
\pgfpathmoveto{\pgfqpoint{4.820422in}{2.300000in}}%
\pgfpathlineto{\pgfqpoint{4.820422in}{2.411111in}}%
\pgfusepath{stroke}%
\end{pgfscope}%
\begin{pgfscope}%
\pgfsetbuttcap%
\pgfsetroundjoin%
\definecolor{currentfill}{rgb}{0.949020,0.372549,0.360784}%
\pgfsetfillcolor{currentfill}%
\pgfsetfillopacity{0.900000}%
\pgfsetlinewidth{1.003750pt}%
\definecolor{currentstroke}{rgb}{0.949020,0.372549,0.360784}%
\pgfsetstrokecolor{currentstroke}%
\pgfsetstrokeopacity{0.900000}%
\pgfsetdash{}{0pt}%
\pgfsys@defobject{currentmarker}{\pgfqpoint{-0.013889in}{-0.000000in}}{\pgfqpoint{0.013889in}{0.000000in}}{%
\pgfpathmoveto{\pgfqpoint{0.013889in}{-0.000000in}}%
\pgfpathlineto{\pgfqpoint{-0.013889in}{0.000000in}}%
\pgfusepath{stroke,fill}%
}%
\begin{pgfscope}%
\pgfsys@transformshift{4.820422in}{2.300000in}%
\pgfsys@useobject{currentmarker}{}%
\end{pgfscope}%
\end{pgfscope}%
\begin{pgfscope}%
\pgfsetbuttcap%
\pgfsetroundjoin%
\definecolor{currentfill}{rgb}{0.949020,0.372549,0.360784}%
\pgfsetfillcolor{currentfill}%
\pgfsetfillopacity{0.900000}%
\pgfsetlinewidth{1.003750pt}%
\definecolor{currentstroke}{rgb}{0.949020,0.372549,0.360784}%
\pgfsetstrokecolor{currentstroke}%
\pgfsetstrokeopacity{0.900000}%
\pgfsetdash{}{0pt}%
\pgfsys@defobject{currentmarker}{\pgfqpoint{-0.013889in}{-0.000000in}}{\pgfqpoint{0.013889in}{0.000000in}}{%
\pgfpathmoveto{\pgfqpoint{0.013889in}{-0.000000in}}%
\pgfpathlineto{\pgfqpoint{-0.013889in}{0.000000in}}%
\pgfusepath{stroke,fill}%
}%
\begin{pgfscope}%
\pgfsys@transformshift{4.820422in}{2.411111in}%
\pgfsys@useobject{currentmarker}{}%
\end{pgfscope}%
\end{pgfscope}%
\begin{pgfscope}%
\pgfsetbuttcap%
\pgfsetroundjoin%
\pgfsetlinewidth{1.505625pt}%
\definecolor{currentstroke}{rgb}{0.949020,0.372549,0.360784}%
\pgfsetstrokecolor{currentstroke}%
\pgfsetstrokeopacity{0.900000}%
\pgfsetdash{{1.500000pt}{2.475000pt}}{0.000000pt}%
\pgfpathmoveto{\pgfqpoint{4.709311in}{2.355556in}}%
\pgfpathlineto{\pgfqpoint{4.931533in}{2.355556in}}%
\pgfusepath{stroke}%
\end{pgfscope}%
\begin{pgfscope}%
\definecolor{textcolor}{rgb}{0.000000,0.000000,0.000000}%
\pgfsetstrokecolor{textcolor}%
\pgfsetfillcolor{textcolor}%
\pgftext[x=5.020422in,y=2.316667in,left,base]{\color{textcolor}\rmfamily\fontsize{8.000000}{9.600000}\selectfont Azimuth \& zenith}%
\end{pgfscope}%
\begin{pgfscope}%
\pgfsetbuttcap%
\pgfsetroundjoin%
\pgfsetlinewidth{1.505625pt}%
\definecolor{currentstroke}{rgb}{1.000000,0.819608,0.101961}%
\pgfsetstrokecolor{currentstroke}%
\pgfsetstrokeopacity{0.900000}%
\pgfsetdash{}{0pt}%
\pgfpathmoveto{\pgfqpoint{4.820422in}{2.145111in}}%
\pgfpathlineto{\pgfqpoint{4.820422in}{2.256222in}}%
\pgfusepath{stroke}%
\end{pgfscope}%
\begin{pgfscope}%
\pgfsetbuttcap%
\pgfsetroundjoin%
\definecolor{currentfill}{rgb}{1.000000,0.819608,0.101961}%
\pgfsetfillcolor{currentfill}%
\pgfsetfillopacity{0.900000}%
\pgfsetlinewidth{1.003750pt}%
\definecolor{currentstroke}{rgb}{1.000000,0.819608,0.101961}%
\pgfsetstrokecolor{currentstroke}%
\pgfsetstrokeopacity{0.900000}%
\pgfsetdash{}{0pt}%
\pgfsys@defobject{currentmarker}{\pgfqpoint{-0.013889in}{-0.000000in}}{\pgfqpoint{0.013889in}{0.000000in}}{%
\pgfpathmoveto{\pgfqpoint{0.013889in}{-0.000000in}}%
\pgfpathlineto{\pgfqpoint{-0.013889in}{0.000000in}}%
\pgfusepath{stroke,fill}%
}%
\begin{pgfscope}%
\pgfsys@transformshift{4.820422in}{2.145111in}%
\pgfsys@useobject{currentmarker}{}%
\end{pgfscope}%
\end{pgfscope}%
\begin{pgfscope}%
\pgfsetbuttcap%
\pgfsetroundjoin%
\definecolor{currentfill}{rgb}{1.000000,0.819608,0.101961}%
\pgfsetfillcolor{currentfill}%
\pgfsetfillopacity{0.900000}%
\pgfsetlinewidth{1.003750pt}%
\definecolor{currentstroke}{rgb}{1.000000,0.819608,0.101961}%
\pgfsetstrokecolor{currentstroke}%
\pgfsetstrokeopacity{0.900000}%
\pgfsetdash{}{0pt}%
\pgfsys@defobject{currentmarker}{\pgfqpoint{-0.013889in}{-0.000000in}}{\pgfqpoint{0.013889in}{0.000000in}}{%
\pgfpathmoveto{\pgfqpoint{0.013889in}{-0.000000in}}%
\pgfpathlineto{\pgfqpoint{-0.013889in}{0.000000in}}%
\pgfusepath{stroke,fill}%
}%
\begin{pgfscope}%
\pgfsys@transformshift{4.820422in}{2.256222in}%
\pgfsys@useobject{currentmarker}{}%
\end{pgfscope}%
\end{pgfscope}%
\begin{pgfscope}%
\pgfsetbuttcap%
\pgfsetroundjoin%
\pgfsetlinewidth{1.505625pt}%
\definecolor{currentstroke}{rgb}{1.000000,0.819608,0.101961}%
\pgfsetstrokecolor{currentstroke}%
\pgfsetstrokeopacity{0.900000}%
\pgfsetdash{{5.550000pt}{2.400000pt}}{0.000000pt}%
\pgfpathmoveto{\pgfqpoint{4.709311in}{2.200667in}}%
\pgfpathlineto{\pgfqpoint{4.931533in}{2.200667in}}%
\pgfusepath{stroke}%
\end{pgfscope}%
\begin{pgfscope}%
\definecolor{textcolor}{rgb}{0.000000,0.000000,0.000000}%
\pgfsetstrokecolor{textcolor}%
\pgfsetfillcolor{textcolor}%
\pgftext[x=5.020422in,y=2.161778in,left,base]{\color{textcolor}\rmfamily\fontsize{8.000000}{9.600000}\selectfont Zenith only}%
\end{pgfscope}%
\end{pgfpicture}%
\makeatother%
\endgroup%

    \caption{Zenith resolution performance for different prediction types.
    Only predicting the zenith coordinate of the neutrino is seen to perform best over the largest range of energies, while the prediction of unit vectors outperforms it at higher energies (\SI{100}{\giga\electronvolt} and above).}\label{fig:prediction_types}
\end{figure}

Another avenue is to only predict the zenith angle, which does not suffer from circular wraparound, as it is restricted to \( \left[ \SI{0}{\degree}, \SI{180}{\degree} \right] \).
This leads to another issue: is it possible to input the raw detector data---which contains Cartesian input coordinates for the DOM positions---and expect the neural network to be able to map it to a spherical coordinate system?

\begin{figure}
    \centering
    %% Creator: Matplotlib, PGF backend
%%
%% To include the figure in your LaTeX document, write
%%   \input{<filename>.pgf}
%%
%% Make sure the required packages are loaded in your preamble
%%   \usepackage{pgf}
%%
%% and, on pdftex
%%   \usepackage[utf8]{inputenc}\DeclareUnicodeCharacter{2212}{-}
%%
%% or, on luatex and xetex
%%   \usepackage{unicode-math}
%%
%% Figures using additional raster images can only be included by \input if
%% they are in the same directory as the main LaTeX file. For loading figures
%% from other directories you can use the `import` package
%%   \usepackage{import}
%%
%% and then include the figures with
%%   \import{<path to file>}{<filename>.pgf}
%%
%% Matplotlib used the following preamble
%%   \usepackage{siunitx} \usepackage{amsmath} \usepackage{bm}
%%   \usepackage{fontspec}
%%
\begingroup%
\makeatletter%
\begin{pgfpicture}%
\pgfpathrectangle{\pgfpointorigin}{\pgfqpoint{6.201200in}{3.000000in}}%
\pgfusepath{use as bounding box, clip}%
\begin{pgfscope}%
\pgfsetbuttcap%
\pgfsetmiterjoin%
\definecolor{currentfill}{rgb}{1.000000,1.000000,1.000000}%
\pgfsetfillcolor{currentfill}%
\pgfsetlinewidth{0.000000pt}%
\definecolor{currentstroke}{rgb}{1.000000,1.000000,1.000000}%
\pgfsetstrokecolor{currentstroke}%
\pgfsetdash{}{0pt}%
\pgfpathmoveto{\pgfqpoint{0.000000in}{0.000000in}}%
\pgfpathlineto{\pgfqpoint{6.201200in}{0.000000in}}%
\pgfpathlineto{\pgfqpoint{6.201200in}{3.000000in}}%
\pgfpathlineto{\pgfqpoint{0.000000in}{3.000000in}}%
\pgfpathclose%
\pgfusepath{fill}%
\end{pgfscope}%
\begin{pgfscope}%
\pgfsetbuttcap%
\pgfsetmiterjoin%
\definecolor{currentfill}{rgb}{1.000000,1.000000,1.000000}%
\pgfsetfillcolor{currentfill}%
\pgfsetlinewidth{0.000000pt}%
\definecolor{currentstroke}{rgb}{0.000000,0.000000,0.000000}%
\pgfsetstrokecolor{currentstroke}%
\pgfsetstrokeopacity{0.000000}%
\pgfsetdash{}{0pt}%
\pgfpathmoveto{\pgfqpoint{0.572918in}{0.553781in}}%
\pgfpathlineto{\pgfqpoint{6.051200in}{0.553781in}}%
\pgfpathlineto{\pgfqpoint{6.051200in}{2.649333in}}%
\pgfpathlineto{\pgfqpoint{0.572918in}{2.649333in}}%
\pgfpathclose%
\pgfusepath{fill}%
\end{pgfscope}%
\begin{pgfscope}%
\pgfpathrectangle{\pgfqpoint{0.572918in}{0.553781in}}{\pgfqpoint{5.478282in}{2.095553in}}%
\pgfusepath{clip}%
\pgfsetbuttcap%
\pgfsetroundjoin%
\pgfsetlinewidth{0.501875pt}%
\definecolor{currentstroke}{rgb}{0.690196,0.690196,0.690196}%
\pgfsetstrokecolor{currentstroke}%
\pgfsetstrokeopacity{0.500000}%
\pgfsetdash{{0.500000pt}{0.825000pt}}{0.000000pt}%
\pgfpathmoveto{\pgfqpoint{0.675453in}{0.553781in}}%
\pgfpathlineto{\pgfqpoint{0.675453in}{2.649333in}}%
\pgfusepath{stroke}%
\end{pgfscope}%
\begin{pgfscope}%
\pgfsetbuttcap%
\pgfsetroundjoin%
\definecolor{currentfill}{rgb}{0.000000,0.000000,0.000000}%
\pgfsetfillcolor{currentfill}%
\pgfsetlinewidth{0.803000pt}%
\definecolor{currentstroke}{rgb}{0.000000,0.000000,0.000000}%
\pgfsetstrokecolor{currentstroke}%
\pgfsetdash{}{0pt}%
\pgfsys@defobject{currentmarker}{\pgfqpoint{0.000000in}{-0.048611in}}{\pgfqpoint{0.000000in}{0.000000in}}{%
\pgfpathmoveto{\pgfqpoint{0.000000in}{0.000000in}}%
\pgfpathlineto{\pgfqpoint{0.000000in}{-0.048611in}}%
\pgfusepath{stroke,fill}%
}%
\begin{pgfscope}%
\pgfsys@transformshift{0.675453in}{0.553781in}%
\pgfsys@useobject{currentmarker}{}%
\end{pgfscope}%
\end{pgfscope}%
\begin{pgfscope}%
\definecolor{textcolor}{rgb}{0.000000,0.000000,0.000000}%
\pgfsetstrokecolor{textcolor}%
\pgfsetfillcolor{textcolor}%
\pgftext[x=0.675453in,y=0.456558in,,top]{\color{textcolor}\rmfamily\fontsize{8.000000}{9.600000}\selectfont \(\displaystyle {0.0}\)}%
\end{pgfscope}%
\begin{pgfscope}%
\pgfpathrectangle{\pgfqpoint{0.572918in}{0.553781in}}{\pgfqpoint{5.478282in}{2.095553in}}%
\pgfusepath{clip}%
\pgfsetbuttcap%
\pgfsetroundjoin%
\pgfsetlinewidth{0.501875pt}%
\definecolor{currentstroke}{rgb}{0.690196,0.690196,0.690196}%
\pgfsetstrokecolor{currentstroke}%
\pgfsetstrokeopacity{0.500000}%
\pgfsetdash{{0.500000pt}{0.825000pt}}{0.000000pt}%
\pgfpathmoveto{\pgfqpoint{1.554322in}{0.553781in}}%
\pgfpathlineto{\pgfqpoint{1.554322in}{2.649333in}}%
\pgfusepath{stroke}%
\end{pgfscope}%
\begin{pgfscope}%
\pgfsetbuttcap%
\pgfsetroundjoin%
\definecolor{currentfill}{rgb}{0.000000,0.000000,0.000000}%
\pgfsetfillcolor{currentfill}%
\pgfsetlinewidth{0.803000pt}%
\definecolor{currentstroke}{rgb}{0.000000,0.000000,0.000000}%
\pgfsetstrokecolor{currentstroke}%
\pgfsetdash{}{0pt}%
\pgfsys@defobject{currentmarker}{\pgfqpoint{0.000000in}{-0.048611in}}{\pgfqpoint{0.000000in}{0.000000in}}{%
\pgfpathmoveto{\pgfqpoint{0.000000in}{0.000000in}}%
\pgfpathlineto{\pgfqpoint{0.000000in}{-0.048611in}}%
\pgfusepath{stroke,fill}%
}%
\begin{pgfscope}%
\pgfsys@transformshift{1.554322in}{0.553781in}%
\pgfsys@useobject{currentmarker}{}%
\end{pgfscope}%
\end{pgfscope}%
\begin{pgfscope}%
\definecolor{textcolor}{rgb}{0.000000,0.000000,0.000000}%
\pgfsetstrokecolor{textcolor}%
\pgfsetfillcolor{textcolor}%
\pgftext[x=1.554322in,y=0.456558in,,top]{\color{textcolor}\rmfamily\fontsize{8.000000}{9.600000}\selectfont \(\displaystyle {0.5}\)}%
\end{pgfscope}%
\begin{pgfscope}%
\pgfpathrectangle{\pgfqpoint{0.572918in}{0.553781in}}{\pgfqpoint{5.478282in}{2.095553in}}%
\pgfusepath{clip}%
\pgfsetbuttcap%
\pgfsetroundjoin%
\pgfsetlinewidth{0.501875pt}%
\definecolor{currentstroke}{rgb}{0.690196,0.690196,0.690196}%
\pgfsetstrokecolor{currentstroke}%
\pgfsetstrokeopacity{0.500000}%
\pgfsetdash{{0.500000pt}{0.825000pt}}{0.000000pt}%
\pgfpathmoveto{\pgfqpoint{2.433190in}{0.553781in}}%
\pgfpathlineto{\pgfqpoint{2.433190in}{2.649333in}}%
\pgfusepath{stroke}%
\end{pgfscope}%
\begin{pgfscope}%
\pgfsetbuttcap%
\pgfsetroundjoin%
\definecolor{currentfill}{rgb}{0.000000,0.000000,0.000000}%
\pgfsetfillcolor{currentfill}%
\pgfsetlinewidth{0.803000pt}%
\definecolor{currentstroke}{rgb}{0.000000,0.000000,0.000000}%
\pgfsetstrokecolor{currentstroke}%
\pgfsetdash{}{0pt}%
\pgfsys@defobject{currentmarker}{\pgfqpoint{0.000000in}{-0.048611in}}{\pgfqpoint{0.000000in}{0.000000in}}{%
\pgfpathmoveto{\pgfqpoint{0.000000in}{0.000000in}}%
\pgfpathlineto{\pgfqpoint{0.000000in}{-0.048611in}}%
\pgfusepath{stroke,fill}%
}%
\begin{pgfscope}%
\pgfsys@transformshift{2.433190in}{0.553781in}%
\pgfsys@useobject{currentmarker}{}%
\end{pgfscope}%
\end{pgfscope}%
\begin{pgfscope}%
\definecolor{textcolor}{rgb}{0.000000,0.000000,0.000000}%
\pgfsetstrokecolor{textcolor}%
\pgfsetfillcolor{textcolor}%
\pgftext[x=2.433190in,y=0.456558in,,top]{\color{textcolor}\rmfamily\fontsize{8.000000}{9.600000}\selectfont \(\displaystyle {1.0}\)}%
\end{pgfscope}%
\begin{pgfscope}%
\pgfpathrectangle{\pgfqpoint{0.572918in}{0.553781in}}{\pgfqpoint{5.478282in}{2.095553in}}%
\pgfusepath{clip}%
\pgfsetbuttcap%
\pgfsetroundjoin%
\pgfsetlinewidth{0.501875pt}%
\definecolor{currentstroke}{rgb}{0.690196,0.690196,0.690196}%
\pgfsetstrokecolor{currentstroke}%
\pgfsetstrokeopacity{0.500000}%
\pgfsetdash{{0.500000pt}{0.825000pt}}{0.000000pt}%
\pgfpathmoveto{\pgfqpoint{3.312059in}{0.553781in}}%
\pgfpathlineto{\pgfqpoint{3.312059in}{2.649333in}}%
\pgfusepath{stroke}%
\end{pgfscope}%
\begin{pgfscope}%
\pgfsetbuttcap%
\pgfsetroundjoin%
\definecolor{currentfill}{rgb}{0.000000,0.000000,0.000000}%
\pgfsetfillcolor{currentfill}%
\pgfsetlinewidth{0.803000pt}%
\definecolor{currentstroke}{rgb}{0.000000,0.000000,0.000000}%
\pgfsetstrokecolor{currentstroke}%
\pgfsetdash{}{0pt}%
\pgfsys@defobject{currentmarker}{\pgfqpoint{0.000000in}{-0.048611in}}{\pgfqpoint{0.000000in}{0.000000in}}{%
\pgfpathmoveto{\pgfqpoint{0.000000in}{0.000000in}}%
\pgfpathlineto{\pgfqpoint{0.000000in}{-0.048611in}}%
\pgfusepath{stroke,fill}%
}%
\begin{pgfscope}%
\pgfsys@transformshift{3.312059in}{0.553781in}%
\pgfsys@useobject{currentmarker}{}%
\end{pgfscope}%
\end{pgfscope}%
\begin{pgfscope}%
\definecolor{textcolor}{rgb}{0.000000,0.000000,0.000000}%
\pgfsetstrokecolor{textcolor}%
\pgfsetfillcolor{textcolor}%
\pgftext[x=3.312059in,y=0.456558in,,top]{\color{textcolor}\rmfamily\fontsize{8.000000}{9.600000}\selectfont \(\displaystyle {1.5}\)}%
\end{pgfscope}%
\begin{pgfscope}%
\pgfpathrectangle{\pgfqpoint{0.572918in}{0.553781in}}{\pgfqpoint{5.478282in}{2.095553in}}%
\pgfusepath{clip}%
\pgfsetbuttcap%
\pgfsetroundjoin%
\pgfsetlinewidth{0.501875pt}%
\definecolor{currentstroke}{rgb}{0.690196,0.690196,0.690196}%
\pgfsetstrokecolor{currentstroke}%
\pgfsetstrokeopacity{0.500000}%
\pgfsetdash{{0.500000pt}{0.825000pt}}{0.000000pt}%
\pgfpathmoveto{\pgfqpoint{4.190928in}{0.553781in}}%
\pgfpathlineto{\pgfqpoint{4.190928in}{2.649333in}}%
\pgfusepath{stroke}%
\end{pgfscope}%
\begin{pgfscope}%
\pgfsetbuttcap%
\pgfsetroundjoin%
\definecolor{currentfill}{rgb}{0.000000,0.000000,0.000000}%
\pgfsetfillcolor{currentfill}%
\pgfsetlinewidth{0.803000pt}%
\definecolor{currentstroke}{rgb}{0.000000,0.000000,0.000000}%
\pgfsetstrokecolor{currentstroke}%
\pgfsetdash{}{0pt}%
\pgfsys@defobject{currentmarker}{\pgfqpoint{0.000000in}{-0.048611in}}{\pgfqpoint{0.000000in}{0.000000in}}{%
\pgfpathmoveto{\pgfqpoint{0.000000in}{0.000000in}}%
\pgfpathlineto{\pgfqpoint{0.000000in}{-0.048611in}}%
\pgfusepath{stroke,fill}%
}%
\begin{pgfscope}%
\pgfsys@transformshift{4.190928in}{0.553781in}%
\pgfsys@useobject{currentmarker}{}%
\end{pgfscope}%
\end{pgfscope}%
\begin{pgfscope}%
\definecolor{textcolor}{rgb}{0.000000,0.000000,0.000000}%
\pgfsetstrokecolor{textcolor}%
\pgfsetfillcolor{textcolor}%
\pgftext[x=4.190928in,y=0.456558in,,top]{\color{textcolor}\rmfamily\fontsize{8.000000}{9.600000}\selectfont \(\displaystyle {2.0}\)}%
\end{pgfscope}%
\begin{pgfscope}%
\pgfpathrectangle{\pgfqpoint{0.572918in}{0.553781in}}{\pgfqpoint{5.478282in}{2.095553in}}%
\pgfusepath{clip}%
\pgfsetbuttcap%
\pgfsetroundjoin%
\pgfsetlinewidth{0.501875pt}%
\definecolor{currentstroke}{rgb}{0.690196,0.690196,0.690196}%
\pgfsetstrokecolor{currentstroke}%
\pgfsetstrokeopacity{0.500000}%
\pgfsetdash{{0.500000pt}{0.825000pt}}{0.000000pt}%
\pgfpathmoveto{\pgfqpoint{5.069797in}{0.553781in}}%
\pgfpathlineto{\pgfqpoint{5.069797in}{2.649333in}}%
\pgfusepath{stroke}%
\end{pgfscope}%
\begin{pgfscope}%
\pgfsetbuttcap%
\pgfsetroundjoin%
\definecolor{currentfill}{rgb}{0.000000,0.000000,0.000000}%
\pgfsetfillcolor{currentfill}%
\pgfsetlinewidth{0.803000pt}%
\definecolor{currentstroke}{rgb}{0.000000,0.000000,0.000000}%
\pgfsetstrokecolor{currentstroke}%
\pgfsetdash{}{0pt}%
\pgfsys@defobject{currentmarker}{\pgfqpoint{0.000000in}{-0.048611in}}{\pgfqpoint{0.000000in}{0.000000in}}{%
\pgfpathmoveto{\pgfqpoint{0.000000in}{0.000000in}}%
\pgfpathlineto{\pgfqpoint{0.000000in}{-0.048611in}}%
\pgfusepath{stroke,fill}%
}%
\begin{pgfscope}%
\pgfsys@transformshift{5.069797in}{0.553781in}%
\pgfsys@useobject{currentmarker}{}%
\end{pgfscope}%
\end{pgfscope}%
\begin{pgfscope}%
\definecolor{textcolor}{rgb}{0.000000,0.000000,0.000000}%
\pgfsetstrokecolor{textcolor}%
\pgfsetfillcolor{textcolor}%
\pgftext[x=5.069797in,y=0.456558in,,top]{\color{textcolor}\rmfamily\fontsize{8.000000}{9.600000}\selectfont \(\displaystyle {2.5}\)}%
\end{pgfscope}%
\begin{pgfscope}%
\pgfpathrectangle{\pgfqpoint{0.572918in}{0.553781in}}{\pgfqpoint{5.478282in}{2.095553in}}%
\pgfusepath{clip}%
\pgfsetbuttcap%
\pgfsetroundjoin%
\pgfsetlinewidth{0.501875pt}%
\definecolor{currentstroke}{rgb}{0.690196,0.690196,0.690196}%
\pgfsetstrokecolor{currentstroke}%
\pgfsetstrokeopacity{0.500000}%
\pgfsetdash{{0.500000pt}{0.825000pt}}{0.000000pt}%
\pgfpathmoveto{\pgfqpoint{5.948665in}{0.553781in}}%
\pgfpathlineto{\pgfqpoint{5.948665in}{2.649333in}}%
\pgfusepath{stroke}%
\end{pgfscope}%
\begin{pgfscope}%
\pgfsetbuttcap%
\pgfsetroundjoin%
\definecolor{currentfill}{rgb}{0.000000,0.000000,0.000000}%
\pgfsetfillcolor{currentfill}%
\pgfsetlinewidth{0.803000pt}%
\definecolor{currentstroke}{rgb}{0.000000,0.000000,0.000000}%
\pgfsetstrokecolor{currentstroke}%
\pgfsetdash{}{0pt}%
\pgfsys@defobject{currentmarker}{\pgfqpoint{0.000000in}{-0.048611in}}{\pgfqpoint{0.000000in}{0.000000in}}{%
\pgfpathmoveto{\pgfqpoint{0.000000in}{0.000000in}}%
\pgfpathlineto{\pgfqpoint{0.000000in}{-0.048611in}}%
\pgfusepath{stroke,fill}%
}%
\begin{pgfscope}%
\pgfsys@transformshift{5.948665in}{0.553781in}%
\pgfsys@useobject{currentmarker}{}%
\end{pgfscope}%
\end{pgfscope}%
\begin{pgfscope}%
\definecolor{textcolor}{rgb}{0.000000,0.000000,0.000000}%
\pgfsetstrokecolor{textcolor}%
\pgfsetfillcolor{textcolor}%
\pgftext[x=5.948665in,y=0.456558in,,top]{\color{textcolor}\rmfamily\fontsize{8.000000}{9.600000}\selectfont \(\displaystyle {3.0}\)}%
\end{pgfscope}%
\begin{pgfscope}%
\definecolor{textcolor}{rgb}{0.000000,0.000000,0.000000}%
\pgfsetstrokecolor{textcolor}%
\pgfsetfillcolor{textcolor}%
\pgftext[x=3.312059in,y=0.302336in,,top]{\color{textcolor}\rmfamily\fontsize{10.950000}{13.140000}\selectfont \(\displaystyle \log_{10}(E_{\textup{true}}) \, \left[ E / \textup{GeV} \right]\)}%
\end{pgfscope}%
\begin{pgfscope}%
\pgfpathrectangle{\pgfqpoint{0.572918in}{0.553781in}}{\pgfqpoint{5.478282in}{2.095553in}}%
\pgfusepath{clip}%
\pgfsetbuttcap%
\pgfsetroundjoin%
\pgfsetlinewidth{0.501875pt}%
\definecolor{currentstroke}{rgb}{0.690196,0.690196,0.690196}%
\pgfsetstrokecolor{currentstroke}%
\pgfsetstrokeopacity{0.500000}%
\pgfsetdash{{0.500000pt}{0.825000pt}}{0.000000pt}%
\pgfpathmoveto{\pgfqpoint{0.572918in}{0.860703in}}%
\pgfpathlineto{\pgfqpoint{6.051200in}{0.860703in}}%
\pgfusepath{stroke}%
\end{pgfscope}%
\begin{pgfscope}%
\pgfsetbuttcap%
\pgfsetroundjoin%
\definecolor{currentfill}{rgb}{0.000000,0.000000,0.000000}%
\pgfsetfillcolor{currentfill}%
\pgfsetlinewidth{0.803000pt}%
\definecolor{currentstroke}{rgb}{0.000000,0.000000,0.000000}%
\pgfsetstrokecolor{currentstroke}%
\pgfsetdash{}{0pt}%
\pgfsys@defobject{currentmarker}{\pgfqpoint{-0.048611in}{0.000000in}}{\pgfqpoint{-0.000000in}{0.000000in}}{%
\pgfpathmoveto{\pgfqpoint{-0.000000in}{0.000000in}}%
\pgfpathlineto{\pgfqpoint{-0.048611in}{0.000000in}}%
\pgfusepath{stroke,fill}%
}%
\begin{pgfscope}%
\pgfsys@transformshift{0.572918in}{0.860703in}%
\pgfsys@useobject{currentmarker}{}%
\end{pgfscope}%
\end{pgfscope}%
\begin{pgfscope}%
\definecolor{textcolor}{rgb}{0.000000,0.000000,0.000000}%
\pgfsetstrokecolor{textcolor}%
\pgfsetfillcolor{textcolor}%
\pgftext[x=0.357639in, y=0.822147in, left, base]{\color{textcolor}\rmfamily\fontsize{8.000000}{9.600000}\selectfont \(\displaystyle {15}\)}%
\end{pgfscope}%
\begin{pgfscope}%
\pgfpathrectangle{\pgfqpoint{0.572918in}{0.553781in}}{\pgfqpoint{5.478282in}{2.095553in}}%
\pgfusepath{clip}%
\pgfsetbuttcap%
\pgfsetroundjoin%
\pgfsetlinewidth{0.501875pt}%
\definecolor{currentstroke}{rgb}{0.690196,0.690196,0.690196}%
\pgfsetstrokecolor{currentstroke}%
\pgfsetstrokeopacity{0.500000}%
\pgfsetdash{{0.500000pt}{0.825000pt}}{0.000000pt}%
\pgfpathmoveto{\pgfqpoint{0.572918in}{1.319044in}}%
\pgfpathlineto{\pgfqpoint{6.051200in}{1.319044in}}%
\pgfusepath{stroke}%
\end{pgfscope}%
\begin{pgfscope}%
\pgfsetbuttcap%
\pgfsetroundjoin%
\definecolor{currentfill}{rgb}{0.000000,0.000000,0.000000}%
\pgfsetfillcolor{currentfill}%
\pgfsetlinewidth{0.803000pt}%
\definecolor{currentstroke}{rgb}{0.000000,0.000000,0.000000}%
\pgfsetstrokecolor{currentstroke}%
\pgfsetdash{}{0pt}%
\pgfsys@defobject{currentmarker}{\pgfqpoint{-0.048611in}{0.000000in}}{\pgfqpoint{-0.000000in}{0.000000in}}{%
\pgfpathmoveto{\pgfqpoint{-0.000000in}{0.000000in}}%
\pgfpathlineto{\pgfqpoint{-0.048611in}{0.000000in}}%
\pgfusepath{stroke,fill}%
}%
\begin{pgfscope}%
\pgfsys@transformshift{0.572918in}{1.319044in}%
\pgfsys@useobject{currentmarker}{}%
\end{pgfscope}%
\end{pgfscope}%
\begin{pgfscope}%
\definecolor{textcolor}{rgb}{0.000000,0.000000,0.000000}%
\pgfsetstrokecolor{textcolor}%
\pgfsetfillcolor{textcolor}%
\pgftext[x=0.357639in, y=1.280488in, left, base]{\color{textcolor}\rmfamily\fontsize{8.000000}{9.600000}\selectfont \(\displaystyle {20}\)}%
\end{pgfscope}%
\begin{pgfscope}%
\pgfpathrectangle{\pgfqpoint{0.572918in}{0.553781in}}{\pgfqpoint{5.478282in}{2.095553in}}%
\pgfusepath{clip}%
\pgfsetbuttcap%
\pgfsetroundjoin%
\pgfsetlinewidth{0.501875pt}%
\definecolor{currentstroke}{rgb}{0.690196,0.690196,0.690196}%
\pgfsetstrokecolor{currentstroke}%
\pgfsetstrokeopacity{0.500000}%
\pgfsetdash{{0.500000pt}{0.825000pt}}{0.000000pt}%
\pgfpathmoveto{\pgfqpoint{0.572918in}{1.777385in}}%
\pgfpathlineto{\pgfqpoint{6.051200in}{1.777385in}}%
\pgfusepath{stroke}%
\end{pgfscope}%
\begin{pgfscope}%
\pgfsetbuttcap%
\pgfsetroundjoin%
\definecolor{currentfill}{rgb}{0.000000,0.000000,0.000000}%
\pgfsetfillcolor{currentfill}%
\pgfsetlinewidth{0.803000pt}%
\definecolor{currentstroke}{rgb}{0.000000,0.000000,0.000000}%
\pgfsetstrokecolor{currentstroke}%
\pgfsetdash{}{0pt}%
\pgfsys@defobject{currentmarker}{\pgfqpoint{-0.048611in}{0.000000in}}{\pgfqpoint{-0.000000in}{0.000000in}}{%
\pgfpathmoveto{\pgfqpoint{-0.000000in}{0.000000in}}%
\pgfpathlineto{\pgfqpoint{-0.048611in}{0.000000in}}%
\pgfusepath{stroke,fill}%
}%
\begin{pgfscope}%
\pgfsys@transformshift{0.572918in}{1.777385in}%
\pgfsys@useobject{currentmarker}{}%
\end{pgfscope}%
\end{pgfscope}%
\begin{pgfscope}%
\definecolor{textcolor}{rgb}{0.000000,0.000000,0.000000}%
\pgfsetstrokecolor{textcolor}%
\pgfsetfillcolor{textcolor}%
\pgftext[x=0.357639in, y=1.738829in, left, base]{\color{textcolor}\rmfamily\fontsize{8.000000}{9.600000}\selectfont \(\displaystyle {25}\)}%
\end{pgfscope}%
\begin{pgfscope}%
\pgfpathrectangle{\pgfqpoint{0.572918in}{0.553781in}}{\pgfqpoint{5.478282in}{2.095553in}}%
\pgfusepath{clip}%
\pgfsetbuttcap%
\pgfsetroundjoin%
\pgfsetlinewidth{0.501875pt}%
\definecolor{currentstroke}{rgb}{0.690196,0.690196,0.690196}%
\pgfsetstrokecolor{currentstroke}%
\pgfsetstrokeopacity{0.500000}%
\pgfsetdash{{0.500000pt}{0.825000pt}}{0.000000pt}%
\pgfpathmoveto{\pgfqpoint{0.572918in}{2.235726in}}%
\pgfpathlineto{\pgfqpoint{6.051200in}{2.235726in}}%
\pgfusepath{stroke}%
\end{pgfscope}%
\begin{pgfscope}%
\pgfsetbuttcap%
\pgfsetroundjoin%
\definecolor{currentfill}{rgb}{0.000000,0.000000,0.000000}%
\pgfsetfillcolor{currentfill}%
\pgfsetlinewidth{0.803000pt}%
\definecolor{currentstroke}{rgb}{0.000000,0.000000,0.000000}%
\pgfsetstrokecolor{currentstroke}%
\pgfsetdash{}{0pt}%
\pgfsys@defobject{currentmarker}{\pgfqpoint{-0.048611in}{0.000000in}}{\pgfqpoint{-0.000000in}{0.000000in}}{%
\pgfpathmoveto{\pgfqpoint{-0.000000in}{0.000000in}}%
\pgfpathlineto{\pgfqpoint{-0.048611in}{0.000000in}}%
\pgfusepath{stroke,fill}%
}%
\begin{pgfscope}%
\pgfsys@transformshift{0.572918in}{2.235726in}%
\pgfsys@useobject{currentmarker}{}%
\end{pgfscope}%
\end{pgfscope}%
\begin{pgfscope}%
\definecolor{textcolor}{rgb}{0.000000,0.000000,0.000000}%
\pgfsetstrokecolor{textcolor}%
\pgfsetfillcolor{textcolor}%
\pgftext[x=0.357639in, y=2.197171in, left, base]{\color{textcolor}\rmfamily\fontsize{8.000000}{9.600000}\selectfont \(\displaystyle {30}\)}%
\end{pgfscope}%
\begin{pgfscope}%
\definecolor{textcolor}{rgb}{0.000000,0.000000,0.000000}%
\pgfsetstrokecolor{textcolor}%
\pgfsetfillcolor{textcolor}%
\pgftext[x=0.302083in,y=1.601557in,,bottom,rotate=90.000000]{\color{textcolor}\rmfamily\fontsize{10.950000}{13.140000}\selectfont IQR / 1.349 \(\displaystyle \left[ \textup{deg} \right]\)}%
\end{pgfscope}%
\begin{pgfscope}%
\pgfpathrectangle{\pgfqpoint{0.572918in}{0.553781in}}{\pgfqpoint{5.478282in}{2.095553in}}%
\pgfusepath{clip}%
\pgfsetbuttcap%
\pgfsetroundjoin%
\pgfsetlinewidth{1.505625pt}%
\definecolor{currentstroke}{rgb}{0.313725,0.317647,0.309804}%
\pgfsetstrokecolor{currentstroke}%
\pgfsetstrokeopacity{0.900000}%
\pgfsetdash{}{0pt}%
\pgfpathmoveto{\pgfqpoint{0.821931in}{1.792309in}}%
\pgfpathlineto{\pgfqpoint{0.821931in}{2.375080in}}%
\pgfusepath{stroke}%
\end{pgfscope}%
\begin{pgfscope}%
\pgfpathrectangle{\pgfqpoint{0.572918in}{0.553781in}}{\pgfqpoint{5.478282in}{2.095553in}}%
\pgfusepath{clip}%
\pgfsetbuttcap%
\pgfsetroundjoin%
\pgfsetlinewidth{1.505625pt}%
\definecolor{currentstroke}{rgb}{0.313725,0.317647,0.309804}%
\pgfsetstrokecolor{currentstroke}%
\pgfsetstrokeopacity{0.900000}%
\pgfsetdash{}{0pt}%
\pgfpathmoveto{\pgfqpoint{1.114887in}{1.842934in}}%
\pgfpathlineto{\pgfqpoint{1.114887in}{2.091621in}}%
\pgfusepath{stroke}%
\end{pgfscope}%
\begin{pgfscope}%
\pgfpathrectangle{\pgfqpoint{0.572918in}{0.553781in}}{\pgfqpoint{5.478282in}{2.095553in}}%
\pgfusepath{clip}%
\pgfsetbuttcap%
\pgfsetroundjoin%
\pgfsetlinewidth{1.505625pt}%
\definecolor{currentstroke}{rgb}{0.313725,0.317647,0.309804}%
\pgfsetstrokecolor{currentstroke}%
\pgfsetstrokeopacity{0.900000}%
\pgfsetdash{}{0pt}%
\pgfpathmoveto{\pgfqpoint{1.407844in}{1.891738in}}%
\pgfpathlineto{\pgfqpoint{1.407844in}{2.042165in}}%
\pgfusepath{stroke}%
\end{pgfscope}%
\begin{pgfscope}%
\pgfpathrectangle{\pgfqpoint{0.572918in}{0.553781in}}{\pgfqpoint{5.478282in}{2.095553in}}%
\pgfusepath{clip}%
\pgfsetbuttcap%
\pgfsetroundjoin%
\pgfsetlinewidth{1.505625pt}%
\definecolor{currentstroke}{rgb}{0.313725,0.317647,0.309804}%
\pgfsetstrokecolor{currentstroke}%
\pgfsetstrokeopacity{0.900000}%
\pgfsetdash{}{0pt}%
\pgfpathmoveto{\pgfqpoint{1.700800in}{1.934637in}}%
\pgfpathlineto{\pgfqpoint{1.700800in}{2.046596in}}%
\pgfusepath{stroke}%
\end{pgfscope}%
\begin{pgfscope}%
\pgfpathrectangle{\pgfqpoint{0.572918in}{0.553781in}}{\pgfqpoint{5.478282in}{2.095553in}}%
\pgfusepath{clip}%
\pgfsetbuttcap%
\pgfsetroundjoin%
\pgfsetlinewidth{1.505625pt}%
\definecolor{currentstroke}{rgb}{0.313725,0.317647,0.309804}%
\pgfsetstrokecolor{currentstroke}%
\pgfsetstrokeopacity{0.900000}%
\pgfsetdash{}{0pt}%
\pgfpathmoveto{\pgfqpoint{1.993756in}{1.938452in}}%
\pgfpathlineto{\pgfqpoint{1.993756in}{2.027104in}}%
\pgfusepath{stroke}%
\end{pgfscope}%
\begin{pgfscope}%
\pgfpathrectangle{\pgfqpoint{0.572918in}{0.553781in}}{\pgfqpoint{5.478282in}{2.095553in}}%
\pgfusepath{clip}%
\pgfsetbuttcap%
\pgfsetroundjoin%
\pgfsetlinewidth{1.505625pt}%
\definecolor{currentstroke}{rgb}{0.313725,0.317647,0.309804}%
\pgfsetstrokecolor{currentstroke}%
\pgfsetstrokeopacity{0.900000}%
\pgfsetdash{}{0pt}%
\pgfpathmoveto{\pgfqpoint{2.286712in}{1.950674in}}%
\pgfpathlineto{\pgfqpoint{2.286712in}{2.027433in}}%
\pgfusepath{stroke}%
\end{pgfscope}%
\begin{pgfscope}%
\pgfpathrectangle{\pgfqpoint{0.572918in}{0.553781in}}{\pgfqpoint{5.478282in}{2.095553in}}%
\pgfusepath{clip}%
\pgfsetbuttcap%
\pgfsetroundjoin%
\pgfsetlinewidth{1.505625pt}%
\definecolor{currentstroke}{rgb}{0.313725,0.317647,0.309804}%
\pgfsetstrokecolor{currentstroke}%
\pgfsetstrokeopacity{0.900000}%
\pgfsetdash{}{0pt}%
\pgfpathmoveto{\pgfqpoint{2.579669in}{1.796111in}}%
\pgfpathlineto{\pgfqpoint{2.579669in}{1.860149in}}%
\pgfusepath{stroke}%
\end{pgfscope}%
\begin{pgfscope}%
\pgfpathrectangle{\pgfqpoint{0.572918in}{0.553781in}}{\pgfqpoint{5.478282in}{2.095553in}}%
\pgfusepath{clip}%
\pgfsetbuttcap%
\pgfsetroundjoin%
\pgfsetlinewidth{1.505625pt}%
\definecolor{currentstroke}{rgb}{0.313725,0.317647,0.309804}%
\pgfsetstrokecolor{currentstroke}%
\pgfsetstrokeopacity{0.900000}%
\pgfsetdash{}{0pt}%
\pgfpathmoveto{\pgfqpoint{2.872625in}{1.739964in}}%
\pgfpathlineto{\pgfqpoint{2.872625in}{1.804273in}}%
\pgfusepath{stroke}%
\end{pgfscope}%
\begin{pgfscope}%
\pgfpathrectangle{\pgfqpoint{0.572918in}{0.553781in}}{\pgfqpoint{5.478282in}{2.095553in}}%
\pgfusepath{clip}%
\pgfsetbuttcap%
\pgfsetroundjoin%
\pgfsetlinewidth{1.505625pt}%
\definecolor{currentstroke}{rgb}{0.313725,0.317647,0.309804}%
\pgfsetstrokecolor{currentstroke}%
\pgfsetstrokeopacity{0.900000}%
\pgfsetdash{}{0pt}%
\pgfpathmoveto{\pgfqpoint{3.165581in}{1.558641in}}%
\pgfpathlineto{\pgfqpoint{3.165581in}{1.625982in}}%
\pgfusepath{stroke}%
\end{pgfscope}%
\begin{pgfscope}%
\pgfpathrectangle{\pgfqpoint{0.572918in}{0.553781in}}{\pgfqpoint{5.478282in}{2.095553in}}%
\pgfusepath{clip}%
\pgfsetbuttcap%
\pgfsetroundjoin%
\pgfsetlinewidth{1.505625pt}%
\definecolor{currentstroke}{rgb}{0.313725,0.317647,0.309804}%
\pgfsetstrokecolor{currentstroke}%
\pgfsetstrokeopacity{0.900000}%
\pgfsetdash{}{0pt}%
\pgfpathmoveto{\pgfqpoint{3.458537in}{1.377567in}}%
\pgfpathlineto{\pgfqpoint{3.458537in}{1.435834in}}%
\pgfusepath{stroke}%
\end{pgfscope}%
\begin{pgfscope}%
\pgfpathrectangle{\pgfqpoint{0.572918in}{0.553781in}}{\pgfqpoint{5.478282in}{2.095553in}}%
\pgfusepath{clip}%
\pgfsetbuttcap%
\pgfsetroundjoin%
\pgfsetlinewidth{1.505625pt}%
\definecolor{currentstroke}{rgb}{0.313725,0.317647,0.309804}%
\pgfsetstrokecolor{currentstroke}%
\pgfsetstrokeopacity{0.900000}%
\pgfsetdash{}{0pt}%
\pgfpathmoveto{\pgfqpoint{3.751494in}{1.260956in}}%
\pgfpathlineto{\pgfqpoint{3.751494in}{1.320675in}}%
\pgfusepath{stroke}%
\end{pgfscope}%
\begin{pgfscope}%
\pgfpathrectangle{\pgfqpoint{0.572918in}{0.553781in}}{\pgfqpoint{5.478282in}{2.095553in}}%
\pgfusepath{clip}%
\pgfsetbuttcap%
\pgfsetroundjoin%
\pgfsetlinewidth{1.505625pt}%
\definecolor{currentstroke}{rgb}{0.313725,0.317647,0.309804}%
\pgfsetstrokecolor{currentstroke}%
\pgfsetstrokeopacity{0.900000}%
\pgfsetdash{}{0pt}%
\pgfpathmoveto{\pgfqpoint{4.044450in}{1.159898in}}%
\pgfpathlineto{\pgfqpoint{4.044450in}{1.217021in}}%
\pgfusepath{stroke}%
\end{pgfscope}%
\begin{pgfscope}%
\pgfpathrectangle{\pgfqpoint{0.572918in}{0.553781in}}{\pgfqpoint{5.478282in}{2.095553in}}%
\pgfusepath{clip}%
\pgfsetbuttcap%
\pgfsetroundjoin%
\pgfsetlinewidth{1.505625pt}%
\definecolor{currentstroke}{rgb}{0.313725,0.317647,0.309804}%
\pgfsetstrokecolor{currentstroke}%
\pgfsetstrokeopacity{0.900000}%
\pgfsetdash{}{0pt}%
\pgfpathmoveto{\pgfqpoint{4.337406in}{0.999984in}}%
\pgfpathlineto{\pgfqpoint{4.337406in}{1.070296in}}%
\pgfusepath{stroke}%
\end{pgfscope}%
\begin{pgfscope}%
\pgfpathrectangle{\pgfqpoint{0.572918in}{0.553781in}}{\pgfqpoint{5.478282in}{2.095553in}}%
\pgfusepath{clip}%
\pgfsetbuttcap%
\pgfsetroundjoin%
\pgfsetlinewidth{1.505625pt}%
\definecolor{currentstroke}{rgb}{0.313725,0.317647,0.309804}%
\pgfsetstrokecolor{currentstroke}%
\pgfsetstrokeopacity{0.900000}%
\pgfsetdash{}{0pt}%
\pgfpathmoveto{\pgfqpoint{4.630362in}{0.885950in}}%
\pgfpathlineto{\pgfqpoint{4.630362in}{0.958433in}}%
\pgfusepath{stroke}%
\end{pgfscope}%
\begin{pgfscope}%
\pgfpathrectangle{\pgfqpoint{0.572918in}{0.553781in}}{\pgfqpoint{5.478282in}{2.095553in}}%
\pgfusepath{clip}%
\pgfsetbuttcap%
\pgfsetroundjoin%
\pgfsetlinewidth{1.505625pt}%
\definecolor{currentstroke}{rgb}{0.313725,0.317647,0.309804}%
\pgfsetstrokecolor{currentstroke}%
\pgfsetstrokeopacity{0.900000}%
\pgfsetdash{}{0pt}%
\pgfpathmoveto{\pgfqpoint{4.923318in}{0.783624in}}%
\pgfpathlineto{\pgfqpoint{4.923318in}{0.864607in}}%
\pgfusepath{stroke}%
\end{pgfscope}%
\begin{pgfscope}%
\pgfpathrectangle{\pgfqpoint{0.572918in}{0.553781in}}{\pgfqpoint{5.478282in}{2.095553in}}%
\pgfusepath{clip}%
\pgfsetbuttcap%
\pgfsetroundjoin%
\pgfsetlinewidth{1.505625pt}%
\definecolor{currentstroke}{rgb}{0.313725,0.317647,0.309804}%
\pgfsetstrokecolor{currentstroke}%
\pgfsetstrokeopacity{0.900000}%
\pgfsetdash{}{0pt}%
\pgfpathmoveto{\pgfqpoint{5.216275in}{0.817513in}}%
\pgfpathlineto{\pgfqpoint{5.216275in}{0.945122in}}%
\pgfusepath{stroke}%
\end{pgfscope}%
\begin{pgfscope}%
\pgfpathrectangle{\pgfqpoint{0.572918in}{0.553781in}}{\pgfqpoint{5.478282in}{2.095553in}}%
\pgfusepath{clip}%
\pgfsetbuttcap%
\pgfsetroundjoin%
\pgfsetlinewidth{1.505625pt}%
\definecolor{currentstroke}{rgb}{0.313725,0.317647,0.309804}%
\pgfsetstrokecolor{currentstroke}%
\pgfsetstrokeopacity{0.900000}%
\pgfsetdash{}{0pt}%
\pgfpathmoveto{\pgfqpoint{5.509231in}{0.717108in}}%
\pgfpathlineto{\pgfqpoint{5.509231in}{0.856322in}}%
\pgfusepath{stroke}%
\end{pgfscope}%
\begin{pgfscope}%
\pgfpathrectangle{\pgfqpoint{0.572918in}{0.553781in}}{\pgfqpoint{5.478282in}{2.095553in}}%
\pgfusepath{clip}%
\pgfsetbuttcap%
\pgfsetroundjoin%
\pgfsetlinewidth{1.505625pt}%
\definecolor{currentstroke}{rgb}{0.313725,0.317647,0.309804}%
\pgfsetstrokecolor{currentstroke}%
\pgfsetstrokeopacity{0.900000}%
\pgfsetdash{}{0pt}%
\pgfpathmoveto{\pgfqpoint{5.802187in}{0.826667in}}%
\pgfpathlineto{\pgfqpoint{5.802187in}{1.012000in}}%
\pgfusepath{stroke}%
\end{pgfscope}%
\begin{pgfscope}%
\pgfpathrectangle{\pgfqpoint{0.572918in}{0.553781in}}{\pgfqpoint{5.478282in}{2.095553in}}%
\pgfusepath{clip}%
\pgfsetbuttcap%
\pgfsetroundjoin%
\pgfsetlinewidth{1.505625pt}%
\definecolor{currentstroke}{rgb}{0.949020,0.372549,0.360784}%
\pgfsetstrokecolor{currentstroke}%
\pgfsetstrokeopacity{0.900000}%
\pgfsetdash{}{0pt}%
\pgfpathmoveto{\pgfqpoint{0.821931in}{1.832104in}}%
\pgfpathlineto{\pgfqpoint{0.821931in}{2.554081in}}%
\pgfusepath{stroke}%
\end{pgfscope}%
\begin{pgfscope}%
\pgfpathrectangle{\pgfqpoint{0.572918in}{0.553781in}}{\pgfqpoint{5.478282in}{2.095553in}}%
\pgfusepath{clip}%
\pgfsetbuttcap%
\pgfsetroundjoin%
\pgfsetlinewidth{1.505625pt}%
\definecolor{currentstroke}{rgb}{0.949020,0.372549,0.360784}%
\pgfsetstrokecolor{currentstroke}%
\pgfsetstrokeopacity{0.900000}%
\pgfsetdash{}{0pt}%
\pgfpathmoveto{\pgfqpoint{1.114887in}{1.948338in}}%
\pgfpathlineto{\pgfqpoint{1.114887in}{2.228136in}}%
\pgfusepath{stroke}%
\end{pgfscope}%
\begin{pgfscope}%
\pgfpathrectangle{\pgfqpoint{0.572918in}{0.553781in}}{\pgfqpoint{5.478282in}{2.095553in}}%
\pgfusepath{clip}%
\pgfsetbuttcap%
\pgfsetroundjoin%
\pgfsetlinewidth{1.505625pt}%
\definecolor{currentstroke}{rgb}{0.949020,0.372549,0.360784}%
\pgfsetstrokecolor{currentstroke}%
\pgfsetstrokeopacity{0.900000}%
\pgfsetdash{}{0pt}%
\pgfpathmoveto{\pgfqpoint{1.407844in}{1.840374in}}%
\pgfpathlineto{\pgfqpoint{1.407844in}{1.999711in}}%
\pgfusepath{stroke}%
\end{pgfscope}%
\begin{pgfscope}%
\pgfpathrectangle{\pgfqpoint{0.572918in}{0.553781in}}{\pgfqpoint{5.478282in}{2.095553in}}%
\pgfusepath{clip}%
\pgfsetbuttcap%
\pgfsetroundjoin%
\pgfsetlinewidth{1.505625pt}%
\definecolor{currentstroke}{rgb}{0.949020,0.372549,0.360784}%
\pgfsetstrokecolor{currentstroke}%
\pgfsetstrokeopacity{0.900000}%
\pgfsetdash{}{0pt}%
\pgfpathmoveto{\pgfqpoint{1.700800in}{1.867913in}}%
\pgfpathlineto{\pgfqpoint{1.700800in}{1.979916in}}%
\pgfusepath{stroke}%
\end{pgfscope}%
\begin{pgfscope}%
\pgfpathrectangle{\pgfqpoint{0.572918in}{0.553781in}}{\pgfqpoint{5.478282in}{2.095553in}}%
\pgfusepath{clip}%
\pgfsetbuttcap%
\pgfsetroundjoin%
\pgfsetlinewidth{1.505625pt}%
\definecolor{currentstroke}{rgb}{0.949020,0.372549,0.360784}%
\pgfsetstrokecolor{currentstroke}%
\pgfsetstrokeopacity{0.900000}%
\pgfsetdash{}{0pt}%
\pgfpathmoveto{\pgfqpoint{1.993756in}{1.963349in}}%
\pgfpathlineto{\pgfqpoint{1.993756in}{2.054461in}}%
\pgfusepath{stroke}%
\end{pgfscope}%
\begin{pgfscope}%
\pgfpathrectangle{\pgfqpoint{0.572918in}{0.553781in}}{\pgfqpoint{5.478282in}{2.095553in}}%
\pgfusepath{clip}%
\pgfsetbuttcap%
\pgfsetroundjoin%
\pgfsetlinewidth{1.505625pt}%
\definecolor{currentstroke}{rgb}{0.949020,0.372549,0.360784}%
\pgfsetstrokecolor{currentstroke}%
\pgfsetstrokeopacity{0.900000}%
\pgfsetdash{}{0pt}%
\pgfpathmoveto{\pgfqpoint{2.286712in}{1.934833in}}%
\pgfpathlineto{\pgfqpoint{2.286712in}{2.001483in}}%
\pgfusepath{stroke}%
\end{pgfscope}%
\begin{pgfscope}%
\pgfpathrectangle{\pgfqpoint{0.572918in}{0.553781in}}{\pgfqpoint{5.478282in}{2.095553in}}%
\pgfusepath{clip}%
\pgfsetbuttcap%
\pgfsetroundjoin%
\pgfsetlinewidth{1.505625pt}%
\definecolor{currentstroke}{rgb}{0.949020,0.372549,0.360784}%
\pgfsetstrokecolor{currentstroke}%
\pgfsetstrokeopacity{0.900000}%
\pgfsetdash{}{0pt}%
\pgfpathmoveto{\pgfqpoint{2.579669in}{1.818698in}}%
\pgfpathlineto{\pgfqpoint{2.579669in}{1.884550in}}%
\pgfusepath{stroke}%
\end{pgfscope}%
\begin{pgfscope}%
\pgfpathrectangle{\pgfqpoint{0.572918in}{0.553781in}}{\pgfqpoint{5.478282in}{2.095553in}}%
\pgfusepath{clip}%
\pgfsetbuttcap%
\pgfsetroundjoin%
\pgfsetlinewidth{1.505625pt}%
\definecolor{currentstroke}{rgb}{0.949020,0.372549,0.360784}%
\pgfsetstrokecolor{currentstroke}%
\pgfsetstrokeopacity{0.900000}%
\pgfsetdash{}{0pt}%
\pgfpathmoveto{\pgfqpoint{2.872625in}{1.733897in}}%
\pgfpathlineto{\pgfqpoint{2.872625in}{1.794023in}}%
\pgfusepath{stroke}%
\end{pgfscope}%
\begin{pgfscope}%
\pgfpathrectangle{\pgfqpoint{0.572918in}{0.553781in}}{\pgfqpoint{5.478282in}{2.095553in}}%
\pgfusepath{clip}%
\pgfsetbuttcap%
\pgfsetroundjoin%
\pgfsetlinewidth{1.505625pt}%
\definecolor{currentstroke}{rgb}{0.949020,0.372549,0.360784}%
\pgfsetstrokecolor{currentstroke}%
\pgfsetstrokeopacity{0.900000}%
\pgfsetdash{}{0pt}%
\pgfpathmoveto{\pgfqpoint{3.165581in}{1.547847in}}%
\pgfpathlineto{\pgfqpoint{3.165581in}{1.610118in}}%
\pgfusepath{stroke}%
\end{pgfscope}%
\begin{pgfscope}%
\pgfpathrectangle{\pgfqpoint{0.572918in}{0.553781in}}{\pgfqpoint{5.478282in}{2.095553in}}%
\pgfusepath{clip}%
\pgfsetbuttcap%
\pgfsetroundjoin%
\pgfsetlinewidth{1.505625pt}%
\definecolor{currentstroke}{rgb}{0.949020,0.372549,0.360784}%
\pgfsetstrokecolor{currentstroke}%
\pgfsetstrokeopacity{0.900000}%
\pgfsetdash{}{0pt}%
\pgfpathmoveto{\pgfqpoint{3.458537in}{1.395274in}}%
\pgfpathlineto{\pgfqpoint{3.458537in}{1.453479in}}%
\pgfusepath{stroke}%
\end{pgfscope}%
\begin{pgfscope}%
\pgfpathrectangle{\pgfqpoint{0.572918in}{0.553781in}}{\pgfqpoint{5.478282in}{2.095553in}}%
\pgfusepath{clip}%
\pgfsetbuttcap%
\pgfsetroundjoin%
\pgfsetlinewidth{1.505625pt}%
\definecolor{currentstroke}{rgb}{0.949020,0.372549,0.360784}%
\pgfsetstrokecolor{currentstroke}%
\pgfsetstrokeopacity{0.900000}%
\pgfsetdash{}{0pt}%
\pgfpathmoveto{\pgfqpoint{3.751494in}{1.265668in}}%
\pgfpathlineto{\pgfqpoint{3.751494in}{1.320713in}}%
\pgfusepath{stroke}%
\end{pgfscope}%
\begin{pgfscope}%
\pgfpathrectangle{\pgfqpoint{0.572918in}{0.553781in}}{\pgfqpoint{5.478282in}{2.095553in}}%
\pgfusepath{clip}%
\pgfsetbuttcap%
\pgfsetroundjoin%
\pgfsetlinewidth{1.505625pt}%
\definecolor{currentstroke}{rgb}{0.949020,0.372549,0.360784}%
\pgfsetstrokecolor{currentstroke}%
\pgfsetstrokeopacity{0.900000}%
\pgfsetdash{}{0pt}%
\pgfpathmoveto{\pgfqpoint{4.044450in}{1.105244in}}%
\pgfpathlineto{\pgfqpoint{4.044450in}{1.171330in}}%
\pgfusepath{stroke}%
\end{pgfscope}%
\begin{pgfscope}%
\pgfpathrectangle{\pgfqpoint{0.572918in}{0.553781in}}{\pgfqpoint{5.478282in}{2.095553in}}%
\pgfusepath{clip}%
\pgfsetbuttcap%
\pgfsetroundjoin%
\pgfsetlinewidth{1.505625pt}%
\definecolor{currentstroke}{rgb}{0.949020,0.372549,0.360784}%
\pgfsetstrokecolor{currentstroke}%
\pgfsetstrokeopacity{0.900000}%
\pgfsetdash{}{0pt}%
\pgfpathmoveto{\pgfqpoint{4.337406in}{0.956867in}}%
\pgfpathlineto{\pgfqpoint{4.337406in}{1.024811in}}%
\pgfusepath{stroke}%
\end{pgfscope}%
\begin{pgfscope}%
\pgfpathrectangle{\pgfqpoint{0.572918in}{0.553781in}}{\pgfqpoint{5.478282in}{2.095553in}}%
\pgfusepath{clip}%
\pgfsetbuttcap%
\pgfsetroundjoin%
\pgfsetlinewidth{1.505625pt}%
\definecolor{currentstroke}{rgb}{0.949020,0.372549,0.360784}%
\pgfsetstrokecolor{currentstroke}%
\pgfsetstrokeopacity{0.900000}%
\pgfsetdash{}{0pt}%
\pgfpathmoveto{\pgfqpoint{4.630362in}{0.791286in}}%
\pgfpathlineto{\pgfqpoint{4.630362in}{0.869426in}}%
\pgfusepath{stroke}%
\end{pgfscope}%
\begin{pgfscope}%
\pgfpathrectangle{\pgfqpoint{0.572918in}{0.553781in}}{\pgfqpoint{5.478282in}{2.095553in}}%
\pgfusepath{clip}%
\pgfsetbuttcap%
\pgfsetroundjoin%
\pgfsetlinewidth{1.505625pt}%
\definecolor{currentstroke}{rgb}{0.949020,0.372549,0.360784}%
\pgfsetstrokecolor{currentstroke}%
\pgfsetstrokeopacity{0.900000}%
\pgfsetdash{}{0pt}%
\pgfpathmoveto{\pgfqpoint{4.923318in}{0.720291in}}%
\pgfpathlineto{\pgfqpoint{4.923318in}{0.809511in}}%
\pgfusepath{stroke}%
\end{pgfscope}%
\begin{pgfscope}%
\pgfpathrectangle{\pgfqpoint{0.572918in}{0.553781in}}{\pgfqpoint{5.478282in}{2.095553in}}%
\pgfusepath{clip}%
\pgfsetbuttcap%
\pgfsetroundjoin%
\pgfsetlinewidth{1.505625pt}%
\definecolor{currentstroke}{rgb}{0.949020,0.372549,0.360784}%
\pgfsetstrokecolor{currentstroke}%
\pgfsetstrokeopacity{0.900000}%
\pgfsetdash{}{0pt}%
\pgfpathmoveto{\pgfqpoint{5.216275in}{0.678524in}}%
\pgfpathlineto{\pgfqpoint{5.216275in}{0.823589in}}%
\pgfusepath{stroke}%
\end{pgfscope}%
\begin{pgfscope}%
\pgfpathrectangle{\pgfqpoint{0.572918in}{0.553781in}}{\pgfqpoint{5.478282in}{2.095553in}}%
\pgfusepath{clip}%
\pgfsetbuttcap%
\pgfsetroundjoin%
\pgfsetlinewidth{1.505625pt}%
\definecolor{currentstroke}{rgb}{0.949020,0.372549,0.360784}%
\pgfsetstrokecolor{currentstroke}%
\pgfsetstrokeopacity{0.900000}%
\pgfsetdash{}{0pt}%
\pgfpathmoveto{\pgfqpoint{5.509231in}{0.649033in}}%
\pgfpathlineto{\pgfqpoint{5.509231in}{0.802225in}}%
\pgfusepath{stroke}%
\end{pgfscope}%
\begin{pgfscope}%
\pgfpathrectangle{\pgfqpoint{0.572918in}{0.553781in}}{\pgfqpoint{5.478282in}{2.095553in}}%
\pgfusepath{clip}%
\pgfsetbuttcap%
\pgfsetroundjoin%
\pgfsetlinewidth{1.505625pt}%
\definecolor{currentstroke}{rgb}{0.949020,0.372549,0.360784}%
\pgfsetstrokecolor{currentstroke}%
\pgfsetstrokeopacity{0.900000}%
\pgfsetdash{}{0pt}%
\pgfpathmoveto{\pgfqpoint{5.802187in}{0.772797in}}%
\pgfpathlineto{\pgfqpoint{5.802187in}{0.976602in}}%
\pgfusepath{stroke}%
\end{pgfscope}%
\begin{pgfscope}%
\pgfpathrectangle{\pgfqpoint{0.572918in}{0.553781in}}{\pgfqpoint{5.478282in}{2.095553in}}%
\pgfusepath{clip}%
\pgfsetbuttcap%
\pgfsetroundjoin%
\definecolor{currentfill}{rgb}{0.313725,0.317647,0.309804}%
\pgfsetfillcolor{currentfill}%
\pgfsetfillopacity{0.900000}%
\pgfsetlinewidth{1.003750pt}%
\definecolor{currentstroke}{rgb}{0.313725,0.317647,0.309804}%
\pgfsetstrokecolor{currentstroke}%
\pgfsetstrokeopacity{0.900000}%
\pgfsetdash{}{0pt}%
\pgfsys@defobject{currentmarker}{\pgfqpoint{-0.013889in}{-0.000000in}}{\pgfqpoint{0.013889in}{0.000000in}}{%
\pgfpathmoveto{\pgfqpoint{0.013889in}{-0.000000in}}%
\pgfpathlineto{\pgfqpoint{-0.013889in}{0.000000in}}%
\pgfusepath{stroke,fill}%
}%
\begin{pgfscope}%
\pgfsys@transformshift{0.821931in}{1.792309in}%
\pgfsys@useobject{currentmarker}{}%
\end{pgfscope}%
\begin{pgfscope}%
\pgfsys@transformshift{1.114887in}{1.842934in}%
\pgfsys@useobject{currentmarker}{}%
\end{pgfscope}%
\begin{pgfscope}%
\pgfsys@transformshift{1.407844in}{1.891738in}%
\pgfsys@useobject{currentmarker}{}%
\end{pgfscope}%
\begin{pgfscope}%
\pgfsys@transformshift{1.700800in}{1.934637in}%
\pgfsys@useobject{currentmarker}{}%
\end{pgfscope}%
\begin{pgfscope}%
\pgfsys@transformshift{1.993756in}{1.938452in}%
\pgfsys@useobject{currentmarker}{}%
\end{pgfscope}%
\begin{pgfscope}%
\pgfsys@transformshift{2.286712in}{1.950674in}%
\pgfsys@useobject{currentmarker}{}%
\end{pgfscope}%
\begin{pgfscope}%
\pgfsys@transformshift{2.579669in}{1.796111in}%
\pgfsys@useobject{currentmarker}{}%
\end{pgfscope}%
\begin{pgfscope}%
\pgfsys@transformshift{2.872625in}{1.739964in}%
\pgfsys@useobject{currentmarker}{}%
\end{pgfscope}%
\begin{pgfscope}%
\pgfsys@transformshift{3.165581in}{1.558641in}%
\pgfsys@useobject{currentmarker}{}%
\end{pgfscope}%
\begin{pgfscope}%
\pgfsys@transformshift{3.458537in}{1.377567in}%
\pgfsys@useobject{currentmarker}{}%
\end{pgfscope}%
\begin{pgfscope}%
\pgfsys@transformshift{3.751494in}{1.260956in}%
\pgfsys@useobject{currentmarker}{}%
\end{pgfscope}%
\begin{pgfscope}%
\pgfsys@transformshift{4.044450in}{1.159898in}%
\pgfsys@useobject{currentmarker}{}%
\end{pgfscope}%
\begin{pgfscope}%
\pgfsys@transformshift{4.337406in}{0.999984in}%
\pgfsys@useobject{currentmarker}{}%
\end{pgfscope}%
\begin{pgfscope}%
\pgfsys@transformshift{4.630362in}{0.885950in}%
\pgfsys@useobject{currentmarker}{}%
\end{pgfscope}%
\begin{pgfscope}%
\pgfsys@transformshift{4.923318in}{0.783624in}%
\pgfsys@useobject{currentmarker}{}%
\end{pgfscope}%
\begin{pgfscope}%
\pgfsys@transformshift{5.216275in}{0.817513in}%
\pgfsys@useobject{currentmarker}{}%
\end{pgfscope}%
\begin{pgfscope}%
\pgfsys@transformshift{5.509231in}{0.717108in}%
\pgfsys@useobject{currentmarker}{}%
\end{pgfscope}%
\begin{pgfscope}%
\pgfsys@transformshift{5.802187in}{0.826667in}%
\pgfsys@useobject{currentmarker}{}%
\end{pgfscope}%
\end{pgfscope}%
\begin{pgfscope}%
\pgfpathrectangle{\pgfqpoint{0.572918in}{0.553781in}}{\pgfqpoint{5.478282in}{2.095553in}}%
\pgfusepath{clip}%
\pgfsetbuttcap%
\pgfsetroundjoin%
\definecolor{currentfill}{rgb}{0.313725,0.317647,0.309804}%
\pgfsetfillcolor{currentfill}%
\pgfsetfillopacity{0.900000}%
\pgfsetlinewidth{1.003750pt}%
\definecolor{currentstroke}{rgb}{0.313725,0.317647,0.309804}%
\pgfsetstrokecolor{currentstroke}%
\pgfsetstrokeopacity{0.900000}%
\pgfsetdash{}{0pt}%
\pgfsys@defobject{currentmarker}{\pgfqpoint{-0.013889in}{-0.000000in}}{\pgfqpoint{0.013889in}{0.000000in}}{%
\pgfpathmoveto{\pgfqpoint{0.013889in}{-0.000000in}}%
\pgfpathlineto{\pgfqpoint{-0.013889in}{0.000000in}}%
\pgfusepath{stroke,fill}%
}%
\begin{pgfscope}%
\pgfsys@transformshift{0.821931in}{2.375080in}%
\pgfsys@useobject{currentmarker}{}%
\end{pgfscope}%
\begin{pgfscope}%
\pgfsys@transformshift{1.114887in}{2.091621in}%
\pgfsys@useobject{currentmarker}{}%
\end{pgfscope}%
\begin{pgfscope}%
\pgfsys@transformshift{1.407844in}{2.042165in}%
\pgfsys@useobject{currentmarker}{}%
\end{pgfscope}%
\begin{pgfscope}%
\pgfsys@transformshift{1.700800in}{2.046596in}%
\pgfsys@useobject{currentmarker}{}%
\end{pgfscope}%
\begin{pgfscope}%
\pgfsys@transformshift{1.993756in}{2.027104in}%
\pgfsys@useobject{currentmarker}{}%
\end{pgfscope}%
\begin{pgfscope}%
\pgfsys@transformshift{2.286712in}{2.027433in}%
\pgfsys@useobject{currentmarker}{}%
\end{pgfscope}%
\begin{pgfscope}%
\pgfsys@transformshift{2.579669in}{1.860149in}%
\pgfsys@useobject{currentmarker}{}%
\end{pgfscope}%
\begin{pgfscope}%
\pgfsys@transformshift{2.872625in}{1.804273in}%
\pgfsys@useobject{currentmarker}{}%
\end{pgfscope}%
\begin{pgfscope}%
\pgfsys@transformshift{3.165581in}{1.625982in}%
\pgfsys@useobject{currentmarker}{}%
\end{pgfscope}%
\begin{pgfscope}%
\pgfsys@transformshift{3.458537in}{1.435834in}%
\pgfsys@useobject{currentmarker}{}%
\end{pgfscope}%
\begin{pgfscope}%
\pgfsys@transformshift{3.751494in}{1.320675in}%
\pgfsys@useobject{currentmarker}{}%
\end{pgfscope}%
\begin{pgfscope}%
\pgfsys@transformshift{4.044450in}{1.217021in}%
\pgfsys@useobject{currentmarker}{}%
\end{pgfscope}%
\begin{pgfscope}%
\pgfsys@transformshift{4.337406in}{1.070296in}%
\pgfsys@useobject{currentmarker}{}%
\end{pgfscope}%
\begin{pgfscope}%
\pgfsys@transformshift{4.630362in}{0.958433in}%
\pgfsys@useobject{currentmarker}{}%
\end{pgfscope}%
\begin{pgfscope}%
\pgfsys@transformshift{4.923318in}{0.864607in}%
\pgfsys@useobject{currentmarker}{}%
\end{pgfscope}%
\begin{pgfscope}%
\pgfsys@transformshift{5.216275in}{0.945122in}%
\pgfsys@useobject{currentmarker}{}%
\end{pgfscope}%
\begin{pgfscope}%
\pgfsys@transformshift{5.509231in}{0.856322in}%
\pgfsys@useobject{currentmarker}{}%
\end{pgfscope}%
\begin{pgfscope}%
\pgfsys@transformshift{5.802187in}{1.012000in}%
\pgfsys@useobject{currentmarker}{}%
\end{pgfscope}%
\end{pgfscope}%
\begin{pgfscope}%
\pgfpathrectangle{\pgfqpoint{0.572918in}{0.553781in}}{\pgfqpoint{5.478282in}{2.095553in}}%
\pgfusepath{clip}%
\pgfsetbuttcap%
\pgfsetroundjoin%
\definecolor{currentfill}{rgb}{0.949020,0.372549,0.360784}%
\pgfsetfillcolor{currentfill}%
\pgfsetfillopacity{0.900000}%
\pgfsetlinewidth{1.003750pt}%
\definecolor{currentstroke}{rgb}{0.949020,0.372549,0.360784}%
\pgfsetstrokecolor{currentstroke}%
\pgfsetstrokeopacity{0.900000}%
\pgfsetdash{}{0pt}%
\pgfsys@defobject{currentmarker}{\pgfqpoint{-0.013889in}{-0.000000in}}{\pgfqpoint{0.013889in}{0.000000in}}{%
\pgfpathmoveto{\pgfqpoint{0.013889in}{-0.000000in}}%
\pgfpathlineto{\pgfqpoint{-0.013889in}{0.000000in}}%
\pgfusepath{stroke,fill}%
}%
\begin{pgfscope}%
\pgfsys@transformshift{0.821931in}{1.832104in}%
\pgfsys@useobject{currentmarker}{}%
\end{pgfscope}%
\begin{pgfscope}%
\pgfsys@transformshift{1.114887in}{1.948338in}%
\pgfsys@useobject{currentmarker}{}%
\end{pgfscope}%
\begin{pgfscope}%
\pgfsys@transformshift{1.407844in}{1.840374in}%
\pgfsys@useobject{currentmarker}{}%
\end{pgfscope}%
\begin{pgfscope}%
\pgfsys@transformshift{1.700800in}{1.867913in}%
\pgfsys@useobject{currentmarker}{}%
\end{pgfscope}%
\begin{pgfscope}%
\pgfsys@transformshift{1.993756in}{1.963349in}%
\pgfsys@useobject{currentmarker}{}%
\end{pgfscope}%
\begin{pgfscope}%
\pgfsys@transformshift{2.286712in}{1.934833in}%
\pgfsys@useobject{currentmarker}{}%
\end{pgfscope}%
\begin{pgfscope}%
\pgfsys@transformshift{2.579669in}{1.818698in}%
\pgfsys@useobject{currentmarker}{}%
\end{pgfscope}%
\begin{pgfscope}%
\pgfsys@transformshift{2.872625in}{1.733897in}%
\pgfsys@useobject{currentmarker}{}%
\end{pgfscope}%
\begin{pgfscope}%
\pgfsys@transformshift{3.165581in}{1.547847in}%
\pgfsys@useobject{currentmarker}{}%
\end{pgfscope}%
\begin{pgfscope}%
\pgfsys@transformshift{3.458537in}{1.395274in}%
\pgfsys@useobject{currentmarker}{}%
\end{pgfscope}%
\begin{pgfscope}%
\pgfsys@transformshift{3.751494in}{1.265668in}%
\pgfsys@useobject{currentmarker}{}%
\end{pgfscope}%
\begin{pgfscope}%
\pgfsys@transformshift{4.044450in}{1.105244in}%
\pgfsys@useobject{currentmarker}{}%
\end{pgfscope}%
\begin{pgfscope}%
\pgfsys@transformshift{4.337406in}{0.956867in}%
\pgfsys@useobject{currentmarker}{}%
\end{pgfscope}%
\begin{pgfscope}%
\pgfsys@transformshift{4.630362in}{0.791286in}%
\pgfsys@useobject{currentmarker}{}%
\end{pgfscope}%
\begin{pgfscope}%
\pgfsys@transformshift{4.923318in}{0.720291in}%
\pgfsys@useobject{currentmarker}{}%
\end{pgfscope}%
\begin{pgfscope}%
\pgfsys@transformshift{5.216275in}{0.678524in}%
\pgfsys@useobject{currentmarker}{}%
\end{pgfscope}%
\begin{pgfscope}%
\pgfsys@transformshift{5.509231in}{0.649033in}%
\pgfsys@useobject{currentmarker}{}%
\end{pgfscope}%
\begin{pgfscope}%
\pgfsys@transformshift{5.802187in}{0.772797in}%
\pgfsys@useobject{currentmarker}{}%
\end{pgfscope}%
\end{pgfscope}%
\begin{pgfscope}%
\pgfpathrectangle{\pgfqpoint{0.572918in}{0.553781in}}{\pgfqpoint{5.478282in}{2.095553in}}%
\pgfusepath{clip}%
\pgfsetbuttcap%
\pgfsetroundjoin%
\definecolor{currentfill}{rgb}{0.949020,0.372549,0.360784}%
\pgfsetfillcolor{currentfill}%
\pgfsetfillopacity{0.900000}%
\pgfsetlinewidth{1.003750pt}%
\definecolor{currentstroke}{rgb}{0.949020,0.372549,0.360784}%
\pgfsetstrokecolor{currentstroke}%
\pgfsetstrokeopacity{0.900000}%
\pgfsetdash{}{0pt}%
\pgfsys@defobject{currentmarker}{\pgfqpoint{-0.013889in}{-0.000000in}}{\pgfqpoint{0.013889in}{0.000000in}}{%
\pgfpathmoveto{\pgfqpoint{0.013889in}{-0.000000in}}%
\pgfpathlineto{\pgfqpoint{-0.013889in}{0.000000in}}%
\pgfusepath{stroke,fill}%
}%
\begin{pgfscope}%
\pgfsys@transformshift{0.821931in}{2.554081in}%
\pgfsys@useobject{currentmarker}{}%
\end{pgfscope}%
\begin{pgfscope}%
\pgfsys@transformshift{1.114887in}{2.228136in}%
\pgfsys@useobject{currentmarker}{}%
\end{pgfscope}%
\begin{pgfscope}%
\pgfsys@transformshift{1.407844in}{1.999711in}%
\pgfsys@useobject{currentmarker}{}%
\end{pgfscope}%
\begin{pgfscope}%
\pgfsys@transformshift{1.700800in}{1.979916in}%
\pgfsys@useobject{currentmarker}{}%
\end{pgfscope}%
\begin{pgfscope}%
\pgfsys@transformshift{1.993756in}{2.054461in}%
\pgfsys@useobject{currentmarker}{}%
\end{pgfscope}%
\begin{pgfscope}%
\pgfsys@transformshift{2.286712in}{2.001483in}%
\pgfsys@useobject{currentmarker}{}%
\end{pgfscope}%
\begin{pgfscope}%
\pgfsys@transformshift{2.579669in}{1.884550in}%
\pgfsys@useobject{currentmarker}{}%
\end{pgfscope}%
\begin{pgfscope}%
\pgfsys@transformshift{2.872625in}{1.794023in}%
\pgfsys@useobject{currentmarker}{}%
\end{pgfscope}%
\begin{pgfscope}%
\pgfsys@transformshift{3.165581in}{1.610118in}%
\pgfsys@useobject{currentmarker}{}%
\end{pgfscope}%
\begin{pgfscope}%
\pgfsys@transformshift{3.458537in}{1.453479in}%
\pgfsys@useobject{currentmarker}{}%
\end{pgfscope}%
\begin{pgfscope}%
\pgfsys@transformshift{3.751494in}{1.320713in}%
\pgfsys@useobject{currentmarker}{}%
\end{pgfscope}%
\begin{pgfscope}%
\pgfsys@transformshift{4.044450in}{1.171330in}%
\pgfsys@useobject{currentmarker}{}%
\end{pgfscope}%
\begin{pgfscope}%
\pgfsys@transformshift{4.337406in}{1.024811in}%
\pgfsys@useobject{currentmarker}{}%
\end{pgfscope}%
\begin{pgfscope}%
\pgfsys@transformshift{4.630362in}{0.869426in}%
\pgfsys@useobject{currentmarker}{}%
\end{pgfscope}%
\begin{pgfscope}%
\pgfsys@transformshift{4.923318in}{0.809511in}%
\pgfsys@useobject{currentmarker}{}%
\end{pgfscope}%
\begin{pgfscope}%
\pgfsys@transformshift{5.216275in}{0.823589in}%
\pgfsys@useobject{currentmarker}{}%
\end{pgfscope}%
\begin{pgfscope}%
\pgfsys@transformshift{5.509231in}{0.802225in}%
\pgfsys@useobject{currentmarker}{}%
\end{pgfscope}%
\begin{pgfscope}%
\pgfsys@transformshift{5.802187in}{0.976602in}%
\pgfsys@useobject{currentmarker}{}%
\end{pgfscope}%
\end{pgfscope}%
\begin{pgfscope}%
\pgfpathrectangle{\pgfqpoint{0.572918in}{0.553781in}}{\pgfqpoint{5.478282in}{2.095553in}}%
\pgfusepath{clip}%
\pgfsetrectcap%
\pgfsetroundjoin%
\pgfsetlinewidth{1.505625pt}%
\definecolor{currentstroke}{rgb}{0.313725,0.317647,0.309804}%
\pgfsetstrokecolor{currentstroke}%
\pgfsetstrokeopacity{0.900000}%
\pgfsetdash{}{0pt}%
\pgfpathmoveto{\pgfqpoint{0.821931in}{2.092589in}}%
\pgfpathlineto{\pgfqpoint{1.114887in}{1.978903in}}%
\pgfpathlineto{\pgfqpoint{1.407844in}{1.964965in}}%
\pgfpathlineto{\pgfqpoint{1.700800in}{1.987525in}}%
\pgfpathlineto{\pgfqpoint{1.993756in}{1.985658in}}%
\pgfpathlineto{\pgfqpoint{2.286712in}{1.986148in}}%
\pgfpathlineto{\pgfqpoint{2.579669in}{1.828223in}}%
\pgfpathlineto{\pgfqpoint{2.872625in}{1.771188in}}%
\pgfpathlineto{\pgfqpoint{3.165581in}{1.590381in}}%
\pgfpathlineto{\pgfqpoint{3.458537in}{1.406914in}}%
\pgfpathlineto{\pgfqpoint{3.751494in}{1.289295in}}%
\pgfpathlineto{\pgfqpoint{4.044450in}{1.190443in}}%
\pgfpathlineto{\pgfqpoint{4.337406in}{1.036798in}}%
\pgfpathlineto{\pgfqpoint{4.630362in}{0.923076in}}%
\pgfpathlineto{\pgfqpoint{4.923318in}{0.822088in}}%
\pgfpathlineto{\pgfqpoint{5.216275in}{0.888400in}}%
\pgfpathlineto{\pgfqpoint{5.509231in}{0.792090in}}%
\pgfpathlineto{\pgfqpoint{5.802187in}{0.928374in}}%
\pgfusepath{stroke}%
\end{pgfscope}%
\begin{pgfscope}%
\pgfpathrectangle{\pgfqpoint{0.572918in}{0.553781in}}{\pgfqpoint{5.478282in}{2.095553in}}%
\pgfusepath{clip}%
\pgfsetbuttcap%
\pgfsetroundjoin%
\pgfsetlinewidth{1.505625pt}%
\definecolor{currentstroke}{rgb}{0.949020,0.372549,0.360784}%
\pgfsetstrokecolor{currentstroke}%
\pgfsetstrokeopacity{0.900000}%
\pgfsetdash{{1.500000pt}{2.475000pt}}{0.000000pt}%
\pgfpathmoveto{\pgfqpoint{0.821931in}{2.126059in}}%
\pgfpathlineto{\pgfqpoint{1.114887in}{2.053659in}}%
\pgfpathlineto{\pgfqpoint{1.407844in}{1.914204in}}%
\pgfpathlineto{\pgfqpoint{1.700800in}{1.929008in}}%
\pgfpathlineto{\pgfqpoint{1.993756in}{2.014239in}}%
\pgfpathlineto{\pgfqpoint{2.286712in}{1.971033in}}%
\pgfpathlineto{\pgfqpoint{2.579669in}{1.849783in}}%
\pgfpathlineto{\pgfqpoint{2.872625in}{1.762912in}}%
\pgfpathlineto{\pgfqpoint{3.165581in}{1.577294in}}%
\pgfpathlineto{\pgfqpoint{3.458537in}{1.423236in}}%
\pgfpathlineto{\pgfqpoint{3.751494in}{1.292686in}}%
\pgfpathlineto{\pgfqpoint{4.044450in}{1.137245in}}%
\pgfpathlineto{\pgfqpoint{4.337406in}{0.989968in}}%
\pgfpathlineto{\pgfqpoint{4.630362in}{0.828010in}}%
\pgfpathlineto{\pgfqpoint{4.923318in}{0.762947in}}%
\pgfpathlineto{\pgfqpoint{5.216275in}{0.750230in}}%
\pgfpathlineto{\pgfqpoint{5.509231in}{0.726517in}}%
\pgfpathlineto{\pgfqpoint{5.802187in}{0.858117in}}%
\pgfusepath{stroke}%
\end{pgfscope}%
\begin{pgfscope}%
\pgfsetrectcap%
\pgfsetmiterjoin%
\pgfsetlinewidth{0.803000pt}%
\definecolor{currentstroke}{rgb}{0.000000,0.000000,0.000000}%
\pgfsetstrokecolor{currentstroke}%
\pgfsetdash{}{0pt}%
\pgfpathmoveto{\pgfqpoint{0.572918in}{0.553781in}}%
\pgfpathlineto{\pgfqpoint{0.572918in}{2.649333in}}%
\pgfusepath{stroke}%
\end{pgfscope}%
\begin{pgfscope}%
\pgfsetrectcap%
\pgfsetmiterjoin%
\pgfsetlinewidth{0.803000pt}%
\definecolor{currentstroke}{rgb}{0.000000,0.000000,0.000000}%
\pgfsetstrokecolor{currentstroke}%
\pgfsetdash{}{0pt}%
\pgfpathmoveto{\pgfqpoint{6.051200in}{0.553781in}}%
\pgfpathlineto{\pgfqpoint{6.051200in}{2.649333in}}%
\pgfusepath{stroke}%
\end{pgfscope}%
\begin{pgfscope}%
\pgfsetrectcap%
\pgfsetmiterjoin%
\pgfsetlinewidth{0.803000pt}%
\definecolor{currentstroke}{rgb}{0.000000,0.000000,0.000000}%
\pgfsetstrokecolor{currentstroke}%
\pgfsetdash{}{0pt}%
\pgfpathmoveto{\pgfqpoint{0.572918in}{0.553781in}}%
\pgfpathlineto{\pgfqpoint{6.051200in}{0.553781in}}%
\pgfusepath{stroke}%
\end{pgfscope}%
\begin{pgfscope}%
\pgfsetrectcap%
\pgfsetmiterjoin%
\pgfsetlinewidth{0.803000pt}%
\definecolor{currentstroke}{rgb}{0.000000,0.000000,0.000000}%
\pgfsetstrokecolor{currentstroke}%
\pgfsetdash{}{0pt}%
\pgfpathmoveto{\pgfqpoint{0.572918in}{2.649333in}}%
\pgfpathlineto{\pgfqpoint{6.051200in}{2.649333in}}%
\pgfusepath{stroke}%
\end{pgfscope}%
\begin{pgfscope}%
\definecolor{textcolor}{rgb}{0.000000,0.000000,0.000000}%
\pgfsetstrokecolor{textcolor}%
\pgfsetfillcolor{textcolor}%
\pgftext[x=0.572918in,y=2.732667in,left,base]{\color{textcolor}\rmfamily\fontsize{12.000000}{14.400000}\selectfont Zenith performance, input type}%
\end{pgfscope}%
\begin{pgfscope}%
\pgfsetbuttcap%
\pgfsetmiterjoin%
\definecolor{currentfill}{rgb}{1.000000,1.000000,1.000000}%
\pgfsetfillcolor{currentfill}%
\pgfsetfillopacity{0.800000}%
\pgfsetlinewidth{1.003750pt}%
\definecolor{currentstroke}{rgb}{0.800000,0.800000,0.800000}%
\pgfsetstrokecolor{currentstroke}%
\pgfsetstrokeopacity{0.800000}%
\pgfsetdash{}{0pt}%
\pgfpathmoveto{\pgfqpoint{4.918035in}{2.227111in}}%
\pgfpathlineto{\pgfqpoint{5.973422in}{2.227111in}}%
\pgfpathquadraticcurveto{\pgfqpoint{5.995644in}{2.227111in}}{\pgfqpoint{5.995644in}{2.249333in}}%
\pgfpathlineto{\pgfqpoint{5.995644in}{2.571556in}}%
\pgfpathquadraticcurveto{\pgfqpoint{5.995644in}{2.593778in}}{\pgfqpoint{5.973422in}{2.593778in}}%
\pgfpathlineto{\pgfqpoint{4.918035in}{2.593778in}}%
\pgfpathquadraticcurveto{\pgfqpoint{4.895813in}{2.593778in}}{\pgfqpoint{4.895813in}{2.571556in}}%
\pgfpathlineto{\pgfqpoint{4.895813in}{2.249333in}}%
\pgfpathquadraticcurveto{\pgfqpoint{4.895813in}{2.227111in}}{\pgfqpoint{4.918035in}{2.227111in}}%
\pgfpathclose%
\pgfusepath{stroke,fill}%
\end{pgfscope}%
\begin{pgfscope}%
\pgfsetbuttcap%
\pgfsetroundjoin%
\pgfsetlinewidth{1.505625pt}%
\definecolor{currentstroke}{rgb}{0.313725,0.317647,0.309804}%
\pgfsetstrokecolor{currentstroke}%
\pgfsetstrokeopacity{0.900000}%
\pgfsetdash{}{0pt}%
\pgfpathmoveto{\pgfqpoint{5.051369in}{2.449333in}}%
\pgfpathlineto{\pgfqpoint{5.051369in}{2.560444in}}%
\pgfusepath{stroke}%
\end{pgfscope}%
\begin{pgfscope}%
\pgfsetbuttcap%
\pgfsetroundjoin%
\definecolor{currentfill}{rgb}{0.313725,0.317647,0.309804}%
\pgfsetfillcolor{currentfill}%
\pgfsetfillopacity{0.900000}%
\pgfsetlinewidth{1.003750pt}%
\definecolor{currentstroke}{rgb}{0.313725,0.317647,0.309804}%
\pgfsetstrokecolor{currentstroke}%
\pgfsetstrokeopacity{0.900000}%
\pgfsetdash{}{0pt}%
\pgfsys@defobject{currentmarker}{\pgfqpoint{-0.013889in}{-0.000000in}}{\pgfqpoint{0.013889in}{0.000000in}}{%
\pgfpathmoveto{\pgfqpoint{0.013889in}{-0.000000in}}%
\pgfpathlineto{\pgfqpoint{-0.013889in}{0.000000in}}%
\pgfusepath{stroke,fill}%
}%
\begin{pgfscope}%
\pgfsys@transformshift{5.051369in}{2.449333in}%
\pgfsys@useobject{currentmarker}{}%
\end{pgfscope}%
\end{pgfscope}%
\begin{pgfscope}%
\pgfsetbuttcap%
\pgfsetroundjoin%
\definecolor{currentfill}{rgb}{0.313725,0.317647,0.309804}%
\pgfsetfillcolor{currentfill}%
\pgfsetfillopacity{0.900000}%
\pgfsetlinewidth{1.003750pt}%
\definecolor{currentstroke}{rgb}{0.313725,0.317647,0.309804}%
\pgfsetstrokecolor{currentstroke}%
\pgfsetstrokeopacity{0.900000}%
\pgfsetdash{}{0pt}%
\pgfsys@defobject{currentmarker}{\pgfqpoint{-0.013889in}{-0.000000in}}{\pgfqpoint{0.013889in}{0.000000in}}{%
\pgfpathmoveto{\pgfqpoint{0.013889in}{-0.000000in}}%
\pgfpathlineto{\pgfqpoint{-0.013889in}{0.000000in}}%
\pgfusepath{stroke,fill}%
}%
\begin{pgfscope}%
\pgfsys@transformshift{5.051369in}{2.560444in}%
\pgfsys@useobject{currentmarker}{}%
\end{pgfscope}%
\end{pgfscope}%
\begin{pgfscope}%
\pgfsetrectcap%
\pgfsetroundjoin%
\pgfsetlinewidth{1.505625pt}%
\definecolor{currentstroke}{rgb}{0.313725,0.317647,0.309804}%
\pgfsetstrokecolor{currentstroke}%
\pgfsetstrokeopacity{0.900000}%
\pgfsetdash{}{0pt}%
\pgfpathmoveto{\pgfqpoint{4.940258in}{2.504889in}}%
\pgfpathlineto{\pgfqpoint{5.162480in}{2.504889in}}%
\pgfusepath{stroke}%
\end{pgfscope}%
\begin{pgfscope}%
\definecolor{textcolor}{rgb}{0.000000,0.000000,0.000000}%
\pgfsetstrokecolor{textcolor}%
\pgfsetfillcolor{textcolor}%
\pgftext[x=5.251369in,y=2.466000in,left,base]{\color{textcolor}\rmfamily\fontsize{8.000000}{9.600000}\selectfont \(\displaystyle  \left(r, \theta, \phi \right) \) input}%
\end{pgfscope}%
\begin{pgfscope}%
\pgfsetbuttcap%
\pgfsetroundjoin%
\pgfsetlinewidth{1.505625pt}%
\definecolor{currentstroke}{rgb}{0.949020,0.372549,0.360784}%
\pgfsetstrokecolor{currentstroke}%
\pgfsetstrokeopacity{0.900000}%
\pgfsetdash{}{0pt}%
\pgfpathmoveto{\pgfqpoint{5.051369in}{2.282667in}}%
\pgfpathlineto{\pgfqpoint{5.051369in}{2.393778in}}%
\pgfusepath{stroke}%
\end{pgfscope}%
\begin{pgfscope}%
\pgfsetbuttcap%
\pgfsetroundjoin%
\definecolor{currentfill}{rgb}{0.949020,0.372549,0.360784}%
\pgfsetfillcolor{currentfill}%
\pgfsetfillopacity{0.900000}%
\pgfsetlinewidth{1.003750pt}%
\definecolor{currentstroke}{rgb}{0.949020,0.372549,0.360784}%
\pgfsetstrokecolor{currentstroke}%
\pgfsetstrokeopacity{0.900000}%
\pgfsetdash{}{0pt}%
\pgfsys@defobject{currentmarker}{\pgfqpoint{-0.013889in}{-0.000000in}}{\pgfqpoint{0.013889in}{0.000000in}}{%
\pgfpathmoveto{\pgfqpoint{0.013889in}{-0.000000in}}%
\pgfpathlineto{\pgfqpoint{-0.013889in}{0.000000in}}%
\pgfusepath{stroke,fill}%
}%
\begin{pgfscope}%
\pgfsys@transformshift{5.051369in}{2.282667in}%
\pgfsys@useobject{currentmarker}{}%
\end{pgfscope}%
\end{pgfscope}%
\begin{pgfscope}%
\pgfsetbuttcap%
\pgfsetroundjoin%
\definecolor{currentfill}{rgb}{0.949020,0.372549,0.360784}%
\pgfsetfillcolor{currentfill}%
\pgfsetfillopacity{0.900000}%
\pgfsetlinewidth{1.003750pt}%
\definecolor{currentstroke}{rgb}{0.949020,0.372549,0.360784}%
\pgfsetstrokecolor{currentstroke}%
\pgfsetstrokeopacity{0.900000}%
\pgfsetdash{}{0pt}%
\pgfsys@defobject{currentmarker}{\pgfqpoint{-0.013889in}{-0.000000in}}{\pgfqpoint{0.013889in}{0.000000in}}{%
\pgfpathmoveto{\pgfqpoint{0.013889in}{-0.000000in}}%
\pgfpathlineto{\pgfqpoint{-0.013889in}{0.000000in}}%
\pgfusepath{stroke,fill}%
}%
\begin{pgfscope}%
\pgfsys@transformshift{5.051369in}{2.393778in}%
\pgfsys@useobject{currentmarker}{}%
\end{pgfscope}%
\end{pgfscope}%
\begin{pgfscope}%
\pgfsetbuttcap%
\pgfsetroundjoin%
\pgfsetlinewidth{1.505625pt}%
\definecolor{currentstroke}{rgb}{0.949020,0.372549,0.360784}%
\pgfsetstrokecolor{currentstroke}%
\pgfsetstrokeopacity{0.900000}%
\pgfsetdash{{1.500000pt}{2.475000pt}}{0.000000pt}%
\pgfpathmoveto{\pgfqpoint{4.940258in}{2.338222in}}%
\pgfpathlineto{\pgfqpoint{5.162480in}{2.338222in}}%
\pgfusepath{stroke}%
\end{pgfscope}%
\begin{pgfscope}%
\definecolor{textcolor}{rgb}{0.000000,0.000000,0.000000}%
\pgfsetstrokecolor{textcolor}%
\pgfsetfillcolor{textcolor}%
\pgftext[x=5.251369in,y=2.299333in,left,base]{\color{textcolor}\rmfamily\fontsize{8.000000}{9.600000}\selectfont \(\displaystyle  \left(x, y, z \right) \) input}%
\end{pgfscope}%
\end{pgfpicture}%
\makeatother%
\endgroup%

    \caption{The difference between inputting the original detector data, in Cartesian coordinates, to the network, or doing a spherical transform beforehand, when predicting the neutrino's zenith directional coordinate.
    The performance is similar.}\label{fig:input_types}
\end{figure}

This was tested, as the thesis was that converting the coordinates beforehand might lessen the burden on the neural network, and the result can be seen in~\vref{fig:input_types}.
The performance is similar, and goes to show the power of a universal approximator: the network will discover the correlations itself, even between coordinate transformations.

Lastly, it is possible to predict \textit{both} the azimuthal and polar angle in the same network; this was, however, seen to lead to a slight decrease in zenith resolution, as seen in~\vref{fig:prediction_types}.

\subsection{Cleaning}

\begin{figure}
     \centering
     \begin{subfigure}[b]{0.49\textwidth}
         \centering
            \includegraphics[width=\textwidth]{./images/design/uncleaned_event.png}
            \caption{}\label{fig:uncleaned_powershovel_event}
     \end{subfigure}
     \hfill
     \begin{subfigure}[b]{0.49\textwidth}
         \centering
            \includegraphics[width=\textwidth]{./images/design/cleaned_event.png}
            \caption{}\label{fig:cleaned_powershovel_event}
     \end{subfigure}
        \caption{Event ID 124492073, a \SI{35.8}{\giga\electronvolt} event shown in Powershovel with no SRT cleaning (\vref{fig:uncleaned_powershovel_event}) and with (\vref{fig:cleaned_powershovel_event}).
        This is a track-like muon neutrino event, the neutrino path indicated by the blue line.
        Each black dot represents a DOM, and the pulses are colored according to the \enquote{plasma} color map with yellower pulses being later in time than bluer pulses.
        The size of the pulse is proportional to its charge.
        As is evident, SRT cleaning removes a large amount of pulses in this example, and the cleaned event leaves a sense of direction in the pulse which can be inferred by a human eye, and surely is easier to infer for a neural network.
        Note that the color scale is different on the two plots; this is caused by a missing feature not yet implemented in Powershovel.}\label{fig:two_powershovel_events}
\end{figure}

\begin{figure}
    \centering
    %% Creator: Matplotlib, PGF backend
%%
%% To include the figure in your LaTeX document, write
%%   \input{<filename>.pgf}
%%
%% Make sure the required packages are loaded in your preamble
%%   \usepackage{pgf}
%%
%% and, on pdftex
%%   \usepackage[utf8]{inputenc}\DeclareUnicodeCharacter{2212}{-}
%%
%% or, on luatex and xetex
%%   \usepackage{unicode-math}
%%
%% Figures using additional raster images can only be included by \input if
%% they are in the same directory as the main LaTeX file. For loading figures
%% from other directories you can use the `import` package
%%   \usepackage{import}
%%
%% and then include the figures with
%%   \import{<path to file>}{<filename>.pgf}
%%
%% Matplotlib used the following preamble
%%   \usepackage{siunitx} \usepackage{amsmath} \usepackage{bm}
%%   \usepackage{fontspec}
%%
\begingroup%
\makeatletter%
\begin{pgfpicture}%
\pgfpathrectangle{\pgfpointorigin}{\pgfqpoint{6.201200in}{3.000000in}}%
\pgfusepath{use as bounding box, clip}%
\begin{pgfscope}%
\pgfsetbuttcap%
\pgfsetmiterjoin%
\definecolor{currentfill}{rgb}{1.000000,1.000000,1.000000}%
\pgfsetfillcolor{currentfill}%
\pgfsetlinewidth{0.000000pt}%
\definecolor{currentstroke}{rgb}{1.000000,1.000000,1.000000}%
\pgfsetstrokecolor{currentstroke}%
\pgfsetdash{}{0pt}%
\pgfpathmoveto{\pgfqpoint{0.000000in}{0.000000in}}%
\pgfpathlineto{\pgfqpoint{6.201200in}{0.000000in}}%
\pgfpathlineto{\pgfqpoint{6.201200in}{3.000000in}}%
\pgfpathlineto{\pgfqpoint{0.000000in}{3.000000in}}%
\pgfpathclose%
\pgfusepath{fill}%
\end{pgfscope}%
\begin{pgfscope}%
\pgfsetbuttcap%
\pgfsetmiterjoin%
\definecolor{currentfill}{rgb}{1.000000,1.000000,1.000000}%
\pgfsetfillcolor{currentfill}%
\pgfsetlinewidth{0.000000pt}%
\definecolor{currentstroke}{rgb}{0.000000,0.000000,0.000000}%
\pgfsetstrokecolor{currentstroke}%
\pgfsetstrokeopacity{0.000000}%
\pgfsetdash{}{0pt}%
\pgfpathmoveto{\pgfqpoint{0.572918in}{0.553781in}}%
\pgfpathlineto{\pgfqpoint{6.051200in}{0.553781in}}%
\pgfpathlineto{\pgfqpoint{6.051200in}{2.649333in}}%
\pgfpathlineto{\pgfqpoint{0.572918in}{2.649333in}}%
\pgfpathclose%
\pgfusepath{fill}%
\end{pgfscope}%
\begin{pgfscope}%
\pgfpathrectangle{\pgfqpoint{0.572918in}{0.553781in}}{\pgfqpoint{5.478282in}{2.095553in}}%
\pgfusepath{clip}%
\pgfsetbuttcap%
\pgfsetroundjoin%
\pgfsetlinewidth{0.501875pt}%
\definecolor{currentstroke}{rgb}{0.690196,0.690196,0.690196}%
\pgfsetstrokecolor{currentstroke}%
\pgfsetstrokeopacity{0.500000}%
\pgfsetdash{{0.500000pt}{0.825000pt}}{0.000000pt}%
\pgfpathmoveto{\pgfqpoint{0.675453in}{0.553781in}}%
\pgfpathlineto{\pgfqpoint{0.675453in}{2.649333in}}%
\pgfusepath{stroke}%
\end{pgfscope}%
\begin{pgfscope}%
\pgfsetbuttcap%
\pgfsetroundjoin%
\definecolor{currentfill}{rgb}{0.000000,0.000000,0.000000}%
\pgfsetfillcolor{currentfill}%
\pgfsetlinewidth{0.803000pt}%
\definecolor{currentstroke}{rgb}{0.000000,0.000000,0.000000}%
\pgfsetstrokecolor{currentstroke}%
\pgfsetdash{}{0pt}%
\pgfsys@defobject{currentmarker}{\pgfqpoint{0.000000in}{-0.048611in}}{\pgfqpoint{0.000000in}{0.000000in}}{%
\pgfpathmoveto{\pgfqpoint{0.000000in}{0.000000in}}%
\pgfpathlineto{\pgfqpoint{0.000000in}{-0.048611in}}%
\pgfusepath{stroke,fill}%
}%
\begin{pgfscope}%
\pgfsys@transformshift{0.675453in}{0.553781in}%
\pgfsys@useobject{currentmarker}{}%
\end{pgfscope}%
\end{pgfscope}%
\begin{pgfscope}%
\definecolor{textcolor}{rgb}{0.000000,0.000000,0.000000}%
\pgfsetstrokecolor{textcolor}%
\pgfsetfillcolor{textcolor}%
\pgftext[x=0.675453in,y=0.456558in,,top]{\color{textcolor}\rmfamily\fontsize{8.000000}{9.600000}\selectfont \(\displaystyle {0.0}\)}%
\end{pgfscope}%
\begin{pgfscope}%
\pgfpathrectangle{\pgfqpoint{0.572918in}{0.553781in}}{\pgfqpoint{5.478282in}{2.095553in}}%
\pgfusepath{clip}%
\pgfsetbuttcap%
\pgfsetroundjoin%
\pgfsetlinewidth{0.501875pt}%
\definecolor{currentstroke}{rgb}{0.690196,0.690196,0.690196}%
\pgfsetstrokecolor{currentstroke}%
\pgfsetstrokeopacity{0.500000}%
\pgfsetdash{{0.500000pt}{0.825000pt}}{0.000000pt}%
\pgfpathmoveto{\pgfqpoint{1.554322in}{0.553781in}}%
\pgfpathlineto{\pgfqpoint{1.554322in}{2.649333in}}%
\pgfusepath{stroke}%
\end{pgfscope}%
\begin{pgfscope}%
\pgfsetbuttcap%
\pgfsetroundjoin%
\definecolor{currentfill}{rgb}{0.000000,0.000000,0.000000}%
\pgfsetfillcolor{currentfill}%
\pgfsetlinewidth{0.803000pt}%
\definecolor{currentstroke}{rgb}{0.000000,0.000000,0.000000}%
\pgfsetstrokecolor{currentstroke}%
\pgfsetdash{}{0pt}%
\pgfsys@defobject{currentmarker}{\pgfqpoint{0.000000in}{-0.048611in}}{\pgfqpoint{0.000000in}{0.000000in}}{%
\pgfpathmoveto{\pgfqpoint{0.000000in}{0.000000in}}%
\pgfpathlineto{\pgfqpoint{0.000000in}{-0.048611in}}%
\pgfusepath{stroke,fill}%
}%
\begin{pgfscope}%
\pgfsys@transformshift{1.554322in}{0.553781in}%
\pgfsys@useobject{currentmarker}{}%
\end{pgfscope}%
\end{pgfscope}%
\begin{pgfscope}%
\definecolor{textcolor}{rgb}{0.000000,0.000000,0.000000}%
\pgfsetstrokecolor{textcolor}%
\pgfsetfillcolor{textcolor}%
\pgftext[x=1.554322in,y=0.456558in,,top]{\color{textcolor}\rmfamily\fontsize{8.000000}{9.600000}\selectfont \(\displaystyle {0.5}\)}%
\end{pgfscope}%
\begin{pgfscope}%
\pgfpathrectangle{\pgfqpoint{0.572918in}{0.553781in}}{\pgfqpoint{5.478282in}{2.095553in}}%
\pgfusepath{clip}%
\pgfsetbuttcap%
\pgfsetroundjoin%
\pgfsetlinewidth{0.501875pt}%
\definecolor{currentstroke}{rgb}{0.690196,0.690196,0.690196}%
\pgfsetstrokecolor{currentstroke}%
\pgfsetstrokeopacity{0.500000}%
\pgfsetdash{{0.500000pt}{0.825000pt}}{0.000000pt}%
\pgfpathmoveto{\pgfqpoint{2.433190in}{0.553781in}}%
\pgfpathlineto{\pgfqpoint{2.433190in}{2.649333in}}%
\pgfusepath{stroke}%
\end{pgfscope}%
\begin{pgfscope}%
\pgfsetbuttcap%
\pgfsetroundjoin%
\definecolor{currentfill}{rgb}{0.000000,0.000000,0.000000}%
\pgfsetfillcolor{currentfill}%
\pgfsetlinewidth{0.803000pt}%
\definecolor{currentstroke}{rgb}{0.000000,0.000000,0.000000}%
\pgfsetstrokecolor{currentstroke}%
\pgfsetdash{}{0pt}%
\pgfsys@defobject{currentmarker}{\pgfqpoint{0.000000in}{-0.048611in}}{\pgfqpoint{0.000000in}{0.000000in}}{%
\pgfpathmoveto{\pgfqpoint{0.000000in}{0.000000in}}%
\pgfpathlineto{\pgfqpoint{0.000000in}{-0.048611in}}%
\pgfusepath{stroke,fill}%
}%
\begin{pgfscope}%
\pgfsys@transformshift{2.433190in}{0.553781in}%
\pgfsys@useobject{currentmarker}{}%
\end{pgfscope}%
\end{pgfscope}%
\begin{pgfscope}%
\definecolor{textcolor}{rgb}{0.000000,0.000000,0.000000}%
\pgfsetstrokecolor{textcolor}%
\pgfsetfillcolor{textcolor}%
\pgftext[x=2.433190in,y=0.456558in,,top]{\color{textcolor}\rmfamily\fontsize{8.000000}{9.600000}\selectfont \(\displaystyle {1.0}\)}%
\end{pgfscope}%
\begin{pgfscope}%
\pgfpathrectangle{\pgfqpoint{0.572918in}{0.553781in}}{\pgfqpoint{5.478282in}{2.095553in}}%
\pgfusepath{clip}%
\pgfsetbuttcap%
\pgfsetroundjoin%
\pgfsetlinewidth{0.501875pt}%
\definecolor{currentstroke}{rgb}{0.690196,0.690196,0.690196}%
\pgfsetstrokecolor{currentstroke}%
\pgfsetstrokeopacity{0.500000}%
\pgfsetdash{{0.500000pt}{0.825000pt}}{0.000000pt}%
\pgfpathmoveto{\pgfqpoint{3.312059in}{0.553781in}}%
\pgfpathlineto{\pgfqpoint{3.312059in}{2.649333in}}%
\pgfusepath{stroke}%
\end{pgfscope}%
\begin{pgfscope}%
\pgfsetbuttcap%
\pgfsetroundjoin%
\definecolor{currentfill}{rgb}{0.000000,0.000000,0.000000}%
\pgfsetfillcolor{currentfill}%
\pgfsetlinewidth{0.803000pt}%
\definecolor{currentstroke}{rgb}{0.000000,0.000000,0.000000}%
\pgfsetstrokecolor{currentstroke}%
\pgfsetdash{}{0pt}%
\pgfsys@defobject{currentmarker}{\pgfqpoint{0.000000in}{-0.048611in}}{\pgfqpoint{0.000000in}{0.000000in}}{%
\pgfpathmoveto{\pgfqpoint{0.000000in}{0.000000in}}%
\pgfpathlineto{\pgfqpoint{0.000000in}{-0.048611in}}%
\pgfusepath{stroke,fill}%
}%
\begin{pgfscope}%
\pgfsys@transformshift{3.312059in}{0.553781in}%
\pgfsys@useobject{currentmarker}{}%
\end{pgfscope}%
\end{pgfscope}%
\begin{pgfscope}%
\definecolor{textcolor}{rgb}{0.000000,0.000000,0.000000}%
\pgfsetstrokecolor{textcolor}%
\pgfsetfillcolor{textcolor}%
\pgftext[x=3.312059in,y=0.456558in,,top]{\color{textcolor}\rmfamily\fontsize{8.000000}{9.600000}\selectfont \(\displaystyle {1.5}\)}%
\end{pgfscope}%
\begin{pgfscope}%
\pgfpathrectangle{\pgfqpoint{0.572918in}{0.553781in}}{\pgfqpoint{5.478282in}{2.095553in}}%
\pgfusepath{clip}%
\pgfsetbuttcap%
\pgfsetroundjoin%
\pgfsetlinewidth{0.501875pt}%
\definecolor{currentstroke}{rgb}{0.690196,0.690196,0.690196}%
\pgfsetstrokecolor{currentstroke}%
\pgfsetstrokeopacity{0.500000}%
\pgfsetdash{{0.500000pt}{0.825000pt}}{0.000000pt}%
\pgfpathmoveto{\pgfqpoint{4.190928in}{0.553781in}}%
\pgfpathlineto{\pgfqpoint{4.190928in}{2.649333in}}%
\pgfusepath{stroke}%
\end{pgfscope}%
\begin{pgfscope}%
\pgfsetbuttcap%
\pgfsetroundjoin%
\definecolor{currentfill}{rgb}{0.000000,0.000000,0.000000}%
\pgfsetfillcolor{currentfill}%
\pgfsetlinewidth{0.803000pt}%
\definecolor{currentstroke}{rgb}{0.000000,0.000000,0.000000}%
\pgfsetstrokecolor{currentstroke}%
\pgfsetdash{}{0pt}%
\pgfsys@defobject{currentmarker}{\pgfqpoint{0.000000in}{-0.048611in}}{\pgfqpoint{0.000000in}{0.000000in}}{%
\pgfpathmoveto{\pgfqpoint{0.000000in}{0.000000in}}%
\pgfpathlineto{\pgfqpoint{0.000000in}{-0.048611in}}%
\pgfusepath{stroke,fill}%
}%
\begin{pgfscope}%
\pgfsys@transformshift{4.190928in}{0.553781in}%
\pgfsys@useobject{currentmarker}{}%
\end{pgfscope}%
\end{pgfscope}%
\begin{pgfscope}%
\definecolor{textcolor}{rgb}{0.000000,0.000000,0.000000}%
\pgfsetstrokecolor{textcolor}%
\pgfsetfillcolor{textcolor}%
\pgftext[x=4.190928in,y=0.456558in,,top]{\color{textcolor}\rmfamily\fontsize{8.000000}{9.600000}\selectfont \(\displaystyle {2.0}\)}%
\end{pgfscope}%
\begin{pgfscope}%
\pgfpathrectangle{\pgfqpoint{0.572918in}{0.553781in}}{\pgfqpoint{5.478282in}{2.095553in}}%
\pgfusepath{clip}%
\pgfsetbuttcap%
\pgfsetroundjoin%
\pgfsetlinewidth{0.501875pt}%
\definecolor{currentstroke}{rgb}{0.690196,0.690196,0.690196}%
\pgfsetstrokecolor{currentstroke}%
\pgfsetstrokeopacity{0.500000}%
\pgfsetdash{{0.500000pt}{0.825000pt}}{0.000000pt}%
\pgfpathmoveto{\pgfqpoint{5.069797in}{0.553781in}}%
\pgfpathlineto{\pgfqpoint{5.069797in}{2.649333in}}%
\pgfusepath{stroke}%
\end{pgfscope}%
\begin{pgfscope}%
\pgfsetbuttcap%
\pgfsetroundjoin%
\definecolor{currentfill}{rgb}{0.000000,0.000000,0.000000}%
\pgfsetfillcolor{currentfill}%
\pgfsetlinewidth{0.803000pt}%
\definecolor{currentstroke}{rgb}{0.000000,0.000000,0.000000}%
\pgfsetstrokecolor{currentstroke}%
\pgfsetdash{}{0pt}%
\pgfsys@defobject{currentmarker}{\pgfqpoint{0.000000in}{-0.048611in}}{\pgfqpoint{0.000000in}{0.000000in}}{%
\pgfpathmoveto{\pgfqpoint{0.000000in}{0.000000in}}%
\pgfpathlineto{\pgfqpoint{0.000000in}{-0.048611in}}%
\pgfusepath{stroke,fill}%
}%
\begin{pgfscope}%
\pgfsys@transformshift{5.069797in}{0.553781in}%
\pgfsys@useobject{currentmarker}{}%
\end{pgfscope}%
\end{pgfscope}%
\begin{pgfscope}%
\definecolor{textcolor}{rgb}{0.000000,0.000000,0.000000}%
\pgfsetstrokecolor{textcolor}%
\pgfsetfillcolor{textcolor}%
\pgftext[x=5.069797in,y=0.456558in,,top]{\color{textcolor}\rmfamily\fontsize{8.000000}{9.600000}\selectfont \(\displaystyle {2.5}\)}%
\end{pgfscope}%
\begin{pgfscope}%
\pgfpathrectangle{\pgfqpoint{0.572918in}{0.553781in}}{\pgfqpoint{5.478282in}{2.095553in}}%
\pgfusepath{clip}%
\pgfsetbuttcap%
\pgfsetroundjoin%
\pgfsetlinewidth{0.501875pt}%
\definecolor{currentstroke}{rgb}{0.690196,0.690196,0.690196}%
\pgfsetstrokecolor{currentstroke}%
\pgfsetstrokeopacity{0.500000}%
\pgfsetdash{{0.500000pt}{0.825000pt}}{0.000000pt}%
\pgfpathmoveto{\pgfqpoint{5.948665in}{0.553781in}}%
\pgfpathlineto{\pgfqpoint{5.948665in}{2.649333in}}%
\pgfusepath{stroke}%
\end{pgfscope}%
\begin{pgfscope}%
\pgfsetbuttcap%
\pgfsetroundjoin%
\definecolor{currentfill}{rgb}{0.000000,0.000000,0.000000}%
\pgfsetfillcolor{currentfill}%
\pgfsetlinewidth{0.803000pt}%
\definecolor{currentstroke}{rgb}{0.000000,0.000000,0.000000}%
\pgfsetstrokecolor{currentstroke}%
\pgfsetdash{}{0pt}%
\pgfsys@defobject{currentmarker}{\pgfqpoint{0.000000in}{-0.048611in}}{\pgfqpoint{0.000000in}{0.000000in}}{%
\pgfpathmoveto{\pgfqpoint{0.000000in}{0.000000in}}%
\pgfpathlineto{\pgfqpoint{0.000000in}{-0.048611in}}%
\pgfusepath{stroke,fill}%
}%
\begin{pgfscope}%
\pgfsys@transformshift{5.948665in}{0.553781in}%
\pgfsys@useobject{currentmarker}{}%
\end{pgfscope}%
\end{pgfscope}%
\begin{pgfscope}%
\definecolor{textcolor}{rgb}{0.000000,0.000000,0.000000}%
\pgfsetstrokecolor{textcolor}%
\pgfsetfillcolor{textcolor}%
\pgftext[x=5.948665in,y=0.456558in,,top]{\color{textcolor}\rmfamily\fontsize{8.000000}{9.600000}\selectfont \(\displaystyle {3.0}\)}%
\end{pgfscope}%
\begin{pgfscope}%
\definecolor{textcolor}{rgb}{0.000000,0.000000,0.000000}%
\pgfsetstrokecolor{textcolor}%
\pgfsetfillcolor{textcolor}%
\pgftext[x=3.312059in,y=0.302336in,,top]{\color{textcolor}\rmfamily\fontsize{10.950000}{13.140000}\selectfont \(\displaystyle \log_{10}(E_{\textup{true}}) \, \left[ E / \textup{GeV} \right]\)}%
\end{pgfscope}%
\begin{pgfscope}%
\pgfpathrectangle{\pgfqpoint{0.572918in}{0.553781in}}{\pgfqpoint{5.478282in}{2.095553in}}%
\pgfusepath{clip}%
\pgfsetbuttcap%
\pgfsetroundjoin%
\pgfsetlinewidth{0.501875pt}%
\definecolor{currentstroke}{rgb}{0.690196,0.690196,0.690196}%
\pgfsetstrokecolor{currentstroke}%
\pgfsetstrokeopacity{0.500000}%
\pgfsetdash{{0.500000pt}{0.825000pt}}{0.000000pt}%
\pgfpathmoveto{\pgfqpoint{0.572918in}{0.825541in}}%
\pgfpathlineto{\pgfqpoint{6.051200in}{0.825541in}}%
\pgfusepath{stroke}%
\end{pgfscope}%
\begin{pgfscope}%
\pgfsetbuttcap%
\pgfsetroundjoin%
\definecolor{currentfill}{rgb}{0.000000,0.000000,0.000000}%
\pgfsetfillcolor{currentfill}%
\pgfsetlinewidth{0.803000pt}%
\definecolor{currentstroke}{rgb}{0.000000,0.000000,0.000000}%
\pgfsetstrokecolor{currentstroke}%
\pgfsetdash{}{0pt}%
\pgfsys@defobject{currentmarker}{\pgfqpoint{-0.048611in}{0.000000in}}{\pgfqpoint{-0.000000in}{0.000000in}}{%
\pgfpathmoveto{\pgfqpoint{-0.000000in}{0.000000in}}%
\pgfpathlineto{\pgfqpoint{-0.048611in}{0.000000in}}%
\pgfusepath{stroke,fill}%
}%
\begin{pgfscope}%
\pgfsys@transformshift{0.572918in}{0.825541in}%
\pgfsys@useobject{currentmarker}{}%
\end{pgfscope}%
\end{pgfscope}%
\begin{pgfscope}%
\definecolor{textcolor}{rgb}{0.000000,0.000000,0.000000}%
\pgfsetstrokecolor{textcolor}%
\pgfsetfillcolor{textcolor}%
\pgftext[x=0.357639in, y=0.786985in, left, base]{\color{textcolor}\rmfamily\fontsize{8.000000}{9.600000}\selectfont \(\displaystyle {15}\)}%
\end{pgfscope}%
\begin{pgfscope}%
\pgfpathrectangle{\pgfqpoint{0.572918in}{0.553781in}}{\pgfqpoint{5.478282in}{2.095553in}}%
\pgfusepath{clip}%
\pgfsetbuttcap%
\pgfsetroundjoin%
\pgfsetlinewidth{0.501875pt}%
\definecolor{currentstroke}{rgb}{0.690196,0.690196,0.690196}%
\pgfsetstrokecolor{currentstroke}%
\pgfsetstrokeopacity{0.500000}%
\pgfsetdash{{0.500000pt}{0.825000pt}}{0.000000pt}%
\pgfpathmoveto{\pgfqpoint{0.572918in}{1.207745in}}%
\pgfpathlineto{\pgfqpoint{6.051200in}{1.207745in}}%
\pgfusepath{stroke}%
\end{pgfscope}%
\begin{pgfscope}%
\pgfsetbuttcap%
\pgfsetroundjoin%
\definecolor{currentfill}{rgb}{0.000000,0.000000,0.000000}%
\pgfsetfillcolor{currentfill}%
\pgfsetlinewidth{0.803000pt}%
\definecolor{currentstroke}{rgb}{0.000000,0.000000,0.000000}%
\pgfsetstrokecolor{currentstroke}%
\pgfsetdash{}{0pt}%
\pgfsys@defobject{currentmarker}{\pgfqpoint{-0.048611in}{0.000000in}}{\pgfqpoint{-0.000000in}{0.000000in}}{%
\pgfpathmoveto{\pgfqpoint{-0.000000in}{0.000000in}}%
\pgfpathlineto{\pgfqpoint{-0.048611in}{0.000000in}}%
\pgfusepath{stroke,fill}%
}%
\begin{pgfscope}%
\pgfsys@transformshift{0.572918in}{1.207745in}%
\pgfsys@useobject{currentmarker}{}%
\end{pgfscope}%
\end{pgfscope}%
\begin{pgfscope}%
\definecolor{textcolor}{rgb}{0.000000,0.000000,0.000000}%
\pgfsetstrokecolor{textcolor}%
\pgfsetfillcolor{textcolor}%
\pgftext[x=0.357639in, y=1.169189in, left, base]{\color{textcolor}\rmfamily\fontsize{8.000000}{9.600000}\selectfont \(\displaystyle {20}\)}%
\end{pgfscope}%
\begin{pgfscope}%
\pgfpathrectangle{\pgfqpoint{0.572918in}{0.553781in}}{\pgfqpoint{5.478282in}{2.095553in}}%
\pgfusepath{clip}%
\pgfsetbuttcap%
\pgfsetroundjoin%
\pgfsetlinewidth{0.501875pt}%
\definecolor{currentstroke}{rgb}{0.690196,0.690196,0.690196}%
\pgfsetstrokecolor{currentstroke}%
\pgfsetstrokeopacity{0.500000}%
\pgfsetdash{{0.500000pt}{0.825000pt}}{0.000000pt}%
\pgfpathmoveto{\pgfqpoint{0.572918in}{1.589948in}}%
\pgfpathlineto{\pgfqpoint{6.051200in}{1.589948in}}%
\pgfusepath{stroke}%
\end{pgfscope}%
\begin{pgfscope}%
\pgfsetbuttcap%
\pgfsetroundjoin%
\definecolor{currentfill}{rgb}{0.000000,0.000000,0.000000}%
\pgfsetfillcolor{currentfill}%
\pgfsetlinewidth{0.803000pt}%
\definecolor{currentstroke}{rgb}{0.000000,0.000000,0.000000}%
\pgfsetstrokecolor{currentstroke}%
\pgfsetdash{}{0pt}%
\pgfsys@defobject{currentmarker}{\pgfqpoint{-0.048611in}{0.000000in}}{\pgfqpoint{-0.000000in}{0.000000in}}{%
\pgfpathmoveto{\pgfqpoint{-0.000000in}{0.000000in}}%
\pgfpathlineto{\pgfqpoint{-0.048611in}{0.000000in}}%
\pgfusepath{stroke,fill}%
}%
\begin{pgfscope}%
\pgfsys@transformshift{0.572918in}{1.589948in}%
\pgfsys@useobject{currentmarker}{}%
\end{pgfscope}%
\end{pgfscope}%
\begin{pgfscope}%
\definecolor{textcolor}{rgb}{0.000000,0.000000,0.000000}%
\pgfsetstrokecolor{textcolor}%
\pgfsetfillcolor{textcolor}%
\pgftext[x=0.357639in, y=1.551393in, left, base]{\color{textcolor}\rmfamily\fontsize{8.000000}{9.600000}\selectfont \(\displaystyle {25}\)}%
\end{pgfscope}%
\begin{pgfscope}%
\pgfpathrectangle{\pgfqpoint{0.572918in}{0.553781in}}{\pgfqpoint{5.478282in}{2.095553in}}%
\pgfusepath{clip}%
\pgfsetbuttcap%
\pgfsetroundjoin%
\pgfsetlinewidth{0.501875pt}%
\definecolor{currentstroke}{rgb}{0.690196,0.690196,0.690196}%
\pgfsetstrokecolor{currentstroke}%
\pgfsetstrokeopacity{0.500000}%
\pgfsetdash{{0.500000pt}{0.825000pt}}{0.000000pt}%
\pgfpathmoveto{\pgfqpoint{0.572918in}{1.972152in}}%
\pgfpathlineto{\pgfqpoint{6.051200in}{1.972152in}}%
\pgfusepath{stroke}%
\end{pgfscope}%
\begin{pgfscope}%
\pgfsetbuttcap%
\pgfsetroundjoin%
\definecolor{currentfill}{rgb}{0.000000,0.000000,0.000000}%
\pgfsetfillcolor{currentfill}%
\pgfsetlinewidth{0.803000pt}%
\definecolor{currentstroke}{rgb}{0.000000,0.000000,0.000000}%
\pgfsetstrokecolor{currentstroke}%
\pgfsetdash{}{0pt}%
\pgfsys@defobject{currentmarker}{\pgfqpoint{-0.048611in}{0.000000in}}{\pgfqpoint{-0.000000in}{0.000000in}}{%
\pgfpathmoveto{\pgfqpoint{-0.000000in}{0.000000in}}%
\pgfpathlineto{\pgfqpoint{-0.048611in}{0.000000in}}%
\pgfusepath{stroke,fill}%
}%
\begin{pgfscope}%
\pgfsys@transformshift{0.572918in}{1.972152in}%
\pgfsys@useobject{currentmarker}{}%
\end{pgfscope}%
\end{pgfscope}%
\begin{pgfscope}%
\definecolor{textcolor}{rgb}{0.000000,0.000000,0.000000}%
\pgfsetstrokecolor{textcolor}%
\pgfsetfillcolor{textcolor}%
\pgftext[x=0.357639in, y=1.933596in, left, base]{\color{textcolor}\rmfamily\fontsize{8.000000}{9.600000}\selectfont \(\displaystyle {30}\)}%
\end{pgfscope}%
\begin{pgfscope}%
\pgfpathrectangle{\pgfqpoint{0.572918in}{0.553781in}}{\pgfqpoint{5.478282in}{2.095553in}}%
\pgfusepath{clip}%
\pgfsetbuttcap%
\pgfsetroundjoin%
\pgfsetlinewidth{0.501875pt}%
\definecolor{currentstroke}{rgb}{0.690196,0.690196,0.690196}%
\pgfsetstrokecolor{currentstroke}%
\pgfsetstrokeopacity{0.500000}%
\pgfsetdash{{0.500000pt}{0.825000pt}}{0.000000pt}%
\pgfpathmoveto{\pgfqpoint{0.572918in}{2.354356in}}%
\pgfpathlineto{\pgfqpoint{6.051200in}{2.354356in}}%
\pgfusepath{stroke}%
\end{pgfscope}%
\begin{pgfscope}%
\pgfsetbuttcap%
\pgfsetroundjoin%
\definecolor{currentfill}{rgb}{0.000000,0.000000,0.000000}%
\pgfsetfillcolor{currentfill}%
\pgfsetlinewidth{0.803000pt}%
\definecolor{currentstroke}{rgb}{0.000000,0.000000,0.000000}%
\pgfsetstrokecolor{currentstroke}%
\pgfsetdash{}{0pt}%
\pgfsys@defobject{currentmarker}{\pgfqpoint{-0.048611in}{0.000000in}}{\pgfqpoint{-0.000000in}{0.000000in}}{%
\pgfpathmoveto{\pgfqpoint{-0.000000in}{0.000000in}}%
\pgfpathlineto{\pgfqpoint{-0.048611in}{0.000000in}}%
\pgfusepath{stroke,fill}%
}%
\begin{pgfscope}%
\pgfsys@transformshift{0.572918in}{2.354356in}%
\pgfsys@useobject{currentmarker}{}%
\end{pgfscope}%
\end{pgfscope}%
\begin{pgfscope}%
\definecolor{textcolor}{rgb}{0.000000,0.000000,0.000000}%
\pgfsetstrokecolor{textcolor}%
\pgfsetfillcolor{textcolor}%
\pgftext[x=0.357639in, y=2.315800in, left, base]{\color{textcolor}\rmfamily\fontsize{8.000000}{9.600000}\selectfont \(\displaystyle {35}\)}%
\end{pgfscope}%
\begin{pgfscope}%
\definecolor{textcolor}{rgb}{0.000000,0.000000,0.000000}%
\pgfsetstrokecolor{textcolor}%
\pgfsetfillcolor{textcolor}%
\pgftext[x=0.302083in,y=1.601557in,,bottom,rotate=90.000000]{\color{textcolor}\rmfamily\fontsize{10.950000}{13.140000}\selectfont IQR / 1.349 \(\displaystyle \left[ \textup{deg} \right]\)}%
\end{pgfscope}%
\begin{pgfscope}%
\pgfpathrectangle{\pgfqpoint{0.572918in}{0.553781in}}{\pgfqpoint{5.478282in}{2.095553in}}%
\pgfusepath{clip}%
\pgfsetbuttcap%
\pgfsetroundjoin%
\pgfsetlinewidth{1.505625pt}%
\definecolor{currentstroke}{rgb}{0.313725,0.317647,0.309804}%
\pgfsetstrokecolor{currentstroke}%
\pgfsetstrokeopacity{0.900000}%
\pgfsetdash{}{0pt}%
\pgfpathmoveto{\pgfqpoint{0.821931in}{1.909657in}}%
\pgfpathlineto{\pgfqpoint{0.821931in}{2.554081in}}%
\pgfusepath{stroke}%
\end{pgfscope}%
\begin{pgfscope}%
\pgfpathrectangle{\pgfqpoint{0.572918in}{0.553781in}}{\pgfqpoint{5.478282in}{2.095553in}}%
\pgfusepath{clip}%
\pgfsetbuttcap%
\pgfsetroundjoin%
\pgfsetlinewidth{1.505625pt}%
\definecolor{currentstroke}{rgb}{0.313725,0.317647,0.309804}%
\pgfsetstrokecolor{currentstroke}%
\pgfsetstrokeopacity{0.900000}%
\pgfsetdash{}{0pt}%
\pgfpathmoveto{\pgfqpoint{1.114887in}{1.845694in}}%
\pgfpathlineto{\pgfqpoint{1.114887in}{2.117529in}}%
\pgfusepath{stroke}%
\end{pgfscope}%
\begin{pgfscope}%
\pgfpathrectangle{\pgfqpoint{0.572918in}{0.553781in}}{\pgfqpoint{5.478282in}{2.095553in}}%
\pgfusepath{clip}%
\pgfsetbuttcap%
\pgfsetroundjoin%
\pgfsetlinewidth{1.505625pt}%
\definecolor{currentstroke}{rgb}{0.313725,0.317647,0.309804}%
\pgfsetstrokecolor{currentstroke}%
\pgfsetstrokeopacity{0.900000}%
\pgfsetdash{}{0pt}%
\pgfpathmoveto{\pgfqpoint{1.407844in}{1.845622in}}%
\pgfpathlineto{\pgfqpoint{1.407844in}{1.986734in}}%
\pgfusepath{stroke}%
\end{pgfscope}%
\begin{pgfscope}%
\pgfpathrectangle{\pgfqpoint{0.572918in}{0.553781in}}{\pgfqpoint{5.478282in}{2.095553in}}%
\pgfusepath{clip}%
\pgfsetbuttcap%
\pgfsetroundjoin%
\pgfsetlinewidth{1.505625pt}%
\definecolor{currentstroke}{rgb}{0.313725,0.317647,0.309804}%
\pgfsetstrokecolor{currentstroke}%
\pgfsetstrokeopacity{0.900000}%
\pgfsetdash{}{0pt}%
\pgfpathmoveto{\pgfqpoint{1.700800in}{2.004430in}}%
\pgfpathlineto{\pgfqpoint{1.700800in}{2.106414in}}%
\pgfusepath{stroke}%
\end{pgfscope}%
\begin{pgfscope}%
\pgfpathrectangle{\pgfqpoint{0.572918in}{0.553781in}}{\pgfqpoint{5.478282in}{2.095553in}}%
\pgfusepath{clip}%
\pgfsetbuttcap%
\pgfsetroundjoin%
\pgfsetlinewidth{1.505625pt}%
\definecolor{currentstroke}{rgb}{0.313725,0.317647,0.309804}%
\pgfsetstrokecolor{currentstroke}%
\pgfsetstrokeopacity{0.900000}%
\pgfsetdash{}{0pt}%
\pgfpathmoveto{\pgfqpoint{1.993756in}{2.029016in}}%
\pgfpathlineto{\pgfqpoint{1.993756in}{2.124830in}}%
\pgfusepath{stroke}%
\end{pgfscope}%
\begin{pgfscope}%
\pgfpathrectangle{\pgfqpoint{0.572918in}{0.553781in}}{\pgfqpoint{5.478282in}{2.095553in}}%
\pgfusepath{clip}%
\pgfsetbuttcap%
\pgfsetroundjoin%
\pgfsetlinewidth{1.505625pt}%
\definecolor{currentstroke}{rgb}{0.313725,0.317647,0.309804}%
\pgfsetstrokecolor{currentstroke}%
\pgfsetstrokeopacity{0.900000}%
\pgfsetdash{}{0pt}%
\pgfpathmoveto{\pgfqpoint{2.286712in}{2.135954in}}%
\pgfpathlineto{\pgfqpoint{2.286712in}{2.214257in}}%
\pgfusepath{stroke}%
\end{pgfscope}%
\begin{pgfscope}%
\pgfpathrectangle{\pgfqpoint{0.572918in}{0.553781in}}{\pgfqpoint{5.478282in}{2.095553in}}%
\pgfusepath{clip}%
\pgfsetbuttcap%
\pgfsetroundjoin%
\pgfsetlinewidth{1.505625pt}%
\definecolor{currentstroke}{rgb}{0.313725,0.317647,0.309804}%
\pgfsetstrokecolor{currentstroke}%
\pgfsetstrokeopacity{0.900000}%
\pgfsetdash{}{0pt}%
\pgfpathmoveto{\pgfqpoint{2.579669in}{2.052734in}}%
\pgfpathlineto{\pgfqpoint{2.579669in}{2.117863in}}%
\pgfusepath{stroke}%
\end{pgfscope}%
\begin{pgfscope}%
\pgfpathrectangle{\pgfqpoint{0.572918in}{0.553781in}}{\pgfqpoint{5.478282in}{2.095553in}}%
\pgfusepath{clip}%
\pgfsetbuttcap%
\pgfsetroundjoin%
\pgfsetlinewidth{1.505625pt}%
\definecolor{currentstroke}{rgb}{0.313725,0.317647,0.309804}%
\pgfsetstrokecolor{currentstroke}%
\pgfsetstrokeopacity{0.900000}%
\pgfsetdash{}{0pt}%
\pgfpathmoveto{\pgfqpoint{2.872625in}{1.958515in}}%
\pgfpathlineto{\pgfqpoint{2.872625in}{2.018180in}}%
\pgfusepath{stroke}%
\end{pgfscope}%
\begin{pgfscope}%
\pgfpathrectangle{\pgfqpoint{0.572918in}{0.553781in}}{\pgfqpoint{5.478282in}{2.095553in}}%
\pgfusepath{clip}%
\pgfsetbuttcap%
\pgfsetroundjoin%
\pgfsetlinewidth{1.505625pt}%
\definecolor{currentstroke}{rgb}{0.313725,0.317647,0.309804}%
\pgfsetstrokecolor{currentstroke}%
\pgfsetstrokeopacity{0.900000}%
\pgfsetdash{}{0pt}%
\pgfpathmoveto{\pgfqpoint{3.165581in}{1.779312in}}%
\pgfpathlineto{\pgfqpoint{3.165581in}{1.841619in}}%
\pgfusepath{stroke}%
\end{pgfscope}%
\begin{pgfscope}%
\pgfpathrectangle{\pgfqpoint{0.572918in}{0.553781in}}{\pgfqpoint{5.478282in}{2.095553in}}%
\pgfusepath{clip}%
\pgfsetbuttcap%
\pgfsetroundjoin%
\pgfsetlinewidth{1.505625pt}%
\definecolor{currentstroke}{rgb}{0.313725,0.317647,0.309804}%
\pgfsetstrokecolor{currentstroke}%
\pgfsetstrokeopacity{0.900000}%
\pgfsetdash{}{0pt}%
\pgfpathmoveto{\pgfqpoint{3.458537in}{1.613942in}}%
\pgfpathlineto{\pgfqpoint{3.458537in}{1.674701in}}%
\pgfusepath{stroke}%
\end{pgfscope}%
\begin{pgfscope}%
\pgfpathrectangle{\pgfqpoint{0.572918in}{0.553781in}}{\pgfqpoint{5.478282in}{2.095553in}}%
\pgfusepath{clip}%
\pgfsetbuttcap%
\pgfsetroundjoin%
\pgfsetlinewidth{1.505625pt}%
\definecolor{currentstroke}{rgb}{0.313725,0.317647,0.309804}%
\pgfsetstrokecolor{currentstroke}%
\pgfsetstrokeopacity{0.900000}%
\pgfsetdash{}{0pt}%
\pgfpathmoveto{\pgfqpoint{3.751494in}{1.471080in}}%
\pgfpathlineto{\pgfqpoint{3.751494in}{1.530224in}}%
\pgfusepath{stroke}%
\end{pgfscope}%
\begin{pgfscope}%
\pgfpathrectangle{\pgfqpoint{0.572918in}{0.553781in}}{\pgfqpoint{5.478282in}{2.095553in}}%
\pgfusepath{clip}%
\pgfsetbuttcap%
\pgfsetroundjoin%
\pgfsetlinewidth{1.505625pt}%
\definecolor{currentstroke}{rgb}{0.313725,0.317647,0.309804}%
\pgfsetstrokecolor{currentstroke}%
\pgfsetstrokeopacity{0.900000}%
\pgfsetdash{}{0pt}%
\pgfpathmoveto{\pgfqpoint{4.044450in}{1.368693in}}%
\pgfpathlineto{\pgfqpoint{4.044450in}{1.432512in}}%
\pgfusepath{stroke}%
\end{pgfscope}%
\begin{pgfscope}%
\pgfpathrectangle{\pgfqpoint{0.572918in}{0.553781in}}{\pgfqpoint{5.478282in}{2.095553in}}%
\pgfusepath{clip}%
\pgfsetbuttcap%
\pgfsetroundjoin%
\pgfsetlinewidth{1.505625pt}%
\definecolor{currentstroke}{rgb}{0.313725,0.317647,0.309804}%
\pgfsetstrokecolor{currentstroke}%
\pgfsetstrokeopacity{0.900000}%
\pgfsetdash{}{0pt}%
\pgfpathmoveto{\pgfqpoint{4.337406in}{1.304155in}}%
\pgfpathlineto{\pgfqpoint{4.337406in}{1.378915in}}%
\pgfusepath{stroke}%
\end{pgfscope}%
\begin{pgfscope}%
\pgfpathrectangle{\pgfqpoint{0.572918in}{0.553781in}}{\pgfqpoint{5.478282in}{2.095553in}}%
\pgfusepath{clip}%
\pgfsetbuttcap%
\pgfsetroundjoin%
\pgfsetlinewidth{1.505625pt}%
\definecolor{currentstroke}{rgb}{0.313725,0.317647,0.309804}%
\pgfsetstrokecolor{currentstroke}%
\pgfsetstrokeopacity{0.900000}%
\pgfsetdash{}{0pt}%
\pgfpathmoveto{\pgfqpoint{4.630362in}{1.227603in}}%
\pgfpathlineto{\pgfqpoint{4.630362in}{1.316322in}}%
\pgfusepath{stroke}%
\end{pgfscope}%
\begin{pgfscope}%
\pgfpathrectangle{\pgfqpoint{0.572918in}{0.553781in}}{\pgfqpoint{5.478282in}{2.095553in}}%
\pgfusepath{clip}%
\pgfsetbuttcap%
\pgfsetroundjoin%
\pgfsetlinewidth{1.505625pt}%
\definecolor{currentstroke}{rgb}{0.313725,0.317647,0.309804}%
\pgfsetstrokecolor{currentstroke}%
\pgfsetstrokeopacity{0.900000}%
\pgfsetdash{}{0pt}%
\pgfpathmoveto{\pgfqpoint{4.923318in}{1.125303in}}%
\pgfpathlineto{\pgfqpoint{4.923318in}{1.213084in}}%
\pgfusepath{stroke}%
\end{pgfscope}%
\begin{pgfscope}%
\pgfpathrectangle{\pgfqpoint{0.572918in}{0.553781in}}{\pgfqpoint{5.478282in}{2.095553in}}%
\pgfusepath{clip}%
\pgfsetbuttcap%
\pgfsetroundjoin%
\pgfsetlinewidth{1.505625pt}%
\definecolor{currentstroke}{rgb}{0.313725,0.317647,0.309804}%
\pgfsetstrokecolor{currentstroke}%
\pgfsetstrokeopacity{0.900000}%
\pgfsetdash{}{0pt}%
\pgfpathmoveto{\pgfqpoint{5.216275in}{1.126958in}}%
\pgfpathlineto{\pgfqpoint{5.216275in}{1.246300in}}%
\pgfusepath{stroke}%
\end{pgfscope}%
\begin{pgfscope}%
\pgfpathrectangle{\pgfqpoint{0.572918in}{0.553781in}}{\pgfqpoint{5.478282in}{2.095553in}}%
\pgfusepath{clip}%
\pgfsetbuttcap%
\pgfsetroundjoin%
\pgfsetlinewidth{1.505625pt}%
\definecolor{currentstroke}{rgb}{0.313725,0.317647,0.309804}%
\pgfsetstrokecolor{currentstroke}%
\pgfsetstrokeopacity{0.900000}%
\pgfsetdash{}{0pt}%
\pgfpathmoveto{\pgfqpoint{5.509231in}{1.075338in}}%
\pgfpathlineto{\pgfqpoint{5.509231in}{1.223011in}}%
\pgfusepath{stroke}%
\end{pgfscope}%
\begin{pgfscope}%
\pgfpathrectangle{\pgfqpoint{0.572918in}{0.553781in}}{\pgfqpoint{5.478282in}{2.095553in}}%
\pgfusepath{clip}%
\pgfsetbuttcap%
\pgfsetroundjoin%
\pgfsetlinewidth{1.505625pt}%
\definecolor{currentstroke}{rgb}{0.313725,0.317647,0.309804}%
\pgfsetstrokecolor{currentstroke}%
\pgfsetstrokeopacity{0.900000}%
\pgfsetdash{}{0pt}%
\pgfpathmoveto{\pgfqpoint{5.802187in}{1.065736in}}%
\pgfpathlineto{\pgfqpoint{5.802187in}{1.281461in}}%
\pgfusepath{stroke}%
\end{pgfscope}%
\begin{pgfscope}%
\pgfpathrectangle{\pgfqpoint{0.572918in}{0.553781in}}{\pgfqpoint{5.478282in}{2.095553in}}%
\pgfusepath{clip}%
\pgfsetbuttcap%
\pgfsetroundjoin%
\pgfsetlinewidth{1.505625pt}%
\definecolor{currentstroke}{rgb}{0.949020,0.372549,0.360784}%
\pgfsetstrokecolor{currentstroke}%
\pgfsetstrokeopacity{0.900000}%
\pgfsetdash{}{0pt}%
\pgfpathmoveto{\pgfqpoint{0.821931in}{1.635578in}}%
\pgfpathlineto{\pgfqpoint{0.821931in}{2.237623in}}%
\pgfusepath{stroke}%
\end{pgfscope}%
\begin{pgfscope}%
\pgfpathrectangle{\pgfqpoint{0.572918in}{0.553781in}}{\pgfqpoint{5.478282in}{2.095553in}}%
\pgfusepath{clip}%
\pgfsetbuttcap%
\pgfsetroundjoin%
\pgfsetlinewidth{1.505625pt}%
\definecolor{currentstroke}{rgb}{0.949020,0.372549,0.360784}%
\pgfsetstrokecolor{currentstroke}%
\pgfsetstrokeopacity{0.900000}%
\pgfsetdash{}{0pt}%
\pgfpathmoveto{\pgfqpoint{1.114887in}{1.732504in}}%
\pgfpathlineto{\pgfqpoint{1.114887in}{1.965823in}}%
\pgfusepath{stroke}%
\end{pgfscope}%
\begin{pgfscope}%
\pgfpathrectangle{\pgfqpoint{0.572918in}{0.553781in}}{\pgfqpoint{5.478282in}{2.095553in}}%
\pgfusepath{clip}%
\pgfsetbuttcap%
\pgfsetroundjoin%
\pgfsetlinewidth{1.505625pt}%
\definecolor{currentstroke}{rgb}{0.949020,0.372549,0.360784}%
\pgfsetstrokecolor{currentstroke}%
\pgfsetstrokeopacity{0.900000}%
\pgfsetdash{}{0pt}%
\pgfpathmoveto{\pgfqpoint{1.407844in}{1.642474in}}%
\pgfpathlineto{\pgfqpoint{1.407844in}{1.775343in}}%
\pgfusepath{stroke}%
\end{pgfscope}%
\begin{pgfscope}%
\pgfpathrectangle{\pgfqpoint{0.572918in}{0.553781in}}{\pgfqpoint{5.478282in}{2.095553in}}%
\pgfusepath{clip}%
\pgfsetbuttcap%
\pgfsetroundjoin%
\pgfsetlinewidth{1.505625pt}%
\definecolor{currentstroke}{rgb}{0.949020,0.372549,0.360784}%
\pgfsetstrokecolor{currentstroke}%
\pgfsetstrokeopacity{0.900000}%
\pgfsetdash{}{0pt}%
\pgfpathmoveto{\pgfqpoint{1.700800in}{1.665438in}}%
\pgfpathlineto{\pgfqpoint{1.700800in}{1.758836in}}%
\pgfusepath{stroke}%
\end{pgfscope}%
\begin{pgfscope}%
\pgfpathrectangle{\pgfqpoint{0.572918in}{0.553781in}}{\pgfqpoint{5.478282in}{2.095553in}}%
\pgfusepath{clip}%
\pgfsetbuttcap%
\pgfsetroundjoin%
\pgfsetlinewidth{1.505625pt}%
\definecolor{currentstroke}{rgb}{0.949020,0.372549,0.360784}%
\pgfsetstrokecolor{currentstroke}%
\pgfsetstrokeopacity{0.900000}%
\pgfsetdash{}{0pt}%
\pgfpathmoveto{\pgfqpoint{1.993756in}{1.745021in}}%
\pgfpathlineto{\pgfqpoint{1.993756in}{1.820998in}}%
\pgfusepath{stroke}%
\end{pgfscope}%
\begin{pgfscope}%
\pgfpathrectangle{\pgfqpoint{0.572918in}{0.553781in}}{\pgfqpoint{5.478282in}{2.095553in}}%
\pgfusepath{clip}%
\pgfsetbuttcap%
\pgfsetroundjoin%
\pgfsetlinewidth{1.505625pt}%
\definecolor{currentstroke}{rgb}{0.949020,0.372549,0.360784}%
\pgfsetstrokecolor{currentstroke}%
\pgfsetstrokeopacity{0.900000}%
\pgfsetdash{}{0pt}%
\pgfpathmoveto{\pgfqpoint{2.286712in}{1.721242in}}%
\pgfpathlineto{\pgfqpoint{2.286712in}{1.776820in}}%
\pgfusepath{stroke}%
\end{pgfscope}%
\begin{pgfscope}%
\pgfpathrectangle{\pgfqpoint{0.572918in}{0.553781in}}{\pgfqpoint{5.478282in}{2.095553in}}%
\pgfusepath{clip}%
\pgfsetbuttcap%
\pgfsetroundjoin%
\pgfsetlinewidth{1.505625pt}%
\definecolor{currentstroke}{rgb}{0.949020,0.372549,0.360784}%
\pgfsetstrokecolor{currentstroke}%
\pgfsetstrokeopacity{0.900000}%
\pgfsetdash{}{0pt}%
\pgfpathmoveto{\pgfqpoint{2.579669in}{1.624399in}}%
\pgfpathlineto{\pgfqpoint{2.579669in}{1.679312in}}%
\pgfusepath{stroke}%
\end{pgfscope}%
\begin{pgfscope}%
\pgfpathrectangle{\pgfqpoint{0.572918in}{0.553781in}}{\pgfqpoint{5.478282in}{2.095553in}}%
\pgfusepath{clip}%
\pgfsetbuttcap%
\pgfsetroundjoin%
\pgfsetlinewidth{1.505625pt}%
\definecolor{currentstroke}{rgb}{0.949020,0.372549,0.360784}%
\pgfsetstrokecolor{currentstroke}%
\pgfsetstrokeopacity{0.900000}%
\pgfsetdash{}{0pt}%
\pgfpathmoveto{\pgfqpoint{2.872625in}{1.553684in}}%
\pgfpathlineto{\pgfqpoint{2.872625in}{1.603823in}}%
\pgfusepath{stroke}%
\end{pgfscope}%
\begin{pgfscope}%
\pgfpathrectangle{\pgfqpoint{0.572918in}{0.553781in}}{\pgfqpoint{5.478282in}{2.095553in}}%
\pgfusepath{clip}%
\pgfsetbuttcap%
\pgfsetroundjoin%
\pgfsetlinewidth{1.505625pt}%
\definecolor{currentstroke}{rgb}{0.949020,0.372549,0.360784}%
\pgfsetstrokecolor{currentstroke}%
\pgfsetstrokeopacity{0.900000}%
\pgfsetdash{}{0pt}%
\pgfpathmoveto{\pgfqpoint{3.165581in}{1.398540in}}%
\pgfpathlineto{\pgfqpoint{3.165581in}{1.450467in}}%
\pgfusepath{stroke}%
\end{pgfscope}%
\begin{pgfscope}%
\pgfpathrectangle{\pgfqpoint{0.572918in}{0.553781in}}{\pgfqpoint{5.478282in}{2.095553in}}%
\pgfusepath{clip}%
\pgfsetbuttcap%
\pgfsetroundjoin%
\pgfsetlinewidth{1.505625pt}%
\definecolor{currentstroke}{rgb}{0.949020,0.372549,0.360784}%
\pgfsetstrokecolor{currentstroke}%
\pgfsetstrokeopacity{0.900000}%
\pgfsetdash{}{0pt}%
\pgfpathmoveto{\pgfqpoint{3.458537in}{1.271312in}}%
\pgfpathlineto{\pgfqpoint{3.458537in}{1.319848in}}%
\pgfusepath{stroke}%
\end{pgfscope}%
\begin{pgfscope}%
\pgfpathrectangle{\pgfqpoint{0.572918in}{0.553781in}}{\pgfqpoint{5.478282in}{2.095553in}}%
\pgfusepath{clip}%
\pgfsetbuttcap%
\pgfsetroundjoin%
\pgfsetlinewidth{1.505625pt}%
\definecolor{currentstroke}{rgb}{0.949020,0.372549,0.360784}%
\pgfsetstrokecolor{currentstroke}%
\pgfsetstrokeopacity{0.900000}%
\pgfsetdash{}{0pt}%
\pgfpathmoveto{\pgfqpoint{3.751494in}{1.163235in}}%
\pgfpathlineto{\pgfqpoint{3.751494in}{1.209137in}}%
\pgfusepath{stroke}%
\end{pgfscope}%
\begin{pgfscope}%
\pgfpathrectangle{\pgfqpoint{0.572918in}{0.553781in}}{\pgfqpoint{5.478282in}{2.095553in}}%
\pgfusepath{clip}%
\pgfsetbuttcap%
\pgfsetroundjoin%
\pgfsetlinewidth{1.505625pt}%
\definecolor{currentstroke}{rgb}{0.949020,0.372549,0.360784}%
\pgfsetstrokecolor{currentstroke}%
\pgfsetstrokeopacity{0.900000}%
\pgfsetdash{}{0pt}%
\pgfpathmoveto{\pgfqpoint{4.044450in}{1.029460in}}%
\pgfpathlineto{\pgfqpoint{4.044450in}{1.084568in}}%
\pgfusepath{stroke}%
\end{pgfscope}%
\begin{pgfscope}%
\pgfpathrectangle{\pgfqpoint{0.572918in}{0.553781in}}{\pgfqpoint{5.478282in}{2.095553in}}%
\pgfusepath{clip}%
\pgfsetbuttcap%
\pgfsetroundjoin%
\pgfsetlinewidth{1.505625pt}%
\definecolor{currentstroke}{rgb}{0.949020,0.372549,0.360784}%
\pgfsetstrokecolor{currentstroke}%
\pgfsetstrokeopacity{0.900000}%
\pgfsetdash{}{0pt}%
\pgfpathmoveto{\pgfqpoint{4.337406in}{0.905731in}}%
\pgfpathlineto{\pgfqpoint{4.337406in}{0.962388in}}%
\pgfusepath{stroke}%
\end{pgfscope}%
\begin{pgfscope}%
\pgfpathrectangle{\pgfqpoint{0.572918in}{0.553781in}}{\pgfqpoint{5.478282in}{2.095553in}}%
\pgfusepath{clip}%
\pgfsetbuttcap%
\pgfsetroundjoin%
\pgfsetlinewidth{1.505625pt}%
\definecolor{currentstroke}{rgb}{0.949020,0.372549,0.360784}%
\pgfsetstrokecolor{currentstroke}%
\pgfsetstrokeopacity{0.900000}%
\pgfsetdash{}{0pt}%
\pgfpathmoveto{\pgfqpoint{4.630362in}{0.767655in}}%
\pgfpathlineto{\pgfqpoint{4.630362in}{0.832815in}}%
\pgfusepath{stroke}%
\end{pgfscope}%
\begin{pgfscope}%
\pgfpathrectangle{\pgfqpoint{0.572918in}{0.553781in}}{\pgfqpoint{5.478282in}{2.095553in}}%
\pgfusepath{clip}%
\pgfsetbuttcap%
\pgfsetroundjoin%
\pgfsetlinewidth{1.505625pt}%
\definecolor{currentstroke}{rgb}{0.949020,0.372549,0.360784}%
\pgfsetstrokecolor{currentstroke}%
\pgfsetstrokeopacity{0.900000}%
\pgfsetdash{}{0pt}%
\pgfpathmoveto{\pgfqpoint{4.923318in}{0.708454in}}%
\pgfpathlineto{\pgfqpoint{4.923318in}{0.782853in}}%
\pgfusepath{stroke}%
\end{pgfscope}%
\begin{pgfscope}%
\pgfpathrectangle{\pgfqpoint{0.572918in}{0.553781in}}{\pgfqpoint{5.478282in}{2.095553in}}%
\pgfusepath{clip}%
\pgfsetbuttcap%
\pgfsetroundjoin%
\pgfsetlinewidth{1.505625pt}%
\definecolor{currentstroke}{rgb}{0.949020,0.372549,0.360784}%
\pgfsetstrokecolor{currentstroke}%
\pgfsetstrokeopacity{0.900000}%
\pgfsetdash{}{0pt}%
\pgfpathmoveto{\pgfqpoint{5.216275in}{0.673625in}}%
\pgfpathlineto{\pgfqpoint{5.216275in}{0.794593in}}%
\pgfusepath{stroke}%
\end{pgfscope}%
\begin{pgfscope}%
\pgfpathrectangle{\pgfqpoint{0.572918in}{0.553781in}}{\pgfqpoint{5.478282in}{2.095553in}}%
\pgfusepath{clip}%
\pgfsetbuttcap%
\pgfsetroundjoin%
\pgfsetlinewidth{1.505625pt}%
\definecolor{currentstroke}{rgb}{0.949020,0.372549,0.360784}%
\pgfsetstrokecolor{currentstroke}%
\pgfsetstrokeopacity{0.900000}%
\pgfsetdash{}{0pt}%
\pgfpathmoveto{\pgfqpoint{5.509231in}{0.649033in}}%
\pgfpathlineto{\pgfqpoint{5.509231in}{0.776777in}}%
\pgfusepath{stroke}%
\end{pgfscope}%
\begin{pgfscope}%
\pgfpathrectangle{\pgfqpoint{0.572918in}{0.553781in}}{\pgfqpoint{5.478282in}{2.095553in}}%
\pgfusepath{clip}%
\pgfsetbuttcap%
\pgfsetroundjoin%
\pgfsetlinewidth{1.505625pt}%
\definecolor{currentstroke}{rgb}{0.949020,0.372549,0.360784}%
\pgfsetstrokecolor{currentstroke}%
\pgfsetstrokeopacity{0.900000}%
\pgfsetdash{}{0pt}%
\pgfpathmoveto{\pgfqpoint{5.802187in}{0.752238in}}%
\pgfpathlineto{\pgfqpoint{5.802187in}{0.922188in}}%
\pgfusepath{stroke}%
\end{pgfscope}%
\begin{pgfscope}%
\pgfpathrectangle{\pgfqpoint{0.572918in}{0.553781in}}{\pgfqpoint{5.478282in}{2.095553in}}%
\pgfusepath{clip}%
\pgfsetbuttcap%
\pgfsetroundjoin%
\definecolor{currentfill}{rgb}{0.313725,0.317647,0.309804}%
\pgfsetfillcolor{currentfill}%
\pgfsetfillopacity{0.900000}%
\pgfsetlinewidth{1.003750pt}%
\definecolor{currentstroke}{rgb}{0.313725,0.317647,0.309804}%
\pgfsetstrokecolor{currentstroke}%
\pgfsetstrokeopacity{0.900000}%
\pgfsetdash{}{0pt}%
\pgfsys@defobject{currentmarker}{\pgfqpoint{-0.013889in}{-0.000000in}}{\pgfqpoint{0.013889in}{0.000000in}}{%
\pgfpathmoveto{\pgfqpoint{0.013889in}{-0.000000in}}%
\pgfpathlineto{\pgfqpoint{-0.013889in}{0.000000in}}%
\pgfusepath{stroke,fill}%
}%
\begin{pgfscope}%
\pgfsys@transformshift{0.821931in}{1.909657in}%
\pgfsys@useobject{currentmarker}{}%
\end{pgfscope}%
\begin{pgfscope}%
\pgfsys@transformshift{1.114887in}{1.845694in}%
\pgfsys@useobject{currentmarker}{}%
\end{pgfscope}%
\begin{pgfscope}%
\pgfsys@transformshift{1.407844in}{1.845622in}%
\pgfsys@useobject{currentmarker}{}%
\end{pgfscope}%
\begin{pgfscope}%
\pgfsys@transformshift{1.700800in}{2.004430in}%
\pgfsys@useobject{currentmarker}{}%
\end{pgfscope}%
\begin{pgfscope}%
\pgfsys@transformshift{1.993756in}{2.029016in}%
\pgfsys@useobject{currentmarker}{}%
\end{pgfscope}%
\begin{pgfscope}%
\pgfsys@transformshift{2.286712in}{2.135954in}%
\pgfsys@useobject{currentmarker}{}%
\end{pgfscope}%
\begin{pgfscope}%
\pgfsys@transformshift{2.579669in}{2.052734in}%
\pgfsys@useobject{currentmarker}{}%
\end{pgfscope}%
\begin{pgfscope}%
\pgfsys@transformshift{2.872625in}{1.958515in}%
\pgfsys@useobject{currentmarker}{}%
\end{pgfscope}%
\begin{pgfscope}%
\pgfsys@transformshift{3.165581in}{1.779312in}%
\pgfsys@useobject{currentmarker}{}%
\end{pgfscope}%
\begin{pgfscope}%
\pgfsys@transformshift{3.458537in}{1.613942in}%
\pgfsys@useobject{currentmarker}{}%
\end{pgfscope}%
\begin{pgfscope}%
\pgfsys@transformshift{3.751494in}{1.471080in}%
\pgfsys@useobject{currentmarker}{}%
\end{pgfscope}%
\begin{pgfscope}%
\pgfsys@transformshift{4.044450in}{1.368693in}%
\pgfsys@useobject{currentmarker}{}%
\end{pgfscope}%
\begin{pgfscope}%
\pgfsys@transformshift{4.337406in}{1.304155in}%
\pgfsys@useobject{currentmarker}{}%
\end{pgfscope}%
\begin{pgfscope}%
\pgfsys@transformshift{4.630362in}{1.227603in}%
\pgfsys@useobject{currentmarker}{}%
\end{pgfscope}%
\begin{pgfscope}%
\pgfsys@transformshift{4.923318in}{1.125303in}%
\pgfsys@useobject{currentmarker}{}%
\end{pgfscope}%
\begin{pgfscope}%
\pgfsys@transformshift{5.216275in}{1.126958in}%
\pgfsys@useobject{currentmarker}{}%
\end{pgfscope}%
\begin{pgfscope}%
\pgfsys@transformshift{5.509231in}{1.075338in}%
\pgfsys@useobject{currentmarker}{}%
\end{pgfscope}%
\begin{pgfscope}%
\pgfsys@transformshift{5.802187in}{1.065736in}%
\pgfsys@useobject{currentmarker}{}%
\end{pgfscope}%
\end{pgfscope}%
\begin{pgfscope}%
\pgfpathrectangle{\pgfqpoint{0.572918in}{0.553781in}}{\pgfqpoint{5.478282in}{2.095553in}}%
\pgfusepath{clip}%
\pgfsetbuttcap%
\pgfsetroundjoin%
\definecolor{currentfill}{rgb}{0.313725,0.317647,0.309804}%
\pgfsetfillcolor{currentfill}%
\pgfsetfillopacity{0.900000}%
\pgfsetlinewidth{1.003750pt}%
\definecolor{currentstroke}{rgb}{0.313725,0.317647,0.309804}%
\pgfsetstrokecolor{currentstroke}%
\pgfsetstrokeopacity{0.900000}%
\pgfsetdash{}{0pt}%
\pgfsys@defobject{currentmarker}{\pgfqpoint{-0.013889in}{-0.000000in}}{\pgfqpoint{0.013889in}{0.000000in}}{%
\pgfpathmoveto{\pgfqpoint{0.013889in}{-0.000000in}}%
\pgfpathlineto{\pgfqpoint{-0.013889in}{0.000000in}}%
\pgfusepath{stroke,fill}%
}%
\begin{pgfscope}%
\pgfsys@transformshift{0.821931in}{2.554081in}%
\pgfsys@useobject{currentmarker}{}%
\end{pgfscope}%
\begin{pgfscope}%
\pgfsys@transformshift{1.114887in}{2.117529in}%
\pgfsys@useobject{currentmarker}{}%
\end{pgfscope}%
\begin{pgfscope}%
\pgfsys@transformshift{1.407844in}{1.986734in}%
\pgfsys@useobject{currentmarker}{}%
\end{pgfscope}%
\begin{pgfscope}%
\pgfsys@transformshift{1.700800in}{2.106414in}%
\pgfsys@useobject{currentmarker}{}%
\end{pgfscope}%
\begin{pgfscope}%
\pgfsys@transformshift{1.993756in}{2.124830in}%
\pgfsys@useobject{currentmarker}{}%
\end{pgfscope}%
\begin{pgfscope}%
\pgfsys@transformshift{2.286712in}{2.214257in}%
\pgfsys@useobject{currentmarker}{}%
\end{pgfscope}%
\begin{pgfscope}%
\pgfsys@transformshift{2.579669in}{2.117863in}%
\pgfsys@useobject{currentmarker}{}%
\end{pgfscope}%
\begin{pgfscope}%
\pgfsys@transformshift{2.872625in}{2.018180in}%
\pgfsys@useobject{currentmarker}{}%
\end{pgfscope}%
\begin{pgfscope}%
\pgfsys@transformshift{3.165581in}{1.841619in}%
\pgfsys@useobject{currentmarker}{}%
\end{pgfscope}%
\begin{pgfscope}%
\pgfsys@transformshift{3.458537in}{1.674701in}%
\pgfsys@useobject{currentmarker}{}%
\end{pgfscope}%
\begin{pgfscope}%
\pgfsys@transformshift{3.751494in}{1.530224in}%
\pgfsys@useobject{currentmarker}{}%
\end{pgfscope}%
\begin{pgfscope}%
\pgfsys@transformshift{4.044450in}{1.432512in}%
\pgfsys@useobject{currentmarker}{}%
\end{pgfscope}%
\begin{pgfscope}%
\pgfsys@transformshift{4.337406in}{1.378915in}%
\pgfsys@useobject{currentmarker}{}%
\end{pgfscope}%
\begin{pgfscope}%
\pgfsys@transformshift{4.630362in}{1.316322in}%
\pgfsys@useobject{currentmarker}{}%
\end{pgfscope}%
\begin{pgfscope}%
\pgfsys@transformshift{4.923318in}{1.213084in}%
\pgfsys@useobject{currentmarker}{}%
\end{pgfscope}%
\begin{pgfscope}%
\pgfsys@transformshift{5.216275in}{1.246300in}%
\pgfsys@useobject{currentmarker}{}%
\end{pgfscope}%
\begin{pgfscope}%
\pgfsys@transformshift{5.509231in}{1.223011in}%
\pgfsys@useobject{currentmarker}{}%
\end{pgfscope}%
\begin{pgfscope}%
\pgfsys@transformshift{5.802187in}{1.281461in}%
\pgfsys@useobject{currentmarker}{}%
\end{pgfscope}%
\end{pgfscope}%
\begin{pgfscope}%
\pgfpathrectangle{\pgfqpoint{0.572918in}{0.553781in}}{\pgfqpoint{5.478282in}{2.095553in}}%
\pgfusepath{clip}%
\pgfsetbuttcap%
\pgfsetroundjoin%
\definecolor{currentfill}{rgb}{0.949020,0.372549,0.360784}%
\pgfsetfillcolor{currentfill}%
\pgfsetfillopacity{0.900000}%
\pgfsetlinewidth{1.003750pt}%
\definecolor{currentstroke}{rgb}{0.949020,0.372549,0.360784}%
\pgfsetstrokecolor{currentstroke}%
\pgfsetstrokeopacity{0.900000}%
\pgfsetdash{}{0pt}%
\pgfsys@defobject{currentmarker}{\pgfqpoint{-0.013889in}{-0.000000in}}{\pgfqpoint{0.013889in}{0.000000in}}{%
\pgfpathmoveto{\pgfqpoint{0.013889in}{-0.000000in}}%
\pgfpathlineto{\pgfqpoint{-0.013889in}{0.000000in}}%
\pgfusepath{stroke,fill}%
}%
\begin{pgfscope}%
\pgfsys@transformshift{0.821931in}{1.635578in}%
\pgfsys@useobject{currentmarker}{}%
\end{pgfscope}%
\begin{pgfscope}%
\pgfsys@transformshift{1.114887in}{1.732504in}%
\pgfsys@useobject{currentmarker}{}%
\end{pgfscope}%
\begin{pgfscope}%
\pgfsys@transformshift{1.407844in}{1.642474in}%
\pgfsys@useobject{currentmarker}{}%
\end{pgfscope}%
\begin{pgfscope}%
\pgfsys@transformshift{1.700800in}{1.665438in}%
\pgfsys@useobject{currentmarker}{}%
\end{pgfscope}%
\begin{pgfscope}%
\pgfsys@transformshift{1.993756in}{1.745021in}%
\pgfsys@useobject{currentmarker}{}%
\end{pgfscope}%
\begin{pgfscope}%
\pgfsys@transformshift{2.286712in}{1.721242in}%
\pgfsys@useobject{currentmarker}{}%
\end{pgfscope}%
\begin{pgfscope}%
\pgfsys@transformshift{2.579669in}{1.624399in}%
\pgfsys@useobject{currentmarker}{}%
\end{pgfscope}%
\begin{pgfscope}%
\pgfsys@transformshift{2.872625in}{1.553684in}%
\pgfsys@useobject{currentmarker}{}%
\end{pgfscope}%
\begin{pgfscope}%
\pgfsys@transformshift{3.165581in}{1.398540in}%
\pgfsys@useobject{currentmarker}{}%
\end{pgfscope}%
\begin{pgfscope}%
\pgfsys@transformshift{3.458537in}{1.271312in}%
\pgfsys@useobject{currentmarker}{}%
\end{pgfscope}%
\begin{pgfscope}%
\pgfsys@transformshift{3.751494in}{1.163235in}%
\pgfsys@useobject{currentmarker}{}%
\end{pgfscope}%
\begin{pgfscope}%
\pgfsys@transformshift{4.044450in}{1.029460in}%
\pgfsys@useobject{currentmarker}{}%
\end{pgfscope}%
\begin{pgfscope}%
\pgfsys@transformshift{4.337406in}{0.905731in}%
\pgfsys@useobject{currentmarker}{}%
\end{pgfscope}%
\begin{pgfscope}%
\pgfsys@transformshift{4.630362in}{0.767655in}%
\pgfsys@useobject{currentmarker}{}%
\end{pgfscope}%
\begin{pgfscope}%
\pgfsys@transformshift{4.923318in}{0.708454in}%
\pgfsys@useobject{currentmarker}{}%
\end{pgfscope}%
\begin{pgfscope}%
\pgfsys@transformshift{5.216275in}{0.673625in}%
\pgfsys@useobject{currentmarker}{}%
\end{pgfscope}%
\begin{pgfscope}%
\pgfsys@transformshift{5.509231in}{0.649033in}%
\pgfsys@useobject{currentmarker}{}%
\end{pgfscope}%
\begin{pgfscope}%
\pgfsys@transformshift{5.802187in}{0.752238in}%
\pgfsys@useobject{currentmarker}{}%
\end{pgfscope}%
\end{pgfscope}%
\begin{pgfscope}%
\pgfpathrectangle{\pgfqpoint{0.572918in}{0.553781in}}{\pgfqpoint{5.478282in}{2.095553in}}%
\pgfusepath{clip}%
\pgfsetbuttcap%
\pgfsetroundjoin%
\definecolor{currentfill}{rgb}{0.949020,0.372549,0.360784}%
\pgfsetfillcolor{currentfill}%
\pgfsetfillopacity{0.900000}%
\pgfsetlinewidth{1.003750pt}%
\definecolor{currentstroke}{rgb}{0.949020,0.372549,0.360784}%
\pgfsetstrokecolor{currentstroke}%
\pgfsetstrokeopacity{0.900000}%
\pgfsetdash{}{0pt}%
\pgfsys@defobject{currentmarker}{\pgfqpoint{-0.013889in}{-0.000000in}}{\pgfqpoint{0.013889in}{0.000000in}}{%
\pgfpathmoveto{\pgfqpoint{0.013889in}{-0.000000in}}%
\pgfpathlineto{\pgfqpoint{-0.013889in}{0.000000in}}%
\pgfusepath{stroke,fill}%
}%
\begin{pgfscope}%
\pgfsys@transformshift{0.821931in}{2.237623in}%
\pgfsys@useobject{currentmarker}{}%
\end{pgfscope}%
\begin{pgfscope}%
\pgfsys@transformshift{1.114887in}{1.965823in}%
\pgfsys@useobject{currentmarker}{}%
\end{pgfscope}%
\begin{pgfscope}%
\pgfsys@transformshift{1.407844in}{1.775343in}%
\pgfsys@useobject{currentmarker}{}%
\end{pgfscope}%
\begin{pgfscope}%
\pgfsys@transformshift{1.700800in}{1.758836in}%
\pgfsys@useobject{currentmarker}{}%
\end{pgfscope}%
\begin{pgfscope}%
\pgfsys@transformshift{1.993756in}{1.820998in}%
\pgfsys@useobject{currentmarker}{}%
\end{pgfscope}%
\begin{pgfscope}%
\pgfsys@transformshift{2.286712in}{1.776820in}%
\pgfsys@useobject{currentmarker}{}%
\end{pgfscope}%
\begin{pgfscope}%
\pgfsys@transformshift{2.579669in}{1.679312in}%
\pgfsys@useobject{currentmarker}{}%
\end{pgfscope}%
\begin{pgfscope}%
\pgfsys@transformshift{2.872625in}{1.603823in}%
\pgfsys@useobject{currentmarker}{}%
\end{pgfscope}%
\begin{pgfscope}%
\pgfsys@transformshift{3.165581in}{1.450467in}%
\pgfsys@useobject{currentmarker}{}%
\end{pgfscope}%
\begin{pgfscope}%
\pgfsys@transformshift{3.458537in}{1.319848in}%
\pgfsys@useobject{currentmarker}{}%
\end{pgfscope}%
\begin{pgfscope}%
\pgfsys@transformshift{3.751494in}{1.209137in}%
\pgfsys@useobject{currentmarker}{}%
\end{pgfscope}%
\begin{pgfscope}%
\pgfsys@transformshift{4.044450in}{1.084568in}%
\pgfsys@useobject{currentmarker}{}%
\end{pgfscope}%
\begin{pgfscope}%
\pgfsys@transformshift{4.337406in}{0.962388in}%
\pgfsys@useobject{currentmarker}{}%
\end{pgfscope}%
\begin{pgfscope}%
\pgfsys@transformshift{4.630362in}{0.832815in}%
\pgfsys@useobject{currentmarker}{}%
\end{pgfscope}%
\begin{pgfscope}%
\pgfsys@transformshift{4.923318in}{0.782853in}%
\pgfsys@useobject{currentmarker}{}%
\end{pgfscope}%
\begin{pgfscope}%
\pgfsys@transformshift{5.216275in}{0.794593in}%
\pgfsys@useobject{currentmarker}{}%
\end{pgfscope}%
\begin{pgfscope}%
\pgfsys@transformshift{5.509231in}{0.776777in}%
\pgfsys@useobject{currentmarker}{}%
\end{pgfscope}%
\begin{pgfscope}%
\pgfsys@transformshift{5.802187in}{0.922188in}%
\pgfsys@useobject{currentmarker}{}%
\end{pgfscope}%
\end{pgfscope}%
\begin{pgfscope}%
\pgfpathrectangle{\pgfqpoint{0.572918in}{0.553781in}}{\pgfqpoint{5.478282in}{2.095553in}}%
\pgfusepath{clip}%
\pgfsetrectcap%
\pgfsetroundjoin%
\pgfsetlinewidth{1.505625pt}%
\definecolor{currentstroke}{rgb}{0.313725,0.317647,0.309804}%
\pgfsetstrokecolor{currentstroke}%
\pgfsetstrokeopacity{0.900000}%
\pgfsetdash{}{0pt}%
\pgfpathmoveto{\pgfqpoint{0.821931in}{2.226735in}}%
\pgfpathlineto{\pgfqpoint{1.114887in}{1.990098in}}%
\pgfpathlineto{\pgfqpoint{1.407844in}{1.913769in}}%
\pgfpathlineto{\pgfqpoint{1.700800in}{2.053321in}}%
\pgfpathlineto{\pgfqpoint{1.993756in}{2.078756in}}%
\pgfpathlineto{\pgfqpoint{2.286712in}{2.171105in}}%
\pgfpathlineto{\pgfqpoint{2.579669in}{2.084946in}}%
\pgfpathlineto{\pgfqpoint{2.872625in}{1.987068in}}%
\pgfpathlineto{\pgfqpoint{3.165581in}{1.810383in}}%
\pgfpathlineto{\pgfqpoint{3.458537in}{1.642735in}}%
\pgfpathlineto{\pgfqpoint{3.751494in}{1.503265in}}%
\pgfpathlineto{\pgfqpoint{4.044450in}{1.402615in}}%
\pgfpathlineto{\pgfqpoint{4.337406in}{1.340742in}}%
\pgfpathlineto{\pgfqpoint{4.630362in}{1.270943in}}%
\pgfpathlineto{\pgfqpoint{4.923318in}{1.169656in}}%
\pgfpathlineto{\pgfqpoint{5.216275in}{1.179274in}}%
\pgfpathlineto{\pgfqpoint{5.509231in}{1.146867in}}%
\pgfpathlineto{\pgfqpoint{5.802187in}{1.156072in}}%
\pgfusepath{stroke}%
\end{pgfscope}%
\begin{pgfscope}%
\pgfpathrectangle{\pgfqpoint{0.572918in}{0.553781in}}{\pgfqpoint{5.478282in}{2.095553in}}%
\pgfusepath{clip}%
\pgfsetbuttcap%
\pgfsetroundjoin%
\pgfsetlinewidth{1.505625pt}%
\definecolor{currentstroke}{rgb}{0.949020,0.372549,0.360784}%
\pgfsetstrokecolor{currentstroke}%
\pgfsetstrokeopacity{0.900000}%
\pgfsetdash{{1.500000pt}{2.475000pt}}{0.000000pt}%
\pgfpathmoveto{\pgfqpoint{0.821931in}{1.880702in}}%
\pgfpathlineto{\pgfqpoint{1.114887in}{1.820329in}}%
\pgfpathlineto{\pgfqpoint{1.407844in}{1.704040in}}%
\pgfpathlineto{\pgfqpoint{1.700800in}{1.716385in}}%
\pgfpathlineto{\pgfqpoint{1.993756in}{1.787457in}}%
\pgfpathlineto{\pgfqpoint{2.286712in}{1.751429in}}%
\pgfpathlineto{\pgfqpoint{2.579669in}{1.650320in}}%
\pgfpathlineto{\pgfqpoint{2.872625in}{1.577880in}}%
\pgfpathlineto{\pgfqpoint{3.165581in}{1.423096in}}%
\pgfpathlineto{\pgfqpoint{3.458537in}{1.294629in}}%
\pgfpathlineto{\pgfqpoint{3.751494in}{1.185765in}}%
\pgfpathlineto{\pgfqpoint{4.044450in}{1.056146in}}%
\pgfpathlineto{\pgfqpoint{4.337406in}{0.933334in}}%
\pgfpathlineto{\pgfqpoint{4.630362in}{0.798280in}}%
\pgfpathlineto{\pgfqpoint{4.923318in}{0.744024in}}%
\pgfpathlineto{\pgfqpoint{5.216275in}{0.733419in}}%
\pgfpathlineto{\pgfqpoint{5.509231in}{0.713646in}}%
\pgfpathlineto{\pgfqpoint{5.802187in}{0.823385in}}%
\pgfusepath{stroke}%
\end{pgfscope}%
\begin{pgfscope}%
\pgfsetrectcap%
\pgfsetmiterjoin%
\pgfsetlinewidth{0.803000pt}%
\definecolor{currentstroke}{rgb}{0.000000,0.000000,0.000000}%
\pgfsetstrokecolor{currentstroke}%
\pgfsetdash{}{0pt}%
\pgfpathmoveto{\pgfqpoint{0.572918in}{0.553781in}}%
\pgfpathlineto{\pgfqpoint{0.572918in}{2.649333in}}%
\pgfusepath{stroke}%
\end{pgfscope}%
\begin{pgfscope}%
\pgfsetrectcap%
\pgfsetmiterjoin%
\pgfsetlinewidth{0.803000pt}%
\definecolor{currentstroke}{rgb}{0.000000,0.000000,0.000000}%
\pgfsetstrokecolor{currentstroke}%
\pgfsetdash{}{0pt}%
\pgfpathmoveto{\pgfqpoint{6.051200in}{0.553781in}}%
\pgfpathlineto{\pgfqpoint{6.051200in}{2.649333in}}%
\pgfusepath{stroke}%
\end{pgfscope}%
\begin{pgfscope}%
\pgfsetrectcap%
\pgfsetmiterjoin%
\pgfsetlinewidth{0.803000pt}%
\definecolor{currentstroke}{rgb}{0.000000,0.000000,0.000000}%
\pgfsetstrokecolor{currentstroke}%
\pgfsetdash{}{0pt}%
\pgfpathmoveto{\pgfqpoint{0.572918in}{0.553781in}}%
\pgfpathlineto{\pgfqpoint{6.051200in}{0.553781in}}%
\pgfusepath{stroke}%
\end{pgfscope}%
\begin{pgfscope}%
\pgfsetrectcap%
\pgfsetmiterjoin%
\pgfsetlinewidth{0.803000pt}%
\definecolor{currentstroke}{rgb}{0.000000,0.000000,0.000000}%
\pgfsetstrokecolor{currentstroke}%
\pgfsetdash{}{0pt}%
\pgfpathmoveto{\pgfqpoint{0.572918in}{2.649333in}}%
\pgfpathlineto{\pgfqpoint{6.051200in}{2.649333in}}%
\pgfusepath{stroke}%
\end{pgfscope}%
\begin{pgfscope}%
\definecolor{textcolor}{rgb}{0.000000,0.000000,0.000000}%
\pgfsetstrokecolor{textcolor}%
\pgfsetfillcolor{textcolor}%
\pgftext[x=0.572918in,y=2.732667in,left,base]{\color{textcolor}\rmfamily\fontsize{12.000000}{14.400000}\selectfont Zenith performance, cleaning}%
\end{pgfscope}%
\begin{pgfscope}%
\pgfsetbuttcap%
\pgfsetmiterjoin%
\definecolor{currentfill}{rgb}{1.000000,1.000000,1.000000}%
\pgfsetfillcolor{currentfill}%
\pgfsetfillopacity{0.800000}%
\pgfsetlinewidth{1.003750pt}%
\definecolor{currentstroke}{rgb}{0.800000,0.800000,0.800000}%
\pgfsetstrokecolor{currentstroke}%
\pgfsetstrokeopacity{0.800000}%
\pgfsetdash{}{0pt}%
\pgfpathmoveto{\pgfqpoint{4.795533in}{2.250222in}}%
\pgfpathlineto{\pgfqpoint{5.973422in}{2.250222in}}%
\pgfpathquadraticcurveto{\pgfqpoint{5.995644in}{2.250222in}}{\pgfqpoint{5.995644in}{2.272444in}}%
\pgfpathlineto{\pgfqpoint{5.995644in}{2.571556in}}%
\pgfpathquadraticcurveto{\pgfqpoint{5.995644in}{2.593778in}}{\pgfqpoint{5.973422in}{2.593778in}}%
\pgfpathlineto{\pgfqpoint{4.795533in}{2.593778in}}%
\pgfpathquadraticcurveto{\pgfqpoint{4.773311in}{2.593778in}}{\pgfqpoint{4.773311in}{2.571556in}}%
\pgfpathlineto{\pgfqpoint{4.773311in}{2.272444in}}%
\pgfpathquadraticcurveto{\pgfqpoint{4.773311in}{2.250222in}}{\pgfqpoint{4.795533in}{2.250222in}}%
\pgfpathclose%
\pgfusepath{stroke,fill}%
\end{pgfscope}%
\begin{pgfscope}%
\pgfsetbuttcap%
\pgfsetroundjoin%
\pgfsetlinewidth{1.505625pt}%
\definecolor{currentstroke}{rgb}{0.313725,0.317647,0.309804}%
\pgfsetstrokecolor{currentstroke}%
\pgfsetstrokeopacity{0.900000}%
\pgfsetdash{}{0pt}%
\pgfpathmoveto{\pgfqpoint{4.928867in}{2.454444in}}%
\pgfpathlineto{\pgfqpoint{4.928867in}{2.565556in}}%
\pgfusepath{stroke}%
\end{pgfscope}%
\begin{pgfscope}%
\pgfsetbuttcap%
\pgfsetroundjoin%
\definecolor{currentfill}{rgb}{0.313725,0.317647,0.309804}%
\pgfsetfillcolor{currentfill}%
\pgfsetfillopacity{0.900000}%
\pgfsetlinewidth{1.003750pt}%
\definecolor{currentstroke}{rgb}{0.313725,0.317647,0.309804}%
\pgfsetstrokecolor{currentstroke}%
\pgfsetstrokeopacity{0.900000}%
\pgfsetdash{}{0pt}%
\pgfsys@defobject{currentmarker}{\pgfqpoint{-0.013889in}{-0.000000in}}{\pgfqpoint{0.013889in}{0.000000in}}{%
\pgfpathmoveto{\pgfqpoint{0.013889in}{-0.000000in}}%
\pgfpathlineto{\pgfqpoint{-0.013889in}{0.000000in}}%
\pgfusepath{stroke,fill}%
}%
\begin{pgfscope}%
\pgfsys@transformshift{4.928867in}{2.454444in}%
\pgfsys@useobject{currentmarker}{}%
\end{pgfscope}%
\end{pgfscope}%
\begin{pgfscope}%
\pgfsetbuttcap%
\pgfsetroundjoin%
\definecolor{currentfill}{rgb}{0.313725,0.317647,0.309804}%
\pgfsetfillcolor{currentfill}%
\pgfsetfillopacity{0.900000}%
\pgfsetlinewidth{1.003750pt}%
\definecolor{currentstroke}{rgb}{0.313725,0.317647,0.309804}%
\pgfsetstrokecolor{currentstroke}%
\pgfsetstrokeopacity{0.900000}%
\pgfsetdash{}{0pt}%
\pgfsys@defobject{currentmarker}{\pgfqpoint{-0.013889in}{-0.000000in}}{\pgfqpoint{0.013889in}{0.000000in}}{%
\pgfpathmoveto{\pgfqpoint{0.013889in}{-0.000000in}}%
\pgfpathlineto{\pgfqpoint{-0.013889in}{0.000000in}}%
\pgfusepath{stroke,fill}%
}%
\begin{pgfscope}%
\pgfsys@transformshift{4.928867in}{2.565556in}%
\pgfsys@useobject{currentmarker}{}%
\end{pgfscope}%
\end{pgfscope}%
\begin{pgfscope}%
\pgfsetrectcap%
\pgfsetroundjoin%
\pgfsetlinewidth{1.505625pt}%
\definecolor{currentstroke}{rgb}{0.313725,0.317647,0.309804}%
\pgfsetstrokecolor{currentstroke}%
\pgfsetstrokeopacity{0.900000}%
\pgfsetdash{}{0pt}%
\pgfpathmoveto{\pgfqpoint{4.817755in}{2.510000in}}%
\pgfpathlineto{\pgfqpoint{5.039978in}{2.510000in}}%
\pgfusepath{stroke}%
\end{pgfscope}%
\begin{pgfscope}%
\definecolor{textcolor}{rgb}{0.000000,0.000000,0.000000}%
\pgfsetstrokecolor{textcolor}%
\pgfsetfillcolor{textcolor}%
\pgftext[x=5.128867in,y=2.471111in,left,base]{\color{textcolor}\rmfamily\fontsize{8.000000}{9.600000}\selectfont SplitInIcePulses}%
\end{pgfscope}%
\begin{pgfscope}%
\pgfsetbuttcap%
\pgfsetroundjoin%
\pgfsetlinewidth{1.505625pt}%
\definecolor{currentstroke}{rgb}{0.949020,0.372549,0.360784}%
\pgfsetstrokecolor{currentstroke}%
\pgfsetstrokeopacity{0.900000}%
\pgfsetdash{}{0pt}%
\pgfpathmoveto{\pgfqpoint{4.928867in}{2.299556in}}%
\pgfpathlineto{\pgfqpoint{4.928867in}{2.410667in}}%
\pgfusepath{stroke}%
\end{pgfscope}%
\begin{pgfscope}%
\pgfsetbuttcap%
\pgfsetroundjoin%
\definecolor{currentfill}{rgb}{0.949020,0.372549,0.360784}%
\pgfsetfillcolor{currentfill}%
\pgfsetfillopacity{0.900000}%
\pgfsetlinewidth{1.003750pt}%
\definecolor{currentstroke}{rgb}{0.949020,0.372549,0.360784}%
\pgfsetstrokecolor{currentstroke}%
\pgfsetstrokeopacity{0.900000}%
\pgfsetdash{}{0pt}%
\pgfsys@defobject{currentmarker}{\pgfqpoint{-0.013889in}{-0.000000in}}{\pgfqpoint{0.013889in}{0.000000in}}{%
\pgfpathmoveto{\pgfqpoint{0.013889in}{-0.000000in}}%
\pgfpathlineto{\pgfqpoint{-0.013889in}{0.000000in}}%
\pgfusepath{stroke,fill}%
}%
\begin{pgfscope}%
\pgfsys@transformshift{4.928867in}{2.299556in}%
\pgfsys@useobject{currentmarker}{}%
\end{pgfscope}%
\end{pgfscope}%
\begin{pgfscope}%
\pgfsetbuttcap%
\pgfsetroundjoin%
\definecolor{currentfill}{rgb}{0.949020,0.372549,0.360784}%
\pgfsetfillcolor{currentfill}%
\pgfsetfillopacity{0.900000}%
\pgfsetlinewidth{1.003750pt}%
\definecolor{currentstroke}{rgb}{0.949020,0.372549,0.360784}%
\pgfsetstrokecolor{currentstroke}%
\pgfsetstrokeopacity{0.900000}%
\pgfsetdash{}{0pt}%
\pgfsys@defobject{currentmarker}{\pgfqpoint{-0.013889in}{-0.000000in}}{\pgfqpoint{0.013889in}{0.000000in}}{%
\pgfpathmoveto{\pgfqpoint{0.013889in}{-0.000000in}}%
\pgfpathlineto{\pgfqpoint{-0.013889in}{0.000000in}}%
\pgfusepath{stroke,fill}%
}%
\begin{pgfscope}%
\pgfsys@transformshift{4.928867in}{2.410667in}%
\pgfsys@useobject{currentmarker}{}%
\end{pgfscope}%
\end{pgfscope}%
\begin{pgfscope}%
\pgfsetbuttcap%
\pgfsetroundjoin%
\pgfsetlinewidth{1.505625pt}%
\definecolor{currentstroke}{rgb}{0.949020,0.372549,0.360784}%
\pgfsetstrokecolor{currentstroke}%
\pgfsetstrokeopacity{0.900000}%
\pgfsetdash{{1.500000pt}{2.475000pt}}{0.000000pt}%
\pgfpathmoveto{\pgfqpoint{4.817755in}{2.355111in}}%
\pgfpathlineto{\pgfqpoint{5.039978in}{2.355111in}}%
\pgfusepath{stroke}%
\end{pgfscope}%
\begin{pgfscope}%
\definecolor{textcolor}{rgb}{0.000000,0.000000,0.000000}%
\pgfsetstrokecolor{textcolor}%
\pgfsetfillcolor{textcolor}%
\pgftext[x=5.128867in,y=2.316222in,left,base]{\color{textcolor}\rmfamily\fontsize{8.000000}{9.600000}\selectfont SRTInIcePulses}%
\end{pgfscope}%
\end{pgfpicture}%
\makeatother%
\endgroup%

    \caption{Impact of cleaning on zenith performance.
    SplitInIcePulses indicates events with no extra pulses removed by SRT cleaning, whilst SRTInIcePulses is the opposite.
    There is a clear difference in performance, which should be expected as the SRT cleaning removes unphysical pulses.}\label{fig:cleaning_type}
\end{figure}

Every event includes extra SRT cleaning, which removes any pulses that are unphysical (see~\vref{sec:l2}).
SRT cleaning tends to remove a large amount of pulses---as is shown in~\vref{fig:two_powershovel_events}---which is only noise, and should thus tend to aid neural networks in training.
Networks were trained with/without SRT cleaning, and it was found to be indeed so; an example is shown in~\vref{fig:cleaning_type}.
As low energy events do not tend to leave long, finely articulated tracks in DeepCore, this cleaning improves the resolution as the network does not need to pay attention to noise.

\subsection{Maximum event length}

\begin{figure}
     \centering
     \begin{subfigure}[b]{0.49\textwidth}
         \centering
         %% Creator: Matplotlib, PGF backend
%%
%% To include the figure in your LaTeX document, write
%%   \input{<filename>.pgf}
%%
%% Make sure the required packages are loaded in your preamble
%%   \usepackage{pgf}
%%
%% and, on pdftex
%%   \usepackage[utf8]{inputenc}\DeclareUnicodeCharacter{2212}{-}
%%
%% or, on luatex and xetex
%%   \usepackage{unicode-math}
%%
%% Figures using additional raster images can only be included by \input if
%% they are in the same directory as the main LaTeX file. For loading figures
%% from other directories you can use the `import` package
%%   \usepackage{import}
%%
%% and then include the figures with
%%   \import{<path to file>}{<filename>.pgf}
%%
%% Matplotlib used the following preamble
%%   \usepackage{siunitx} \usepackage{amsmath} \usepackage{bm}
%%   \usepackage{fontspec}
%%
\begingroup%
\makeatletter%
\begin{pgfpicture}%
\pgfpathrectangle{\pgfpointorigin}{\pgfqpoint{3.038588in}{2.500000in}}%
\pgfusepath{use as bounding box, clip}%
\begin{pgfscope}%
\pgfsetbuttcap%
\pgfsetmiterjoin%
\definecolor{currentfill}{rgb}{1.000000,1.000000,1.000000}%
\pgfsetfillcolor{currentfill}%
\pgfsetlinewidth{0.000000pt}%
\definecolor{currentstroke}{rgb}{1.000000,1.000000,1.000000}%
\pgfsetstrokecolor{currentstroke}%
\pgfsetdash{}{0pt}%
\pgfpathmoveto{\pgfqpoint{0.000000in}{0.000000in}}%
\pgfpathlineto{\pgfqpoint{3.038588in}{0.000000in}}%
\pgfpathlineto{\pgfqpoint{3.038588in}{2.500000in}}%
\pgfpathlineto{\pgfqpoint{0.000000in}{2.500000in}}%
\pgfpathclose%
\pgfusepath{fill}%
\end{pgfscope}%
\begin{pgfscope}%
\pgfsetbuttcap%
\pgfsetmiterjoin%
\definecolor{currentfill}{rgb}{1.000000,1.000000,1.000000}%
\pgfsetfillcolor{currentfill}%
\pgfsetlinewidth{0.000000pt}%
\definecolor{currentstroke}{rgb}{0.000000,0.000000,0.000000}%
\pgfsetstrokecolor{currentstroke}%
\pgfsetstrokeopacity{0.000000}%
\pgfsetdash{}{0pt}%
\pgfpathmoveto{\pgfqpoint{0.706736in}{0.538319in}}%
\pgfpathlineto{\pgfqpoint{2.836878in}{0.538319in}}%
\pgfpathlineto{\pgfqpoint{2.836878in}{2.151000in}}%
\pgfpathlineto{\pgfqpoint{0.706736in}{2.151000in}}%
\pgfpathclose%
\pgfusepath{fill}%
\end{pgfscope}%
\begin{pgfscope}%
\pgfsetbuttcap%
\pgfsetroundjoin%
\definecolor{currentfill}{rgb}{0.000000,0.000000,0.000000}%
\pgfsetfillcolor{currentfill}%
\pgfsetlinewidth{0.803000pt}%
\definecolor{currentstroke}{rgb}{0.000000,0.000000,0.000000}%
\pgfsetstrokecolor{currentstroke}%
\pgfsetdash{}{0pt}%
\pgfsys@defobject{currentmarker}{\pgfqpoint{0.000000in}{-0.048611in}}{\pgfqpoint{0.000000in}{0.000000in}}{%
\pgfpathmoveto{\pgfqpoint{0.000000in}{0.000000in}}%
\pgfpathlineto{\pgfqpoint{0.000000in}{-0.048611in}}%
\pgfusepath{stroke,fill}%
}%
\begin{pgfscope}%
\pgfsys@transformshift{0.870156in}{0.538319in}%
\pgfsys@useobject{currentmarker}{}%
\end{pgfscope}%
\end{pgfscope}%
\begin{pgfscope}%
\definecolor{textcolor}{rgb}{0.000000,0.000000,0.000000}%
\pgfsetstrokecolor{textcolor}%
\pgfsetfillcolor{textcolor}%
\pgftext[x=0.870156in,y=0.441097in,,top]{\color{textcolor}\rmfamily\fontsize{8.000000}{9.600000}\selectfont \(\displaystyle {0}\)}%
\end{pgfscope}%
\begin{pgfscope}%
\pgfsetbuttcap%
\pgfsetroundjoin%
\definecolor{currentfill}{rgb}{0.000000,0.000000,0.000000}%
\pgfsetfillcolor{currentfill}%
\pgfsetlinewidth{0.803000pt}%
\definecolor{currentstroke}{rgb}{0.000000,0.000000,0.000000}%
\pgfsetstrokecolor{currentstroke}%
\pgfsetdash{}{0pt}%
\pgfsys@defobject{currentmarker}{\pgfqpoint{0.000000in}{-0.048611in}}{\pgfqpoint{0.000000in}{0.000000in}}{%
\pgfpathmoveto{\pgfqpoint{0.000000in}{0.000000in}}%
\pgfpathlineto{\pgfqpoint{0.000000in}{-0.048611in}}%
\pgfusepath{stroke,fill}%
}%
\begin{pgfscope}%
\pgfsys@transformshift{1.084978in}{0.538319in}%
\pgfsys@useobject{currentmarker}{}%
\end{pgfscope}%
\end{pgfscope}%
\begin{pgfscope}%
\definecolor{textcolor}{rgb}{0.000000,0.000000,0.000000}%
\pgfsetstrokecolor{textcolor}%
\pgfsetfillcolor{textcolor}%
\pgftext[x=1.084978in,y=0.441097in,,top]{\color{textcolor}\rmfamily\fontsize{8.000000}{9.600000}\selectfont \(\displaystyle {80}\)}%
\end{pgfscope}%
\begin{pgfscope}%
\pgfsetbuttcap%
\pgfsetroundjoin%
\definecolor{currentfill}{rgb}{0.000000,0.000000,0.000000}%
\pgfsetfillcolor{currentfill}%
\pgfsetlinewidth{0.803000pt}%
\definecolor{currentstroke}{rgb}{0.000000,0.000000,0.000000}%
\pgfsetstrokecolor{currentstroke}%
\pgfsetdash{}{0pt}%
\pgfsys@defobject{currentmarker}{\pgfqpoint{0.000000in}{-0.048611in}}{\pgfqpoint{0.000000in}{0.000000in}}{%
\pgfpathmoveto{\pgfqpoint{0.000000in}{0.000000in}}%
\pgfpathlineto{\pgfqpoint{0.000000in}{-0.048611in}}%
\pgfusepath{stroke,fill}%
}%
\begin{pgfscope}%
\pgfsys@transformshift{1.299801in}{0.538319in}%
\pgfsys@useobject{currentmarker}{}%
\end{pgfscope}%
\end{pgfscope}%
\begin{pgfscope}%
\definecolor{textcolor}{rgb}{0.000000,0.000000,0.000000}%
\pgfsetstrokecolor{textcolor}%
\pgfsetfillcolor{textcolor}%
\pgftext[x=1.299801in,y=0.441097in,,top]{\color{textcolor}\rmfamily\fontsize{8.000000}{9.600000}\selectfont \(\displaystyle {160}\)}%
\end{pgfscope}%
\begin{pgfscope}%
\pgfsetbuttcap%
\pgfsetroundjoin%
\definecolor{currentfill}{rgb}{0.000000,0.000000,0.000000}%
\pgfsetfillcolor{currentfill}%
\pgfsetlinewidth{0.803000pt}%
\definecolor{currentstroke}{rgb}{0.000000,0.000000,0.000000}%
\pgfsetstrokecolor{currentstroke}%
\pgfsetdash{}{0pt}%
\pgfsys@defobject{currentmarker}{\pgfqpoint{0.000000in}{-0.048611in}}{\pgfqpoint{0.000000in}{0.000000in}}{%
\pgfpathmoveto{\pgfqpoint{0.000000in}{0.000000in}}%
\pgfpathlineto{\pgfqpoint{0.000000in}{-0.048611in}}%
\pgfusepath{stroke,fill}%
}%
\begin{pgfscope}%
\pgfsys@transformshift{1.514624in}{0.538319in}%
\pgfsys@useobject{currentmarker}{}%
\end{pgfscope}%
\end{pgfscope}%
\begin{pgfscope}%
\definecolor{textcolor}{rgb}{0.000000,0.000000,0.000000}%
\pgfsetstrokecolor{textcolor}%
\pgfsetfillcolor{textcolor}%
\pgftext[x=1.514624in,y=0.441097in,,top]{\color{textcolor}\rmfamily\fontsize{8.000000}{9.600000}\selectfont \(\displaystyle {240}\)}%
\end{pgfscope}%
\begin{pgfscope}%
\pgfsetbuttcap%
\pgfsetroundjoin%
\definecolor{currentfill}{rgb}{0.000000,0.000000,0.000000}%
\pgfsetfillcolor{currentfill}%
\pgfsetlinewidth{0.803000pt}%
\definecolor{currentstroke}{rgb}{0.000000,0.000000,0.000000}%
\pgfsetstrokecolor{currentstroke}%
\pgfsetdash{}{0pt}%
\pgfsys@defobject{currentmarker}{\pgfqpoint{0.000000in}{-0.048611in}}{\pgfqpoint{0.000000in}{0.000000in}}{%
\pgfpathmoveto{\pgfqpoint{0.000000in}{0.000000in}}%
\pgfpathlineto{\pgfqpoint{0.000000in}{-0.048611in}}%
\pgfusepath{stroke,fill}%
}%
\begin{pgfscope}%
\pgfsys@transformshift{1.729447in}{0.538319in}%
\pgfsys@useobject{currentmarker}{}%
\end{pgfscope}%
\end{pgfscope}%
\begin{pgfscope}%
\definecolor{textcolor}{rgb}{0.000000,0.000000,0.000000}%
\pgfsetstrokecolor{textcolor}%
\pgfsetfillcolor{textcolor}%
\pgftext[x=1.729447in,y=0.441097in,,top]{\color{textcolor}\rmfamily\fontsize{8.000000}{9.600000}\selectfont \(\displaystyle {320}\)}%
\end{pgfscope}%
\begin{pgfscope}%
\pgfsetbuttcap%
\pgfsetroundjoin%
\definecolor{currentfill}{rgb}{0.000000,0.000000,0.000000}%
\pgfsetfillcolor{currentfill}%
\pgfsetlinewidth{0.803000pt}%
\definecolor{currentstroke}{rgb}{0.000000,0.000000,0.000000}%
\pgfsetstrokecolor{currentstroke}%
\pgfsetdash{}{0pt}%
\pgfsys@defobject{currentmarker}{\pgfqpoint{0.000000in}{-0.048611in}}{\pgfqpoint{0.000000in}{0.000000in}}{%
\pgfpathmoveto{\pgfqpoint{0.000000in}{0.000000in}}%
\pgfpathlineto{\pgfqpoint{0.000000in}{-0.048611in}}%
\pgfusepath{stroke,fill}%
}%
\begin{pgfscope}%
\pgfsys@transformshift{1.944270in}{0.538319in}%
\pgfsys@useobject{currentmarker}{}%
\end{pgfscope}%
\end{pgfscope}%
\begin{pgfscope}%
\definecolor{textcolor}{rgb}{0.000000,0.000000,0.000000}%
\pgfsetstrokecolor{textcolor}%
\pgfsetfillcolor{textcolor}%
\pgftext[x=1.944270in,y=0.441097in,,top]{\color{textcolor}\rmfamily\fontsize{8.000000}{9.600000}\selectfont \(\displaystyle {400}\)}%
\end{pgfscope}%
\begin{pgfscope}%
\pgfsetbuttcap%
\pgfsetroundjoin%
\definecolor{currentfill}{rgb}{0.000000,0.000000,0.000000}%
\pgfsetfillcolor{currentfill}%
\pgfsetlinewidth{0.803000pt}%
\definecolor{currentstroke}{rgb}{0.000000,0.000000,0.000000}%
\pgfsetstrokecolor{currentstroke}%
\pgfsetdash{}{0pt}%
\pgfsys@defobject{currentmarker}{\pgfqpoint{0.000000in}{-0.048611in}}{\pgfqpoint{0.000000in}{0.000000in}}{%
\pgfpathmoveto{\pgfqpoint{0.000000in}{0.000000in}}%
\pgfpathlineto{\pgfqpoint{0.000000in}{-0.048611in}}%
\pgfusepath{stroke,fill}%
}%
\begin{pgfscope}%
\pgfsys@transformshift{2.159092in}{0.538319in}%
\pgfsys@useobject{currentmarker}{}%
\end{pgfscope}%
\end{pgfscope}%
\begin{pgfscope}%
\definecolor{textcolor}{rgb}{0.000000,0.000000,0.000000}%
\pgfsetstrokecolor{textcolor}%
\pgfsetfillcolor{textcolor}%
\pgftext[x=2.159092in,y=0.441097in,,top]{\color{textcolor}\rmfamily\fontsize{8.000000}{9.600000}\selectfont \(\displaystyle {480}\)}%
\end{pgfscope}%
\begin{pgfscope}%
\pgfsetbuttcap%
\pgfsetroundjoin%
\definecolor{currentfill}{rgb}{0.000000,0.000000,0.000000}%
\pgfsetfillcolor{currentfill}%
\pgfsetlinewidth{0.803000pt}%
\definecolor{currentstroke}{rgb}{0.000000,0.000000,0.000000}%
\pgfsetstrokecolor{currentstroke}%
\pgfsetdash{}{0pt}%
\pgfsys@defobject{currentmarker}{\pgfqpoint{0.000000in}{-0.048611in}}{\pgfqpoint{0.000000in}{0.000000in}}{%
\pgfpathmoveto{\pgfqpoint{0.000000in}{0.000000in}}%
\pgfpathlineto{\pgfqpoint{0.000000in}{-0.048611in}}%
\pgfusepath{stroke,fill}%
}%
\begin{pgfscope}%
\pgfsys@transformshift{2.373915in}{0.538319in}%
\pgfsys@useobject{currentmarker}{}%
\end{pgfscope}%
\end{pgfscope}%
\begin{pgfscope}%
\definecolor{textcolor}{rgb}{0.000000,0.000000,0.000000}%
\pgfsetstrokecolor{textcolor}%
\pgfsetfillcolor{textcolor}%
\pgftext[x=2.373915in,y=0.441097in,,top]{\color{textcolor}\rmfamily\fontsize{8.000000}{9.600000}\selectfont \(\displaystyle {560}\)}%
\end{pgfscope}%
\begin{pgfscope}%
\pgfsetbuttcap%
\pgfsetroundjoin%
\definecolor{currentfill}{rgb}{0.000000,0.000000,0.000000}%
\pgfsetfillcolor{currentfill}%
\pgfsetlinewidth{0.803000pt}%
\definecolor{currentstroke}{rgb}{0.000000,0.000000,0.000000}%
\pgfsetstrokecolor{currentstroke}%
\pgfsetdash{}{0pt}%
\pgfsys@defobject{currentmarker}{\pgfqpoint{0.000000in}{-0.048611in}}{\pgfqpoint{0.000000in}{0.000000in}}{%
\pgfpathmoveto{\pgfqpoint{0.000000in}{0.000000in}}%
\pgfpathlineto{\pgfqpoint{0.000000in}{-0.048611in}}%
\pgfusepath{stroke,fill}%
}%
\begin{pgfscope}%
\pgfsys@transformshift{2.588738in}{0.538319in}%
\pgfsys@useobject{currentmarker}{}%
\end{pgfscope}%
\end{pgfscope}%
\begin{pgfscope}%
\definecolor{textcolor}{rgb}{0.000000,0.000000,0.000000}%
\pgfsetstrokecolor{textcolor}%
\pgfsetfillcolor{textcolor}%
\pgftext[x=2.588738in,y=0.441097in,,top]{\color{textcolor}\rmfamily\fontsize{8.000000}{9.600000}\selectfont \(\displaystyle {640}\)}%
\end{pgfscope}%
\begin{pgfscope}%
\pgfsetbuttcap%
\pgfsetroundjoin%
\definecolor{currentfill}{rgb}{0.000000,0.000000,0.000000}%
\pgfsetfillcolor{currentfill}%
\pgfsetlinewidth{0.803000pt}%
\definecolor{currentstroke}{rgb}{0.000000,0.000000,0.000000}%
\pgfsetstrokecolor{currentstroke}%
\pgfsetdash{}{0pt}%
\pgfsys@defobject{currentmarker}{\pgfqpoint{0.000000in}{-0.048611in}}{\pgfqpoint{0.000000in}{0.000000in}}{%
\pgfpathmoveto{\pgfqpoint{0.000000in}{0.000000in}}%
\pgfpathlineto{\pgfqpoint{0.000000in}{-0.048611in}}%
\pgfusepath{stroke,fill}%
}%
\begin{pgfscope}%
\pgfsys@transformshift{2.803561in}{0.538319in}%
\pgfsys@useobject{currentmarker}{}%
\end{pgfscope}%
\end{pgfscope}%
\begin{pgfscope}%
\definecolor{textcolor}{rgb}{0.000000,0.000000,0.000000}%
\pgfsetstrokecolor{textcolor}%
\pgfsetfillcolor{textcolor}%
\pgftext[x=2.803561in,y=0.441097in,,top]{\color{textcolor}\rmfamily\fontsize{8.000000}{9.600000}\selectfont \(\displaystyle {720}\)}%
\end{pgfscope}%
\begin{pgfscope}%
\definecolor{textcolor}{rgb}{0.000000,0.000000,0.000000}%
\pgfsetstrokecolor{textcolor}%
\pgfsetfillcolor{textcolor}%
\pgftext[x=1.771807in,y=0.286875in,,top]{\color{textcolor}\rmfamily\fontsize{10.950000}{13.140000}\selectfont Event length}%
\end{pgfscope}%
\begin{pgfscope}%
\pgfsetbuttcap%
\pgfsetroundjoin%
\definecolor{currentfill}{rgb}{0.000000,0.000000,0.000000}%
\pgfsetfillcolor{currentfill}%
\pgfsetlinewidth{0.803000pt}%
\definecolor{currentstroke}{rgb}{0.000000,0.000000,0.000000}%
\pgfsetstrokecolor{currentstroke}%
\pgfsetdash{}{0pt}%
\pgfsys@defobject{currentmarker}{\pgfqpoint{-0.048611in}{0.000000in}}{\pgfqpoint{-0.000000in}{0.000000in}}{%
\pgfpathmoveto{\pgfqpoint{-0.000000in}{0.000000in}}%
\pgfpathlineto{\pgfqpoint{-0.048611in}{0.000000in}}%
\pgfusepath{stroke,fill}%
}%
\begin{pgfscope}%
\pgfsys@transformshift{0.706736in}{0.611623in}%
\pgfsys@useobject{currentmarker}{}%
\end{pgfscope}%
\end{pgfscope}%
\begin{pgfscope}%
\definecolor{textcolor}{rgb}{0.000000,0.000000,0.000000}%
\pgfsetstrokecolor{textcolor}%
\pgfsetfillcolor{textcolor}%
\pgftext[x=0.340606in, y=0.573068in, left, base]{\color{textcolor}\rmfamily\fontsize{8.000000}{9.600000}\selectfont \(\displaystyle {0.000}\)}%
\end{pgfscope}%
\begin{pgfscope}%
\pgfsetbuttcap%
\pgfsetroundjoin%
\definecolor{currentfill}{rgb}{0.000000,0.000000,0.000000}%
\pgfsetfillcolor{currentfill}%
\pgfsetlinewidth{0.803000pt}%
\definecolor{currentstroke}{rgb}{0.000000,0.000000,0.000000}%
\pgfsetstrokecolor{currentstroke}%
\pgfsetdash{}{0pt}%
\pgfsys@defobject{currentmarker}{\pgfqpoint{-0.048611in}{0.000000in}}{\pgfqpoint{-0.000000in}{0.000000in}}{%
\pgfpathmoveto{\pgfqpoint{-0.000000in}{0.000000in}}%
\pgfpathlineto{\pgfqpoint{-0.048611in}{0.000000in}}%
\pgfusepath{stroke,fill}%
}%
\begin{pgfscope}%
\pgfsys@transformshift{0.706736in}{0.824158in}%
\pgfsys@useobject{currentmarker}{}%
\end{pgfscope}%
\end{pgfscope}%
\begin{pgfscope}%
\definecolor{textcolor}{rgb}{0.000000,0.000000,0.000000}%
\pgfsetstrokecolor{textcolor}%
\pgfsetfillcolor{textcolor}%
\pgftext[x=0.340606in, y=0.785602in, left, base]{\color{textcolor}\rmfamily\fontsize{8.000000}{9.600000}\selectfont \(\displaystyle {0.008}\)}%
\end{pgfscope}%
\begin{pgfscope}%
\pgfsetbuttcap%
\pgfsetroundjoin%
\definecolor{currentfill}{rgb}{0.000000,0.000000,0.000000}%
\pgfsetfillcolor{currentfill}%
\pgfsetlinewidth{0.803000pt}%
\definecolor{currentstroke}{rgb}{0.000000,0.000000,0.000000}%
\pgfsetstrokecolor{currentstroke}%
\pgfsetdash{}{0pt}%
\pgfsys@defobject{currentmarker}{\pgfqpoint{-0.048611in}{0.000000in}}{\pgfqpoint{-0.000000in}{0.000000in}}{%
\pgfpathmoveto{\pgfqpoint{-0.000000in}{0.000000in}}%
\pgfpathlineto{\pgfqpoint{-0.048611in}{0.000000in}}%
\pgfusepath{stroke,fill}%
}%
\begin{pgfscope}%
\pgfsys@transformshift{0.706736in}{1.036692in}%
\pgfsys@useobject{currentmarker}{}%
\end{pgfscope}%
\end{pgfscope}%
\begin{pgfscope}%
\definecolor{textcolor}{rgb}{0.000000,0.000000,0.000000}%
\pgfsetstrokecolor{textcolor}%
\pgfsetfillcolor{textcolor}%
\pgftext[x=0.340606in, y=0.998136in, left, base]{\color{textcolor}\rmfamily\fontsize{8.000000}{9.600000}\selectfont \(\displaystyle {0.016}\)}%
\end{pgfscope}%
\begin{pgfscope}%
\pgfsetbuttcap%
\pgfsetroundjoin%
\definecolor{currentfill}{rgb}{0.000000,0.000000,0.000000}%
\pgfsetfillcolor{currentfill}%
\pgfsetlinewidth{0.803000pt}%
\definecolor{currentstroke}{rgb}{0.000000,0.000000,0.000000}%
\pgfsetstrokecolor{currentstroke}%
\pgfsetdash{}{0pt}%
\pgfsys@defobject{currentmarker}{\pgfqpoint{-0.048611in}{0.000000in}}{\pgfqpoint{-0.000000in}{0.000000in}}{%
\pgfpathmoveto{\pgfqpoint{-0.000000in}{0.000000in}}%
\pgfpathlineto{\pgfqpoint{-0.048611in}{0.000000in}}%
\pgfusepath{stroke,fill}%
}%
\begin{pgfscope}%
\pgfsys@transformshift{0.706736in}{1.249226in}%
\pgfsys@useobject{currentmarker}{}%
\end{pgfscope}%
\end{pgfscope}%
\begin{pgfscope}%
\definecolor{textcolor}{rgb}{0.000000,0.000000,0.000000}%
\pgfsetstrokecolor{textcolor}%
\pgfsetfillcolor{textcolor}%
\pgftext[x=0.340606in, y=1.210671in, left, base]{\color{textcolor}\rmfamily\fontsize{8.000000}{9.600000}\selectfont \(\displaystyle {0.024}\)}%
\end{pgfscope}%
\begin{pgfscope}%
\pgfsetbuttcap%
\pgfsetroundjoin%
\definecolor{currentfill}{rgb}{0.000000,0.000000,0.000000}%
\pgfsetfillcolor{currentfill}%
\pgfsetlinewidth{0.803000pt}%
\definecolor{currentstroke}{rgb}{0.000000,0.000000,0.000000}%
\pgfsetstrokecolor{currentstroke}%
\pgfsetdash{}{0pt}%
\pgfsys@defobject{currentmarker}{\pgfqpoint{-0.048611in}{0.000000in}}{\pgfqpoint{-0.000000in}{0.000000in}}{%
\pgfpathmoveto{\pgfqpoint{-0.000000in}{0.000000in}}%
\pgfpathlineto{\pgfqpoint{-0.048611in}{0.000000in}}%
\pgfusepath{stroke,fill}%
}%
\begin{pgfscope}%
\pgfsys@transformshift{0.706736in}{1.461761in}%
\pgfsys@useobject{currentmarker}{}%
\end{pgfscope}%
\end{pgfscope}%
\begin{pgfscope}%
\definecolor{textcolor}{rgb}{0.000000,0.000000,0.000000}%
\pgfsetstrokecolor{textcolor}%
\pgfsetfillcolor{textcolor}%
\pgftext[x=0.340606in, y=1.423205in, left, base]{\color{textcolor}\rmfamily\fontsize{8.000000}{9.600000}\selectfont \(\displaystyle {0.032}\)}%
\end{pgfscope}%
\begin{pgfscope}%
\pgfsetbuttcap%
\pgfsetroundjoin%
\definecolor{currentfill}{rgb}{0.000000,0.000000,0.000000}%
\pgfsetfillcolor{currentfill}%
\pgfsetlinewidth{0.803000pt}%
\definecolor{currentstroke}{rgb}{0.000000,0.000000,0.000000}%
\pgfsetstrokecolor{currentstroke}%
\pgfsetdash{}{0pt}%
\pgfsys@defobject{currentmarker}{\pgfqpoint{-0.048611in}{0.000000in}}{\pgfqpoint{-0.000000in}{0.000000in}}{%
\pgfpathmoveto{\pgfqpoint{-0.000000in}{0.000000in}}%
\pgfpathlineto{\pgfqpoint{-0.048611in}{0.000000in}}%
\pgfusepath{stroke,fill}%
}%
\begin{pgfscope}%
\pgfsys@transformshift{0.706736in}{1.674295in}%
\pgfsys@useobject{currentmarker}{}%
\end{pgfscope}%
\end{pgfscope}%
\begin{pgfscope}%
\definecolor{textcolor}{rgb}{0.000000,0.000000,0.000000}%
\pgfsetstrokecolor{textcolor}%
\pgfsetfillcolor{textcolor}%
\pgftext[x=0.340606in, y=1.635740in, left, base]{\color{textcolor}\rmfamily\fontsize{8.000000}{9.600000}\selectfont \(\displaystyle {0.040}\)}%
\end{pgfscope}%
\begin{pgfscope}%
\pgfsetbuttcap%
\pgfsetroundjoin%
\definecolor{currentfill}{rgb}{0.000000,0.000000,0.000000}%
\pgfsetfillcolor{currentfill}%
\pgfsetlinewidth{0.803000pt}%
\definecolor{currentstroke}{rgb}{0.000000,0.000000,0.000000}%
\pgfsetstrokecolor{currentstroke}%
\pgfsetdash{}{0pt}%
\pgfsys@defobject{currentmarker}{\pgfqpoint{-0.048611in}{0.000000in}}{\pgfqpoint{-0.000000in}{0.000000in}}{%
\pgfpathmoveto{\pgfqpoint{-0.000000in}{0.000000in}}%
\pgfpathlineto{\pgfqpoint{-0.048611in}{0.000000in}}%
\pgfusepath{stroke,fill}%
}%
\begin{pgfscope}%
\pgfsys@transformshift{0.706736in}{1.886829in}%
\pgfsys@useobject{currentmarker}{}%
\end{pgfscope}%
\end{pgfscope}%
\begin{pgfscope}%
\definecolor{textcolor}{rgb}{0.000000,0.000000,0.000000}%
\pgfsetstrokecolor{textcolor}%
\pgfsetfillcolor{textcolor}%
\pgftext[x=0.340606in, y=1.848274in, left, base]{\color{textcolor}\rmfamily\fontsize{8.000000}{9.600000}\selectfont \(\displaystyle {0.048}\)}%
\end{pgfscope}%
\begin{pgfscope}%
\pgfsetbuttcap%
\pgfsetroundjoin%
\definecolor{currentfill}{rgb}{0.000000,0.000000,0.000000}%
\pgfsetfillcolor{currentfill}%
\pgfsetlinewidth{0.803000pt}%
\definecolor{currentstroke}{rgb}{0.000000,0.000000,0.000000}%
\pgfsetstrokecolor{currentstroke}%
\pgfsetdash{}{0pt}%
\pgfsys@defobject{currentmarker}{\pgfqpoint{-0.048611in}{0.000000in}}{\pgfqpoint{-0.000000in}{0.000000in}}{%
\pgfpathmoveto{\pgfqpoint{-0.000000in}{0.000000in}}%
\pgfpathlineto{\pgfqpoint{-0.048611in}{0.000000in}}%
\pgfusepath{stroke,fill}%
}%
\begin{pgfscope}%
\pgfsys@transformshift{0.706736in}{2.099364in}%
\pgfsys@useobject{currentmarker}{}%
\end{pgfscope}%
\end{pgfscope}%
\begin{pgfscope}%
\definecolor{textcolor}{rgb}{0.000000,0.000000,0.000000}%
\pgfsetstrokecolor{textcolor}%
\pgfsetfillcolor{textcolor}%
\pgftext[x=0.340606in, y=2.060808in, left, base]{\color{textcolor}\rmfamily\fontsize{8.000000}{9.600000}\selectfont \(\displaystyle {0.056}\)}%
\end{pgfscope}%
\begin{pgfscope}%
\definecolor{textcolor}{rgb}{0.000000,0.000000,0.000000}%
\pgfsetstrokecolor{textcolor}%
\pgfsetfillcolor{textcolor}%
\pgftext[x=0.285050in,y=1.344660in,,bottom,rotate=90.000000]{\color{textcolor}\rmfamily\fontsize{10.950000}{13.140000}\selectfont Density}%
\end{pgfscope}%
\begin{pgfscope}%
\pgfpathrectangle{\pgfqpoint{0.706736in}{0.538319in}}{\pgfqpoint{2.130142in}{1.612681in}}%
\pgfusepath{clip}%
\pgfsetrectcap%
\pgfsetroundjoin%
\pgfsetlinewidth{0.501875pt}%
\definecolor{currentstroke}{rgb}{0.313725,0.317647,0.309804}%
\pgfsetstrokecolor{currentstroke}%
\pgfsetdash{}{0pt}%
\pgfpathmoveto{\pgfqpoint{0.828937in}{0.611623in}}%
\pgfpathlineto{\pgfqpoint{0.920476in}{0.612220in}}%
\pgfpathlineto{\pgfqpoint{0.926080in}{0.614606in}}%
\pgfpathlineto{\pgfqpoint{0.929817in}{0.619069in}}%
\pgfpathlineto{\pgfqpoint{0.933553in}{0.626606in}}%
\pgfpathlineto{\pgfqpoint{0.937289in}{0.639391in}}%
\pgfpathlineto{\pgfqpoint{0.941026in}{0.660246in}}%
\pgfpathlineto{\pgfqpoint{0.946630in}{0.716580in}}%
\pgfpathlineto{\pgfqpoint{0.952234in}{0.803882in}}%
\pgfpathlineto{\pgfqpoint{0.961575in}{1.027199in}}%
\pgfpathlineto{\pgfqpoint{0.976520in}{1.376473in}}%
\pgfpathlineto{\pgfqpoint{0.980257in}{1.453024in}}%
\pgfpathlineto{\pgfqpoint{0.987729in}{1.485302in}}%
\pgfpathlineto{\pgfqpoint{0.989597in}{1.478790in}}%
\pgfpathlineto{\pgfqpoint{0.991466in}{1.479455in}}%
\pgfpathlineto{\pgfqpoint{0.993334in}{1.471133in}}%
\pgfpathlineto{\pgfqpoint{0.995202in}{1.453007in}}%
\pgfpathlineto{\pgfqpoint{0.998938in}{1.435947in}}%
\pgfpathlineto{\pgfqpoint{1.013883in}{1.207299in}}%
\pgfpathlineto{\pgfqpoint{1.019488in}{1.127842in}}%
\pgfpathlineto{\pgfqpoint{1.034433in}{0.956777in}}%
\pgfpathlineto{\pgfqpoint{1.049378in}{0.850355in}}%
\pgfpathlineto{\pgfqpoint{1.058719in}{0.797960in}}%
\pgfpathlineto{\pgfqpoint{1.086741in}{0.719268in}}%
\pgfpathlineto{\pgfqpoint{1.112895in}{0.681877in}}%
\pgfpathlineto{\pgfqpoint{1.116632in}{0.678311in}}%
\pgfpathlineto{\pgfqpoint{1.120368in}{0.674489in}}%
\pgfpathlineto{\pgfqpoint{1.127841in}{0.669102in}}%
\pgfpathlineto{\pgfqpoint{1.131577in}{0.667750in}}%
\pgfpathlineto{\pgfqpoint{1.135313in}{0.663169in}}%
\pgfpathlineto{\pgfqpoint{1.137181in}{0.661018in}}%
\pgfpathlineto{\pgfqpoint{1.142786in}{0.658926in}}%
\pgfpathlineto{\pgfqpoint{1.157731in}{0.652220in}}%
\pgfpathlineto{\pgfqpoint{1.161467in}{0.650440in}}%
\pgfpathlineto{\pgfqpoint{1.170808in}{0.644173in}}%
\pgfpathlineto{\pgfqpoint{1.176412in}{0.642749in}}%
\pgfpathlineto{\pgfqpoint{1.200698in}{0.634003in}}%
\pgfpathlineto{\pgfqpoint{1.213775in}{0.632630in}}%
\pgfpathlineto{\pgfqpoint{1.223116in}{0.629937in}}%
\pgfpathlineto{\pgfqpoint{1.232457in}{0.629442in}}%
\pgfpathlineto{\pgfqpoint{1.243666in}{0.625808in}}%
\pgfpathlineto{\pgfqpoint{1.256743in}{0.624486in}}%
\pgfpathlineto{\pgfqpoint{1.260479in}{0.623764in}}%
\pgfpathlineto{\pgfqpoint{1.267952in}{0.624118in}}%
\pgfpathlineto{\pgfqpoint{1.284765in}{0.622863in}}%
\pgfpathlineto{\pgfqpoint{1.288501in}{0.622882in}}%
\pgfpathlineto{\pgfqpoint{1.299710in}{0.620332in}}%
\pgfpathlineto{\pgfqpoint{1.309051in}{0.620017in}}%
\pgfpathlineto{\pgfqpoint{1.316524in}{0.618784in}}%
\pgfpathlineto{\pgfqpoint{1.350150in}{0.617223in}}%
\pgfpathlineto{\pgfqpoint{1.359491in}{0.616990in}}%
\pgfpathlineto{\pgfqpoint{1.365096in}{0.616427in}}%
\pgfpathlineto{\pgfqpoint{1.372568in}{0.615903in}}%
\pgfpathlineto{\pgfqpoint{1.398722in}{0.614676in}}%
\pgfpathlineto{\pgfqpoint{1.406195in}{0.614568in}}%
\pgfpathlineto{\pgfqpoint{1.413667in}{0.614586in}}%
\pgfpathlineto{\pgfqpoint{1.421140in}{0.613572in}}%
\pgfpathlineto{\pgfqpoint{1.436085in}{0.614661in}}%
\pgfpathlineto{\pgfqpoint{1.518284in}{0.612754in}}%
\pgfpathlineto{\pgfqpoint{1.531361in}{0.612791in}}%
\pgfpathlineto{\pgfqpoint{1.538833in}{0.612996in}}%
\pgfpathlineto{\pgfqpoint{1.548174in}{0.612545in}}%
\pgfpathlineto{\pgfqpoint{2.740054in}{0.611623in}}%
\pgfpathlineto{\pgfqpoint{2.740054in}{0.611623in}}%
\pgfusepath{stroke}%
\end{pgfscope}%
\begin{pgfscope}%
\pgfpathrectangle{\pgfqpoint{0.706736in}{0.538319in}}{\pgfqpoint{2.130142in}{1.612681in}}%
\pgfusepath{clip}%
\pgfsetbuttcap%
\pgfsetroundjoin%
\pgfsetlinewidth{0.501875pt}%
\definecolor{currentstroke}{rgb}{0.949020,0.372549,0.360784}%
\pgfsetstrokecolor{currentstroke}%
\pgfsetdash{{0.500000pt}{0.825000pt}}{0.000000pt}%
\pgfpathmoveto{\pgfqpoint{0.803561in}{0.611623in}}%
\pgfpathlineto{\pgfqpoint{0.876485in}{0.611946in}}%
\pgfpathlineto{\pgfqpoint{0.878264in}{0.614160in}}%
\pgfpathlineto{\pgfqpoint{0.880042in}{0.624942in}}%
\pgfpathlineto{\pgfqpoint{0.881821in}{0.660807in}}%
\pgfpathlineto{\pgfqpoint{0.883599in}{0.746276in}}%
\pgfpathlineto{\pgfqpoint{0.887157in}{1.111784in}}%
\pgfpathlineto{\pgfqpoint{0.894271in}{1.906196in}}%
\pgfpathlineto{\pgfqpoint{0.897829in}{2.052031in}}%
\pgfpathlineto{\pgfqpoint{0.899607in}{2.077696in}}%
\pgfpathlineto{\pgfqpoint{0.901386in}{2.060953in}}%
\pgfpathlineto{\pgfqpoint{0.904943in}{1.983980in}}%
\pgfpathlineto{\pgfqpoint{0.924508in}{1.296499in}}%
\pgfpathlineto{\pgfqpoint{0.936959in}{1.060436in}}%
\pgfpathlineto{\pgfqpoint{0.944073in}{0.969522in}}%
\pgfpathlineto{\pgfqpoint{0.949409in}{0.923415in}}%
\pgfpathlineto{\pgfqpoint{0.965417in}{0.817757in}}%
\pgfpathlineto{\pgfqpoint{0.981425in}{0.756107in}}%
\pgfpathlineto{\pgfqpoint{0.993875in}{0.724779in}}%
\pgfpathlineto{\pgfqpoint{0.997432in}{0.718985in}}%
\pgfpathlineto{\pgfqpoint{1.002768in}{0.708249in}}%
\pgfpathlineto{\pgfqpoint{1.006326in}{0.703544in}}%
\pgfpathlineto{\pgfqpoint{1.015219in}{0.689381in}}%
\pgfpathlineto{\pgfqpoint{1.024112in}{0.679818in}}%
\pgfpathlineto{\pgfqpoint{1.027669in}{0.677029in}}%
\pgfpathlineto{\pgfqpoint{1.031227in}{0.673703in}}%
\pgfpathlineto{\pgfqpoint{1.034784in}{0.672788in}}%
\pgfpathlineto{\pgfqpoint{1.036562in}{0.672482in}}%
\pgfpathlineto{\pgfqpoint{1.049013in}{0.660286in}}%
\pgfpathlineto{\pgfqpoint{1.066799in}{0.649601in}}%
\pgfpathlineto{\pgfqpoint{1.070357in}{0.648085in}}%
\pgfpathlineto{\pgfqpoint{1.073914in}{0.647586in}}%
\pgfpathlineto{\pgfqpoint{1.082807in}{0.643351in}}%
\pgfpathlineto{\pgfqpoint{1.091700in}{0.640858in}}%
\pgfpathlineto{\pgfqpoint{1.098815in}{0.637705in}}%
\pgfpathlineto{\pgfqpoint{1.104151in}{0.635428in}}%
\pgfpathlineto{\pgfqpoint{1.111265in}{0.632111in}}%
\pgfpathlineto{\pgfqpoint{1.116601in}{0.632083in}}%
\pgfpathlineto{\pgfqpoint{1.121937in}{0.632269in}}%
\pgfpathlineto{\pgfqpoint{1.127273in}{0.630104in}}%
\pgfpathlineto{\pgfqpoint{1.137945in}{0.628647in}}%
\pgfpathlineto{\pgfqpoint{1.171739in}{0.624017in}}%
\pgfpathlineto{\pgfqpoint{1.175296in}{0.624548in}}%
\pgfpathlineto{\pgfqpoint{1.180632in}{0.623201in}}%
\pgfpathlineto{\pgfqpoint{1.185968in}{0.623594in}}%
\pgfpathlineto{\pgfqpoint{1.198419in}{0.621058in}}%
\pgfpathlineto{\pgfqpoint{1.205533in}{0.620418in}}%
\pgfpathlineto{\pgfqpoint{1.214426in}{0.619629in}}%
\pgfpathlineto{\pgfqpoint{1.225098in}{0.618717in}}%
\pgfpathlineto{\pgfqpoint{1.230434in}{0.618468in}}%
\pgfpathlineto{\pgfqpoint{1.235770in}{0.618264in}}%
\pgfpathlineto{\pgfqpoint{1.244663in}{0.617320in}}%
\pgfpathlineto{\pgfqpoint{1.264228in}{0.616684in}}%
\pgfpathlineto{\pgfqpoint{1.283793in}{0.615522in}}%
\pgfpathlineto{\pgfqpoint{1.324702in}{0.614033in}}%
\pgfpathlineto{\pgfqpoint{1.335374in}{0.614775in}}%
\pgfpathlineto{\pgfqpoint{1.344267in}{0.614270in}}%
\pgfpathlineto{\pgfqpoint{1.362054in}{0.613969in}}%
\pgfpathlineto{\pgfqpoint{1.369168in}{0.614125in}}%
\pgfpathlineto{\pgfqpoint{1.383397in}{0.613064in}}%
\pgfpathlineto{\pgfqpoint{1.586162in}{0.612295in}}%
\pgfpathlineto{\pgfqpoint{2.623109in}{0.611623in}}%
\pgfpathlineto{\pgfqpoint{2.623109in}{0.611623in}}%
\pgfusepath{stroke}%
\end{pgfscope}%
\begin{pgfscope}%
\pgfsetrectcap%
\pgfsetmiterjoin%
\pgfsetlinewidth{0.803000pt}%
\definecolor{currentstroke}{rgb}{0.000000,0.000000,0.000000}%
\pgfsetstrokecolor{currentstroke}%
\pgfsetdash{}{0pt}%
\pgfpathmoveto{\pgfqpoint{0.706736in}{0.538319in}}%
\pgfpathlineto{\pgfqpoint{0.706736in}{2.151000in}}%
\pgfusepath{stroke}%
\end{pgfscope}%
\begin{pgfscope}%
\pgfsetrectcap%
\pgfsetmiterjoin%
\pgfsetlinewidth{0.803000pt}%
\definecolor{currentstroke}{rgb}{0.000000,0.000000,0.000000}%
\pgfsetstrokecolor{currentstroke}%
\pgfsetdash{}{0pt}%
\pgfpathmoveto{\pgfqpoint{2.836878in}{0.538319in}}%
\pgfpathlineto{\pgfqpoint{2.836878in}{2.151000in}}%
\pgfusepath{stroke}%
\end{pgfscope}%
\begin{pgfscope}%
\pgfsetrectcap%
\pgfsetmiterjoin%
\pgfsetlinewidth{0.803000pt}%
\definecolor{currentstroke}{rgb}{0.000000,0.000000,0.000000}%
\pgfsetstrokecolor{currentstroke}%
\pgfsetdash{}{0pt}%
\pgfpathmoveto{\pgfqpoint{0.706736in}{0.538319in}}%
\pgfpathlineto{\pgfqpoint{2.836878in}{0.538319in}}%
\pgfusepath{stroke}%
\end{pgfscope}%
\begin{pgfscope}%
\pgfsetrectcap%
\pgfsetmiterjoin%
\pgfsetlinewidth{0.803000pt}%
\definecolor{currentstroke}{rgb}{0.000000,0.000000,0.000000}%
\pgfsetstrokecolor{currentstroke}%
\pgfsetdash{}{0pt}%
\pgfpathmoveto{\pgfqpoint{0.706736in}{2.151000in}}%
\pgfpathlineto{\pgfqpoint{2.836878in}{2.151000in}}%
\pgfusepath{stroke}%
\end{pgfscope}%
\begin{pgfscope}%
\definecolor{textcolor}{rgb}{0.000000,0.000000,0.000000}%
\pgfsetstrokecolor{textcolor}%
\pgfsetfillcolor{textcolor}%
\pgftext[x=0.706736in,y=2.234333in,left,base]{\color{textcolor}\rmfamily\fontsize{12.000000}{14.400000}\selectfont Event lengths}%
\end{pgfscope}%
\begin{pgfscope}%
\pgfsetbuttcap%
\pgfsetmiterjoin%
\definecolor{currentfill}{rgb}{1.000000,1.000000,1.000000}%
\pgfsetfillcolor{currentfill}%
\pgfsetfillopacity{0.800000}%
\pgfsetlinewidth{1.003750pt}%
\definecolor{currentstroke}{rgb}{0.800000,0.800000,0.800000}%
\pgfsetstrokecolor{currentstroke}%
\pgfsetstrokeopacity{0.800000}%
\pgfsetdash{}{0pt}%
\pgfpathmoveto{\pgfqpoint{1.243211in}{1.751889in}}%
\pgfpathlineto{\pgfqpoint{2.759101in}{1.751889in}}%
\pgfpathquadraticcurveto{\pgfqpoint{2.781323in}{1.751889in}}{\pgfqpoint{2.781323in}{1.774111in}}%
\pgfpathlineto{\pgfqpoint{2.781323in}{2.073222in}}%
\pgfpathquadraticcurveto{\pgfqpoint{2.781323in}{2.095444in}}{\pgfqpoint{2.759101in}{2.095444in}}%
\pgfpathlineto{\pgfqpoint{1.243211in}{2.095444in}}%
\pgfpathquadraticcurveto{\pgfqpoint{1.220989in}{2.095444in}}{\pgfqpoint{1.220989in}{2.073222in}}%
\pgfpathlineto{\pgfqpoint{1.220989in}{1.774111in}}%
\pgfpathquadraticcurveto{\pgfqpoint{1.220989in}{1.751889in}}{\pgfqpoint{1.243211in}{1.751889in}}%
\pgfpathclose%
\pgfusepath{stroke,fill}%
\end{pgfscope}%
\begin{pgfscope}%
\pgfsetrectcap%
\pgfsetroundjoin%
\pgfsetlinewidth{0.501875pt}%
\definecolor{currentstroke}{rgb}{0.313725,0.317647,0.309804}%
\pgfsetstrokecolor{currentstroke}%
\pgfsetdash{}{0pt}%
\pgfpathmoveto{\pgfqpoint{1.265434in}{2.012111in}}%
\pgfpathlineto{\pgfqpoint{1.487656in}{2.012111in}}%
\pgfusepath{stroke}%
\end{pgfscope}%
\begin{pgfscope}%
\definecolor{textcolor}{rgb}{0.000000,0.000000,0.000000}%
\pgfsetstrokecolor{textcolor}%
\pgfsetfillcolor{textcolor}%
\pgftext[x=1.576545in,y=1.973222in,left,base]{\color{textcolor}\rmfamily\fontsize{8.000000}{9.600000}\selectfont Raw, median=50.0}%
\end{pgfscope}%
\begin{pgfscope}%
\pgfsetbuttcap%
\pgfsetroundjoin%
\pgfsetlinewidth{0.501875pt}%
\definecolor{currentstroke}{rgb}{0.949020,0.372549,0.360784}%
\pgfsetstrokecolor{currentstroke}%
\pgfsetdash{{0.500000pt}{0.825000pt}}{0.000000pt}%
\pgfpathmoveto{\pgfqpoint{1.265434in}{1.856778in}}%
\pgfpathlineto{\pgfqpoint{1.487656in}{1.856778in}}%
\pgfusepath{stroke}%
\end{pgfscope}%
\begin{pgfscope}%
\definecolor{textcolor}{rgb}{0.000000,0.000000,0.000000}%
\pgfsetstrokecolor{textcolor}%
\pgfsetfillcolor{textcolor}%
\pgftext[x=1.576545in,y=1.817889in,left,base]{\color{textcolor}\rmfamily\fontsize{8.000000}{9.600000}\selectfont Cleaned, median=17.0}%
\end{pgfscope}%
\end{pgfpicture}%
\makeatother%
\endgroup%

         \caption{}\label{fig:event_length}
     \end{subfigure}
     \hfill
     \begin{subfigure}[b]{0.49\textwidth}
         \centering
         %% Creator: Matplotlib, PGF backend
%%
%% To include the figure in your LaTeX document, write
%%   \input{<filename>.pgf}
%%
%% Make sure the required packages are loaded in your preamble
%%   \usepackage{pgf}
%%
%% and, on pdftex
%%   \usepackage[utf8]{inputenc}\DeclareUnicodeCharacter{2212}{-}
%%
%% or, on luatex and xetex
%%   \usepackage{unicode-math}
%%
%% Figures using additional raster images can only be included by \input if
%% they are in the same directory as the main LaTeX file. For loading figures
%% from other directories you can use the `import` package
%%   \usepackage{import}
%%
%% and then include the figures with
%%   \import{<path to file>}{<filename>.pgf}
%%
%% Matplotlib used the following preamble
%%   \usepackage{siunitx} \usepackage{amsmath} \usepackage{bm}
%%   \usepackage{fontspec}
%%
\begingroup%
\makeatletter%
\begin{pgfpicture}%
\pgfpathrectangle{\pgfpointorigin}{\pgfqpoint{6.201200in}{3.000000in}}%
\pgfusepath{use as bounding box, clip}%
\begin{pgfscope}%
\pgfsetbuttcap%
\pgfsetmiterjoin%
\definecolor{currentfill}{rgb}{1.000000,1.000000,1.000000}%
\pgfsetfillcolor{currentfill}%
\pgfsetlinewidth{0.000000pt}%
\definecolor{currentstroke}{rgb}{1.000000,1.000000,1.000000}%
\pgfsetstrokecolor{currentstroke}%
\pgfsetdash{}{0pt}%
\pgfpathmoveto{\pgfqpoint{0.000000in}{0.000000in}}%
\pgfpathlineto{\pgfqpoint{6.201200in}{0.000000in}}%
\pgfpathlineto{\pgfqpoint{6.201200in}{3.000000in}}%
\pgfpathlineto{\pgfqpoint{0.000000in}{3.000000in}}%
\pgfpathclose%
\pgfusepath{fill}%
\end{pgfscope}%
\begin{pgfscope}%
\pgfsetbuttcap%
\pgfsetmiterjoin%
\definecolor{currentfill}{rgb}{1.000000,1.000000,1.000000}%
\pgfsetfillcolor{currentfill}%
\pgfsetlinewidth{0.000000pt}%
\definecolor{currentstroke}{rgb}{0.000000,0.000000,0.000000}%
\pgfsetstrokecolor{currentstroke}%
\pgfsetstrokeopacity{0.000000}%
\pgfsetdash{}{0pt}%
\pgfpathmoveto{\pgfqpoint{0.572918in}{0.553781in}}%
\pgfpathlineto{\pgfqpoint{6.051200in}{0.553781in}}%
\pgfpathlineto{\pgfqpoint{6.051200in}{2.649333in}}%
\pgfpathlineto{\pgfqpoint{0.572918in}{2.649333in}}%
\pgfpathclose%
\pgfusepath{fill}%
\end{pgfscope}%
\begin{pgfscope}%
\pgfpathrectangle{\pgfqpoint{0.572918in}{0.553781in}}{\pgfqpoint{5.478282in}{2.095553in}}%
\pgfusepath{clip}%
\pgfsetbuttcap%
\pgfsetroundjoin%
\pgfsetlinewidth{0.501875pt}%
\definecolor{currentstroke}{rgb}{0.690196,0.690196,0.690196}%
\pgfsetstrokecolor{currentstroke}%
\pgfsetstrokeopacity{0.500000}%
\pgfsetdash{{0.500000pt}{0.825000pt}}{0.000000pt}%
\pgfpathmoveto{\pgfqpoint{0.675453in}{0.553781in}}%
\pgfpathlineto{\pgfqpoint{0.675453in}{2.649333in}}%
\pgfusepath{stroke}%
\end{pgfscope}%
\begin{pgfscope}%
\pgfsetbuttcap%
\pgfsetroundjoin%
\definecolor{currentfill}{rgb}{0.000000,0.000000,0.000000}%
\pgfsetfillcolor{currentfill}%
\pgfsetlinewidth{0.803000pt}%
\definecolor{currentstroke}{rgb}{0.000000,0.000000,0.000000}%
\pgfsetstrokecolor{currentstroke}%
\pgfsetdash{}{0pt}%
\pgfsys@defobject{currentmarker}{\pgfqpoint{0.000000in}{-0.048611in}}{\pgfqpoint{0.000000in}{0.000000in}}{%
\pgfpathmoveto{\pgfqpoint{0.000000in}{0.000000in}}%
\pgfpathlineto{\pgfqpoint{0.000000in}{-0.048611in}}%
\pgfusepath{stroke,fill}%
}%
\begin{pgfscope}%
\pgfsys@transformshift{0.675453in}{0.553781in}%
\pgfsys@useobject{currentmarker}{}%
\end{pgfscope}%
\end{pgfscope}%
\begin{pgfscope}%
\definecolor{textcolor}{rgb}{0.000000,0.000000,0.000000}%
\pgfsetstrokecolor{textcolor}%
\pgfsetfillcolor{textcolor}%
\pgftext[x=0.675453in,y=0.456558in,,top]{\color{textcolor}\rmfamily\fontsize{8.000000}{9.600000}\selectfont \(\displaystyle {0.0}\)}%
\end{pgfscope}%
\begin{pgfscope}%
\pgfpathrectangle{\pgfqpoint{0.572918in}{0.553781in}}{\pgfqpoint{5.478282in}{2.095553in}}%
\pgfusepath{clip}%
\pgfsetbuttcap%
\pgfsetroundjoin%
\pgfsetlinewidth{0.501875pt}%
\definecolor{currentstroke}{rgb}{0.690196,0.690196,0.690196}%
\pgfsetstrokecolor{currentstroke}%
\pgfsetstrokeopacity{0.500000}%
\pgfsetdash{{0.500000pt}{0.825000pt}}{0.000000pt}%
\pgfpathmoveto{\pgfqpoint{1.554322in}{0.553781in}}%
\pgfpathlineto{\pgfqpoint{1.554322in}{2.649333in}}%
\pgfusepath{stroke}%
\end{pgfscope}%
\begin{pgfscope}%
\pgfsetbuttcap%
\pgfsetroundjoin%
\definecolor{currentfill}{rgb}{0.000000,0.000000,0.000000}%
\pgfsetfillcolor{currentfill}%
\pgfsetlinewidth{0.803000pt}%
\definecolor{currentstroke}{rgb}{0.000000,0.000000,0.000000}%
\pgfsetstrokecolor{currentstroke}%
\pgfsetdash{}{0pt}%
\pgfsys@defobject{currentmarker}{\pgfqpoint{0.000000in}{-0.048611in}}{\pgfqpoint{0.000000in}{0.000000in}}{%
\pgfpathmoveto{\pgfqpoint{0.000000in}{0.000000in}}%
\pgfpathlineto{\pgfqpoint{0.000000in}{-0.048611in}}%
\pgfusepath{stroke,fill}%
}%
\begin{pgfscope}%
\pgfsys@transformshift{1.554322in}{0.553781in}%
\pgfsys@useobject{currentmarker}{}%
\end{pgfscope}%
\end{pgfscope}%
\begin{pgfscope}%
\definecolor{textcolor}{rgb}{0.000000,0.000000,0.000000}%
\pgfsetstrokecolor{textcolor}%
\pgfsetfillcolor{textcolor}%
\pgftext[x=1.554322in,y=0.456558in,,top]{\color{textcolor}\rmfamily\fontsize{8.000000}{9.600000}\selectfont \(\displaystyle {0.5}\)}%
\end{pgfscope}%
\begin{pgfscope}%
\pgfpathrectangle{\pgfqpoint{0.572918in}{0.553781in}}{\pgfqpoint{5.478282in}{2.095553in}}%
\pgfusepath{clip}%
\pgfsetbuttcap%
\pgfsetroundjoin%
\pgfsetlinewidth{0.501875pt}%
\definecolor{currentstroke}{rgb}{0.690196,0.690196,0.690196}%
\pgfsetstrokecolor{currentstroke}%
\pgfsetstrokeopacity{0.500000}%
\pgfsetdash{{0.500000pt}{0.825000pt}}{0.000000pt}%
\pgfpathmoveto{\pgfqpoint{2.433190in}{0.553781in}}%
\pgfpathlineto{\pgfqpoint{2.433190in}{2.649333in}}%
\pgfusepath{stroke}%
\end{pgfscope}%
\begin{pgfscope}%
\pgfsetbuttcap%
\pgfsetroundjoin%
\definecolor{currentfill}{rgb}{0.000000,0.000000,0.000000}%
\pgfsetfillcolor{currentfill}%
\pgfsetlinewidth{0.803000pt}%
\definecolor{currentstroke}{rgb}{0.000000,0.000000,0.000000}%
\pgfsetstrokecolor{currentstroke}%
\pgfsetdash{}{0pt}%
\pgfsys@defobject{currentmarker}{\pgfqpoint{0.000000in}{-0.048611in}}{\pgfqpoint{0.000000in}{0.000000in}}{%
\pgfpathmoveto{\pgfqpoint{0.000000in}{0.000000in}}%
\pgfpathlineto{\pgfqpoint{0.000000in}{-0.048611in}}%
\pgfusepath{stroke,fill}%
}%
\begin{pgfscope}%
\pgfsys@transformshift{2.433190in}{0.553781in}%
\pgfsys@useobject{currentmarker}{}%
\end{pgfscope}%
\end{pgfscope}%
\begin{pgfscope}%
\definecolor{textcolor}{rgb}{0.000000,0.000000,0.000000}%
\pgfsetstrokecolor{textcolor}%
\pgfsetfillcolor{textcolor}%
\pgftext[x=2.433190in,y=0.456558in,,top]{\color{textcolor}\rmfamily\fontsize{8.000000}{9.600000}\selectfont \(\displaystyle {1.0}\)}%
\end{pgfscope}%
\begin{pgfscope}%
\pgfpathrectangle{\pgfqpoint{0.572918in}{0.553781in}}{\pgfqpoint{5.478282in}{2.095553in}}%
\pgfusepath{clip}%
\pgfsetbuttcap%
\pgfsetroundjoin%
\pgfsetlinewidth{0.501875pt}%
\definecolor{currentstroke}{rgb}{0.690196,0.690196,0.690196}%
\pgfsetstrokecolor{currentstroke}%
\pgfsetstrokeopacity{0.500000}%
\pgfsetdash{{0.500000pt}{0.825000pt}}{0.000000pt}%
\pgfpathmoveto{\pgfqpoint{3.312059in}{0.553781in}}%
\pgfpathlineto{\pgfqpoint{3.312059in}{2.649333in}}%
\pgfusepath{stroke}%
\end{pgfscope}%
\begin{pgfscope}%
\pgfsetbuttcap%
\pgfsetroundjoin%
\definecolor{currentfill}{rgb}{0.000000,0.000000,0.000000}%
\pgfsetfillcolor{currentfill}%
\pgfsetlinewidth{0.803000pt}%
\definecolor{currentstroke}{rgb}{0.000000,0.000000,0.000000}%
\pgfsetstrokecolor{currentstroke}%
\pgfsetdash{}{0pt}%
\pgfsys@defobject{currentmarker}{\pgfqpoint{0.000000in}{-0.048611in}}{\pgfqpoint{0.000000in}{0.000000in}}{%
\pgfpathmoveto{\pgfqpoint{0.000000in}{0.000000in}}%
\pgfpathlineto{\pgfqpoint{0.000000in}{-0.048611in}}%
\pgfusepath{stroke,fill}%
}%
\begin{pgfscope}%
\pgfsys@transformshift{3.312059in}{0.553781in}%
\pgfsys@useobject{currentmarker}{}%
\end{pgfscope}%
\end{pgfscope}%
\begin{pgfscope}%
\definecolor{textcolor}{rgb}{0.000000,0.000000,0.000000}%
\pgfsetstrokecolor{textcolor}%
\pgfsetfillcolor{textcolor}%
\pgftext[x=3.312059in,y=0.456558in,,top]{\color{textcolor}\rmfamily\fontsize{8.000000}{9.600000}\selectfont \(\displaystyle {1.5}\)}%
\end{pgfscope}%
\begin{pgfscope}%
\pgfpathrectangle{\pgfqpoint{0.572918in}{0.553781in}}{\pgfqpoint{5.478282in}{2.095553in}}%
\pgfusepath{clip}%
\pgfsetbuttcap%
\pgfsetroundjoin%
\pgfsetlinewidth{0.501875pt}%
\definecolor{currentstroke}{rgb}{0.690196,0.690196,0.690196}%
\pgfsetstrokecolor{currentstroke}%
\pgfsetstrokeopacity{0.500000}%
\pgfsetdash{{0.500000pt}{0.825000pt}}{0.000000pt}%
\pgfpathmoveto{\pgfqpoint{4.190928in}{0.553781in}}%
\pgfpathlineto{\pgfqpoint{4.190928in}{2.649333in}}%
\pgfusepath{stroke}%
\end{pgfscope}%
\begin{pgfscope}%
\pgfsetbuttcap%
\pgfsetroundjoin%
\definecolor{currentfill}{rgb}{0.000000,0.000000,0.000000}%
\pgfsetfillcolor{currentfill}%
\pgfsetlinewidth{0.803000pt}%
\definecolor{currentstroke}{rgb}{0.000000,0.000000,0.000000}%
\pgfsetstrokecolor{currentstroke}%
\pgfsetdash{}{0pt}%
\pgfsys@defobject{currentmarker}{\pgfqpoint{0.000000in}{-0.048611in}}{\pgfqpoint{0.000000in}{0.000000in}}{%
\pgfpathmoveto{\pgfqpoint{0.000000in}{0.000000in}}%
\pgfpathlineto{\pgfqpoint{0.000000in}{-0.048611in}}%
\pgfusepath{stroke,fill}%
}%
\begin{pgfscope}%
\pgfsys@transformshift{4.190928in}{0.553781in}%
\pgfsys@useobject{currentmarker}{}%
\end{pgfscope}%
\end{pgfscope}%
\begin{pgfscope}%
\definecolor{textcolor}{rgb}{0.000000,0.000000,0.000000}%
\pgfsetstrokecolor{textcolor}%
\pgfsetfillcolor{textcolor}%
\pgftext[x=4.190928in,y=0.456558in,,top]{\color{textcolor}\rmfamily\fontsize{8.000000}{9.600000}\selectfont \(\displaystyle {2.0}\)}%
\end{pgfscope}%
\begin{pgfscope}%
\pgfpathrectangle{\pgfqpoint{0.572918in}{0.553781in}}{\pgfqpoint{5.478282in}{2.095553in}}%
\pgfusepath{clip}%
\pgfsetbuttcap%
\pgfsetroundjoin%
\pgfsetlinewidth{0.501875pt}%
\definecolor{currentstroke}{rgb}{0.690196,0.690196,0.690196}%
\pgfsetstrokecolor{currentstroke}%
\pgfsetstrokeopacity{0.500000}%
\pgfsetdash{{0.500000pt}{0.825000pt}}{0.000000pt}%
\pgfpathmoveto{\pgfqpoint{5.069797in}{0.553781in}}%
\pgfpathlineto{\pgfqpoint{5.069797in}{2.649333in}}%
\pgfusepath{stroke}%
\end{pgfscope}%
\begin{pgfscope}%
\pgfsetbuttcap%
\pgfsetroundjoin%
\definecolor{currentfill}{rgb}{0.000000,0.000000,0.000000}%
\pgfsetfillcolor{currentfill}%
\pgfsetlinewidth{0.803000pt}%
\definecolor{currentstroke}{rgb}{0.000000,0.000000,0.000000}%
\pgfsetstrokecolor{currentstroke}%
\pgfsetdash{}{0pt}%
\pgfsys@defobject{currentmarker}{\pgfqpoint{0.000000in}{-0.048611in}}{\pgfqpoint{0.000000in}{0.000000in}}{%
\pgfpathmoveto{\pgfqpoint{0.000000in}{0.000000in}}%
\pgfpathlineto{\pgfqpoint{0.000000in}{-0.048611in}}%
\pgfusepath{stroke,fill}%
}%
\begin{pgfscope}%
\pgfsys@transformshift{5.069797in}{0.553781in}%
\pgfsys@useobject{currentmarker}{}%
\end{pgfscope}%
\end{pgfscope}%
\begin{pgfscope}%
\definecolor{textcolor}{rgb}{0.000000,0.000000,0.000000}%
\pgfsetstrokecolor{textcolor}%
\pgfsetfillcolor{textcolor}%
\pgftext[x=5.069797in,y=0.456558in,,top]{\color{textcolor}\rmfamily\fontsize{8.000000}{9.600000}\selectfont \(\displaystyle {2.5}\)}%
\end{pgfscope}%
\begin{pgfscope}%
\pgfpathrectangle{\pgfqpoint{0.572918in}{0.553781in}}{\pgfqpoint{5.478282in}{2.095553in}}%
\pgfusepath{clip}%
\pgfsetbuttcap%
\pgfsetroundjoin%
\pgfsetlinewidth{0.501875pt}%
\definecolor{currentstroke}{rgb}{0.690196,0.690196,0.690196}%
\pgfsetstrokecolor{currentstroke}%
\pgfsetstrokeopacity{0.500000}%
\pgfsetdash{{0.500000pt}{0.825000pt}}{0.000000pt}%
\pgfpathmoveto{\pgfqpoint{5.948665in}{0.553781in}}%
\pgfpathlineto{\pgfqpoint{5.948665in}{2.649333in}}%
\pgfusepath{stroke}%
\end{pgfscope}%
\begin{pgfscope}%
\pgfsetbuttcap%
\pgfsetroundjoin%
\definecolor{currentfill}{rgb}{0.000000,0.000000,0.000000}%
\pgfsetfillcolor{currentfill}%
\pgfsetlinewidth{0.803000pt}%
\definecolor{currentstroke}{rgb}{0.000000,0.000000,0.000000}%
\pgfsetstrokecolor{currentstroke}%
\pgfsetdash{}{0pt}%
\pgfsys@defobject{currentmarker}{\pgfqpoint{0.000000in}{-0.048611in}}{\pgfqpoint{0.000000in}{0.000000in}}{%
\pgfpathmoveto{\pgfqpoint{0.000000in}{0.000000in}}%
\pgfpathlineto{\pgfqpoint{0.000000in}{-0.048611in}}%
\pgfusepath{stroke,fill}%
}%
\begin{pgfscope}%
\pgfsys@transformshift{5.948665in}{0.553781in}%
\pgfsys@useobject{currentmarker}{}%
\end{pgfscope}%
\end{pgfscope}%
\begin{pgfscope}%
\definecolor{textcolor}{rgb}{0.000000,0.000000,0.000000}%
\pgfsetstrokecolor{textcolor}%
\pgfsetfillcolor{textcolor}%
\pgftext[x=5.948665in,y=0.456558in,,top]{\color{textcolor}\rmfamily\fontsize{8.000000}{9.600000}\selectfont \(\displaystyle {3.0}\)}%
\end{pgfscope}%
\begin{pgfscope}%
\definecolor{textcolor}{rgb}{0.000000,0.000000,0.000000}%
\pgfsetstrokecolor{textcolor}%
\pgfsetfillcolor{textcolor}%
\pgftext[x=3.312059in,y=0.302336in,,top]{\color{textcolor}\rmfamily\fontsize{10.950000}{13.140000}\selectfont \(\displaystyle \log_{10}(E_{\textup{true}}) \, \left[ E / \textup{GeV} \right]\)}%
\end{pgfscope}%
\begin{pgfscope}%
\pgfpathrectangle{\pgfqpoint{0.572918in}{0.553781in}}{\pgfqpoint{5.478282in}{2.095553in}}%
\pgfusepath{clip}%
\pgfsetbuttcap%
\pgfsetroundjoin%
\pgfsetlinewidth{0.501875pt}%
\definecolor{currentstroke}{rgb}{0.690196,0.690196,0.690196}%
\pgfsetstrokecolor{currentstroke}%
\pgfsetstrokeopacity{0.500000}%
\pgfsetdash{{0.500000pt}{0.825000pt}}{0.000000pt}%
\pgfpathmoveto{\pgfqpoint{0.572918in}{0.939927in}}%
\pgfpathlineto{\pgfqpoint{6.051200in}{0.939927in}}%
\pgfusepath{stroke}%
\end{pgfscope}%
\begin{pgfscope}%
\pgfsetbuttcap%
\pgfsetroundjoin%
\definecolor{currentfill}{rgb}{0.000000,0.000000,0.000000}%
\pgfsetfillcolor{currentfill}%
\pgfsetlinewidth{0.803000pt}%
\definecolor{currentstroke}{rgb}{0.000000,0.000000,0.000000}%
\pgfsetstrokecolor{currentstroke}%
\pgfsetdash{}{0pt}%
\pgfsys@defobject{currentmarker}{\pgfqpoint{-0.048611in}{0.000000in}}{\pgfqpoint{-0.000000in}{0.000000in}}{%
\pgfpathmoveto{\pgfqpoint{-0.000000in}{0.000000in}}%
\pgfpathlineto{\pgfqpoint{-0.048611in}{0.000000in}}%
\pgfusepath{stroke,fill}%
}%
\begin{pgfscope}%
\pgfsys@transformshift{0.572918in}{0.939927in}%
\pgfsys@useobject{currentmarker}{}%
\end{pgfscope}%
\end{pgfscope}%
\begin{pgfscope}%
\definecolor{textcolor}{rgb}{0.000000,0.000000,0.000000}%
\pgfsetstrokecolor{textcolor}%
\pgfsetfillcolor{textcolor}%
\pgftext[x=0.357639in, y=0.901372in, left, base]{\color{textcolor}\rmfamily\fontsize{8.000000}{9.600000}\selectfont \(\displaystyle {15}\)}%
\end{pgfscope}%
\begin{pgfscope}%
\pgfpathrectangle{\pgfqpoint{0.572918in}{0.553781in}}{\pgfqpoint{5.478282in}{2.095553in}}%
\pgfusepath{clip}%
\pgfsetbuttcap%
\pgfsetroundjoin%
\pgfsetlinewidth{0.501875pt}%
\definecolor{currentstroke}{rgb}{0.690196,0.690196,0.690196}%
\pgfsetstrokecolor{currentstroke}%
\pgfsetstrokeopacity{0.500000}%
\pgfsetdash{{0.500000pt}{0.825000pt}}{0.000000pt}%
\pgfpathmoveto{\pgfqpoint{0.572918in}{1.376825in}}%
\pgfpathlineto{\pgfqpoint{6.051200in}{1.376825in}}%
\pgfusepath{stroke}%
\end{pgfscope}%
\begin{pgfscope}%
\pgfsetbuttcap%
\pgfsetroundjoin%
\definecolor{currentfill}{rgb}{0.000000,0.000000,0.000000}%
\pgfsetfillcolor{currentfill}%
\pgfsetlinewidth{0.803000pt}%
\definecolor{currentstroke}{rgb}{0.000000,0.000000,0.000000}%
\pgfsetstrokecolor{currentstroke}%
\pgfsetdash{}{0pt}%
\pgfsys@defobject{currentmarker}{\pgfqpoint{-0.048611in}{0.000000in}}{\pgfqpoint{-0.000000in}{0.000000in}}{%
\pgfpathmoveto{\pgfqpoint{-0.000000in}{0.000000in}}%
\pgfpathlineto{\pgfqpoint{-0.048611in}{0.000000in}}%
\pgfusepath{stroke,fill}%
}%
\begin{pgfscope}%
\pgfsys@transformshift{0.572918in}{1.376825in}%
\pgfsys@useobject{currentmarker}{}%
\end{pgfscope}%
\end{pgfscope}%
\begin{pgfscope}%
\definecolor{textcolor}{rgb}{0.000000,0.000000,0.000000}%
\pgfsetstrokecolor{textcolor}%
\pgfsetfillcolor{textcolor}%
\pgftext[x=0.357639in, y=1.338269in, left, base]{\color{textcolor}\rmfamily\fontsize{8.000000}{9.600000}\selectfont \(\displaystyle {20}\)}%
\end{pgfscope}%
\begin{pgfscope}%
\pgfpathrectangle{\pgfqpoint{0.572918in}{0.553781in}}{\pgfqpoint{5.478282in}{2.095553in}}%
\pgfusepath{clip}%
\pgfsetbuttcap%
\pgfsetroundjoin%
\pgfsetlinewidth{0.501875pt}%
\definecolor{currentstroke}{rgb}{0.690196,0.690196,0.690196}%
\pgfsetstrokecolor{currentstroke}%
\pgfsetstrokeopacity{0.500000}%
\pgfsetdash{{0.500000pt}{0.825000pt}}{0.000000pt}%
\pgfpathmoveto{\pgfqpoint{0.572918in}{1.813723in}}%
\pgfpathlineto{\pgfqpoint{6.051200in}{1.813723in}}%
\pgfusepath{stroke}%
\end{pgfscope}%
\begin{pgfscope}%
\pgfsetbuttcap%
\pgfsetroundjoin%
\definecolor{currentfill}{rgb}{0.000000,0.000000,0.000000}%
\pgfsetfillcolor{currentfill}%
\pgfsetlinewidth{0.803000pt}%
\definecolor{currentstroke}{rgb}{0.000000,0.000000,0.000000}%
\pgfsetstrokecolor{currentstroke}%
\pgfsetdash{}{0pt}%
\pgfsys@defobject{currentmarker}{\pgfqpoint{-0.048611in}{0.000000in}}{\pgfqpoint{-0.000000in}{0.000000in}}{%
\pgfpathmoveto{\pgfqpoint{-0.000000in}{0.000000in}}%
\pgfpathlineto{\pgfqpoint{-0.048611in}{0.000000in}}%
\pgfusepath{stroke,fill}%
}%
\begin{pgfscope}%
\pgfsys@transformshift{0.572918in}{1.813723in}%
\pgfsys@useobject{currentmarker}{}%
\end{pgfscope}%
\end{pgfscope}%
\begin{pgfscope}%
\definecolor{textcolor}{rgb}{0.000000,0.000000,0.000000}%
\pgfsetstrokecolor{textcolor}%
\pgfsetfillcolor{textcolor}%
\pgftext[x=0.357639in, y=1.775167in, left, base]{\color{textcolor}\rmfamily\fontsize{8.000000}{9.600000}\selectfont \(\displaystyle {25}\)}%
\end{pgfscope}%
\begin{pgfscope}%
\pgfpathrectangle{\pgfqpoint{0.572918in}{0.553781in}}{\pgfqpoint{5.478282in}{2.095553in}}%
\pgfusepath{clip}%
\pgfsetbuttcap%
\pgfsetroundjoin%
\pgfsetlinewidth{0.501875pt}%
\definecolor{currentstroke}{rgb}{0.690196,0.690196,0.690196}%
\pgfsetstrokecolor{currentstroke}%
\pgfsetstrokeopacity{0.500000}%
\pgfsetdash{{0.500000pt}{0.825000pt}}{0.000000pt}%
\pgfpathmoveto{\pgfqpoint{0.572918in}{2.250620in}}%
\pgfpathlineto{\pgfqpoint{6.051200in}{2.250620in}}%
\pgfusepath{stroke}%
\end{pgfscope}%
\begin{pgfscope}%
\pgfsetbuttcap%
\pgfsetroundjoin%
\definecolor{currentfill}{rgb}{0.000000,0.000000,0.000000}%
\pgfsetfillcolor{currentfill}%
\pgfsetlinewidth{0.803000pt}%
\definecolor{currentstroke}{rgb}{0.000000,0.000000,0.000000}%
\pgfsetstrokecolor{currentstroke}%
\pgfsetdash{}{0pt}%
\pgfsys@defobject{currentmarker}{\pgfqpoint{-0.048611in}{0.000000in}}{\pgfqpoint{-0.000000in}{0.000000in}}{%
\pgfpathmoveto{\pgfqpoint{-0.000000in}{0.000000in}}%
\pgfpathlineto{\pgfqpoint{-0.048611in}{0.000000in}}%
\pgfusepath{stroke,fill}%
}%
\begin{pgfscope}%
\pgfsys@transformshift{0.572918in}{2.250620in}%
\pgfsys@useobject{currentmarker}{}%
\end{pgfscope}%
\end{pgfscope}%
\begin{pgfscope}%
\definecolor{textcolor}{rgb}{0.000000,0.000000,0.000000}%
\pgfsetstrokecolor{textcolor}%
\pgfsetfillcolor{textcolor}%
\pgftext[x=0.357639in, y=2.212065in, left, base]{\color{textcolor}\rmfamily\fontsize{8.000000}{9.600000}\selectfont \(\displaystyle {30}\)}%
\end{pgfscope}%
\begin{pgfscope}%
\definecolor{textcolor}{rgb}{0.000000,0.000000,0.000000}%
\pgfsetstrokecolor{textcolor}%
\pgfsetfillcolor{textcolor}%
\pgftext[x=0.302083in,y=1.601557in,,bottom,rotate=90.000000]{\color{textcolor}\rmfamily\fontsize{10.950000}{13.140000}\selectfont IQR / 1.349 \(\displaystyle \left[ \textup{deg} \right]\)}%
\end{pgfscope}%
\begin{pgfscope}%
\pgfpathrectangle{\pgfqpoint{0.572918in}{0.553781in}}{\pgfqpoint{5.478282in}{2.095553in}}%
\pgfusepath{clip}%
\pgfsetbuttcap%
\pgfsetroundjoin%
\pgfsetlinewidth{1.505625pt}%
\definecolor{currentstroke}{rgb}{0.313725,0.317647,0.309804}%
\pgfsetstrokecolor{currentstroke}%
\pgfsetstrokeopacity{0.900000}%
\pgfsetdash{}{0pt}%
\pgfpathmoveto{\pgfqpoint{0.821931in}{1.876502in}}%
\pgfpathlineto{\pgfqpoint{0.821931in}{2.530549in}}%
\pgfusepath{stroke}%
\end{pgfscope}%
\begin{pgfscope}%
\pgfpathrectangle{\pgfqpoint{0.572918in}{0.553781in}}{\pgfqpoint{5.478282in}{2.095553in}}%
\pgfusepath{clip}%
\pgfsetbuttcap%
\pgfsetroundjoin%
\pgfsetlinewidth{1.505625pt}%
\definecolor{currentstroke}{rgb}{0.313725,0.317647,0.309804}%
\pgfsetstrokecolor{currentstroke}%
\pgfsetstrokeopacity{0.900000}%
\pgfsetdash{}{0pt}%
\pgfpathmoveto{\pgfqpoint{1.114887in}{1.960242in}}%
\pgfpathlineto{\pgfqpoint{1.114887in}{2.241571in}}%
\pgfusepath{stroke}%
\end{pgfscope}%
\begin{pgfscope}%
\pgfpathrectangle{\pgfqpoint{0.572918in}{0.553781in}}{\pgfqpoint{5.478282in}{2.095553in}}%
\pgfusepath{clip}%
\pgfsetbuttcap%
\pgfsetroundjoin%
\pgfsetlinewidth{1.505625pt}%
\definecolor{currentstroke}{rgb}{0.313725,0.317647,0.309804}%
\pgfsetstrokecolor{currentstroke}%
\pgfsetstrokeopacity{0.900000}%
\pgfsetdash{}{0pt}%
\pgfpathmoveto{\pgfqpoint{1.407844in}{1.924596in}}%
\pgfpathlineto{\pgfqpoint{1.407844in}{2.103930in}}%
\pgfusepath{stroke}%
\end{pgfscope}%
\begin{pgfscope}%
\pgfpathrectangle{\pgfqpoint{0.572918in}{0.553781in}}{\pgfqpoint{5.478282in}{2.095553in}}%
\pgfusepath{clip}%
\pgfsetbuttcap%
\pgfsetroundjoin%
\pgfsetlinewidth{1.505625pt}%
\definecolor{currentstroke}{rgb}{0.313725,0.317647,0.309804}%
\pgfsetstrokecolor{currentstroke}%
\pgfsetstrokeopacity{0.900000}%
\pgfsetdash{}{0pt}%
\pgfpathmoveto{\pgfqpoint{1.700800in}{2.044423in}}%
\pgfpathlineto{\pgfqpoint{1.700800in}{2.146716in}}%
\pgfusepath{stroke}%
\end{pgfscope}%
\begin{pgfscope}%
\pgfpathrectangle{\pgfqpoint{0.572918in}{0.553781in}}{\pgfqpoint{5.478282in}{2.095553in}}%
\pgfusepath{clip}%
\pgfsetbuttcap%
\pgfsetroundjoin%
\pgfsetlinewidth{1.505625pt}%
\definecolor{currentstroke}{rgb}{0.313725,0.317647,0.309804}%
\pgfsetstrokecolor{currentstroke}%
\pgfsetstrokeopacity{0.900000}%
\pgfsetdash{}{0pt}%
\pgfpathmoveto{\pgfqpoint{1.993756in}{2.138952in}}%
\pgfpathlineto{\pgfqpoint{1.993756in}{2.230297in}}%
\pgfusepath{stroke}%
\end{pgfscope}%
\begin{pgfscope}%
\pgfpathrectangle{\pgfqpoint{0.572918in}{0.553781in}}{\pgfqpoint{5.478282in}{2.095553in}}%
\pgfusepath{clip}%
\pgfsetbuttcap%
\pgfsetroundjoin%
\pgfsetlinewidth{1.505625pt}%
\definecolor{currentstroke}{rgb}{0.313725,0.317647,0.309804}%
\pgfsetstrokecolor{currentstroke}%
\pgfsetstrokeopacity{0.900000}%
\pgfsetdash{}{0pt}%
\pgfpathmoveto{\pgfqpoint{2.286712in}{2.132477in}}%
\pgfpathlineto{\pgfqpoint{2.286712in}{2.209609in}}%
\pgfusepath{stroke}%
\end{pgfscope}%
\begin{pgfscope}%
\pgfpathrectangle{\pgfqpoint{0.572918in}{0.553781in}}{\pgfqpoint{5.478282in}{2.095553in}}%
\pgfusepath{clip}%
\pgfsetbuttcap%
\pgfsetroundjoin%
\pgfsetlinewidth{1.505625pt}%
\definecolor{currentstroke}{rgb}{0.313725,0.317647,0.309804}%
\pgfsetstrokecolor{currentstroke}%
\pgfsetstrokeopacity{0.900000}%
\pgfsetdash{}{0pt}%
\pgfpathmoveto{\pgfqpoint{2.579669in}{2.040047in}}%
\pgfpathlineto{\pgfqpoint{2.579669in}{2.103224in}}%
\pgfusepath{stroke}%
\end{pgfscope}%
\begin{pgfscope}%
\pgfpathrectangle{\pgfqpoint{0.572918in}{0.553781in}}{\pgfqpoint{5.478282in}{2.095553in}}%
\pgfusepath{clip}%
\pgfsetbuttcap%
\pgfsetroundjoin%
\pgfsetlinewidth{1.505625pt}%
\definecolor{currentstroke}{rgb}{0.313725,0.317647,0.309804}%
\pgfsetstrokecolor{currentstroke}%
\pgfsetstrokeopacity{0.900000}%
\pgfsetdash{}{0pt}%
\pgfpathmoveto{\pgfqpoint{2.872625in}{1.948060in}}%
\pgfpathlineto{\pgfqpoint{2.872625in}{2.013760in}}%
\pgfusepath{stroke}%
\end{pgfscope}%
\begin{pgfscope}%
\pgfpathrectangle{\pgfqpoint{0.572918in}{0.553781in}}{\pgfqpoint{5.478282in}{2.095553in}}%
\pgfusepath{clip}%
\pgfsetbuttcap%
\pgfsetroundjoin%
\pgfsetlinewidth{1.505625pt}%
\definecolor{currentstroke}{rgb}{0.313725,0.317647,0.309804}%
\pgfsetstrokecolor{currentstroke}%
\pgfsetstrokeopacity{0.900000}%
\pgfsetdash{}{0pt}%
\pgfpathmoveto{\pgfqpoint{3.165581in}{1.752391in}}%
\pgfpathlineto{\pgfqpoint{3.165581in}{1.810813in}}%
\pgfusepath{stroke}%
\end{pgfscope}%
\begin{pgfscope}%
\pgfpathrectangle{\pgfqpoint{0.572918in}{0.553781in}}{\pgfqpoint{5.478282in}{2.095553in}}%
\pgfusepath{clip}%
\pgfsetbuttcap%
\pgfsetroundjoin%
\pgfsetlinewidth{1.505625pt}%
\definecolor{currentstroke}{rgb}{0.313725,0.317647,0.309804}%
\pgfsetstrokecolor{currentstroke}%
\pgfsetstrokeopacity{0.900000}%
\pgfsetdash{}{0pt}%
\pgfpathmoveto{\pgfqpoint{3.458537in}{1.577617in}}%
\pgfpathlineto{\pgfqpoint{3.458537in}{1.635123in}}%
\pgfusepath{stroke}%
\end{pgfscope}%
\begin{pgfscope}%
\pgfpathrectangle{\pgfqpoint{0.572918in}{0.553781in}}{\pgfqpoint{5.478282in}{2.095553in}}%
\pgfusepath{clip}%
\pgfsetbuttcap%
\pgfsetroundjoin%
\pgfsetlinewidth{1.505625pt}%
\definecolor{currentstroke}{rgb}{0.313725,0.317647,0.309804}%
\pgfsetstrokecolor{currentstroke}%
\pgfsetstrokeopacity{0.900000}%
\pgfsetdash{}{0pt}%
\pgfpathmoveto{\pgfqpoint{3.751494in}{1.401942in}}%
\pgfpathlineto{\pgfqpoint{3.751494in}{1.463563in}}%
\pgfusepath{stroke}%
\end{pgfscope}%
\begin{pgfscope}%
\pgfpathrectangle{\pgfqpoint{0.572918in}{0.553781in}}{\pgfqpoint{5.478282in}{2.095553in}}%
\pgfusepath{clip}%
\pgfsetbuttcap%
\pgfsetroundjoin%
\pgfsetlinewidth{1.505625pt}%
\definecolor{currentstroke}{rgb}{0.313725,0.317647,0.309804}%
\pgfsetstrokecolor{currentstroke}%
\pgfsetstrokeopacity{0.900000}%
\pgfsetdash{}{0pt}%
\pgfpathmoveto{\pgfqpoint{4.044450in}{1.274577in}}%
\pgfpathlineto{\pgfqpoint{4.044450in}{1.337350in}}%
\pgfusepath{stroke}%
\end{pgfscope}%
\begin{pgfscope}%
\pgfpathrectangle{\pgfqpoint{0.572918in}{0.553781in}}{\pgfqpoint{5.478282in}{2.095553in}}%
\pgfusepath{clip}%
\pgfsetbuttcap%
\pgfsetroundjoin%
\pgfsetlinewidth{1.505625pt}%
\definecolor{currentstroke}{rgb}{0.313725,0.317647,0.309804}%
\pgfsetstrokecolor{currentstroke}%
\pgfsetstrokeopacity{0.900000}%
\pgfsetdash{}{0pt}%
\pgfpathmoveto{\pgfqpoint{4.337406in}{1.130452in}}%
\pgfpathlineto{\pgfqpoint{4.337406in}{1.199946in}}%
\pgfusepath{stroke}%
\end{pgfscope}%
\begin{pgfscope}%
\pgfpathrectangle{\pgfqpoint{0.572918in}{0.553781in}}{\pgfqpoint{5.478282in}{2.095553in}}%
\pgfusepath{clip}%
\pgfsetbuttcap%
\pgfsetroundjoin%
\pgfsetlinewidth{1.505625pt}%
\definecolor{currentstroke}{rgb}{0.313725,0.317647,0.309804}%
\pgfsetstrokecolor{currentstroke}%
\pgfsetstrokeopacity{0.900000}%
\pgfsetdash{}{0pt}%
\pgfpathmoveto{\pgfqpoint{4.630362in}{1.063926in}}%
\pgfpathlineto{\pgfqpoint{4.630362in}{1.148751in}}%
\pgfusepath{stroke}%
\end{pgfscope}%
\begin{pgfscope}%
\pgfpathrectangle{\pgfqpoint{0.572918in}{0.553781in}}{\pgfqpoint{5.478282in}{2.095553in}}%
\pgfusepath{clip}%
\pgfsetbuttcap%
\pgfsetroundjoin%
\pgfsetlinewidth{1.505625pt}%
\definecolor{currentstroke}{rgb}{0.313725,0.317647,0.309804}%
\pgfsetstrokecolor{currentstroke}%
\pgfsetstrokeopacity{0.900000}%
\pgfsetdash{}{0pt}%
\pgfpathmoveto{\pgfqpoint{4.923318in}{0.943440in}}%
\pgfpathlineto{\pgfqpoint{4.923318in}{1.032083in}}%
\pgfusepath{stroke}%
\end{pgfscope}%
\begin{pgfscope}%
\pgfpathrectangle{\pgfqpoint{0.572918in}{0.553781in}}{\pgfqpoint{5.478282in}{2.095553in}}%
\pgfusepath{clip}%
\pgfsetbuttcap%
\pgfsetroundjoin%
\pgfsetlinewidth{1.505625pt}%
\definecolor{currentstroke}{rgb}{0.313725,0.317647,0.309804}%
\pgfsetstrokecolor{currentstroke}%
\pgfsetstrokeopacity{0.900000}%
\pgfsetdash{}{0pt}%
\pgfpathmoveto{\pgfqpoint{5.216275in}{0.982409in}}%
\pgfpathlineto{\pgfqpoint{5.216275in}{1.106934in}}%
\pgfusepath{stroke}%
\end{pgfscope}%
\begin{pgfscope}%
\pgfpathrectangle{\pgfqpoint{0.572918in}{0.553781in}}{\pgfqpoint{5.478282in}{2.095553in}}%
\pgfusepath{clip}%
\pgfsetbuttcap%
\pgfsetroundjoin%
\pgfsetlinewidth{1.505625pt}%
\definecolor{currentstroke}{rgb}{0.313725,0.317647,0.309804}%
\pgfsetstrokecolor{currentstroke}%
\pgfsetstrokeopacity{0.900000}%
\pgfsetdash{}{0pt}%
\pgfpathmoveto{\pgfqpoint{5.509231in}{0.817690in}}%
\pgfpathlineto{\pgfqpoint{5.509231in}{0.973846in}}%
\pgfusepath{stroke}%
\end{pgfscope}%
\begin{pgfscope}%
\pgfpathrectangle{\pgfqpoint{0.572918in}{0.553781in}}{\pgfqpoint{5.478282in}{2.095553in}}%
\pgfusepath{clip}%
\pgfsetbuttcap%
\pgfsetroundjoin%
\pgfsetlinewidth{1.505625pt}%
\definecolor{currentstroke}{rgb}{0.313725,0.317647,0.309804}%
\pgfsetstrokecolor{currentstroke}%
\pgfsetstrokeopacity{0.900000}%
\pgfsetdash{}{0pt}%
\pgfpathmoveto{\pgfqpoint{5.802187in}{0.830056in}}%
\pgfpathlineto{\pgfqpoint{5.802187in}{1.053850in}}%
\pgfusepath{stroke}%
\end{pgfscope}%
\begin{pgfscope}%
\pgfpathrectangle{\pgfqpoint{0.572918in}{0.553781in}}{\pgfqpoint{5.478282in}{2.095553in}}%
\pgfusepath{clip}%
\pgfsetbuttcap%
\pgfsetroundjoin%
\pgfsetlinewidth{1.505625pt}%
\definecolor{currentstroke}{rgb}{0.949020,0.372549,0.360784}%
\pgfsetstrokecolor{currentstroke}%
\pgfsetstrokeopacity{0.900000}%
\pgfsetdash{}{0pt}%
\pgfpathmoveto{\pgfqpoint{0.821931in}{1.865882in}}%
\pgfpathlineto{\pgfqpoint{0.821931in}{2.554081in}}%
\pgfusepath{stroke}%
\end{pgfscope}%
\begin{pgfscope}%
\pgfpathrectangle{\pgfqpoint{0.572918in}{0.553781in}}{\pgfqpoint{5.478282in}{2.095553in}}%
\pgfusepath{clip}%
\pgfsetbuttcap%
\pgfsetroundjoin%
\pgfsetlinewidth{1.505625pt}%
\definecolor{currentstroke}{rgb}{0.949020,0.372549,0.360784}%
\pgfsetstrokecolor{currentstroke}%
\pgfsetstrokeopacity{0.900000}%
\pgfsetdash{}{0pt}%
\pgfpathmoveto{\pgfqpoint{1.114887in}{1.976678in}}%
\pgfpathlineto{\pgfqpoint{1.114887in}{2.243386in}}%
\pgfusepath{stroke}%
\end{pgfscope}%
\begin{pgfscope}%
\pgfpathrectangle{\pgfqpoint{0.572918in}{0.553781in}}{\pgfqpoint{5.478282in}{2.095553in}}%
\pgfusepath{clip}%
\pgfsetbuttcap%
\pgfsetroundjoin%
\pgfsetlinewidth{1.505625pt}%
\definecolor{currentstroke}{rgb}{0.949020,0.372549,0.360784}%
\pgfsetstrokecolor{currentstroke}%
\pgfsetstrokeopacity{0.900000}%
\pgfsetdash{}{0pt}%
\pgfpathmoveto{\pgfqpoint{1.407844in}{1.873765in}}%
\pgfpathlineto{\pgfqpoint{1.407844in}{2.025647in}}%
\pgfusepath{stroke}%
\end{pgfscope}%
\begin{pgfscope}%
\pgfpathrectangle{\pgfqpoint{0.572918in}{0.553781in}}{\pgfqpoint{5.478282in}{2.095553in}}%
\pgfusepath{clip}%
\pgfsetbuttcap%
\pgfsetroundjoin%
\pgfsetlinewidth{1.505625pt}%
\definecolor{currentstroke}{rgb}{0.949020,0.372549,0.360784}%
\pgfsetstrokecolor{currentstroke}%
\pgfsetstrokeopacity{0.900000}%
\pgfsetdash{}{0pt}%
\pgfpathmoveto{\pgfqpoint{1.700800in}{1.900015in}}%
\pgfpathlineto{\pgfqpoint{1.700800in}{2.006778in}}%
\pgfusepath{stroke}%
\end{pgfscope}%
\begin{pgfscope}%
\pgfpathrectangle{\pgfqpoint{0.572918in}{0.553781in}}{\pgfqpoint{5.478282in}{2.095553in}}%
\pgfusepath{clip}%
\pgfsetbuttcap%
\pgfsetroundjoin%
\pgfsetlinewidth{1.505625pt}%
\definecolor{currentstroke}{rgb}{0.949020,0.372549,0.360784}%
\pgfsetstrokecolor{currentstroke}%
\pgfsetstrokeopacity{0.900000}%
\pgfsetdash{}{0pt}%
\pgfpathmoveto{\pgfqpoint{1.993756in}{1.990986in}}%
\pgfpathlineto{\pgfqpoint{1.993756in}{2.077835in}}%
\pgfusepath{stroke}%
\end{pgfscope}%
\begin{pgfscope}%
\pgfpathrectangle{\pgfqpoint{0.572918in}{0.553781in}}{\pgfqpoint{5.478282in}{2.095553in}}%
\pgfusepath{clip}%
\pgfsetbuttcap%
\pgfsetroundjoin%
\pgfsetlinewidth{1.505625pt}%
\definecolor{currentstroke}{rgb}{0.949020,0.372549,0.360784}%
\pgfsetstrokecolor{currentstroke}%
\pgfsetstrokeopacity{0.900000}%
\pgfsetdash{}{0pt}%
\pgfpathmoveto{\pgfqpoint{2.286712in}{1.963804in}}%
\pgfpathlineto{\pgfqpoint{2.286712in}{2.027336in}}%
\pgfusepath{stroke}%
\end{pgfscope}%
\begin{pgfscope}%
\pgfpathrectangle{\pgfqpoint{0.572918in}{0.553781in}}{\pgfqpoint{5.478282in}{2.095553in}}%
\pgfusepath{clip}%
\pgfsetbuttcap%
\pgfsetroundjoin%
\pgfsetlinewidth{1.505625pt}%
\definecolor{currentstroke}{rgb}{0.949020,0.372549,0.360784}%
\pgfsetstrokecolor{currentstroke}%
\pgfsetstrokeopacity{0.900000}%
\pgfsetdash{}{0pt}%
\pgfpathmoveto{\pgfqpoint{2.579669in}{1.853103in}}%
\pgfpathlineto{\pgfqpoint{2.579669in}{1.915874in}}%
\pgfusepath{stroke}%
\end{pgfscope}%
\begin{pgfscope}%
\pgfpathrectangle{\pgfqpoint{0.572918in}{0.553781in}}{\pgfqpoint{5.478282in}{2.095553in}}%
\pgfusepath{clip}%
\pgfsetbuttcap%
\pgfsetroundjoin%
\pgfsetlinewidth{1.505625pt}%
\definecolor{currentstroke}{rgb}{0.949020,0.372549,0.360784}%
\pgfsetstrokecolor{currentstroke}%
\pgfsetstrokeopacity{0.900000}%
\pgfsetdash{}{0pt}%
\pgfpathmoveto{\pgfqpoint{2.872625in}{1.772269in}}%
\pgfpathlineto{\pgfqpoint{2.872625in}{1.829583in}}%
\pgfusepath{stroke}%
\end{pgfscope}%
\begin{pgfscope}%
\pgfpathrectangle{\pgfqpoint{0.572918in}{0.553781in}}{\pgfqpoint{5.478282in}{2.095553in}}%
\pgfusepath{clip}%
\pgfsetbuttcap%
\pgfsetroundjoin%
\pgfsetlinewidth{1.505625pt}%
\definecolor{currentstroke}{rgb}{0.949020,0.372549,0.360784}%
\pgfsetstrokecolor{currentstroke}%
\pgfsetstrokeopacity{0.900000}%
\pgfsetdash{}{0pt}%
\pgfpathmoveto{\pgfqpoint{3.165581in}{1.594924in}}%
\pgfpathlineto{\pgfqpoint{3.165581in}{1.654281in}}%
\pgfusepath{stroke}%
\end{pgfscope}%
\begin{pgfscope}%
\pgfpathrectangle{\pgfqpoint{0.572918in}{0.553781in}}{\pgfqpoint{5.478282in}{2.095553in}}%
\pgfusepath{clip}%
\pgfsetbuttcap%
\pgfsetroundjoin%
\pgfsetlinewidth{1.505625pt}%
\definecolor{currentstroke}{rgb}{0.949020,0.372549,0.360784}%
\pgfsetstrokecolor{currentstroke}%
\pgfsetstrokeopacity{0.900000}%
\pgfsetdash{}{0pt}%
\pgfpathmoveto{\pgfqpoint{3.458537in}{1.449489in}}%
\pgfpathlineto{\pgfqpoint{3.458537in}{1.504971in}}%
\pgfusepath{stroke}%
\end{pgfscope}%
\begin{pgfscope}%
\pgfpathrectangle{\pgfqpoint{0.572918in}{0.553781in}}{\pgfqpoint{5.478282in}{2.095553in}}%
\pgfusepath{clip}%
\pgfsetbuttcap%
\pgfsetroundjoin%
\pgfsetlinewidth{1.505625pt}%
\definecolor{currentstroke}{rgb}{0.949020,0.372549,0.360784}%
\pgfsetstrokecolor{currentstroke}%
\pgfsetstrokeopacity{0.900000}%
\pgfsetdash{}{0pt}%
\pgfpathmoveto{\pgfqpoint{3.751494in}{1.325946in}}%
\pgfpathlineto{\pgfqpoint{3.751494in}{1.378416in}}%
\pgfusepath{stroke}%
\end{pgfscope}%
\begin{pgfscope}%
\pgfpathrectangle{\pgfqpoint{0.572918in}{0.553781in}}{\pgfqpoint{5.478282in}{2.095553in}}%
\pgfusepath{clip}%
\pgfsetbuttcap%
\pgfsetroundjoin%
\pgfsetlinewidth{1.505625pt}%
\definecolor{currentstroke}{rgb}{0.949020,0.372549,0.360784}%
\pgfsetstrokecolor{currentstroke}%
\pgfsetstrokeopacity{0.900000}%
\pgfsetdash{}{0pt}%
\pgfpathmoveto{\pgfqpoint{4.044450in}{1.173028in}}%
\pgfpathlineto{\pgfqpoint{4.044450in}{1.236022in}}%
\pgfusepath{stroke}%
\end{pgfscope}%
\begin{pgfscope}%
\pgfpathrectangle{\pgfqpoint{0.572918in}{0.553781in}}{\pgfqpoint{5.478282in}{2.095553in}}%
\pgfusepath{clip}%
\pgfsetbuttcap%
\pgfsetroundjoin%
\pgfsetlinewidth{1.505625pt}%
\definecolor{currentstroke}{rgb}{0.949020,0.372549,0.360784}%
\pgfsetstrokecolor{currentstroke}%
\pgfsetstrokeopacity{0.900000}%
\pgfsetdash{}{0pt}%
\pgfpathmoveto{\pgfqpoint{4.337406in}{1.031593in}}%
\pgfpathlineto{\pgfqpoint{4.337406in}{1.096358in}}%
\pgfusepath{stroke}%
\end{pgfscope}%
\begin{pgfscope}%
\pgfpathrectangle{\pgfqpoint{0.572918in}{0.553781in}}{\pgfqpoint{5.478282in}{2.095553in}}%
\pgfusepath{clip}%
\pgfsetbuttcap%
\pgfsetroundjoin%
\pgfsetlinewidth{1.505625pt}%
\definecolor{currentstroke}{rgb}{0.949020,0.372549,0.360784}%
\pgfsetstrokecolor{currentstroke}%
\pgfsetstrokeopacity{0.900000}%
\pgfsetdash{}{0pt}%
\pgfpathmoveto{\pgfqpoint{4.630362in}{0.873758in}}%
\pgfpathlineto{\pgfqpoint{4.630362in}{0.948242in}}%
\pgfusepath{stroke}%
\end{pgfscope}%
\begin{pgfscope}%
\pgfpathrectangle{\pgfqpoint{0.572918in}{0.553781in}}{\pgfqpoint{5.478282in}{2.095553in}}%
\pgfusepath{clip}%
\pgfsetbuttcap%
\pgfsetroundjoin%
\pgfsetlinewidth{1.505625pt}%
\definecolor{currentstroke}{rgb}{0.949020,0.372549,0.360784}%
\pgfsetstrokecolor{currentstroke}%
\pgfsetstrokeopacity{0.900000}%
\pgfsetdash{}{0pt}%
\pgfpathmoveto{\pgfqpoint{4.923318in}{0.806085in}}%
\pgfpathlineto{\pgfqpoint{4.923318in}{0.891131in}}%
\pgfusepath{stroke}%
\end{pgfscope}%
\begin{pgfscope}%
\pgfpathrectangle{\pgfqpoint{0.572918in}{0.553781in}}{\pgfqpoint{5.478282in}{2.095553in}}%
\pgfusepath{clip}%
\pgfsetbuttcap%
\pgfsetroundjoin%
\pgfsetlinewidth{1.505625pt}%
\definecolor{currentstroke}{rgb}{0.949020,0.372549,0.360784}%
\pgfsetstrokecolor{currentstroke}%
\pgfsetstrokeopacity{0.900000}%
\pgfsetdash{}{0pt}%
\pgfpathmoveto{\pgfqpoint{5.216275in}{0.766272in}}%
\pgfpathlineto{\pgfqpoint{5.216275in}{0.904551in}}%
\pgfusepath{stroke}%
\end{pgfscope}%
\begin{pgfscope}%
\pgfpathrectangle{\pgfqpoint{0.572918in}{0.553781in}}{\pgfqpoint{5.478282in}{2.095553in}}%
\pgfusepath{clip}%
\pgfsetbuttcap%
\pgfsetroundjoin%
\pgfsetlinewidth{1.505625pt}%
\definecolor{currentstroke}{rgb}{0.949020,0.372549,0.360784}%
\pgfsetstrokecolor{currentstroke}%
\pgfsetstrokeopacity{0.900000}%
\pgfsetdash{}{0pt}%
\pgfpathmoveto{\pgfqpoint{5.509231in}{0.738161in}}%
\pgfpathlineto{\pgfqpoint{5.509231in}{0.884185in}}%
\pgfusepath{stroke}%
\end{pgfscope}%
\begin{pgfscope}%
\pgfpathrectangle{\pgfqpoint{0.572918in}{0.553781in}}{\pgfqpoint{5.478282in}{2.095553in}}%
\pgfusepath{clip}%
\pgfsetbuttcap%
\pgfsetroundjoin%
\pgfsetlinewidth{1.505625pt}%
\definecolor{currentstroke}{rgb}{0.949020,0.372549,0.360784}%
\pgfsetstrokecolor{currentstroke}%
\pgfsetstrokeopacity{0.900000}%
\pgfsetdash{}{0pt}%
\pgfpathmoveto{\pgfqpoint{5.802187in}{0.856135in}}%
\pgfpathlineto{\pgfqpoint{5.802187in}{1.050405in}}%
\pgfusepath{stroke}%
\end{pgfscope}%
\begin{pgfscope}%
\pgfpathrectangle{\pgfqpoint{0.572918in}{0.553781in}}{\pgfqpoint{5.478282in}{2.095553in}}%
\pgfusepath{clip}%
\pgfsetbuttcap%
\pgfsetroundjoin%
\pgfsetlinewidth{1.505625pt}%
\definecolor{currentstroke}{rgb}{1.000000,0.819608,0.101961}%
\pgfsetstrokecolor{currentstroke}%
\pgfsetstrokeopacity{0.900000}%
\pgfsetdash{}{0pt}%
\pgfpathmoveto{\pgfqpoint{0.821931in}{1.933177in}}%
\pgfpathlineto{\pgfqpoint{0.821931in}{2.464097in}}%
\pgfusepath{stroke}%
\end{pgfscope}%
\begin{pgfscope}%
\pgfpathrectangle{\pgfqpoint{0.572918in}{0.553781in}}{\pgfqpoint{5.478282in}{2.095553in}}%
\pgfusepath{clip}%
\pgfsetbuttcap%
\pgfsetroundjoin%
\pgfsetlinewidth{1.505625pt}%
\definecolor{currentstroke}{rgb}{1.000000,0.819608,0.101961}%
\pgfsetstrokecolor{currentstroke}%
\pgfsetstrokeopacity{0.900000}%
\pgfsetdash{}{0pt}%
\pgfpathmoveto{\pgfqpoint{1.114887in}{1.845694in}}%
\pgfpathlineto{\pgfqpoint{1.114887in}{2.119473in}}%
\pgfusepath{stroke}%
\end{pgfscope}%
\begin{pgfscope}%
\pgfpathrectangle{\pgfqpoint{0.572918in}{0.553781in}}{\pgfqpoint{5.478282in}{2.095553in}}%
\pgfusepath{clip}%
\pgfsetbuttcap%
\pgfsetroundjoin%
\pgfsetlinewidth{1.505625pt}%
\definecolor{currentstroke}{rgb}{1.000000,0.819608,0.101961}%
\pgfsetstrokecolor{currentstroke}%
\pgfsetstrokeopacity{0.900000}%
\pgfsetdash{}{0pt}%
\pgfpathmoveto{\pgfqpoint{1.407844in}{1.857270in}}%
\pgfpathlineto{\pgfqpoint{1.407844in}{1.996922in}}%
\pgfusepath{stroke}%
\end{pgfscope}%
\begin{pgfscope}%
\pgfpathrectangle{\pgfqpoint{0.572918in}{0.553781in}}{\pgfqpoint{5.478282in}{2.095553in}}%
\pgfusepath{clip}%
\pgfsetbuttcap%
\pgfsetroundjoin%
\pgfsetlinewidth{1.505625pt}%
\definecolor{currentstroke}{rgb}{1.000000,0.819608,0.101961}%
\pgfsetstrokecolor{currentstroke}%
\pgfsetstrokeopacity{0.900000}%
\pgfsetdash{}{0pt}%
\pgfpathmoveto{\pgfqpoint{1.700800in}{1.927427in}}%
\pgfpathlineto{\pgfqpoint{1.700800in}{2.029553in}}%
\pgfusepath{stroke}%
\end{pgfscope}%
\begin{pgfscope}%
\pgfpathrectangle{\pgfqpoint{0.572918in}{0.553781in}}{\pgfqpoint{5.478282in}{2.095553in}}%
\pgfusepath{clip}%
\pgfsetbuttcap%
\pgfsetroundjoin%
\pgfsetlinewidth{1.505625pt}%
\definecolor{currentstroke}{rgb}{1.000000,0.819608,0.101961}%
\pgfsetstrokecolor{currentstroke}%
\pgfsetstrokeopacity{0.900000}%
\pgfsetdash{}{0pt}%
\pgfpathmoveto{\pgfqpoint{1.993756in}{1.966691in}}%
\pgfpathlineto{\pgfqpoint{1.993756in}{2.056628in}}%
\pgfusepath{stroke}%
\end{pgfscope}%
\begin{pgfscope}%
\pgfpathrectangle{\pgfqpoint{0.572918in}{0.553781in}}{\pgfqpoint{5.478282in}{2.095553in}}%
\pgfusepath{clip}%
\pgfsetbuttcap%
\pgfsetroundjoin%
\pgfsetlinewidth{1.505625pt}%
\definecolor{currentstroke}{rgb}{1.000000,0.819608,0.101961}%
\pgfsetstrokecolor{currentstroke}%
\pgfsetstrokeopacity{0.900000}%
\pgfsetdash{}{0pt}%
\pgfpathmoveto{\pgfqpoint{2.286712in}{1.949186in}}%
\pgfpathlineto{\pgfqpoint{2.286712in}{2.021969in}}%
\pgfusepath{stroke}%
\end{pgfscope}%
\begin{pgfscope}%
\pgfpathrectangle{\pgfqpoint{0.572918in}{0.553781in}}{\pgfqpoint{5.478282in}{2.095553in}}%
\pgfusepath{clip}%
\pgfsetbuttcap%
\pgfsetroundjoin%
\pgfsetlinewidth{1.505625pt}%
\definecolor{currentstroke}{rgb}{1.000000,0.819608,0.101961}%
\pgfsetstrokecolor{currentstroke}%
\pgfsetstrokeopacity{0.900000}%
\pgfsetdash{}{0pt}%
\pgfpathmoveto{\pgfqpoint{2.579669in}{1.836146in}}%
\pgfpathlineto{\pgfqpoint{2.579669in}{1.897323in}}%
\pgfusepath{stroke}%
\end{pgfscope}%
\begin{pgfscope}%
\pgfpathrectangle{\pgfqpoint{0.572918in}{0.553781in}}{\pgfqpoint{5.478282in}{2.095553in}}%
\pgfusepath{clip}%
\pgfsetbuttcap%
\pgfsetroundjoin%
\pgfsetlinewidth{1.505625pt}%
\definecolor{currentstroke}{rgb}{1.000000,0.819608,0.101961}%
\pgfsetstrokecolor{currentstroke}%
\pgfsetstrokeopacity{0.900000}%
\pgfsetdash{}{0pt}%
\pgfpathmoveto{\pgfqpoint{2.872625in}{1.781188in}}%
\pgfpathlineto{\pgfqpoint{2.872625in}{1.841292in}}%
\pgfusepath{stroke}%
\end{pgfscope}%
\begin{pgfscope}%
\pgfpathrectangle{\pgfqpoint{0.572918in}{0.553781in}}{\pgfqpoint{5.478282in}{2.095553in}}%
\pgfusepath{clip}%
\pgfsetbuttcap%
\pgfsetroundjoin%
\pgfsetlinewidth{1.505625pt}%
\definecolor{currentstroke}{rgb}{1.000000,0.819608,0.101961}%
\pgfsetstrokecolor{currentstroke}%
\pgfsetstrokeopacity{0.900000}%
\pgfsetdash{}{0pt}%
\pgfpathmoveto{\pgfqpoint{3.165581in}{1.583574in}}%
\pgfpathlineto{\pgfqpoint{3.165581in}{1.640033in}}%
\pgfusepath{stroke}%
\end{pgfscope}%
\begin{pgfscope}%
\pgfpathrectangle{\pgfqpoint{0.572918in}{0.553781in}}{\pgfqpoint{5.478282in}{2.095553in}}%
\pgfusepath{clip}%
\pgfsetbuttcap%
\pgfsetroundjoin%
\pgfsetlinewidth{1.505625pt}%
\definecolor{currentstroke}{rgb}{1.000000,0.819608,0.101961}%
\pgfsetstrokecolor{currentstroke}%
\pgfsetstrokeopacity{0.900000}%
\pgfsetdash{}{0pt}%
\pgfpathmoveto{\pgfqpoint{3.458537in}{1.437153in}}%
\pgfpathlineto{\pgfqpoint{3.458537in}{1.488326in}}%
\pgfusepath{stroke}%
\end{pgfscope}%
\begin{pgfscope}%
\pgfpathrectangle{\pgfqpoint{0.572918in}{0.553781in}}{\pgfqpoint{5.478282in}{2.095553in}}%
\pgfusepath{clip}%
\pgfsetbuttcap%
\pgfsetroundjoin%
\pgfsetlinewidth{1.505625pt}%
\definecolor{currentstroke}{rgb}{1.000000,0.819608,0.101961}%
\pgfsetstrokecolor{currentstroke}%
\pgfsetstrokeopacity{0.900000}%
\pgfsetdash{}{0pt}%
\pgfpathmoveto{\pgfqpoint{3.751494in}{1.310464in}}%
\pgfpathlineto{\pgfqpoint{3.751494in}{1.366965in}}%
\pgfusepath{stroke}%
\end{pgfscope}%
\begin{pgfscope}%
\pgfpathrectangle{\pgfqpoint{0.572918in}{0.553781in}}{\pgfqpoint{5.478282in}{2.095553in}}%
\pgfusepath{clip}%
\pgfsetbuttcap%
\pgfsetroundjoin%
\pgfsetlinewidth{1.505625pt}%
\definecolor{currentstroke}{rgb}{1.000000,0.819608,0.101961}%
\pgfsetstrokecolor{currentstroke}%
\pgfsetstrokeopacity{0.900000}%
\pgfsetdash{}{0pt}%
\pgfpathmoveto{\pgfqpoint{4.044450in}{1.190002in}}%
\pgfpathlineto{\pgfqpoint{4.044450in}{1.252633in}}%
\pgfusepath{stroke}%
\end{pgfscope}%
\begin{pgfscope}%
\pgfpathrectangle{\pgfqpoint{0.572918in}{0.553781in}}{\pgfqpoint{5.478282in}{2.095553in}}%
\pgfusepath{clip}%
\pgfsetbuttcap%
\pgfsetroundjoin%
\pgfsetlinewidth{1.505625pt}%
\definecolor{currentstroke}{rgb}{1.000000,0.819608,0.101961}%
\pgfsetstrokecolor{currentstroke}%
\pgfsetstrokeopacity{0.900000}%
\pgfsetdash{}{0pt}%
\pgfpathmoveto{\pgfqpoint{4.337406in}{1.063963in}}%
\pgfpathlineto{\pgfqpoint{4.337406in}{1.133046in}}%
\pgfusepath{stroke}%
\end{pgfscope}%
\begin{pgfscope}%
\pgfpathrectangle{\pgfqpoint{0.572918in}{0.553781in}}{\pgfqpoint{5.478282in}{2.095553in}}%
\pgfusepath{clip}%
\pgfsetbuttcap%
\pgfsetroundjoin%
\pgfsetlinewidth{1.505625pt}%
\definecolor{currentstroke}{rgb}{1.000000,0.819608,0.101961}%
\pgfsetstrokecolor{currentstroke}%
\pgfsetstrokeopacity{0.900000}%
\pgfsetdash{}{0pt}%
\pgfpathmoveto{\pgfqpoint{4.630362in}{0.951248in}}%
\pgfpathlineto{\pgfqpoint{4.630362in}{1.030747in}}%
\pgfusepath{stroke}%
\end{pgfscope}%
\begin{pgfscope}%
\pgfpathrectangle{\pgfqpoint{0.572918in}{0.553781in}}{\pgfqpoint{5.478282in}{2.095553in}}%
\pgfusepath{clip}%
\pgfsetbuttcap%
\pgfsetroundjoin%
\pgfsetlinewidth{1.505625pt}%
\definecolor{currentstroke}{rgb}{1.000000,0.819608,0.101961}%
\pgfsetstrokecolor{currentstroke}%
\pgfsetstrokeopacity{0.900000}%
\pgfsetdash{}{0pt}%
\pgfpathmoveto{\pgfqpoint{4.923318in}{0.822881in}}%
\pgfpathlineto{\pgfqpoint{4.923318in}{0.902755in}}%
\pgfusepath{stroke}%
\end{pgfscope}%
\begin{pgfscope}%
\pgfpathrectangle{\pgfqpoint{0.572918in}{0.553781in}}{\pgfqpoint{5.478282in}{2.095553in}}%
\pgfusepath{clip}%
\pgfsetbuttcap%
\pgfsetroundjoin%
\pgfsetlinewidth{1.505625pt}%
\definecolor{currentstroke}{rgb}{1.000000,0.819608,0.101961}%
\pgfsetstrokecolor{currentstroke}%
\pgfsetstrokeopacity{0.900000}%
\pgfsetdash{}{0pt}%
\pgfpathmoveto{\pgfqpoint{5.216275in}{0.786669in}}%
\pgfpathlineto{\pgfqpoint{5.216275in}{0.885982in}}%
\pgfusepath{stroke}%
\end{pgfscope}%
\begin{pgfscope}%
\pgfpathrectangle{\pgfqpoint{0.572918in}{0.553781in}}{\pgfqpoint{5.478282in}{2.095553in}}%
\pgfusepath{clip}%
\pgfsetbuttcap%
\pgfsetroundjoin%
\pgfsetlinewidth{1.505625pt}%
\definecolor{currentstroke}{rgb}{1.000000,0.819608,0.101961}%
\pgfsetstrokecolor{currentstroke}%
\pgfsetstrokeopacity{0.900000}%
\pgfsetdash{}{0pt}%
\pgfpathmoveto{\pgfqpoint{5.509231in}{0.671880in}}%
\pgfpathlineto{\pgfqpoint{5.509231in}{0.773529in}}%
\pgfusepath{stroke}%
\end{pgfscope}%
\begin{pgfscope}%
\pgfpathrectangle{\pgfqpoint{0.572918in}{0.553781in}}{\pgfqpoint{5.478282in}{2.095553in}}%
\pgfusepath{clip}%
\pgfsetbuttcap%
\pgfsetroundjoin%
\pgfsetlinewidth{1.505625pt}%
\definecolor{currentstroke}{rgb}{1.000000,0.819608,0.101961}%
\pgfsetstrokecolor{currentstroke}%
\pgfsetstrokeopacity{0.900000}%
\pgfsetdash{}{0pt}%
\pgfpathmoveto{\pgfqpoint{5.802187in}{0.649033in}}%
\pgfpathlineto{\pgfqpoint{5.802187in}{0.806739in}}%
\pgfusepath{stroke}%
\end{pgfscope}%
\begin{pgfscope}%
\pgfpathrectangle{\pgfqpoint{0.572918in}{0.553781in}}{\pgfqpoint{5.478282in}{2.095553in}}%
\pgfusepath{clip}%
\pgfsetbuttcap%
\pgfsetroundjoin%
\definecolor{currentfill}{rgb}{0.313725,0.317647,0.309804}%
\pgfsetfillcolor{currentfill}%
\pgfsetfillopacity{0.900000}%
\pgfsetlinewidth{1.003750pt}%
\definecolor{currentstroke}{rgb}{0.313725,0.317647,0.309804}%
\pgfsetstrokecolor{currentstroke}%
\pgfsetstrokeopacity{0.900000}%
\pgfsetdash{}{0pt}%
\pgfsys@defobject{currentmarker}{\pgfqpoint{-0.013889in}{-0.000000in}}{\pgfqpoint{0.013889in}{0.000000in}}{%
\pgfpathmoveto{\pgfqpoint{0.013889in}{-0.000000in}}%
\pgfpathlineto{\pgfqpoint{-0.013889in}{0.000000in}}%
\pgfusepath{stroke,fill}%
}%
\begin{pgfscope}%
\pgfsys@transformshift{0.821931in}{1.876502in}%
\pgfsys@useobject{currentmarker}{}%
\end{pgfscope}%
\begin{pgfscope}%
\pgfsys@transformshift{1.114887in}{1.960242in}%
\pgfsys@useobject{currentmarker}{}%
\end{pgfscope}%
\begin{pgfscope}%
\pgfsys@transformshift{1.407844in}{1.924596in}%
\pgfsys@useobject{currentmarker}{}%
\end{pgfscope}%
\begin{pgfscope}%
\pgfsys@transformshift{1.700800in}{2.044423in}%
\pgfsys@useobject{currentmarker}{}%
\end{pgfscope}%
\begin{pgfscope}%
\pgfsys@transformshift{1.993756in}{2.138952in}%
\pgfsys@useobject{currentmarker}{}%
\end{pgfscope}%
\begin{pgfscope}%
\pgfsys@transformshift{2.286712in}{2.132477in}%
\pgfsys@useobject{currentmarker}{}%
\end{pgfscope}%
\begin{pgfscope}%
\pgfsys@transformshift{2.579669in}{2.040047in}%
\pgfsys@useobject{currentmarker}{}%
\end{pgfscope}%
\begin{pgfscope}%
\pgfsys@transformshift{2.872625in}{1.948060in}%
\pgfsys@useobject{currentmarker}{}%
\end{pgfscope}%
\begin{pgfscope}%
\pgfsys@transformshift{3.165581in}{1.752391in}%
\pgfsys@useobject{currentmarker}{}%
\end{pgfscope}%
\begin{pgfscope}%
\pgfsys@transformshift{3.458537in}{1.577617in}%
\pgfsys@useobject{currentmarker}{}%
\end{pgfscope}%
\begin{pgfscope}%
\pgfsys@transformshift{3.751494in}{1.401942in}%
\pgfsys@useobject{currentmarker}{}%
\end{pgfscope}%
\begin{pgfscope}%
\pgfsys@transformshift{4.044450in}{1.274577in}%
\pgfsys@useobject{currentmarker}{}%
\end{pgfscope}%
\begin{pgfscope}%
\pgfsys@transformshift{4.337406in}{1.130452in}%
\pgfsys@useobject{currentmarker}{}%
\end{pgfscope}%
\begin{pgfscope}%
\pgfsys@transformshift{4.630362in}{1.063926in}%
\pgfsys@useobject{currentmarker}{}%
\end{pgfscope}%
\begin{pgfscope}%
\pgfsys@transformshift{4.923318in}{0.943440in}%
\pgfsys@useobject{currentmarker}{}%
\end{pgfscope}%
\begin{pgfscope}%
\pgfsys@transformshift{5.216275in}{0.982409in}%
\pgfsys@useobject{currentmarker}{}%
\end{pgfscope}%
\begin{pgfscope}%
\pgfsys@transformshift{5.509231in}{0.817690in}%
\pgfsys@useobject{currentmarker}{}%
\end{pgfscope}%
\begin{pgfscope}%
\pgfsys@transformshift{5.802187in}{0.830056in}%
\pgfsys@useobject{currentmarker}{}%
\end{pgfscope}%
\end{pgfscope}%
\begin{pgfscope}%
\pgfpathrectangle{\pgfqpoint{0.572918in}{0.553781in}}{\pgfqpoint{5.478282in}{2.095553in}}%
\pgfusepath{clip}%
\pgfsetbuttcap%
\pgfsetroundjoin%
\definecolor{currentfill}{rgb}{0.313725,0.317647,0.309804}%
\pgfsetfillcolor{currentfill}%
\pgfsetfillopacity{0.900000}%
\pgfsetlinewidth{1.003750pt}%
\definecolor{currentstroke}{rgb}{0.313725,0.317647,0.309804}%
\pgfsetstrokecolor{currentstroke}%
\pgfsetstrokeopacity{0.900000}%
\pgfsetdash{}{0pt}%
\pgfsys@defobject{currentmarker}{\pgfqpoint{-0.013889in}{-0.000000in}}{\pgfqpoint{0.013889in}{0.000000in}}{%
\pgfpathmoveto{\pgfqpoint{0.013889in}{-0.000000in}}%
\pgfpathlineto{\pgfqpoint{-0.013889in}{0.000000in}}%
\pgfusepath{stroke,fill}%
}%
\begin{pgfscope}%
\pgfsys@transformshift{0.821931in}{2.530549in}%
\pgfsys@useobject{currentmarker}{}%
\end{pgfscope}%
\begin{pgfscope}%
\pgfsys@transformshift{1.114887in}{2.241571in}%
\pgfsys@useobject{currentmarker}{}%
\end{pgfscope}%
\begin{pgfscope}%
\pgfsys@transformshift{1.407844in}{2.103930in}%
\pgfsys@useobject{currentmarker}{}%
\end{pgfscope}%
\begin{pgfscope}%
\pgfsys@transformshift{1.700800in}{2.146716in}%
\pgfsys@useobject{currentmarker}{}%
\end{pgfscope}%
\begin{pgfscope}%
\pgfsys@transformshift{1.993756in}{2.230297in}%
\pgfsys@useobject{currentmarker}{}%
\end{pgfscope}%
\begin{pgfscope}%
\pgfsys@transformshift{2.286712in}{2.209609in}%
\pgfsys@useobject{currentmarker}{}%
\end{pgfscope}%
\begin{pgfscope}%
\pgfsys@transformshift{2.579669in}{2.103224in}%
\pgfsys@useobject{currentmarker}{}%
\end{pgfscope}%
\begin{pgfscope}%
\pgfsys@transformshift{2.872625in}{2.013760in}%
\pgfsys@useobject{currentmarker}{}%
\end{pgfscope}%
\begin{pgfscope}%
\pgfsys@transformshift{3.165581in}{1.810813in}%
\pgfsys@useobject{currentmarker}{}%
\end{pgfscope}%
\begin{pgfscope}%
\pgfsys@transformshift{3.458537in}{1.635123in}%
\pgfsys@useobject{currentmarker}{}%
\end{pgfscope}%
\begin{pgfscope}%
\pgfsys@transformshift{3.751494in}{1.463563in}%
\pgfsys@useobject{currentmarker}{}%
\end{pgfscope}%
\begin{pgfscope}%
\pgfsys@transformshift{4.044450in}{1.337350in}%
\pgfsys@useobject{currentmarker}{}%
\end{pgfscope}%
\begin{pgfscope}%
\pgfsys@transformshift{4.337406in}{1.199946in}%
\pgfsys@useobject{currentmarker}{}%
\end{pgfscope}%
\begin{pgfscope}%
\pgfsys@transformshift{4.630362in}{1.148751in}%
\pgfsys@useobject{currentmarker}{}%
\end{pgfscope}%
\begin{pgfscope}%
\pgfsys@transformshift{4.923318in}{1.032083in}%
\pgfsys@useobject{currentmarker}{}%
\end{pgfscope}%
\begin{pgfscope}%
\pgfsys@transformshift{5.216275in}{1.106934in}%
\pgfsys@useobject{currentmarker}{}%
\end{pgfscope}%
\begin{pgfscope}%
\pgfsys@transformshift{5.509231in}{0.973846in}%
\pgfsys@useobject{currentmarker}{}%
\end{pgfscope}%
\begin{pgfscope}%
\pgfsys@transformshift{5.802187in}{1.053850in}%
\pgfsys@useobject{currentmarker}{}%
\end{pgfscope}%
\end{pgfscope}%
\begin{pgfscope}%
\pgfpathrectangle{\pgfqpoint{0.572918in}{0.553781in}}{\pgfqpoint{5.478282in}{2.095553in}}%
\pgfusepath{clip}%
\pgfsetbuttcap%
\pgfsetroundjoin%
\definecolor{currentfill}{rgb}{0.949020,0.372549,0.360784}%
\pgfsetfillcolor{currentfill}%
\pgfsetfillopacity{0.900000}%
\pgfsetlinewidth{1.003750pt}%
\definecolor{currentstroke}{rgb}{0.949020,0.372549,0.360784}%
\pgfsetstrokecolor{currentstroke}%
\pgfsetstrokeopacity{0.900000}%
\pgfsetdash{}{0pt}%
\pgfsys@defobject{currentmarker}{\pgfqpoint{-0.013889in}{-0.000000in}}{\pgfqpoint{0.013889in}{0.000000in}}{%
\pgfpathmoveto{\pgfqpoint{0.013889in}{-0.000000in}}%
\pgfpathlineto{\pgfqpoint{-0.013889in}{0.000000in}}%
\pgfusepath{stroke,fill}%
}%
\begin{pgfscope}%
\pgfsys@transformshift{0.821931in}{1.865882in}%
\pgfsys@useobject{currentmarker}{}%
\end{pgfscope}%
\begin{pgfscope}%
\pgfsys@transformshift{1.114887in}{1.976678in}%
\pgfsys@useobject{currentmarker}{}%
\end{pgfscope}%
\begin{pgfscope}%
\pgfsys@transformshift{1.407844in}{1.873765in}%
\pgfsys@useobject{currentmarker}{}%
\end{pgfscope}%
\begin{pgfscope}%
\pgfsys@transformshift{1.700800in}{1.900015in}%
\pgfsys@useobject{currentmarker}{}%
\end{pgfscope}%
\begin{pgfscope}%
\pgfsys@transformshift{1.993756in}{1.990986in}%
\pgfsys@useobject{currentmarker}{}%
\end{pgfscope}%
\begin{pgfscope}%
\pgfsys@transformshift{2.286712in}{1.963804in}%
\pgfsys@useobject{currentmarker}{}%
\end{pgfscope}%
\begin{pgfscope}%
\pgfsys@transformshift{2.579669in}{1.853103in}%
\pgfsys@useobject{currentmarker}{}%
\end{pgfscope}%
\begin{pgfscope}%
\pgfsys@transformshift{2.872625in}{1.772269in}%
\pgfsys@useobject{currentmarker}{}%
\end{pgfscope}%
\begin{pgfscope}%
\pgfsys@transformshift{3.165581in}{1.594924in}%
\pgfsys@useobject{currentmarker}{}%
\end{pgfscope}%
\begin{pgfscope}%
\pgfsys@transformshift{3.458537in}{1.449489in}%
\pgfsys@useobject{currentmarker}{}%
\end{pgfscope}%
\begin{pgfscope}%
\pgfsys@transformshift{3.751494in}{1.325946in}%
\pgfsys@useobject{currentmarker}{}%
\end{pgfscope}%
\begin{pgfscope}%
\pgfsys@transformshift{4.044450in}{1.173028in}%
\pgfsys@useobject{currentmarker}{}%
\end{pgfscope}%
\begin{pgfscope}%
\pgfsys@transformshift{4.337406in}{1.031593in}%
\pgfsys@useobject{currentmarker}{}%
\end{pgfscope}%
\begin{pgfscope}%
\pgfsys@transformshift{4.630362in}{0.873758in}%
\pgfsys@useobject{currentmarker}{}%
\end{pgfscope}%
\begin{pgfscope}%
\pgfsys@transformshift{4.923318in}{0.806085in}%
\pgfsys@useobject{currentmarker}{}%
\end{pgfscope}%
\begin{pgfscope}%
\pgfsys@transformshift{5.216275in}{0.766272in}%
\pgfsys@useobject{currentmarker}{}%
\end{pgfscope}%
\begin{pgfscope}%
\pgfsys@transformshift{5.509231in}{0.738161in}%
\pgfsys@useobject{currentmarker}{}%
\end{pgfscope}%
\begin{pgfscope}%
\pgfsys@transformshift{5.802187in}{0.856135in}%
\pgfsys@useobject{currentmarker}{}%
\end{pgfscope}%
\end{pgfscope}%
\begin{pgfscope}%
\pgfpathrectangle{\pgfqpoint{0.572918in}{0.553781in}}{\pgfqpoint{5.478282in}{2.095553in}}%
\pgfusepath{clip}%
\pgfsetbuttcap%
\pgfsetroundjoin%
\definecolor{currentfill}{rgb}{0.949020,0.372549,0.360784}%
\pgfsetfillcolor{currentfill}%
\pgfsetfillopacity{0.900000}%
\pgfsetlinewidth{1.003750pt}%
\definecolor{currentstroke}{rgb}{0.949020,0.372549,0.360784}%
\pgfsetstrokecolor{currentstroke}%
\pgfsetstrokeopacity{0.900000}%
\pgfsetdash{}{0pt}%
\pgfsys@defobject{currentmarker}{\pgfqpoint{-0.013889in}{-0.000000in}}{\pgfqpoint{0.013889in}{0.000000in}}{%
\pgfpathmoveto{\pgfqpoint{0.013889in}{-0.000000in}}%
\pgfpathlineto{\pgfqpoint{-0.013889in}{0.000000in}}%
\pgfusepath{stroke,fill}%
}%
\begin{pgfscope}%
\pgfsys@transformshift{0.821931in}{2.554081in}%
\pgfsys@useobject{currentmarker}{}%
\end{pgfscope}%
\begin{pgfscope}%
\pgfsys@transformshift{1.114887in}{2.243386in}%
\pgfsys@useobject{currentmarker}{}%
\end{pgfscope}%
\begin{pgfscope}%
\pgfsys@transformshift{1.407844in}{2.025647in}%
\pgfsys@useobject{currentmarker}{}%
\end{pgfscope}%
\begin{pgfscope}%
\pgfsys@transformshift{1.700800in}{2.006778in}%
\pgfsys@useobject{currentmarker}{}%
\end{pgfscope}%
\begin{pgfscope}%
\pgfsys@transformshift{1.993756in}{2.077835in}%
\pgfsys@useobject{currentmarker}{}%
\end{pgfscope}%
\begin{pgfscope}%
\pgfsys@transformshift{2.286712in}{2.027336in}%
\pgfsys@useobject{currentmarker}{}%
\end{pgfscope}%
\begin{pgfscope}%
\pgfsys@transformshift{2.579669in}{1.915874in}%
\pgfsys@useobject{currentmarker}{}%
\end{pgfscope}%
\begin{pgfscope}%
\pgfsys@transformshift{2.872625in}{1.829583in}%
\pgfsys@useobject{currentmarker}{}%
\end{pgfscope}%
\begin{pgfscope}%
\pgfsys@transformshift{3.165581in}{1.654281in}%
\pgfsys@useobject{currentmarker}{}%
\end{pgfscope}%
\begin{pgfscope}%
\pgfsys@transformshift{3.458537in}{1.504971in}%
\pgfsys@useobject{currentmarker}{}%
\end{pgfscope}%
\begin{pgfscope}%
\pgfsys@transformshift{3.751494in}{1.378416in}%
\pgfsys@useobject{currentmarker}{}%
\end{pgfscope}%
\begin{pgfscope}%
\pgfsys@transformshift{4.044450in}{1.236022in}%
\pgfsys@useobject{currentmarker}{}%
\end{pgfscope}%
\begin{pgfscope}%
\pgfsys@transformshift{4.337406in}{1.096358in}%
\pgfsys@useobject{currentmarker}{}%
\end{pgfscope}%
\begin{pgfscope}%
\pgfsys@transformshift{4.630362in}{0.948242in}%
\pgfsys@useobject{currentmarker}{}%
\end{pgfscope}%
\begin{pgfscope}%
\pgfsys@transformshift{4.923318in}{0.891131in}%
\pgfsys@useobject{currentmarker}{}%
\end{pgfscope}%
\begin{pgfscope}%
\pgfsys@transformshift{5.216275in}{0.904551in}%
\pgfsys@useobject{currentmarker}{}%
\end{pgfscope}%
\begin{pgfscope}%
\pgfsys@transformshift{5.509231in}{0.884185in}%
\pgfsys@useobject{currentmarker}{}%
\end{pgfscope}%
\begin{pgfscope}%
\pgfsys@transformshift{5.802187in}{1.050405in}%
\pgfsys@useobject{currentmarker}{}%
\end{pgfscope}%
\end{pgfscope}%
\begin{pgfscope}%
\pgfpathrectangle{\pgfqpoint{0.572918in}{0.553781in}}{\pgfqpoint{5.478282in}{2.095553in}}%
\pgfusepath{clip}%
\pgfsetbuttcap%
\pgfsetroundjoin%
\definecolor{currentfill}{rgb}{1.000000,0.819608,0.101961}%
\pgfsetfillcolor{currentfill}%
\pgfsetfillopacity{0.900000}%
\pgfsetlinewidth{1.003750pt}%
\definecolor{currentstroke}{rgb}{1.000000,0.819608,0.101961}%
\pgfsetstrokecolor{currentstroke}%
\pgfsetstrokeopacity{0.900000}%
\pgfsetdash{}{0pt}%
\pgfsys@defobject{currentmarker}{\pgfqpoint{-0.013889in}{-0.000000in}}{\pgfqpoint{0.013889in}{0.000000in}}{%
\pgfpathmoveto{\pgfqpoint{0.013889in}{-0.000000in}}%
\pgfpathlineto{\pgfqpoint{-0.013889in}{0.000000in}}%
\pgfusepath{stroke,fill}%
}%
\begin{pgfscope}%
\pgfsys@transformshift{0.821931in}{1.933177in}%
\pgfsys@useobject{currentmarker}{}%
\end{pgfscope}%
\begin{pgfscope}%
\pgfsys@transformshift{1.114887in}{1.845694in}%
\pgfsys@useobject{currentmarker}{}%
\end{pgfscope}%
\begin{pgfscope}%
\pgfsys@transformshift{1.407844in}{1.857270in}%
\pgfsys@useobject{currentmarker}{}%
\end{pgfscope}%
\begin{pgfscope}%
\pgfsys@transformshift{1.700800in}{1.927427in}%
\pgfsys@useobject{currentmarker}{}%
\end{pgfscope}%
\begin{pgfscope}%
\pgfsys@transformshift{1.993756in}{1.966691in}%
\pgfsys@useobject{currentmarker}{}%
\end{pgfscope}%
\begin{pgfscope}%
\pgfsys@transformshift{2.286712in}{1.949186in}%
\pgfsys@useobject{currentmarker}{}%
\end{pgfscope}%
\begin{pgfscope}%
\pgfsys@transformshift{2.579669in}{1.836146in}%
\pgfsys@useobject{currentmarker}{}%
\end{pgfscope}%
\begin{pgfscope}%
\pgfsys@transformshift{2.872625in}{1.781188in}%
\pgfsys@useobject{currentmarker}{}%
\end{pgfscope}%
\begin{pgfscope}%
\pgfsys@transformshift{3.165581in}{1.583574in}%
\pgfsys@useobject{currentmarker}{}%
\end{pgfscope}%
\begin{pgfscope}%
\pgfsys@transformshift{3.458537in}{1.437153in}%
\pgfsys@useobject{currentmarker}{}%
\end{pgfscope}%
\begin{pgfscope}%
\pgfsys@transformshift{3.751494in}{1.310464in}%
\pgfsys@useobject{currentmarker}{}%
\end{pgfscope}%
\begin{pgfscope}%
\pgfsys@transformshift{4.044450in}{1.190002in}%
\pgfsys@useobject{currentmarker}{}%
\end{pgfscope}%
\begin{pgfscope}%
\pgfsys@transformshift{4.337406in}{1.063963in}%
\pgfsys@useobject{currentmarker}{}%
\end{pgfscope}%
\begin{pgfscope}%
\pgfsys@transformshift{4.630362in}{0.951248in}%
\pgfsys@useobject{currentmarker}{}%
\end{pgfscope}%
\begin{pgfscope}%
\pgfsys@transformshift{4.923318in}{0.822881in}%
\pgfsys@useobject{currentmarker}{}%
\end{pgfscope}%
\begin{pgfscope}%
\pgfsys@transformshift{5.216275in}{0.786669in}%
\pgfsys@useobject{currentmarker}{}%
\end{pgfscope}%
\begin{pgfscope}%
\pgfsys@transformshift{5.509231in}{0.671880in}%
\pgfsys@useobject{currentmarker}{}%
\end{pgfscope}%
\begin{pgfscope}%
\pgfsys@transformshift{5.802187in}{0.649033in}%
\pgfsys@useobject{currentmarker}{}%
\end{pgfscope}%
\end{pgfscope}%
\begin{pgfscope}%
\pgfpathrectangle{\pgfqpoint{0.572918in}{0.553781in}}{\pgfqpoint{5.478282in}{2.095553in}}%
\pgfusepath{clip}%
\pgfsetbuttcap%
\pgfsetroundjoin%
\definecolor{currentfill}{rgb}{1.000000,0.819608,0.101961}%
\pgfsetfillcolor{currentfill}%
\pgfsetfillopacity{0.900000}%
\pgfsetlinewidth{1.003750pt}%
\definecolor{currentstroke}{rgb}{1.000000,0.819608,0.101961}%
\pgfsetstrokecolor{currentstroke}%
\pgfsetstrokeopacity{0.900000}%
\pgfsetdash{}{0pt}%
\pgfsys@defobject{currentmarker}{\pgfqpoint{-0.013889in}{-0.000000in}}{\pgfqpoint{0.013889in}{0.000000in}}{%
\pgfpathmoveto{\pgfqpoint{0.013889in}{-0.000000in}}%
\pgfpathlineto{\pgfqpoint{-0.013889in}{0.000000in}}%
\pgfusepath{stroke,fill}%
}%
\begin{pgfscope}%
\pgfsys@transformshift{0.821931in}{2.464097in}%
\pgfsys@useobject{currentmarker}{}%
\end{pgfscope}%
\begin{pgfscope}%
\pgfsys@transformshift{1.114887in}{2.119473in}%
\pgfsys@useobject{currentmarker}{}%
\end{pgfscope}%
\begin{pgfscope}%
\pgfsys@transformshift{1.407844in}{1.996922in}%
\pgfsys@useobject{currentmarker}{}%
\end{pgfscope}%
\begin{pgfscope}%
\pgfsys@transformshift{1.700800in}{2.029553in}%
\pgfsys@useobject{currentmarker}{}%
\end{pgfscope}%
\begin{pgfscope}%
\pgfsys@transformshift{1.993756in}{2.056628in}%
\pgfsys@useobject{currentmarker}{}%
\end{pgfscope}%
\begin{pgfscope}%
\pgfsys@transformshift{2.286712in}{2.021969in}%
\pgfsys@useobject{currentmarker}{}%
\end{pgfscope}%
\begin{pgfscope}%
\pgfsys@transformshift{2.579669in}{1.897323in}%
\pgfsys@useobject{currentmarker}{}%
\end{pgfscope}%
\begin{pgfscope}%
\pgfsys@transformshift{2.872625in}{1.841292in}%
\pgfsys@useobject{currentmarker}{}%
\end{pgfscope}%
\begin{pgfscope}%
\pgfsys@transformshift{3.165581in}{1.640033in}%
\pgfsys@useobject{currentmarker}{}%
\end{pgfscope}%
\begin{pgfscope}%
\pgfsys@transformshift{3.458537in}{1.488326in}%
\pgfsys@useobject{currentmarker}{}%
\end{pgfscope}%
\begin{pgfscope}%
\pgfsys@transformshift{3.751494in}{1.366965in}%
\pgfsys@useobject{currentmarker}{}%
\end{pgfscope}%
\begin{pgfscope}%
\pgfsys@transformshift{4.044450in}{1.252633in}%
\pgfsys@useobject{currentmarker}{}%
\end{pgfscope}%
\begin{pgfscope}%
\pgfsys@transformshift{4.337406in}{1.133046in}%
\pgfsys@useobject{currentmarker}{}%
\end{pgfscope}%
\begin{pgfscope}%
\pgfsys@transformshift{4.630362in}{1.030747in}%
\pgfsys@useobject{currentmarker}{}%
\end{pgfscope}%
\begin{pgfscope}%
\pgfsys@transformshift{4.923318in}{0.902755in}%
\pgfsys@useobject{currentmarker}{}%
\end{pgfscope}%
\begin{pgfscope}%
\pgfsys@transformshift{5.216275in}{0.885982in}%
\pgfsys@useobject{currentmarker}{}%
\end{pgfscope}%
\begin{pgfscope}%
\pgfsys@transformshift{5.509231in}{0.773529in}%
\pgfsys@useobject{currentmarker}{}%
\end{pgfscope}%
\begin{pgfscope}%
\pgfsys@transformshift{5.802187in}{0.806739in}%
\pgfsys@useobject{currentmarker}{}%
\end{pgfscope}%
\end{pgfscope}%
\begin{pgfscope}%
\pgfpathrectangle{\pgfqpoint{0.572918in}{0.553781in}}{\pgfqpoint{5.478282in}{2.095553in}}%
\pgfusepath{clip}%
\pgfsetrectcap%
\pgfsetroundjoin%
\pgfsetlinewidth{1.505625pt}%
\definecolor{currentstroke}{rgb}{0.313725,0.317647,0.309804}%
\pgfsetstrokecolor{currentstroke}%
\pgfsetstrokeopacity{0.900000}%
\pgfsetdash{}{0pt}%
\pgfpathmoveto{\pgfqpoint{0.821931in}{2.198210in}}%
\pgfpathlineto{\pgfqpoint{1.114887in}{2.081854in}}%
\pgfpathlineto{\pgfqpoint{1.407844in}{2.012423in}}%
\pgfpathlineto{\pgfqpoint{1.700800in}{2.097248in}}%
\pgfpathlineto{\pgfqpoint{1.993756in}{2.186506in}}%
\pgfpathlineto{\pgfqpoint{2.286712in}{2.169260in}}%
\pgfpathlineto{\pgfqpoint{2.579669in}{2.072999in}}%
\pgfpathlineto{\pgfqpoint{2.872625in}{1.981079in}}%
\pgfpathlineto{\pgfqpoint{3.165581in}{1.782179in}}%
\pgfpathlineto{\pgfqpoint{3.458537in}{1.605444in}}%
\pgfpathlineto{\pgfqpoint{3.751494in}{1.434500in}}%
\pgfpathlineto{\pgfqpoint{4.044450in}{1.303325in}}%
\pgfpathlineto{\pgfqpoint{4.337406in}{1.170664in}}%
\pgfpathlineto{\pgfqpoint{4.630362in}{1.105630in}}%
\pgfpathlineto{\pgfqpoint{4.923318in}{0.989290in}}%
\pgfpathlineto{\pgfqpoint{5.216275in}{1.054475in}}%
\pgfpathlineto{\pgfqpoint{5.509231in}{0.897197in}}%
\pgfpathlineto{\pgfqpoint{5.802187in}{0.928426in}}%
\pgfusepath{stroke}%
\end{pgfscope}%
\begin{pgfscope}%
\pgfpathrectangle{\pgfqpoint{0.572918in}{0.553781in}}{\pgfqpoint{5.478282in}{2.095553in}}%
\pgfusepath{clip}%
\pgfsetbuttcap%
\pgfsetroundjoin%
\pgfsetlinewidth{1.505625pt}%
\definecolor{currentstroke}{rgb}{0.949020,0.372549,0.360784}%
\pgfsetstrokecolor{currentstroke}%
\pgfsetstrokeopacity{0.900000}%
\pgfsetdash{{1.500000pt}{2.475000pt}}{0.000000pt}%
\pgfpathmoveto{\pgfqpoint{0.821931in}{2.146084in}}%
\pgfpathlineto{\pgfqpoint{1.114887in}{2.077071in}}%
\pgfpathlineto{\pgfqpoint{1.407844in}{1.944141in}}%
\pgfpathlineto{\pgfqpoint{1.700800in}{1.958253in}}%
\pgfpathlineto{\pgfqpoint{1.993756in}{2.039496in}}%
\pgfpathlineto{\pgfqpoint{2.286712in}{1.998311in}}%
\pgfpathlineto{\pgfqpoint{2.579669in}{1.882734in}}%
\pgfpathlineto{\pgfqpoint{2.872625in}{1.799927in}}%
\pgfpathlineto{\pgfqpoint{3.165581in}{1.622993in}}%
\pgfpathlineto{\pgfqpoint{3.458537in}{1.476143in}}%
\pgfpathlineto{\pgfqpoint{3.751494in}{1.351700in}}%
\pgfpathlineto{\pgfqpoint{4.044450in}{1.203532in}}%
\pgfpathlineto{\pgfqpoint{4.337406in}{1.063145in}}%
\pgfpathlineto{\pgfqpoint{4.630362in}{0.908765in}}%
\pgfpathlineto{\pgfqpoint{4.923318in}{0.846745in}}%
\pgfpathlineto{\pgfqpoint{5.216275in}{0.834623in}}%
\pgfpathlineto{\pgfqpoint{5.509231in}{0.812020in}}%
\pgfpathlineto{\pgfqpoint{5.802187in}{0.937462in}}%
\pgfusepath{stroke}%
\end{pgfscope}%
\begin{pgfscope}%
\pgfpathrectangle{\pgfqpoint{0.572918in}{0.553781in}}{\pgfqpoint{5.478282in}{2.095553in}}%
\pgfusepath{clip}%
\pgfsetbuttcap%
\pgfsetroundjoin%
\pgfsetlinewidth{1.505625pt}%
\definecolor{currentstroke}{rgb}{1.000000,0.819608,0.101961}%
\pgfsetstrokecolor{currentstroke}%
\pgfsetstrokeopacity{0.900000}%
\pgfsetdash{{5.550000pt}{2.400000pt}}{0.000000pt}%
\pgfpathmoveto{\pgfqpoint{0.821931in}{2.179459in}}%
\pgfpathlineto{\pgfqpoint{1.114887in}{1.993082in}}%
\pgfpathlineto{\pgfqpoint{1.407844in}{1.925953in}}%
\pgfpathlineto{\pgfqpoint{1.700800in}{1.982617in}}%
\pgfpathlineto{\pgfqpoint{1.993756in}{2.014789in}}%
\pgfpathlineto{\pgfqpoint{2.286712in}{1.982326in}}%
\pgfpathlineto{\pgfqpoint{2.579669in}{1.864768in}}%
\pgfpathlineto{\pgfqpoint{2.872625in}{1.814423in}}%
\pgfpathlineto{\pgfqpoint{3.165581in}{1.610897in}}%
\pgfpathlineto{\pgfqpoint{3.458537in}{1.463461in}}%
\pgfpathlineto{\pgfqpoint{3.751494in}{1.336902in}}%
\pgfpathlineto{\pgfqpoint{4.044450in}{1.219198in}}%
\pgfpathlineto{\pgfqpoint{4.337406in}{1.096847in}}%
\pgfpathlineto{\pgfqpoint{4.630362in}{0.992532in}}%
\pgfpathlineto{\pgfqpoint{4.923318in}{0.860809in}}%
\pgfpathlineto{\pgfqpoint{5.216275in}{0.839350in}}%
\pgfpathlineto{\pgfqpoint{5.509231in}{0.725624in}}%
\pgfpathlineto{\pgfqpoint{5.802187in}{0.727653in}}%
\pgfusepath{stroke}%
\end{pgfscope}%
\begin{pgfscope}%
\pgfsetrectcap%
\pgfsetmiterjoin%
\pgfsetlinewidth{0.803000pt}%
\definecolor{currentstroke}{rgb}{0.000000,0.000000,0.000000}%
\pgfsetstrokecolor{currentstroke}%
\pgfsetdash{}{0pt}%
\pgfpathmoveto{\pgfqpoint{0.572918in}{0.553781in}}%
\pgfpathlineto{\pgfqpoint{0.572918in}{2.649333in}}%
\pgfusepath{stroke}%
\end{pgfscope}%
\begin{pgfscope}%
\pgfsetrectcap%
\pgfsetmiterjoin%
\pgfsetlinewidth{0.803000pt}%
\definecolor{currentstroke}{rgb}{0.000000,0.000000,0.000000}%
\pgfsetstrokecolor{currentstroke}%
\pgfsetdash{}{0pt}%
\pgfpathmoveto{\pgfqpoint{6.051200in}{0.553781in}}%
\pgfpathlineto{\pgfqpoint{6.051200in}{2.649333in}}%
\pgfusepath{stroke}%
\end{pgfscope}%
\begin{pgfscope}%
\pgfsetrectcap%
\pgfsetmiterjoin%
\pgfsetlinewidth{0.803000pt}%
\definecolor{currentstroke}{rgb}{0.000000,0.000000,0.000000}%
\pgfsetstrokecolor{currentstroke}%
\pgfsetdash{}{0pt}%
\pgfpathmoveto{\pgfqpoint{0.572918in}{0.553781in}}%
\pgfpathlineto{\pgfqpoint{6.051200in}{0.553781in}}%
\pgfusepath{stroke}%
\end{pgfscope}%
\begin{pgfscope}%
\pgfsetrectcap%
\pgfsetmiterjoin%
\pgfsetlinewidth{0.803000pt}%
\definecolor{currentstroke}{rgb}{0.000000,0.000000,0.000000}%
\pgfsetstrokecolor{currentstroke}%
\pgfsetdash{}{0pt}%
\pgfpathmoveto{\pgfqpoint{0.572918in}{2.649333in}}%
\pgfpathlineto{\pgfqpoint{6.051200in}{2.649333in}}%
\pgfusepath{stroke}%
\end{pgfscope}%
\begin{pgfscope}%
\definecolor{textcolor}{rgb}{0.000000,0.000000,0.000000}%
\pgfsetstrokecolor{textcolor}%
\pgfsetfillcolor{textcolor}%
\pgftext[x=0.572918in,y=2.732667in,left,base]{\color{textcolor}\rmfamily\fontsize{12.000000}{14.400000}\selectfont Zenith performance}%
\end{pgfscope}%
\begin{pgfscope}%
\pgfsetbuttcap%
\pgfsetmiterjoin%
\definecolor{currentfill}{rgb}{1.000000,1.000000,1.000000}%
\pgfsetfillcolor{currentfill}%
\pgfsetfillopacity{0.800000}%
\pgfsetlinewidth{1.003750pt}%
\definecolor{currentstroke}{rgb}{0.800000,0.800000,0.800000}%
\pgfsetstrokecolor{currentstroke}%
\pgfsetstrokeopacity{0.800000}%
\pgfsetdash{}{0pt}%
\pgfpathmoveto{\pgfqpoint{5.440867in}{2.095778in}}%
\pgfpathlineto{\pgfqpoint{5.973422in}{2.095778in}}%
\pgfpathquadraticcurveto{\pgfqpoint{5.995644in}{2.095778in}}{\pgfqpoint{5.995644in}{2.118000in}}%
\pgfpathlineto{\pgfqpoint{5.995644in}{2.571556in}}%
\pgfpathquadraticcurveto{\pgfqpoint{5.995644in}{2.593778in}}{\pgfqpoint{5.973422in}{2.593778in}}%
\pgfpathlineto{\pgfqpoint{5.440867in}{2.593778in}}%
\pgfpathquadraticcurveto{\pgfqpoint{5.418644in}{2.593778in}}{\pgfqpoint{5.418644in}{2.571556in}}%
\pgfpathlineto{\pgfqpoint{5.418644in}{2.118000in}}%
\pgfpathquadraticcurveto{\pgfqpoint{5.418644in}{2.095778in}}{\pgfqpoint{5.440867in}{2.095778in}}%
\pgfpathclose%
\pgfusepath{stroke,fill}%
\end{pgfscope}%
\begin{pgfscope}%
\pgfsetbuttcap%
\pgfsetroundjoin%
\pgfsetlinewidth{1.505625pt}%
\definecolor{currentstroke}{rgb}{0.313725,0.317647,0.309804}%
\pgfsetstrokecolor{currentstroke}%
\pgfsetstrokeopacity{0.900000}%
\pgfsetdash{}{0pt}%
\pgfpathmoveto{\pgfqpoint{5.574200in}{2.454889in}}%
\pgfpathlineto{\pgfqpoint{5.574200in}{2.566000in}}%
\pgfusepath{stroke}%
\end{pgfscope}%
\begin{pgfscope}%
\pgfsetbuttcap%
\pgfsetroundjoin%
\definecolor{currentfill}{rgb}{0.313725,0.317647,0.309804}%
\pgfsetfillcolor{currentfill}%
\pgfsetfillopacity{0.900000}%
\pgfsetlinewidth{1.003750pt}%
\definecolor{currentstroke}{rgb}{0.313725,0.317647,0.309804}%
\pgfsetstrokecolor{currentstroke}%
\pgfsetstrokeopacity{0.900000}%
\pgfsetdash{}{0pt}%
\pgfsys@defobject{currentmarker}{\pgfqpoint{-0.013889in}{-0.000000in}}{\pgfqpoint{0.013889in}{0.000000in}}{%
\pgfpathmoveto{\pgfqpoint{0.013889in}{-0.000000in}}%
\pgfpathlineto{\pgfqpoint{-0.013889in}{0.000000in}}%
\pgfusepath{stroke,fill}%
}%
\begin{pgfscope}%
\pgfsys@transformshift{5.574200in}{2.454889in}%
\pgfsys@useobject{currentmarker}{}%
\end{pgfscope}%
\end{pgfscope}%
\begin{pgfscope}%
\pgfsetbuttcap%
\pgfsetroundjoin%
\definecolor{currentfill}{rgb}{0.313725,0.317647,0.309804}%
\pgfsetfillcolor{currentfill}%
\pgfsetfillopacity{0.900000}%
\pgfsetlinewidth{1.003750pt}%
\definecolor{currentstroke}{rgb}{0.313725,0.317647,0.309804}%
\pgfsetstrokecolor{currentstroke}%
\pgfsetstrokeopacity{0.900000}%
\pgfsetdash{}{0pt}%
\pgfsys@defobject{currentmarker}{\pgfqpoint{-0.013889in}{-0.000000in}}{\pgfqpoint{0.013889in}{0.000000in}}{%
\pgfpathmoveto{\pgfqpoint{0.013889in}{-0.000000in}}%
\pgfpathlineto{\pgfqpoint{-0.013889in}{0.000000in}}%
\pgfusepath{stroke,fill}%
}%
\begin{pgfscope}%
\pgfsys@transformshift{5.574200in}{2.566000in}%
\pgfsys@useobject{currentmarker}{}%
\end{pgfscope}%
\end{pgfscope}%
\begin{pgfscope}%
\pgfsetrectcap%
\pgfsetroundjoin%
\pgfsetlinewidth{1.505625pt}%
\definecolor{currentstroke}{rgb}{0.313725,0.317647,0.309804}%
\pgfsetstrokecolor{currentstroke}%
\pgfsetstrokeopacity{0.900000}%
\pgfsetdash{}{0pt}%
\pgfpathmoveto{\pgfqpoint{5.463089in}{2.510444in}}%
\pgfpathlineto{\pgfqpoint{5.685311in}{2.510444in}}%
\pgfusepath{stroke}%
\end{pgfscope}%
\begin{pgfscope}%
\definecolor{textcolor}{rgb}{0.000000,0.000000,0.000000}%
\pgfsetstrokecolor{textcolor}%
\pgfsetfillcolor{textcolor}%
\pgftext[x=5.774200in,y=2.471556in,left,base]{\color{textcolor}\rmfamily\fontsize{8.000000}{9.600000}\selectfont 100}%
\end{pgfscope}%
\begin{pgfscope}%
\pgfsetbuttcap%
\pgfsetroundjoin%
\pgfsetlinewidth{1.505625pt}%
\definecolor{currentstroke}{rgb}{0.949020,0.372549,0.360784}%
\pgfsetstrokecolor{currentstroke}%
\pgfsetstrokeopacity{0.900000}%
\pgfsetdash{}{0pt}%
\pgfpathmoveto{\pgfqpoint{5.574200in}{2.300000in}}%
\pgfpathlineto{\pgfqpoint{5.574200in}{2.411111in}}%
\pgfusepath{stroke}%
\end{pgfscope}%
\begin{pgfscope}%
\pgfsetbuttcap%
\pgfsetroundjoin%
\definecolor{currentfill}{rgb}{0.949020,0.372549,0.360784}%
\pgfsetfillcolor{currentfill}%
\pgfsetfillopacity{0.900000}%
\pgfsetlinewidth{1.003750pt}%
\definecolor{currentstroke}{rgb}{0.949020,0.372549,0.360784}%
\pgfsetstrokecolor{currentstroke}%
\pgfsetstrokeopacity{0.900000}%
\pgfsetdash{}{0pt}%
\pgfsys@defobject{currentmarker}{\pgfqpoint{-0.013889in}{-0.000000in}}{\pgfqpoint{0.013889in}{0.000000in}}{%
\pgfpathmoveto{\pgfqpoint{0.013889in}{-0.000000in}}%
\pgfpathlineto{\pgfqpoint{-0.013889in}{0.000000in}}%
\pgfusepath{stroke,fill}%
}%
\begin{pgfscope}%
\pgfsys@transformshift{5.574200in}{2.300000in}%
\pgfsys@useobject{currentmarker}{}%
\end{pgfscope}%
\end{pgfscope}%
\begin{pgfscope}%
\pgfsetbuttcap%
\pgfsetroundjoin%
\definecolor{currentfill}{rgb}{0.949020,0.372549,0.360784}%
\pgfsetfillcolor{currentfill}%
\pgfsetfillopacity{0.900000}%
\pgfsetlinewidth{1.003750pt}%
\definecolor{currentstroke}{rgb}{0.949020,0.372549,0.360784}%
\pgfsetstrokecolor{currentstroke}%
\pgfsetstrokeopacity{0.900000}%
\pgfsetdash{}{0pt}%
\pgfsys@defobject{currentmarker}{\pgfqpoint{-0.013889in}{-0.000000in}}{\pgfqpoint{0.013889in}{0.000000in}}{%
\pgfpathmoveto{\pgfqpoint{0.013889in}{-0.000000in}}%
\pgfpathlineto{\pgfqpoint{-0.013889in}{0.000000in}}%
\pgfusepath{stroke,fill}%
}%
\begin{pgfscope}%
\pgfsys@transformshift{5.574200in}{2.411111in}%
\pgfsys@useobject{currentmarker}{}%
\end{pgfscope}%
\end{pgfscope}%
\begin{pgfscope}%
\pgfsetbuttcap%
\pgfsetroundjoin%
\pgfsetlinewidth{1.505625pt}%
\definecolor{currentstroke}{rgb}{0.949020,0.372549,0.360784}%
\pgfsetstrokecolor{currentstroke}%
\pgfsetstrokeopacity{0.900000}%
\pgfsetdash{{1.500000pt}{2.475000pt}}{0.000000pt}%
\pgfpathmoveto{\pgfqpoint{5.463089in}{2.355556in}}%
\pgfpathlineto{\pgfqpoint{5.685311in}{2.355556in}}%
\pgfusepath{stroke}%
\end{pgfscope}%
\begin{pgfscope}%
\definecolor{textcolor}{rgb}{0.000000,0.000000,0.000000}%
\pgfsetstrokecolor{textcolor}%
\pgfsetfillcolor{textcolor}%
\pgftext[x=5.774200in,y=2.316667in,left,base]{\color{textcolor}\rmfamily\fontsize{8.000000}{9.600000}\selectfont 200}%
\end{pgfscope}%
\begin{pgfscope}%
\pgfsetbuttcap%
\pgfsetroundjoin%
\pgfsetlinewidth{1.505625pt}%
\definecolor{currentstroke}{rgb}{1.000000,0.819608,0.101961}%
\pgfsetstrokecolor{currentstroke}%
\pgfsetstrokeopacity{0.900000}%
\pgfsetdash{}{0pt}%
\pgfpathmoveto{\pgfqpoint{5.574200in}{2.145111in}}%
\pgfpathlineto{\pgfqpoint{5.574200in}{2.256222in}}%
\pgfusepath{stroke}%
\end{pgfscope}%
\begin{pgfscope}%
\pgfsetbuttcap%
\pgfsetroundjoin%
\definecolor{currentfill}{rgb}{1.000000,0.819608,0.101961}%
\pgfsetfillcolor{currentfill}%
\pgfsetfillopacity{0.900000}%
\pgfsetlinewidth{1.003750pt}%
\definecolor{currentstroke}{rgb}{1.000000,0.819608,0.101961}%
\pgfsetstrokecolor{currentstroke}%
\pgfsetstrokeopacity{0.900000}%
\pgfsetdash{}{0pt}%
\pgfsys@defobject{currentmarker}{\pgfqpoint{-0.013889in}{-0.000000in}}{\pgfqpoint{0.013889in}{0.000000in}}{%
\pgfpathmoveto{\pgfqpoint{0.013889in}{-0.000000in}}%
\pgfpathlineto{\pgfqpoint{-0.013889in}{0.000000in}}%
\pgfusepath{stroke,fill}%
}%
\begin{pgfscope}%
\pgfsys@transformshift{5.574200in}{2.145111in}%
\pgfsys@useobject{currentmarker}{}%
\end{pgfscope}%
\end{pgfscope}%
\begin{pgfscope}%
\pgfsetbuttcap%
\pgfsetroundjoin%
\definecolor{currentfill}{rgb}{1.000000,0.819608,0.101961}%
\pgfsetfillcolor{currentfill}%
\pgfsetfillopacity{0.900000}%
\pgfsetlinewidth{1.003750pt}%
\definecolor{currentstroke}{rgb}{1.000000,0.819608,0.101961}%
\pgfsetstrokecolor{currentstroke}%
\pgfsetstrokeopacity{0.900000}%
\pgfsetdash{}{0pt}%
\pgfsys@defobject{currentmarker}{\pgfqpoint{-0.013889in}{-0.000000in}}{\pgfqpoint{0.013889in}{0.000000in}}{%
\pgfpathmoveto{\pgfqpoint{0.013889in}{-0.000000in}}%
\pgfpathlineto{\pgfqpoint{-0.013889in}{0.000000in}}%
\pgfusepath{stroke,fill}%
}%
\begin{pgfscope}%
\pgfsys@transformshift{5.574200in}{2.256222in}%
\pgfsys@useobject{currentmarker}{}%
\end{pgfscope}%
\end{pgfscope}%
\begin{pgfscope}%
\pgfsetbuttcap%
\pgfsetroundjoin%
\pgfsetlinewidth{1.505625pt}%
\definecolor{currentstroke}{rgb}{1.000000,0.819608,0.101961}%
\pgfsetstrokecolor{currentstroke}%
\pgfsetstrokeopacity{0.900000}%
\pgfsetdash{{5.550000pt}{2.400000pt}}{0.000000pt}%
\pgfpathmoveto{\pgfqpoint{5.463089in}{2.200667in}}%
\pgfpathlineto{\pgfqpoint{5.685311in}{2.200667in}}%
\pgfusepath{stroke}%
\end{pgfscope}%
\begin{pgfscope}%
\definecolor{textcolor}{rgb}{0.000000,0.000000,0.000000}%
\pgfsetstrokecolor{textcolor}%
\pgfsetfillcolor{textcolor}%
\pgftext[x=5.774200in,y=2.161778in,left,base]{\color{textcolor}\rmfamily\fontsize{8.000000}{9.600000}\selectfont 400}%
\end{pgfscope}%
\end{pgfpicture}%
\makeatother%
\endgroup%

         \caption{}\label{fig:performance_length}
     \end{subfigure}
        \caption{The distributions of event lengths with/without SRT cleaning (Raw/Cleaned) as KDEs are shown in~\vref{fig:event_length}.
        \Vref{fig:performance_length} shows the effect of different maximum event length on the performance of the network.
    For 200 and 400 the performance is similar, while somewhat decreased for 100.
    This may be caused by the limitation of a maximum length of 100 \enquote{throws away} too much event information.
    This is reinforced by the fact that at the highest energy bin the performance of 200 decreases, probably because higher energy events are typically longer, while the performances are very similar at low energies.}\label{fig:length}
\end{figure}

The tensors that are input into the neural network need to be of the same shape, and thus some padding is necessary.
All tensors are therefore padded up to some pre-set maximum length, and events that are longer than this limit are instead composed of a random subsample of the total event.

The median length of a raw event is 50 and 17 for SRT cleaned ones.
200 was chosen as the best performant maximum length, when training time, memory usage and resolution performance was taken into account; see~\vref{fig:length}.

\subsection{Array padding}

\begin{figure}
    \centering
    %% Creator: Matplotlib, PGF backend
%%
%% To include the figure in your LaTeX document, write
%%   \input{<filename>.pgf}
%%
%% Make sure the required packages are loaded in your preamble
%%   \usepackage{pgf}
%%
%% and, on pdftex
%%   \usepackage[utf8]{inputenc}\DeclareUnicodeCharacter{2212}{-}
%%
%% or, on luatex and xetex
%%   \usepackage{unicode-math}
%%
%% Figures using additional raster images can only be included by \input if
%% they are in the same directory as the main LaTeX file. For loading figures
%% from other directories you can use the `import` package
%%   \usepackage{import}
%%
%% and then include the figures with
%%   \import{<path to file>}{<filename>.pgf}
%%
%% Matplotlib used the following preamble
%%   \usepackage{siunitx} \usepackage{amsmath} \usepackage{bm}
%%   \usepackage{fontspec}
%%
\begingroup%
\makeatletter%
\begin{pgfpicture}%
\pgfpathrectangle{\pgfpointorigin}{\pgfqpoint{6.201200in}{3.000000in}}%
\pgfusepath{use as bounding box, clip}%
\begin{pgfscope}%
\pgfsetbuttcap%
\pgfsetmiterjoin%
\definecolor{currentfill}{rgb}{1.000000,1.000000,1.000000}%
\pgfsetfillcolor{currentfill}%
\pgfsetlinewidth{0.000000pt}%
\definecolor{currentstroke}{rgb}{1.000000,1.000000,1.000000}%
\pgfsetstrokecolor{currentstroke}%
\pgfsetdash{}{0pt}%
\pgfpathmoveto{\pgfqpoint{0.000000in}{0.000000in}}%
\pgfpathlineto{\pgfqpoint{6.201200in}{0.000000in}}%
\pgfpathlineto{\pgfqpoint{6.201200in}{3.000000in}}%
\pgfpathlineto{\pgfqpoint{0.000000in}{3.000000in}}%
\pgfpathclose%
\pgfusepath{fill}%
\end{pgfscope}%
\begin{pgfscope}%
\pgfsetbuttcap%
\pgfsetmiterjoin%
\definecolor{currentfill}{rgb}{1.000000,1.000000,1.000000}%
\pgfsetfillcolor{currentfill}%
\pgfsetlinewidth{0.000000pt}%
\definecolor{currentstroke}{rgb}{0.000000,0.000000,0.000000}%
\pgfsetstrokecolor{currentstroke}%
\pgfsetstrokeopacity{0.000000}%
\pgfsetdash{}{0pt}%
\pgfpathmoveto{\pgfqpoint{0.572918in}{0.553781in}}%
\pgfpathlineto{\pgfqpoint{6.051200in}{0.553781in}}%
\pgfpathlineto{\pgfqpoint{6.051200in}{2.649333in}}%
\pgfpathlineto{\pgfqpoint{0.572918in}{2.649333in}}%
\pgfpathclose%
\pgfusepath{fill}%
\end{pgfscope}%
\begin{pgfscope}%
\pgfpathrectangle{\pgfqpoint{0.572918in}{0.553781in}}{\pgfqpoint{5.478282in}{2.095553in}}%
\pgfusepath{clip}%
\pgfsetbuttcap%
\pgfsetroundjoin%
\pgfsetlinewidth{0.501875pt}%
\definecolor{currentstroke}{rgb}{0.690196,0.690196,0.690196}%
\pgfsetstrokecolor{currentstroke}%
\pgfsetstrokeopacity{0.500000}%
\pgfsetdash{{0.500000pt}{0.825000pt}}{0.000000pt}%
\pgfpathmoveto{\pgfqpoint{0.675453in}{0.553781in}}%
\pgfpathlineto{\pgfqpoint{0.675453in}{2.649333in}}%
\pgfusepath{stroke}%
\end{pgfscope}%
\begin{pgfscope}%
\pgfsetbuttcap%
\pgfsetroundjoin%
\definecolor{currentfill}{rgb}{0.000000,0.000000,0.000000}%
\pgfsetfillcolor{currentfill}%
\pgfsetlinewidth{0.803000pt}%
\definecolor{currentstroke}{rgb}{0.000000,0.000000,0.000000}%
\pgfsetstrokecolor{currentstroke}%
\pgfsetdash{}{0pt}%
\pgfsys@defobject{currentmarker}{\pgfqpoint{0.000000in}{-0.048611in}}{\pgfqpoint{0.000000in}{0.000000in}}{%
\pgfpathmoveto{\pgfqpoint{0.000000in}{0.000000in}}%
\pgfpathlineto{\pgfqpoint{0.000000in}{-0.048611in}}%
\pgfusepath{stroke,fill}%
}%
\begin{pgfscope}%
\pgfsys@transformshift{0.675453in}{0.553781in}%
\pgfsys@useobject{currentmarker}{}%
\end{pgfscope}%
\end{pgfscope}%
\begin{pgfscope}%
\definecolor{textcolor}{rgb}{0.000000,0.000000,0.000000}%
\pgfsetstrokecolor{textcolor}%
\pgfsetfillcolor{textcolor}%
\pgftext[x=0.675453in,y=0.456558in,,top]{\color{textcolor}\rmfamily\fontsize{8.000000}{9.600000}\selectfont \(\displaystyle {0.0}\)}%
\end{pgfscope}%
\begin{pgfscope}%
\pgfpathrectangle{\pgfqpoint{0.572918in}{0.553781in}}{\pgfqpoint{5.478282in}{2.095553in}}%
\pgfusepath{clip}%
\pgfsetbuttcap%
\pgfsetroundjoin%
\pgfsetlinewidth{0.501875pt}%
\definecolor{currentstroke}{rgb}{0.690196,0.690196,0.690196}%
\pgfsetstrokecolor{currentstroke}%
\pgfsetstrokeopacity{0.500000}%
\pgfsetdash{{0.500000pt}{0.825000pt}}{0.000000pt}%
\pgfpathmoveto{\pgfqpoint{1.554322in}{0.553781in}}%
\pgfpathlineto{\pgfqpoint{1.554322in}{2.649333in}}%
\pgfusepath{stroke}%
\end{pgfscope}%
\begin{pgfscope}%
\pgfsetbuttcap%
\pgfsetroundjoin%
\definecolor{currentfill}{rgb}{0.000000,0.000000,0.000000}%
\pgfsetfillcolor{currentfill}%
\pgfsetlinewidth{0.803000pt}%
\definecolor{currentstroke}{rgb}{0.000000,0.000000,0.000000}%
\pgfsetstrokecolor{currentstroke}%
\pgfsetdash{}{0pt}%
\pgfsys@defobject{currentmarker}{\pgfqpoint{0.000000in}{-0.048611in}}{\pgfqpoint{0.000000in}{0.000000in}}{%
\pgfpathmoveto{\pgfqpoint{0.000000in}{0.000000in}}%
\pgfpathlineto{\pgfqpoint{0.000000in}{-0.048611in}}%
\pgfusepath{stroke,fill}%
}%
\begin{pgfscope}%
\pgfsys@transformshift{1.554322in}{0.553781in}%
\pgfsys@useobject{currentmarker}{}%
\end{pgfscope}%
\end{pgfscope}%
\begin{pgfscope}%
\definecolor{textcolor}{rgb}{0.000000,0.000000,0.000000}%
\pgfsetstrokecolor{textcolor}%
\pgfsetfillcolor{textcolor}%
\pgftext[x=1.554322in,y=0.456558in,,top]{\color{textcolor}\rmfamily\fontsize{8.000000}{9.600000}\selectfont \(\displaystyle {0.5}\)}%
\end{pgfscope}%
\begin{pgfscope}%
\pgfpathrectangle{\pgfqpoint{0.572918in}{0.553781in}}{\pgfqpoint{5.478282in}{2.095553in}}%
\pgfusepath{clip}%
\pgfsetbuttcap%
\pgfsetroundjoin%
\pgfsetlinewidth{0.501875pt}%
\definecolor{currentstroke}{rgb}{0.690196,0.690196,0.690196}%
\pgfsetstrokecolor{currentstroke}%
\pgfsetstrokeopacity{0.500000}%
\pgfsetdash{{0.500000pt}{0.825000pt}}{0.000000pt}%
\pgfpathmoveto{\pgfqpoint{2.433190in}{0.553781in}}%
\pgfpathlineto{\pgfqpoint{2.433190in}{2.649333in}}%
\pgfusepath{stroke}%
\end{pgfscope}%
\begin{pgfscope}%
\pgfsetbuttcap%
\pgfsetroundjoin%
\definecolor{currentfill}{rgb}{0.000000,0.000000,0.000000}%
\pgfsetfillcolor{currentfill}%
\pgfsetlinewidth{0.803000pt}%
\definecolor{currentstroke}{rgb}{0.000000,0.000000,0.000000}%
\pgfsetstrokecolor{currentstroke}%
\pgfsetdash{}{0pt}%
\pgfsys@defobject{currentmarker}{\pgfqpoint{0.000000in}{-0.048611in}}{\pgfqpoint{0.000000in}{0.000000in}}{%
\pgfpathmoveto{\pgfqpoint{0.000000in}{0.000000in}}%
\pgfpathlineto{\pgfqpoint{0.000000in}{-0.048611in}}%
\pgfusepath{stroke,fill}%
}%
\begin{pgfscope}%
\pgfsys@transformshift{2.433190in}{0.553781in}%
\pgfsys@useobject{currentmarker}{}%
\end{pgfscope}%
\end{pgfscope}%
\begin{pgfscope}%
\definecolor{textcolor}{rgb}{0.000000,0.000000,0.000000}%
\pgfsetstrokecolor{textcolor}%
\pgfsetfillcolor{textcolor}%
\pgftext[x=2.433190in,y=0.456558in,,top]{\color{textcolor}\rmfamily\fontsize{8.000000}{9.600000}\selectfont \(\displaystyle {1.0}\)}%
\end{pgfscope}%
\begin{pgfscope}%
\pgfpathrectangle{\pgfqpoint{0.572918in}{0.553781in}}{\pgfqpoint{5.478282in}{2.095553in}}%
\pgfusepath{clip}%
\pgfsetbuttcap%
\pgfsetroundjoin%
\pgfsetlinewidth{0.501875pt}%
\definecolor{currentstroke}{rgb}{0.690196,0.690196,0.690196}%
\pgfsetstrokecolor{currentstroke}%
\pgfsetstrokeopacity{0.500000}%
\pgfsetdash{{0.500000pt}{0.825000pt}}{0.000000pt}%
\pgfpathmoveto{\pgfqpoint{3.312059in}{0.553781in}}%
\pgfpathlineto{\pgfqpoint{3.312059in}{2.649333in}}%
\pgfusepath{stroke}%
\end{pgfscope}%
\begin{pgfscope}%
\pgfsetbuttcap%
\pgfsetroundjoin%
\definecolor{currentfill}{rgb}{0.000000,0.000000,0.000000}%
\pgfsetfillcolor{currentfill}%
\pgfsetlinewidth{0.803000pt}%
\definecolor{currentstroke}{rgb}{0.000000,0.000000,0.000000}%
\pgfsetstrokecolor{currentstroke}%
\pgfsetdash{}{0pt}%
\pgfsys@defobject{currentmarker}{\pgfqpoint{0.000000in}{-0.048611in}}{\pgfqpoint{0.000000in}{0.000000in}}{%
\pgfpathmoveto{\pgfqpoint{0.000000in}{0.000000in}}%
\pgfpathlineto{\pgfqpoint{0.000000in}{-0.048611in}}%
\pgfusepath{stroke,fill}%
}%
\begin{pgfscope}%
\pgfsys@transformshift{3.312059in}{0.553781in}%
\pgfsys@useobject{currentmarker}{}%
\end{pgfscope}%
\end{pgfscope}%
\begin{pgfscope}%
\definecolor{textcolor}{rgb}{0.000000,0.000000,0.000000}%
\pgfsetstrokecolor{textcolor}%
\pgfsetfillcolor{textcolor}%
\pgftext[x=3.312059in,y=0.456558in,,top]{\color{textcolor}\rmfamily\fontsize{8.000000}{9.600000}\selectfont \(\displaystyle {1.5}\)}%
\end{pgfscope}%
\begin{pgfscope}%
\pgfpathrectangle{\pgfqpoint{0.572918in}{0.553781in}}{\pgfqpoint{5.478282in}{2.095553in}}%
\pgfusepath{clip}%
\pgfsetbuttcap%
\pgfsetroundjoin%
\pgfsetlinewidth{0.501875pt}%
\definecolor{currentstroke}{rgb}{0.690196,0.690196,0.690196}%
\pgfsetstrokecolor{currentstroke}%
\pgfsetstrokeopacity{0.500000}%
\pgfsetdash{{0.500000pt}{0.825000pt}}{0.000000pt}%
\pgfpathmoveto{\pgfqpoint{4.190928in}{0.553781in}}%
\pgfpathlineto{\pgfqpoint{4.190928in}{2.649333in}}%
\pgfusepath{stroke}%
\end{pgfscope}%
\begin{pgfscope}%
\pgfsetbuttcap%
\pgfsetroundjoin%
\definecolor{currentfill}{rgb}{0.000000,0.000000,0.000000}%
\pgfsetfillcolor{currentfill}%
\pgfsetlinewidth{0.803000pt}%
\definecolor{currentstroke}{rgb}{0.000000,0.000000,0.000000}%
\pgfsetstrokecolor{currentstroke}%
\pgfsetdash{}{0pt}%
\pgfsys@defobject{currentmarker}{\pgfqpoint{0.000000in}{-0.048611in}}{\pgfqpoint{0.000000in}{0.000000in}}{%
\pgfpathmoveto{\pgfqpoint{0.000000in}{0.000000in}}%
\pgfpathlineto{\pgfqpoint{0.000000in}{-0.048611in}}%
\pgfusepath{stroke,fill}%
}%
\begin{pgfscope}%
\pgfsys@transformshift{4.190928in}{0.553781in}%
\pgfsys@useobject{currentmarker}{}%
\end{pgfscope}%
\end{pgfscope}%
\begin{pgfscope}%
\definecolor{textcolor}{rgb}{0.000000,0.000000,0.000000}%
\pgfsetstrokecolor{textcolor}%
\pgfsetfillcolor{textcolor}%
\pgftext[x=4.190928in,y=0.456558in,,top]{\color{textcolor}\rmfamily\fontsize{8.000000}{9.600000}\selectfont \(\displaystyle {2.0}\)}%
\end{pgfscope}%
\begin{pgfscope}%
\pgfpathrectangle{\pgfqpoint{0.572918in}{0.553781in}}{\pgfqpoint{5.478282in}{2.095553in}}%
\pgfusepath{clip}%
\pgfsetbuttcap%
\pgfsetroundjoin%
\pgfsetlinewidth{0.501875pt}%
\definecolor{currentstroke}{rgb}{0.690196,0.690196,0.690196}%
\pgfsetstrokecolor{currentstroke}%
\pgfsetstrokeopacity{0.500000}%
\pgfsetdash{{0.500000pt}{0.825000pt}}{0.000000pt}%
\pgfpathmoveto{\pgfqpoint{5.069797in}{0.553781in}}%
\pgfpathlineto{\pgfqpoint{5.069797in}{2.649333in}}%
\pgfusepath{stroke}%
\end{pgfscope}%
\begin{pgfscope}%
\pgfsetbuttcap%
\pgfsetroundjoin%
\definecolor{currentfill}{rgb}{0.000000,0.000000,0.000000}%
\pgfsetfillcolor{currentfill}%
\pgfsetlinewidth{0.803000pt}%
\definecolor{currentstroke}{rgb}{0.000000,0.000000,0.000000}%
\pgfsetstrokecolor{currentstroke}%
\pgfsetdash{}{0pt}%
\pgfsys@defobject{currentmarker}{\pgfqpoint{0.000000in}{-0.048611in}}{\pgfqpoint{0.000000in}{0.000000in}}{%
\pgfpathmoveto{\pgfqpoint{0.000000in}{0.000000in}}%
\pgfpathlineto{\pgfqpoint{0.000000in}{-0.048611in}}%
\pgfusepath{stroke,fill}%
}%
\begin{pgfscope}%
\pgfsys@transformshift{5.069797in}{0.553781in}%
\pgfsys@useobject{currentmarker}{}%
\end{pgfscope}%
\end{pgfscope}%
\begin{pgfscope}%
\definecolor{textcolor}{rgb}{0.000000,0.000000,0.000000}%
\pgfsetstrokecolor{textcolor}%
\pgfsetfillcolor{textcolor}%
\pgftext[x=5.069797in,y=0.456558in,,top]{\color{textcolor}\rmfamily\fontsize{8.000000}{9.600000}\selectfont \(\displaystyle {2.5}\)}%
\end{pgfscope}%
\begin{pgfscope}%
\pgfpathrectangle{\pgfqpoint{0.572918in}{0.553781in}}{\pgfqpoint{5.478282in}{2.095553in}}%
\pgfusepath{clip}%
\pgfsetbuttcap%
\pgfsetroundjoin%
\pgfsetlinewidth{0.501875pt}%
\definecolor{currentstroke}{rgb}{0.690196,0.690196,0.690196}%
\pgfsetstrokecolor{currentstroke}%
\pgfsetstrokeopacity{0.500000}%
\pgfsetdash{{0.500000pt}{0.825000pt}}{0.000000pt}%
\pgfpathmoveto{\pgfqpoint{5.948665in}{0.553781in}}%
\pgfpathlineto{\pgfqpoint{5.948665in}{2.649333in}}%
\pgfusepath{stroke}%
\end{pgfscope}%
\begin{pgfscope}%
\pgfsetbuttcap%
\pgfsetroundjoin%
\definecolor{currentfill}{rgb}{0.000000,0.000000,0.000000}%
\pgfsetfillcolor{currentfill}%
\pgfsetlinewidth{0.803000pt}%
\definecolor{currentstroke}{rgb}{0.000000,0.000000,0.000000}%
\pgfsetstrokecolor{currentstroke}%
\pgfsetdash{}{0pt}%
\pgfsys@defobject{currentmarker}{\pgfqpoint{0.000000in}{-0.048611in}}{\pgfqpoint{0.000000in}{0.000000in}}{%
\pgfpathmoveto{\pgfqpoint{0.000000in}{0.000000in}}%
\pgfpathlineto{\pgfqpoint{0.000000in}{-0.048611in}}%
\pgfusepath{stroke,fill}%
}%
\begin{pgfscope}%
\pgfsys@transformshift{5.948665in}{0.553781in}%
\pgfsys@useobject{currentmarker}{}%
\end{pgfscope}%
\end{pgfscope}%
\begin{pgfscope}%
\definecolor{textcolor}{rgb}{0.000000,0.000000,0.000000}%
\pgfsetstrokecolor{textcolor}%
\pgfsetfillcolor{textcolor}%
\pgftext[x=5.948665in,y=0.456558in,,top]{\color{textcolor}\rmfamily\fontsize{8.000000}{9.600000}\selectfont \(\displaystyle {3.0}\)}%
\end{pgfscope}%
\begin{pgfscope}%
\definecolor{textcolor}{rgb}{0.000000,0.000000,0.000000}%
\pgfsetstrokecolor{textcolor}%
\pgfsetfillcolor{textcolor}%
\pgftext[x=3.312059in,y=0.302336in,,top]{\color{textcolor}\rmfamily\fontsize{10.950000}{13.140000}\selectfont \(\displaystyle \log_{10}(E_{\textup{true}}) \, \left[ E / \textup{GeV} \right]\)}%
\end{pgfscope}%
\begin{pgfscope}%
\pgfpathrectangle{\pgfqpoint{0.572918in}{0.553781in}}{\pgfqpoint{5.478282in}{2.095553in}}%
\pgfusepath{clip}%
\pgfsetbuttcap%
\pgfsetroundjoin%
\pgfsetlinewidth{0.501875pt}%
\definecolor{currentstroke}{rgb}{0.690196,0.690196,0.690196}%
\pgfsetstrokecolor{currentstroke}%
\pgfsetstrokeopacity{0.500000}%
\pgfsetdash{{0.500000pt}{0.825000pt}}{0.000000pt}%
\pgfpathmoveto{\pgfqpoint{0.572918in}{0.860703in}}%
\pgfpathlineto{\pgfqpoint{6.051200in}{0.860703in}}%
\pgfusepath{stroke}%
\end{pgfscope}%
\begin{pgfscope}%
\pgfsetbuttcap%
\pgfsetroundjoin%
\definecolor{currentfill}{rgb}{0.000000,0.000000,0.000000}%
\pgfsetfillcolor{currentfill}%
\pgfsetlinewidth{0.803000pt}%
\definecolor{currentstroke}{rgb}{0.000000,0.000000,0.000000}%
\pgfsetstrokecolor{currentstroke}%
\pgfsetdash{}{0pt}%
\pgfsys@defobject{currentmarker}{\pgfqpoint{-0.048611in}{0.000000in}}{\pgfqpoint{-0.000000in}{0.000000in}}{%
\pgfpathmoveto{\pgfqpoint{-0.000000in}{0.000000in}}%
\pgfpathlineto{\pgfqpoint{-0.048611in}{0.000000in}}%
\pgfusepath{stroke,fill}%
}%
\begin{pgfscope}%
\pgfsys@transformshift{0.572918in}{0.860703in}%
\pgfsys@useobject{currentmarker}{}%
\end{pgfscope}%
\end{pgfscope}%
\begin{pgfscope}%
\definecolor{textcolor}{rgb}{0.000000,0.000000,0.000000}%
\pgfsetstrokecolor{textcolor}%
\pgfsetfillcolor{textcolor}%
\pgftext[x=0.357639in, y=0.822147in, left, base]{\color{textcolor}\rmfamily\fontsize{8.000000}{9.600000}\selectfont \(\displaystyle {15}\)}%
\end{pgfscope}%
\begin{pgfscope}%
\pgfpathrectangle{\pgfqpoint{0.572918in}{0.553781in}}{\pgfqpoint{5.478282in}{2.095553in}}%
\pgfusepath{clip}%
\pgfsetbuttcap%
\pgfsetroundjoin%
\pgfsetlinewidth{0.501875pt}%
\definecolor{currentstroke}{rgb}{0.690196,0.690196,0.690196}%
\pgfsetstrokecolor{currentstroke}%
\pgfsetstrokeopacity{0.500000}%
\pgfsetdash{{0.500000pt}{0.825000pt}}{0.000000pt}%
\pgfpathmoveto{\pgfqpoint{0.572918in}{1.319044in}}%
\pgfpathlineto{\pgfqpoint{6.051200in}{1.319044in}}%
\pgfusepath{stroke}%
\end{pgfscope}%
\begin{pgfscope}%
\pgfsetbuttcap%
\pgfsetroundjoin%
\definecolor{currentfill}{rgb}{0.000000,0.000000,0.000000}%
\pgfsetfillcolor{currentfill}%
\pgfsetlinewidth{0.803000pt}%
\definecolor{currentstroke}{rgb}{0.000000,0.000000,0.000000}%
\pgfsetstrokecolor{currentstroke}%
\pgfsetdash{}{0pt}%
\pgfsys@defobject{currentmarker}{\pgfqpoint{-0.048611in}{0.000000in}}{\pgfqpoint{-0.000000in}{0.000000in}}{%
\pgfpathmoveto{\pgfqpoint{-0.000000in}{0.000000in}}%
\pgfpathlineto{\pgfqpoint{-0.048611in}{0.000000in}}%
\pgfusepath{stroke,fill}%
}%
\begin{pgfscope}%
\pgfsys@transformshift{0.572918in}{1.319044in}%
\pgfsys@useobject{currentmarker}{}%
\end{pgfscope}%
\end{pgfscope}%
\begin{pgfscope}%
\definecolor{textcolor}{rgb}{0.000000,0.000000,0.000000}%
\pgfsetstrokecolor{textcolor}%
\pgfsetfillcolor{textcolor}%
\pgftext[x=0.357639in, y=1.280488in, left, base]{\color{textcolor}\rmfamily\fontsize{8.000000}{9.600000}\selectfont \(\displaystyle {20}\)}%
\end{pgfscope}%
\begin{pgfscope}%
\pgfpathrectangle{\pgfqpoint{0.572918in}{0.553781in}}{\pgfqpoint{5.478282in}{2.095553in}}%
\pgfusepath{clip}%
\pgfsetbuttcap%
\pgfsetroundjoin%
\pgfsetlinewidth{0.501875pt}%
\definecolor{currentstroke}{rgb}{0.690196,0.690196,0.690196}%
\pgfsetstrokecolor{currentstroke}%
\pgfsetstrokeopacity{0.500000}%
\pgfsetdash{{0.500000pt}{0.825000pt}}{0.000000pt}%
\pgfpathmoveto{\pgfqpoint{0.572918in}{1.777385in}}%
\pgfpathlineto{\pgfqpoint{6.051200in}{1.777385in}}%
\pgfusepath{stroke}%
\end{pgfscope}%
\begin{pgfscope}%
\pgfsetbuttcap%
\pgfsetroundjoin%
\definecolor{currentfill}{rgb}{0.000000,0.000000,0.000000}%
\pgfsetfillcolor{currentfill}%
\pgfsetlinewidth{0.803000pt}%
\definecolor{currentstroke}{rgb}{0.000000,0.000000,0.000000}%
\pgfsetstrokecolor{currentstroke}%
\pgfsetdash{}{0pt}%
\pgfsys@defobject{currentmarker}{\pgfqpoint{-0.048611in}{0.000000in}}{\pgfqpoint{-0.000000in}{0.000000in}}{%
\pgfpathmoveto{\pgfqpoint{-0.000000in}{0.000000in}}%
\pgfpathlineto{\pgfqpoint{-0.048611in}{0.000000in}}%
\pgfusepath{stroke,fill}%
}%
\begin{pgfscope}%
\pgfsys@transformshift{0.572918in}{1.777385in}%
\pgfsys@useobject{currentmarker}{}%
\end{pgfscope}%
\end{pgfscope}%
\begin{pgfscope}%
\definecolor{textcolor}{rgb}{0.000000,0.000000,0.000000}%
\pgfsetstrokecolor{textcolor}%
\pgfsetfillcolor{textcolor}%
\pgftext[x=0.357639in, y=1.738829in, left, base]{\color{textcolor}\rmfamily\fontsize{8.000000}{9.600000}\selectfont \(\displaystyle {25}\)}%
\end{pgfscope}%
\begin{pgfscope}%
\pgfpathrectangle{\pgfqpoint{0.572918in}{0.553781in}}{\pgfqpoint{5.478282in}{2.095553in}}%
\pgfusepath{clip}%
\pgfsetbuttcap%
\pgfsetroundjoin%
\pgfsetlinewidth{0.501875pt}%
\definecolor{currentstroke}{rgb}{0.690196,0.690196,0.690196}%
\pgfsetstrokecolor{currentstroke}%
\pgfsetstrokeopacity{0.500000}%
\pgfsetdash{{0.500000pt}{0.825000pt}}{0.000000pt}%
\pgfpathmoveto{\pgfqpoint{0.572918in}{2.235726in}}%
\pgfpathlineto{\pgfqpoint{6.051200in}{2.235726in}}%
\pgfusepath{stroke}%
\end{pgfscope}%
\begin{pgfscope}%
\pgfsetbuttcap%
\pgfsetroundjoin%
\definecolor{currentfill}{rgb}{0.000000,0.000000,0.000000}%
\pgfsetfillcolor{currentfill}%
\pgfsetlinewidth{0.803000pt}%
\definecolor{currentstroke}{rgb}{0.000000,0.000000,0.000000}%
\pgfsetstrokecolor{currentstroke}%
\pgfsetdash{}{0pt}%
\pgfsys@defobject{currentmarker}{\pgfqpoint{-0.048611in}{0.000000in}}{\pgfqpoint{-0.000000in}{0.000000in}}{%
\pgfpathmoveto{\pgfqpoint{-0.000000in}{0.000000in}}%
\pgfpathlineto{\pgfqpoint{-0.048611in}{0.000000in}}%
\pgfusepath{stroke,fill}%
}%
\begin{pgfscope}%
\pgfsys@transformshift{0.572918in}{2.235726in}%
\pgfsys@useobject{currentmarker}{}%
\end{pgfscope}%
\end{pgfscope}%
\begin{pgfscope}%
\definecolor{textcolor}{rgb}{0.000000,0.000000,0.000000}%
\pgfsetstrokecolor{textcolor}%
\pgfsetfillcolor{textcolor}%
\pgftext[x=0.357639in, y=2.197171in, left, base]{\color{textcolor}\rmfamily\fontsize{8.000000}{9.600000}\selectfont \(\displaystyle {30}\)}%
\end{pgfscope}%
\begin{pgfscope}%
\definecolor{textcolor}{rgb}{0.000000,0.000000,0.000000}%
\pgfsetstrokecolor{textcolor}%
\pgfsetfillcolor{textcolor}%
\pgftext[x=0.302083in,y=1.601557in,,bottom,rotate=90.000000]{\color{textcolor}\rmfamily\fontsize{10.950000}{13.140000}\selectfont IQR / 1.349 \(\displaystyle \left[ \textup{deg} \right]\)}%
\end{pgfscope}%
\begin{pgfscope}%
\pgfpathrectangle{\pgfqpoint{0.572918in}{0.553781in}}{\pgfqpoint{5.478282in}{2.095553in}}%
\pgfusepath{clip}%
\pgfsetbuttcap%
\pgfsetroundjoin%
\pgfsetlinewidth{1.505625pt}%
\definecolor{currentstroke}{rgb}{0.313725,0.317647,0.309804}%
\pgfsetstrokecolor{currentstroke}%
\pgfsetstrokeopacity{0.900000}%
\pgfsetdash{}{0pt}%
\pgfpathmoveto{\pgfqpoint{0.821931in}{1.832104in}}%
\pgfpathlineto{\pgfqpoint{0.821931in}{2.554081in}}%
\pgfusepath{stroke}%
\end{pgfscope}%
\begin{pgfscope}%
\pgfpathrectangle{\pgfqpoint{0.572918in}{0.553781in}}{\pgfqpoint{5.478282in}{2.095553in}}%
\pgfusepath{clip}%
\pgfsetbuttcap%
\pgfsetroundjoin%
\pgfsetlinewidth{1.505625pt}%
\definecolor{currentstroke}{rgb}{0.313725,0.317647,0.309804}%
\pgfsetstrokecolor{currentstroke}%
\pgfsetstrokeopacity{0.900000}%
\pgfsetdash{}{0pt}%
\pgfpathmoveto{\pgfqpoint{1.114887in}{1.948338in}}%
\pgfpathlineto{\pgfqpoint{1.114887in}{2.228136in}}%
\pgfusepath{stroke}%
\end{pgfscope}%
\begin{pgfscope}%
\pgfpathrectangle{\pgfqpoint{0.572918in}{0.553781in}}{\pgfqpoint{5.478282in}{2.095553in}}%
\pgfusepath{clip}%
\pgfsetbuttcap%
\pgfsetroundjoin%
\pgfsetlinewidth{1.505625pt}%
\definecolor{currentstroke}{rgb}{0.313725,0.317647,0.309804}%
\pgfsetstrokecolor{currentstroke}%
\pgfsetstrokeopacity{0.900000}%
\pgfsetdash{}{0pt}%
\pgfpathmoveto{\pgfqpoint{1.407844in}{1.840374in}}%
\pgfpathlineto{\pgfqpoint{1.407844in}{1.999711in}}%
\pgfusepath{stroke}%
\end{pgfscope}%
\begin{pgfscope}%
\pgfpathrectangle{\pgfqpoint{0.572918in}{0.553781in}}{\pgfqpoint{5.478282in}{2.095553in}}%
\pgfusepath{clip}%
\pgfsetbuttcap%
\pgfsetroundjoin%
\pgfsetlinewidth{1.505625pt}%
\definecolor{currentstroke}{rgb}{0.313725,0.317647,0.309804}%
\pgfsetstrokecolor{currentstroke}%
\pgfsetstrokeopacity{0.900000}%
\pgfsetdash{}{0pt}%
\pgfpathmoveto{\pgfqpoint{1.700800in}{1.867913in}}%
\pgfpathlineto{\pgfqpoint{1.700800in}{1.979916in}}%
\pgfusepath{stroke}%
\end{pgfscope}%
\begin{pgfscope}%
\pgfpathrectangle{\pgfqpoint{0.572918in}{0.553781in}}{\pgfqpoint{5.478282in}{2.095553in}}%
\pgfusepath{clip}%
\pgfsetbuttcap%
\pgfsetroundjoin%
\pgfsetlinewidth{1.505625pt}%
\definecolor{currentstroke}{rgb}{0.313725,0.317647,0.309804}%
\pgfsetstrokecolor{currentstroke}%
\pgfsetstrokeopacity{0.900000}%
\pgfsetdash{}{0pt}%
\pgfpathmoveto{\pgfqpoint{1.993756in}{1.963349in}}%
\pgfpathlineto{\pgfqpoint{1.993756in}{2.054461in}}%
\pgfusepath{stroke}%
\end{pgfscope}%
\begin{pgfscope}%
\pgfpathrectangle{\pgfqpoint{0.572918in}{0.553781in}}{\pgfqpoint{5.478282in}{2.095553in}}%
\pgfusepath{clip}%
\pgfsetbuttcap%
\pgfsetroundjoin%
\pgfsetlinewidth{1.505625pt}%
\definecolor{currentstroke}{rgb}{0.313725,0.317647,0.309804}%
\pgfsetstrokecolor{currentstroke}%
\pgfsetstrokeopacity{0.900000}%
\pgfsetdash{}{0pt}%
\pgfpathmoveto{\pgfqpoint{2.286712in}{1.934833in}}%
\pgfpathlineto{\pgfqpoint{2.286712in}{2.001483in}}%
\pgfusepath{stroke}%
\end{pgfscope}%
\begin{pgfscope}%
\pgfpathrectangle{\pgfqpoint{0.572918in}{0.553781in}}{\pgfqpoint{5.478282in}{2.095553in}}%
\pgfusepath{clip}%
\pgfsetbuttcap%
\pgfsetroundjoin%
\pgfsetlinewidth{1.505625pt}%
\definecolor{currentstroke}{rgb}{0.313725,0.317647,0.309804}%
\pgfsetstrokecolor{currentstroke}%
\pgfsetstrokeopacity{0.900000}%
\pgfsetdash{}{0pt}%
\pgfpathmoveto{\pgfqpoint{2.579669in}{1.818698in}}%
\pgfpathlineto{\pgfqpoint{2.579669in}{1.884550in}}%
\pgfusepath{stroke}%
\end{pgfscope}%
\begin{pgfscope}%
\pgfpathrectangle{\pgfqpoint{0.572918in}{0.553781in}}{\pgfqpoint{5.478282in}{2.095553in}}%
\pgfusepath{clip}%
\pgfsetbuttcap%
\pgfsetroundjoin%
\pgfsetlinewidth{1.505625pt}%
\definecolor{currentstroke}{rgb}{0.313725,0.317647,0.309804}%
\pgfsetstrokecolor{currentstroke}%
\pgfsetstrokeopacity{0.900000}%
\pgfsetdash{}{0pt}%
\pgfpathmoveto{\pgfqpoint{2.872625in}{1.733897in}}%
\pgfpathlineto{\pgfqpoint{2.872625in}{1.794023in}}%
\pgfusepath{stroke}%
\end{pgfscope}%
\begin{pgfscope}%
\pgfpathrectangle{\pgfqpoint{0.572918in}{0.553781in}}{\pgfqpoint{5.478282in}{2.095553in}}%
\pgfusepath{clip}%
\pgfsetbuttcap%
\pgfsetroundjoin%
\pgfsetlinewidth{1.505625pt}%
\definecolor{currentstroke}{rgb}{0.313725,0.317647,0.309804}%
\pgfsetstrokecolor{currentstroke}%
\pgfsetstrokeopacity{0.900000}%
\pgfsetdash{}{0pt}%
\pgfpathmoveto{\pgfqpoint{3.165581in}{1.547847in}}%
\pgfpathlineto{\pgfqpoint{3.165581in}{1.610118in}}%
\pgfusepath{stroke}%
\end{pgfscope}%
\begin{pgfscope}%
\pgfpathrectangle{\pgfqpoint{0.572918in}{0.553781in}}{\pgfqpoint{5.478282in}{2.095553in}}%
\pgfusepath{clip}%
\pgfsetbuttcap%
\pgfsetroundjoin%
\pgfsetlinewidth{1.505625pt}%
\definecolor{currentstroke}{rgb}{0.313725,0.317647,0.309804}%
\pgfsetstrokecolor{currentstroke}%
\pgfsetstrokeopacity{0.900000}%
\pgfsetdash{}{0pt}%
\pgfpathmoveto{\pgfqpoint{3.458537in}{1.395274in}}%
\pgfpathlineto{\pgfqpoint{3.458537in}{1.453479in}}%
\pgfusepath{stroke}%
\end{pgfscope}%
\begin{pgfscope}%
\pgfpathrectangle{\pgfqpoint{0.572918in}{0.553781in}}{\pgfqpoint{5.478282in}{2.095553in}}%
\pgfusepath{clip}%
\pgfsetbuttcap%
\pgfsetroundjoin%
\pgfsetlinewidth{1.505625pt}%
\definecolor{currentstroke}{rgb}{0.313725,0.317647,0.309804}%
\pgfsetstrokecolor{currentstroke}%
\pgfsetstrokeopacity{0.900000}%
\pgfsetdash{}{0pt}%
\pgfpathmoveto{\pgfqpoint{3.751494in}{1.265668in}}%
\pgfpathlineto{\pgfqpoint{3.751494in}{1.320713in}}%
\pgfusepath{stroke}%
\end{pgfscope}%
\begin{pgfscope}%
\pgfpathrectangle{\pgfqpoint{0.572918in}{0.553781in}}{\pgfqpoint{5.478282in}{2.095553in}}%
\pgfusepath{clip}%
\pgfsetbuttcap%
\pgfsetroundjoin%
\pgfsetlinewidth{1.505625pt}%
\definecolor{currentstroke}{rgb}{0.313725,0.317647,0.309804}%
\pgfsetstrokecolor{currentstroke}%
\pgfsetstrokeopacity{0.900000}%
\pgfsetdash{}{0pt}%
\pgfpathmoveto{\pgfqpoint{4.044450in}{1.105244in}}%
\pgfpathlineto{\pgfqpoint{4.044450in}{1.171330in}}%
\pgfusepath{stroke}%
\end{pgfscope}%
\begin{pgfscope}%
\pgfpathrectangle{\pgfqpoint{0.572918in}{0.553781in}}{\pgfqpoint{5.478282in}{2.095553in}}%
\pgfusepath{clip}%
\pgfsetbuttcap%
\pgfsetroundjoin%
\pgfsetlinewidth{1.505625pt}%
\definecolor{currentstroke}{rgb}{0.313725,0.317647,0.309804}%
\pgfsetstrokecolor{currentstroke}%
\pgfsetstrokeopacity{0.900000}%
\pgfsetdash{}{0pt}%
\pgfpathmoveto{\pgfqpoint{4.337406in}{0.956867in}}%
\pgfpathlineto{\pgfqpoint{4.337406in}{1.024811in}}%
\pgfusepath{stroke}%
\end{pgfscope}%
\begin{pgfscope}%
\pgfpathrectangle{\pgfqpoint{0.572918in}{0.553781in}}{\pgfqpoint{5.478282in}{2.095553in}}%
\pgfusepath{clip}%
\pgfsetbuttcap%
\pgfsetroundjoin%
\pgfsetlinewidth{1.505625pt}%
\definecolor{currentstroke}{rgb}{0.313725,0.317647,0.309804}%
\pgfsetstrokecolor{currentstroke}%
\pgfsetstrokeopacity{0.900000}%
\pgfsetdash{}{0pt}%
\pgfpathmoveto{\pgfqpoint{4.630362in}{0.791286in}}%
\pgfpathlineto{\pgfqpoint{4.630362in}{0.869426in}}%
\pgfusepath{stroke}%
\end{pgfscope}%
\begin{pgfscope}%
\pgfpathrectangle{\pgfqpoint{0.572918in}{0.553781in}}{\pgfqpoint{5.478282in}{2.095553in}}%
\pgfusepath{clip}%
\pgfsetbuttcap%
\pgfsetroundjoin%
\pgfsetlinewidth{1.505625pt}%
\definecolor{currentstroke}{rgb}{0.313725,0.317647,0.309804}%
\pgfsetstrokecolor{currentstroke}%
\pgfsetstrokeopacity{0.900000}%
\pgfsetdash{}{0pt}%
\pgfpathmoveto{\pgfqpoint{4.923318in}{0.720291in}}%
\pgfpathlineto{\pgfqpoint{4.923318in}{0.809511in}}%
\pgfusepath{stroke}%
\end{pgfscope}%
\begin{pgfscope}%
\pgfpathrectangle{\pgfqpoint{0.572918in}{0.553781in}}{\pgfqpoint{5.478282in}{2.095553in}}%
\pgfusepath{clip}%
\pgfsetbuttcap%
\pgfsetroundjoin%
\pgfsetlinewidth{1.505625pt}%
\definecolor{currentstroke}{rgb}{0.313725,0.317647,0.309804}%
\pgfsetstrokecolor{currentstroke}%
\pgfsetstrokeopacity{0.900000}%
\pgfsetdash{}{0pt}%
\pgfpathmoveto{\pgfqpoint{5.216275in}{0.678524in}}%
\pgfpathlineto{\pgfqpoint{5.216275in}{0.823589in}}%
\pgfusepath{stroke}%
\end{pgfscope}%
\begin{pgfscope}%
\pgfpathrectangle{\pgfqpoint{0.572918in}{0.553781in}}{\pgfqpoint{5.478282in}{2.095553in}}%
\pgfusepath{clip}%
\pgfsetbuttcap%
\pgfsetroundjoin%
\pgfsetlinewidth{1.505625pt}%
\definecolor{currentstroke}{rgb}{0.313725,0.317647,0.309804}%
\pgfsetstrokecolor{currentstroke}%
\pgfsetstrokeopacity{0.900000}%
\pgfsetdash{}{0pt}%
\pgfpathmoveto{\pgfqpoint{5.509231in}{0.649033in}}%
\pgfpathlineto{\pgfqpoint{5.509231in}{0.802225in}}%
\pgfusepath{stroke}%
\end{pgfscope}%
\begin{pgfscope}%
\pgfpathrectangle{\pgfqpoint{0.572918in}{0.553781in}}{\pgfqpoint{5.478282in}{2.095553in}}%
\pgfusepath{clip}%
\pgfsetbuttcap%
\pgfsetroundjoin%
\pgfsetlinewidth{1.505625pt}%
\definecolor{currentstroke}{rgb}{0.313725,0.317647,0.309804}%
\pgfsetstrokecolor{currentstroke}%
\pgfsetstrokeopacity{0.900000}%
\pgfsetdash{}{0pt}%
\pgfpathmoveto{\pgfqpoint{5.802187in}{0.772797in}}%
\pgfpathlineto{\pgfqpoint{5.802187in}{0.976602in}}%
\pgfusepath{stroke}%
\end{pgfscope}%
\begin{pgfscope}%
\pgfpathrectangle{\pgfqpoint{0.572918in}{0.553781in}}{\pgfqpoint{5.478282in}{2.095553in}}%
\pgfusepath{clip}%
\pgfsetbuttcap%
\pgfsetroundjoin%
\pgfsetlinewidth{1.505625pt}%
\definecolor{currentstroke}{rgb}{0.949020,0.372549,0.360784}%
\pgfsetstrokecolor{currentstroke}%
\pgfsetstrokeopacity{0.900000}%
\pgfsetdash{}{0pt}%
\pgfpathmoveto{\pgfqpoint{0.821931in}{1.953690in}}%
\pgfpathlineto{\pgfqpoint{0.821931in}{2.472437in}}%
\pgfusepath{stroke}%
\end{pgfscope}%
\begin{pgfscope}%
\pgfpathrectangle{\pgfqpoint{0.572918in}{0.553781in}}{\pgfqpoint{5.478282in}{2.095553in}}%
\pgfusepath{clip}%
\pgfsetbuttcap%
\pgfsetroundjoin%
\pgfsetlinewidth{1.505625pt}%
\definecolor{currentstroke}{rgb}{0.949020,0.372549,0.360784}%
\pgfsetstrokecolor{currentstroke}%
\pgfsetstrokeopacity{0.900000}%
\pgfsetdash{}{0pt}%
\pgfpathmoveto{\pgfqpoint{1.114887in}{1.902501in}}%
\pgfpathlineto{\pgfqpoint{1.114887in}{2.235890in}}%
\pgfusepath{stroke}%
\end{pgfscope}%
\begin{pgfscope}%
\pgfpathrectangle{\pgfqpoint{0.572918in}{0.553781in}}{\pgfqpoint{5.478282in}{2.095553in}}%
\pgfusepath{clip}%
\pgfsetbuttcap%
\pgfsetroundjoin%
\pgfsetlinewidth{1.505625pt}%
\definecolor{currentstroke}{rgb}{0.949020,0.372549,0.360784}%
\pgfsetstrokecolor{currentstroke}%
\pgfsetstrokeopacity{0.900000}%
\pgfsetdash{}{0pt}%
\pgfpathmoveto{\pgfqpoint{1.407844in}{1.875880in}}%
\pgfpathlineto{\pgfqpoint{1.407844in}{2.039843in}}%
\pgfusepath{stroke}%
\end{pgfscope}%
\begin{pgfscope}%
\pgfpathrectangle{\pgfqpoint{0.572918in}{0.553781in}}{\pgfqpoint{5.478282in}{2.095553in}}%
\pgfusepath{clip}%
\pgfsetbuttcap%
\pgfsetroundjoin%
\pgfsetlinewidth{1.505625pt}%
\definecolor{currentstroke}{rgb}{0.949020,0.372549,0.360784}%
\pgfsetstrokecolor{currentstroke}%
\pgfsetstrokeopacity{0.900000}%
\pgfsetdash{}{0pt}%
\pgfpathmoveto{\pgfqpoint{1.700800in}{1.963755in}}%
\pgfpathlineto{\pgfqpoint{1.700800in}{2.095315in}}%
\pgfusepath{stroke}%
\end{pgfscope}%
\begin{pgfscope}%
\pgfpathrectangle{\pgfqpoint{0.572918in}{0.553781in}}{\pgfqpoint{5.478282in}{2.095553in}}%
\pgfusepath{clip}%
\pgfsetbuttcap%
\pgfsetroundjoin%
\pgfsetlinewidth{1.505625pt}%
\definecolor{currentstroke}{rgb}{0.949020,0.372549,0.360784}%
\pgfsetstrokecolor{currentstroke}%
\pgfsetstrokeopacity{0.900000}%
\pgfsetdash{}{0pt}%
\pgfpathmoveto{\pgfqpoint{1.993756in}{1.996520in}}%
\pgfpathlineto{\pgfqpoint{1.993756in}{2.092320in}}%
\pgfusepath{stroke}%
\end{pgfscope}%
\begin{pgfscope}%
\pgfpathrectangle{\pgfqpoint{0.572918in}{0.553781in}}{\pgfqpoint{5.478282in}{2.095553in}}%
\pgfusepath{clip}%
\pgfsetbuttcap%
\pgfsetroundjoin%
\pgfsetlinewidth{1.505625pt}%
\definecolor{currentstroke}{rgb}{0.949020,0.372549,0.360784}%
\pgfsetstrokecolor{currentstroke}%
\pgfsetstrokeopacity{0.900000}%
\pgfsetdash{}{0pt}%
\pgfpathmoveto{\pgfqpoint{2.286712in}{1.979066in}}%
\pgfpathlineto{\pgfqpoint{2.286712in}{2.060142in}}%
\pgfusepath{stroke}%
\end{pgfscope}%
\begin{pgfscope}%
\pgfpathrectangle{\pgfqpoint{0.572918in}{0.553781in}}{\pgfqpoint{5.478282in}{2.095553in}}%
\pgfusepath{clip}%
\pgfsetbuttcap%
\pgfsetroundjoin%
\pgfsetlinewidth{1.505625pt}%
\definecolor{currentstroke}{rgb}{0.949020,0.372549,0.360784}%
\pgfsetstrokecolor{currentstroke}%
\pgfsetstrokeopacity{0.900000}%
\pgfsetdash{}{0pt}%
\pgfpathmoveto{\pgfqpoint{2.579669in}{1.879980in}}%
\pgfpathlineto{\pgfqpoint{2.579669in}{1.946145in}}%
\pgfusepath{stroke}%
\end{pgfscope}%
\begin{pgfscope}%
\pgfpathrectangle{\pgfqpoint{0.572918in}{0.553781in}}{\pgfqpoint{5.478282in}{2.095553in}}%
\pgfusepath{clip}%
\pgfsetbuttcap%
\pgfsetroundjoin%
\pgfsetlinewidth{1.505625pt}%
\definecolor{currentstroke}{rgb}{0.949020,0.372549,0.360784}%
\pgfsetstrokecolor{currentstroke}%
\pgfsetstrokeopacity{0.900000}%
\pgfsetdash{}{0pt}%
\pgfpathmoveto{\pgfqpoint{2.872625in}{1.764300in}}%
\pgfpathlineto{\pgfqpoint{2.872625in}{1.831262in}}%
\pgfusepath{stroke}%
\end{pgfscope}%
\begin{pgfscope}%
\pgfpathrectangle{\pgfqpoint{0.572918in}{0.553781in}}{\pgfqpoint{5.478282in}{2.095553in}}%
\pgfusepath{clip}%
\pgfsetbuttcap%
\pgfsetroundjoin%
\pgfsetlinewidth{1.505625pt}%
\definecolor{currentstroke}{rgb}{0.949020,0.372549,0.360784}%
\pgfsetstrokecolor{currentstroke}%
\pgfsetstrokeopacity{0.900000}%
\pgfsetdash{}{0pt}%
\pgfpathmoveto{\pgfqpoint{3.165581in}{1.575657in}}%
\pgfpathlineto{\pgfqpoint{3.165581in}{1.631798in}}%
\pgfusepath{stroke}%
\end{pgfscope}%
\begin{pgfscope}%
\pgfpathrectangle{\pgfqpoint{0.572918in}{0.553781in}}{\pgfqpoint{5.478282in}{2.095553in}}%
\pgfusepath{clip}%
\pgfsetbuttcap%
\pgfsetroundjoin%
\pgfsetlinewidth{1.505625pt}%
\definecolor{currentstroke}{rgb}{0.949020,0.372549,0.360784}%
\pgfsetstrokecolor{currentstroke}%
\pgfsetstrokeopacity{0.900000}%
\pgfsetdash{}{0pt}%
\pgfpathmoveto{\pgfqpoint{3.458537in}{1.442712in}}%
\pgfpathlineto{\pgfqpoint{3.458537in}{1.501731in}}%
\pgfusepath{stroke}%
\end{pgfscope}%
\begin{pgfscope}%
\pgfpathrectangle{\pgfqpoint{0.572918in}{0.553781in}}{\pgfqpoint{5.478282in}{2.095553in}}%
\pgfusepath{clip}%
\pgfsetbuttcap%
\pgfsetroundjoin%
\pgfsetlinewidth{1.505625pt}%
\definecolor{currentstroke}{rgb}{0.949020,0.372549,0.360784}%
\pgfsetstrokecolor{currentstroke}%
\pgfsetstrokeopacity{0.900000}%
\pgfsetdash{}{0pt}%
\pgfpathmoveto{\pgfqpoint{3.751494in}{1.293041in}}%
\pgfpathlineto{\pgfqpoint{3.751494in}{1.352306in}}%
\pgfusepath{stroke}%
\end{pgfscope}%
\begin{pgfscope}%
\pgfpathrectangle{\pgfqpoint{0.572918in}{0.553781in}}{\pgfqpoint{5.478282in}{2.095553in}}%
\pgfusepath{clip}%
\pgfsetbuttcap%
\pgfsetroundjoin%
\pgfsetlinewidth{1.505625pt}%
\definecolor{currentstroke}{rgb}{0.949020,0.372549,0.360784}%
\pgfsetstrokecolor{currentstroke}%
\pgfsetstrokeopacity{0.900000}%
\pgfsetdash{}{0pt}%
\pgfpathmoveto{\pgfqpoint{4.044450in}{1.188594in}}%
\pgfpathlineto{\pgfqpoint{4.044450in}{1.259355in}}%
\pgfusepath{stroke}%
\end{pgfscope}%
\begin{pgfscope}%
\pgfpathrectangle{\pgfqpoint{0.572918in}{0.553781in}}{\pgfqpoint{5.478282in}{2.095553in}}%
\pgfusepath{clip}%
\pgfsetbuttcap%
\pgfsetroundjoin%
\pgfsetlinewidth{1.505625pt}%
\definecolor{currentstroke}{rgb}{0.949020,0.372549,0.360784}%
\pgfsetstrokecolor{currentstroke}%
\pgfsetstrokeopacity{0.900000}%
\pgfsetdash{}{0pt}%
\pgfpathmoveto{\pgfqpoint{4.337406in}{1.046456in}}%
\pgfpathlineto{\pgfqpoint{4.337406in}{1.123900in}}%
\pgfusepath{stroke}%
\end{pgfscope}%
\begin{pgfscope}%
\pgfpathrectangle{\pgfqpoint{0.572918in}{0.553781in}}{\pgfqpoint{5.478282in}{2.095553in}}%
\pgfusepath{clip}%
\pgfsetbuttcap%
\pgfsetroundjoin%
\pgfsetlinewidth{1.505625pt}%
\definecolor{currentstroke}{rgb}{0.949020,0.372549,0.360784}%
\pgfsetstrokecolor{currentstroke}%
\pgfsetstrokeopacity{0.900000}%
\pgfsetdash{}{0pt}%
\pgfpathmoveto{\pgfqpoint{4.630362in}{0.899390in}}%
\pgfpathlineto{\pgfqpoint{4.630362in}{0.981067in}}%
\pgfusepath{stroke}%
\end{pgfscope}%
\begin{pgfscope}%
\pgfpathrectangle{\pgfqpoint{0.572918in}{0.553781in}}{\pgfqpoint{5.478282in}{2.095553in}}%
\pgfusepath{clip}%
\pgfsetbuttcap%
\pgfsetroundjoin%
\pgfsetlinewidth{1.505625pt}%
\definecolor{currentstroke}{rgb}{0.949020,0.372549,0.360784}%
\pgfsetstrokecolor{currentstroke}%
\pgfsetstrokeopacity{0.900000}%
\pgfsetdash{}{0pt}%
\pgfpathmoveto{\pgfqpoint{4.923318in}{0.822040in}}%
\pgfpathlineto{\pgfqpoint{4.923318in}{0.925711in}}%
\pgfusepath{stroke}%
\end{pgfscope}%
\begin{pgfscope}%
\pgfpathrectangle{\pgfqpoint{0.572918in}{0.553781in}}{\pgfqpoint{5.478282in}{2.095553in}}%
\pgfusepath{clip}%
\pgfsetbuttcap%
\pgfsetroundjoin%
\pgfsetlinewidth{1.505625pt}%
\definecolor{currentstroke}{rgb}{0.949020,0.372549,0.360784}%
\pgfsetstrokecolor{currentstroke}%
\pgfsetstrokeopacity{0.900000}%
\pgfsetdash{}{0pt}%
\pgfpathmoveto{\pgfqpoint{5.216275in}{0.828332in}}%
\pgfpathlineto{\pgfqpoint{5.216275in}{0.958950in}}%
\pgfusepath{stroke}%
\end{pgfscope}%
\begin{pgfscope}%
\pgfpathrectangle{\pgfqpoint{0.572918in}{0.553781in}}{\pgfqpoint{5.478282in}{2.095553in}}%
\pgfusepath{clip}%
\pgfsetbuttcap%
\pgfsetroundjoin%
\pgfsetlinewidth{1.505625pt}%
\definecolor{currentstroke}{rgb}{0.949020,0.372549,0.360784}%
\pgfsetstrokecolor{currentstroke}%
\pgfsetstrokeopacity{0.900000}%
\pgfsetdash{}{0pt}%
\pgfpathmoveto{\pgfqpoint{5.509231in}{0.701657in}}%
\pgfpathlineto{\pgfqpoint{5.509231in}{0.829812in}}%
\pgfusepath{stroke}%
\end{pgfscope}%
\begin{pgfscope}%
\pgfpathrectangle{\pgfqpoint{0.572918in}{0.553781in}}{\pgfqpoint{5.478282in}{2.095553in}}%
\pgfusepath{clip}%
\pgfsetbuttcap%
\pgfsetroundjoin%
\pgfsetlinewidth{1.505625pt}%
\definecolor{currentstroke}{rgb}{0.949020,0.372549,0.360784}%
\pgfsetstrokecolor{currentstroke}%
\pgfsetstrokeopacity{0.900000}%
\pgfsetdash{}{0pt}%
\pgfpathmoveto{\pgfqpoint{5.802187in}{0.717960in}}%
\pgfpathlineto{\pgfqpoint{5.802187in}{0.895768in}}%
\pgfusepath{stroke}%
\end{pgfscope}%
\begin{pgfscope}%
\pgfpathrectangle{\pgfqpoint{0.572918in}{0.553781in}}{\pgfqpoint{5.478282in}{2.095553in}}%
\pgfusepath{clip}%
\pgfsetbuttcap%
\pgfsetroundjoin%
\definecolor{currentfill}{rgb}{0.313725,0.317647,0.309804}%
\pgfsetfillcolor{currentfill}%
\pgfsetfillopacity{0.900000}%
\pgfsetlinewidth{1.003750pt}%
\definecolor{currentstroke}{rgb}{0.313725,0.317647,0.309804}%
\pgfsetstrokecolor{currentstroke}%
\pgfsetstrokeopacity{0.900000}%
\pgfsetdash{}{0pt}%
\pgfsys@defobject{currentmarker}{\pgfqpoint{-0.013889in}{-0.000000in}}{\pgfqpoint{0.013889in}{0.000000in}}{%
\pgfpathmoveto{\pgfqpoint{0.013889in}{-0.000000in}}%
\pgfpathlineto{\pgfqpoint{-0.013889in}{0.000000in}}%
\pgfusepath{stroke,fill}%
}%
\begin{pgfscope}%
\pgfsys@transformshift{0.821931in}{1.832104in}%
\pgfsys@useobject{currentmarker}{}%
\end{pgfscope}%
\begin{pgfscope}%
\pgfsys@transformshift{1.114887in}{1.948338in}%
\pgfsys@useobject{currentmarker}{}%
\end{pgfscope}%
\begin{pgfscope}%
\pgfsys@transformshift{1.407844in}{1.840374in}%
\pgfsys@useobject{currentmarker}{}%
\end{pgfscope}%
\begin{pgfscope}%
\pgfsys@transformshift{1.700800in}{1.867913in}%
\pgfsys@useobject{currentmarker}{}%
\end{pgfscope}%
\begin{pgfscope}%
\pgfsys@transformshift{1.993756in}{1.963349in}%
\pgfsys@useobject{currentmarker}{}%
\end{pgfscope}%
\begin{pgfscope}%
\pgfsys@transformshift{2.286712in}{1.934833in}%
\pgfsys@useobject{currentmarker}{}%
\end{pgfscope}%
\begin{pgfscope}%
\pgfsys@transformshift{2.579669in}{1.818698in}%
\pgfsys@useobject{currentmarker}{}%
\end{pgfscope}%
\begin{pgfscope}%
\pgfsys@transformshift{2.872625in}{1.733897in}%
\pgfsys@useobject{currentmarker}{}%
\end{pgfscope}%
\begin{pgfscope}%
\pgfsys@transformshift{3.165581in}{1.547847in}%
\pgfsys@useobject{currentmarker}{}%
\end{pgfscope}%
\begin{pgfscope}%
\pgfsys@transformshift{3.458537in}{1.395274in}%
\pgfsys@useobject{currentmarker}{}%
\end{pgfscope}%
\begin{pgfscope}%
\pgfsys@transformshift{3.751494in}{1.265668in}%
\pgfsys@useobject{currentmarker}{}%
\end{pgfscope}%
\begin{pgfscope}%
\pgfsys@transformshift{4.044450in}{1.105244in}%
\pgfsys@useobject{currentmarker}{}%
\end{pgfscope}%
\begin{pgfscope}%
\pgfsys@transformshift{4.337406in}{0.956867in}%
\pgfsys@useobject{currentmarker}{}%
\end{pgfscope}%
\begin{pgfscope}%
\pgfsys@transformshift{4.630362in}{0.791286in}%
\pgfsys@useobject{currentmarker}{}%
\end{pgfscope}%
\begin{pgfscope}%
\pgfsys@transformshift{4.923318in}{0.720291in}%
\pgfsys@useobject{currentmarker}{}%
\end{pgfscope}%
\begin{pgfscope}%
\pgfsys@transformshift{5.216275in}{0.678524in}%
\pgfsys@useobject{currentmarker}{}%
\end{pgfscope}%
\begin{pgfscope}%
\pgfsys@transformshift{5.509231in}{0.649033in}%
\pgfsys@useobject{currentmarker}{}%
\end{pgfscope}%
\begin{pgfscope}%
\pgfsys@transformshift{5.802187in}{0.772797in}%
\pgfsys@useobject{currentmarker}{}%
\end{pgfscope}%
\end{pgfscope}%
\begin{pgfscope}%
\pgfpathrectangle{\pgfqpoint{0.572918in}{0.553781in}}{\pgfqpoint{5.478282in}{2.095553in}}%
\pgfusepath{clip}%
\pgfsetbuttcap%
\pgfsetroundjoin%
\definecolor{currentfill}{rgb}{0.313725,0.317647,0.309804}%
\pgfsetfillcolor{currentfill}%
\pgfsetfillopacity{0.900000}%
\pgfsetlinewidth{1.003750pt}%
\definecolor{currentstroke}{rgb}{0.313725,0.317647,0.309804}%
\pgfsetstrokecolor{currentstroke}%
\pgfsetstrokeopacity{0.900000}%
\pgfsetdash{}{0pt}%
\pgfsys@defobject{currentmarker}{\pgfqpoint{-0.013889in}{-0.000000in}}{\pgfqpoint{0.013889in}{0.000000in}}{%
\pgfpathmoveto{\pgfqpoint{0.013889in}{-0.000000in}}%
\pgfpathlineto{\pgfqpoint{-0.013889in}{0.000000in}}%
\pgfusepath{stroke,fill}%
}%
\begin{pgfscope}%
\pgfsys@transformshift{0.821931in}{2.554081in}%
\pgfsys@useobject{currentmarker}{}%
\end{pgfscope}%
\begin{pgfscope}%
\pgfsys@transformshift{1.114887in}{2.228136in}%
\pgfsys@useobject{currentmarker}{}%
\end{pgfscope}%
\begin{pgfscope}%
\pgfsys@transformshift{1.407844in}{1.999711in}%
\pgfsys@useobject{currentmarker}{}%
\end{pgfscope}%
\begin{pgfscope}%
\pgfsys@transformshift{1.700800in}{1.979916in}%
\pgfsys@useobject{currentmarker}{}%
\end{pgfscope}%
\begin{pgfscope}%
\pgfsys@transformshift{1.993756in}{2.054461in}%
\pgfsys@useobject{currentmarker}{}%
\end{pgfscope}%
\begin{pgfscope}%
\pgfsys@transformshift{2.286712in}{2.001483in}%
\pgfsys@useobject{currentmarker}{}%
\end{pgfscope}%
\begin{pgfscope}%
\pgfsys@transformshift{2.579669in}{1.884550in}%
\pgfsys@useobject{currentmarker}{}%
\end{pgfscope}%
\begin{pgfscope}%
\pgfsys@transformshift{2.872625in}{1.794023in}%
\pgfsys@useobject{currentmarker}{}%
\end{pgfscope}%
\begin{pgfscope}%
\pgfsys@transformshift{3.165581in}{1.610118in}%
\pgfsys@useobject{currentmarker}{}%
\end{pgfscope}%
\begin{pgfscope}%
\pgfsys@transformshift{3.458537in}{1.453479in}%
\pgfsys@useobject{currentmarker}{}%
\end{pgfscope}%
\begin{pgfscope}%
\pgfsys@transformshift{3.751494in}{1.320713in}%
\pgfsys@useobject{currentmarker}{}%
\end{pgfscope}%
\begin{pgfscope}%
\pgfsys@transformshift{4.044450in}{1.171330in}%
\pgfsys@useobject{currentmarker}{}%
\end{pgfscope}%
\begin{pgfscope}%
\pgfsys@transformshift{4.337406in}{1.024811in}%
\pgfsys@useobject{currentmarker}{}%
\end{pgfscope}%
\begin{pgfscope}%
\pgfsys@transformshift{4.630362in}{0.869426in}%
\pgfsys@useobject{currentmarker}{}%
\end{pgfscope}%
\begin{pgfscope}%
\pgfsys@transformshift{4.923318in}{0.809511in}%
\pgfsys@useobject{currentmarker}{}%
\end{pgfscope}%
\begin{pgfscope}%
\pgfsys@transformshift{5.216275in}{0.823589in}%
\pgfsys@useobject{currentmarker}{}%
\end{pgfscope}%
\begin{pgfscope}%
\pgfsys@transformshift{5.509231in}{0.802225in}%
\pgfsys@useobject{currentmarker}{}%
\end{pgfscope}%
\begin{pgfscope}%
\pgfsys@transformshift{5.802187in}{0.976602in}%
\pgfsys@useobject{currentmarker}{}%
\end{pgfscope}%
\end{pgfscope}%
\begin{pgfscope}%
\pgfpathrectangle{\pgfqpoint{0.572918in}{0.553781in}}{\pgfqpoint{5.478282in}{2.095553in}}%
\pgfusepath{clip}%
\pgfsetbuttcap%
\pgfsetroundjoin%
\definecolor{currentfill}{rgb}{0.949020,0.372549,0.360784}%
\pgfsetfillcolor{currentfill}%
\pgfsetfillopacity{0.900000}%
\pgfsetlinewidth{1.003750pt}%
\definecolor{currentstroke}{rgb}{0.949020,0.372549,0.360784}%
\pgfsetstrokecolor{currentstroke}%
\pgfsetstrokeopacity{0.900000}%
\pgfsetdash{}{0pt}%
\pgfsys@defobject{currentmarker}{\pgfqpoint{-0.013889in}{-0.000000in}}{\pgfqpoint{0.013889in}{0.000000in}}{%
\pgfpathmoveto{\pgfqpoint{0.013889in}{-0.000000in}}%
\pgfpathlineto{\pgfqpoint{-0.013889in}{0.000000in}}%
\pgfusepath{stroke,fill}%
}%
\begin{pgfscope}%
\pgfsys@transformshift{0.821931in}{1.953690in}%
\pgfsys@useobject{currentmarker}{}%
\end{pgfscope}%
\begin{pgfscope}%
\pgfsys@transformshift{1.114887in}{1.902501in}%
\pgfsys@useobject{currentmarker}{}%
\end{pgfscope}%
\begin{pgfscope}%
\pgfsys@transformshift{1.407844in}{1.875880in}%
\pgfsys@useobject{currentmarker}{}%
\end{pgfscope}%
\begin{pgfscope}%
\pgfsys@transformshift{1.700800in}{1.963755in}%
\pgfsys@useobject{currentmarker}{}%
\end{pgfscope}%
\begin{pgfscope}%
\pgfsys@transformshift{1.993756in}{1.996520in}%
\pgfsys@useobject{currentmarker}{}%
\end{pgfscope}%
\begin{pgfscope}%
\pgfsys@transformshift{2.286712in}{1.979066in}%
\pgfsys@useobject{currentmarker}{}%
\end{pgfscope}%
\begin{pgfscope}%
\pgfsys@transformshift{2.579669in}{1.879980in}%
\pgfsys@useobject{currentmarker}{}%
\end{pgfscope}%
\begin{pgfscope}%
\pgfsys@transformshift{2.872625in}{1.764300in}%
\pgfsys@useobject{currentmarker}{}%
\end{pgfscope}%
\begin{pgfscope}%
\pgfsys@transformshift{3.165581in}{1.575657in}%
\pgfsys@useobject{currentmarker}{}%
\end{pgfscope}%
\begin{pgfscope}%
\pgfsys@transformshift{3.458537in}{1.442712in}%
\pgfsys@useobject{currentmarker}{}%
\end{pgfscope}%
\begin{pgfscope}%
\pgfsys@transformshift{3.751494in}{1.293041in}%
\pgfsys@useobject{currentmarker}{}%
\end{pgfscope}%
\begin{pgfscope}%
\pgfsys@transformshift{4.044450in}{1.188594in}%
\pgfsys@useobject{currentmarker}{}%
\end{pgfscope}%
\begin{pgfscope}%
\pgfsys@transformshift{4.337406in}{1.046456in}%
\pgfsys@useobject{currentmarker}{}%
\end{pgfscope}%
\begin{pgfscope}%
\pgfsys@transformshift{4.630362in}{0.899390in}%
\pgfsys@useobject{currentmarker}{}%
\end{pgfscope}%
\begin{pgfscope}%
\pgfsys@transformshift{4.923318in}{0.822040in}%
\pgfsys@useobject{currentmarker}{}%
\end{pgfscope}%
\begin{pgfscope}%
\pgfsys@transformshift{5.216275in}{0.828332in}%
\pgfsys@useobject{currentmarker}{}%
\end{pgfscope}%
\begin{pgfscope}%
\pgfsys@transformshift{5.509231in}{0.701657in}%
\pgfsys@useobject{currentmarker}{}%
\end{pgfscope}%
\begin{pgfscope}%
\pgfsys@transformshift{5.802187in}{0.717960in}%
\pgfsys@useobject{currentmarker}{}%
\end{pgfscope}%
\end{pgfscope}%
\begin{pgfscope}%
\pgfpathrectangle{\pgfqpoint{0.572918in}{0.553781in}}{\pgfqpoint{5.478282in}{2.095553in}}%
\pgfusepath{clip}%
\pgfsetbuttcap%
\pgfsetroundjoin%
\definecolor{currentfill}{rgb}{0.949020,0.372549,0.360784}%
\pgfsetfillcolor{currentfill}%
\pgfsetfillopacity{0.900000}%
\pgfsetlinewidth{1.003750pt}%
\definecolor{currentstroke}{rgb}{0.949020,0.372549,0.360784}%
\pgfsetstrokecolor{currentstroke}%
\pgfsetstrokeopacity{0.900000}%
\pgfsetdash{}{0pt}%
\pgfsys@defobject{currentmarker}{\pgfqpoint{-0.013889in}{-0.000000in}}{\pgfqpoint{0.013889in}{0.000000in}}{%
\pgfpathmoveto{\pgfqpoint{0.013889in}{-0.000000in}}%
\pgfpathlineto{\pgfqpoint{-0.013889in}{0.000000in}}%
\pgfusepath{stroke,fill}%
}%
\begin{pgfscope}%
\pgfsys@transformshift{0.821931in}{2.472437in}%
\pgfsys@useobject{currentmarker}{}%
\end{pgfscope}%
\begin{pgfscope}%
\pgfsys@transformshift{1.114887in}{2.235890in}%
\pgfsys@useobject{currentmarker}{}%
\end{pgfscope}%
\begin{pgfscope}%
\pgfsys@transformshift{1.407844in}{2.039843in}%
\pgfsys@useobject{currentmarker}{}%
\end{pgfscope}%
\begin{pgfscope}%
\pgfsys@transformshift{1.700800in}{2.095315in}%
\pgfsys@useobject{currentmarker}{}%
\end{pgfscope}%
\begin{pgfscope}%
\pgfsys@transformshift{1.993756in}{2.092320in}%
\pgfsys@useobject{currentmarker}{}%
\end{pgfscope}%
\begin{pgfscope}%
\pgfsys@transformshift{2.286712in}{2.060142in}%
\pgfsys@useobject{currentmarker}{}%
\end{pgfscope}%
\begin{pgfscope}%
\pgfsys@transformshift{2.579669in}{1.946145in}%
\pgfsys@useobject{currentmarker}{}%
\end{pgfscope}%
\begin{pgfscope}%
\pgfsys@transformshift{2.872625in}{1.831262in}%
\pgfsys@useobject{currentmarker}{}%
\end{pgfscope}%
\begin{pgfscope}%
\pgfsys@transformshift{3.165581in}{1.631798in}%
\pgfsys@useobject{currentmarker}{}%
\end{pgfscope}%
\begin{pgfscope}%
\pgfsys@transformshift{3.458537in}{1.501731in}%
\pgfsys@useobject{currentmarker}{}%
\end{pgfscope}%
\begin{pgfscope}%
\pgfsys@transformshift{3.751494in}{1.352306in}%
\pgfsys@useobject{currentmarker}{}%
\end{pgfscope}%
\begin{pgfscope}%
\pgfsys@transformshift{4.044450in}{1.259355in}%
\pgfsys@useobject{currentmarker}{}%
\end{pgfscope}%
\begin{pgfscope}%
\pgfsys@transformshift{4.337406in}{1.123900in}%
\pgfsys@useobject{currentmarker}{}%
\end{pgfscope}%
\begin{pgfscope}%
\pgfsys@transformshift{4.630362in}{0.981067in}%
\pgfsys@useobject{currentmarker}{}%
\end{pgfscope}%
\begin{pgfscope}%
\pgfsys@transformshift{4.923318in}{0.925711in}%
\pgfsys@useobject{currentmarker}{}%
\end{pgfscope}%
\begin{pgfscope}%
\pgfsys@transformshift{5.216275in}{0.958950in}%
\pgfsys@useobject{currentmarker}{}%
\end{pgfscope}%
\begin{pgfscope}%
\pgfsys@transformshift{5.509231in}{0.829812in}%
\pgfsys@useobject{currentmarker}{}%
\end{pgfscope}%
\begin{pgfscope}%
\pgfsys@transformshift{5.802187in}{0.895768in}%
\pgfsys@useobject{currentmarker}{}%
\end{pgfscope}%
\end{pgfscope}%
\begin{pgfscope}%
\pgfpathrectangle{\pgfqpoint{0.572918in}{0.553781in}}{\pgfqpoint{5.478282in}{2.095553in}}%
\pgfusepath{clip}%
\pgfsetrectcap%
\pgfsetroundjoin%
\pgfsetlinewidth{1.505625pt}%
\definecolor{currentstroke}{rgb}{0.313725,0.317647,0.309804}%
\pgfsetstrokecolor{currentstroke}%
\pgfsetstrokeopacity{0.900000}%
\pgfsetdash{}{0pt}%
\pgfpathmoveto{\pgfqpoint{0.821931in}{2.126059in}}%
\pgfpathlineto{\pgfqpoint{1.114887in}{2.053659in}}%
\pgfpathlineto{\pgfqpoint{1.407844in}{1.914204in}}%
\pgfpathlineto{\pgfqpoint{1.700800in}{1.929008in}}%
\pgfpathlineto{\pgfqpoint{1.993756in}{2.014239in}}%
\pgfpathlineto{\pgfqpoint{2.286712in}{1.971033in}}%
\pgfpathlineto{\pgfqpoint{2.579669in}{1.849783in}}%
\pgfpathlineto{\pgfqpoint{2.872625in}{1.762912in}}%
\pgfpathlineto{\pgfqpoint{3.165581in}{1.577294in}}%
\pgfpathlineto{\pgfqpoint{3.458537in}{1.423236in}}%
\pgfpathlineto{\pgfqpoint{3.751494in}{1.292686in}}%
\pgfpathlineto{\pgfqpoint{4.044450in}{1.137245in}}%
\pgfpathlineto{\pgfqpoint{4.337406in}{0.989968in}}%
\pgfpathlineto{\pgfqpoint{4.630362in}{0.828010in}}%
\pgfpathlineto{\pgfqpoint{4.923318in}{0.762947in}}%
\pgfpathlineto{\pgfqpoint{5.216275in}{0.750230in}}%
\pgfpathlineto{\pgfqpoint{5.509231in}{0.726517in}}%
\pgfpathlineto{\pgfqpoint{5.802187in}{0.858117in}}%
\pgfusepath{stroke}%
\end{pgfscope}%
\begin{pgfscope}%
\pgfpathrectangle{\pgfqpoint{0.572918in}{0.553781in}}{\pgfqpoint{5.478282in}{2.095553in}}%
\pgfusepath{clip}%
\pgfsetbuttcap%
\pgfsetroundjoin%
\pgfsetlinewidth{1.505625pt}%
\definecolor{currentstroke}{rgb}{0.949020,0.372549,0.360784}%
\pgfsetstrokecolor{currentstroke}%
\pgfsetstrokeopacity{0.900000}%
\pgfsetdash{{1.500000pt}{2.475000pt}}{0.000000pt}%
\pgfpathmoveto{\pgfqpoint{0.821931in}{2.232053in}}%
\pgfpathlineto{\pgfqpoint{1.114887in}{2.065689in}}%
\pgfpathlineto{\pgfqpoint{1.407844in}{1.952369in}}%
\pgfpathlineto{\pgfqpoint{1.700800in}{2.025191in}}%
\pgfpathlineto{\pgfqpoint{1.993756in}{2.044647in}}%
\pgfpathlineto{\pgfqpoint{2.286712in}{2.021149in}}%
\pgfpathlineto{\pgfqpoint{2.579669in}{1.912523in}}%
\pgfpathlineto{\pgfqpoint{2.872625in}{1.798461in}}%
\pgfpathlineto{\pgfqpoint{3.165581in}{1.602192in}}%
\pgfpathlineto{\pgfqpoint{3.458537in}{1.470726in}}%
\pgfpathlineto{\pgfqpoint{3.751494in}{1.324593in}}%
\pgfpathlineto{\pgfqpoint{4.044450in}{1.221919in}}%
\pgfpathlineto{\pgfqpoint{4.337406in}{1.084844in}}%
\pgfpathlineto{\pgfqpoint{4.630362in}{0.941006in}}%
\pgfpathlineto{\pgfqpoint{4.923318in}{0.872027in}}%
\pgfpathlineto{\pgfqpoint{5.216275in}{0.885870in}}%
\pgfpathlineto{\pgfqpoint{5.509231in}{0.760266in}}%
\pgfpathlineto{\pgfqpoint{5.802187in}{0.777854in}}%
\pgfusepath{stroke}%
\end{pgfscope}%
\begin{pgfscope}%
\pgfsetrectcap%
\pgfsetmiterjoin%
\pgfsetlinewidth{0.803000pt}%
\definecolor{currentstroke}{rgb}{0.000000,0.000000,0.000000}%
\pgfsetstrokecolor{currentstroke}%
\pgfsetdash{}{0pt}%
\pgfpathmoveto{\pgfqpoint{0.572918in}{0.553781in}}%
\pgfpathlineto{\pgfqpoint{0.572918in}{2.649333in}}%
\pgfusepath{stroke}%
\end{pgfscope}%
\begin{pgfscope}%
\pgfsetrectcap%
\pgfsetmiterjoin%
\pgfsetlinewidth{0.803000pt}%
\definecolor{currentstroke}{rgb}{0.000000,0.000000,0.000000}%
\pgfsetstrokecolor{currentstroke}%
\pgfsetdash{}{0pt}%
\pgfpathmoveto{\pgfqpoint{6.051200in}{0.553781in}}%
\pgfpathlineto{\pgfqpoint{6.051200in}{2.649333in}}%
\pgfusepath{stroke}%
\end{pgfscope}%
\begin{pgfscope}%
\pgfsetrectcap%
\pgfsetmiterjoin%
\pgfsetlinewidth{0.803000pt}%
\definecolor{currentstroke}{rgb}{0.000000,0.000000,0.000000}%
\pgfsetstrokecolor{currentstroke}%
\pgfsetdash{}{0pt}%
\pgfpathmoveto{\pgfqpoint{0.572918in}{0.553781in}}%
\pgfpathlineto{\pgfqpoint{6.051200in}{0.553781in}}%
\pgfusepath{stroke}%
\end{pgfscope}%
\begin{pgfscope}%
\pgfsetrectcap%
\pgfsetmiterjoin%
\pgfsetlinewidth{0.803000pt}%
\definecolor{currentstroke}{rgb}{0.000000,0.000000,0.000000}%
\pgfsetstrokecolor{currentstroke}%
\pgfsetdash{}{0pt}%
\pgfpathmoveto{\pgfqpoint{0.572918in}{2.649333in}}%
\pgfpathlineto{\pgfqpoint{6.051200in}{2.649333in}}%
\pgfusepath{stroke}%
\end{pgfscope}%
\begin{pgfscope}%
\definecolor{textcolor}{rgb}{0.000000,0.000000,0.000000}%
\pgfsetstrokecolor{textcolor}%
\pgfsetfillcolor{textcolor}%
\pgftext[x=0.572918in,y=2.732667in,left,base]{\color{textcolor}\rmfamily\fontsize{12.000000}{14.400000}\selectfont Zenith performance}%
\end{pgfscope}%
\begin{pgfscope}%
\pgfsetbuttcap%
\pgfsetmiterjoin%
\definecolor{currentfill}{rgb}{1.000000,1.000000,1.000000}%
\pgfsetfillcolor{currentfill}%
\pgfsetfillopacity{0.800000}%
\pgfsetlinewidth{1.003750pt}%
\definecolor{currentstroke}{rgb}{0.800000,0.800000,0.800000}%
\pgfsetstrokecolor{currentstroke}%
\pgfsetstrokeopacity{0.800000}%
\pgfsetdash{}{0pt}%
\pgfpathmoveto{\pgfqpoint{5.260644in}{2.250667in}}%
\pgfpathlineto{\pgfqpoint{5.973422in}{2.250667in}}%
\pgfpathquadraticcurveto{\pgfqpoint{5.995644in}{2.250667in}}{\pgfqpoint{5.995644in}{2.272889in}}%
\pgfpathlineto{\pgfqpoint{5.995644in}{2.571556in}}%
\pgfpathquadraticcurveto{\pgfqpoint{5.995644in}{2.593778in}}{\pgfqpoint{5.973422in}{2.593778in}}%
\pgfpathlineto{\pgfqpoint{5.260644in}{2.593778in}}%
\pgfpathquadraticcurveto{\pgfqpoint{5.238422in}{2.593778in}}{\pgfqpoint{5.238422in}{2.571556in}}%
\pgfpathlineto{\pgfqpoint{5.238422in}{2.272889in}}%
\pgfpathquadraticcurveto{\pgfqpoint{5.238422in}{2.250667in}}{\pgfqpoint{5.260644in}{2.250667in}}%
\pgfpathclose%
\pgfusepath{stroke,fill}%
\end{pgfscope}%
\begin{pgfscope}%
\pgfsetbuttcap%
\pgfsetroundjoin%
\pgfsetlinewidth{1.505625pt}%
\definecolor{currentstroke}{rgb}{0.313725,0.317647,0.309804}%
\pgfsetstrokecolor{currentstroke}%
\pgfsetstrokeopacity{0.900000}%
\pgfsetdash{}{0pt}%
\pgfpathmoveto{\pgfqpoint{5.393978in}{2.454889in}}%
\pgfpathlineto{\pgfqpoint{5.393978in}{2.566000in}}%
\pgfusepath{stroke}%
\end{pgfscope}%
\begin{pgfscope}%
\pgfsetbuttcap%
\pgfsetroundjoin%
\definecolor{currentfill}{rgb}{0.313725,0.317647,0.309804}%
\pgfsetfillcolor{currentfill}%
\pgfsetfillopacity{0.900000}%
\pgfsetlinewidth{1.003750pt}%
\definecolor{currentstroke}{rgb}{0.313725,0.317647,0.309804}%
\pgfsetstrokecolor{currentstroke}%
\pgfsetstrokeopacity{0.900000}%
\pgfsetdash{}{0pt}%
\pgfsys@defobject{currentmarker}{\pgfqpoint{-0.013889in}{-0.000000in}}{\pgfqpoint{0.013889in}{0.000000in}}{%
\pgfpathmoveto{\pgfqpoint{0.013889in}{-0.000000in}}%
\pgfpathlineto{\pgfqpoint{-0.013889in}{0.000000in}}%
\pgfusepath{stroke,fill}%
}%
\begin{pgfscope}%
\pgfsys@transformshift{5.393978in}{2.454889in}%
\pgfsys@useobject{currentmarker}{}%
\end{pgfscope}%
\end{pgfscope}%
\begin{pgfscope}%
\pgfsetbuttcap%
\pgfsetroundjoin%
\definecolor{currentfill}{rgb}{0.313725,0.317647,0.309804}%
\pgfsetfillcolor{currentfill}%
\pgfsetfillopacity{0.900000}%
\pgfsetlinewidth{1.003750pt}%
\definecolor{currentstroke}{rgb}{0.313725,0.317647,0.309804}%
\pgfsetstrokecolor{currentstroke}%
\pgfsetstrokeopacity{0.900000}%
\pgfsetdash{}{0pt}%
\pgfsys@defobject{currentmarker}{\pgfqpoint{-0.013889in}{-0.000000in}}{\pgfqpoint{0.013889in}{0.000000in}}{%
\pgfpathmoveto{\pgfqpoint{0.013889in}{-0.000000in}}%
\pgfpathlineto{\pgfqpoint{-0.013889in}{0.000000in}}%
\pgfusepath{stroke,fill}%
}%
\begin{pgfscope}%
\pgfsys@transformshift{5.393978in}{2.566000in}%
\pgfsys@useobject{currentmarker}{}%
\end{pgfscope}%
\end{pgfscope}%
\begin{pgfscope}%
\pgfsetrectcap%
\pgfsetroundjoin%
\pgfsetlinewidth{1.505625pt}%
\definecolor{currentstroke}{rgb}{0.313725,0.317647,0.309804}%
\pgfsetstrokecolor{currentstroke}%
\pgfsetstrokeopacity{0.900000}%
\pgfsetdash{}{0pt}%
\pgfpathmoveto{\pgfqpoint{5.282867in}{2.510444in}}%
\pgfpathlineto{\pgfqpoint{5.505089in}{2.510444in}}%
\pgfusepath{stroke}%
\end{pgfscope}%
\begin{pgfscope}%
\definecolor{textcolor}{rgb}{0.000000,0.000000,0.000000}%
\pgfsetstrokecolor{textcolor}%
\pgfsetfillcolor{textcolor}%
\pgftext[x=5.593978in,y=2.471556in,left,base]{\color{textcolor}\rmfamily\fontsize{8.000000}{9.600000}\selectfont End}%
\end{pgfscope}%
\begin{pgfscope}%
\pgfsetbuttcap%
\pgfsetroundjoin%
\pgfsetlinewidth{1.505625pt}%
\definecolor{currentstroke}{rgb}{0.949020,0.372549,0.360784}%
\pgfsetstrokecolor{currentstroke}%
\pgfsetstrokeopacity{0.900000}%
\pgfsetdash{}{0pt}%
\pgfpathmoveto{\pgfqpoint{5.393978in}{2.300000in}}%
\pgfpathlineto{\pgfqpoint{5.393978in}{2.411111in}}%
\pgfusepath{stroke}%
\end{pgfscope}%
\begin{pgfscope}%
\pgfsetbuttcap%
\pgfsetroundjoin%
\definecolor{currentfill}{rgb}{0.949020,0.372549,0.360784}%
\pgfsetfillcolor{currentfill}%
\pgfsetfillopacity{0.900000}%
\pgfsetlinewidth{1.003750pt}%
\definecolor{currentstroke}{rgb}{0.949020,0.372549,0.360784}%
\pgfsetstrokecolor{currentstroke}%
\pgfsetstrokeopacity{0.900000}%
\pgfsetdash{}{0pt}%
\pgfsys@defobject{currentmarker}{\pgfqpoint{-0.013889in}{-0.000000in}}{\pgfqpoint{0.013889in}{0.000000in}}{%
\pgfpathmoveto{\pgfqpoint{0.013889in}{-0.000000in}}%
\pgfpathlineto{\pgfqpoint{-0.013889in}{0.000000in}}%
\pgfusepath{stroke,fill}%
}%
\begin{pgfscope}%
\pgfsys@transformshift{5.393978in}{2.300000in}%
\pgfsys@useobject{currentmarker}{}%
\end{pgfscope}%
\end{pgfscope}%
\begin{pgfscope}%
\pgfsetbuttcap%
\pgfsetroundjoin%
\definecolor{currentfill}{rgb}{0.949020,0.372549,0.360784}%
\pgfsetfillcolor{currentfill}%
\pgfsetfillopacity{0.900000}%
\pgfsetlinewidth{1.003750pt}%
\definecolor{currentstroke}{rgb}{0.949020,0.372549,0.360784}%
\pgfsetstrokecolor{currentstroke}%
\pgfsetstrokeopacity{0.900000}%
\pgfsetdash{}{0pt}%
\pgfsys@defobject{currentmarker}{\pgfqpoint{-0.013889in}{-0.000000in}}{\pgfqpoint{0.013889in}{0.000000in}}{%
\pgfpathmoveto{\pgfqpoint{0.013889in}{-0.000000in}}%
\pgfpathlineto{\pgfqpoint{-0.013889in}{0.000000in}}%
\pgfusepath{stroke,fill}%
}%
\begin{pgfscope}%
\pgfsys@transformshift{5.393978in}{2.411111in}%
\pgfsys@useobject{currentmarker}{}%
\end{pgfscope}%
\end{pgfscope}%
\begin{pgfscope}%
\pgfsetbuttcap%
\pgfsetroundjoin%
\pgfsetlinewidth{1.505625pt}%
\definecolor{currentstroke}{rgb}{0.949020,0.372549,0.360784}%
\pgfsetstrokecolor{currentstroke}%
\pgfsetstrokeopacity{0.900000}%
\pgfsetdash{{1.500000pt}{2.475000pt}}{0.000000pt}%
\pgfpathmoveto{\pgfqpoint{5.282867in}{2.355556in}}%
\pgfpathlineto{\pgfqpoint{5.505089in}{2.355556in}}%
\pgfusepath{stroke}%
\end{pgfscope}%
\begin{pgfscope}%
\definecolor{textcolor}{rgb}{0.000000,0.000000,0.000000}%
\pgfsetstrokecolor{textcolor}%
\pgfsetfillcolor{textcolor}%
\pgftext[x=5.593978in,y=2.316667in,left,base]{\color{textcolor}\rmfamily\fontsize{8.000000}{9.600000}\selectfont Middle}%
\end{pgfscope}%
\end{pgfpicture}%
\makeatother%
\endgroup%

    \caption{Network performance with different padding types.
    Because all input tensors are required to have the same shape, each event is zero-padded up to the maximum event length.
    \enquote{End} padding pads the end of the array, while the poorly named \enquote{middle} places the event in the middle, and pads both ends.
    The performance is similar for both.}\label{fig:padding}
\end{figure}

The style of zero-padding was also varied, although with no discernible effect as shown in~\vref{fig:padding}.
Either the event was padded with zeros after the last entry and up to the maximum length, or the event was put into the middle of the array, with both ends then padded.
The network, however, seems to understand that zeros indicate no pulse, as the performance is very similar between the two.

\subsection{Weighting}

\begin{figure}
     \centering
     \begin{subfigure}[b]{0.49\textwidth}
         \centering
         %% Creator: Matplotlib, PGF backend
%%
%% To include the figure in your LaTeX document, write
%%   \input{<filename>.pgf}
%%
%% Make sure the required packages are loaded in your preamble
%%   \usepackage{pgf}
%%
%% and, on pdftex
%%   \usepackage[utf8]{inputenc}\DeclareUnicodeCharacter{2212}{-}
%%
%% or, on luatex and xetex
%%   \usepackage{unicode-math}
%%
%% Figures using additional raster images can only be included by \input if
%% they are in the same directory as the main LaTeX file. For loading figures
%% from other directories you can use the `import` package
%%   \usepackage{import}
%%
%% and then include the figures with
%%   \import{<path to file>}{<filename>.pgf}
%%
%% Matplotlib used the following preamble
%%   \usepackage{siunitx} \usepackage{amsmath} \usepackage{bm}
%%   \usepackage{fontspec}
%%
\begingroup%
\makeatletter%
\begin{pgfpicture}%
\pgfpathrectangle{\pgfpointorigin}{\pgfqpoint{3.038588in}{2.500000in}}%
\pgfusepath{use as bounding box, clip}%
\begin{pgfscope}%
\pgfsetbuttcap%
\pgfsetmiterjoin%
\definecolor{currentfill}{rgb}{1.000000,1.000000,1.000000}%
\pgfsetfillcolor{currentfill}%
\pgfsetlinewidth{0.000000pt}%
\definecolor{currentstroke}{rgb}{1.000000,1.000000,1.000000}%
\pgfsetstrokecolor{currentstroke}%
\pgfsetdash{}{0pt}%
\pgfpathmoveto{\pgfqpoint{0.000000in}{0.000000in}}%
\pgfpathlineto{\pgfqpoint{3.038588in}{0.000000in}}%
\pgfpathlineto{\pgfqpoint{3.038588in}{2.500000in}}%
\pgfpathlineto{\pgfqpoint{0.000000in}{2.500000in}}%
\pgfpathclose%
\pgfusepath{fill}%
\end{pgfscope}%
\begin{pgfscope}%
\pgfsetbuttcap%
\pgfsetmiterjoin%
\definecolor{currentfill}{rgb}{1.000000,1.000000,1.000000}%
\pgfsetfillcolor{currentfill}%
\pgfsetlinewidth{0.000000pt}%
\definecolor{currentstroke}{rgb}{0.000000,0.000000,0.000000}%
\pgfsetstrokecolor{currentstroke}%
\pgfsetstrokeopacity{0.000000}%
\pgfsetdash{}{0pt}%
\pgfpathmoveto{\pgfqpoint{0.588679in}{0.553781in}}%
\pgfpathlineto{\pgfqpoint{2.888588in}{0.553781in}}%
\pgfpathlineto{\pgfqpoint{2.888588in}{2.151000in}}%
\pgfpathlineto{\pgfqpoint{0.588679in}{2.151000in}}%
\pgfpathclose%
\pgfusepath{fill}%
\end{pgfscope}%
\begin{pgfscope}%
\pgfpathrectangle{\pgfqpoint{0.588679in}{0.553781in}}{\pgfqpoint{2.299909in}{1.597219in}}%
\pgfusepath{clip}%
\pgfsetbuttcap%
\pgfsetroundjoin%
\pgfsetlinewidth{0.501875pt}%
\definecolor{currentstroke}{rgb}{0.690196,0.690196,0.690196}%
\pgfsetstrokecolor{currentstroke}%
\pgfsetstrokeopacity{0.500000}%
\pgfsetdash{{0.500000pt}{0.825000pt}}{0.000000pt}%
\pgfpathmoveto{\pgfqpoint{0.693220in}{0.553781in}}%
\pgfpathlineto{\pgfqpoint{0.693220in}{2.151000in}}%
\pgfusepath{stroke}%
\end{pgfscope}%
\begin{pgfscope}%
\pgfsetbuttcap%
\pgfsetroundjoin%
\definecolor{currentfill}{rgb}{0.000000,0.000000,0.000000}%
\pgfsetfillcolor{currentfill}%
\pgfsetlinewidth{0.803000pt}%
\definecolor{currentstroke}{rgb}{0.000000,0.000000,0.000000}%
\pgfsetstrokecolor{currentstroke}%
\pgfsetdash{}{0pt}%
\pgfsys@defobject{currentmarker}{\pgfqpoint{0.000000in}{-0.048611in}}{\pgfqpoint{0.000000in}{0.000000in}}{%
\pgfpathmoveto{\pgfqpoint{0.000000in}{0.000000in}}%
\pgfpathlineto{\pgfqpoint{0.000000in}{-0.048611in}}%
\pgfusepath{stroke,fill}%
}%
\begin{pgfscope}%
\pgfsys@transformshift{0.693220in}{0.553781in}%
\pgfsys@useobject{currentmarker}{}%
\end{pgfscope}%
\end{pgfscope}%
\begin{pgfscope}%
\definecolor{textcolor}{rgb}{0.000000,0.000000,0.000000}%
\pgfsetstrokecolor{textcolor}%
\pgfsetfillcolor{textcolor}%
\pgftext[x=0.693220in,y=0.456558in,,top]{\color{textcolor}\rmfamily\fontsize{8.000000}{9.600000}\selectfont \(\displaystyle {0}\)}%
\end{pgfscope}%
\begin{pgfscope}%
\pgfpathrectangle{\pgfqpoint{0.588679in}{0.553781in}}{\pgfqpoint{2.299909in}{1.597219in}}%
\pgfusepath{clip}%
\pgfsetbuttcap%
\pgfsetroundjoin%
\pgfsetlinewidth{0.501875pt}%
\definecolor{currentstroke}{rgb}{0.690196,0.690196,0.690196}%
\pgfsetstrokecolor{currentstroke}%
\pgfsetstrokeopacity{0.500000}%
\pgfsetdash{{0.500000pt}{0.825000pt}}{0.000000pt}%
\pgfpathmoveto{\pgfqpoint{1.390164in}{0.553781in}}%
\pgfpathlineto{\pgfqpoint{1.390164in}{2.151000in}}%
\pgfusepath{stroke}%
\end{pgfscope}%
\begin{pgfscope}%
\pgfsetbuttcap%
\pgfsetroundjoin%
\definecolor{currentfill}{rgb}{0.000000,0.000000,0.000000}%
\pgfsetfillcolor{currentfill}%
\pgfsetlinewidth{0.803000pt}%
\definecolor{currentstroke}{rgb}{0.000000,0.000000,0.000000}%
\pgfsetstrokecolor{currentstroke}%
\pgfsetdash{}{0pt}%
\pgfsys@defobject{currentmarker}{\pgfqpoint{0.000000in}{-0.048611in}}{\pgfqpoint{0.000000in}{0.000000in}}{%
\pgfpathmoveto{\pgfqpoint{0.000000in}{0.000000in}}%
\pgfpathlineto{\pgfqpoint{0.000000in}{-0.048611in}}%
\pgfusepath{stroke,fill}%
}%
\begin{pgfscope}%
\pgfsys@transformshift{1.390164in}{0.553781in}%
\pgfsys@useobject{currentmarker}{}%
\end{pgfscope}%
\end{pgfscope}%
\begin{pgfscope}%
\definecolor{textcolor}{rgb}{0.000000,0.000000,0.000000}%
\pgfsetstrokecolor{textcolor}%
\pgfsetfillcolor{textcolor}%
\pgftext[x=1.390164in,y=0.456558in,,top]{\color{textcolor}\rmfamily\fontsize{8.000000}{9.600000}\selectfont \(\displaystyle {1}\)}%
\end{pgfscope}%
\begin{pgfscope}%
\pgfpathrectangle{\pgfqpoint{0.588679in}{0.553781in}}{\pgfqpoint{2.299909in}{1.597219in}}%
\pgfusepath{clip}%
\pgfsetbuttcap%
\pgfsetroundjoin%
\pgfsetlinewidth{0.501875pt}%
\definecolor{currentstroke}{rgb}{0.690196,0.690196,0.690196}%
\pgfsetstrokecolor{currentstroke}%
\pgfsetstrokeopacity{0.500000}%
\pgfsetdash{{0.500000pt}{0.825000pt}}{0.000000pt}%
\pgfpathmoveto{\pgfqpoint{2.087109in}{0.553781in}}%
\pgfpathlineto{\pgfqpoint{2.087109in}{2.151000in}}%
\pgfusepath{stroke}%
\end{pgfscope}%
\begin{pgfscope}%
\pgfsetbuttcap%
\pgfsetroundjoin%
\definecolor{currentfill}{rgb}{0.000000,0.000000,0.000000}%
\pgfsetfillcolor{currentfill}%
\pgfsetlinewidth{0.803000pt}%
\definecolor{currentstroke}{rgb}{0.000000,0.000000,0.000000}%
\pgfsetstrokecolor{currentstroke}%
\pgfsetdash{}{0pt}%
\pgfsys@defobject{currentmarker}{\pgfqpoint{0.000000in}{-0.048611in}}{\pgfqpoint{0.000000in}{0.000000in}}{%
\pgfpathmoveto{\pgfqpoint{0.000000in}{0.000000in}}%
\pgfpathlineto{\pgfqpoint{0.000000in}{-0.048611in}}%
\pgfusepath{stroke,fill}%
}%
\begin{pgfscope}%
\pgfsys@transformshift{2.087109in}{0.553781in}%
\pgfsys@useobject{currentmarker}{}%
\end{pgfscope}%
\end{pgfscope}%
\begin{pgfscope}%
\definecolor{textcolor}{rgb}{0.000000,0.000000,0.000000}%
\pgfsetstrokecolor{textcolor}%
\pgfsetfillcolor{textcolor}%
\pgftext[x=2.087109in,y=0.456558in,,top]{\color{textcolor}\rmfamily\fontsize{8.000000}{9.600000}\selectfont \(\displaystyle {2}\)}%
\end{pgfscope}%
\begin{pgfscope}%
\pgfpathrectangle{\pgfqpoint{0.588679in}{0.553781in}}{\pgfqpoint{2.299909in}{1.597219in}}%
\pgfusepath{clip}%
\pgfsetbuttcap%
\pgfsetroundjoin%
\pgfsetlinewidth{0.501875pt}%
\definecolor{currentstroke}{rgb}{0.690196,0.690196,0.690196}%
\pgfsetstrokecolor{currentstroke}%
\pgfsetstrokeopacity{0.500000}%
\pgfsetdash{{0.500000pt}{0.825000pt}}{0.000000pt}%
\pgfpathmoveto{\pgfqpoint{2.784053in}{0.553781in}}%
\pgfpathlineto{\pgfqpoint{2.784053in}{2.151000in}}%
\pgfusepath{stroke}%
\end{pgfscope}%
\begin{pgfscope}%
\pgfsetbuttcap%
\pgfsetroundjoin%
\definecolor{currentfill}{rgb}{0.000000,0.000000,0.000000}%
\pgfsetfillcolor{currentfill}%
\pgfsetlinewidth{0.803000pt}%
\definecolor{currentstroke}{rgb}{0.000000,0.000000,0.000000}%
\pgfsetstrokecolor{currentstroke}%
\pgfsetdash{}{0pt}%
\pgfsys@defobject{currentmarker}{\pgfqpoint{0.000000in}{-0.048611in}}{\pgfqpoint{0.000000in}{0.000000in}}{%
\pgfpathmoveto{\pgfqpoint{0.000000in}{0.000000in}}%
\pgfpathlineto{\pgfqpoint{0.000000in}{-0.048611in}}%
\pgfusepath{stroke,fill}%
}%
\begin{pgfscope}%
\pgfsys@transformshift{2.784053in}{0.553781in}%
\pgfsys@useobject{currentmarker}{}%
\end{pgfscope}%
\end{pgfscope}%
\begin{pgfscope}%
\definecolor{textcolor}{rgb}{0.000000,0.000000,0.000000}%
\pgfsetstrokecolor{textcolor}%
\pgfsetfillcolor{textcolor}%
\pgftext[x=2.784053in,y=0.456558in,,top]{\color{textcolor}\rmfamily\fontsize{8.000000}{9.600000}\selectfont \(\displaystyle {3}\)}%
\end{pgfscope}%
\begin{pgfscope}%
\definecolor{textcolor}{rgb}{0.000000,0.000000,0.000000}%
\pgfsetstrokecolor{textcolor}%
\pgfsetfillcolor{textcolor}%
\pgftext[x=1.738633in,y=0.302336in,,top]{\color{textcolor}\rmfamily\fontsize{10.950000}{13.140000}\selectfont \(\displaystyle \log_{10}(E_{\textup{true}}) \, \left[ E / \textup{GeV} \right]\)}%
\end{pgfscope}%
\begin{pgfscope}%
\pgfpathrectangle{\pgfqpoint{0.588679in}{0.553781in}}{\pgfqpoint{2.299909in}{1.597219in}}%
\pgfusepath{clip}%
\pgfsetbuttcap%
\pgfsetroundjoin%
\pgfsetlinewidth{0.501875pt}%
\definecolor{currentstroke}{rgb}{0.690196,0.690196,0.690196}%
\pgfsetstrokecolor{currentstroke}%
\pgfsetstrokeopacity{0.500000}%
\pgfsetdash{{0.500000pt}{0.825000pt}}{0.000000pt}%
\pgfpathmoveto{\pgfqpoint{0.588679in}{0.553781in}}%
\pgfpathlineto{\pgfqpoint{2.888588in}{0.553781in}}%
\pgfusepath{stroke}%
\end{pgfscope}%
\begin{pgfscope}%
\pgfsetbuttcap%
\pgfsetroundjoin%
\definecolor{currentfill}{rgb}{0.000000,0.000000,0.000000}%
\pgfsetfillcolor{currentfill}%
\pgfsetlinewidth{0.803000pt}%
\definecolor{currentstroke}{rgb}{0.000000,0.000000,0.000000}%
\pgfsetstrokecolor{currentstroke}%
\pgfsetdash{}{0pt}%
\pgfsys@defobject{currentmarker}{\pgfqpoint{-0.048611in}{0.000000in}}{\pgfqpoint{-0.000000in}{0.000000in}}{%
\pgfpathmoveto{\pgfqpoint{-0.000000in}{0.000000in}}%
\pgfpathlineto{\pgfqpoint{-0.048611in}{0.000000in}}%
\pgfusepath{stroke,fill}%
}%
\begin{pgfscope}%
\pgfsys@transformshift{0.588679in}{0.553781in}%
\pgfsys@useobject{currentmarker}{}%
\end{pgfscope}%
\end{pgfscope}%
\begin{pgfscope}%
\definecolor{textcolor}{rgb}{0.000000,0.000000,0.000000}%
\pgfsetstrokecolor{textcolor}%
\pgfsetfillcolor{textcolor}%
\pgftext[x=0.340606in, y=0.515225in, left, base]{\color{textcolor}\rmfamily\fontsize{8.000000}{9.600000}\selectfont \(\displaystyle {0.0}\)}%
\end{pgfscope}%
\begin{pgfscope}%
\pgfpathrectangle{\pgfqpoint{0.588679in}{0.553781in}}{\pgfqpoint{2.299909in}{1.597219in}}%
\pgfusepath{clip}%
\pgfsetbuttcap%
\pgfsetroundjoin%
\pgfsetlinewidth{0.501875pt}%
\definecolor{currentstroke}{rgb}{0.690196,0.690196,0.690196}%
\pgfsetstrokecolor{currentstroke}%
\pgfsetstrokeopacity{0.500000}%
\pgfsetdash{{0.500000pt}{0.825000pt}}{0.000000pt}%
\pgfpathmoveto{\pgfqpoint{0.588679in}{0.937596in}}%
\pgfpathlineto{\pgfqpoint{2.888588in}{0.937596in}}%
\pgfusepath{stroke}%
\end{pgfscope}%
\begin{pgfscope}%
\pgfsetbuttcap%
\pgfsetroundjoin%
\definecolor{currentfill}{rgb}{0.000000,0.000000,0.000000}%
\pgfsetfillcolor{currentfill}%
\pgfsetlinewidth{0.803000pt}%
\definecolor{currentstroke}{rgb}{0.000000,0.000000,0.000000}%
\pgfsetstrokecolor{currentstroke}%
\pgfsetdash{}{0pt}%
\pgfsys@defobject{currentmarker}{\pgfqpoint{-0.048611in}{0.000000in}}{\pgfqpoint{-0.000000in}{0.000000in}}{%
\pgfpathmoveto{\pgfqpoint{-0.000000in}{0.000000in}}%
\pgfpathlineto{\pgfqpoint{-0.048611in}{0.000000in}}%
\pgfusepath{stroke,fill}%
}%
\begin{pgfscope}%
\pgfsys@transformshift{0.588679in}{0.937596in}%
\pgfsys@useobject{currentmarker}{}%
\end{pgfscope}%
\end{pgfscope}%
\begin{pgfscope}%
\definecolor{textcolor}{rgb}{0.000000,0.000000,0.000000}%
\pgfsetstrokecolor{textcolor}%
\pgfsetfillcolor{textcolor}%
\pgftext[x=0.340606in, y=0.899041in, left, base]{\color{textcolor}\rmfamily\fontsize{8.000000}{9.600000}\selectfont \(\displaystyle {0.2}\)}%
\end{pgfscope}%
\begin{pgfscope}%
\pgfpathrectangle{\pgfqpoint{0.588679in}{0.553781in}}{\pgfqpoint{2.299909in}{1.597219in}}%
\pgfusepath{clip}%
\pgfsetbuttcap%
\pgfsetroundjoin%
\pgfsetlinewidth{0.501875pt}%
\definecolor{currentstroke}{rgb}{0.690196,0.690196,0.690196}%
\pgfsetstrokecolor{currentstroke}%
\pgfsetstrokeopacity{0.500000}%
\pgfsetdash{{0.500000pt}{0.825000pt}}{0.000000pt}%
\pgfpathmoveto{\pgfqpoint{0.588679in}{1.321412in}}%
\pgfpathlineto{\pgfqpoint{2.888588in}{1.321412in}}%
\pgfusepath{stroke}%
\end{pgfscope}%
\begin{pgfscope}%
\pgfsetbuttcap%
\pgfsetroundjoin%
\definecolor{currentfill}{rgb}{0.000000,0.000000,0.000000}%
\pgfsetfillcolor{currentfill}%
\pgfsetlinewidth{0.803000pt}%
\definecolor{currentstroke}{rgb}{0.000000,0.000000,0.000000}%
\pgfsetstrokecolor{currentstroke}%
\pgfsetdash{}{0pt}%
\pgfsys@defobject{currentmarker}{\pgfqpoint{-0.048611in}{0.000000in}}{\pgfqpoint{-0.000000in}{0.000000in}}{%
\pgfpathmoveto{\pgfqpoint{-0.000000in}{0.000000in}}%
\pgfpathlineto{\pgfqpoint{-0.048611in}{0.000000in}}%
\pgfusepath{stroke,fill}%
}%
\begin{pgfscope}%
\pgfsys@transformshift{0.588679in}{1.321412in}%
\pgfsys@useobject{currentmarker}{}%
\end{pgfscope}%
\end{pgfscope}%
\begin{pgfscope}%
\definecolor{textcolor}{rgb}{0.000000,0.000000,0.000000}%
\pgfsetstrokecolor{textcolor}%
\pgfsetfillcolor{textcolor}%
\pgftext[x=0.340606in, y=1.282857in, left, base]{\color{textcolor}\rmfamily\fontsize{8.000000}{9.600000}\selectfont \(\displaystyle {0.4}\)}%
\end{pgfscope}%
\begin{pgfscope}%
\pgfpathrectangle{\pgfqpoint{0.588679in}{0.553781in}}{\pgfqpoint{2.299909in}{1.597219in}}%
\pgfusepath{clip}%
\pgfsetbuttcap%
\pgfsetroundjoin%
\pgfsetlinewidth{0.501875pt}%
\definecolor{currentstroke}{rgb}{0.690196,0.690196,0.690196}%
\pgfsetstrokecolor{currentstroke}%
\pgfsetstrokeopacity{0.500000}%
\pgfsetdash{{0.500000pt}{0.825000pt}}{0.000000pt}%
\pgfpathmoveto{\pgfqpoint{0.588679in}{1.705228in}}%
\pgfpathlineto{\pgfqpoint{2.888588in}{1.705228in}}%
\pgfusepath{stroke}%
\end{pgfscope}%
\begin{pgfscope}%
\pgfsetbuttcap%
\pgfsetroundjoin%
\definecolor{currentfill}{rgb}{0.000000,0.000000,0.000000}%
\pgfsetfillcolor{currentfill}%
\pgfsetlinewidth{0.803000pt}%
\definecolor{currentstroke}{rgb}{0.000000,0.000000,0.000000}%
\pgfsetstrokecolor{currentstroke}%
\pgfsetdash{}{0pt}%
\pgfsys@defobject{currentmarker}{\pgfqpoint{-0.048611in}{0.000000in}}{\pgfqpoint{-0.000000in}{0.000000in}}{%
\pgfpathmoveto{\pgfqpoint{-0.000000in}{0.000000in}}%
\pgfpathlineto{\pgfqpoint{-0.048611in}{0.000000in}}%
\pgfusepath{stroke,fill}%
}%
\begin{pgfscope}%
\pgfsys@transformshift{0.588679in}{1.705228in}%
\pgfsys@useobject{currentmarker}{}%
\end{pgfscope}%
\end{pgfscope}%
\begin{pgfscope}%
\definecolor{textcolor}{rgb}{0.000000,0.000000,0.000000}%
\pgfsetstrokecolor{textcolor}%
\pgfsetfillcolor{textcolor}%
\pgftext[x=0.340606in, y=1.666673in, left, base]{\color{textcolor}\rmfamily\fontsize{8.000000}{9.600000}\selectfont \(\displaystyle {0.6}\)}%
\end{pgfscope}%
\begin{pgfscope}%
\pgfpathrectangle{\pgfqpoint{0.588679in}{0.553781in}}{\pgfqpoint{2.299909in}{1.597219in}}%
\pgfusepath{clip}%
\pgfsetbuttcap%
\pgfsetroundjoin%
\pgfsetlinewidth{0.501875pt}%
\definecolor{currentstroke}{rgb}{0.690196,0.690196,0.690196}%
\pgfsetstrokecolor{currentstroke}%
\pgfsetstrokeopacity{0.500000}%
\pgfsetdash{{0.500000pt}{0.825000pt}}{0.000000pt}%
\pgfpathmoveto{\pgfqpoint{0.588679in}{2.089044in}}%
\pgfpathlineto{\pgfqpoint{2.888588in}{2.089044in}}%
\pgfusepath{stroke}%
\end{pgfscope}%
\begin{pgfscope}%
\pgfsetbuttcap%
\pgfsetroundjoin%
\definecolor{currentfill}{rgb}{0.000000,0.000000,0.000000}%
\pgfsetfillcolor{currentfill}%
\pgfsetlinewidth{0.803000pt}%
\definecolor{currentstroke}{rgb}{0.000000,0.000000,0.000000}%
\pgfsetstrokecolor{currentstroke}%
\pgfsetdash{}{0pt}%
\pgfsys@defobject{currentmarker}{\pgfqpoint{-0.048611in}{0.000000in}}{\pgfqpoint{-0.000000in}{0.000000in}}{%
\pgfpathmoveto{\pgfqpoint{-0.000000in}{0.000000in}}%
\pgfpathlineto{\pgfqpoint{-0.048611in}{0.000000in}}%
\pgfusepath{stroke,fill}%
}%
\begin{pgfscope}%
\pgfsys@transformshift{0.588679in}{2.089044in}%
\pgfsys@useobject{currentmarker}{}%
\end{pgfscope}%
\end{pgfscope}%
\begin{pgfscope}%
\definecolor{textcolor}{rgb}{0.000000,0.000000,0.000000}%
\pgfsetstrokecolor{textcolor}%
\pgfsetfillcolor{textcolor}%
\pgftext[x=0.340606in, y=2.050488in, left, base]{\color{textcolor}\rmfamily\fontsize{8.000000}{9.600000}\selectfont \(\displaystyle {0.8}\)}%
\end{pgfscope}%
\begin{pgfscope}%
\definecolor{textcolor}{rgb}{0.000000,0.000000,0.000000}%
\pgfsetstrokecolor{textcolor}%
\pgfsetfillcolor{textcolor}%
\pgftext[x=0.285050in,y=1.352390in,,bottom,rotate=90.000000]{\color{textcolor}\rmfamily\fontsize{10.950000}{13.140000}\selectfont Density}%
\end{pgfscope}%
\begin{pgfscope}%
\pgfpathrectangle{\pgfqpoint{0.588679in}{0.553781in}}{\pgfqpoint{2.299909in}{1.597219in}}%
\pgfusepath{clip}%
\pgfsetbuttcap%
\pgfsetmiterjoin%
\pgfsetlinewidth{1.003750pt}%
\definecolor{currentstroke}{rgb}{0.313725,0.317647,0.309804}%
\pgfsetstrokecolor{currentstroke}%
\pgfsetdash{}{0pt}%
\pgfpathmoveto{\pgfqpoint{0.693220in}{0.553781in}}%
\pgfpathlineto{\pgfqpoint{0.693220in}{0.563637in}}%
\pgfpathlineto{\pgfqpoint{0.703674in}{0.563637in}}%
\pgfpathlineto{\pgfqpoint{0.703674in}{0.564958in}}%
\pgfpathlineto{\pgfqpoint{0.714128in}{0.564958in}}%
\pgfpathlineto{\pgfqpoint{0.714128in}{0.566548in}}%
\pgfpathlineto{\pgfqpoint{0.724582in}{0.566548in}}%
\pgfpathlineto{\pgfqpoint{0.724582in}{0.568823in}}%
\pgfpathlineto{\pgfqpoint{0.735037in}{0.568823in}}%
\pgfpathlineto{\pgfqpoint{0.735037in}{0.570951in}}%
\pgfpathlineto{\pgfqpoint{0.745491in}{0.570951in}}%
\pgfpathlineto{\pgfqpoint{0.745491in}{0.573127in}}%
\pgfpathlineto{\pgfqpoint{0.755945in}{0.573127in}}%
\pgfpathlineto{\pgfqpoint{0.755945in}{0.576014in}}%
\pgfpathlineto{\pgfqpoint{0.766399in}{0.576014in}}%
\pgfpathlineto{\pgfqpoint{0.766399in}{0.580661in}}%
\pgfpathlineto{\pgfqpoint{0.776853in}{0.580661in}}%
\pgfpathlineto{\pgfqpoint{0.776853in}{0.585112in}}%
\pgfpathlineto{\pgfqpoint{0.787307in}{0.585112in}}%
\pgfpathlineto{\pgfqpoint{0.787307in}{0.587167in}}%
\pgfpathlineto{\pgfqpoint{0.797761in}{0.587167in}}%
\pgfpathlineto{\pgfqpoint{0.797761in}{0.592817in}}%
\pgfpathlineto{\pgfqpoint{0.808216in}{0.592817in}}%
\pgfpathlineto{\pgfqpoint{0.808216in}{0.598320in}}%
\pgfpathlineto{\pgfqpoint{0.818670in}{0.598320in}}%
\pgfpathlineto{\pgfqpoint{0.818670in}{0.604630in}}%
\pgfpathlineto{\pgfqpoint{0.829124in}{0.604630in}}%
\pgfpathlineto{\pgfqpoint{0.829124in}{0.607908in}}%
\pgfpathlineto{\pgfqpoint{0.839578in}{0.607908in}}%
\pgfpathlineto{\pgfqpoint{0.839578in}{0.616786in}}%
\pgfpathlineto{\pgfqpoint{0.850032in}{0.616786in}}%
\pgfpathlineto{\pgfqpoint{0.850032in}{0.622974in}}%
\pgfpathlineto{\pgfqpoint{0.860486in}{0.622974in}}%
\pgfpathlineto{\pgfqpoint{0.860486in}{0.632758in}}%
\pgfpathlineto{\pgfqpoint{0.870940in}{0.632758in}}%
\pgfpathlineto{\pgfqpoint{0.870940in}{0.642101in}}%
\pgfpathlineto{\pgfqpoint{0.881394in}{0.642101in}}%
\pgfpathlineto{\pgfqpoint{0.881394in}{0.649267in}}%
\pgfpathlineto{\pgfqpoint{0.891849in}{0.649267in}}%
\pgfpathlineto{\pgfqpoint{0.891849in}{0.659442in}}%
\pgfpathlineto{\pgfqpoint{0.902303in}{0.659442in}}%
\pgfpathlineto{\pgfqpoint{0.902303in}{0.670473in}}%
\pgfpathlineto{\pgfqpoint{0.912757in}{0.670473in}}%
\pgfpathlineto{\pgfqpoint{0.912757in}{0.685809in}}%
\pgfpathlineto{\pgfqpoint{0.923211in}{0.685809in}}%
\pgfpathlineto{\pgfqpoint{0.923211in}{0.697696in}}%
\pgfpathlineto{\pgfqpoint{0.933665in}{0.697696in}}%
\pgfpathlineto{\pgfqpoint{0.933665in}{0.710341in}}%
\pgfpathlineto{\pgfqpoint{0.944119in}{0.710341in}}%
\pgfpathlineto{\pgfqpoint{0.944119in}{0.728220in}}%
\pgfpathlineto{\pgfqpoint{0.954573in}{0.728220in}}%
\pgfpathlineto{\pgfqpoint{0.954573in}{0.745464in}}%
\pgfpathlineto{\pgfqpoint{0.965028in}{0.745464in}}%
\pgfpathlineto{\pgfqpoint{0.965028in}{0.760750in}}%
\pgfpathlineto{\pgfqpoint{0.975482in}{0.760750in}}%
\pgfpathlineto{\pgfqpoint{0.975482in}{0.779681in}}%
\pgfpathlineto{\pgfqpoint{0.985936in}{0.779681in}}%
\pgfpathlineto{\pgfqpoint{0.985936in}{0.800545in}}%
\pgfpathlineto{\pgfqpoint{0.996390in}{0.800545in}}%
\pgfpathlineto{\pgfqpoint{0.996390in}{0.815880in}}%
\pgfpathlineto{\pgfqpoint{1.006844in}{0.815880in}}%
\pgfpathlineto{\pgfqpoint{1.006844in}{0.838871in}}%
\pgfpathlineto{\pgfqpoint{1.017298in}{0.838871in}}%
\pgfpathlineto{\pgfqpoint{1.017298in}{0.862768in}}%
\pgfpathlineto{\pgfqpoint{1.027752in}{0.862768in}}%
\pgfpathlineto{\pgfqpoint{1.027752in}{0.887911in}}%
\pgfpathlineto{\pgfqpoint{1.038206in}{0.887911in}}%
\pgfpathlineto{\pgfqpoint{1.038206in}{0.908236in}}%
\pgfpathlineto{\pgfqpoint{1.048661in}{0.908236in}}%
\pgfpathlineto{\pgfqpoint{1.048661in}{0.935214in}}%
\pgfpathlineto{\pgfqpoint{1.059115in}{0.935214in}}%
\pgfpathlineto{\pgfqpoint{1.059115in}{0.961385in}}%
\pgfpathlineto{\pgfqpoint{1.069569in}{0.961385in}}%
\pgfpathlineto{\pgfqpoint{1.069569in}{0.986627in}}%
\pgfpathlineto{\pgfqpoint{1.080023in}{0.986627in}}%
\pgfpathlineto{\pgfqpoint{1.080023in}{1.021578in}}%
\pgfpathlineto{\pgfqpoint{1.090477in}{1.021578in}}%
\pgfpathlineto{\pgfqpoint{1.090477in}{1.041634in}}%
\pgfpathlineto{\pgfqpoint{1.100931in}{1.041634in}}%
\pgfpathlineto{\pgfqpoint{1.100931in}{1.073871in}}%
\pgfpathlineto{\pgfqpoint{1.111385in}{1.073871in}}%
\pgfpathlineto{\pgfqpoint{1.111385in}{1.112638in}}%
\pgfpathlineto{\pgfqpoint{1.121840in}{1.112638in}}%
\pgfpathlineto{\pgfqpoint{1.121840in}{1.130566in}}%
\pgfpathlineto{\pgfqpoint{1.132294in}{1.130566in}}%
\pgfpathlineto{\pgfqpoint{1.132294in}{1.168599in}}%
\pgfpathlineto{\pgfqpoint{1.142748in}{1.168599in}}%
\pgfpathlineto{\pgfqpoint{1.142748in}{1.197412in}}%
\pgfpathlineto{\pgfqpoint{1.153202in}{1.197412in}}%
\pgfpathlineto{\pgfqpoint{1.153202in}{1.237084in}}%
\pgfpathlineto{\pgfqpoint{1.163656in}{1.237084in}}%
\pgfpathlineto{\pgfqpoint{1.163656in}{1.263499in}}%
\pgfpathlineto{\pgfqpoint{1.174110in}{1.263499in}}%
\pgfpathlineto{\pgfqpoint{1.174110in}{1.289328in}}%
\pgfpathlineto{\pgfqpoint{1.184564in}{1.289328in}}%
\pgfpathlineto{\pgfqpoint{1.184564in}{1.304614in}}%
\pgfpathlineto{\pgfqpoint{1.195018in}{1.304614in}}%
\pgfpathlineto{\pgfqpoint{1.195018in}{1.349349in}}%
\pgfpathlineto{\pgfqpoint{1.205473in}{1.349349in}}%
\pgfpathlineto{\pgfqpoint{1.205473in}{1.385206in}}%
\pgfpathlineto{\pgfqpoint{1.215927in}{1.385206in}}%
\pgfpathlineto{\pgfqpoint{1.215927in}{1.406876in}}%
\pgfpathlineto{\pgfqpoint{1.226381in}{1.406876in}}%
\pgfpathlineto{\pgfqpoint{1.226381in}{1.434784in}}%
\pgfpathlineto{\pgfqpoint{1.236835in}{1.434784in}}%
\pgfpathlineto{\pgfqpoint{1.236835in}{1.480937in}}%
\pgfpathlineto{\pgfqpoint{1.247289in}{1.480937in}}%
\pgfpathlineto{\pgfqpoint{1.247289in}{1.507720in}}%
\pgfpathlineto{\pgfqpoint{1.257743in}{1.507720in}}%
\pgfpathlineto{\pgfqpoint{1.257743in}{1.547465in}}%
\pgfpathlineto{\pgfqpoint{1.268197in}{1.547465in}}%
\pgfpathlineto{\pgfqpoint{1.268197in}{1.569551in}}%
\pgfpathlineto{\pgfqpoint{1.278652in}{1.569551in}}%
\pgfpathlineto{\pgfqpoint{1.278652in}{1.603255in}}%
\pgfpathlineto{\pgfqpoint{1.289106in}{1.603255in}}%
\pgfpathlineto{\pgfqpoint{1.289106in}{1.641484in}}%
\pgfpathlineto{\pgfqpoint{1.299560in}{1.641484in}}%
\pgfpathlineto{\pgfqpoint{1.299560in}{1.670517in}}%
\pgfpathlineto{\pgfqpoint{1.310014in}{1.670517in}}%
\pgfpathlineto{\pgfqpoint{1.310014in}{1.691992in}}%
\pgfpathlineto{\pgfqpoint{1.320468in}{1.691992in}}%
\pgfpathlineto{\pgfqpoint{1.320468in}{1.723323in}}%
\pgfpathlineto{\pgfqpoint{1.330922in}{1.723323in}}%
\pgfpathlineto{\pgfqpoint{1.330922in}{1.754777in}}%
\pgfpathlineto{\pgfqpoint{1.341376in}{1.754777in}}%
\pgfpathlineto{\pgfqpoint{1.341376in}{1.788579in}}%
\pgfpathlineto{\pgfqpoint{1.351830in}{1.788579in}}%
\pgfpathlineto{\pgfqpoint{1.351830in}{1.806899in}}%
\pgfpathlineto{\pgfqpoint{1.362285in}{1.806899in}}%
\pgfpathlineto{\pgfqpoint{1.362285in}{1.825267in}}%
\pgfpathlineto{\pgfqpoint{1.372739in}{1.825267in}}%
\pgfpathlineto{\pgfqpoint{1.372739in}{1.864793in}}%
\pgfpathlineto{\pgfqpoint{1.383193in}{1.864793in}}%
\pgfpathlineto{\pgfqpoint{1.383193in}{1.879639in}}%
\pgfpathlineto{\pgfqpoint{1.393647in}{1.879639in}}%
\pgfpathlineto{\pgfqpoint{1.393647in}{1.898277in}}%
\pgfpathlineto{\pgfqpoint{1.404101in}{1.898277in}}%
\pgfpathlineto{\pgfqpoint{1.404101in}{1.927480in}}%
\pgfpathlineto{\pgfqpoint{1.414555in}{1.927480in}}%
\pgfpathlineto{\pgfqpoint{1.414555in}{1.937557in}}%
\pgfpathlineto{\pgfqpoint{1.425009in}{1.937557in}}%
\pgfpathlineto{\pgfqpoint{1.425009in}{1.957467in}}%
\pgfpathlineto{\pgfqpoint{1.435464in}{1.957467in}}%
\pgfpathlineto{\pgfqpoint{1.435464in}{1.984934in}}%
\pgfpathlineto{\pgfqpoint{1.445918in}{1.984934in}}%
\pgfpathlineto{\pgfqpoint{1.445918in}{2.003082in}}%
\pgfpathlineto{\pgfqpoint{1.456372in}{2.003082in}}%
\pgfpathlineto{\pgfqpoint{1.456372in}{2.009368in}}%
\pgfpathlineto{\pgfqpoint{1.466826in}{2.009368in}}%
\pgfpathlineto{\pgfqpoint{1.466826in}{2.013722in}}%
\pgfpathlineto{\pgfqpoint{1.477280in}{2.013722in}}%
\pgfpathlineto{\pgfqpoint{1.477280in}{2.023530in}}%
\pgfpathlineto{\pgfqpoint{1.487734in}{2.023530in}}%
\pgfpathlineto{\pgfqpoint{1.487734in}{2.039624in}}%
\pgfpathlineto{\pgfqpoint{1.498188in}{2.039624in}}%
\pgfpathlineto{\pgfqpoint{1.498188in}{2.056696in}}%
\pgfpathlineto{\pgfqpoint{1.508642in}{2.056696in}}%
\pgfpathlineto{\pgfqpoint{1.508642in}{2.064645in}}%
\pgfpathlineto{\pgfqpoint{1.519097in}{2.064645in}}%
\pgfpathlineto{\pgfqpoint{1.519097in}{2.066724in}}%
\pgfpathlineto{\pgfqpoint{1.529551in}{2.066724in}}%
\pgfpathlineto{\pgfqpoint{1.529551in}{2.051926in}}%
\pgfpathlineto{\pgfqpoint{1.540005in}{2.051926in}}%
\pgfpathlineto{\pgfqpoint{1.540005in}{2.074942in}}%
\pgfpathlineto{\pgfqpoint{1.550459in}{2.074942in}}%
\pgfpathlineto{\pgfqpoint{1.550459in}{2.060707in}}%
\pgfpathlineto{\pgfqpoint{1.560913in}{2.060707in}}%
\pgfpathlineto{\pgfqpoint{1.560913in}{2.063960in}}%
\pgfpathlineto{\pgfqpoint{1.571367in}{2.063960in}}%
\pgfpathlineto{\pgfqpoint{1.571367in}{2.060976in}}%
\pgfpathlineto{\pgfqpoint{1.581821in}{2.060976in}}%
\pgfpathlineto{\pgfqpoint{1.581821in}{2.052293in}}%
\pgfpathlineto{\pgfqpoint{1.592276in}{2.052293in}}%
\pgfpathlineto{\pgfqpoint{1.592276in}{2.049407in}}%
\pgfpathlineto{\pgfqpoint{1.602730in}{2.049407in}}%
\pgfpathlineto{\pgfqpoint{1.602730in}{2.054005in}}%
\pgfpathlineto{\pgfqpoint{1.613184in}{2.054005in}}%
\pgfpathlineto{\pgfqpoint{1.613184in}{2.027272in}}%
\pgfpathlineto{\pgfqpoint{1.623638in}{2.027272in}}%
\pgfpathlineto{\pgfqpoint{1.623638in}{2.024508in}}%
\pgfpathlineto{\pgfqpoint{1.634092in}{2.024508in}}%
\pgfpathlineto{\pgfqpoint{1.634092in}{2.022209in}}%
\pgfpathlineto{\pgfqpoint{1.644546in}{2.022209in}}%
\pgfpathlineto{\pgfqpoint{1.644546in}{1.996283in}}%
\pgfpathlineto{\pgfqpoint{1.655000in}{1.996283in}}%
\pgfpathlineto{\pgfqpoint{1.655000in}{1.991391in}}%
\pgfpathlineto{\pgfqpoint{1.665454in}{1.991391in}}%
\pgfpathlineto{\pgfqpoint{1.665454in}{1.979700in}}%
\pgfpathlineto{\pgfqpoint{1.686363in}{1.979944in}}%
\pgfpathlineto{\pgfqpoint{1.686363in}{1.958836in}}%
\pgfpathlineto{\pgfqpoint{1.696817in}{1.958836in}}%
\pgfpathlineto{\pgfqpoint{1.696817in}{1.942547in}}%
\pgfpathlineto{\pgfqpoint{1.707271in}{1.942547in}}%
\pgfpathlineto{\pgfqpoint{1.707271in}{1.920876in}}%
\pgfpathlineto{\pgfqpoint{1.717725in}{1.920876in}}%
\pgfpathlineto{\pgfqpoint{1.717725in}{1.904220in}}%
\pgfpathlineto{\pgfqpoint{1.728179in}{1.904220in}}%
\pgfpathlineto{\pgfqpoint{1.728179in}{1.876117in}}%
\pgfpathlineto{\pgfqpoint{1.738633in}{1.876117in}}%
\pgfpathlineto{\pgfqpoint{1.738633in}{1.865869in}}%
\pgfpathlineto{\pgfqpoint{1.749088in}{1.865869in}}%
\pgfpathlineto{\pgfqpoint{1.749088in}{1.844272in}}%
\pgfpathlineto{\pgfqpoint{1.759542in}{1.844272in}}%
\pgfpathlineto{\pgfqpoint{1.759542in}{1.829572in}}%
\pgfpathlineto{\pgfqpoint{1.769996in}{1.829572in}}%
\pgfpathlineto{\pgfqpoint{1.769996in}{1.818663in}}%
\pgfpathlineto{\pgfqpoint{1.780450in}{1.818663in}}%
\pgfpathlineto{\pgfqpoint{1.780450in}{1.781608in}}%
\pgfpathlineto{\pgfqpoint{1.801358in}{1.780801in}}%
\pgfpathlineto{\pgfqpoint{1.801358in}{1.743526in}}%
\pgfpathlineto{\pgfqpoint{1.811812in}{1.743526in}}%
\pgfpathlineto{\pgfqpoint{1.811812in}{1.726503in}}%
\pgfpathlineto{\pgfqpoint{1.822266in}{1.726503in}}%
\pgfpathlineto{\pgfqpoint{1.822266in}{1.702802in}}%
\pgfpathlineto{\pgfqpoint{1.832721in}{1.702802in}}%
\pgfpathlineto{\pgfqpoint{1.832721in}{1.692236in}}%
\pgfpathlineto{\pgfqpoint{1.843175in}{1.692236in}}%
\pgfpathlineto{\pgfqpoint{1.843175in}{1.667998in}}%
\pgfpathlineto{\pgfqpoint{1.853629in}{1.667998in}}%
\pgfpathlineto{\pgfqpoint{1.853629in}{1.642536in}}%
\pgfpathlineto{\pgfqpoint{1.864083in}{1.642536in}}%
\pgfpathlineto{\pgfqpoint{1.864083in}{1.611743in}}%
\pgfpathlineto{\pgfqpoint{1.874537in}{1.611743in}}%
\pgfpathlineto{\pgfqpoint{1.874537in}{1.591711in}}%
\pgfpathlineto{\pgfqpoint{1.884991in}{1.591711in}}%
\pgfpathlineto{\pgfqpoint{1.884991in}{1.575373in}}%
\pgfpathlineto{\pgfqpoint{1.895445in}{1.575373in}}%
\pgfpathlineto{\pgfqpoint{1.895445in}{1.544334in}}%
\pgfpathlineto{\pgfqpoint{1.905900in}{1.544334in}}%
\pgfpathlineto{\pgfqpoint{1.905900in}{1.528265in}}%
\pgfpathlineto{\pgfqpoint{1.916354in}{1.528265in}}%
\pgfpathlineto{\pgfqpoint{1.916354in}{1.503268in}}%
\pgfpathlineto{\pgfqpoint{1.926808in}{1.503268in}}%
\pgfpathlineto{\pgfqpoint{1.926808in}{1.483359in}}%
\pgfpathlineto{\pgfqpoint{1.937262in}{1.483359in}}%
\pgfpathlineto{\pgfqpoint{1.937262in}{1.459291in}}%
\pgfpathlineto{\pgfqpoint{1.947716in}{1.459291in}}%
\pgfpathlineto{\pgfqpoint{1.947716in}{1.432680in}}%
\pgfpathlineto{\pgfqpoint{1.958170in}{1.432680in}}%
\pgfpathlineto{\pgfqpoint{1.958170in}{1.412991in}}%
\pgfpathlineto{\pgfqpoint{1.968624in}{1.412991in}}%
\pgfpathlineto{\pgfqpoint{1.968624in}{1.392250in}}%
\pgfpathlineto{\pgfqpoint{1.979078in}{1.392250in}}%
\pgfpathlineto{\pgfqpoint{1.979078in}{1.369039in}}%
\pgfpathlineto{\pgfqpoint{1.989533in}{1.369039in}}%
\pgfpathlineto{\pgfqpoint{1.989533in}{1.337732in}}%
\pgfpathlineto{\pgfqpoint{1.999987in}{1.337732in}}%
\pgfpathlineto{\pgfqpoint{1.999987in}{1.321515in}}%
\pgfpathlineto{\pgfqpoint{2.010441in}{1.321515in}}%
\pgfpathlineto{\pgfqpoint{2.010441in}{1.293143in}}%
\pgfpathlineto{\pgfqpoint{2.020895in}{1.293143in}}%
\pgfpathlineto{\pgfqpoint{2.020895in}{1.279422in}}%
\pgfpathlineto{\pgfqpoint{2.031349in}{1.279422in}}%
\pgfpathlineto{\pgfqpoint{2.031349in}{1.263181in}}%
\pgfpathlineto{\pgfqpoint{2.041803in}{1.263181in}}%
\pgfpathlineto{\pgfqpoint{2.041803in}{1.226322in}}%
\pgfpathlineto{\pgfqpoint{2.052257in}{1.226322in}}%
\pgfpathlineto{\pgfqpoint{2.052257in}{1.217933in}}%
\pgfpathlineto{\pgfqpoint{2.062711in}{1.217933in}}%
\pgfpathlineto{\pgfqpoint{2.062711in}{1.194893in}}%
\pgfpathlineto{\pgfqpoint{2.073166in}{1.194893in}}%
\pgfpathlineto{\pgfqpoint{2.073166in}{1.178970in}}%
\pgfpathlineto{\pgfqpoint{2.083620in}{1.178970in}}%
\pgfpathlineto{\pgfqpoint{2.083620in}{1.152701in}}%
\pgfpathlineto{\pgfqpoint{2.094074in}{1.152701in}}%
\pgfpathlineto{\pgfqpoint{2.094074in}{1.139004in}}%
\pgfpathlineto{\pgfqpoint{2.104528in}{1.139004in}}%
\pgfpathlineto{\pgfqpoint{2.104528in}{1.124036in}}%
\pgfpathlineto{\pgfqpoint{2.114982in}{1.124036in}}%
\pgfpathlineto{\pgfqpoint{2.114982in}{1.104493in}}%
\pgfpathlineto{\pgfqpoint{2.125436in}{1.104493in}}%
\pgfpathlineto{\pgfqpoint{2.125436in}{1.082529in}}%
\pgfpathlineto{\pgfqpoint{2.135890in}{1.082529in}}%
\pgfpathlineto{\pgfqpoint{2.135890in}{1.072036in}}%
\pgfpathlineto{\pgfqpoint{2.146345in}{1.072036in}}%
\pgfpathlineto{\pgfqpoint{2.146345in}{1.051809in}}%
\pgfpathlineto{\pgfqpoint{2.156799in}{1.051809in}}%
\pgfpathlineto{\pgfqpoint{2.156799in}{1.042662in}}%
\pgfpathlineto{\pgfqpoint{2.167253in}{1.042662in}}%
\pgfpathlineto{\pgfqpoint{2.167253in}{1.014461in}}%
\pgfpathlineto{\pgfqpoint{2.177707in}{1.014461in}}%
\pgfpathlineto{\pgfqpoint{2.177707in}{1.007025in}}%
\pgfpathlineto{\pgfqpoint{2.188161in}{1.007025in}}%
\pgfpathlineto{\pgfqpoint{2.188161in}{0.995212in}}%
\pgfpathlineto{\pgfqpoint{2.198615in}{0.995212in}}%
\pgfpathlineto{\pgfqpoint{2.198615in}{0.973590in}}%
\pgfpathlineto{\pgfqpoint{2.209069in}{0.973590in}}%
\pgfpathlineto{\pgfqpoint{2.209069in}{0.964614in}}%
\pgfpathlineto{\pgfqpoint{2.219523in}{0.964614in}}%
\pgfpathlineto{\pgfqpoint{2.219523in}{0.954292in}}%
\pgfpathlineto{\pgfqpoint{2.229978in}{0.954292in}}%
\pgfpathlineto{\pgfqpoint{2.229978in}{0.933233in}}%
\pgfpathlineto{\pgfqpoint{2.240432in}{0.933233in}}%
\pgfpathlineto{\pgfqpoint{2.240432in}{0.925480in}}%
\pgfpathlineto{\pgfqpoint{2.250886in}{0.925480in}}%
\pgfpathlineto{\pgfqpoint{2.250886in}{0.909802in}}%
\pgfpathlineto{\pgfqpoint{2.261340in}{0.909802in}}%
\pgfpathlineto{\pgfqpoint{2.261340in}{0.895909in}}%
\pgfpathlineto{\pgfqpoint{2.271794in}{0.895909in}}%
\pgfpathlineto{\pgfqpoint{2.271794in}{0.888963in}}%
\pgfpathlineto{\pgfqpoint{2.282248in}{0.888963in}}%
\pgfpathlineto{\pgfqpoint{2.282248in}{0.873114in}}%
\pgfpathlineto{\pgfqpoint{2.292702in}{0.873114in}}%
\pgfpathlineto{\pgfqpoint{2.292702in}{0.859319in}}%
\pgfpathlineto{\pgfqpoint{2.303157in}{0.859319in}}%
\pgfpathlineto{\pgfqpoint{2.303157in}{0.850074in}}%
\pgfpathlineto{\pgfqpoint{2.313611in}{0.850074in}}%
\pgfpathlineto{\pgfqpoint{2.313611in}{0.835961in}}%
\pgfpathlineto{\pgfqpoint{2.324065in}{0.835961in}}%
\pgfpathlineto{\pgfqpoint{2.324065in}{0.832316in}}%
\pgfpathlineto{\pgfqpoint{2.334519in}{0.832316in}}%
\pgfpathlineto{\pgfqpoint{2.334519in}{0.818742in}}%
\pgfpathlineto{\pgfqpoint{2.344973in}{0.818742in}}%
\pgfpathlineto{\pgfqpoint{2.344973in}{0.804091in}}%
\pgfpathlineto{\pgfqpoint{2.355427in}{0.804091in}}%
\pgfpathlineto{\pgfqpoint{2.355427in}{0.797487in}}%
\pgfpathlineto{\pgfqpoint{2.365881in}{0.797487in}}%
\pgfpathlineto{\pgfqpoint{2.365881in}{0.790076in}}%
\pgfpathlineto{\pgfqpoint{2.376335in}{0.790076in}}%
\pgfpathlineto{\pgfqpoint{2.376335in}{0.781932in}}%
\pgfpathlineto{\pgfqpoint{2.386790in}{0.781932in}}%
\pgfpathlineto{\pgfqpoint{2.386790in}{0.776551in}}%
\pgfpathlineto{\pgfqpoint{2.397244in}{0.776551in}}%
\pgfpathlineto{\pgfqpoint{2.397244in}{0.762829in}}%
\pgfpathlineto{\pgfqpoint{2.407698in}{0.762829in}}%
\pgfpathlineto{\pgfqpoint{2.407698in}{0.755051in}}%
\pgfpathlineto{\pgfqpoint{2.418152in}{0.755051in}}%
\pgfpathlineto{\pgfqpoint{2.418152in}{0.748814in}}%
\pgfpathlineto{\pgfqpoint{2.428606in}{0.748814in}}%
\pgfpathlineto{\pgfqpoint{2.428606in}{0.737319in}}%
\pgfpathlineto{\pgfqpoint{2.439060in}{0.737319in}}%
\pgfpathlineto{\pgfqpoint{2.439060in}{0.731742in}}%
\pgfpathlineto{\pgfqpoint{2.449514in}{0.731742in}}%
\pgfpathlineto{\pgfqpoint{2.449514in}{0.725358in}}%
\pgfpathlineto{\pgfqpoint{2.459969in}{0.725358in}}%
\pgfpathlineto{\pgfqpoint{2.459969in}{0.721103in}}%
\pgfpathlineto{\pgfqpoint{2.470423in}{0.721103in}}%
\pgfpathlineto{\pgfqpoint{2.470423in}{0.713716in}}%
\pgfpathlineto{\pgfqpoint{2.480877in}{0.713716in}}%
\pgfpathlineto{\pgfqpoint{2.480877in}{0.702856in}}%
\pgfpathlineto{\pgfqpoint{2.491331in}{0.702856in}}%
\pgfpathlineto{\pgfqpoint{2.491331in}{0.699750in}}%
\pgfpathlineto{\pgfqpoint{2.501785in}{0.699750in}}%
\pgfpathlineto{\pgfqpoint{2.501785in}{0.694663in}}%
\pgfpathlineto{\pgfqpoint{2.512239in}{0.694663in}}%
\pgfpathlineto{\pgfqpoint{2.512239in}{0.688426in}}%
\pgfpathlineto{\pgfqpoint{2.522693in}{0.688426in}}%
\pgfpathlineto{\pgfqpoint{2.522693in}{0.684977in}}%
\pgfpathlineto{\pgfqpoint{2.533147in}{0.684977in}}%
\pgfpathlineto{\pgfqpoint{2.533147in}{0.678300in}}%
\pgfpathlineto{\pgfqpoint{2.543602in}{0.678300in}}%
\pgfpathlineto{\pgfqpoint{2.543602in}{0.672870in}}%
\pgfpathlineto{\pgfqpoint{2.554056in}{0.672870in}}%
\pgfpathlineto{\pgfqpoint{2.554056in}{0.667807in}}%
\pgfpathlineto{\pgfqpoint{2.564510in}{0.667807in}}%
\pgfpathlineto{\pgfqpoint{2.564510in}{0.662793in}}%
\pgfpathlineto{\pgfqpoint{2.574964in}{0.662793in}}%
\pgfpathlineto{\pgfqpoint{2.574964in}{0.655700in}}%
\pgfpathlineto{\pgfqpoint{2.585418in}{0.655700in}}%
\pgfpathlineto{\pgfqpoint{2.585418in}{0.653915in}}%
\pgfpathlineto{\pgfqpoint{2.595872in}{0.653915in}}%
\pgfpathlineto{\pgfqpoint{2.595872in}{0.650808in}}%
\pgfpathlineto{\pgfqpoint{2.606326in}{0.650808in}}%
\pgfpathlineto{\pgfqpoint{2.606326in}{0.643691in}}%
\pgfpathlineto{\pgfqpoint{2.616781in}{0.643691in}}%
\pgfpathlineto{\pgfqpoint{2.616781in}{0.645452in}}%
\pgfpathlineto{\pgfqpoint{2.627235in}{0.645452in}}%
\pgfpathlineto{\pgfqpoint{2.627235in}{0.636867in}}%
\pgfpathlineto{\pgfqpoint{2.637689in}{0.636867in}}%
\pgfpathlineto{\pgfqpoint{2.637689in}{0.634397in}}%
\pgfpathlineto{\pgfqpoint{2.648143in}{0.634397in}}%
\pgfpathlineto{\pgfqpoint{2.648143in}{0.632195in}}%
\pgfpathlineto{\pgfqpoint{2.658597in}{0.632195in}}%
\pgfpathlineto{\pgfqpoint{2.658597in}{0.626081in}}%
\pgfpathlineto{\pgfqpoint{2.679505in}{0.625860in}}%
\pgfpathlineto{\pgfqpoint{2.679505in}{0.619795in}}%
\pgfpathlineto{\pgfqpoint{2.689959in}{0.619795in}}%
\pgfpathlineto{\pgfqpoint{2.689959in}{0.616909in}}%
\pgfpathlineto{\pgfqpoint{2.700414in}{0.616909in}}%
\pgfpathlineto{\pgfqpoint{2.700414in}{0.614805in}}%
\pgfpathlineto{\pgfqpoint{2.710868in}{0.614805in}}%
\pgfpathlineto{\pgfqpoint{2.710868in}{0.609180in}}%
\pgfpathlineto{\pgfqpoint{2.742230in}{0.608593in}}%
\pgfpathlineto{\pgfqpoint{2.742230in}{0.604826in}}%
\pgfpathlineto{\pgfqpoint{2.763138in}{0.604043in}}%
\pgfpathlineto{\pgfqpoint{2.763138in}{0.599641in}}%
\pgfpathlineto{\pgfqpoint{2.784047in}{0.600423in}}%
\pgfpathlineto{\pgfqpoint{2.784047in}{0.553781in}}%
\pgfpathlineto{\pgfqpoint{2.784047in}{0.553781in}}%
\pgfusepath{stroke}%
\end{pgfscope}%
\begin{pgfscope}%
\pgfpathrectangle{\pgfqpoint{0.588679in}{0.553781in}}{\pgfqpoint{2.299909in}{1.597219in}}%
\pgfusepath{clip}%
\pgfsetbuttcap%
\pgfsetmiterjoin%
\pgfsetlinewidth{1.003750pt}%
\definecolor{currentstroke}{rgb}{0.949020,0.372549,0.360784}%
\pgfsetstrokecolor{currentstroke}%
\pgfsetdash{{1.000000pt}{1.650000pt}}{0.000000pt}%
\pgfpathmoveto{\pgfqpoint{0.693220in}{0.553781in}}%
\pgfpathlineto{\pgfqpoint{0.693220in}{0.877958in}}%
\pgfpathlineto{\pgfqpoint{0.703674in}{0.877958in}}%
\pgfpathlineto{\pgfqpoint{0.703674in}{0.921251in}}%
\pgfpathlineto{\pgfqpoint{0.714128in}{0.921251in}}%
\pgfpathlineto{\pgfqpoint{0.714128in}{0.935931in}}%
\pgfpathlineto{\pgfqpoint{0.724582in}{0.935931in}}%
\pgfpathlineto{\pgfqpoint{0.724582in}{1.003235in}}%
\pgfpathlineto{\pgfqpoint{0.735037in}{1.003235in}}%
\pgfpathlineto{\pgfqpoint{0.735037in}{0.989429in}}%
\pgfpathlineto{\pgfqpoint{0.745491in}{0.989429in}}%
\pgfpathlineto{\pgfqpoint{0.745491in}{1.043507in}}%
\pgfpathlineto{\pgfqpoint{0.755945in}{1.043507in}}%
\pgfpathlineto{\pgfqpoint{0.755945in}{1.013042in}}%
\pgfpathlineto{\pgfqpoint{0.766399in}{1.013042in}}%
\pgfpathlineto{\pgfqpoint{0.766399in}{1.107828in}}%
\pgfpathlineto{\pgfqpoint{0.776853in}{1.107828in}}%
\pgfpathlineto{\pgfqpoint{0.776853in}{1.071931in}}%
\pgfpathlineto{\pgfqpoint{0.787307in}{1.071931in}}%
\pgfpathlineto{\pgfqpoint{0.787307in}{1.101999in}}%
\pgfpathlineto{\pgfqpoint{0.797761in}{1.101999in}}%
\pgfpathlineto{\pgfqpoint{0.797761in}{1.069369in}}%
\pgfpathlineto{\pgfqpoint{0.808216in}{1.069369in}}%
\pgfpathlineto{\pgfqpoint{0.808216in}{1.137523in}}%
\pgfpathlineto{\pgfqpoint{0.818670in}{1.137523in}}%
\pgfpathlineto{\pgfqpoint{0.818670in}{1.092676in}}%
\pgfpathlineto{\pgfqpoint{0.829124in}{1.092676in}}%
\pgfpathlineto{\pgfqpoint{0.829124in}{1.121747in}}%
\pgfpathlineto{\pgfqpoint{0.839578in}{1.121747in}}%
\pgfpathlineto{\pgfqpoint{0.839578in}{1.093554in}}%
\pgfpathlineto{\pgfqpoint{0.850032in}{1.093554in}}%
\pgfpathlineto{\pgfqpoint{0.850032in}{1.141811in}}%
\pgfpathlineto{\pgfqpoint{0.860486in}{1.141811in}}%
\pgfpathlineto{\pgfqpoint{0.860486in}{1.105497in}}%
\pgfpathlineto{\pgfqpoint{0.870940in}{1.105497in}}%
\pgfpathlineto{\pgfqpoint{0.870940in}{1.166553in}}%
\pgfpathlineto{\pgfqpoint{0.881394in}{1.166553in}}%
\pgfpathlineto{\pgfqpoint{0.881394in}{1.102471in}}%
\pgfpathlineto{\pgfqpoint{0.891849in}{1.102471in}}%
\pgfpathlineto{\pgfqpoint{0.891849in}{1.157467in}}%
\pgfpathlineto{\pgfqpoint{0.902303in}{1.157467in}}%
\pgfpathlineto{\pgfqpoint{0.902303in}{1.110093in}}%
\pgfpathlineto{\pgfqpoint{0.912757in}{1.110093in}}%
\pgfpathlineto{\pgfqpoint{0.912757in}{1.180386in}}%
\pgfpathlineto{\pgfqpoint{0.923211in}{1.180386in}}%
\pgfpathlineto{\pgfqpoint{0.923211in}{1.127893in}}%
\pgfpathlineto{\pgfqpoint{0.933665in}{1.127893in}}%
\pgfpathlineto{\pgfqpoint{0.933665in}{1.175305in}}%
\pgfpathlineto{\pgfqpoint{0.944119in}{1.175305in}}%
\pgfpathlineto{\pgfqpoint{0.944119in}{1.140952in}}%
\pgfpathlineto{\pgfqpoint{0.954573in}{1.140952in}}%
\pgfpathlineto{\pgfqpoint{0.954573in}{1.194483in}}%
\pgfpathlineto{\pgfqpoint{0.965028in}{1.194483in}}%
\pgfpathlineto{\pgfqpoint{0.965028in}{1.146447in}}%
\pgfpathlineto{\pgfqpoint{0.975482in}{1.146447in}}%
\pgfpathlineto{\pgfqpoint{0.975482in}{1.195723in}}%
\pgfpathlineto{\pgfqpoint{0.985936in}{1.195723in}}%
\pgfpathlineto{\pgfqpoint{0.985936in}{1.159834in}}%
\pgfpathlineto{\pgfqpoint{0.996390in}{1.159834in}}%
\pgfpathlineto{\pgfqpoint{0.996390in}{1.192083in}}%
\pgfpathlineto{\pgfqpoint{1.006844in}{1.192083in}}%
\pgfpathlineto{\pgfqpoint{1.006844in}{1.159190in}}%
\pgfpathlineto{\pgfqpoint{1.017298in}{1.159190in}}%
\pgfpathlineto{\pgfqpoint{1.017298in}{1.205846in}}%
\pgfpathlineto{\pgfqpoint{1.027752in}{1.205846in}}%
\pgfpathlineto{\pgfqpoint{1.027752in}{1.172206in}}%
\pgfpathlineto{\pgfqpoint{1.038206in}{1.172206in}}%
\pgfpathlineto{\pgfqpoint{1.038206in}{1.204365in}}%
\pgfpathlineto{\pgfqpoint{1.048661in}{1.204365in}}%
\pgfpathlineto{\pgfqpoint{1.048661in}{1.174096in}}%
\pgfpathlineto{\pgfqpoint{1.059115in}{1.174096in}}%
\pgfpathlineto{\pgfqpoint{1.059115in}{1.211436in}}%
\pgfpathlineto{\pgfqpoint{1.069569in}{1.211436in}}%
\pgfpathlineto{\pgfqpoint{1.069569in}{1.177227in}}%
\pgfpathlineto{\pgfqpoint{1.080023in}{1.177227in}}%
\pgfpathlineto{\pgfqpoint{1.080023in}{1.223612in}}%
\pgfpathlineto{\pgfqpoint{1.090477in}{1.223612in}}%
\pgfpathlineto{\pgfqpoint{1.090477in}{1.181150in}}%
\pgfpathlineto{\pgfqpoint{1.100931in}{1.181150in}}%
\pgfpathlineto{\pgfqpoint{1.100931in}{1.219374in}}%
\pgfpathlineto{\pgfqpoint{1.111385in}{1.219374in}}%
\pgfpathlineto{\pgfqpoint{1.111385in}{1.200500in}}%
\pgfpathlineto{\pgfqpoint{1.121840in}{1.200500in}}%
\pgfpathlineto{\pgfqpoint{1.121840in}{1.218801in}}%
\pgfpathlineto{\pgfqpoint{1.132294in}{1.218801in}}%
\pgfpathlineto{\pgfqpoint{1.132294in}{1.198938in}}%
\pgfpathlineto{\pgfqpoint{1.142748in}{1.198938in}}%
\pgfpathlineto{\pgfqpoint{1.142748in}{1.228139in}}%
\pgfpathlineto{\pgfqpoint{1.153202in}{1.228139in}}%
\pgfpathlineto{\pgfqpoint{1.153202in}{1.208625in}}%
\pgfpathlineto{\pgfqpoint{1.163656in}{1.208625in}}%
\pgfpathlineto{\pgfqpoint{1.163656in}{1.232579in}}%
\pgfpathlineto{\pgfqpoint{1.174110in}{1.232579in}}%
\pgfpathlineto{\pgfqpoint{1.174110in}{1.201764in}}%
\pgfpathlineto{\pgfqpoint{1.184564in}{1.201764in}}%
\pgfpathlineto{\pgfqpoint{1.184564in}{1.215001in}}%
\pgfpathlineto{\pgfqpoint{1.195018in}{1.215001in}}%
\pgfpathlineto{\pgfqpoint{1.195018in}{1.201947in}}%
\pgfpathlineto{\pgfqpoint{1.205473in}{1.201947in}}%
\pgfpathlineto{\pgfqpoint{1.205473in}{1.231160in}}%
\pgfpathlineto{\pgfqpoint{1.215927in}{1.231160in}}%
\pgfpathlineto{\pgfqpoint{1.215927in}{1.200183in}}%
\pgfpathlineto{\pgfqpoint{1.226381in}{1.200183in}}%
\pgfpathlineto{\pgfqpoint{1.226381in}{1.221284in}}%
\pgfpathlineto{\pgfqpoint{1.236835in}{1.221284in}}%
\pgfpathlineto{\pgfqpoint{1.236835in}{1.210494in}}%
\pgfpathlineto{\pgfqpoint{1.247289in}{1.210494in}}%
\pgfpathlineto{\pgfqpoint{1.247289in}{1.229428in}}%
\pgfpathlineto{\pgfqpoint{1.257743in}{1.229428in}}%
\pgfpathlineto{\pgfqpoint{1.257743in}{1.215612in}}%
\pgfpathlineto{\pgfqpoint{1.268197in}{1.215612in}}%
\pgfpathlineto{\pgfqpoint{1.268197in}{1.229687in}}%
\pgfpathlineto{\pgfqpoint{1.278652in}{1.229687in}}%
\pgfpathlineto{\pgfqpoint{1.278652in}{1.213598in}}%
\pgfpathlineto{\pgfqpoint{1.289106in}{1.213598in}}%
\pgfpathlineto{\pgfqpoint{1.289106in}{1.237069in}}%
\pgfpathlineto{\pgfqpoint{1.299560in}{1.237069in}}%
\pgfpathlineto{\pgfqpoint{1.299560in}{1.219903in}}%
\pgfpathlineto{\pgfqpoint{1.310014in}{1.219903in}}%
\pgfpathlineto{\pgfqpoint{1.310014in}{1.232066in}}%
\pgfpathlineto{\pgfqpoint{1.320468in}{1.232066in}}%
\pgfpathlineto{\pgfqpoint{1.320468in}{1.218089in}}%
\pgfpathlineto{\pgfqpoint{1.330922in}{1.218089in}}%
\pgfpathlineto{\pgfqpoint{1.330922in}{1.235618in}}%
\pgfpathlineto{\pgfqpoint{1.341376in}{1.235618in}}%
\pgfpathlineto{\pgfqpoint{1.341376in}{1.225284in}}%
\pgfpathlineto{\pgfqpoint{1.351830in}{1.225284in}}%
\pgfpathlineto{\pgfqpoint{1.351830in}{1.235247in}}%
\pgfpathlineto{\pgfqpoint{1.362285in}{1.235247in}}%
\pgfpathlineto{\pgfqpoint{1.362285in}{1.218976in}}%
\pgfpathlineto{\pgfqpoint{1.372739in}{1.218976in}}%
\pgfpathlineto{\pgfqpoint{1.372739in}{1.239419in}}%
\pgfpathlineto{\pgfqpoint{1.383193in}{1.239419in}}%
\pgfpathlineto{\pgfqpoint{1.383193in}{1.223694in}}%
\pgfpathlineto{\pgfqpoint{1.393647in}{1.223694in}}%
\pgfpathlineto{\pgfqpoint{1.393647in}{1.232665in}}%
\pgfpathlineto{\pgfqpoint{1.404101in}{1.232665in}}%
\pgfpathlineto{\pgfqpoint{1.404101in}{1.227191in}}%
\pgfpathlineto{\pgfqpoint{1.414555in}{1.227191in}}%
\pgfpathlineto{\pgfqpoint{1.414555in}{1.231686in}}%
\pgfpathlineto{\pgfqpoint{1.425009in}{1.231686in}}%
\pgfpathlineto{\pgfqpoint{1.425009in}{1.224285in}}%
\pgfpathlineto{\pgfqpoint{1.435464in}{1.224285in}}%
\pgfpathlineto{\pgfqpoint{1.435464in}{1.237091in}}%
\pgfpathlineto{\pgfqpoint{1.445918in}{1.237091in}}%
\pgfpathlineto{\pgfqpoint{1.445918in}{1.231273in}}%
\pgfpathlineto{\pgfqpoint{1.456372in}{1.231273in}}%
\pgfpathlineto{\pgfqpoint{1.456372in}{1.233960in}}%
\pgfpathlineto{\pgfqpoint{1.466826in}{1.233960in}}%
\pgfpathlineto{\pgfqpoint{1.466826in}{1.224530in}}%
\pgfpathlineto{\pgfqpoint{1.477280in}{1.224530in}}%
\pgfpathlineto{\pgfqpoint{1.477280in}{1.228906in}}%
\pgfpathlineto{\pgfqpoint{1.487734in}{1.228906in}}%
\pgfpathlineto{\pgfqpoint{1.487734in}{1.227535in}}%
\pgfpathlineto{\pgfqpoint{1.498188in}{1.227535in}}%
\pgfpathlineto{\pgfqpoint{1.498188in}{1.235205in}}%
\pgfpathlineto{\pgfqpoint{1.508642in}{1.235205in}}%
\pgfpathlineto{\pgfqpoint{1.508642in}{1.232722in}}%
\pgfpathlineto{\pgfqpoint{1.529551in}{1.233574in}}%
\pgfpathlineto{\pgfqpoint{1.529551in}{1.223633in}}%
\pgfpathlineto{\pgfqpoint{1.540005in}{1.223633in}}%
\pgfpathlineto{\pgfqpoint{1.540005in}{1.233899in}}%
\pgfpathlineto{\pgfqpoint{1.550459in}{1.233899in}}%
\pgfpathlineto{\pgfqpoint{1.550459in}{1.226748in}}%
\pgfpathlineto{\pgfqpoint{1.560913in}{1.226748in}}%
\pgfpathlineto{\pgfqpoint{1.560913in}{1.228238in}}%
\pgfpathlineto{\pgfqpoint{1.581821in}{1.228593in}}%
\pgfpathlineto{\pgfqpoint{1.581821in}{1.224763in}}%
\pgfpathlineto{\pgfqpoint{1.592276in}{1.224763in}}%
\pgfpathlineto{\pgfqpoint{1.592276in}{1.227588in}}%
\pgfpathlineto{\pgfqpoint{1.602730in}{1.227588in}}%
\pgfpathlineto{\pgfqpoint{1.602730in}{1.229667in}}%
\pgfpathlineto{\pgfqpoint{1.613184in}{1.229667in}}%
\pgfpathlineto{\pgfqpoint{1.613184in}{1.224106in}}%
\pgfpathlineto{\pgfqpoint{1.623638in}{1.224106in}}%
\pgfpathlineto{\pgfqpoint{1.623638in}{1.222849in}}%
\pgfpathlineto{\pgfqpoint{1.634092in}{1.222849in}}%
\pgfpathlineto{\pgfqpoint{1.634092in}{1.230467in}}%
\pgfpathlineto{\pgfqpoint{1.644546in}{1.230467in}}%
\pgfpathlineto{\pgfqpoint{1.644546in}{1.218610in}}%
\pgfpathlineto{\pgfqpoint{1.655000in}{1.218610in}}%
\pgfpathlineto{\pgfqpoint{1.655000in}{1.226990in}}%
\pgfpathlineto{\pgfqpoint{1.665454in}{1.226990in}}%
\pgfpathlineto{\pgfqpoint{1.665454in}{1.221624in}}%
\pgfpathlineto{\pgfqpoint{1.675909in}{1.221624in}}%
\pgfpathlineto{\pgfqpoint{1.675909in}{1.234415in}}%
\pgfpathlineto{\pgfqpoint{1.686363in}{1.234415in}}%
\pgfpathlineto{\pgfqpoint{1.686363in}{1.224438in}}%
\pgfpathlineto{\pgfqpoint{1.696817in}{1.224438in}}%
\pgfpathlineto{\pgfqpoint{1.696817in}{1.231083in}}%
\pgfpathlineto{\pgfqpoint{1.707271in}{1.231083in}}%
\pgfpathlineto{\pgfqpoint{1.707271in}{1.220677in}}%
\pgfpathlineto{\pgfqpoint{1.717725in}{1.220677in}}%
\pgfpathlineto{\pgfqpoint{1.717725in}{1.228562in}}%
\pgfpathlineto{\pgfqpoint{1.728179in}{1.228562in}}%
\pgfpathlineto{\pgfqpoint{1.728179in}{1.214524in}}%
\pgfpathlineto{\pgfqpoint{1.738633in}{1.214524in}}%
\pgfpathlineto{\pgfqpoint{1.738633in}{1.227080in}}%
\pgfpathlineto{\pgfqpoint{1.749088in}{1.227080in}}%
\pgfpathlineto{\pgfqpoint{1.749088in}{1.216019in}}%
\pgfpathlineto{\pgfqpoint{1.759542in}{1.216019in}}%
\pgfpathlineto{\pgfqpoint{1.759542in}{1.227739in}}%
\pgfpathlineto{\pgfqpoint{1.769996in}{1.227739in}}%
\pgfpathlineto{\pgfqpoint{1.769996in}{1.221976in}}%
\pgfpathlineto{\pgfqpoint{1.780450in}{1.221976in}}%
\pgfpathlineto{\pgfqpoint{1.780450in}{1.223133in}}%
\pgfpathlineto{\pgfqpoint{1.801358in}{1.222872in}}%
\pgfpathlineto{\pgfqpoint{1.801358in}{1.224722in}}%
\pgfpathlineto{\pgfqpoint{1.811812in}{1.224722in}}%
\pgfpathlineto{\pgfqpoint{1.811812in}{1.215122in}}%
\pgfpathlineto{\pgfqpoint{1.822266in}{1.215122in}}%
\pgfpathlineto{\pgfqpoint{1.822266in}{1.225721in}}%
\pgfpathlineto{\pgfqpoint{1.832721in}{1.225721in}}%
\pgfpathlineto{\pgfqpoint{1.832721in}{1.219576in}}%
\pgfpathlineto{\pgfqpoint{1.843175in}{1.219576in}}%
\pgfpathlineto{\pgfqpoint{1.843175in}{1.231112in}}%
\pgfpathlineto{\pgfqpoint{1.853629in}{1.231112in}}%
\pgfpathlineto{\pgfqpoint{1.853629in}{1.215698in}}%
\pgfpathlineto{\pgfqpoint{1.864083in}{1.215698in}}%
\pgfpathlineto{\pgfqpoint{1.864083in}{1.223862in}}%
\pgfpathlineto{\pgfqpoint{1.874537in}{1.223862in}}%
\pgfpathlineto{\pgfqpoint{1.874537in}{1.211753in}}%
\pgfpathlineto{\pgfqpoint{1.884991in}{1.211753in}}%
\pgfpathlineto{\pgfqpoint{1.884991in}{1.229753in}}%
\pgfpathlineto{\pgfqpoint{1.895445in}{1.229753in}}%
\pgfpathlineto{\pgfqpoint{1.895445in}{1.209286in}}%
\pgfpathlineto{\pgfqpoint{1.905900in}{1.209286in}}%
\pgfpathlineto{\pgfqpoint{1.905900in}{1.228750in}}%
\pgfpathlineto{\pgfqpoint{1.916354in}{1.228750in}}%
\pgfpathlineto{\pgfqpoint{1.916354in}{1.211632in}}%
\pgfpathlineto{\pgfqpoint{1.926808in}{1.211632in}}%
\pgfpathlineto{\pgfqpoint{1.926808in}{1.229230in}}%
\pgfpathlineto{\pgfqpoint{1.937262in}{1.229230in}}%
\pgfpathlineto{\pgfqpoint{1.937262in}{1.212527in}}%
\pgfpathlineto{\pgfqpoint{1.947716in}{1.212527in}}%
\pgfpathlineto{\pgfqpoint{1.947716in}{1.225169in}}%
\pgfpathlineto{\pgfqpoint{1.958170in}{1.225169in}}%
\pgfpathlineto{\pgfqpoint{1.958170in}{1.211072in}}%
\pgfpathlineto{\pgfqpoint{1.968624in}{1.211072in}}%
\pgfpathlineto{\pgfqpoint{1.968624in}{1.228501in}}%
\pgfpathlineto{\pgfqpoint{1.979078in}{1.228501in}}%
\pgfpathlineto{\pgfqpoint{1.979078in}{1.210416in}}%
\pgfpathlineto{\pgfqpoint{1.989533in}{1.210416in}}%
\pgfpathlineto{\pgfqpoint{1.989533in}{1.219628in}}%
\pgfpathlineto{\pgfqpoint{1.999987in}{1.219628in}}%
\pgfpathlineto{\pgfqpoint{1.999987in}{1.207167in}}%
\pgfpathlineto{\pgfqpoint{2.010441in}{1.207167in}}%
\pgfpathlineto{\pgfqpoint{2.010441in}{1.217770in}}%
\pgfpathlineto{\pgfqpoint{2.020895in}{1.217770in}}%
\pgfpathlineto{\pgfqpoint{2.020895in}{1.206417in}}%
\pgfpathlineto{\pgfqpoint{2.031349in}{1.206417in}}%
\pgfpathlineto{\pgfqpoint{2.031349in}{1.228502in}}%
\pgfpathlineto{\pgfqpoint{2.041803in}{1.228502in}}%
\pgfpathlineto{\pgfqpoint{2.041803in}{1.194117in}}%
\pgfpathlineto{\pgfqpoint{2.052257in}{1.194117in}}%
\pgfpathlineto{\pgfqpoint{2.052257in}{1.223723in}}%
\pgfpathlineto{\pgfqpoint{2.062711in}{1.223723in}}%
\pgfpathlineto{\pgfqpoint{2.062711in}{1.201713in}}%
\pgfpathlineto{\pgfqpoint{2.073166in}{1.201713in}}%
\pgfpathlineto{\pgfqpoint{2.073166in}{1.223552in}}%
\pgfpathlineto{\pgfqpoint{2.083620in}{1.223552in}}%
\pgfpathlineto{\pgfqpoint{2.083620in}{1.196317in}}%
\pgfpathlineto{\pgfqpoint{2.094074in}{1.196317in}}%
\pgfpathlineto{\pgfqpoint{2.094074in}{1.220421in}}%
\pgfpathlineto{\pgfqpoint{2.104528in}{1.220421in}}%
\pgfpathlineto{\pgfqpoint{2.104528in}{1.204218in}}%
\pgfpathlineto{\pgfqpoint{2.114982in}{1.204218in}}%
\pgfpathlineto{\pgfqpoint{2.114982in}{1.221557in}}%
\pgfpathlineto{\pgfqpoint{2.125436in}{1.221557in}}%
\pgfpathlineto{\pgfqpoint{2.125436in}{1.195924in}}%
\pgfpathlineto{\pgfqpoint{2.135890in}{1.195924in}}%
\pgfpathlineto{\pgfqpoint{2.135890in}{1.223421in}}%
\pgfpathlineto{\pgfqpoint{2.146345in}{1.223421in}}%
\pgfpathlineto{\pgfqpoint{2.146345in}{1.198529in}}%
\pgfpathlineto{\pgfqpoint{2.156799in}{1.198529in}}%
\pgfpathlineto{\pgfqpoint{2.156799in}{1.227552in}}%
\pgfpathlineto{\pgfqpoint{2.167253in}{1.227552in}}%
\pgfpathlineto{\pgfqpoint{2.167253in}{1.190196in}}%
\pgfpathlineto{\pgfqpoint{2.177707in}{1.190196in}}%
\pgfpathlineto{\pgfqpoint{2.177707in}{1.220648in}}%
\pgfpathlineto{\pgfqpoint{2.188161in}{1.220648in}}%
\pgfpathlineto{\pgfqpoint{2.188161in}{1.204927in}}%
\pgfpathlineto{\pgfqpoint{2.198615in}{1.204927in}}%
\pgfpathlineto{\pgfqpoint{2.198615in}{1.213885in}}%
\pgfpathlineto{\pgfqpoint{2.209069in}{1.213885in}}%
\pgfpathlineto{\pgfqpoint{2.209069in}{1.201933in}}%
\pgfpathlineto{\pgfqpoint{2.219523in}{1.201933in}}%
\pgfpathlineto{\pgfqpoint{2.219523in}{1.227538in}}%
\pgfpathlineto{\pgfqpoint{2.229978in}{1.227538in}}%
\pgfpathlineto{\pgfqpoint{2.229978in}{1.194376in}}%
\pgfpathlineto{\pgfqpoint{2.240432in}{1.194376in}}%
\pgfpathlineto{\pgfqpoint{2.240432in}{1.223583in}}%
\pgfpathlineto{\pgfqpoint{2.250886in}{1.223583in}}%
\pgfpathlineto{\pgfqpoint{2.250886in}{1.197576in}}%
\pgfpathlineto{\pgfqpoint{2.261340in}{1.197576in}}%
\pgfpathlineto{\pgfqpoint{2.261340in}{1.215156in}}%
\pgfpathlineto{\pgfqpoint{2.271794in}{1.215156in}}%
\pgfpathlineto{\pgfqpoint{2.271794in}{1.203746in}}%
\pgfpathlineto{\pgfqpoint{2.282248in}{1.203746in}}%
\pgfpathlineto{\pgfqpoint{2.282248in}{1.216984in}}%
\pgfpathlineto{\pgfqpoint{2.292702in}{1.216984in}}%
\pgfpathlineto{\pgfqpoint{2.292702in}{1.190008in}}%
\pgfpathlineto{\pgfqpoint{2.303157in}{1.190008in}}%
\pgfpathlineto{\pgfqpoint{2.303157in}{1.215842in}}%
\pgfpathlineto{\pgfqpoint{2.313611in}{1.215842in}}%
\pgfpathlineto{\pgfqpoint{2.313611in}{1.185466in}}%
\pgfpathlineto{\pgfqpoint{2.324065in}{1.185466in}}%
\pgfpathlineto{\pgfqpoint{2.324065in}{1.224336in}}%
\pgfpathlineto{\pgfqpoint{2.334519in}{1.224336in}}%
\pgfpathlineto{\pgfqpoint{2.334519in}{1.192546in}}%
\pgfpathlineto{\pgfqpoint{2.344973in}{1.192546in}}%
\pgfpathlineto{\pgfqpoint{2.344973in}{1.203805in}}%
\pgfpathlineto{\pgfqpoint{2.355427in}{1.203805in}}%
\pgfpathlineto{\pgfqpoint{2.355427in}{1.187355in}}%
\pgfpathlineto{\pgfqpoint{2.365881in}{1.187355in}}%
\pgfpathlineto{\pgfqpoint{2.365881in}{1.216408in}}%
\pgfpathlineto{\pgfqpoint{2.376335in}{1.216408in}}%
\pgfpathlineto{\pgfqpoint{2.376335in}{1.194372in}}%
\pgfpathlineto{\pgfqpoint{2.386790in}{1.194372in}}%
\pgfpathlineto{\pgfqpoint{2.386790in}{1.228937in}}%
\pgfpathlineto{\pgfqpoint{2.397244in}{1.228937in}}%
\pgfpathlineto{\pgfqpoint{2.397244in}{1.188459in}}%
\pgfpathlineto{\pgfqpoint{2.407698in}{1.188459in}}%
\pgfpathlineto{\pgfqpoint{2.407698in}{1.213591in}}%
\pgfpathlineto{\pgfqpoint{2.418152in}{1.213591in}}%
\pgfpathlineto{\pgfqpoint{2.418152in}{1.193944in}}%
\pgfpathlineto{\pgfqpoint{2.428606in}{1.193944in}}%
\pgfpathlineto{\pgfqpoint{2.428606in}{1.205012in}}%
\pgfpathlineto{\pgfqpoint{2.439060in}{1.205012in}}%
\pgfpathlineto{\pgfqpoint{2.439060in}{1.186283in}}%
\pgfpathlineto{\pgfqpoint{2.449514in}{1.186283in}}%
\pgfpathlineto{\pgfqpoint{2.449514in}{1.213062in}}%
\pgfpathlineto{\pgfqpoint{2.459969in}{1.213062in}}%
\pgfpathlineto{\pgfqpoint{2.459969in}{1.198302in}}%
\pgfpathlineto{\pgfqpoint{2.470423in}{1.198302in}}%
\pgfpathlineto{\pgfqpoint{2.470423in}{1.219633in}}%
\pgfpathlineto{\pgfqpoint{2.480877in}{1.219633in}}%
\pgfpathlineto{\pgfqpoint{2.480877in}{1.175535in}}%
\pgfpathlineto{\pgfqpoint{2.491331in}{1.175535in}}%
\pgfpathlineto{\pgfqpoint{2.491331in}{1.212627in}}%
\pgfpathlineto{\pgfqpoint{2.501785in}{1.212627in}}%
\pgfpathlineto{\pgfqpoint{2.501785in}{1.190838in}}%
\pgfpathlineto{\pgfqpoint{2.512239in}{1.190838in}}%
\pgfpathlineto{\pgfqpoint{2.512239in}{1.213215in}}%
\pgfpathlineto{\pgfqpoint{2.522693in}{1.213215in}}%
\pgfpathlineto{\pgfqpoint{2.522693in}{1.197377in}}%
\pgfpathlineto{\pgfqpoint{2.533147in}{1.197377in}}%
\pgfpathlineto{\pgfqpoint{2.533147in}{1.216128in}}%
\pgfpathlineto{\pgfqpoint{2.543602in}{1.216128in}}%
\pgfpathlineto{\pgfqpoint{2.543602in}{1.187757in}}%
\pgfpathlineto{\pgfqpoint{2.554056in}{1.187757in}}%
\pgfpathlineto{\pgfqpoint{2.554056in}{1.213327in}}%
\pgfpathlineto{\pgfqpoint{2.564510in}{1.213327in}}%
\pgfpathlineto{\pgfqpoint{2.564510in}{1.184325in}}%
\pgfpathlineto{\pgfqpoint{2.574964in}{1.184325in}}%
\pgfpathlineto{\pgfqpoint{2.574964in}{1.195341in}}%
\pgfpathlineto{\pgfqpoint{2.585418in}{1.195341in}}%
\pgfpathlineto{\pgfqpoint{2.585418in}{1.184386in}}%
\pgfpathlineto{\pgfqpoint{2.595872in}{1.184386in}}%
\pgfpathlineto{\pgfqpoint{2.595872in}{1.218685in}}%
\pgfpathlineto{\pgfqpoint{2.606326in}{1.218685in}}%
\pgfpathlineto{\pgfqpoint{2.606326in}{1.171043in}}%
\pgfpathlineto{\pgfqpoint{2.616781in}{1.171043in}}%
\pgfpathlineto{\pgfqpoint{2.616781in}{1.239303in}}%
\pgfpathlineto{\pgfqpoint{2.627235in}{1.239303in}}%
\pgfpathlineto{\pgfqpoint{2.627235in}{1.176242in}}%
\pgfpathlineto{\pgfqpoint{2.637689in}{1.176242in}}%
\pgfpathlineto{\pgfqpoint{2.637689in}{1.211972in}}%
\pgfpathlineto{\pgfqpoint{2.648143in}{1.211972in}}%
\pgfpathlineto{\pgfqpoint{2.648143in}{1.195195in}}%
\pgfpathlineto{\pgfqpoint{2.658597in}{1.195195in}}%
\pgfpathlineto{\pgfqpoint{2.658597in}{1.198147in}}%
\pgfpathlineto{\pgfqpoint{2.679505in}{1.197038in}}%
\pgfpathlineto{\pgfqpoint{2.679505in}{1.193132in}}%
\pgfpathlineto{\pgfqpoint{2.689959in}{1.193132in}}%
\pgfpathlineto{\pgfqpoint{2.689959in}{1.166666in}}%
\pgfpathlineto{\pgfqpoint{2.700414in}{1.166666in}}%
\pgfpathlineto{\pgfqpoint{2.700414in}{1.193131in}}%
\pgfpathlineto{\pgfqpoint{2.710868in}{1.193131in}}%
\pgfpathlineto{\pgfqpoint{2.710868in}{1.135134in}}%
\pgfpathlineto{\pgfqpoint{2.721322in}{1.135134in}}%
\pgfpathlineto{\pgfqpoint{2.721322in}{1.182433in}}%
\pgfpathlineto{\pgfqpoint{2.731776in}{1.182433in}}%
\pgfpathlineto{\pgfqpoint{2.731776in}{1.168352in}}%
\pgfpathlineto{\pgfqpoint{2.742230in}{1.168352in}}%
\pgfpathlineto{\pgfqpoint{2.742230in}{1.154166in}}%
\pgfpathlineto{\pgfqpoint{2.752684in}{1.154166in}}%
\pgfpathlineto{\pgfqpoint{2.752684in}{1.145569in}}%
\pgfpathlineto{\pgfqpoint{2.763138in}{1.145569in}}%
\pgfpathlineto{\pgfqpoint{2.763138in}{1.107815in}}%
\pgfpathlineto{\pgfqpoint{2.773593in}{1.107815in}}%
\pgfpathlineto{\pgfqpoint{2.773593in}{1.117425in}}%
\pgfpathlineto{\pgfqpoint{2.784047in}{1.117425in}}%
\pgfpathlineto{\pgfqpoint{2.784047in}{0.553781in}}%
\pgfpathlineto{\pgfqpoint{2.784047in}{0.553781in}}%
\pgfusepath{stroke}%
\end{pgfscope}%
\begin{pgfscope}%
\pgfsetrectcap%
\pgfsetmiterjoin%
\pgfsetlinewidth{0.803000pt}%
\definecolor{currentstroke}{rgb}{0.000000,0.000000,0.000000}%
\pgfsetstrokecolor{currentstroke}%
\pgfsetdash{}{0pt}%
\pgfpathmoveto{\pgfqpoint{0.588679in}{0.553781in}}%
\pgfpathlineto{\pgfqpoint{0.588679in}{2.151000in}}%
\pgfusepath{stroke}%
\end{pgfscope}%
\begin{pgfscope}%
\pgfsetrectcap%
\pgfsetmiterjoin%
\pgfsetlinewidth{0.803000pt}%
\definecolor{currentstroke}{rgb}{0.000000,0.000000,0.000000}%
\pgfsetstrokecolor{currentstroke}%
\pgfsetdash{}{0pt}%
\pgfpathmoveto{\pgfqpoint{2.888588in}{0.553781in}}%
\pgfpathlineto{\pgfqpoint{2.888588in}{2.151000in}}%
\pgfusepath{stroke}%
\end{pgfscope}%
\begin{pgfscope}%
\pgfsetrectcap%
\pgfsetmiterjoin%
\pgfsetlinewidth{0.803000pt}%
\definecolor{currentstroke}{rgb}{0.000000,0.000000,0.000000}%
\pgfsetstrokecolor{currentstroke}%
\pgfsetdash{}{0pt}%
\pgfpathmoveto{\pgfqpoint{0.588679in}{0.553781in}}%
\pgfpathlineto{\pgfqpoint{2.888588in}{0.553781in}}%
\pgfusepath{stroke}%
\end{pgfscope}%
\begin{pgfscope}%
\pgfsetrectcap%
\pgfsetmiterjoin%
\pgfsetlinewidth{0.803000pt}%
\definecolor{currentstroke}{rgb}{0.000000,0.000000,0.000000}%
\pgfsetstrokecolor{currentstroke}%
\pgfsetdash{}{0pt}%
\pgfpathmoveto{\pgfqpoint{0.588679in}{2.151000in}}%
\pgfpathlineto{\pgfqpoint{2.888588in}{2.151000in}}%
\pgfusepath{stroke}%
\end{pgfscope}%
\begin{pgfscope}%
\definecolor{textcolor}{rgb}{0.000000,0.000000,0.000000}%
\pgfsetstrokecolor{textcolor}%
\pgfsetfillcolor{textcolor}%
\pgftext[x=0.588679in,y=2.234333in,left,base]{\color{textcolor}\rmfamily\fontsize{12.000000}{14.400000}\selectfont Weighted energy distribution}%
\end{pgfscope}%
\begin{pgfscope}%
\pgfsetbuttcap%
\pgfsetmiterjoin%
\definecolor{currentfill}{rgb}{1.000000,1.000000,1.000000}%
\pgfsetfillcolor{currentfill}%
\pgfsetfillopacity{0.800000}%
\pgfsetlinewidth{1.003750pt}%
\definecolor{currentstroke}{rgb}{0.800000,0.800000,0.800000}%
\pgfsetstrokecolor{currentstroke}%
\pgfsetstrokeopacity{0.800000}%
\pgfsetdash{}{0pt}%
\pgfpathmoveto{\pgfqpoint{1.852366in}{1.749889in}}%
\pgfpathlineto{\pgfqpoint{2.810810in}{1.749889in}}%
\pgfpathquadraticcurveto{\pgfqpoint{2.833032in}{1.749889in}}{\pgfqpoint{2.833032in}{1.772111in}}%
\pgfpathlineto{\pgfqpoint{2.833032in}{2.073222in}}%
\pgfpathquadraticcurveto{\pgfqpoint{2.833032in}{2.095444in}}{\pgfqpoint{2.810810in}{2.095444in}}%
\pgfpathlineto{\pgfqpoint{1.852366in}{2.095444in}}%
\pgfpathquadraticcurveto{\pgfqpoint{1.830144in}{2.095444in}}{\pgfqpoint{1.830144in}{2.073222in}}%
\pgfpathlineto{\pgfqpoint{1.830144in}{1.772111in}}%
\pgfpathquadraticcurveto{\pgfqpoint{1.830144in}{1.749889in}}{\pgfqpoint{1.852366in}{1.749889in}}%
\pgfpathclose%
\pgfusepath{stroke,fill}%
\end{pgfscope}%
\begin{pgfscope}%
\pgfsetbuttcap%
\pgfsetmiterjoin%
\pgfsetlinewidth{1.003750pt}%
\definecolor{currentstroke}{rgb}{0.313725,0.317647,0.309804}%
\pgfsetstrokecolor{currentstroke}%
\pgfsetdash{}{0pt}%
\pgfpathmoveto{\pgfqpoint{1.874588in}{1.973222in}}%
\pgfpathlineto{\pgfqpoint{2.096810in}{1.973222in}}%
\pgfpathlineto{\pgfqpoint{2.096810in}{2.051000in}}%
\pgfpathlineto{\pgfqpoint{1.874588in}{2.051000in}}%
\pgfpathclose%
\pgfusepath{stroke}%
\end{pgfscope}%
\begin{pgfscope}%
\definecolor{textcolor}{rgb}{0.000000,0.000000,0.000000}%
\pgfsetstrokecolor{textcolor}%
\pgfsetfillcolor{textcolor}%
\pgftext[x=2.185699in,y=1.973222in,left,base]{\color{textcolor}\rmfamily\fontsize{8.000000}{9.600000}\selectfont Unweighted}%
\end{pgfscope}%
\begin{pgfscope}%
\pgfsetbuttcap%
\pgfsetmiterjoin%
\pgfsetlinewidth{1.003750pt}%
\definecolor{currentstroke}{rgb}{0.949020,0.372549,0.360784}%
\pgfsetstrokecolor{currentstroke}%
\pgfsetdash{{1.000000pt}{1.650000pt}}{0.000000pt}%
\pgfpathmoveto{\pgfqpoint{1.874588in}{1.817111in}}%
\pgfpathlineto{\pgfqpoint{2.096810in}{1.817111in}}%
\pgfpathlineto{\pgfqpoint{2.096810in}{1.894889in}}%
\pgfpathlineto{\pgfqpoint{1.874588in}{1.894889in}}%
\pgfpathclose%
\pgfusepath{stroke}%
\end{pgfscope}%
\begin{pgfscope}%
\definecolor{textcolor}{rgb}{0.000000,0.000000,0.000000}%
\pgfsetstrokecolor{textcolor}%
\pgfsetfillcolor{textcolor}%
\pgftext[x=2.185699in,y=1.817111in,left,base]{\color{textcolor}\rmfamily\fontsize{8.000000}{9.600000}\selectfont Weighted}%
\end{pgfscope}%
\end{pgfpicture}%
\makeatother%
\endgroup%

         \caption{}\label{fig:weighted_distribution}
     \end{subfigure}
     \hfill
     \begin{subfigure}[b]{0.49\textwidth}
         \centering
         %% Creator: Matplotlib, PGF backend
%%
%% To include the figure in your LaTeX document, write
%%   \input{<filename>.pgf}
%%
%% Make sure the required packages are loaded in your preamble
%%   \usepackage{pgf}
%%
%% and, on pdftex
%%   \usepackage[utf8]{inputenc}\DeclareUnicodeCharacter{2212}{-}
%%
%% or, on luatex and xetex
%%   \usepackage{unicode-math}
%%
%% Figures using additional raster images can only be included by \input if
%% they are in the same directory as the main LaTeX file. For loading figures
%% from other directories you can use the `import` package
%%   \usepackage{import}
%%
%% and then include the figures with
%%   \import{<path to file>}{<filename>.pgf}
%%
%% Matplotlib used the following preamble
%%   \usepackage{siunitx} \usepackage{amsmath} \usepackage{bm}
%%   \usepackage{fontspec}
%%
\begingroup%
\makeatletter%
\begin{pgfpicture}%
\pgfpathrectangle{\pgfpointorigin}{\pgfqpoint{3.038588in}{2.500000in}}%
\pgfusepath{use as bounding box, clip}%
\begin{pgfscope}%
\pgfsetbuttcap%
\pgfsetmiterjoin%
\definecolor{currentfill}{rgb}{1.000000,1.000000,1.000000}%
\pgfsetfillcolor{currentfill}%
\pgfsetlinewidth{0.000000pt}%
\definecolor{currentstroke}{rgb}{1.000000,1.000000,1.000000}%
\pgfsetstrokecolor{currentstroke}%
\pgfsetdash{}{0pt}%
\pgfpathmoveto{\pgfqpoint{0.000000in}{0.000000in}}%
\pgfpathlineto{\pgfqpoint{3.038588in}{0.000000in}}%
\pgfpathlineto{\pgfqpoint{3.038588in}{2.500000in}}%
\pgfpathlineto{\pgfqpoint{0.000000in}{2.500000in}}%
\pgfpathclose%
\pgfusepath{fill}%
\end{pgfscope}%
\begin{pgfscope}%
\pgfsetbuttcap%
\pgfsetmiterjoin%
\definecolor{currentfill}{rgb}{1.000000,1.000000,1.000000}%
\pgfsetfillcolor{currentfill}%
\pgfsetlinewidth{0.000000pt}%
\definecolor{currentstroke}{rgb}{0.000000,0.000000,0.000000}%
\pgfsetstrokecolor{currentstroke}%
\pgfsetstrokeopacity{0.000000}%
\pgfsetdash{}{0pt}%
\pgfpathmoveto{\pgfqpoint{0.572918in}{0.553781in}}%
\pgfpathlineto{\pgfqpoint{2.857238in}{0.553781in}}%
\pgfpathlineto{\pgfqpoint{2.857238in}{2.149333in}}%
\pgfpathlineto{\pgfqpoint{0.572918in}{2.149333in}}%
\pgfpathclose%
\pgfusepath{fill}%
\end{pgfscope}%
\begin{pgfscope}%
\pgfpathrectangle{\pgfqpoint{0.572918in}{0.553781in}}{\pgfqpoint{2.284320in}{1.595553in}}%
\pgfusepath{clip}%
\pgfsetbuttcap%
\pgfsetroundjoin%
\pgfsetlinewidth{0.501875pt}%
\definecolor{currentstroke}{rgb}{0.690196,0.690196,0.690196}%
\pgfsetstrokecolor{currentstroke}%
\pgfsetstrokeopacity{0.500000}%
\pgfsetdash{{0.500000pt}{0.825000pt}}{0.000000pt}%
\pgfpathmoveto{\pgfqpoint{0.615673in}{0.553781in}}%
\pgfpathlineto{\pgfqpoint{0.615673in}{2.149333in}}%
\pgfusepath{stroke}%
\end{pgfscope}%
\begin{pgfscope}%
\pgfsetbuttcap%
\pgfsetroundjoin%
\definecolor{currentfill}{rgb}{0.000000,0.000000,0.000000}%
\pgfsetfillcolor{currentfill}%
\pgfsetlinewidth{0.803000pt}%
\definecolor{currentstroke}{rgb}{0.000000,0.000000,0.000000}%
\pgfsetstrokecolor{currentstroke}%
\pgfsetdash{}{0pt}%
\pgfsys@defobject{currentmarker}{\pgfqpoint{0.000000in}{-0.048611in}}{\pgfqpoint{0.000000in}{0.000000in}}{%
\pgfpathmoveto{\pgfqpoint{0.000000in}{0.000000in}}%
\pgfpathlineto{\pgfqpoint{0.000000in}{-0.048611in}}%
\pgfusepath{stroke,fill}%
}%
\begin{pgfscope}%
\pgfsys@transformshift{0.615673in}{0.553781in}%
\pgfsys@useobject{currentmarker}{}%
\end{pgfscope}%
\end{pgfscope}%
\begin{pgfscope}%
\definecolor{textcolor}{rgb}{0.000000,0.000000,0.000000}%
\pgfsetstrokecolor{textcolor}%
\pgfsetfillcolor{textcolor}%
\pgftext[x=0.615673in,y=0.456558in,,top]{\color{textcolor}\rmfamily\fontsize{8.000000}{9.600000}\selectfont \(\displaystyle {0}\)}%
\end{pgfscope}%
\begin{pgfscope}%
\pgfpathrectangle{\pgfqpoint{0.572918in}{0.553781in}}{\pgfqpoint{2.284320in}{1.595553in}}%
\pgfusepath{clip}%
\pgfsetbuttcap%
\pgfsetroundjoin%
\pgfsetlinewidth{0.501875pt}%
\definecolor{currentstroke}{rgb}{0.690196,0.690196,0.690196}%
\pgfsetstrokecolor{currentstroke}%
\pgfsetstrokeopacity{0.500000}%
\pgfsetdash{{0.500000pt}{0.825000pt}}{0.000000pt}%
\pgfpathmoveto{\pgfqpoint{1.348610in}{0.553781in}}%
\pgfpathlineto{\pgfqpoint{1.348610in}{2.149333in}}%
\pgfusepath{stroke}%
\end{pgfscope}%
\begin{pgfscope}%
\pgfsetbuttcap%
\pgfsetroundjoin%
\definecolor{currentfill}{rgb}{0.000000,0.000000,0.000000}%
\pgfsetfillcolor{currentfill}%
\pgfsetlinewidth{0.803000pt}%
\definecolor{currentstroke}{rgb}{0.000000,0.000000,0.000000}%
\pgfsetstrokecolor{currentstroke}%
\pgfsetdash{}{0pt}%
\pgfsys@defobject{currentmarker}{\pgfqpoint{0.000000in}{-0.048611in}}{\pgfqpoint{0.000000in}{0.000000in}}{%
\pgfpathmoveto{\pgfqpoint{0.000000in}{0.000000in}}%
\pgfpathlineto{\pgfqpoint{0.000000in}{-0.048611in}}%
\pgfusepath{stroke,fill}%
}%
\begin{pgfscope}%
\pgfsys@transformshift{1.348610in}{0.553781in}%
\pgfsys@useobject{currentmarker}{}%
\end{pgfscope}%
\end{pgfscope}%
\begin{pgfscope}%
\definecolor{textcolor}{rgb}{0.000000,0.000000,0.000000}%
\pgfsetstrokecolor{textcolor}%
\pgfsetfillcolor{textcolor}%
\pgftext[x=1.348610in,y=0.456558in,,top]{\color{textcolor}\rmfamily\fontsize{8.000000}{9.600000}\selectfont \(\displaystyle {1}\)}%
\end{pgfscope}%
\begin{pgfscope}%
\pgfpathrectangle{\pgfqpoint{0.572918in}{0.553781in}}{\pgfqpoint{2.284320in}{1.595553in}}%
\pgfusepath{clip}%
\pgfsetbuttcap%
\pgfsetroundjoin%
\pgfsetlinewidth{0.501875pt}%
\definecolor{currentstroke}{rgb}{0.690196,0.690196,0.690196}%
\pgfsetstrokecolor{currentstroke}%
\pgfsetstrokeopacity{0.500000}%
\pgfsetdash{{0.500000pt}{0.825000pt}}{0.000000pt}%
\pgfpathmoveto{\pgfqpoint{2.081547in}{0.553781in}}%
\pgfpathlineto{\pgfqpoint{2.081547in}{2.149333in}}%
\pgfusepath{stroke}%
\end{pgfscope}%
\begin{pgfscope}%
\pgfsetbuttcap%
\pgfsetroundjoin%
\definecolor{currentfill}{rgb}{0.000000,0.000000,0.000000}%
\pgfsetfillcolor{currentfill}%
\pgfsetlinewidth{0.803000pt}%
\definecolor{currentstroke}{rgb}{0.000000,0.000000,0.000000}%
\pgfsetstrokecolor{currentstroke}%
\pgfsetdash{}{0pt}%
\pgfsys@defobject{currentmarker}{\pgfqpoint{0.000000in}{-0.048611in}}{\pgfqpoint{0.000000in}{0.000000in}}{%
\pgfpathmoveto{\pgfqpoint{0.000000in}{0.000000in}}%
\pgfpathlineto{\pgfqpoint{0.000000in}{-0.048611in}}%
\pgfusepath{stroke,fill}%
}%
\begin{pgfscope}%
\pgfsys@transformshift{2.081547in}{0.553781in}%
\pgfsys@useobject{currentmarker}{}%
\end{pgfscope}%
\end{pgfscope}%
\begin{pgfscope}%
\definecolor{textcolor}{rgb}{0.000000,0.000000,0.000000}%
\pgfsetstrokecolor{textcolor}%
\pgfsetfillcolor{textcolor}%
\pgftext[x=2.081547in,y=0.456558in,,top]{\color{textcolor}\rmfamily\fontsize{8.000000}{9.600000}\selectfont \(\displaystyle {2}\)}%
\end{pgfscope}%
\begin{pgfscope}%
\pgfpathrectangle{\pgfqpoint{0.572918in}{0.553781in}}{\pgfqpoint{2.284320in}{1.595553in}}%
\pgfusepath{clip}%
\pgfsetbuttcap%
\pgfsetroundjoin%
\pgfsetlinewidth{0.501875pt}%
\definecolor{currentstroke}{rgb}{0.690196,0.690196,0.690196}%
\pgfsetstrokecolor{currentstroke}%
\pgfsetstrokeopacity{0.500000}%
\pgfsetdash{{0.500000pt}{0.825000pt}}{0.000000pt}%
\pgfpathmoveto{\pgfqpoint{2.814484in}{0.553781in}}%
\pgfpathlineto{\pgfqpoint{2.814484in}{2.149333in}}%
\pgfusepath{stroke}%
\end{pgfscope}%
\begin{pgfscope}%
\pgfsetbuttcap%
\pgfsetroundjoin%
\definecolor{currentfill}{rgb}{0.000000,0.000000,0.000000}%
\pgfsetfillcolor{currentfill}%
\pgfsetlinewidth{0.803000pt}%
\definecolor{currentstroke}{rgb}{0.000000,0.000000,0.000000}%
\pgfsetstrokecolor{currentstroke}%
\pgfsetdash{}{0pt}%
\pgfsys@defobject{currentmarker}{\pgfqpoint{0.000000in}{-0.048611in}}{\pgfqpoint{0.000000in}{0.000000in}}{%
\pgfpathmoveto{\pgfqpoint{0.000000in}{0.000000in}}%
\pgfpathlineto{\pgfqpoint{0.000000in}{-0.048611in}}%
\pgfusepath{stroke,fill}%
}%
\begin{pgfscope}%
\pgfsys@transformshift{2.814484in}{0.553781in}%
\pgfsys@useobject{currentmarker}{}%
\end{pgfscope}%
\end{pgfscope}%
\begin{pgfscope}%
\definecolor{textcolor}{rgb}{0.000000,0.000000,0.000000}%
\pgfsetstrokecolor{textcolor}%
\pgfsetfillcolor{textcolor}%
\pgftext[x=2.814484in,y=0.456558in,,top]{\color{textcolor}\rmfamily\fontsize{8.000000}{9.600000}\selectfont \(\displaystyle {3}\)}%
\end{pgfscope}%
\begin{pgfscope}%
\definecolor{textcolor}{rgb}{0.000000,0.000000,0.000000}%
\pgfsetstrokecolor{textcolor}%
\pgfsetfillcolor{textcolor}%
\pgftext[x=1.715078in,y=0.302336in,,top]{\color{textcolor}\rmfamily\fontsize{10.950000}{13.140000}\selectfont \(\displaystyle \log_{10}(E_{\textup{true}}) \, \left[ E / \textup{GeV} \right]\)}%
\end{pgfscope}%
\begin{pgfscope}%
\pgfpathrectangle{\pgfqpoint{0.572918in}{0.553781in}}{\pgfqpoint{2.284320in}{1.595553in}}%
\pgfusepath{clip}%
\pgfsetbuttcap%
\pgfsetroundjoin%
\pgfsetlinewidth{0.501875pt}%
\definecolor{currentstroke}{rgb}{0.690196,0.690196,0.690196}%
\pgfsetstrokecolor{currentstroke}%
\pgfsetstrokeopacity{0.500000}%
\pgfsetdash{{0.500000pt}{0.825000pt}}{0.000000pt}%
\pgfpathmoveto{\pgfqpoint{0.572918in}{0.850021in}}%
\pgfpathlineto{\pgfqpoint{2.857238in}{0.850021in}}%
\pgfusepath{stroke}%
\end{pgfscope}%
\begin{pgfscope}%
\pgfsetbuttcap%
\pgfsetroundjoin%
\definecolor{currentfill}{rgb}{0.000000,0.000000,0.000000}%
\pgfsetfillcolor{currentfill}%
\pgfsetlinewidth{0.803000pt}%
\definecolor{currentstroke}{rgb}{0.000000,0.000000,0.000000}%
\pgfsetstrokecolor{currentstroke}%
\pgfsetdash{}{0pt}%
\pgfsys@defobject{currentmarker}{\pgfqpoint{-0.048611in}{0.000000in}}{\pgfqpoint{-0.000000in}{0.000000in}}{%
\pgfpathmoveto{\pgfqpoint{-0.000000in}{0.000000in}}%
\pgfpathlineto{\pgfqpoint{-0.048611in}{0.000000in}}%
\pgfusepath{stroke,fill}%
}%
\begin{pgfscope}%
\pgfsys@transformshift{0.572918in}{0.850021in}%
\pgfsys@useobject{currentmarker}{}%
\end{pgfscope}%
\end{pgfscope}%
\begin{pgfscope}%
\definecolor{textcolor}{rgb}{0.000000,0.000000,0.000000}%
\pgfsetstrokecolor{textcolor}%
\pgfsetfillcolor{textcolor}%
\pgftext[x=0.357639in, y=0.811466in, left, base]{\color{textcolor}\rmfamily\fontsize{8.000000}{9.600000}\selectfont \(\displaystyle {15}\)}%
\end{pgfscope}%
\begin{pgfscope}%
\pgfpathrectangle{\pgfqpoint{0.572918in}{0.553781in}}{\pgfqpoint{2.284320in}{1.595553in}}%
\pgfusepath{clip}%
\pgfsetbuttcap%
\pgfsetroundjoin%
\pgfsetlinewidth{0.501875pt}%
\definecolor{currentstroke}{rgb}{0.690196,0.690196,0.690196}%
\pgfsetstrokecolor{currentstroke}%
\pgfsetstrokeopacity{0.500000}%
\pgfsetdash{{0.500000pt}{0.825000pt}}{0.000000pt}%
\pgfpathmoveto{\pgfqpoint{0.572918in}{1.225479in}}%
\pgfpathlineto{\pgfqpoint{2.857238in}{1.225479in}}%
\pgfusepath{stroke}%
\end{pgfscope}%
\begin{pgfscope}%
\pgfsetbuttcap%
\pgfsetroundjoin%
\definecolor{currentfill}{rgb}{0.000000,0.000000,0.000000}%
\pgfsetfillcolor{currentfill}%
\pgfsetlinewidth{0.803000pt}%
\definecolor{currentstroke}{rgb}{0.000000,0.000000,0.000000}%
\pgfsetstrokecolor{currentstroke}%
\pgfsetdash{}{0pt}%
\pgfsys@defobject{currentmarker}{\pgfqpoint{-0.048611in}{0.000000in}}{\pgfqpoint{-0.000000in}{0.000000in}}{%
\pgfpathmoveto{\pgfqpoint{-0.000000in}{0.000000in}}%
\pgfpathlineto{\pgfqpoint{-0.048611in}{0.000000in}}%
\pgfusepath{stroke,fill}%
}%
\begin{pgfscope}%
\pgfsys@transformshift{0.572918in}{1.225479in}%
\pgfsys@useobject{currentmarker}{}%
\end{pgfscope}%
\end{pgfscope}%
\begin{pgfscope}%
\definecolor{textcolor}{rgb}{0.000000,0.000000,0.000000}%
\pgfsetstrokecolor{textcolor}%
\pgfsetfillcolor{textcolor}%
\pgftext[x=0.357639in, y=1.186923in, left, base]{\color{textcolor}\rmfamily\fontsize{8.000000}{9.600000}\selectfont \(\displaystyle {20}\)}%
\end{pgfscope}%
\begin{pgfscope}%
\pgfpathrectangle{\pgfqpoint{0.572918in}{0.553781in}}{\pgfqpoint{2.284320in}{1.595553in}}%
\pgfusepath{clip}%
\pgfsetbuttcap%
\pgfsetroundjoin%
\pgfsetlinewidth{0.501875pt}%
\definecolor{currentstroke}{rgb}{0.690196,0.690196,0.690196}%
\pgfsetstrokecolor{currentstroke}%
\pgfsetstrokeopacity{0.500000}%
\pgfsetdash{{0.500000pt}{0.825000pt}}{0.000000pt}%
\pgfpathmoveto{\pgfqpoint{0.572918in}{1.600937in}}%
\pgfpathlineto{\pgfqpoint{2.857238in}{1.600937in}}%
\pgfusepath{stroke}%
\end{pgfscope}%
\begin{pgfscope}%
\pgfsetbuttcap%
\pgfsetroundjoin%
\definecolor{currentfill}{rgb}{0.000000,0.000000,0.000000}%
\pgfsetfillcolor{currentfill}%
\pgfsetlinewidth{0.803000pt}%
\definecolor{currentstroke}{rgb}{0.000000,0.000000,0.000000}%
\pgfsetstrokecolor{currentstroke}%
\pgfsetdash{}{0pt}%
\pgfsys@defobject{currentmarker}{\pgfqpoint{-0.048611in}{0.000000in}}{\pgfqpoint{-0.000000in}{0.000000in}}{%
\pgfpathmoveto{\pgfqpoint{-0.000000in}{0.000000in}}%
\pgfpathlineto{\pgfqpoint{-0.048611in}{0.000000in}}%
\pgfusepath{stroke,fill}%
}%
\begin{pgfscope}%
\pgfsys@transformshift{0.572918in}{1.600937in}%
\pgfsys@useobject{currentmarker}{}%
\end{pgfscope}%
\end{pgfscope}%
\begin{pgfscope}%
\definecolor{textcolor}{rgb}{0.000000,0.000000,0.000000}%
\pgfsetstrokecolor{textcolor}%
\pgfsetfillcolor{textcolor}%
\pgftext[x=0.357639in, y=1.562381in, left, base]{\color{textcolor}\rmfamily\fontsize{8.000000}{9.600000}\selectfont \(\displaystyle {25}\)}%
\end{pgfscope}%
\begin{pgfscope}%
\pgfpathrectangle{\pgfqpoint{0.572918in}{0.553781in}}{\pgfqpoint{2.284320in}{1.595553in}}%
\pgfusepath{clip}%
\pgfsetbuttcap%
\pgfsetroundjoin%
\pgfsetlinewidth{0.501875pt}%
\definecolor{currentstroke}{rgb}{0.690196,0.690196,0.690196}%
\pgfsetstrokecolor{currentstroke}%
\pgfsetstrokeopacity{0.500000}%
\pgfsetdash{{0.500000pt}{0.825000pt}}{0.000000pt}%
\pgfpathmoveto{\pgfqpoint{0.572918in}{1.976394in}}%
\pgfpathlineto{\pgfqpoint{2.857238in}{1.976394in}}%
\pgfusepath{stroke}%
\end{pgfscope}%
\begin{pgfscope}%
\pgfsetbuttcap%
\pgfsetroundjoin%
\definecolor{currentfill}{rgb}{0.000000,0.000000,0.000000}%
\pgfsetfillcolor{currentfill}%
\pgfsetlinewidth{0.803000pt}%
\definecolor{currentstroke}{rgb}{0.000000,0.000000,0.000000}%
\pgfsetstrokecolor{currentstroke}%
\pgfsetdash{}{0pt}%
\pgfsys@defobject{currentmarker}{\pgfqpoint{-0.048611in}{0.000000in}}{\pgfqpoint{-0.000000in}{0.000000in}}{%
\pgfpathmoveto{\pgfqpoint{-0.000000in}{0.000000in}}%
\pgfpathlineto{\pgfqpoint{-0.048611in}{0.000000in}}%
\pgfusepath{stroke,fill}%
}%
\begin{pgfscope}%
\pgfsys@transformshift{0.572918in}{1.976394in}%
\pgfsys@useobject{currentmarker}{}%
\end{pgfscope}%
\end{pgfscope}%
\begin{pgfscope}%
\definecolor{textcolor}{rgb}{0.000000,0.000000,0.000000}%
\pgfsetstrokecolor{textcolor}%
\pgfsetfillcolor{textcolor}%
\pgftext[x=0.357639in, y=1.937839in, left, base]{\color{textcolor}\rmfamily\fontsize{8.000000}{9.600000}\selectfont \(\displaystyle {30}\)}%
\end{pgfscope}%
\begin{pgfscope}%
\definecolor{textcolor}{rgb}{0.000000,0.000000,0.000000}%
\pgfsetstrokecolor{textcolor}%
\pgfsetfillcolor{textcolor}%
\pgftext[x=0.302083in,y=1.351557in,,bottom,rotate=90.000000]{\color{textcolor}\rmfamily\fontsize{10.950000}{13.140000}\selectfont IQR / 1.349 \(\displaystyle \left[ \textup{deg} \right]\)}%
\end{pgfscope}%
\begin{pgfscope}%
\pgfpathrectangle{\pgfqpoint{0.572918in}{0.553781in}}{\pgfqpoint{2.284320in}{1.595553in}}%
\pgfusepath{clip}%
\pgfsetbuttcap%
\pgfsetroundjoin%
\pgfsetlinewidth{1.505625pt}%
\definecolor{currentstroke}{rgb}{0.313725,0.317647,0.309804}%
\pgfsetstrokecolor{currentstroke}%
\pgfsetstrokeopacity{0.900000}%
\pgfsetdash{}{0pt}%
\pgfpathmoveto{\pgfqpoint{0.676751in}{1.616799in}}%
\pgfpathlineto{\pgfqpoint{0.676751in}{2.076808in}}%
\pgfusepath{stroke}%
\end{pgfscope}%
\begin{pgfscope}%
\pgfpathrectangle{\pgfqpoint{0.572918in}{0.553781in}}{\pgfqpoint{2.284320in}{1.595553in}}%
\pgfusepath{clip}%
\pgfsetbuttcap%
\pgfsetroundjoin%
\pgfsetlinewidth{1.505625pt}%
\definecolor{currentstroke}{rgb}{0.313725,0.317647,0.309804}%
\pgfsetstrokecolor{currentstroke}%
\pgfsetstrokeopacity{0.900000}%
\pgfsetdash{}{0pt}%
\pgfpathmoveto{\pgfqpoint{0.798907in}{1.724935in}}%
\pgfpathlineto{\pgfqpoint{0.798907in}{1.940323in}}%
\pgfusepath{stroke}%
\end{pgfscope}%
\begin{pgfscope}%
\pgfpathrectangle{\pgfqpoint{0.572918in}{0.553781in}}{\pgfqpoint{2.284320in}{1.595553in}}%
\pgfusepath{clip}%
\pgfsetbuttcap%
\pgfsetroundjoin%
\pgfsetlinewidth{1.505625pt}%
\definecolor{currentstroke}{rgb}{0.313725,0.317647,0.309804}%
\pgfsetstrokecolor{currentstroke}%
\pgfsetstrokeopacity{0.900000}%
\pgfsetdash{}{0pt}%
\pgfpathmoveto{\pgfqpoint{0.921063in}{1.617610in}}%
\pgfpathlineto{\pgfqpoint{0.921063in}{1.743657in}}%
\pgfusepath{stroke}%
\end{pgfscope}%
\begin{pgfscope}%
\pgfpathrectangle{\pgfqpoint{0.572918in}{0.553781in}}{\pgfqpoint{2.284320in}{1.595553in}}%
\pgfusepath{clip}%
\pgfsetbuttcap%
\pgfsetroundjoin%
\pgfsetlinewidth{1.505625pt}%
\definecolor{currentstroke}{rgb}{0.313725,0.317647,0.309804}%
\pgfsetstrokecolor{currentstroke}%
\pgfsetstrokeopacity{0.900000}%
\pgfsetdash{}{0pt}%
\pgfpathmoveto{\pgfqpoint{1.043220in}{1.711033in}}%
\pgfpathlineto{\pgfqpoint{1.043220in}{1.800703in}}%
\pgfusepath{stroke}%
\end{pgfscope}%
\begin{pgfscope}%
\pgfpathrectangle{\pgfqpoint{0.572918in}{0.553781in}}{\pgfqpoint{2.284320in}{1.595553in}}%
\pgfusepath{clip}%
\pgfsetbuttcap%
\pgfsetroundjoin%
\pgfsetlinewidth{1.505625pt}%
\definecolor{currentstroke}{rgb}{0.313725,0.317647,0.309804}%
\pgfsetstrokecolor{currentstroke}%
\pgfsetstrokeopacity{0.900000}%
\pgfsetdash{}{0pt}%
\pgfpathmoveto{\pgfqpoint{1.165376in}{1.754637in}}%
\pgfpathlineto{\pgfqpoint{1.165376in}{1.835703in}}%
\pgfusepath{stroke}%
\end{pgfscope}%
\begin{pgfscope}%
\pgfpathrectangle{\pgfqpoint{0.572918in}{0.553781in}}{\pgfqpoint{2.284320in}{1.595553in}}%
\pgfusepath{clip}%
\pgfsetbuttcap%
\pgfsetroundjoin%
\pgfsetlinewidth{1.505625pt}%
\definecolor{currentstroke}{rgb}{0.313725,0.317647,0.309804}%
\pgfsetstrokecolor{currentstroke}%
\pgfsetstrokeopacity{0.900000}%
\pgfsetdash{}{0pt}%
\pgfpathmoveto{\pgfqpoint{1.287532in}{1.725490in}}%
\pgfpathlineto{\pgfqpoint{1.287532in}{1.780662in}}%
\pgfusepath{stroke}%
\end{pgfscope}%
\begin{pgfscope}%
\pgfpathrectangle{\pgfqpoint{0.572918in}{0.553781in}}{\pgfqpoint{2.284320in}{1.595553in}}%
\pgfusepath{clip}%
\pgfsetbuttcap%
\pgfsetroundjoin%
\pgfsetlinewidth{1.505625pt}%
\definecolor{currentstroke}{rgb}{0.313725,0.317647,0.309804}%
\pgfsetstrokecolor{currentstroke}%
\pgfsetstrokeopacity{0.900000}%
\pgfsetdash{}{0pt}%
\pgfpathmoveto{\pgfqpoint{1.409688in}{1.644147in}}%
\pgfpathlineto{\pgfqpoint{1.409688in}{1.695716in}}%
\pgfusepath{stroke}%
\end{pgfscope}%
\begin{pgfscope}%
\pgfpathrectangle{\pgfqpoint{0.572918in}{0.553781in}}{\pgfqpoint{2.284320in}{1.595553in}}%
\pgfusepath{clip}%
\pgfsetbuttcap%
\pgfsetroundjoin%
\pgfsetlinewidth{1.505625pt}%
\definecolor{currentstroke}{rgb}{0.313725,0.317647,0.309804}%
\pgfsetstrokecolor{currentstroke}%
\pgfsetstrokeopacity{0.900000}%
\pgfsetdash{}{0pt}%
\pgfpathmoveto{\pgfqpoint{1.531844in}{1.579060in}}%
\pgfpathlineto{\pgfqpoint{1.531844in}{1.631678in}}%
\pgfusepath{stroke}%
\end{pgfscope}%
\begin{pgfscope}%
\pgfpathrectangle{\pgfqpoint{0.572918in}{0.553781in}}{\pgfqpoint{2.284320in}{1.595553in}}%
\pgfusepath{clip}%
\pgfsetbuttcap%
\pgfsetroundjoin%
\pgfsetlinewidth{1.505625pt}%
\definecolor{currentstroke}{rgb}{0.313725,0.317647,0.309804}%
\pgfsetstrokecolor{currentstroke}%
\pgfsetstrokeopacity{0.900000}%
\pgfsetdash{}{0pt}%
\pgfpathmoveto{\pgfqpoint{1.654000in}{1.427271in}}%
\pgfpathlineto{\pgfqpoint{1.654000in}{1.473512in}}%
\pgfusepath{stroke}%
\end{pgfscope}%
\begin{pgfscope}%
\pgfpathrectangle{\pgfqpoint{0.572918in}{0.553781in}}{\pgfqpoint{2.284320in}{1.595553in}}%
\pgfusepath{clip}%
\pgfsetbuttcap%
\pgfsetroundjoin%
\pgfsetlinewidth{1.505625pt}%
\definecolor{currentstroke}{rgb}{0.313725,0.317647,0.309804}%
\pgfsetstrokecolor{currentstroke}%
\pgfsetstrokeopacity{0.900000}%
\pgfsetdash{}{0pt}%
\pgfpathmoveto{\pgfqpoint{1.776156in}{1.324813in}}%
\pgfpathlineto{\pgfqpoint{1.776156in}{1.371594in}}%
\pgfusepath{stroke}%
\end{pgfscope}%
\begin{pgfscope}%
\pgfpathrectangle{\pgfqpoint{0.572918in}{0.553781in}}{\pgfqpoint{2.284320in}{1.595553in}}%
\pgfusepath{clip}%
\pgfsetbuttcap%
\pgfsetroundjoin%
\pgfsetlinewidth{1.505625pt}%
\definecolor{currentstroke}{rgb}{0.313725,0.317647,0.309804}%
\pgfsetstrokecolor{currentstroke}%
\pgfsetstrokeopacity{0.900000}%
\pgfsetdash{}{0pt}%
\pgfpathmoveto{\pgfqpoint{1.898313in}{1.239259in}}%
\pgfpathlineto{\pgfqpoint{1.898313in}{1.290063in}}%
\pgfusepath{stroke}%
\end{pgfscope}%
\begin{pgfscope}%
\pgfpathrectangle{\pgfqpoint{0.572918in}{0.553781in}}{\pgfqpoint{2.284320in}{1.595553in}}%
\pgfusepath{clip}%
\pgfsetbuttcap%
\pgfsetroundjoin%
\pgfsetlinewidth{1.505625pt}%
\definecolor{currentstroke}{rgb}{0.313725,0.317647,0.309804}%
\pgfsetstrokecolor{currentstroke}%
\pgfsetstrokeopacity{0.900000}%
\pgfsetdash{}{0pt}%
\pgfpathmoveto{\pgfqpoint{2.020469in}{1.158760in}}%
\pgfpathlineto{\pgfqpoint{2.020469in}{1.216663in}}%
\pgfusepath{stroke}%
\end{pgfscope}%
\begin{pgfscope}%
\pgfpathrectangle{\pgfqpoint{0.572918in}{0.553781in}}{\pgfqpoint{2.284320in}{1.595553in}}%
\pgfusepath{clip}%
\pgfsetbuttcap%
\pgfsetroundjoin%
\pgfsetlinewidth{1.505625pt}%
\definecolor{currentstroke}{rgb}{0.313725,0.317647,0.309804}%
\pgfsetstrokecolor{currentstroke}%
\pgfsetstrokeopacity{0.900000}%
\pgfsetdash{}{0pt}%
\pgfpathmoveto{\pgfqpoint{2.142625in}{1.052714in}}%
\pgfpathlineto{\pgfqpoint{2.142625in}{1.108101in}}%
\pgfusepath{stroke}%
\end{pgfscope}%
\begin{pgfscope}%
\pgfpathrectangle{\pgfqpoint{0.572918in}{0.553781in}}{\pgfqpoint{2.284320in}{1.595553in}}%
\pgfusepath{clip}%
\pgfsetbuttcap%
\pgfsetroundjoin%
\pgfsetlinewidth{1.505625pt}%
\definecolor{currentstroke}{rgb}{0.313725,0.317647,0.309804}%
\pgfsetstrokecolor{currentstroke}%
\pgfsetstrokeopacity{0.900000}%
\pgfsetdash{}{0pt}%
\pgfpathmoveto{\pgfqpoint{2.264781in}{0.965507in}}%
\pgfpathlineto{\pgfqpoint{2.264781in}{1.035051in}}%
\pgfusepath{stroke}%
\end{pgfscope}%
\begin{pgfscope}%
\pgfpathrectangle{\pgfqpoint{0.572918in}{0.553781in}}{\pgfqpoint{2.284320in}{1.595553in}}%
\pgfusepath{clip}%
\pgfsetbuttcap%
\pgfsetroundjoin%
\pgfsetlinewidth{1.505625pt}%
\definecolor{currentstroke}{rgb}{0.313725,0.317647,0.309804}%
\pgfsetstrokecolor{currentstroke}%
\pgfsetstrokeopacity{0.900000}%
\pgfsetdash{}{0pt}%
\pgfpathmoveto{\pgfqpoint{2.386937in}{0.885101in}}%
\pgfpathlineto{\pgfqpoint{2.386937in}{0.962103in}}%
\pgfusepath{stroke}%
\end{pgfscope}%
\begin{pgfscope}%
\pgfpathrectangle{\pgfqpoint{0.572918in}{0.553781in}}{\pgfqpoint{2.284320in}{1.595553in}}%
\pgfusepath{clip}%
\pgfsetbuttcap%
\pgfsetroundjoin%
\pgfsetlinewidth{1.505625pt}%
\definecolor{currentstroke}{rgb}{0.313725,0.317647,0.309804}%
\pgfsetstrokecolor{currentstroke}%
\pgfsetstrokeopacity{0.900000}%
\pgfsetdash{}{0pt}%
\pgfpathmoveto{\pgfqpoint{2.509093in}{0.921260in}}%
\pgfpathlineto{\pgfqpoint{2.509093in}{1.025799in}}%
\pgfusepath{stroke}%
\end{pgfscope}%
\begin{pgfscope}%
\pgfpathrectangle{\pgfqpoint{0.572918in}{0.553781in}}{\pgfqpoint{2.284320in}{1.595553in}}%
\pgfusepath{clip}%
\pgfsetbuttcap%
\pgfsetroundjoin%
\pgfsetlinewidth{1.505625pt}%
\definecolor{currentstroke}{rgb}{0.313725,0.317647,0.309804}%
\pgfsetstrokecolor{currentstroke}%
\pgfsetstrokeopacity{0.900000}%
\pgfsetdash{}{0pt}%
\pgfpathmoveto{\pgfqpoint{2.631249in}{0.854516in}}%
\pgfpathlineto{\pgfqpoint{2.631249in}{0.976566in}}%
\pgfusepath{stroke}%
\end{pgfscope}%
\begin{pgfscope}%
\pgfpathrectangle{\pgfqpoint{0.572918in}{0.553781in}}{\pgfqpoint{2.284320in}{1.595553in}}%
\pgfusepath{clip}%
\pgfsetbuttcap%
\pgfsetroundjoin%
\pgfsetlinewidth{1.505625pt}%
\definecolor{currentstroke}{rgb}{0.313725,0.317647,0.309804}%
\pgfsetstrokecolor{currentstroke}%
\pgfsetstrokeopacity{0.900000}%
\pgfsetdash{}{0pt}%
\pgfpathmoveto{\pgfqpoint{2.753406in}{0.960084in}}%
\pgfpathlineto{\pgfqpoint{2.753406in}{1.129302in}}%
\pgfusepath{stroke}%
\end{pgfscope}%
\begin{pgfscope}%
\pgfpathrectangle{\pgfqpoint{0.572918in}{0.553781in}}{\pgfqpoint{2.284320in}{1.595553in}}%
\pgfusepath{clip}%
\pgfsetbuttcap%
\pgfsetroundjoin%
\pgfsetlinewidth{1.505625pt}%
\definecolor{currentstroke}{rgb}{0.949020,0.372549,0.360784}%
\pgfsetstrokecolor{currentstroke}%
\pgfsetstrokeopacity{0.900000}%
\pgfsetdash{}{0pt}%
\pgfpathmoveto{\pgfqpoint{0.676751in}{1.222027in}}%
\pgfpathlineto{\pgfqpoint{0.676751in}{1.601249in}}%
\pgfusepath{stroke}%
\end{pgfscope}%
\begin{pgfscope}%
\pgfpathrectangle{\pgfqpoint{0.572918in}{0.553781in}}{\pgfqpoint{2.284320in}{1.595553in}}%
\pgfusepath{clip}%
\pgfsetbuttcap%
\pgfsetroundjoin%
\pgfsetlinewidth{1.505625pt}%
\definecolor{currentstroke}{rgb}{0.949020,0.372549,0.360784}%
\pgfsetstrokecolor{currentstroke}%
\pgfsetstrokeopacity{0.900000}%
\pgfsetdash{}{0pt}%
\pgfpathmoveto{\pgfqpoint{0.798907in}{1.464649in}}%
\pgfpathlineto{\pgfqpoint{0.798907in}{1.642828in}}%
\pgfusepath{stroke}%
\end{pgfscope}%
\begin{pgfscope}%
\pgfpathrectangle{\pgfqpoint{0.572918in}{0.553781in}}{\pgfqpoint{2.284320in}{1.595553in}}%
\pgfusepath{clip}%
\pgfsetbuttcap%
\pgfsetroundjoin%
\pgfsetlinewidth{1.505625pt}%
\definecolor{currentstroke}{rgb}{0.949020,0.372549,0.360784}%
\pgfsetstrokecolor{currentstroke}%
\pgfsetstrokeopacity{0.900000}%
\pgfsetdash{}{0pt}%
\pgfpathmoveto{\pgfqpoint{0.921063in}{1.641057in}}%
\pgfpathlineto{\pgfqpoint{0.921063in}{1.770490in}}%
\pgfusepath{stroke}%
\end{pgfscope}%
\begin{pgfscope}%
\pgfpathrectangle{\pgfqpoint{0.572918in}{0.553781in}}{\pgfqpoint{2.284320in}{1.595553in}}%
\pgfusepath{clip}%
\pgfsetbuttcap%
\pgfsetroundjoin%
\pgfsetlinewidth{1.505625pt}%
\definecolor{currentstroke}{rgb}{0.949020,0.372549,0.360784}%
\pgfsetstrokecolor{currentstroke}%
\pgfsetstrokeopacity{0.900000}%
\pgfsetdash{}{0pt}%
\pgfpathmoveto{\pgfqpoint{1.043220in}{1.690597in}}%
\pgfpathlineto{\pgfqpoint{1.043220in}{1.792942in}}%
\pgfusepath{stroke}%
\end{pgfscope}%
\begin{pgfscope}%
\pgfpathrectangle{\pgfqpoint{0.572918in}{0.553781in}}{\pgfqpoint{2.284320in}{1.595553in}}%
\pgfusepath{clip}%
\pgfsetbuttcap%
\pgfsetroundjoin%
\pgfsetlinewidth{1.505625pt}%
\definecolor{currentstroke}{rgb}{0.949020,0.372549,0.360784}%
\pgfsetstrokecolor{currentstroke}%
\pgfsetstrokeopacity{0.900000}%
\pgfsetdash{}{0pt}%
\pgfpathmoveto{\pgfqpoint{1.165376in}{1.817406in}}%
\pgfpathlineto{\pgfqpoint{1.165376in}{1.885963in}}%
\pgfusepath{stroke}%
\end{pgfscope}%
\begin{pgfscope}%
\pgfpathrectangle{\pgfqpoint{0.572918in}{0.553781in}}{\pgfqpoint{2.284320in}{1.595553in}}%
\pgfusepath{clip}%
\pgfsetbuttcap%
\pgfsetroundjoin%
\pgfsetlinewidth{1.505625pt}%
\definecolor{currentstroke}{rgb}{0.949020,0.372549,0.360784}%
\pgfsetstrokecolor{currentstroke}%
\pgfsetstrokeopacity{0.900000}%
\pgfsetdash{}{0pt}%
\pgfpathmoveto{\pgfqpoint{1.287532in}{1.834100in}}%
\pgfpathlineto{\pgfqpoint{1.287532in}{1.893427in}}%
\pgfusepath{stroke}%
\end{pgfscope}%
\begin{pgfscope}%
\pgfpathrectangle{\pgfqpoint{0.572918in}{0.553781in}}{\pgfqpoint{2.284320in}{1.595553in}}%
\pgfusepath{clip}%
\pgfsetbuttcap%
\pgfsetroundjoin%
\pgfsetlinewidth{1.505625pt}%
\definecolor{currentstroke}{rgb}{0.949020,0.372549,0.360784}%
\pgfsetstrokecolor{currentstroke}%
\pgfsetstrokeopacity{0.900000}%
\pgfsetdash{}{0pt}%
\pgfpathmoveto{\pgfqpoint{1.409688in}{1.734241in}}%
\pgfpathlineto{\pgfqpoint{1.409688in}{1.788576in}}%
\pgfusepath{stroke}%
\end{pgfscope}%
\begin{pgfscope}%
\pgfpathrectangle{\pgfqpoint{0.572918in}{0.553781in}}{\pgfqpoint{2.284320in}{1.595553in}}%
\pgfusepath{clip}%
\pgfsetbuttcap%
\pgfsetroundjoin%
\pgfsetlinewidth{1.505625pt}%
\definecolor{currentstroke}{rgb}{0.949020,0.372549,0.360784}%
\pgfsetstrokecolor{currentstroke}%
\pgfsetstrokeopacity{0.900000}%
\pgfsetdash{}{0pt}%
\pgfpathmoveto{\pgfqpoint{1.531844in}{1.678007in}}%
\pgfpathlineto{\pgfqpoint{1.531844in}{1.734244in}}%
\pgfusepath{stroke}%
\end{pgfscope}%
\begin{pgfscope}%
\pgfpathrectangle{\pgfqpoint{0.572918in}{0.553781in}}{\pgfqpoint{2.284320in}{1.595553in}}%
\pgfusepath{clip}%
\pgfsetbuttcap%
\pgfsetroundjoin%
\pgfsetlinewidth{1.505625pt}%
\definecolor{currentstroke}{rgb}{0.949020,0.372549,0.360784}%
\pgfsetstrokecolor{currentstroke}%
\pgfsetstrokeopacity{0.900000}%
\pgfsetdash{}{0pt}%
\pgfpathmoveto{\pgfqpoint{1.654000in}{1.540229in}}%
\pgfpathlineto{\pgfqpoint{1.654000in}{1.596922in}}%
\pgfusepath{stroke}%
\end{pgfscope}%
\begin{pgfscope}%
\pgfpathrectangle{\pgfqpoint{0.572918in}{0.553781in}}{\pgfqpoint{2.284320in}{1.595553in}}%
\pgfusepath{clip}%
\pgfsetbuttcap%
\pgfsetroundjoin%
\pgfsetlinewidth{1.505625pt}%
\definecolor{currentstroke}{rgb}{0.949020,0.372549,0.360784}%
\pgfsetstrokecolor{currentstroke}%
\pgfsetstrokeopacity{0.900000}%
\pgfsetdash{}{0pt}%
\pgfpathmoveto{\pgfqpoint{1.776156in}{1.419618in}}%
\pgfpathlineto{\pgfqpoint{1.776156in}{1.468669in}}%
\pgfusepath{stroke}%
\end{pgfscope}%
\begin{pgfscope}%
\pgfpathrectangle{\pgfqpoint{0.572918in}{0.553781in}}{\pgfqpoint{2.284320in}{1.595553in}}%
\pgfusepath{clip}%
\pgfsetbuttcap%
\pgfsetroundjoin%
\pgfsetlinewidth{1.505625pt}%
\definecolor{currentstroke}{rgb}{0.949020,0.372549,0.360784}%
\pgfsetstrokecolor{currentstroke}%
\pgfsetstrokeopacity{0.900000}%
\pgfsetdash{}{0pt}%
\pgfpathmoveto{\pgfqpoint{1.898313in}{1.287824in}}%
\pgfpathlineto{\pgfqpoint{1.898313in}{1.344514in}}%
\pgfusepath{stroke}%
\end{pgfscope}%
\begin{pgfscope}%
\pgfpathrectangle{\pgfqpoint{0.572918in}{0.553781in}}{\pgfqpoint{2.284320in}{1.595553in}}%
\pgfusepath{clip}%
\pgfsetbuttcap%
\pgfsetroundjoin%
\pgfsetlinewidth{1.505625pt}%
\definecolor{currentstroke}{rgb}{0.949020,0.372549,0.360784}%
\pgfsetstrokecolor{currentstroke}%
\pgfsetstrokeopacity{0.900000}%
\pgfsetdash{}{0pt}%
\pgfpathmoveto{\pgfqpoint{2.020469in}{1.221152in}}%
\pgfpathlineto{\pgfqpoint{2.020469in}{1.281900in}}%
\pgfusepath{stroke}%
\end{pgfscope}%
\begin{pgfscope}%
\pgfpathrectangle{\pgfqpoint{0.572918in}{0.553781in}}{\pgfqpoint{2.284320in}{1.595553in}}%
\pgfusepath{clip}%
\pgfsetbuttcap%
\pgfsetroundjoin%
\pgfsetlinewidth{1.505625pt}%
\definecolor{currentstroke}{rgb}{0.949020,0.372549,0.360784}%
\pgfsetstrokecolor{currentstroke}%
\pgfsetstrokeopacity{0.900000}%
\pgfsetdash{}{0pt}%
\pgfpathmoveto{\pgfqpoint{2.142625in}{1.093822in}}%
\pgfpathlineto{\pgfqpoint{2.142625in}{1.158288in}}%
\pgfusepath{stroke}%
\end{pgfscope}%
\begin{pgfscope}%
\pgfpathrectangle{\pgfqpoint{0.572918in}{0.553781in}}{\pgfqpoint{2.284320in}{1.595553in}}%
\pgfusepath{clip}%
\pgfsetbuttcap%
\pgfsetroundjoin%
\pgfsetlinewidth{1.505625pt}%
\definecolor{currentstroke}{rgb}{0.949020,0.372549,0.360784}%
\pgfsetstrokecolor{currentstroke}%
\pgfsetstrokeopacity{0.900000}%
\pgfsetdash{}{0pt}%
\pgfpathmoveto{\pgfqpoint{2.264781in}{0.950068in}}%
\pgfpathlineto{\pgfqpoint{2.264781in}{1.013998in}}%
\pgfusepath{stroke}%
\end{pgfscope}%
\begin{pgfscope}%
\pgfpathrectangle{\pgfqpoint{0.572918in}{0.553781in}}{\pgfqpoint{2.284320in}{1.595553in}}%
\pgfusepath{clip}%
\pgfsetbuttcap%
\pgfsetroundjoin%
\pgfsetlinewidth{1.505625pt}%
\definecolor{currentstroke}{rgb}{0.949020,0.372549,0.360784}%
\pgfsetstrokecolor{currentstroke}%
\pgfsetstrokeopacity{0.900000}%
\pgfsetdash{}{0pt}%
\pgfpathmoveto{\pgfqpoint{2.386937in}{0.825329in}}%
\pgfpathlineto{\pgfqpoint{2.386937in}{0.901882in}}%
\pgfusepath{stroke}%
\end{pgfscope}%
\begin{pgfscope}%
\pgfpathrectangle{\pgfqpoint{0.572918in}{0.553781in}}{\pgfqpoint{2.284320in}{1.595553in}}%
\pgfusepath{clip}%
\pgfsetbuttcap%
\pgfsetroundjoin%
\pgfsetlinewidth{1.505625pt}%
\definecolor{currentstroke}{rgb}{0.949020,0.372549,0.360784}%
\pgfsetstrokecolor{currentstroke}%
\pgfsetstrokeopacity{0.900000}%
\pgfsetdash{}{0pt}%
\pgfpathmoveto{\pgfqpoint{2.509093in}{0.786051in}}%
\pgfpathlineto{\pgfqpoint{2.509093in}{0.883016in}}%
\pgfusepath{stroke}%
\end{pgfscope}%
\begin{pgfscope}%
\pgfpathrectangle{\pgfqpoint{0.572918in}{0.553781in}}{\pgfqpoint{2.284320in}{1.595553in}}%
\pgfusepath{clip}%
\pgfsetbuttcap%
\pgfsetroundjoin%
\pgfsetlinewidth{1.505625pt}%
\definecolor{currentstroke}{rgb}{0.949020,0.372549,0.360784}%
\pgfsetstrokecolor{currentstroke}%
\pgfsetstrokeopacity{0.900000}%
\pgfsetdash{}{0pt}%
\pgfpathmoveto{\pgfqpoint{2.631249in}{0.644190in}}%
\pgfpathlineto{\pgfqpoint{2.631249in}{0.738656in}}%
\pgfusepath{stroke}%
\end{pgfscope}%
\begin{pgfscope}%
\pgfpathrectangle{\pgfqpoint{0.572918in}{0.553781in}}{\pgfqpoint{2.284320in}{1.595553in}}%
\pgfusepath{clip}%
\pgfsetbuttcap%
\pgfsetroundjoin%
\pgfsetlinewidth{1.505625pt}%
\definecolor{currentstroke}{rgb}{0.949020,0.372549,0.360784}%
\pgfsetstrokecolor{currentstroke}%
\pgfsetstrokeopacity{0.900000}%
\pgfsetdash{}{0pt}%
\pgfpathmoveto{\pgfqpoint{2.753406in}{0.626306in}}%
\pgfpathlineto{\pgfqpoint{2.753406in}{0.754489in}}%
\pgfusepath{stroke}%
\end{pgfscope}%
\begin{pgfscope}%
\pgfpathrectangle{\pgfqpoint{0.572918in}{0.553781in}}{\pgfqpoint{2.284320in}{1.595553in}}%
\pgfusepath{clip}%
\pgfsetbuttcap%
\pgfsetroundjoin%
\definecolor{currentfill}{rgb}{0.313725,0.317647,0.309804}%
\pgfsetfillcolor{currentfill}%
\pgfsetfillopacity{0.900000}%
\pgfsetlinewidth{1.003750pt}%
\definecolor{currentstroke}{rgb}{0.313725,0.317647,0.309804}%
\pgfsetstrokecolor{currentstroke}%
\pgfsetstrokeopacity{0.900000}%
\pgfsetdash{}{0pt}%
\pgfsys@defobject{currentmarker}{\pgfqpoint{-0.013889in}{-0.000000in}}{\pgfqpoint{0.013889in}{0.000000in}}{%
\pgfpathmoveto{\pgfqpoint{0.013889in}{-0.000000in}}%
\pgfpathlineto{\pgfqpoint{-0.013889in}{0.000000in}}%
\pgfusepath{stroke,fill}%
}%
\begin{pgfscope}%
\pgfsys@transformshift{0.676751in}{1.616799in}%
\pgfsys@useobject{currentmarker}{}%
\end{pgfscope}%
\begin{pgfscope}%
\pgfsys@transformshift{0.798907in}{1.724935in}%
\pgfsys@useobject{currentmarker}{}%
\end{pgfscope}%
\begin{pgfscope}%
\pgfsys@transformshift{0.921063in}{1.617610in}%
\pgfsys@useobject{currentmarker}{}%
\end{pgfscope}%
\begin{pgfscope}%
\pgfsys@transformshift{1.043220in}{1.711033in}%
\pgfsys@useobject{currentmarker}{}%
\end{pgfscope}%
\begin{pgfscope}%
\pgfsys@transformshift{1.165376in}{1.754637in}%
\pgfsys@useobject{currentmarker}{}%
\end{pgfscope}%
\begin{pgfscope}%
\pgfsys@transformshift{1.287532in}{1.725490in}%
\pgfsys@useobject{currentmarker}{}%
\end{pgfscope}%
\begin{pgfscope}%
\pgfsys@transformshift{1.409688in}{1.644147in}%
\pgfsys@useobject{currentmarker}{}%
\end{pgfscope}%
\begin{pgfscope}%
\pgfsys@transformshift{1.531844in}{1.579060in}%
\pgfsys@useobject{currentmarker}{}%
\end{pgfscope}%
\begin{pgfscope}%
\pgfsys@transformshift{1.654000in}{1.427271in}%
\pgfsys@useobject{currentmarker}{}%
\end{pgfscope}%
\begin{pgfscope}%
\pgfsys@transformshift{1.776156in}{1.324813in}%
\pgfsys@useobject{currentmarker}{}%
\end{pgfscope}%
\begin{pgfscope}%
\pgfsys@transformshift{1.898313in}{1.239259in}%
\pgfsys@useobject{currentmarker}{}%
\end{pgfscope}%
\begin{pgfscope}%
\pgfsys@transformshift{2.020469in}{1.158760in}%
\pgfsys@useobject{currentmarker}{}%
\end{pgfscope}%
\begin{pgfscope}%
\pgfsys@transformshift{2.142625in}{1.052714in}%
\pgfsys@useobject{currentmarker}{}%
\end{pgfscope}%
\begin{pgfscope}%
\pgfsys@transformshift{2.264781in}{0.965507in}%
\pgfsys@useobject{currentmarker}{}%
\end{pgfscope}%
\begin{pgfscope}%
\pgfsys@transformshift{2.386937in}{0.885101in}%
\pgfsys@useobject{currentmarker}{}%
\end{pgfscope}%
\begin{pgfscope}%
\pgfsys@transformshift{2.509093in}{0.921260in}%
\pgfsys@useobject{currentmarker}{}%
\end{pgfscope}%
\begin{pgfscope}%
\pgfsys@transformshift{2.631249in}{0.854516in}%
\pgfsys@useobject{currentmarker}{}%
\end{pgfscope}%
\begin{pgfscope}%
\pgfsys@transformshift{2.753406in}{0.960084in}%
\pgfsys@useobject{currentmarker}{}%
\end{pgfscope}%
\end{pgfscope}%
\begin{pgfscope}%
\pgfpathrectangle{\pgfqpoint{0.572918in}{0.553781in}}{\pgfqpoint{2.284320in}{1.595553in}}%
\pgfusepath{clip}%
\pgfsetbuttcap%
\pgfsetroundjoin%
\definecolor{currentfill}{rgb}{0.313725,0.317647,0.309804}%
\pgfsetfillcolor{currentfill}%
\pgfsetfillopacity{0.900000}%
\pgfsetlinewidth{1.003750pt}%
\definecolor{currentstroke}{rgb}{0.313725,0.317647,0.309804}%
\pgfsetstrokecolor{currentstroke}%
\pgfsetstrokeopacity{0.900000}%
\pgfsetdash{}{0pt}%
\pgfsys@defobject{currentmarker}{\pgfqpoint{-0.013889in}{-0.000000in}}{\pgfqpoint{0.013889in}{0.000000in}}{%
\pgfpathmoveto{\pgfqpoint{0.013889in}{-0.000000in}}%
\pgfpathlineto{\pgfqpoint{-0.013889in}{0.000000in}}%
\pgfusepath{stroke,fill}%
}%
\begin{pgfscope}%
\pgfsys@transformshift{0.676751in}{2.076808in}%
\pgfsys@useobject{currentmarker}{}%
\end{pgfscope}%
\begin{pgfscope}%
\pgfsys@transformshift{0.798907in}{1.940323in}%
\pgfsys@useobject{currentmarker}{}%
\end{pgfscope}%
\begin{pgfscope}%
\pgfsys@transformshift{0.921063in}{1.743657in}%
\pgfsys@useobject{currentmarker}{}%
\end{pgfscope}%
\begin{pgfscope}%
\pgfsys@transformshift{1.043220in}{1.800703in}%
\pgfsys@useobject{currentmarker}{}%
\end{pgfscope}%
\begin{pgfscope}%
\pgfsys@transformshift{1.165376in}{1.835703in}%
\pgfsys@useobject{currentmarker}{}%
\end{pgfscope}%
\begin{pgfscope}%
\pgfsys@transformshift{1.287532in}{1.780662in}%
\pgfsys@useobject{currentmarker}{}%
\end{pgfscope}%
\begin{pgfscope}%
\pgfsys@transformshift{1.409688in}{1.695716in}%
\pgfsys@useobject{currentmarker}{}%
\end{pgfscope}%
\begin{pgfscope}%
\pgfsys@transformshift{1.531844in}{1.631678in}%
\pgfsys@useobject{currentmarker}{}%
\end{pgfscope}%
\begin{pgfscope}%
\pgfsys@transformshift{1.654000in}{1.473512in}%
\pgfsys@useobject{currentmarker}{}%
\end{pgfscope}%
\begin{pgfscope}%
\pgfsys@transformshift{1.776156in}{1.371594in}%
\pgfsys@useobject{currentmarker}{}%
\end{pgfscope}%
\begin{pgfscope}%
\pgfsys@transformshift{1.898313in}{1.290063in}%
\pgfsys@useobject{currentmarker}{}%
\end{pgfscope}%
\begin{pgfscope}%
\pgfsys@transformshift{2.020469in}{1.216663in}%
\pgfsys@useobject{currentmarker}{}%
\end{pgfscope}%
\begin{pgfscope}%
\pgfsys@transformshift{2.142625in}{1.108101in}%
\pgfsys@useobject{currentmarker}{}%
\end{pgfscope}%
\begin{pgfscope}%
\pgfsys@transformshift{2.264781in}{1.035051in}%
\pgfsys@useobject{currentmarker}{}%
\end{pgfscope}%
\begin{pgfscope}%
\pgfsys@transformshift{2.386937in}{0.962103in}%
\pgfsys@useobject{currentmarker}{}%
\end{pgfscope}%
\begin{pgfscope}%
\pgfsys@transformshift{2.509093in}{1.025799in}%
\pgfsys@useobject{currentmarker}{}%
\end{pgfscope}%
\begin{pgfscope}%
\pgfsys@transformshift{2.631249in}{0.976566in}%
\pgfsys@useobject{currentmarker}{}%
\end{pgfscope}%
\begin{pgfscope}%
\pgfsys@transformshift{2.753406in}{1.129302in}%
\pgfsys@useobject{currentmarker}{}%
\end{pgfscope}%
\end{pgfscope}%
\begin{pgfscope}%
\pgfpathrectangle{\pgfqpoint{0.572918in}{0.553781in}}{\pgfqpoint{2.284320in}{1.595553in}}%
\pgfusepath{clip}%
\pgfsetbuttcap%
\pgfsetroundjoin%
\definecolor{currentfill}{rgb}{0.949020,0.372549,0.360784}%
\pgfsetfillcolor{currentfill}%
\pgfsetfillopacity{0.900000}%
\pgfsetlinewidth{1.003750pt}%
\definecolor{currentstroke}{rgb}{0.949020,0.372549,0.360784}%
\pgfsetstrokecolor{currentstroke}%
\pgfsetstrokeopacity{0.900000}%
\pgfsetdash{}{0pt}%
\pgfsys@defobject{currentmarker}{\pgfqpoint{-0.013889in}{-0.000000in}}{\pgfqpoint{0.013889in}{0.000000in}}{%
\pgfpathmoveto{\pgfqpoint{0.013889in}{-0.000000in}}%
\pgfpathlineto{\pgfqpoint{-0.013889in}{0.000000in}}%
\pgfusepath{stroke,fill}%
}%
\begin{pgfscope}%
\pgfsys@transformshift{0.676751in}{1.222027in}%
\pgfsys@useobject{currentmarker}{}%
\end{pgfscope}%
\begin{pgfscope}%
\pgfsys@transformshift{0.798907in}{1.464649in}%
\pgfsys@useobject{currentmarker}{}%
\end{pgfscope}%
\begin{pgfscope}%
\pgfsys@transformshift{0.921063in}{1.641057in}%
\pgfsys@useobject{currentmarker}{}%
\end{pgfscope}%
\begin{pgfscope}%
\pgfsys@transformshift{1.043220in}{1.690597in}%
\pgfsys@useobject{currentmarker}{}%
\end{pgfscope}%
\begin{pgfscope}%
\pgfsys@transformshift{1.165376in}{1.817406in}%
\pgfsys@useobject{currentmarker}{}%
\end{pgfscope}%
\begin{pgfscope}%
\pgfsys@transformshift{1.287532in}{1.834100in}%
\pgfsys@useobject{currentmarker}{}%
\end{pgfscope}%
\begin{pgfscope}%
\pgfsys@transformshift{1.409688in}{1.734241in}%
\pgfsys@useobject{currentmarker}{}%
\end{pgfscope}%
\begin{pgfscope}%
\pgfsys@transformshift{1.531844in}{1.678007in}%
\pgfsys@useobject{currentmarker}{}%
\end{pgfscope}%
\begin{pgfscope}%
\pgfsys@transformshift{1.654000in}{1.540229in}%
\pgfsys@useobject{currentmarker}{}%
\end{pgfscope}%
\begin{pgfscope}%
\pgfsys@transformshift{1.776156in}{1.419618in}%
\pgfsys@useobject{currentmarker}{}%
\end{pgfscope}%
\begin{pgfscope}%
\pgfsys@transformshift{1.898313in}{1.287824in}%
\pgfsys@useobject{currentmarker}{}%
\end{pgfscope}%
\begin{pgfscope}%
\pgfsys@transformshift{2.020469in}{1.221152in}%
\pgfsys@useobject{currentmarker}{}%
\end{pgfscope}%
\begin{pgfscope}%
\pgfsys@transformshift{2.142625in}{1.093822in}%
\pgfsys@useobject{currentmarker}{}%
\end{pgfscope}%
\begin{pgfscope}%
\pgfsys@transformshift{2.264781in}{0.950068in}%
\pgfsys@useobject{currentmarker}{}%
\end{pgfscope}%
\begin{pgfscope}%
\pgfsys@transformshift{2.386937in}{0.825329in}%
\pgfsys@useobject{currentmarker}{}%
\end{pgfscope}%
\begin{pgfscope}%
\pgfsys@transformshift{2.509093in}{0.786051in}%
\pgfsys@useobject{currentmarker}{}%
\end{pgfscope}%
\begin{pgfscope}%
\pgfsys@transformshift{2.631249in}{0.644190in}%
\pgfsys@useobject{currentmarker}{}%
\end{pgfscope}%
\begin{pgfscope}%
\pgfsys@transformshift{2.753406in}{0.626306in}%
\pgfsys@useobject{currentmarker}{}%
\end{pgfscope}%
\end{pgfscope}%
\begin{pgfscope}%
\pgfpathrectangle{\pgfqpoint{0.572918in}{0.553781in}}{\pgfqpoint{2.284320in}{1.595553in}}%
\pgfusepath{clip}%
\pgfsetbuttcap%
\pgfsetroundjoin%
\definecolor{currentfill}{rgb}{0.949020,0.372549,0.360784}%
\pgfsetfillcolor{currentfill}%
\pgfsetfillopacity{0.900000}%
\pgfsetlinewidth{1.003750pt}%
\definecolor{currentstroke}{rgb}{0.949020,0.372549,0.360784}%
\pgfsetstrokecolor{currentstroke}%
\pgfsetstrokeopacity{0.900000}%
\pgfsetdash{}{0pt}%
\pgfsys@defobject{currentmarker}{\pgfqpoint{-0.013889in}{-0.000000in}}{\pgfqpoint{0.013889in}{0.000000in}}{%
\pgfpathmoveto{\pgfqpoint{0.013889in}{-0.000000in}}%
\pgfpathlineto{\pgfqpoint{-0.013889in}{0.000000in}}%
\pgfusepath{stroke,fill}%
}%
\begin{pgfscope}%
\pgfsys@transformshift{0.676751in}{1.601249in}%
\pgfsys@useobject{currentmarker}{}%
\end{pgfscope}%
\begin{pgfscope}%
\pgfsys@transformshift{0.798907in}{1.642828in}%
\pgfsys@useobject{currentmarker}{}%
\end{pgfscope}%
\begin{pgfscope}%
\pgfsys@transformshift{0.921063in}{1.770490in}%
\pgfsys@useobject{currentmarker}{}%
\end{pgfscope}%
\begin{pgfscope}%
\pgfsys@transformshift{1.043220in}{1.792942in}%
\pgfsys@useobject{currentmarker}{}%
\end{pgfscope}%
\begin{pgfscope}%
\pgfsys@transformshift{1.165376in}{1.885963in}%
\pgfsys@useobject{currentmarker}{}%
\end{pgfscope}%
\begin{pgfscope}%
\pgfsys@transformshift{1.287532in}{1.893427in}%
\pgfsys@useobject{currentmarker}{}%
\end{pgfscope}%
\begin{pgfscope}%
\pgfsys@transformshift{1.409688in}{1.788576in}%
\pgfsys@useobject{currentmarker}{}%
\end{pgfscope}%
\begin{pgfscope}%
\pgfsys@transformshift{1.531844in}{1.734244in}%
\pgfsys@useobject{currentmarker}{}%
\end{pgfscope}%
\begin{pgfscope}%
\pgfsys@transformshift{1.654000in}{1.596922in}%
\pgfsys@useobject{currentmarker}{}%
\end{pgfscope}%
\begin{pgfscope}%
\pgfsys@transformshift{1.776156in}{1.468669in}%
\pgfsys@useobject{currentmarker}{}%
\end{pgfscope}%
\begin{pgfscope}%
\pgfsys@transformshift{1.898313in}{1.344514in}%
\pgfsys@useobject{currentmarker}{}%
\end{pgfscope}%
\begin{pgfscope}%
\pgfsys@transformshift{2.020469in}{1.281900in}%
\pgfsys@useobject{currentmarker}{}%
\end{pgfscope}%
\begin{pgfscope}%
\pgfsys@transformshift{2.142625in}{1.158288in}%
\pgfsys@useobject{currentmarker}{}%
\end{pgfscope}%
\begin{pgfscope}%
\pgfsys@transformshift{2.264781in}{1.013998in}%
\pgfsys@useobject{currentmarker}{}%
\end{pgfscope}%
\begin{pgfscope}%
\pgfsys@transformshift{2.386937in}{0.901882in}%
\pgfsys@useobject{currentmarker}{}%
\end{pgfscope}%
\begin{pgfscope}%
\pgfsys@transformshift{2.509093in}{0.883016in}%
\pgfsys@useobject{currentmarker}{}%
\end{pgfscope}%
\begin{pgfscope}%
\pgfsys@transformshift{2.631249in}{0.738656in}%
\pgfsys@useobject{currentmarker}{}%
\end{pgfscope}%
\begin{pgfscope}%
\pgfsys@transformshift{2.753406in}{0.754489in}%
\pgfsys@useobject{currentmarker}{}%
\end{pgfscope}%
\end{pgfscope}%
\begin{pgfscope}%
\pgfpathrectangle{\pgfqpoint{0.572918in}{0.553781in}}{\pgfqpoint{2.284320in}{1.595553in}}%
\pgfusepath{clip}%
\pgfsetrectcap%
\pgfsetroundjoin%
\pgfsetlinewidth{1.505625pt}%
\definecolor{currentstroke}{rgb}{0.313725,0.317647,0.309804}%
\pgfsetstrokecolor{currentstroke}%
\pgfsetstrokeopacity{0.900000}%
\pgfsetdash{}{0pt}%
\pgfpathmoveto{\pgfqpoint{0.676751in}{1.852487in}}%
\pgfpathlineto{\pgfqpoint{0.798907in}{1.815838in}}%
\pgfpathlineto{\pgfqpoint{0.921063in}{1.680504in}}%
\pgfpathlineto{\pgfqpoint{1.043220in}{1.751764in}}%
\pgfpathlineto{\pgfqpoint{1.165376in}{1.789683in}}%
\pgfpathlineto{\pgfqpoint{1.287532in}{1.751397in}}%
\pgfpathlineto{\pgfqpoint{1.409688in}{1.671349in}}%
\pgfpathlineto{\pgfqpoint{1.531844in}{1.604920in}}%
\pgfpathlineto{\pgfqpoint{1.654000in}{1.452508in}}%
\pgfpathlineto{\pgfqpoint{1.776156in}{1.350093in}}%
\pgfpathlineto{\pgfqpoint{1.898313in}{1.265898in}}%
\pgfpathlineto{\pgfqpoint{2.020469in}{1.185787in}}%
\pgfpathlineto{\pgfqpoint{2.142625in}{1.080259in}}%
\pgfpathlineto{\pgfqpoint{2.264781in}{1.002922in}}%
\pgfpathlineto{\pgfqpoint{2.386937in}{0.928342in}}%
\pgfpathlineto{\pgfqpoint{2.509093in}{0.968176in}}%
\pgfpathlineto{\pgfqpoint{2.631249in}{0.911175in}}%
\pgfpathlineto{\pgfqpoint{2.753406in}{1.045448in}}%
\pgfusepath{stroke}%
\end{pgfscope}%
\begin{pgfscope}%
\pgfpathrectangle{\pgfqpoint{0.572918in}{0.553781in}}{\pgfqpoint{2.284320in}{1.595553in}}%
\pgfusepath{clip}%
\pgfsetbuttcap%
\pgfsetroundjoin%
\pgfsetlinewidth{1.505625pt}%
\definecolor{currentstroke}{rgb}{0.949020,0.372549,0.360784}%
\pgfsetstrokecolor{currentstroke}%
\pgfsetstrokeopacity{0.900000}%
\pgfsetdash{{1.500000pt}{2.475000pt}}{0.000000pt}%
\pgfpathmoveto{\pgfqpoint{0.676751in}{1.385317in}}%
\pgfpathlineto{\pgfqpoint{0.798907in}{1.550721in}}%
\pgfpathlineto{\pgfqpoint{0.921063in}{1.705232in}}%
\pgfpathlineto{\pgfqpoint{1.043220in}{1.740839in}}%
\pgfpathlineto{\pgfqpoint{1.165376in}{1.850166in}}%
\pgfpathlineto{\pgfqpoint{1.287532in}{1.862863in}}%
\pgfpathlineto{\pgfqpoint{1.409688in}{1.762536in}}%
\pgfpathlineto{\pgfqpoint{1.531844in}{1.709625in}}%
\pgfpathlineto{\pgfqpoint{1.654000in}{1.569089in}}%
\pgfpathlineto{\pgfqpoint{1.776156in}{1.444263in}}%
\pgfpathlineto{\pgfqpoint{1.898313in}{1.314200in}}%
\pgfpathlineto{\pgfqpoint{2.020469in}{1.253492in}}%
\pgfpathlineto{\pgfqpoint{2.142625in}{1.127449in}}%
\pgfpathlineto{\pgfqpoint{2.264781in}{0.980344in}}%
\pgfpathlineto{\pgfqpoint{2.386937in}{0.864249in}}%
\pgfpathlineto{\pgfqpoint{2.509093in}{0.834795in}}%
\pgfpathlineto{\pgfqpoint{2.631249in}{0.693355in}}%
\pgfpathlineto{\pgfqpoint{2.753406in}{0.693626in}}%
\pgfusepath{stroke}%
\end{pgfscope}%
\begin{pgfscope}%
\pgfsetrectcap%
\pgfsetmiterjoin%
\pgfsetlinewidth{0.803000pt}%
\definecolor{currentstroke}{rgb}{0.000000,0.000000,0.000000}%
\pgfsetstrokecolor{currentstroke}%
\pgfsetdash{}{0pt}%
\pgfpathmoveto{\pgfqpoint{0.572918in}{0.553781in}}%
\pgfpathlineto{\pgfqpoint{0.572918in}{2.149333in}}%
\pgfusepath{stroke}%
\end{pgfscope}%
\begin{pgfscope}%
\pgfsetrectcap%
\pgfsetmiterjoin%
\pgfsetlinewidth{0.803000pt}%
\definecolor{currentstroke}{rgb}{0.000000,0.000000,0.000000}%
\pgfsetstrokecolor{currentstroke}%
\pgfsetdash{}{0pt}%
\pgfpathmoveto{\pgfqpoint{2.857238in}{0.553781in}}%
\pgfpathlineto{\pgfqpoint{2.857238in}{2.149333in}}%
\pgfusepath{stroke}%
\end{pgfscope}%
\begin{pgfscope}%
\pgfsetrectcap%
\pgfsetmiterjoin%
\pgfsetlinewidth{0.803000pt}%
\definecolor{currentstroke}{rgb}{0.000000,0.000000,0.000000}%
\pgfsetstrokecolor{currentstroke}%
\pgfsetdash{}{0pt}%
\pgfpathmoveto{\pgfqpoint{0.572918in}{0.553781in}}%
\pgfpathlineto{\pgfqpoint{2.857238in}{0.553781in}}%
\pgfusepath{stroke}%
\end{pgfscope}%
\begin{pgfscope}%
\pgfsetrectcap%
\pgfsetmiterjoin%
\pgfsetlinewidth{0.803000pt}%
\definecolor{currentstroke}{rgb}{0.000000,0.000000,0.000000}%
\pgfsetstrokecolor{currentstroke}%
\pgfsetdash{}{0pt}%
\pgfpathmoveto{\pgfqpoint{0.572918in}{2.149333in}}%
\pgfpathlineto{\pgfqpoint{2.857238in}{2.149333in}}%
\pgfusepath{stroke}%
\end{pgfscope}%
\begin{pgfscope}%
\definecolor{textcolor}{rgb}{0.000000,0.000000,0.000000}%
\pgfsetstrokecolor{textcolor}%
\pgfsetfillcolor{textcolor}%
\pgftext[x=0.572918in,y=2.232667in,left,base]{\color{textcolor}\rmfamily\fontsize{12.000000}{14.400000}\selectfont Zenith performance, weighting}%
\end{pgfscope}%
\begin{pgfscope}%
\pgfsetbuttcap%
\pgfsetmiterjoin%
\definecolor{currentfill}{rgb}{1.000000,1.000000,1.000000}%
\pgfsetfillcolor{currentfill}%
\pgfsetfillopacity{0.800000}%
\pgfsetlinewidth{1.003750pt}%
\definecolor{currentstroke}{rgb}{0.800000,0.800000,0.800000}%
\pgfsetstrokecolor{currentstroke}%
\pgfsetstrokeopacity{0.800000}%
\pgfsetdash{}{0pt}%
\pgfpathmoveto{\pgfqpoint{1.821016in}{1.748222in}}%
\pgfpathlineto{\pgfqpoint{2.779461in}{1.748222in}}%
\pgfpathquadraticcurveto{\pgfqpoint{2.801683in}{1.748222in}}{\pgfqpoint{2.801683in}{1.770444in}}%
\pgfpathlineto{\pgfqpoint{2.801683in}{2.071556in}}%
\pgfpathquadraticcurveto{\pgfqpoint{2.801683in}{2.093778in}}{\pgfqpoint{2.779461in}{2.093778in}}%
\pgfpathlineto{\pgfqpoint{1.821016in}{2.093778in}}%
\pgfpathquadraticcurveto{\pgfqpoint{1.798794in}{2.093778in}}{\pgfqpoint{1.798794in}{2.071556in}}%
\pgfpathlineto{\pgfqpoint{1.798794in}{1.770444in}}%
\pgfpathquadraticcurveto{\pgfqpoint{1.798794in}{1.748222in}}{\pgfqpoint{1.821016in}{1.748222in}}%
\pgfpathclose%
\pgfusepath{stroke,fill}%
\end{pgfscope}%
\begin{pgfscope}%
\pgfsetbuttcap%
\pgfsetroundjoin%
\pgfsetlinewidth{1.505625pt}%
\definecolor{currentstroke}{rgb}{0.313725,0.317647,0.309804}%
\pgfsetstrokecolor{currentstroke}%
\pgfsetstrokeopacity{0.900000}%
\pgfsetdash{}{0pt}%
\pgfpathmoveto{\pgfqpoint{1.954350in}{1.954889in}}%
\pgfpathlineto{\pgfqpoint{1.954350in}{2.066000in}}%
\pgfusepath{stroke}%
\end{pgfscope}%
\begin{pgfscope}%
\pgfsetbuttcap%
\pgfsetroundjoin%
\definecolor{currentfill}{rgb}{0.313725,0.317647,0.309804}%
\pgfsetfillcolor{currentfill}%
\pgfsetfillopacity{0.900000}%
\pgfsetlinewidth{1.003750pt}%
\definecolor{currentstroke}{rgb}{0.313725,0.317647,0.309804}%
\pgfsetstrokecolor{currentstroke}%
\pgfsetstrokeopacity{0.900000}%
\pgfsetdash{}{0pt}%
\pgfsys@defobject{currentmarker}{\pgfqpoint{-0.013889in}{-0.000000in}}{\pgfqpoint{0.013889in}{0.000000in}}{%
\pgfpathmoveto{\pgfqpoint{0.013889in}{-0.000000in}}%
\pgfpathlineto{\pgfqpoint{-0.013889in}{0.000000in}}%
\pgfusepath{stroke,fill}%
}%
\begin{pgfscope}%
\pgfsys@transformshift{1.954350in}{1.954889in}%
\pgfsys@useobject{currentmarker}{}%
\end{pgfscope}%
\end{pgfscope}%
\begin{pgfscope}%
\pgfsetbuttcap%
\pgfsetroundjoin%
\definecolor{currentfill}{rgb}{0.313725,0.317647,0.309804}%
\pgfsetfillcolor{currentfill}%
\pgfsetfillopacity{0.900000}%
\pgfsetlinewidth{1.003750pt}%
\definecolor{currentstroke}{rgb}{0.313725,0.317647,0.309804}%
\pgfsetstrokecolor{currentstroke}%
\pgfsetstrokeopacity{0.900000}%
\pgfsetdash{}{0pt}%
\pgfsys@defobject{currentmarker}{\pgfqpoint{-0.013889in}{-0.000000in}}{\pgfqpoint{0.013889in}{0.000000in}}{%
\pgfpathmoveto{\pgfqpoint{0.013889in}{-0.000000in}}%
\pgfpathlineto{\pgfqpoint{-0.013889in}{0.000000in}}%
\pgfusepath{stroke,fill}%
}%
\begin{pgfscope}%
\pgfsys@transformshift{1.954350in}{2.066000in}%
\pgfsys@useobject{currentmarker}{}%
\end{pgfscope}%
\end{pgfscope}%
\begin{pgfscope}%
\pgfsetrectcap%
\pgfsetroundjoin%
\pgfsetlinewidth{1.505625pt}%
\definecolor{currentstroke}{rgb}{0.313725,0.317647,0.309804}%
\pgfsetstrokecolor{currentstroke}%
\pgfsetstrokeopacity{0.900000}%
\pgfsetdash{}{0pt}%
\pgfpathmoveto{\pgfqpoint{1.843238in}{2.010444in}}%
\pgfpathlineto{\pgfqpoint{2.065461in}{2.010444in}}%
\pgfusepath{stroke}%
\end{pgfscope}%
\begin{pgfscope}%
\definecolor{textcolor}{rgb}{0.000000,0.000000,0.000000}%
\pgfsetstrokecolor{textcolor}%
\pgfsetfillcolor{textcolor}%
\pgftext[x=2.154350in,y=1.971556in,left,base]{\color{textcolor}\rmfamily\fontsize{8.000000}{9.600000}\selectfont Unweighted}%
\end{pgfscope}%
\begin{pgfscope}%
\pgfsetbuttcap%
\pgfsetroundjoin%
\pgfsetlinewidth{1.505625pt}%
\definecolor{currentstroke}{rgb}{0.949020,0.372549,0.360784}%
\pgfsetstrokecolor{currentstroke}%
\pgfsetstrokeopacity{0.900000}%
\pgfsetdash{}{0pt}%
\pgfpathmoveto{\pgfqpoint{1.954350in}{1.798778in}}%
\pgfpathlineto{\pgfqpoint{1.954350in}{1.909889in}}%
\pgfusepath{stroke}%
\end{pgfscope}%
\begin{pgfscope}%
\pgfsetbuttcap%
\pgfsetroundjoin%
\definecolor{currentfill}{rgb}{0.949020,0.372549,0.360784}%
\pgfsetfillcolor{currentfill}%
\pgfsetfillopacity{0.900000}%
\pgfsetlinewidth{1.003750pt}%
\definecolor{currentstroke}{rgb}{0.949020,0.372549,0.360784}%
\pgfsetstrokecolor{currentstroke}%
\pgfsetstrokeopacity{0.900000}%
\pgfsetdash{}{0pt}%
\pgfsys@defobject{currentmarker}{\pgfqpoint{-0.013889in}{-0.000000in}}{\pgfqpoint{0.013889in}{0.000000in}}{%
\pgfpathmoveto{\pgfqpoint{0.013889in}{-0.000000in}}%
\pgfpathlineto{\pgfqpoint{-0.013889in}{0.000000in}}%
\pgfusepath{stroke,fill}%
}%
\begin{pgfscope}%
\pgfsys@transformshift{1.954350in}{1.798778in}%
\pgfsys@useobject{currentmarker}{}%
\end{pgfscope}%
\end{pgfscope}%
\begin{pgfscope}%
\pgfsetbuttcap%
\pgfsetroundjoin%
\definecolor{currentfill}{rgb}{0.949020,0.372549,0.360784}%
\pgfsetfillcolor{currentfill}%
\pgfsetfillopacity{0.900000}%
\pgfsetlinewidth{1.003750pt}%
\definecolor{currentstroke}{rgb}{0.949020,0.372549,0.360784}%
\pgfsetstrokecolor{currentstroke}%
\pgfsetstrokeopacity{0.900000}%
\pgfsetdash{}{0pt}%
\pgfsys@defobject{currentmarker}{\pgfqpoint{-0.013889in}{-0.000000in}}{\pgfqpoint{0.013889in}{0.000000in}}{%
\pgfpathmoveto{\pgfqpoint{0.013889in}{-0.000000in}}%
\pgfpathlineto{\pgfqpoint{-0.013889in}{0.000000in}}%
\pgfusepath{stroke,fill}%
}%
\begin{pgfscope}%
\pgfsys@transformshift{1.954350in}{1.909889in}%
\pgfsys@useobject{currentmarker}{}%
\end{pgfscope}%
\end{pgfscope}%
\begin{pgfscope}%
\pgfsetbuttcap%
\pgfsetroundjoin%
\pgfsetlinewidth{1.505625pt}%
\definecolor{currentstroke}{rgb}{0.949020,0.372549,0.360784}%
\pgfsetstrokecolor{currentstroke}%
\pgfsetstrokeopacity{0.900000}%
\pgfsetdash{{1.500000pt}{2.475000pt}}{0.000000pt}%
\pgfpathmoveto{\pgfqpoint{1.843238in}{1.854333in}}%
\pgfpathlineto{\pgfqpoint{2.065461in}{1.854333in}}%
\pgfusepath{stroke}%
\end{pgfscope}%
\begin{pgfscope}%
\definecolor{textcolor}{rgb}{0.000000,0.000000,0.000000}%
\pgfsetstrokecolor{textcolor}%
\pgfsetfillcolor{textcolor}%
\pgftext[x=2.154350in,y=1.815444in,left,base]{\color{textcolor}\rmfamily\fontsize{8.000000}{9.600000}\selectfont Weighted}%
\end{pgfscope}%
\end{pgfpicture}%
\makeatother%
\endgroup%

         \caption{}\label{fig:weighting_performance}
     \end{subfigure}
        \caption{\Vref{fig:weighted_distribution} shows the energy distribution of the largest DeepCore muon neutrino dataset.
        Superimposed is the weighted distribution, where each entry counts with the value of its weight instead of 1.
        The distribution has been weighted to a uniform distribution.
        \Vref{fig:weighting_performance} shows performance with/without re-weighting by energy bins.
    The tails naturally improve their performance, while the bulk performs worse.}\label{fig:weighting}
\end{figure}

It is common practice in both machine learning and high energy physics to re-weight a distribution.
In the case of machine learning, it is often done such that distribution imbalances are evened out,
and for IceCube most neutrinos are around the \SIrange{10}{100}{\giga\electronvolt} mark, as seen in~\vref{fig:weighted_distribution}.
Thus re-weighting to a uniform distribution is very useful for extremely low energy events, under \SI{10}{\giga\electronvolt}, although performance suffers most everywhere else; see~\vref{fig:weighting}.

\subsection{Model architecture}

\begin{figure}
    \centering
    %% Creator: Matplotlib, PGF backend
%%
%% To include the figure in your LaTeX document, write
%%   \input{<filename>.pgf}
%%
%% Make sure the required packages are loaded in your preamble
%%   \usepackage{pgf}
%%
%% and, on pdftex
%%   \usepackage[utf8]{inputenc}\DeclareUnicodeCharacter{2212}{-}
%%
%% or, on luatex and xetex
%%   \usepackage{unicode-math}
%%
%% Figures using additional raster images can only be included by \input if
%% they are in the same directory as the main LaTeX file. For loading figures
%% from other directories you can use the `import` package
%%   \usepackage{import}
%%
%% and then include the figures with
%%   \import{<path to file>}{<filename>.pgf}
%%
%% Matplotlib used the following preamble
%%   \usepackage{siunitx} \usepackage{amsmath} \usepackage{bm}
%%   \usepackage{fontspec}
%%
\begingroup%
\makeatletter%
\begin{pgfpicture}%
\pgfpathrectangle{\pgfpointorigin}{\pgfqpoint{6.201200in}{3.000000in}}%
\pgfusepath{use as bounding box, clip}%
\begin{pgfscope}%
\pgfsetbuttcap%
\pgfsetmiterjoin%
\definecolor{currentfill}{rgb}{1.000000,1.000000,1.000000}%
\pgfsetfillcolor{currentfill}%
\pgfsetlinewidth{0.000000pt}%
\definecolor{currentstroke}{rgb}{1.000000,1.000000,1.000000}%
\pgfsetstrokecolor{currentstroke}%
\pgfsetdash{}{0pt}%
\pgfpathmoveto{\pgfqpoint{0.000000in}{0.000000in}}%
\pgfpathlineto{\pgfqpoint{6.201200in}{0.000000in}}%
\pgfpathlineto{\pgfqpoint{6.201200in}{3.000000in}}%
\pgfpathlineto{\pgfqpoint{0.000000in}{3.000000in}}%
\pgfpathclose%
\pgfusepath{fill}%
\end{pgfscope}%
\begin{pgfscope}%
\pgfsetbuttcap%
\pgfsetmiterjoin%
\definecolor{currentfill}{rgb}{1.000000,1.000000,1.000000}%
\pgfsetfillcolor{currentfill}%
\pgfsetlinewidth{0.000000pt}%
\definecolor{currentstroke}{rgb}{0.000000,0.000000,0.000000}%
\pgfsetstrokecolor{currentstroke}%
\pgfsetstrokeopacity{0.000000}%
\pgfsetdash{}{0pt}%
\pgfpathmoveto{\pgfqpoint{0.572918in}{0.553781in}}%
\pgfpathlineto{\pgfqpoint{6.051200in}{0.553781in}}%
\pgfpathlineto{\pgfqpoint{6.051200in}{2.649333in}}%
\pgfpathlineto{\pgfqpoint{0.572918in}{2.649333in}}%
\pgfpathclose%
\pgfusepath{fill}%
\end{pgfscope}%
\begin{pgfscope}%
\pgfpathrectangle{\pgfqpoint{0.572918in}{0.553781in}}{\pgfqpoint{5.478282in}{2.095553in}}%
\pgfusepath{clip}%
\pgfsetbuttcap%
\pgfsetroundjoin%
\pgfsetlinewidth{0.501875pt}%
\definecolor{currentstroke}{rgb}{0.690196,0.690196,0.690196}%
\pgfsetstrokecolor{currentstroke}%
\pgfsetstrokeopacity{0.500000}%
\pgfsetdash{{0.500000pt}{0.825000pt}}{0.000000pt}%
\pgfpathmoveto{\pgfqpoint{0.675453in}{0.553781in}}%
\pgfpathlineto{\pgfqpoint{0.675453in}{2.649333in}}%
\pgfusepath{stroke}%
\end{pgfscope}%
\begin{pgfscope}%
\pgfsetbuttcap%
\pgfsetroundjoin%
\definecolor{currentfill}{rgb}{0.000000,0.000000,0.000000}%
\pgfsetfillcolor{currentfill}%
\pgfsetlinewidth{0.803000pt}%
\definecolor{currentstroke}{rgb}{0.000000,0.000000,0.000000}%
\pgfsetstrokecolor{currentstroke}%
\pgfsetdash{}{0pt}%
\pgfsys@defobject{currentmarker}{\pgfqpoint{0.000000in}{-0.048611in}}{\pgfqpoint{0.000000in}{0.000000in}}{%
\pgfpathmoveto{\pgfqpoint{0.000000in}{0.000000in}}%
\pgfpathlineto{\pgfqpoint{0.000000in}{-0.048611in}}%
\pgfusepath{stroke,fill}%
}%
\begin{pgfscope}%
\pgfsys@transformshift{0.675453in}{0.553781in}%
\pgfsys@useobject{currentmarker}{}%
\end{pgfscope}%
\end{pgfscope}%
\begin{pgfscope}%
\definecolor{textcolor}{rgb}{0.000000,0.000000,0.000000}%
\pgfsetstrokecolor{textcolor}%
\pgfsetfillcolor{textcolor}%
\pgftext[x=0.675453in,y=0.456558in,,top]{\color{textcolor}\rmfamily\fontsize{8.000000}{9.600000}\selectfont \(\displaystyle {0.0}\)}%
\end{pgfscope}%
\begin{pgfscope}%
\pgfpathrectangle{\pgfqpoint{0.572918in}{0.553781in}}{\pgfqpoint{5.478282in}{2.095553in}}%
\pgfusepath{clip}%
\pgfsetbuttcap%
\pgfsetroundjoin%
\pgfsetlinewidth{0.501875pt}%
\definecolor{currentstroke}{rgb}{0.690196,0.690196,0.690196}%
\pgfsetstrokecolor{currentstroke}%
\pgfsetstrokeopacity{0.500000}%
\pgfsetdash{{0.500000pt}{0.825000pt}}{0.000000pt}%
\pgfpathmoveto{\pgfqpoint{1.554322in}{0.553781in}}%
\pgfpathlineto{\pgfqpoint{1.554322in}{2.649333in}}%
\pgfusepath{stroke}%
\end{pgfscope}%
\begin{pgfscope}%
\pgfsetbuttcap%
\pgfsetroundjoin%
\definecolor{currentfill}{rgb}{0.000000,0.000000,0.000000}%
\pgfsetfillcolor{currentfill}%
\pgfsetlinewidth{0.803000pt}%
\definecolor{currentstroke}{rgb}{0.000000,0.000000,0.000000}%
\pgfsetstrokecolor{currentstroke}%
\pgfsetdash{}{0pt}%
\pgfsys@defobject{currentmarker}{\pgfqpoint{0.000000in}{-0.048611in}}{\pgfqpoint{0.000000in}{0.000000in}}{%
\pgfpathmoveto{\pgfqpoint{0.000000in}{0.000000in}}%
\pgfpathlineto{\pgfqpoint{0.000000in}{-0.048611in}}%
\pgfusepath{stroke,fill}%
}%
\begin{pgfscope}%
\pgfsys@transformshift{1.554322in}{0.553781in}%
\pgfsys@useobject{currentmarker}{}%
\end{pgfscope}%
\end{pgfscope}%
\begin{pgfscope}%
\definecolor{textcolor}{rgb}{0.000000,0.000000,0.000000}%
\pgfsetstrokecolor{textcolor}%
\pgfsetfillcolor{textcolor}%
\pgftext[x=1.554322in,y=0.456558in,,top]{\color{textcolor}\rmfamily\fontsize{8.000000}{9.600000}\selectfont \(\displaystyle {0.5}\)}%
\end{pgfscope}%
\begin{pgfscope}%
\pgfpathrectangle{\pgfqpoint{0.572918in}{0.553781in}}{\pgfqpoint{5.478282in}{2.095553in}}%
\pgfusepath{clip}%
\pgfsetbuttcap%
\pgfsetroundjoin%
\pgfsetlinewidth{0.501875pt}%
\definecolor{currentstroke}{rgb}{0.690196,0.690196,0.690196}%
\pgfsetstrokecolor{currentstroke}%
\pgfsetstrokeopacity{0.500000}%
\pgfsetdash{{0.500000pt}{0.825000pt}}{0.000000pt}%
\pgfpathmoveto{\pgfqpoint{2.433190in}{0.553781in}}%
\pgfpathlineto{\pgfqpoint{2.433190in}{2.649333in}}%
\pgfusepath{stroke}%
\end{pgfscope}%
\begin{pgfscope}%
\pgfsetbuttcap%
\pgfsetroundjoin%
\definecolor{currentfill}{rgb}{0.000000,0.000000,0.000000}%
\pgfsetfillcolor{currentfill}%
\pgfsetlinewidth{0.803000pt}%
\definecolor{currentstroke}{rgb}{0.000000,0.000000,0.000000}%
\pgfsetstrokecolor{currentstroke}%
\pgfsetdash{}{0pt}%
\pgfsys@defobject{currentmarker}{\pgfqpoint{0.000000in}{-0.048611in}}{\pgfqpoint{0.000000in}{0.000000in}}{%
\pgfpathmoveto{\pgfqpoint{0.000000in}{0.000000in}}%
\pgfpathlineto{\pgfqpoint{0.000000in}{-0.048611in}}%
\pgfusepath{stroke,fill}%
}%
\begin{pgfscope}%
\pgfsys@transformshift{2.433190in}{0.553781in}%
\pgfsys@useobject{currentmarker}{}%
\end{pgfscope}%
\end{pgfscope}%
\begin{pgfscope}%
\definecolor{textcolor}{rgb}{0.000000,0.000000,0.000000}%
\pgfsetstrokecolor{textcolor}%
\pgfsetfillcolor{textcolor}%
\pgftext[x=2.433190in,y=0.456558in,,top]{\color{textcolor}\rmfamily\fontsize{8.000000}{9.600000}\selectfont \(\displaystyle {1.0}\)}%
\end{pgfscope}%
\begin{pgfscope}%
\pgfpathrectangle{\pgfqpoint{0.572918in}{0.553781in}}{\pgfqpoint{5.478282in}{2.095553in}}%
\pgfusepath{clip}%
\pgfsetbuttcap%
\pgfsetroundjoin%
\pgfsetlinewidth{0.501875pt}%
\definecolor{currentstroke}{rgb}{0.690196,0.690196,0.690196}%
\pgfsetstrokecolor{currentstroke}%
\pgfsetstrokeopacity{0.500000}%
\pgfsetdash{{0.500000pt}{0.825000pt}}{0.000000pt}%
\pgfpathmoveto{\pgfqpoint{3.312059in}{0.553781in}}%
\pgfpathlineto{\pgfqpoint{3.312059in}{2.649333in}}%
\pgfusepath{stroke}%
\end{pgfscope}%
\begin{pgfscope}%
\pgfsetbuttcap%
\pgfsetroundjoin%
\definecolor{currentfill}{rgb}{0.000000,0.000000,0.000000}%
\pgfsetfillcolor{currentfill}%
\pgfsetlinewidth{0.803000pt}%
\definecolor{currentstroke}{rgb}{0.000000,0.000000,0.000000}%
\pgfsetstrokecolor{currentstroke}%
\pgfsetdash{}{0pt}%
\pgfsys@defobject{currentmarker}{\pgfqpoint{0.000000in}{-0.048611in}}{\pgfqpoint{0.000000in}{0.000000in}}{%
\pgfpathmoveto{\pgfqpoint{0.000000in}{0.000000in}}%
\pgfpathlineto{\pgfqpoint{0.000000in}{-0.048611in}}%
\pgfusepath{stroke,fill}%
}%
\begin{pgfscope}%
\pgfsys@transformshift{3.312059in}{0.553781in}%
\pgfsys@useobject{currentmarker}{}%
\end{pgfscope}%
\end{pgfscope}%
\begin{pgfscope}%
\definecolor{textcolor}{rgb}{0.000000,0.000000,0.000000}%
\pgfsetstrokecolor{textcolor}%
\pgfsetfillcolor{textcolor}%
\pgftext[x=3.312059in,y=0.456558in,,top]{\color{textcolor}\rmfamily\fontsize{8.000000}{9.600000}\selectfont \(\displaystyle {1.5}\)}%
\end{pgfscope}%
\begin{pgfscope}%
\pgfpathrectangle{\pgfqpoint{0.572918in}{0.553781in}}{\pgfqpoint{5.478282in}{2.095553in}}%
\pgfusepath{clip}%
\pgfsetbuttcap%
\pgfsetroundjoin%
\pgfsetlinewidth{0.501875pt}%
\definecolor{currentstroke}{rgb}{0.690196,0.690196,0.690196}%
\pgfsetstrokecolor{currentstroke}%
\pgfsetstrokeopacity{0.500000}%
\pgfsetdash{{0.500000pt}{0.825000pt}}{0.000000pt}%
\pgfpathmoveto{\pgfqpoint{4.190928in}{0.553781in}}%
\pgfpathlineto{\pgfqpoint{4.190928in}{2.649333in}}%
\pgfusepath{stroke}%
\end{pgfscope}%
\begin{pgfscope}%
\pgfsetbuttcap%
\pgfsetroundjoin%
\definecolor{currentfill}{rgb}{0.000000,0.000000,0.000000}%
\pgfsetfillcolor{currentfill}%
\pgfsetlinewidth{0.803000pt}%
\definecolor{currentstroke}{rgb}{0.000000,0.000000,0.000000}%
\pgfsetstrokecolor{currentstroke}%
\pgfsetdash{}{0pt}%
\pgfsys@defobject{currentmarker}{\pgfqpoint{0.000000in}{-0.048611in}}{\pgfqpoint{0.000000in}{0.000000in}}{%
\pgfpathmoveto{\pgfqpoint{0.000000in}{0.000000in}}%
\pgfpathlineto{\pgfqpoint{0.000000in}{-0.048611in}}%
\pgfusepath{stroke,fill}%
}%
\begin{pgfscope}%
\pgfsys@transformshift{4.190928in}{0.553781in}%
\pgfsys@useobject{currentmarker}{}%
\end{pgfscope}%
\end{pgfscope}%
\begin{pgfscope}%
\definecolor{textcolor}{rgb}{0.000000,0.000000,0.000000}%
\pgfsetstrokecolor{textcolor}%
\pgfsetfillcolor{textcolor}%
\pgftext[x=4.190928in,y=0.456558in,,top]{\color{textcolor}\rmfamily\fontsize{8.000000}{9.600000}\selectfont \(\displaystyle {2.0}\)}%
\end{pgfscope}%
\begin{pgfscope}%
\pgfpathrectangle{\pgfqpoint{0.572918in}{0.553781in}}{\pgfqpoint{5.478282in}{2.095553in}}%
\pgfusepath{clip}%
\pgfsetbuttcap%
\pgfsetroundjoin%
\pgfsetlinewidth{0.501875pt}%
\definecolor{currentstroke}{rgb}{0.690196,0.690196,0.690196}%
\pgfsetstrokecolor{currentstroke}%
\pgfsetstrokeopacity{0.500000}%
\pgfsetdash{{0.500000pt}{0.825000pt}}{0.000000pt}%
\pgfpathmoveto{\pgfqpoint{5.069797in}{0.553781in}}%
\pgfpathlineto{\pgfqpoint{5.069797in}{2.649333in}}%
\pgfusepath{stroke}%
\end{pgfscope}%
\begin{pgfscope}%
\pgfsetbuttcap%
\pgfsetroundjoin%
\definecolor{currentfill}{rgb}{0.000000,0.000000,0.000000}%
\pgfsetfillcolor{currentfill}%
\pgfsetlinewidth{0.803000pt}%
\definecolor{currentstroke}{rgb}{0.000000,0.000000,0.000000}%
\pgfsetstrokecolor{currentstroke}%
\pgfsetdash{}{0pt}%
\pgfsys@defobject{currentmarker}{\pgfqpoint{0.000000in}{-0.048611in}}{\pgfqpoint{0.000000in}{0.000000in}}{%
\pgfpathmoveto{\pgfqpoint{0.000000in}{0.000000in}}%
\pgfpathlineto{\pgfqpoint{0.000000in}{-0.048611in}}%
\pgfusepath{stroke,fill}%
}%
\begin{pgfscope}%
\pgfsys@transformshift{5.069797in}{0.553781in}%
\pgfsys@useobject{currentmarker}{}%
\end{pgfscope}%
\end{pgfscope}%
\begin{pgfscope}%
\definecolor{textcolor}{rgb}{0.000000,0.000000,0.000000}%
\pgfsetstrokecolor{textcolor}%
\pgfsetfillcolor{textcolor}%
\pgftext[x=5.069797in,y=0.456558in,,top]{\color{textcolor}\rmfamily\fontsize{8.000000}{9.600000}\selectfont \(\displaystyle {2.5}\)}%
\end{pgfscope}%
\begin{pgfscope}%
\pgfpathrectangle{\pgfqpoint{0.572918in}{0.553781in}}{\pgfqpoint{5.478282in}{2.095553in}}%
\pgfusepath{clip}%
\pgfsetbuttcap%
\pgfsetroundjoin%
\pgfsetlinewidth{0.501875pt}%
\definecolor{currentstroke}{rgb}{0.690196,0.690196,0.690196}%
\pgfsetstrokecolor{currentstroke}%
\pgfsetstrokeopacity{0.500000}%
\pgfsetdash{{0.500000pt}{0.825000pt}}{0.000000pt}%
\pgfpathmoveto{\pgfqpoint{5.948665in}{0.553781in}}%
\pgfpathlineto{\pgfqpoint{5.948665in}{2.649333in}}%
\pgfusepath{stroke}%
\end{pgfscope}%
\begin{pgfscope}%
\pgfsetbuttcap%
\pgfsetroundjoin%
\definecolor{currentfill}{rgb}{0.000000,0.000000,0.000000}%
\pgfsetfillcolor{currentfill}%
\pgfsetlinewidth{0.803000pt}%
\definecolor{currentstroke}{rgb}{0.000000,0.000000,0.000000}%
\pgfsetstrokecolor{currentstroke}%
\pgfsetdash{}{0pt}%
\pgfsys@defobject{currentmarker}{\pgfqpoint{0.000000in}{-0.048611in}}{\pgfqpoint{0.000000in}{0.000000in}}{%
\pgfpathmoveto{\pgfqpoint{0.000000in}{0.000000in}}%
\pgfpathlineto{\pgfqpoint{0.000000in}{-0.048611in}}%
\pgfusepath{stroke,fill}%
}%
\begin{pgfscope}%
\pgfsys@transformshift{5.948665in}{0.553781in}%
\pgfsys@useobject{currentmarker}{}%
\end{pgfscope}%
\end{pgfscope}%
\begin{pgfscope}%
\definecolor{textcolor}{rgb}{0.000000,0.000000,0.000000}%
\pgfsetstrokecolor{textcolor}%
\pgfsetfillcolor{textcolor}%
\pgftext[x=5.948665in,y=0.456558in,,top]{\color{textcolor}\rmfamily\fontsize{8.000000}{9.600000}\selectfont \(\displaystyle {3.0}\)}%
\end{pgfscope}%
\begin{pgfscope}%
\definecolor{textcolor}{rgb}{0.000000,0.000000,0.000000}%
\pgfsetstrokecolor{textcolor}%
\pgfsetfillcolor{textcolor}%
\pgftext[x=3.312059in,y=0.302336in,,top]{\color{textcolor}\rmfamily\fontsize{10.950000}{13.140000}\selectfont \(\displaystyle \log_{10}(E_{\textup{true}}) \, \left[ E / \textup{GeV} \right]\)}%
\end{pgfscope}%
\begin{pgfscope}%
\pgfpathrectangle{\pgfqpoint{0.572918in}{0.553781in}}{\pgfqpoint{5.478282in}{2.095553in}}%
\pgfusepath{clip}%
\pgfsetbuttcap%
\pgfsetroundjoin%
\pgfsetlinewidth{0.501875pt}%
\definecolor{currentstroke}{rgb}{0.690196,0.690196,0.690196}%
\pgfsetstrokecolor{currentstroke}%
\pgfsetstrokeopacity{0.500000}%
\pgfsetdash{{0.500000pt}{0.825000pt}}{0.000000pt}%
\pgfpathmoveto{\pgfqpoint{0.572918in}{0.853750in}}%
\pgfpathlineto{\pgfqpoint{6.051200in}{0.853750in}}%
\pgfusepath{stroke}%
\end{pgfscope}%
\begin{pgfscope}%
\pgfsetbuttcap%
\pgfsetroundjoin%
\definecolor{currentfill}{rgb}{0.000000,0.000000,0.000000}%
\pgfsetfillcolor{currentfill}%
\pgfsetlinewidth{0.803000pt}%
\definecolor{currentstroke}{rgb}{0.000000,0.000000,0.000000}%
\pgfsetstrokecolor{currentstroke}%
\pgfsetdash{}{0pt}%
\pgfsys@defobject{currentmarker}{\pgfqpoint{-0.048611in}{0.000000in}}{\pgfqpoint{-0.000000in}{0.000000in}}{%
\pgfpathmoveto{\pgfqpoint{-0.000000in}{0.000000in}}%
\pgfpathlineto{\pgfqpoint{-0.048611in}{0.000000in}}%
\pgfusepath{stroke,fill}%
}%
\begin{pgfscope}%
\pgfsys@transformshift{0.572918in}{0.853750in}%
\pgfsys@useobject{currentmarker}{}%
\end{pgfscope}%
\end{pgfscope}%
\begin{pgfscope}%
\definecolor{textcolor}{rgb}{0.000000,0.000000,0.000000}%
\pgfsetstrokecolor{textcolor}%
\pgfsetfillcolor{textcolor}%
\pgftext[x=0.357639in, y=0.815194in, left, base]{\color{textcolor}\rmfamily\fontsize{8.000000}{9.600000}\selectfont \(\displaystyle {15}\)}%
\end{pgfscope}%
\begin{pgfscope}%
\pgfpathrectangle{\pgfqpoint{0.572918in}{0.553781in}}{\pgfqpoint{5.478282in}{2.095553in}}%
\pgfusepath{clip}%
\pgfsetbuttcap%
\pgfsetroundjoin%
\pgfsetlinewidth{0.501875pt}%
\definecolor{currentstroke}{rgb}{0.690196,0.690196,0.690196}%
\pgfsetstrokecolor{currentstroke}%
\pgfsetstrokeopacity{0.500000}%
\pgfsetdash{{0.500000pt}{0.825000pt}}{0.000000pt}%
\pgfpathmoveto{\pgfqpoint{0.572918in}{1.297035in}}%
\pgfpathlineto{\pgfqpoint{6.051200in}{1.297035in}}%
\pgfusepath{stroke}%
\end{pgfscope}%
\begin{pgfscope}%
\pgfsetbuttcap%
\pgfsetroundjoin%
\definecolor{currentfill}{rgb}{0.000000,0.000000,0.000000}%
\pgfsetfillcolor{currentfill}%
\pgfsetlinewidth{0.803000pt}%
\definecolor{currentstroke}{rgb}{0.000000,0.000000,0.000000}%
\pgfsetstrokecolor{currentstroke}%
\pgfsetdash{}{0pt}%
\pgfsys@defobject{currentmarker}{\pgfqpoint{-0.048611in}{0.000000in}}{\pgfqpoint{-0.000000in}{0.000000in}}{%
\pgfpathmoveto{\pgfqpoint{-0.000000in}{0.000000in}}%
\pgfpathlineto{\pgfqpoint{-0.048611in}{0.000000in}}%
\pgfusepath{stroke,fill}%
}%
\begin{pgfscope}%
\pgfsys@transformshift{0.572918in}{1.297035in}%
\pgfsys@useobject{currentmarker}{}%
\end{pgfscope}%
\end{pgfscope}%
\begin{pgfscope}%
\definecolor{textcolor}{rgb}{0.000000,0.000000,0.000000}%
\pgfsetstrokecolor{textcolor}%
\pgfsetfillcolor{textcolor}%
\pgftext[x=0.357639in, y=1.258480in, left, base]{\color{textcolor}\rmfamily\fontsize{8.000000}{9.600000}\selectfont \(\displaystyle {20}\)}%
\end{pgfscope}%
\begin{pgfscope}%
\pgfpathrectangle{\pgfqpoint{0.572918in}{0.553781in}}{\pgfqpoint{5.478282in}{2.095553in}}%
\pgfusepath{clip}%
\pgfsetbuttcap%
\pgfsetroundjoin%
\pgfsetlinewidth{0.501875pt}%
\definecolor{currentstroke}{rgb}{0.690196,0.690196,0.690196}%
\pgfsetstrokecolor{currentstroke}%
\pgfsetstrokeopacity{0.500000}%
\pgfsetdash{{0.500000pt}{0.825000pt}}{0.000000pt}%
\pgfpathmoveto{\pgfqpoint{0.572918in}{1.740321in}}%
\pgfpathlineto{\pgfqpoint{6.051200in}{1.740321in}}%
\pgfusepath{stroke}%
\end{pgfscope}%
\begin{pgfscope}%
\pgfsetbuttcap%
\pgfsetroundjoin%
\definecolor{currentfill}{rgb}{0.000000,0.000000,0.000000}%
\pgfsetfillcolor{currentfill}%
\pgfsetlinewidth{0.803000pt}%
\definecolor{currentstroke}{rgb}{0.000000,0.000000,0.000000}%
\pgfsetstrokecolor{currentstroke}%
\pgfsetdash{}{0pt}%
\pgfsys@defobject{currentmarker}{\pgfqpoint{-0.048611in}{0.000000in}}{\pgfqpoint{-0.000000in}{0.000000in}}{%
\pgfpathmoveto{\pgfqpoint{-0.000000in}{0.000000in}}%
\pgfpathlineto{\pgfqpoint{-0.048611in}{0.000000in}}%
\pgfusepath{stroke,fill}%
}%
\begin{pgfscope}%
\pgfsys@transformshift{0.572918in}{1.740321in}%
\pgfsys@useobject{currentmarker}{}%
\end{pgfscope}%
\end{pgfscope}%
\begin{pgfscope}%
\definecolor{textcolor}{rgb}{0.000000,0.000000,0.000000}%
\pgfsetstrokecolor{textcolor}%
\pgfsetfillcolor{textcolor}%
\pgftext[x=0.357639in, y=1.701765in, left, base]{\color{textcolor}\rmfamily\fontsize{8.000000}{9.600000}\selectfont \(\displaystyle {25}\)}%
\end{pgfscope}%
\begin{pgfscope}%
\pgfpathrectangle{\pgfqpoint{0.572918in}{0.553781in}}{\pgfqpoint{5.478282in}{2.095553in}}%
\pgfusepath{clip}%
\pgfsetbuttcap%
\pgfsetroundjoin%
\pgfsetlinewidth{0.501875pt}%
\definecolor{currentstroke}{rgb}{0.690196,0.690196,0.690196}%
\pgfsetstrokecolor{currentstroke}%
\pgfsetstrokeopacity{0.500000}%
\pgfsetdash{{0.500000pt}{0.825000pt}}{0.000000pt}%
\pgfpathmoveto{\pgfqpoint{0.572918in}{2.183606in}}%
\pgfpathlineto{\pgfqpoint{6.051200in}{2.183606in}}%
\pgfusepath{stroke}%
\end{pgfscope}%
\begin{pgfscope}%
\pgfsetbuttcap%
\pgfsetroundjoin%
\definecolor{currentfill}{rgb}{0.000000,0.000000,0.000000}%
\pgfsetfillcolor{currentfill}%
\pgfsetlinewidth{0.803000pt}%
\definecolor{currentstroke}{rgb}{0.000000,0.000000,0.000000}%
\pgfsetstrokecolor{currentstroke}%
\pgfsetdash{}{0pt}%
\pgfsys@defobject{currentmarker}{\pgfqpoint{-0.048611in}{0.000000in}}{\pgfqpoint{-0.000000in}{0.000000in}}{%
\pgfpathmoveto{\pgfqpoint{-0.000000in}{0.000000in}}%
\pgfpathlineto{\pgfqpoint{-0.048611in}{0.000000in}}%
\pgfusepath{stroke,fill}%
}%
\begin{pgfscope}%
\pgfsys@transformshift{0.572918in}{2.183606in}%
\pgfsys@useobject{currentmarker}{}%
\end{pgfscope}%
\end{pgfscope}%
\begin{pgfscope}%
\definecolor{textcolor}{rgb}{0.000000,0.000000,0.000000}%
\pgfsetstrokecolor{textcolor}%
\pgfsetfillcolor{textcolor}%
\pgftext[x=0.357639in, y=2.145051in, left, base]{\color{textcolor}\rmfamily\fontsize{8.000000}{9.600000}\selectfont \(\displaystyle {30}\)}%
\end{pgfscope}%
\begin{pgfscope}%
\pgfpathrectangle{\pgfqpoint{0.572918in}{0.553781in}}{\pgfqpoint{5.478282in}{2.095553in}}%
\pgfusepath{clip}%
\pgfsetbuttcap%
\pgfsetroundjoin%
\pgfsetlinewidth{0.501875pt}%
\definecolor{currentstroke}{rgb}{0.690196,0.690196,0.690196}%
\pgfsetstrokecolor{currentstroke}%
\pgfsetstrokeopacity{0.500000}%
\pgfsetdash{{0.500000pt}{0.825000pt}}{0.000000pt}%
\pgfpathmoveto{\pgfqpoint{0.572918in}{2.626892in}}%
\pgfpathlineto{\pgfqpoint{6.051200in}{2.626892in}}%
\pgfusepath{stroke}%
\end{pgfscope}%
\begin{pgfscope}%
\pgfsetbuttcap%
\pgfsetroundjoin%
\definecolor{currentfill}{rgb}{0.000000,0.000000,0.000000}%
\pgfsetfillcolor{currentfill}%
\pgfsetlinewidth{0.803000pt}%
\definecolor{currentstroke}{rgb}{0.000000,0.000000,0.000000}%
\pgfsetstrokecolor{currentstroke}%
\pgfsetdash{}{0pt}%
\pgfsys@defobject{currentmarker}{\pgfqpoint{-0.048611in}{0.000000in}}{\pgfqpoint{-0.000000in}{0.000000in}}{%
\pgfpathmoveto{\pgfqpoint{-0.000000in}{0.000000in}}%
\pgfpathlineto{\pgfqpoint{-0.048611in}{0.000000in}}%
\pgfusepath{stroke,fill}%
}%
\begin{pgfscope}%
\pgfsys@transformshift{0.572918in}{2.626892in}%
\pgfsys@useobject{currentmarker}{}%
\end{pgfscope}%
\end{pgfscope}%
\begin{pgfscope}%
\definecolor{textcolor}{rgb}{0.000000,0.000000,0.000000}%
\pgfsetstrokecolor{textcolor}%
\pgfsetfillcolor{textcolor}%
\pgftext[x=0.357639in, y=2.588336in, left, base]{\color{textcolor}\rmfamily\fontsize{8.000000}{9.600000}\selectfont \(\displaystyle {35}\)}%
\end{pgfscope}%
\begin{pgfscope}%
\definecolor{textcolor}{rgb}{0.000000,0.000000,0.000000}%
\pgfsetstrokecolor{textcolor}%
\pgfsetfillcolor{textcolor}%
\pgftext[x=0.302083in,y=1.601557in,,bottom,rotate=90.000000]{\color{textcolor}\rmfamily\fontsize{10.950000}{13.140000}\selectfont IQR / 1.349 \(\displaystyle \left[ \textup{deg} \right]\)}%
\end{pgfscope}%
\begin{pgfscope}%
\pgfpathrectangle{\pgfqpoint{0.572918in}{0.553781in}}{\pgfqpoint{5.478282in}{2.095553in}}%
\pgfusepath{clip}%
\pgfsetbuttcap%
\pgfsetroundjoin%
\pgfsetlinewidth{1.505625pt}%
\definecolor{currentstroke}{rgb}{0.313725,0.317647,0.309804}%
\pgfsetstrokecolor{currentstroke}%
\pgfsetstrokeopacity{0.900000}%
\pgfsetdash{}{0pt}%
\pgfpathmoveto{\pgfqpoint{0.821931in}{1.793243in}}%
\pgfpathlineto{\pgfqpoint{0.821931in}{2.491504in}}%
\pgfusepath{stroke}%
\end{pgfscope}%
\begin{pgfscope}%
\pgfpathrectangle{\pgfqpoint{0.572918in}{0.553781in}}{\pgfqpoint{5.478282in}{2.095553in}}%
\pgfusepath{clip}%
\pgfsetbuttcap%
\pgfsetroundjoin%
\pgfsetlinewidth{1.505625pt}%
\definecolor{currentstroke}{rgb}{0.313725,0.317647,0.309804}%
\pgfsetstrokecolor{currentstroke}%
\pgfsetstrokeopacity{0.900000}%
\pgfsetdash{}{0pt}%
\pgfpathmoveto{\pgfqpoint{1.114887in}{1.905658in}}%
\pgfpathlineto{\pgfqpoint{1.114887in}{2.176266in}}%
\pgfusepath{stroke}%
\end{pgfscope}%
\begin{pgfscope}%
\pgfpathrectangle{\pgfqpoint{0.572918in}{0.553781in}}{\pgfqpoint{5.478282in}{2.095553in}}%
\pgfusepath{clip}%
\pgfsetbuttcap%
\pgfsetroundjoin%
\pgfsetlinewidth{1.505625pt}%
\definecolor{currentstroke}{rgb}{0.313725,0.317647,0.309804}%
\pgfsetstrokecolor{currentstroke}%
\pgfsetstrokeopacity{0.900000}%
\pgfsetdash{}{0pt}%
\pgfpathmoveto{\pgfqpoint{1.407844in}{1.801241in}}%
\pgfpathlineto{\pgfqpoint{1.407844in}{1.955344in}}%
\pgfusepath{stroke}%
\end{pgfscope}%
\begin{pgfscope}%
\pgfpathrectangle{\pgfqpoint{0.572918in}{0.553781in}}{\pgfqpoint{5.478282in}{2.095553in}}%
\pgfusepath{clip}%
\pgfsetbuttcap%
\pgfsetroundjoin%
\pgfsetlinewidth{1.505625pt}%
\definecolor{currentstroke}{rgb}{0.313725,0.317647,0.309804}%
\pgfsetstrokecolor{currentstroke}%
\pgfsetstrokeopacity{0.900000}%
\pgfsetdash{}{0pt}%
\pgfpathmoveto{\pgfqpoint{1.700800in}{1.827875in}}%
\pgfpathlineto{\pgfqpoint{1.700800in}{1.936199in}}%
\pgfusepath{stroke}%
\end{pgfscope}%
\begin{pgfscope}%
\pgfpathrectangle{\pgfqpoint{0.572918in}{0.553781in}}{\pgfqpoint{5.478282in}{2.095553in}}%
\pgfusepath{clip}%
\pgfsetbuttcap%
\pgfsetroundjoin%
\pgfsetlinewidth{1.505625pt}%
\definecolor{currentstroke}{rgb}{0.313725,0.317647,0.309804}%
\pgfsetstrokecolor{currentstroke}%
\pgfsetstrokeopacity{0.900000}%
\pgfsetdash{}{0pt}%
\pgfpathmoveto{\pgfqpoint{1.993756in}{1.920176in}}%
\pgfpathlineto{\pgfqpoint{1.993756in}{2.008295in}}%
\pgfusepath{stroke}%
\end{pgfscope}%
\begin{pgfscope}%
\pgfpathrectangle{\pgfqpoint{0.572918in}{0.553781in}}{\pgfqpoint{5.478282in}{2.095553in}}%
\pgfusepath{clip}%
\pgfsetbuttcap%
\pgfsetroundjoin%
\pgfsetlinewidth{1.505625pt}%
\definecolor{currentstroke}{rgb}{0.313725,0.317647,0.309804}%
\pgfsetstrokecolor{currentstroke}%
\pgfsetstrokeopacity{0.900000}%
\pgfsetdash{}{0pt}%
\pgfpathmoveto{\pgfqpoint{2.286712in}{1.892597in}}%
\pgfpathlineto{\pgfqpoint{2.286712in}{1.957057in}}%
\pgfusepath{stroke}%
\end{pgfscope}%
\begin{pgfscope}%
\pgfpathrectangle{\pgfqpoint{0.572918in}{0.553781in}}{\pgfqpoint{5.478282in}{2.095553in}}%
\pgfusepath{clip}%
\pgfsetbuttcap%
\pgfsetroundjoin%
\pgfsetlinewidth{1.505625pt}%
\definecolor{currentstroke}{rgb}{0.313725,0.317647,0.309804}%
\pgfsetstrokecolor{currentstroke}%
\pgfsetstrokeopacity{0.900000}%
\pgfsetdash{}{0pt}%
\pgfpathmoveto{\pgfqpoint{2.579669in}{1.780277in}}%
\pgfpathlineto{\pgfqpoint{2.579669in}{1.843966in}}%
\pgfusepath{stroke}%
\end{pgfscope}%
\begin{pgfscope}%
\pgfpathrectangle{\pgfqpoint{0.572918in}{0.553781in}}{\pgfqpoint{5.478282in}{2.095553in}}%
\pgfusepath{clip}%
\pgfsetbuttcap%
\pgfsetroundjoin%
\pgfsetlinewidth{1.505625pt}%
\definecolor{currentstroke}{rgb}{0.313725,0.317647,0.309804}%
\pgfsetstrokecolor{currentstroke}%
\pgfsetstrokeopacity{0.900000}%
\pgfsetdash{}{0pt}%
\pgfpathmoveto{\pgfqpoint{2.872625in}{1.698261in}}%
\pgfpathlineto{\pgfqpoint{2.872625in}{1.756413in}}%
\pgfusepath{stroke}%
\end{pgfscope}%
\begin{pgfscope}%
\pgfpathrectangle{\pgfqpoint{0.572918in}{0.553781in}}{\pgfqpoint{5.478282in}{2.095553in}}%
\pgfusepath{clip}%
\pgfsetbuttcap%
\pgfsetroundjoin%
\pgfsetlinewidth{1.505625pt}%
\definecolor{currentstroke}{rgb}{0.313725,0.317647,0.309804}%
\pgfsetstrokecolor{currentstroke}%
\pgfsetstrokeopacity{0.900000}%
\pgfsetdash{}{0pt}%
\pgfpathmoveto{\pgfqpoint{3.165581in}{1.518322in}}%
\pgfpathlineto{\pgfqpoint{3.165581in}{1.578548in}}%
\pgfusepath{stroke}%
\end{pgfscope}%
\begin{pgfscope}%
\pgfpathrectangle{\pgfqpoint{0.572918in}{0.553781in}}{\pgfqpoint{5.478282in}{2.095553in}}%
\pgfusepath{clip}%
\pgfsetbuttcap%
\pgfsetroundjoin%
\pgfsetlinewidth{1.505625pt}%
\definecolor{currentstroke}{rgb}{0.313725,0.317647,0.309804}%
\pgfsetstrokecolor{currentstroke}%
\pgfsetstrokeopacity{0.900000}%
\pgfsetdash{}{0pt}%
\pgfpathmoveto{\pgfqpoint{3.458537in}{1.370761in}}%
\pgfpathlineto{\pgfqpoint{3.458537in}{1.427054in}}%
\pgfusepath{stroke}%
\end{pgfscope}%
\begin{pgfscope}%
\pgfpathrectangle{\pgfqpoint{0.572918in}{0.553781in}}{\pgfqpoint{5.478282in}{2.095553in}}%
\pgfusepath{clip}%
\pgfsetbuttcap%
\pgfsetroundjoin%
\pgfsetlinewidth{1.505625pt}%
\definecolor{currentstroke}{rgb}{0.313725,0.317647,0.309804}%
\pgfsetstrokecolor{currentstroke}%
\pgfsetstrokeopacity{0.900000}%
\pgfsetdash{}{0pt}%
\pgfpathmoveto{\pgfqpoint{3.751494in}{1.245412in}}%
\pgfpathlineto{\pgfqpoint{3.751494in}{1.298650in}}%
\pgfusepath{stroke}%
\end{pgfscope}%
\begin{pgfscope}%
\pgfpathrectangle{\pgfqpoint{0.572918in}{0.553781in}}{\pgfqpoint{5.478282in}{2.095553in}}%
\pgfusepath{clip}%
\pgfsetbuttcap%
\pgfsetroundjoin%
\pgfsetlinewidth{1.505625pt}%
\definecolor{currentstroke}{rgb}{0.313725,0.317647,0.309804}%
\pgfsetstrokecolor{currentstroke}%
\pgfsetstrokeopacity{0.900000}%
\pgfsetdash{}{0pt}%
\pgfpathmoveto{\pgfqpoint{4.044450in}{1.090258in}}%
\pgfpathlineto{\pgfqpoint{4.044450in}{1.154173in}}%
\pgfusepath{stroke}%
\end{pgfscope}%
\begin{pgfscope}%
\pgfpathrectangle{\pgfqpoint{0.572918in}{0.553781in}}{\pgfqpoint{5.478282in}{2.095553in}}%
\pgfusepath{clip}%
\pgfsetbuttcap%
\pgfsetroundjoin%
\pgfsetlinewidth{1.505625pt}%
\definecolor{currentstroke}{rgb}{0.313725,0.317647,0.309804}%
\pgfsetstrokecolor{currentstroke}%
\pgfsetstrokeopacity{0.900000}%
\pgfsetdash{}{0pt}%
\pgfpathmoveto{\pgfqpoint{4.337406in}{0.946755in}}%
\pgfpathlineto{\pgfqpoint{4.337406in}{1.012467in}}%
\pgfusepath{stroke}%
\end{pgfscope}%
\begin{pgfscope}%
\pgfpathrectangle{\pgfqpoint{0.572918in}{0.553781in}}{\pgfqpoint{5.478282in}{2.095553in}}%
\pgfusepath{clip}%
\pgfsetbuttcap%
\pgfsetroundjoin%
\pgfsetlinewidth{1.505625pt}%
\definecolor{currentstroke}{rgb}{0.313725,0.317647,0.309804}%
\pgfsetstrokecolor{currentstroke}%
\pgfsetstrokeopacity{0.900000}%
\pgfsetdash{}{0pt}%
\pgfpathmoveto{\pgfqpoint{4.630362in}{0.786613in}}%
\pgfpathlineto{\pgfqpoint{4.630362in}{0.862186in}}%
\pgfusepath{stroke}%
\end{pgfscope}%
\begin{pgfscope}%
\pgfpathrectangle{\pgfqpoint{0.572918in}{0.553781in}}{\pgfqpoint{5.478282in}{2.095553in}}%
\pgfusepath{clip}%
\pgfsetbuttcap%
\pgfsetroundjoin%
\pgfsetlinewidth{1.505625pt}%
\definecolor{currentstroke}{rgb}{0.313725,0.317647,0.309804}%
\pgfsetstrokecolor{currentstroke}%
\pgfsetstrokeopacity{0.900000}%
\pgfsetdash{}{0pt}%
\pgfpathmoveto{\pgfqpoint{4.923318in}{0.717951in}}%
\pgfpathlineto{\pgfqpoint{4.923318in}{0.804240in}}%
\pgfusepath{stroke}%
\end{pgfscope}%
\begin{pgfscope}%
\pgfpathrectangle{\pgfqpoint{0.572918in}{0.553781in}}{\pgfqpoint{5.478282in}{2.095553in}}%
\pgfusepath{clip}%
\pgfsetbuttcap%
\pgfsetroundjoin%
\pgfsetlinewidth{1.505625pt}%
\definecolor{currentstroke}{rgb}{0.313725,0.317647,0.309804}%
\pgfsetstrokecolor{currentstroke}%
\pgfsetstrokeopacity{0.900000}%
\pgfsetdash{}{0pt}%
\pgfpathmoveto{\pgfqpoint{5.216275in}{0.677555in}}%
\pgfpathlineto{\pgfqpoint{5.216275in}{0.817856in}}%
\pgfusepath{stroke}%
\end{pgfscope}%
\begin{pgfscope}%
\pgfpathrectangle{\pgfqpoint{0.572918in}{0.553781in}}{\pgfqpoint{5.478282in}{2.095553in}}%
\pgfusepath{clip}%
\pgfsetbuttcap%
\pgfsetroundjoin%
\pgfsetlinewidth{1.505625pt}%
\definecolor{currentstroke}{rgb}{0.313725,0.317647,0.309804}%
\pgfsetstrokecolor{currentstroke}%
\pgfsetstrokeopacity{0.900000}%
\pgfsetdash{}{0pt}%
\pgfpathmoveto{\pgfqpoint{5.509231in}{0.649033in}}%
\pgfpathlineto{\pgfqpoint{5.509231in}{0.797193in}}%
\pgfusepath{stroke}%
\end{pgfscope}%
\begin{pgfscope}%
\pgfpathrectangle{\pgfqpoint{0.572918in}{0.553781in}}{\pgfqpoint{5.478282in}{2.095553in}}%
\pgfusepath{clip}%
\pgfsetbuttcap%
\pgfsetroundjoin%
\pgfsetlinewidth{1.505625pt}%
\definecolor{currentstroke}{rgb}{0.313725,0.317647,0.309804}%
\pgfsetstrokecolor{currentstroke}%
\pgfsetstrokeopacity{0.900000}%
\pgfsetdash{}{0pt}%
\pgfpathmoveto{\pgfqpoint{5.802187in}{0.768732in}}%
\pgfpathlineto{\pgfqpoint{5.802187in}{0.965842in}}%
\pgfusepath{stroke}%
\end{pgfscope}%
\begin{pgfscope}%
\pgfpathrectangle{\pgfqpoint{0.572918in}{0.553781in}}{\pgfqpoint{5.478282in}{2.095553in}}%
\pgfusepath{clip}%
\pgfsetbuttcap%
\pgfsetroundjoin%
\pgfsetlinewidth{1.505625pt}%
\definecolor{currentstroke}{rgb}{0.949020,0.372549,0.360784}%
\pgfsetstrokecolor{currentstroke}%
\pgfsetstrokeopacity{0.900000}%
\pgfsetdash{}{0pt}%
\pgfpathmoveto{\pgfqpoint{0.821931in}{1.926023in}}%
\pgfpathlineto{\pgfqpoint{0.821931in}{2.554081in}}%
\pgfusepath{stroke}%
\end{pgfscope}%
\begin{pgfscope}%
\pgfpathrectangle{\pgfqpoint{0.572918in}{0.553781in}}{\pgfqpoint{5.478282in}{2.095553in}}%
\pgfusepath{clip}%
\pgfsetbuttcap%
\pgfsetroundjoin%
\pgfsetlinewidth{1.505625pt}%
\definecolor{currentstroke}{rgb}{0.949020,0.372549,0.360784}%
\pgfsetstrokecolor{currentstroke}%
\pgfsetstrokeopacity{0.900000}%
\pgfsetdash{}{0pt}%
\pgfpathmoveto{\pgfqpoint{1.114887in}{1.868958in}}%
\pgfpathlineto{\pgfqpoint{1.114887in}{2.198651in}}%
\pgfusepath{stroke}%
\end{pgfscope}%
\begin{pgfscope}%
\pgfpathrectangle{\pgfqpoint{0.572918in}{0.553781in}}{\pgfqpoint{5.478282in}{2.095553in}}%
\pgfusepath{clip}%
\pgfsetbuttcap%
\pgfsetroundjoin%
\pgfsetlinewidth{1.505625pt}%
\definecolor{currentstroke}{rgb}{0.949020,0.372549,0.360784}%
\pgfsetstrokecolor{currentstroke}%
\pgfsetstrokeopacity{0.900000}%
\pgfsetdash{}{0pt}%
\pgfpathmoveto{\pgfqpoint{1.407844in}{1.916610in}}%
\pgfpathlineto{\pgfqpoint{1.407844in}{2.091522in}}%
\pgfusepath{stroke}%
\end{pgfscope}%
\begin{pgfscope}%
\pgfpathrectangle{\pgfqpoint{0.572918in}{0.553781in}}{\pgfqpoint{5.478282in}{2.095553in}}%
\pgfusepath{clip}%
\pgfsetbuttcap%
\pgfsetroundjoin%
\pgfsetlinewidth{1.505625pt}%
\definecolor{currentstroke}{rgb}{0.949020,0.372549,0.360784}%
\pgfsetstrokecolor{currentstroke}%
\pgfsetstrokeopacity{0.900000}%
\pgfsetdash{}{0pt}%
\pgfpathmoveto{\pgfqpoint{1.700800in}{2.047022in}}%
\pgfpathlineto{\pgfqpoint{1.700800in}{2.170552in}}%
\pgfusepath{stroke}%
\end{pgfscope}%
\begin{pgfscope}%
\pgfpathrectangle{\pgfqpoint{0.572918in}{0.553781in}}{\pgfqpoint{5.478282in}{2.095553in}}%
\pgfusepath{clip}%
\pgfsetbuttcap%
\pgfsetroundjoin%
\pgfsetlinewidth{1.505625pt}%
\definecolor{currentstroke}{rgb}{0.949020,0.372549,0.360784}%
\pgfsetstrokecolor{currentstroke}%
\pgfsetstrokeopacity{0.900000}%
\pgfsetdash{}{0pt}%
\pgfpathmoveto{\pgfqpoint{1.993756in}{2.114712in}}%
\pgfpathlineto{\pgfqpoint{1.993756in}{2.206146in}}%
\pgfusepath{stroke}%
\end{pgfscope}%
\begin{pgfscope}%
\pgfpathrectangle{\pgfqpoint{0.572918in}{0.553781in}}{\pgfqpoint{5.478282in}{2.095553in}}%
\pgfusepath{clip}%
\pgfsetbuttcap%
\pgfsetroundjoin%
\pgfsetlinewidth{1.505625pt}%
\definecolor{currentstroke}{rgb}{0.949020,0.372549,0.360784}%
\pgfsetstrokecolor{currentstroke}%
\pgfsetstrokeopacity{0.900000}%
\pgfsetdash{}{0pt}%
\pgfpathmoveto{\pgfqpoint{2.286712in}{2.139318in}}%
\pgfpathlineto{\pgfqpoint{2.286712in}{2.212870in}}%
\pgfusepath{stroke}%
\end{pgfscope}%
\begin{pgfscope}%
\pgfpathrectangle{\pgfqpoint{0.572918in}{0.553781in}}{\pgfqpoint{5.478282in}{2.095553in}}%
\pgfusepath{clip}%
\pgfsetbuttcap%
\pgfsetroundjoin%
\pgfsetlinewidth{1.505625pt}%
\definecolor{currentstroke}{rgb}{0.949020,0.372549,0.360784}%
\pgfsetstrokecolor{currentstroke}%
\pgfsetstrokeopacity{0.900000}%
\pgfsetdash{}{0pt}%
\pgfpathmoveto{\pgfqpoint{2.579669in}{1.985918in}}%
\pgfpathlineto{\pgfqpoint{2.579669in}{2.052150in}}%
\pgfusepath{stroke}%
\end{pgfscope}%
\begin{pgfscope}%
\pgfpathrectangle{\pgfqpoint{0.572918in}{0.553781in}}{\pgfqpoint{5.478282in}{2.095553in}}%
\pgfusepath{clip}%
\pgfsetbuttcap%
\pgfsetroundjoin%
\pgfsetlinewidth{1.505625pt}%
\definecolor{currentstroke}{rgb}{0.949020,0.372549,0.360784}%
\pgfsetstrokecolor{currentstroke}%
\pgfsetstrokeopacity{0.900000}%
\pgfsetdash{}{0pt}%
\pgfpathmoveto{\pgfqpoint{2.872625in}{1.883001in}}%
\pgfpathlineto{\pgfqpoint{2.872625in}{1.954018in}}%
\pgfusepath{stroke}%
\end{pgfscope}%
\begin{pgfscope}%
\pgfpathrectangle{\pgfqpoint{0.572918in}{0.553781in}}{\pgfqpoint{5.478282in}{2.095553in}}%
\pgfusepath{clip}%
\pgfsetbuttcap%
\pgfsetroundjoin%
\pgfsetlinewidth{1.505625pt}%
\definecolor{currentstroke}{rgb}{0.949020,0.372549,0.360784}%
\pgfsetstrokecolor{currentstroke}%
\pgfsetstrokeopacity{0.900000}%
\pgfsetdash{}{0pt}%
\pgfpathmoveto{\pgfqpoint{3.165581in}{1.719692in}}%
\pgfpathlineto{\pgfqpoint{3.165581in}{1.785707in}}%
\pgfusepath{stroke}%
\end{pgfscope}%
\begin{pgfscope}%
\pgfpathrectangle{\pgfqpoint{0.572918in}{0.553781in}}{\pgfqpoint{5.478282in}{2.095553in}}%
\pgfusepath{clip}%
\pgfsetbuttcap%
\pgfsetroundjoin%
\pgfsetlinewidth{1.505625pt}%
\definecolor{currentstroke}{rgb}{0.949020,0.372549,0.360784}%
\pgfsetstrokecolor{currentstroke}%
\pgfsetstrokeopacity{0.900000}%
\pgfsetdash{}{0pt}%
\pgfpathmoveto{\pgfqpoint{3.458537in}{1.592957in}}%
\pgfpathlineto{\pgfqpoint{3.458537in}{1.650216in}}%
\pgfusepath{stroke}%
\end{pgfscope}%
\begin{pgfscope}%
\pgfpathrectangle{\pgfqpoint{0.572918in}{0.553781in}}{\pgfqpoint{5.478282in}{2.095553in}}%
\pgfusepath{clip}%
\pgfsetbuttcap%
\pgfsetroundjoin%
\pgfsetlinewidth{1.505625pt}%
\definecolor{currentstroke}{rgb}{0.949020,0.372549,0.360784}%
\pgfsetstrokecolor{currentstroke}%
\pgfsetstrokeopacity{0.900000}%
\pgfsetdash{}{0pt}%
\pgfpathmoveto{\pgfqpoint{3.751494in}{1.488064in}}%
\pgfpathlineto{\pgfqpoint{3.751494in}{1.551427in}}%
\pgfusepath{stroke}%
\end{pgfscope}%
\begin{pgfscope}%
\pgfpathrectangle{\pgfqpoint{0.572918in}{0.553781in}}{\pgfqpoint{5.478282in}{2.095553in}}%
\pgfusepath{clip}%
\pgfsetbuttcap%
\pgfsetroundjoin%
\pgfsetlinewidth{1.505625pt}%
\definecolor{currentstroke}{rgb}{0.949020,0.372549,0.360784}%
\pgfsetstrokecolor{currentstroke}%
\pgfsetstrokeopacity{0.900000}%
\pgfsetdash{}{0pt}%
\pgfpathmoveto{\pgfqpoint{4.044450in}{1.430698in}}%
\pgfpathlineto{\pgfqpoint{4.044450in}{1.501249in}}%
\pgfusepath{stroke}%
\end{pgfscope}%
\begin{pgfscope}%
\pgfpathrectangle{\pgfqpoint{0.572918in}{0.553781in}}{\pgfqpoint{5.478282in}{2.095553in}}%
\pgfusepath{clip}%
\pgfsetbuttcap%
\pgfsetroundjoin%
\pgfsetlinewidth{1.505625pt}%
\definecolor{currentstroke}{rgb}{0.949020,0.372549,0.360784}%
\pgfsetstrokecolor{currentstroke}%
\pgfsetstrokeopacity{0.900000}%
\pgfsetdash{}{0pt}%
\pgfpathmoveto{\pgfqpoint{4.337406in}{1.359842in}}%
\pgfpathlineto{\pgfqpoint{4.337406in}{1.453341in}}%
\pgfusepath{stroke}%
\end{pgfscope}%
\begin{pgfscope}%
\pgfpathrectangle{\pgfqpoint{0.572918in}{0.553781in}}{\pgfqpoint{5.478282in}{2.095553in}}%
\pgfusepath{clip}%
\pgfsetbuttcap%
\pgfsetroundjoin%
\pgfsetlinewidth{1.505625pt}%
\definecolor{currentstroke}{rgb}{0.949020,0.372549,0.360784}%
\pgfsetstrokecolor{currentstroke}%
\pgfsetstrokeopacity{0.900000}%
\pgfsetdash{}{0pt}%
\pgfpathmoveto{\pgfqpoint{4.630362in}{1.262182in}}%
\pgfpathlineto{\pgfqpoint{4.630362in}{1.364769in}}%
\pgfusepath{stroke}%
\end{pgfscope}%
\begin{pgfscope}%
\pgfpathrectangle{\pgfqpoint{0.572918in}{0.553781in}}{\pgfqpoint{5.478282in}{2.095553in}}%
\pgfusepath{clip}%
\pgfsetbuttcap%
\pgfsetroundjoin%
\pgfsetlinewidth{1.505625pt}%
\definecolor{currentstroke}{rgb}{0.949020,0.372549,0.360784}%
\pgfsetstrokecolor{currentstroke}%
\pgfsetstrokeopacity{0.900000}%
\pgfsetdash{}{0pt}%
\pgfpathmoveto{\pgfqpoint{4.923318in}{1.215097in}}%
\pgfpathlineto{\pgfqpoint{4.923318in}{1.307425in}}%
\pgfusepath{stroke}%
\end{pgfscope}%
\begin{pgfscope}%
\pgfpathrectangle{\pgfqpoint{0.572918in}{0.553781in}}{\pgfqpoint{5.478282in}{2.095553in}}%
\pgfusepath{clip}%
\pgfsetbuttcap%
\pgfsetroundjoin%
\pgfsetlinewidth{1.505625pt}%
\definecolor{currentstroke}{rgb}{0.949020,0.372549,0.360784}%
\pgfsetstrokecolor{currentstroke}%
\pgfsetstrokeopacity{0.900000}%
\pgfsetdash{}{0pt}%
\pgfpathmoveto{\pgfqpoint{5.216275in}{1.169651in}}%
\pgfpathlineto{\pgfqpoint{5.216275in}{1.319654in}}%
\pgfusepath{stroke}%
\end{pgfscope}%
\begin{pgfscope}%
\pgfpathrectangle{\pgfqpoint{0.572918in}{0.553781in}}{\pgfqpoint{5.478282in}{2.095553in}}%
\pgfusepath{clip}%
\pgfsetbuttcap%
\pgfsetroundjoin%
\pgfsetlinewidth{1.505625pt}%
\definecolor{currentstroke}{rgb}{0.949020,0.372549,0.360784}%
\pgfsetstrokecolor{currentstroke}%
\pgfsetstrokeopacity{0.900000}%
\pgfsetdash{}{0pt}%
\pgfpathmoveto{\pgfqpoint{5.509231in}{1.052616in}}%
\pgfpathlineto{\pgfqpoint{5.509231in}{1.235690in}}%
\pgfusepath{stroke}%
\end{pgfscope}%
\begin{pgfscope}%
\pgfpathrectangle{\pgfqpoint{0.572918in}{0.553781in}}{\pgfqpoint{5.478282in}{2.095553in}}%
\pgfusepath{clip}%
\pgfsetbuttcap%
\pgfsetroundjoin%
\pgfsetlinewidth{1.505625pt}%
\definecolor{currentstroke}{rgb}{0.949020,0.372549,0.360784}%
\pgfsetstrokecolor{currentstroke}%
\pgfsetstrokeopacity{0.900000}%
\pgfsetdash{}{0pt}%
\pgfpathmoveto{\pgfqpoint{5.802187in}{1.066653in}}%
\pgfpathlineto{\pgfqpoint{5.802187in}{1.301442in}}%
\pgfusepath{stroke}%
\end{pgfscope}%
\begin{pgfscope}%
\pgfpathrectangle{\pgfqpoint{0.572918in}{0.553781in}}{\pgfqpoint{5.478282in}{2.095553in}}%
\pgfusepath{clip}%
\pgfsetbuttcap%
\pgfsetroundjoin%
\definecolor{currentfill}{rgb}{0.313725,0.317647,0.309804}%
\pgfsetfillcolor{currentfill}%
\pgfsetfillopacity{0.900000}%
\pgfsetlinewidth{1.003750pt}%
\definecolor{currentstroke}{rgb}{0.313725,0.317647,0.309804}%
\pgfsetstrokecolor{currentstroke}%
\pgfsetstrokeopacity{0.900000}%
\pgfsetdash{}{0pt}%
\pgfsys@defobject{currentmarker}{\pgfqpoint{-0.013889in}{-0.000000in}}{\pgfqpoint{0.013889in}{0.000000in}}{%
\pgfpathmoveto{\pgfqpoint{0.013889in}{-0.000000in}}%
\pgfpathlineto{\pgfqpoint{-0.013889in}{0.000000in}}%
\pgfusepath{stroke,fill}%
}%
\begin{pgfscope}%
\pgfsys@transformshift{0.821931in}{1.793243in}%
\pgfsys@useobject{currentmarker}{}%
\end{pgfscope}%
\begin{pgfscope}%
\pgfsys@transformshift{1.114887in}{1.905658in}%
\pgfsys@useobject{currentmarker}{}%
\end{pgfscope}%
\begin{pgfscope}%
\pgfsys@transformshift{1.407844in}{1.801241in}%
\pgfsys@useobject{currentmarker}{}%
\end{pgfscope}%
\begin{pgfscope}%
\pgfsys@transformshift{1.700800in}{1.827875in}%
\pgfsys@useobject{currentmarker}{}%
\end{pgfscope}%
\begin{pgfscope}%
\pgfsys@transformshift{1.993756in}{1.920176in}%
\pgfsys@useobject{currentmarker}{}%
\end{pgfscope}%
\begin{pgfscope}%
\pgfsys@transformshift{2.286712in}{1.892597in}%
\pgfsys@useobject{currentmarker}{}%
\end{pgfscope}%
\begin{pgfscope}%
\pgfsys@transformshift{2.579669in}{1.780277in}%
\pgfsys@useobject{currentmarker}{}%
\end{pgfscope}%
\begin{pgfscope}%
\pgfsys@transformshift{2.872625in}{1.698261in}%
\pgfsys@useobject{currentmarker}{}%
\end{pgfscope}%
\begin{pgfscope}%
\pgfsys@transformshift{3.165581in}{1.518322in}%
\pgfsys@useobject{currentmarker}{}%
\end{pgfscope}%
\begin{pgfscope}%
\pgfsys@transformshift{3.458537in}{1.370761in}%
\pgfsys@useobject{currentmarker}{}%
\end{pgfscope}%
\begin{pgfscope}%
\pgfsys@transformshift{3.751494in}{1.245412in}%
\pgfsys@useobject{currentmarker}{}%
\end{pgfscope}%
\begin{pgfscope}%
\pgfsys@transformshift{4.044450in}{1.090258in}%
\pgfsys@useobject{currentmarker}{}%
\end{pgfscope}%
\begin{pgfscope}%
\pgfsys@transformshift{4.337406in}{0.946755in}%
\pgfsys@useobject{currentmarker}{}%
\end{pgfscope}%
\begin{pgfscope}%
\pgfsys@transformshift{4.630362in}{0.786613in}%
\pgfsys@useobject{currentmarker}{}%
\end{pgfscope}%
\begin{pgfscope}%
\pgfsys@transformshift{4.923318in}{0.717951in}%
\pgfsys@useobject{currentmarker}{}%
\end{pgfscope}%
\begin{pgfscope}%
\pgfsys@transformshift{5.216275in}{0.677555in}%
\pgfsys@useobject{currentmarker}{}%
\end{pgfscope}%
\begin{pgfscope}%
\pgfsys@transformshift{5.509231in}{0.649033in}%
\pgfsys@useobject{currentmarker}{}%
\end{pgfscope}%
\begin{pgfscope}%
\pgfsys@transformshift{5.802187in}{0.768732in}%
\pgfsys@useobject{currentmarker}{}%
\end{pgfscope}%
\end{pgfscope}%
\begin{pgfscope}%
\pgfpathrectangle{\pgfqpoint{0.572918in}{0.553781in}}{\pgfqpoint{5.478282in}{2.095553in}}%
\pgfusepath{clip}%
\pgfsetbuttcap%
\pgfsetroundjoin%
\definecolor{currentfill}{rgb}{0.313725,0.317647,0.309804}%
\pgfsetfillcolor{currentfill}%
\pgfsetfillopacity{0.900000}%
\pgfsetlinewidth{1.003750pt}%
\definecolor{currentstroke}{rgb}{0.313725,0.317647,0.309804}%
\pgfsetstrokecolor{currentstroke}%
\pgfsetstrokeopacity{0.900000}%
\pgfsetdash{}{0pt}%
\pgfsys@defobject{currentmarker}{\pgfqpoint{-0.013889in}{-0.000000in}}{\pgfqpoint{0.013889in}{0.000000in}}{%
\pgfpathmoveto{\pgfqpoint{0.013889in}{-0.000000in}}%
\pgfpathlineto{\pgfqpoint{-0.013889in}{0.000000in}}%
\pgfusepath{stroke,fill}%
}%
\begin{pgfscope}%
\pgfsys@transformshift{0.821931in}{2.491504in}%
\pgfsys@useobject{currentmarker}{}%
\end{pgfscope}%
\begin{pgfscope}%
\pgfsys@transformshift{1.114887in}{2.176266in}%
\pgfsys@useobject{currentmarker}{}%
\end{pgfscope}%
\begin{pgfscope}%
\pgfsys@transformshift{1.407844in}{1.955344in}%
\pgfsys@useobject{currentmarker}{}%
\end{pgfscope}%
\begin{pgfscope}%
\pgfsys@transformshift{1.700800in}{1.936199in}%
\pgfsys@useobject{currentmarker}{}%
\end{pgfscope}%
\begin{pgfscope}%
\pgfsys@transformshift{1.993756in}{2.008295in}%
\pgfsys@useobject{currentmarker}{}%
\end{pgfscope}%
\begin{pgfscope}%
\pgfsys@transformshift{2.286712in}{1.957057in}%
\pgfsys@useobject{currentmarker}{}%
\end{pgfscope}%
\begin{pgfscope}%
\pgfsys@transformshift{2.579669in}{1.843966in}%
\pgfsys@useobject{currentmarker}{}%
\end{pgfscope}%
\begin{pgfscope}%
\pgfsys@transformshift{2.872625in}{1.756413in}%
\pgfsys@useobject{currentmarker}{}%
\end{pgfscope}%
\begin{pgfscope}%
\pgfsys@transformshift{3.165581in}{1.578548in}%
\pgfsys@useobject{currentmarker}{}%
\end{pgfscope}%
\begin{pgfscope}%
\pgfsys@transformshift{3.458537in}{1.427054in}%
\pgfsys@useobject{currentmarker}{}%
\end{pgfscope}%
\begin{pgfscope}%
\pgfsys@transformshift{3.751494in}{1.298650in}%
\pgfsys@useobject{currentmarker}{}%
\end{pgfscope}%
\begin{pgfscope}%
\pgfsys@transformshift{4.044450in}{1.154173in}%
\pgfsys@useobject{currentmarker}{}%
\end{pgfscope}%
\begin{pgfscope}%
\pgfsys@transformshift{4.337406in}{1.012467in}%
\pgfsys@useobject{currentmarker}{}%
\end{pgfscope}%
\begin{pgfscope}%
\pgfsys@transformshift{4.630362in}{0.862186in}%
\pgfsys@useobject{currentmarker}{}%
\end{pgfscope}%
\begin{pgfscope}%
\pgfsys@transformshift{4.923318in}{0.804240in}%
\pgfsys@useobject{currentmarker}{}%
\end{pgfscope}%
\begin{pgfscope}%
\pgfsys@transformshift{5.216275in}{0.817856in}%
\pgfsys@useobject{currentmarker}{}%
\end{pgfscope}%
\begin{pgfscope}%
\pgfsys@transformshift{5.509231in}{0.797193in}%
\pgfsys@useobject{currentmarker}{}%
\end{pgfscope}%
\begin{pgfscope}%
\pgfsys@transformshift{5.802187in}{0.965842in}%
\pgfsys@useobject{currentmarker}{}%
\end{pgfscope}%
\end{pgfscope}%
\begin{pgfscope}%
\pgfpathrectangle{\pgfqpoint{0.572918in}{0.553781in}}{\pgfqpoint{5.478282in}{2.095553in}}%
\pgfusepath{clip}%
\pgfsetbuttcap%
\pgfsetroundjoin%
\definecolor{currentfill}{rgb}{0.949020,0.372549,0.360784}%
\pgfsetfillcolor{currentfill}%
\pgfsetfillopacity{0.900000}%
\pgfsetlinewidth{1.003750pt}%
\definecolor{currentstroke}{rgb}{0.949020,0.372549,0.360784}%
\pgfsetstrokecolor{currentstroke}%
\pgfsetstrokeopacity{0.900000}%
\pgfsetdash{}{0pt}%
\pgfsys@defobject{currentmarker}{\pgfqpoint{-0.013889in}{-0.000000in}}{\pgfqpoint{0.013889in}{0.000000in}}{%
\pgfpathmoveto{\pgfqpoint{0.013889in}{-0.000000in}}%
\pgfpathlineto{\pgfqpoint{-0.013889in}{0.000000in}}%
\pgfusepath{stroke,fill}%
}%
\begin{pgfscope}%
\pgfsys@transformshift{0.821931in}{1.926023in}%
\pgfsys@useobject{currentmarker}{}%
\end{pgfscope}%
\begin{pgfscope}%
\pgfsys@transformshift{1.114887in}{1.868958in}%
\pgfsys@useobject{currentmarker}{}%
\end{pgfscope}%
\begin{pgfscope}%
\pgfsys@transformshift{1.407844in}{1.916610in}%
\pgfsys@useobject{currentmarker}{}%
\end{pgfscope}%
\begin{pgfscope}%
\pgfsys@transformshift{1.700800in}{2.047022in}%
\pgfsys@useobject{currentmarker}{}%
\end{pgfscope}%
\begin{pgfscope}%
\pgfsys@transformshift{1.993756in}{2.114712in}%
\pgfsys@useobject{currentmarker}{}%
\end{pgfscope}%
\begin{pgfscope}%
\pgfsys@transformshift{2.286712in}{2.139318in}%
\pgfsys@useobject{currentmarker}{}%
\end{pgfscope}%
\begin{pgfscope}%
\pgfsys@transformshift{2.579669in}{1.985918in}%
\pgfsys@useobject{currentmarker}{}%
\end{pgfscope}%
\begin{pgfscope}%
\pgfsys@transformshift{2.872625in}{1.883001in}%
\pgfsys@useobject{currentmarker}{}%
\end{pgfscope}%
\begin{pgfscope}%
\pgfsys@transformshift{3.165581in}{1.719692in}%
\pgfsys@useobject{currentmarker}{}%
\end{pgfscope}%
\begin{pgfscope}%
\pgfsys@transformshift{3.458537in}{1.592957in}%
\pgfsys@useobject{currentmarker}{}%
\end{pgfscope}%
\begin{pgfscope}%
\pgfsys@transformshift{3.751494in}{1.488064in}%
\pgfsys@useobject{currentmarker}{}%
\end{pgfscope}%
\begin{pgfscope}%
\pgfsys@transformshift{4.044450in}{1.430698in}%
\pgfsys@useobject{currentmarker}{}%
\end{pgfscope}%
\begin{pgfscope}%
\pgfsys@transformshift{4.337406in}{1.359842in}%
\pgfsys@useobject{currentmarker}{}%
\end{pgfscope}%
\begin{pgfscope}%
\pgfsys@transformshift{4.630362in}{1.262182in}%
\pgfsys@useobject{currentmarker}{}%
\end{pgfscope}%
\begin{pgfscope}%
\pgfsys@transformshift{4.923318in}{1.215097in}%
\pgfsys@useobject{currentmarker}{}%
\end{pgfscope}%
\begin{pgfscope}%
\pgfsys@transformshift{5.216275in}{1.169651in}%
\pgfsys@useobject{currentmarker}{}%
\end{pgfscope}%
\begin{pgfscope}%
\pgfsys@transformshift{5.509231in}{1.052616in}%
\pgfsys@useobject{currentmarker}{}%
\end{pgfscope}%
\begin{pgfscope}%
\pgfsys@transformshift{5.802187in}{1.066653in}%
\pgfsys@useobject{currentmarker}{}%
\end{pgfscope}%
\end{pgfscope}%
\begin{pgfscope}%
\pgfpathrectangle{\pgfqpoint{0.572918in}{0.553781in}}{\pgfqpoint{5.478282in}{2.095553in}}%
\pgfusepath{clip}%
\pgfsetbuttcap%
\pgfsetroundjoin%
\definecolor{currentfill}{rgb}{0.949020,0.372549,0.360784}%
\pgfsetfillcolor{currentfill}%
\pgfsetfillopacity{0.900000}%
\pgfsetlinewidth{1.003750pt}%
\definecolor{currentstroke}{rgb}{0.949020,0.372549,0.360784}%
\pgfsetstrokecolor{currentstroke}%
\pgfsetstrokeopacity{0.900000}%
\pgfsetdash{}{0pt}%
\pgfsys@defobject{currentmarker}{\pgfqpoint{-0.013889in}{-0.000000in}}{\pgfqpoint{0.013889in}{0.000000in}}{%
\pgfpathmoveto{\pgfqpoint{0.013889in}{-0.000000in}}%
\pgfpathlineto{\pgfqpoint{-0.013889in}{0.000000in}}%
\pgfusepath{stroke,fill}%
}%
\begin{pgfscope}%
\pgfsys@transformshift{0.821931in}{2.554081in}%
\pgfsys@useobject{currentmarker}{}%
\end{pgfscope}%
\begin{pgfscope}%
\pgfsys@transformshift{1.114887in}{2.198651in}%
\pgfsys@useobject{currentmarker}{}%
\end{pgfscope}%
\begin{pgfscope}%
\pgfsys@transformshift{1.407844in}{2.091522in}%
\pgfsys@useobject{currentmarker}{}%
\end{pgfscope}%
\begin{pgfscope}%
\pgfsys@transformshift{1.700800in}{2.170552in}%
\pgfsys@useobject{currentmarker}{}%
\end{pgfscope}%
\begin{pgfscope}%
\pgfsys@transformshift{1.993756in}{2.206146in}%
\pgfsys@useobject{currentmarker}{}%
\end{pgfscope}%
\begin{pgfscope}%
\pgfsys@transformshift{2.286712in}{2.212870in}%
\pgfsys@useobject{currentmarker}{}%
\end{pgfscope}%
\begin{pgfscope}%
\pgfsys@transformshift{2.579669in}{2.052150in}%
\pgfsys@useobject{currentmarker}{}%
\end{pgfscope}%
\begin{pgfscope}%
\pgfsys@transformshift{2.872625in}{1.954018in}%
\pgfsys@useobject{currentmarker}{}%
\end{pgfscope}%
\begin{pgfscope}%
\pgfsys@transformshift{3.165581in}{1.785707in}%
\pgfsys@useobject{currentmarker}{}%
\end{pgfscope}%
\begin{pgfscope}%
\pgfsys@transformshift{3.458537in}{1.650216in}%
\pgfsys@useobject{currentmarker}{}%
\end{pgfscope}%
\begin{pgfscope}%
\pgfsys@transformshift{3.751494in}{1.551427in}%
\pgfsys@useobject{currentmarker}{}%
\end{pgfscope}%
\begin{pgfscope}%
\pgfsys@transformshift{4.044450in}{1.501249in}%
\pgfsys@useobject{currentmarker}{}%
\end{pgfscope}%
\begin{pgfscope}%
\pgfsys@transformshift{4.337406in}{1.453341in}%
\pgfsys@useobject{currentmarker}{}%
\end{pgfscope}%
\begin{pgfscope}%
\pgfsys@transformshift{4.630362in}{1.364769in}%
\pgfsys@useobject{currentmarker}{}%
\end{pgfscope}%
\begin{pgfscope}%
\pgfsys@transformshift{4.923318in}{1.307425in}%
\pgfsys@useobject{currentmarker}{}%
\end{pgfscope}%
\begin{pgfscope}%
\pgfsys@transformshift{5.216275in}{1.319654in}%
\pgfsys@useobject{currentmarker}{}%
\end{pgfscope}%
\begin{pgfscope}%
\pgfsys@transformshift{5.509231in}{1.235690in}%
\pgfsys@useobject{currentmarker}{}%
\end{pgfscope}%
\begin{pgfscope}%
\pgfsys@transformshift{5.802187in}{1.301442in}%
\pgfsys@useobject{currentmarker}{}%
\end{pgfscope}%
\end{pgfscope}%
\begin{pgfscope}%
\pgfpathrectangle{\pgfqpoint{0.572918in}{0.553781in}}{\pgfqpoint{5.478282in}{2.095553in}}%
\pgfusepath{clip}%
\pgfsetrectcap%
\pgfsetroundjoin%
\pgfsetlinewidth{1.505625pt}%
\definecolor{currentstroke}{rgb}{0.313725,0.317647,0.309804}%
\pgfsetstrokecolor{currentstroke}%
\pgfsetstrokeopacity{0.900000}%
\pgfsetdash{}{0pt}%
\pgfpathmoveto{\pgfqpoint{0.821931in}{2.077541in}}%
\pgfpathlineto{\pgfqpoint{1.114887in}{2.007520in}}%
\pgfpathlineto{\pgfqpoint{1.407844in}{1.872645in}}%
\pgfpathlineto{\pgfqpoint{1.700800in}{1.886964in}}%
\pgfpathlineto{\pgfqpoint{1.993756in}{1.969395in}}%
\pgfpathlineto{\pgfqpoint{2.286712in}{1.927608in}}%
\pgfpathlineto{\pgfqpoint{2.579669in}{1.810341in}}%
\pgfpathlineto{\pgfqpoint{2.872625in}{1.726323in}}%
\pgfpathlineto{\pgfqpoint{3.165581in}{1.546802in}}%
\pgfpathlineto{\pgfqpoint{3.458537in}{1.397805in}}%
\pgfpathlineto{\pgfqpoint{3.751494in}{1.271543in}}%
\pgfpathlineto{\pgfqpoint{4.044450in}{1.121208in}}%
\pgfpathlineto{\pgfqpoint{4.337406in}{0.978769in}}%
\pgfpathlineto{\pgfqpoint{4.630362in}{0.822131in}}%
\pgfpathlineto{\pgfqpoint{4.923318in}{0.759205in}}%
\pgfpathlineto{\pgfqpoint{5.216275in}{0.746905in}}%
\pgfpathlineto{\pgfqpoint{5.509231in}{0.723972in}}%
\pgfpathlineto{\pgfqpoint{5.802187in}{0.851249in}}%
\pgfusepath{stroke}%
\end{pgfscope}%
\begin{pgfscope}%
\pgfpathrectangle{\pgfqpoint{0.572918in}{0.553781in}}{\pgfqpoint{5.478282in}{2.095553in}}%
\pgfusepath{clip}%
\pgfsetbuttcap%
\pgfsetroundjoin%
\pgfsetlinewidth{1.505625pt}%
\definecolor{currentstroke}{rgb}{0.949020,0.372549,0.360784}%
\pgfsetstrokecolor{currentstroke}%
\pgfsetstrokeopacity{0.900000}%
\pgfsetdash{{1.500000pt}{2.475000pt}}{0.000000pt}%
\pgfpathmoveto{\pgfqpoint{0.821931in}{2.228294in}}%
\pgfpathlineto{\pgfqpoint{1.114887in}{2.026779in}}%
\pgfpathlineto{\pgfqpoint{1.407844in}{2.007644in}}%
\pgfpathlineto{\pgfqpoint{1.700800in}{2.107648in}}%
\pgfpathlineto{\pgfqpoint{1.993756in}{2.161632in}}%
\pgfpathlineto{\pgfqpoint{2.286712in}{2.179738in}}%
\pgfpathlineto{\pgfqpoint{2.579669in}{2.020901in}}%
\pgfpathlineto{\pgfqpoint{2.872625in}{1.919151in}}%
\pgfpathlineto{\pgfqpoint{3.165581in}{1.752775in}}%
\pgfpathlineto{\pgfqpoint{3.458537in}{1.620302in}}%
\pgfpathlineto{\pgfqpoint{3.751494in}{1.520440in}}%
\pgfpathlineto{\pgfqpoint{4.044450in}{1.468148in}}%
\pgfpathlineto{\pgfqpoint{4.337406in}{1.406309in}}%
\pgfpathlineto{\pgfqpoint{4.630362in}{1.313032in}}%
\pgfpathlineto{\pgfqpoint{4.923318in}{1.255081in}}%
\pgfpathlineto{\pgfqpoint{5.216275in}{1.244156in}}%
\pgfpathlineto{\pgfqpoint{5.509231in}{1.141299in}}%
\pgfpathlineto{\pgfqpoint{5.802187in}{1.190557in}}%
\pgfusepath{stroke}%
\end{pgfscope}%
\begin{pgfscope}%
\pgfsetrectcap%
\pgfsetmiterjoin%
\pgfsetlinewidth{0.803000pt}%
\definecolor{currentstroke}{rgb}{0.000000,0.000000,0.000000}%
\pgfsetstrokecolor{currentstroke}%
\pgfsetdash{}{0pt}%
\pgfpathmoveto{\pgfqpoint{0.572918in}{0.553781in}}%
\pgfpathlineto{\pgfqpoint{0.572918in}{2.649333in}}%
\pgfusepath{stroke}%
\end{pgfscope}%
\begin{pgfscope}%
\pgfsetrectcap%
\pgfsetmiterjoin%
\pgfsetlinewidth{0.803000pt}%
\definecolor{currentstroke}{rgb}{0.000000,0.000000,0.000000}%
\pgfsetstrokecolor{currentstroke}%
\pgfsetdash{}{0pt}%
\pgfpathmoveto{\pgfqpoint{6.051200in}{0.553781in}}%
\pgfpathlineto{\pgfqpoint{6.051200in}{2.649333in}}%
\pgfusepath{stroke}%
\end{pgfscope}%
\begin{pgfscope}%
\pgfsetrectcap%
\pgfsetmiterjoin%
\pgfsetlinewidth{0.803000pt}%
\definecolor{currentstroke}{rgb}{0.000000,0.000000,0.000000}%
\pgfsetstrokecolor{currentstroke}%
\pgfsetdash{}{0pt}%
\pgfpathmoveto{\pgfqpoint{0.572918in}{0.553781in}}%
\pgfpathlineto{\pgfqpoint{6.051200in}{0.553781in}}%
\pgfusepath{stroke}%
\end{pgfscope}%
\begin{pgfscope}%
\pgfsetrectcap%
\pgfsetmiterjoin%
\pgfsetlinewidth{0.803000pt}%
\definecolor{currentstroke}{rgb}{0.000000,0.000000,0.000000}%
\pgfsetstrokecolor{currentstroke}%
\pgfsetdash{}{0pt}%
\pgfpathmoveto{\pgfqpoint{0.572918in}{2.649333in}}%
\pgfpathlineto{\pgfqpoint{6.051200in}{2.649333in}}%
\pgfusepath{stroke}%
\end{pgfscope}%
\begin{pgfscope}%
\definecolor{textcolor}{rgb}{0.000000,0.000000,0.000000}%
\pgfsetstrokecolor{textcolor}%
\pgfsetfillcolor{textcolor}%
\pgftext[x=0.572918in,y=2.732667in,left,base]{\color{textcolor}\rmfamily\fontsize{12.000000}{14.400000}\selectfont Zenith performance}%
\end{pgfscope}%
\begin{pgfscope}%
\pgfsetbuttcap%
\pgfsetmiterjoin%
\definecolor{currentfill}{rgb}{1.000000,1.000000,1.000000}%
\pgfsetfillcolor{currentfill}%
\pgfsetfillopacity{0.800000}%
\pgfsetlinewidth{1.003750pt}%
\definecolor{currentstroke}{rgb}{0.800000,0.800000,0.800000}%
\pgfsetstrokecolor{currentstroke}%
\pgfsetstrokeopacity{0.800000}%
\pgfsetdash{}{0pt}%
\pgfpathmoveto{\pgfqpoint{5.165644in}{2.250667in}}%
\pgfpathlineto{\pgfqpoint{5.973422in}{2.250667in}}%
\pgfpathquadraticcurveto{\pgfqpoint{5.995644in}{2.250667in}}{\pgfqpoint{5.995644in}{2.272889in}}%
\pgfpathlineto{\pgfqpoint{5.995644in}{2.571556in}}%
\pgfpathquadraticcurveto{\pgfqpoint{5.995644in}{2.593778in}}{\pgfqpoint{5.973422in}{2.593778in}}%
\pgfpathlineto{\pgfqpoint{5.165644in}{2.593778in}}%
\pgfpathquadraticcurveto{\pgfqpoint{5.143422in}{2.593778in}}{\pgfqpoint{5.143422in}{2.571556in}}%
\pgfpathlineto{\pgfqpoint{5.143422in}{2.272889in}}%
\pgfpathquadraticcurveto{\pgfqpoint{5.143422in}{2.250667in}}{\pgfqpoint{5.165644in}{2.250667in}}%
\pgfpathclose%
\pgfusepath{stroke,fill}%
\end{pgfscope}%
\begin{pgfscope}%
\pgfsetbuttcap%
\pgfsetroundjoin%
\pgfsetlinewidth{1.505625pt}%
\definecolor{currentstroke}{rgb}{0.313725,0.317647,0.309804}%
\pgfsetstrokecolor{currentstroke}%
\pgfsetstrokeopacity{0.900000}%
\pgfsetdash{}{0pt}%
\pgfpathmoveto{\pgfqpoint{5.298978in}{2.454889in}}%
\pgfpathlineto{\pgfqpoint{5.298978in}{2.566000in}}%
\pgfusepath{stroke}%
\end{pgfscope}%
\begin{pgfscope}%
\pgfsetbuttcap%
\pgfsetroundjoin%
\definecolor{currentfill}{rgb}{0.313725,0.317647,0.309804}%
\pgfsetfillcolor{currentfill}%
\pgfsetfillopacity{0.900000}%
\pgfsetlinewidth{1.003750pt}%
\definecolor{currentstroke}{rgb}{0.313725,0.317647,0.309804}%
\pgfsetstrokecolor{currentstroke}%
\pgfsetstrokeopacity{0.900000}%
\pgfsetdash{}{0pt}%
\pgfsys@defobject{currentmarker}{\pgfqpoint{-0.013889in}{-0.000000in}}{\pgfqpoint{0.013889in}{0.000000in}}{%
\pgfpathmoveto{\pgfqpoint{0.013889in}{-0.000000in}}%
\pgfpathlineto{\pgfqpoint{-0.013889in}{0.000000in}}%
\pgfusepath{stroke,fill}%
}%
\begin{pgfscope}%
\pgfsys@transformshift{5.298978in}{2.454889in}%
\pgfsys@useobject{currentmarker}{}%
\end{pgfscope}%
\end{pgfscope}%
\begin{pgfscope}%
\pgfsetbuttcap%
\pgfsetroundjoin%
\definecolor{currentfill}{rgb}{0.313725,0.317647,0.309804}%
\pgfsetfillcolor{currentfill}%
\pgfsetfillopacity{0.900000}%
\pgfsetlinewidth{1.003750pt}%
\definecolor{currentstroke}{rgb}{0.313725,0.317647,0.309804}%
\pgfsetstrokecolor{currentstroke}%
\pgfsetstrokeopacity{0.900000}%
\pgfsetdash{}{0pt}%
\pgfsys@defobject{currentmarker}{\pgfqpoint{-0.013889in}{-0.000000in}}{\pgfqpoint{0.013889in}{0.000000in}}{%
\pgfpathmoveto{\pgfqpoint{0.013889in}{-0.000000in}}%
\pgfpathlineto{\pgfqpoint{-0.013889in}{0.000000in}}%
\pgfusepath{stroke,fill}%
}%
\begin{pgfscope}%
\pgfsys@transformshift{5.298978in}{2.566000in}%
\pgfsys@useobject{currentmarker}{}%
\end{pgfscope}%
\end{pgfscope}%
\begin{pgfscope}%
\pgfsetrectcap%
\pgfsetroundjoin%
\pgfsetlinewidth{1.505625pt}%
\definecolor{currentstroke}{rgb}{0.313725,0.317647,0.309804}%
\pgfsetstrokecolor{currentstroke}%
\pgfsetstrokeopacity{0.900000}%
\pgfsetdash{}{0pt}%
\pgfpathmoveto{\pgfqpoint{5.187867in}{2.510444in}}%
\pgfpathlineto{\pgfqpoint{5.410089in}{2.510444in}}%
\pgfusepath{stroke}%
\end{pgfscope}%
\begin{pgfscope}%
\definecolor{textcolor}{rgb}{0.000000,0.000000,0.000000}%
\pgfsetstrokecolor{textcolor}%
\pgfsetfillcolor{textcolor}%
\pgftext[x=5.498978in,y=2.471556in,left,base]{\color{textcolor}\rmfamily\fontsize{8.000000}{9.600000}\selectfont CNN 1.0}%
\end{pgfscope}%
\begin{pgfscope}%
\pgfsetbuttcap%
\pgfsetroundjoin%
\pgfsetlinewidth{1.505625pt}%
\definecolor{currentstroke}{rgb}{0.949020,0.372549,0.360784}%
\pgfsetstrokecolor{currentstroke}%
\pgfsetstrokeopacity{0.900000}%
\pgfsetdash{}{0pt}%
\pgfpathmoveto{\pgfqpoint{5.298978in}{2.300000in}}%
\pgfpathlineto{\pgfqpoint{5.298978in}{2.411111in}}%
\pgfusepath{stroke}%
\end{pgfscope}%
\begin{pgfscope}%
\pgfsetbuttcap%
\pgfsetroundjoin%
\definecolor{currentfill}{rgb}{0.949020,0.372549,0.360784}%
\pgfsetfillcolor{currentfill}%
\pgfsetfillopacity{0.900000}%
\pgfsetlinewidth{1.003750pt}%
\definecolor{currentstroke}{rgb}{0.949020,0.372549,0.360784}%
\pgfsetstrokecolor{currentstroke}%
\pgfsetstrokeopacity{0.900000}%
\pgfsetdash{}{0pt}%
\pgfsys@defobject{currentmarker}{\pgfqpoint{-0.013889in}{-0.000000in}}{\pgfqpoint{0.013889in}{0.000000in}}{%
\pgfpathmoveto{\pgfqpoint{0.013889in}{-0.000000in}}%
\pgfpathlineto{\pgfqpoint{-0.013889in}{0.000000in}}%
\pgfusepath{stroke,fill}%
}%
\begin{pgfscope}%
\pgfsys@transformshift{5.298978in}{2.300000in}%
\pgfsys@useobject{currentmarker}{}%
\end{pgfscope}%
\end{pgfscope}%
\begin{pgfscope}%
\pgfsetbuttcap%
\pgfsetroundjoin%
\definecolor{currentfill}{rgb}{0.949020,0.372549,0.360784}%
\pgfsetfillcolor{currentfill}%
\pgfsetfillopacity{0.900000}%
\pgfsetlinewidth{1.003750pt}%
\definecolor{currentstroke}{rgb}{0.949020,0.372549,0.360784}%
\pgfsetstrokecolor{currentstroke}%
\pgfsetstrokeopacity{0.900000}%
\pgfsetdash{}{0pt}%
\pgfsys@defobject{currentmarker}{\pgfqpoint{-0.013889in}{-0.000000in}}{\pgfqpoint{0.013889in}{0.000000in}}{%
\pgfpathmoveto{\pgfqpoint{0.013889in}{-0.000000in}}%
\pgfpathlineto{\pgfqpoint{-0.013889in}{0.000000in}}%
\pgfusepath{stroke,fill}%
}%
\begin{pgfscope}%
\pgfsys@transformshift{5.298978in}{2.411111in}%
\pgfsys@useobject{currentmarker}{}%
\end{pgfscope}%
\end{pgfscope}%
\begin{pgfscope}%
\pgfsetbuttcap%
\pgfsetroundjoin%
\pgfsetlinewidth{1.505625pt}%
\definecolor{currentstroke}{rgb}{0.949020,0.372549,0.360784}%
\pgfsetstrokecolor{currentstroke}%
\pgfsetstrokeopacity{0.900000}%
\pgfsetdash{{1.500000pt}{2.475000pt}}{0.000000pt}%
\pgfpathmoveto{\pgfqpoint{5.187867in}{2.355556in}}%
\pgfpathlineto{\pgfqpoint{5.410089in}{2.355556in}}%
\pgfusepath{stroke}%
\end{pgfscope}%
\begin{pgfscope}%
\definecolor{textcolor}{rgb}{0.000000,0.000000,0.000000}%
\pgfsetstrokecolor{textcolor}%
\pgfsetfillcolor{textcolor}%
\pgftext[x=5.498978in,y=2.316667in,left,base]{\color{textcolor}\rmfamily\fontsize{8.000000}{9.600000}\selectfont CNN 2.0}%
\end{pgfscope}%
\end{pgfpicture}%
\makeatother%
\endgroup%

    \caption{Performance of four different networks.
    The low-parameter TCN 1.0 performs promising vs. CNNs with orders of magnitude more parameters, but fails at higher energies; this is remedied by TCN 2.0, which trades low amount of parameters for performance.}\label{fig:models}
\end{figure}

The preceding discussion holds as well for zenith prediction as for energy prediction (bar prediction type and input type, which are not energy relevant).
The same goes for the design of the network; where performance increased on zenith prediction, performance increased on energy prediction.

Several designs have been tested out, both of the CNN and TCN variety.
The first working version, \enquote{CNN 1.0}, consists of five convolutional layers (kernel size 5), each doubling the number of filters starting with 32, using ReLUs, max pool and batch normalization in each block.
This is then fed through four fully connected layers until terminating in an output layer.

The network was found to much improve by removing most convolutional layers, with \enquote{CNN 2.0} only containing two of these, with two fully connected layers afterwards.
The difference in performance, seen in~\vref{fig:models}, is rather striking everywhere but the lowest energy bins.

CNN 1.0 is a large network, containing \num{5735009} trainable parameters, cut down to \num{1718241} in CNN 2.0.

\enquote{TCN 1.0} is a temporal convolutional network with one stack of 64 filters (kernel size 2), dilations (1, 2, 4, 8, 16, 32), ReLUs and causal padding.
It only has \num{92161} trainable parameters, yet performs as well as CNN 2.0 until around \SI{100}{\giga\electronvolt}.
Upping the filters to 256, the number of stacks to 2, and adding batch normalization results in \num{3038209} trainable parameters for \enquote{TCN 2.0}\footnote{More---many more!---network designs were trialed than two CNNs and two TCNs; for reasons of brevity, they are not discussed here.}, the winning network of the bunch.

These models are compared visually in~\vref{fig:models}.
With the best network---architecture and hyperparameters---in hand, the next task is to compare it to IceCube's current reconstruction algorithm, Retro Reco.

\end{document}
