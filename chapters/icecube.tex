\documentclass[../main.tex]{subfiles}

\begin{document}

IceCube's raison d'être is the study of neutrinos, its singular tool the conversion of light pulses into particle properties.
This is not an easy task, inasmuch as nature rarely exhibits the simplicity of its governing equations.
IceCube consists of two primary detectors, IceCube and DeepCore, with a third, \enquote{the Upgrade}, expected to be installed in the next few years.

\section{Detector}\label{sec:detector}

As we have seen, the neutrino only participates weakly in the Standard Model.
This makes the neutrino a very elusive particle, able to generally travel very far before interacting.
The rate of neutrino interactions is a function of the flux \( \phi \), the cross section \( \sigma \),  and the number of target particles \( N_{\text{target}} \), i.e.
\begin{equation}
	N_{\Pnu} \propto \phi \times \sigma \times N_{\text{target}},
\end{equation}
which means that if your flux is \SI{2e38}{\per\second}, and the cross section \SI{1e-44}{\centi\meter\squared}, you, to paraphrase Roy Schneider, are gonna need a bigger boat.
But if one thing is a big enough detector, containing a sufficient number of target particles, another is background noise: you generally want to ensure that the only thing interacting within your detector volume is by-products of neutrinos.
In many cases these secondary particles are muons, but those are plentiful in nature from other sources, and your detector will thus require screening from the outside world.

There are generally two ways of solving this.
One is to build a container and fill it up with whatever substance suits the purpose, and then finding somewhere to stow it away such that it is shielded from outside charged leptons, typically done by positioning the container inside an abandoned mine deep underground.
The other way is to utilise nature to its fullest, and find somewhere that both provides the interaction substance \textit{and} shields from unwanted particles.

The Homestake experiment, mentioned in \vref{sec:neutrino_oscillations}, made use of the first type of detector in the late 1960s.
Many others---such as the Kamioka Observatory in Japan, and the Sudbury Neutrino Observatory in Canada---followed this path, but a race of sorts was soon on to build a detector of the second variety; detailed in~\autocite{Bowen2017}.
Starting with DUMAND in the 70s, one set of scientists were focused on water Cherenkov detectors.
In this case on suspends strings of detectors in deep water to observe Cherenkov radiation by charged particles travelling faster than \( c \) through the water.
DUMAND was never successfully completed, but another team---led by Francis Halzen---was inspired to use ice instead of water, and in 1996, a year after DUMAND's cancellation, set up AMANDA on the South Pole.
When this proved a succes, funding for the IceCube Neutrino Observatory was approved, and construction completed in 2010.

The IceCube instigators recognised that the Antarctic ice provides an excellent source of neutrino interaction material, and the Earth provides a near perfect shield from background muons.
\begin{figure}
     \centering
     \begin{subfigure}[b]{0.49\textwidth}
         \centering
		    \includegraphics[width=\textwidth]{./images/icecube/Icecube-architecture-diagram2009.PNG}
		    \caption{Image from \url{https://commons.wikimedia.org/wiki/File:Icecube-architecture-diagram2009.PNG}}\label{fig:icecube_detector}
     \end{subfigure}
     \hfill
     \begin{subfigure}[b]{0.4\textwidth}
         \centering
		    \includegraphics[width=\textwidth]{./images/icecube/e88_1.png}
		    \caption{Image from \url{https://physics.aps.org/articles/v7/88}}\label{fig:icecube_event_1}
     \end{subfigure}
        \caption{\Vref{fig:icecube_detector} shows an artist rendition of the IceCube detector as a scale comparison to the Eiffel Tower. \Vref{fig:icecube_event_1} shows a neutrino event in IceCube. This is a high energy track-like event, for which angular reconstruction is possible even by eye.}
        \label{fig:icecube}
\end{figure}
The detector is built by using hot water to drill holes in the Antarctic ice \SI{2.5}{\kilo\meter} deep, and lowering down long strings on which are situated the actual Cherenkov light detectors.
Each string consists of 60 of these detectors, called digital optical modules (DOMs), and each DOM consists of assorted electronics, a photomultiplier tube that does the actual light detection, and shielding from the environment.
The DOM is further detailed in \vref{sec:doms}.
\begin{figure}
     \centering
     \begin{subfigure}[b]{0.45\textwidth}
         \centering
		    \includegraphics[width=\textwidth]{./images/icecube/10052_2018_6369_Fig3_HTML.png}
		    \caption{}\label{fig:ray_types}
     \end{subfigure}
     \hfill
     \begin{subfigure}[b]{0.45\textwidth}
         \centering
		    \includegraphics[width=\textwidth]{./images/icecube/Figs_all-nu-spectrum-mod.png}
		    \caption{}\label{fig:neutrino_spectrum}
     \end{subfigure}
        \caption{\Vref{fig:ray_types}~\autocite{Ahlers2018} shows the types of rays relevant to IceCube, defining up- and down-going muons. Neutrinos---created by atmospheric interaction with cosmic rays or in cosmic processes---interact either in the atmosphere (creating background muons), or in the Earth (creating up-going muons). \Vref{fig:neutrino_spectrum}~\autocite{Spiering2012} shows the energy spectrum of neutrinos from different sources. Atmospheric neutrinos dominate in the energy region supported by IceCube.}
        \label{fig:rays_and_spectrum}
\end{figure}
IceCube is split into the in-ice array, i.e. the stringed DOMs, and an IceTop array.
This consists of detectors on the surface, housing water Cherenkov detectors that may act as a useful veto on particles: if they were seen by IceTop, they were probably down-going particles, and thus noise, while the neutrino events present as up-going particles created via interactions in the Earth, where only neutrinos penetrate; see \vref{fig:ray_types}.
The detector is tuned to particles in the \si{\giga\electronvolt} to \si{\peta\electronvolt} range, where atmospheric neutrinos dominate, as seen in \vref{fig:neutrino_spectrum}.

IceCube records data based on certain triggers (\vref{sec:triggers}), and sends filtered data to the Northern Hemisphere by satellite (about \SI{100}{\giga\byte\per\day}~\autocite{Kelley2014}), with the raw data sent by sneakernet\footnote{That is, data transported by sneakers, i.e. shipped via physical media rather than the internet.} (about \SI{1}{\tera\byte\per\day}~\autocite{Kelley2014}).
The surface also hosts the IceCube Lab, containing supporting electronics such as computers and communication systems.
The whole detector is located near, and supported by, the Amundsen-Scott South Pole station, situated at the Geographic South Pole and administered by the US National Science Foundation.
\begin{figure}[ht]
    \centering
    \includegraphics[width=0.7\textwidth]{./images/icecube/cherenkov_icecube.png}
    \caption{Detection modes of the AMANDA detector (built following the same principles as IceCube). This makes clear that the Cherenkov light is what is actually observed, from a muon in the left figure, and a cascade in the right. The track-like signatures are detected from Cherenkov cones, while the cascades present as Cherenkov spheres. Image from~\autocite{Ahrens2004}.}\label{fig:cherenkov_icecube}
\end{figure}
The observatory detects Cherenkov light from secondary particles---visualised in \vref{fig:cherenkov_icecube}---from both charged- and neutral current neutrino interactions,
with the large volume of ice providing a plentiful basin of potentially interacting ice molecules.
In the case of neutral current interactions, the event signature is that of a cascade: the neutrino scatters of a charged lepton, escaping the detector again, with the deposited energy resulting in electromagnetic and hadronic cascades; hence the name.
These cascades can be contained fully within the detector, meaning the energy can be easily inferred, but it is difficult to reconstruct a track that points to the origin of the neutrino.
Charged current interactions, on the other hand, present the possibility of track-like signals, which make for a much better pointing resolution.
However, the flavour of the neutrino in this case determines the signature:
\begin{itemize}
	\item An electron neutrino will produce an electron, which will create an electromagnetic shower, as the electron instigates pair production, and is slowed down by bremsstrahlung. This looks like a cascade signature
	\item A tau lepton will live very shortly (on the order of \SI{1e-13}{\second}) owing to its massive nature, and decays again creating a cascade
	\item Muons are special: they do not decay immediately, as taus, and are less affected by the processes stopping electrons. They are thus able to leave long tracks in the detector, a track-like signature
\end{itemize}
All three flavours create a hadronic cascade at the interaction vertex, \( X \) in \( \Pnu_{\alpha} + N \rightarrow \Plepton_{\alpha} + X \) with \( N \) a nucleon, \( \Pnu_{\alpha} \) a neutrino and \( \Plepton_{\alpha} \) a lepton, both of flavour \( \alpha \).
\todo[inline]{Gotta explain e-m and hadronic cascades at some point}
In the case of a tau neutrino, there is the possibility of another (as of yet unseen) signature: the double-bang.
A tau will have a decay length of \SI{50}{\meter\per\peta\electronvolt}~\autocite{Ahlers2018}, so very high energy tau neutrinos may interact and first create a hadronic cascade, then a track from a high energy tau, which then decays \SI{50}{\meter} or more away.
\todo[inline]{Maybe expound on double-bangs and what they're good for? Maybe not.}
The signatures can be seen in ~\vref{fig:event_signatures}.
\begin{figure}[ht]
    \centering
    \includegraphics[width=\textwidth]{./images/icecube/Typical-event-signatures-observed-by-IceCube-a-track-like-event-left-a-showerlike.png}
    \caption{Three IceCube event signatures. The size of each coloured sphere is proportional to the charge of the pulse detected by the PMT, and the colour is related to the timing of the signal; red hits are earlier than blue. The left figure shows a track-like muon event, while the signature in the middle is a cascade. On the right is shown a double-bang event, where two cascades and a track is seen. The two left-most events are real, the double-bang is simulated. Image from~\autocite{Meagher2016}.}\label{fig:event_signatures}
\end{figure}

\section{DOMs}\label{sec:doms}

The DOMs are situated on \SI{2.5}{\kilo\meter} long strings, spaced \SI{17}{\meter} apart (\SI{7}{\meter} for DeepCore, \SI{2.4}{\meter} for Upgrade).
They are essentially in-ice, high quality digital cameras, capable of detecting single photons~\autocite{Aartsen2017}.
Being self-contained in the sense that all recording capabilities are on-device, and according to pre-set triggers (\vref{sec:triggers}), they will record and save pulses and transmit them via cabling to IceCube Lab on the surface.
When a photon is detected, the DOM starts a recording, lasting for up to \SI{6.4}{\micro\second}, such that later arriving photons are included in the \enquote{hit}~\autocite{Aartsen2017}.
This operation is independent, and as such each DOM does not know what other DOMs are seeing.
However, the PMT hit rate is recorded in millisecond intervals, because an overall increase in hits might indicate a cosmic event, which must be acted upon promptly, and a separate local coincidence wiring---connecting closest DOMs---ensures quick recognition of (near)simultaneous hits~\autocite{Aartsen2017}.

The PMT waveforms are digitised by two circuits, ATWD (Analog Transient Waveform Digitiser) and fADC (fast Analog to Digital Converter).
These differ in their sampling time, \SI{427}{\nano\second} for ATWD, \SI{6.4}{\micro\second} for fADC.
\todo[inline]{Expound on digitisers.}

The DOMs also contain \enquote{flasher boards} outfitted with LEDs, used for calibration purposes such as measuring their positions and the optical properties of the ice.

It should be noted that the DOMs are pretty reliable; by 2016 87 DOMs were not operational~\autocite{Aartsen2017}; this is rather fortuitous, inasmuch as they are non-repairable (at least they're impossible to retrieve from the ice).

The location of the detector, in the dark ice deep beneath the Antarctic, leads to the majority of noise in the PMTs being \enquote{dark noise}, that is quantum effects such as thermal emission, radioactive decays and the luminescence of the glass; these effects have no outside sources, but have a rate of \SIrange{560}{780}{\hertz}~\autocite{Aartsen2017}.
The background muons, for comparison, are at a rate of \SIrange{5}{25}{\hertz}~\autocite{Aartsen2017}.

Interestingly, the noise rate decreases after installation, as the DOMs \enquote{freeze in}~\autocite{Aartsen2017}.

As can be seen in \vref{fig:dom_schematic}, the entire southern hemisphere of the DOMs is taken up by the PMT.
Accordingly, all DOMs point downwards, a design choice changed in the Upgrade (\vref{sec:icecube_upgrade}).
Communication and power is handled by the cabling---clearly seen in \vref{fig:dom_pic}---, connecting all DOMs on a string, terminating on the surface in the IceCube Lab.

\begin{figure}
     \centering
     \begin{subfigure}[b]{0.4\textwidth}
         \centering
		    \includegraphics[width=\textwidth]{./images/icecube/ICECUBE_dom_taklampa.jpg}
		    \caption{Image from \url{https://commons.wikimedia.org/wiki/File:ICECUBE_dom_taklampa.jpg}}\label{fig:dom_pic}
     \end{subfigure}
     \hfill
     \begin{subfigure}[b]{0.4\textwidth}
         \centering
		    \includegraphics[width=\textwidth]{./images/icecube/5-DOM-Picture.png}
		    \caption{}\label{fig:dom_schematic}
     \end{subfigure}
        \caption{\Vref{fig:dom_pic} shows a picture of a DOM, \vref{fig:dom_schematic}~\autocite{Abbasi2009} shows a schematic of a DOM.}
        \label{fig:doms}
\end{figure}

\section{Triggers and on-pole reconstruction}\label{sec:triggers}

As any other particle detector, IceCube employs triggers to maintain the otherwise unmanageable data levels at a reasonable size.
The raw data rate, after triggers have been applied, is \SI{1}{\tera\byte\per\day}, which is whittled down to around \SI{100}{\giga\byte} by event selections performed by the on-Pole servers, appropriate for transfer via satellite to the Northern Hemisphere~\autocite{Kelley2014}, where events can be analysed further.
There are several triggers at work, with different functionalities in mind; one trigger, the \enquote{Single Multiplicity Trigger} (SMT), uses the local coincidence wiring and acts when \( N \) or more locally coincident hits (LC) are detected within a sliding window \( n \) \si{\micro\second}, extended until the local coincidence condition is no longer satisfied.
With \( N = 8 \) and \( n = 5 \) the hit rate of this particular trigger is \SI{2100}{\hertz}~\autocite{Kelley2014}.
Another is the string trigger, looking for a certain number of hits along a string, useful for slow-moving, completely up-going, muons.
In general, the triggers rely on the speed of light in ice to figure out whether hits are causally connected, or appear to be noise.
Additionally, because an event may fulfil the conditions set out in several triggers, hit-merging is performed, ensuring that unique hits are not present in several events.
On the other extreme, events can be coincident, needing algorithms to split events from each other.

When triggers have been activated, detector data is sent to IceCube Lab, which uses on-premise servers to perform fast event selection and preliminary reconstruction to determine a) if a muon is up- or down-going and b) whether several muons are present in one event.
The reconstruction is important as a filter, because the largest source of noise in the detector comes from background atmospheric muons.
Using earth-filtering is the easiest way of determining whether a muon is atmospheric or caused by a neutrino, as the ground stops all of the former, but allows passage of the latter, but this does present a challenge: if one wants to determine (approximately) whence a muon came, one needs to assemble the disparate hits in the detector into something resembling a line.
As several thousand events are read out each second, the algorithm must be fast, and fast on the available hardware (which tends not to be replaced often, seeing as how it is situated at one of the hardest to reach locations on Earth).
The servers at the IceCube Lab perform first a quick reconstruction, \enquote{linefit}~\autocite{Aartsen2014}---a minimizer using a Huber penalty---used as a seed for the \enquote{Single-Photo-Electron-Fit}, a maximum likelihood estimation~\autocite{Ahrens2004}.
This reconstruction has a median angular resolution (the arc-distance between reconstruction and the true track) on Monte Carlo (MC, \vref{sec:monte_carlo_simulation}) data of \( \theta_{\text{med}} = \SI{4.211}{\degree} \)~\autocite{Aartsen2014}.

\section{Post processing}\label{sec:post_processing}

%\section{Levels}\label{sec:levels}

When data arrives from the Pole, different analysis groups apply different post-processing measures.
In the following we shall focus on the methods applied by the oscillation working group, a suite of processing software termed \enquote{oscNext}, documented in~\autocite{Blot2020}.
Six levels are used, each removing different types of noise and muons using several different methods (some of which are machine learning based).
The muon and noise is reduced by a factor \( \mathcal{O}(10^{5}) \) by the post-processing, and at the final level a neutrino rate of \( \mathcal{O}(\SI{1}{\milli\hertz}) \) survives with a few \% background muons and almost no noise.

The post-processing levels are shortly touched on in the following.

\subsection{L2}\label{sec:l2}

L2 is run on files received from the Pole.
It first passes finds events passing the SMT3 trigger, meaning the SMT described in \vref{sec:triggers} with \( N = 3 \).
Next, a Seeded Radius-Time (SRT) cleaning algorithm is run, which relies on the fact that Cherenkov photons will be clustered in space and time, while noise is random.
The SRT algorithm uses this on the LC hits, which are used as seeds, with non-LC hits outside some pre-set space-time volume discarded, while those inside are kept and used as seeds for further iteration.
Lastly, a DeepCore fiducial volume, and a DeepCore veto volume are defined, and using a center of gravity calculation using hits in the fiducial volume, hits from the veto region are discarded based on their derived velocity, on the basis that such a hit might be related to a crossing muon.

\subsection{L3}\label{sec:l3}

At level 3, a number of cuts are made on several derived variables, described in~\autocite{Blot2020}.
The main point of this level is to bring the data into agreement with MC, the discrepancies mainly stemming from processes for which there is no simulation.
\todo[inline]{More about L3}

L3 removes around \SI{95}{\percent} of the background muon events, \SI{99}{\percent} of pure noise and keeps \SI{60}{\percent} of atmospheric neutrino events relative to level 2, such that the event rate is below \SI{1}{\hertz}~\autocite{Blot2020}.

\subsection{L4}\label{sec:l4}

\section{Current reconstruction methods}\label{sec:current_reconstruction_methods}

\section{I3 file format}\label{sec:i3_file_format}

\section{DeepCore}\label{sec:deepcore}

\section{IceCube upgrade}\label{sec:icecube_upgrade}

\begin{figure}
     \centering
     \begin{subfigure}[b]{0.4\textwidth}
         \centering
		    \includegraphics[width=\textwidth]{./images/icecube/UpG_OpticalSensors_SensorsOnly_v2.png}
		    \caption{}\label{fig:upgrade_all_doms}
     \end{subfigure}
     \hfill
     \begin{subfigure}[b]{0.4\textwidth}
         \centering
		    \includegraphics[width=0.5\textwidth]{./images/icecube/mDOM_exploded_view-2.pdf.png}
		    \caption{}\label{fig:mdom_schematic}
     \end{subfigure}
        \caption{\Vref{fig:upgrade_all_doms} shows DOMs installed with the IceCube upgrade, \vref{fig:mdom_schematic}~\autocite{Classen2018} shows an exploded view of an mDOM.
        \Vref{fig:upgrade_all_doms} is from the IceCube Collaboration, fetched from \url{https://icecube.wisc.edu/gallery/press/view/2328}}\label{fig:upgrade_doms}
\end{figure}

\begin{figure}
	\begin{subfigure}{\linewidth}
		\centering
		\includegraphics[width=0.5\linewidth]{./images/icecube/UpG_OpticalSensors_SensorsOnly_v2.png}
		\caption{}\label{fig:upgrade_all_doms}
	\end{subfigure}\\[1ex]
	\begin{subfigure}{\linewidth}
		\centering
		\includegraphics[width=0.5\linewidth]{./images/icecube/comparison_updated.JPG}
		\caption{}\label{fig:improved_event_upgrade}
	\end{subfigure}
	\caption{\Vref{fig:upgrade_all_doms} shows new DOM types installed with the IceCube upgrade. \Vref{fig:improved_event_upgrade} shows a simulated \SI{3.8}{\giga\electronvolt} track-like event and its Cherenkov cone. More DOMs light up in the Upgrade configuration, giving more information for track reconstruction. Figures are from the IceCube Collaboration, \url{https://icecube.wisc.edu/gallery/press/view/2328}}\label{fig:upgrade_doms}
\end{figure}

\begin{figure}
     \centering
     \begin{subfigure}[b]{0.4\textwidth}
         \centering
		    \includegraphics[width=\textwidth]{./images/icecube/4-VetoView2_labeledStringswIcetops.jpg}
		    \caption{Image from \url{https://icecube.wisc.edu/gallery/view/140}}\label{fig:icecube_string_pattern}
     \end{subfigure}
     \hfill
     \begin{subfigure}[b]{0.4\textwidth}
         \centering
		    \includegraphics[width=\textwidth]{./images/icecube/Upgrade_Spacing.jpg}
		    \caption{Image from \url{https://icecube.wisc.edu/news/view/605}}\label{fig:upgrade_string_pattern}
     \end{subfigure}
        \caption{\Vref{fig:icecube_string_pattern} shows the string pattern of IceCube and DeepCore, \vref{fig:upgrade_string_pattern} shows the proposed upgrade.}
        \label{fig:string_patterns}
\end{figure}

\begin{figure}[ht]
    \centering
    \includegraphics[width=\textwidth]{./images/icecube/DGV3K.png}
    \caption{Stopping power}\label{fig:stopping_power}
\end{figure}

\begin{figure}[ht]
    \centering
    \includegraphics[width=\textwidth]{./images/icecube/6-Figure1-1.png}
    \caption{Glashow resonance. Image from \url{https://www.semanticscholar.org/paper/Glashow-resonance-as-a-window-into-cosmic-neutrino-Barger-Fu/9922dfd291f92a97219a0abbee9c0303ee9fbb36}}\label{fig:glashow_resonance}
\end{figure}

\end{document}
