%!TEX root = ../main.tex
%!TEX spellcheck en-us
\documentclass[../main.tex]{subfiles}

\begin{document}

One of IceCube's raison d'êtres is the study of neutrinos, its primary tool the conversion of light pulses into particle properties.
This is not an easy task, inasmuch as nature rarely exhibits the simplicity of its governing equations.
IceCube consists of two primary detectors, IceCube and DeepCore, with a third, \enquote{the Upgrade}, expected to be installed in the next few years.

\section{Detector}\label{sec:detector}

As we have seen, the neutrino only participates weakly in the Standard Model.
This makes the neutrino a very elusive particle, able to generally travel very far before interacting.
The rate of neutrino interactions is a function of the flux \( \phi \), the cross section \( \sigma \),  and the number of target particles \( N_{\text{target}} \), i.e.
\begin{equation}
	N_{\Pnu} \propto \phi \times \sigma \times N_{\text{target}},
\end{equation}
which means that if your flux is \SI{2e38}{\per\second}, and the cross section \SI{1e-44}{\centi\meter\squared}, you, to paraphrase Roy Schneider, are gonna need a bigger boat.
But if one thing is a big enough detector, containing a sufficient number of target particles, another is background noise: you generally want to ensure that the only thing interacting within your detector volume is by-products of neutrinos.
In many cases these secondary particles are muons, but those are plentiful in nature from other sources, and your detector will thus require screening from the outside world.

There are generally two ways of solving this.
One is to build a container and fill it up with whatever substance suits the purpose, and then finding somewhere to stow it away such that it is shielded from outside charged leptons, typically done by positioning the container inside an abandoned mine deep underground.
The other way is to utilize nature to its fullest, and find somewhere that both provides the interaction substance \textit{and} shields from unwanted particles.

The Homestake experiment, mentioned in~\vref{sec:neutrino_oscillations}, made use of the first type of detector in the late 1960s.
Many others---such as the Kamioka Observatory in Japan, and the Sudbury Neutrino Observatory in Canada---followed this path, but a race of sorts was soon on to build a detector of the second variety; detailed in~\cite{Bowen2017}.
Starting with DUMAND in the 70s, one set of scientists were focused on water Cherenkov detectors.
In this case one suspends strings of detectors in deep water to observe Cherenkov radiation by charged particles travelling faster than \( c \) through the water.
DUMAND was never successfully completed, but another team---led by Francis Halzen---was inspired to use ice instead of water, and in 1996, a year after DUMAND's cancellation, set up AMANDA on the South Pole.
When this proved a success, funding for the IceCube Neutrino Observatory was approved, and construction completed in 2010.

The IceCube instigators recognized that the Antarctic ice provides an excellent source of neutrino interaction material, and the Earth provides a near perfect shield from background muons.
\begin{figure}
     \centering
     \begin{subfigure}[b]{0.49\textwidth}
         \centering
		    \includegraphics[width=\textwidth]{./images/icecube/Icecube-architecture-diagram2009.PNG}
		    \caption{}\label{fig:icecube_detector}
     \end{subfigure}
     \hfill
     \begin{subfigure}[b]{0.4\textwidth}
         \centering
		    \includegraphics[width=\textwidth]{./images/icecube/e88_1.png}
		    \caption{}\label{fig:icecube_event_1}
     \end{subfigure}
        \caption{(a), from~\cite{WikiIcecube1}, shows an artist rendition of the IceCube detector as a scale comparison to the Eiffel Tower.
        (b), from~\cite{IcecubeNeutrinoSpectrum}, shows a neutrino event in IceCube.
        This is a high energy track-like event, for which angular reconstruction is possible even by eye.}\label{fig:icecube}
\end{figure}
The detector is built by using hot water to drill holes in the Antarctic ice \SI{2.5}{\kilo\meter} deep, and lowering down long strings on which are situated the actual Cherenkov light detectors.
Each string consists of 60 of these detectors, called digital optical modules (DOMs), and each DOM consists of assorted electronics, a photomultiplier tube that does the actual light detection, and shielding from the environment.
The DOM is further detailed in~\vref{sec:doms}.
\begin{figure}
     \centering
     \begin{subfigure}[b]{0.45\textwidth}
         \centering
		    \includegraphics[width=\textwidth]{./images/icecube/10052_2018_6369_Fig3_HTML.png}
		    \caption{}\label{fig:ray_types}
     \end{subfigure}
     \hfill
     \begin{subfigure}[b]{0.45\textwidth}
         \centering
		    \includegraphics[width=\textwidth]{./images/icecube/Figs_all-nu-spectrum-mod.png}
		    \caption{}\label{fig:neutrino_spectrum}
     \end{subfigure}
        \caption{\Vref{fig:ray_types}~\cite{Ahlers2018} shows the types of rays relevant to IceCube, defining up- and down-going muons. Neutrinos---created by atmospheric interaction with cosmic rays or in cosmic processes---interact either in the atmosphere (creating background muons), or in the Earth (creating up-going muons). \Vref{fig:neutrino_spectrum}~\cite{Spiering2012} shows the energy spectrum of neutrinos from different sources. Atmospheric neutrinos dominate in the energy region supported by IceCube.}\label{fig:rays_and_spectrum}
\end{figure}
IceCube is split into the in-ice array, i.e.\ the stringed DOMs, and an IceTop array.
This consists of detectors on the surface, housing water Cherenkov detectors that may act as a useful veto on particles: if they were seen by IceTop, they were probably down-going particles, and thus background, while the neutrino events present as up-going particles created via interactions in the Earth, where only neutrinos penetrate; see~\vref{fig:ray_types}.
The detector is tuned to particles in the \si{\giga\electronvolt} to \si{\peta\electronvolt} range, where atmospheric neutrinos dominate, as seen in~\vref{fig:neutrino_spectrum}.

IceCube records data based on certain triggers (\vref{sec:triggers}), and sends filtered data to the Northern Hemisphere by satellite (about \SI{100}{\giga\byte\per\day}~\cite{Kelley2014}), with the raw data sent by sneakernet\footnote{That is, data transported by sneakers, i.e.\ shipped via physical media rather than the internet.} (about \SI{1}{\tera\byte\per\day}~\cite{Kelley2014}).
The surface also hosts the IceCube Lab, containing supporting electronics such as computers and communication systems.
The whole detector is located near, and supported by, the Amundsen-Scott South Pole station, situated at the Geographic South Pole and administered by the US National Science Foundation.
\begin{figure}[ht]
    \centering
    \includegraphics[width=0.7\textwidth]{./images/icecube/cherenkov_icecube.png}
    \caption{Detection modes of the AMANDA detector (built following the same principles as IceCube). This makes clear that the Cherenkov light is what is actually observed, from a muon in the left figure, and a cascade in the right. The track-like signatures are detected from Cherenkov cones, while the cascades present as Cherenkov spheres. Image from~\cite{Ahrens2004}.}\label{fig:cherenkov_icecube}
\end{figure}
The observatory detects Cherenkov light from secondary particles---visualized in~\vref{fig:cherenkov_icecube}---from both charged- and neutral current neutrino interactions,
with the large volume of ice providing a plentiful basin of potentially interacting ice molecules.
In the case of neutral current interactions, the event signature is that of a cascade: the neutrino scatters of a charged lepton, escaping the detector again, with the deposited energy resulting in electromagnetic and hadronic cascades; hence the name.
These cascades can be contained fully within the detector, meaning the energy can be easily inferred, but it is difficult to reconstruct a track that points to the origin of the neutrino.
Charged current (CC) interactions, on the other hand, present the possibility of track-like signals, which make for a much better pointing resolution.
However, the flavor of the neutrino in this case determines the signature:
\begin{itemize}
	\item An electron neutrino CC interaction will produce an electron, which will create an electromagnetic shower, as the electron instigates pair production, and is slowed down by bremsstrahlung. This looks like a cascade signature
	\item A tau lepton will live very shortly (on the order of \SI{1e-13}{\second}) owing to its massive nature, and decays again creating a cascade\todo{From Troels: \ldots or a muon track (20 percent). Explain\ldots}
	\item Muons are special: they do not decay immediately, as taus, and are less affected by the processes stopping electrons. They are thus able to leave long tracks in the detector, a track-like signature\todo{From Troels: You need to explain why muons don't loose much energy in matter (like electrons)}
\end{itemize}
All three flavors create a hadronic cascade at the interaction vertex, \( X \) in \( \Pnu_{\alpha} + N \rightarrow \Plepton_{\alpha} + X \) with \( N \) a nucleon, \( \Pnu_{\alpha} \) a neutrino and \( \Plepton_{\alpha} \) a lepton, both of flavor \( \alpha \).
\todo[inline]{Gotta explain e-m and hadronic cascades}
In the case of a tau neutrino, there is the possibility of another (as of yet unseen) signature: the double-bang.
A tau will have a decay length of \SI{50}{\meter\per\peta\electronvolt}~\cite{Ahlers2018}, so very high energy tau neutrinos may interact and first create a hadronic cascade, then a track from a high energy tau, which then decays \SI{50}{\meter} or more away.
The signatures can be seen in~\vref{fig:event_signatures}.

\begin{figure}[ht]
    \centering
    \includegraphics[width=\textwidth]{./images/icecube/Typical-event-signatures-observed-by-IceCube-a-track-like-event-left-a-showerlike.png}
    \caption{Three IceCube event signatures. The size of each colored sphere is proportional to the charge of the pulse detected by the PMT, and the color is related to the timing of the signal; red hits are earlier than blue. The left figure shows a track-like muon event, while the signature in the middle is a cascade. On the right is shown a double-bang event, where two cascades and a track is seen. The two left-most events are real, the double-bang is simulated. Image from~\cite{Meagher2016}.}\label{fig:event_signatures}
\end{figure}

\section{DOMs}\label{sec:doms}

The DOMs are situated on \SI{2.5}{\kilo\meter} long strings, spaced \SI{17}{\meter} apart (\SI{7}{\meter} for DeepCore, \SI{2.4}{\meter} for Upgrade).
They are essentially in-ice, high quality light detectors, capable of detecting single photons~\cite{Aartsen2017}.
Being self-contained in the sense that all recording capabilities are on-device, and according to pre-set triggers (\vref{sec:triggers}), they will record and save pulses and transmit them via cabling to IceCube Lab on the surface.
When a photon is detected, the DOM starts a recording, lasting for up to \SI{6.4}{\micro\second}, such that later arriving photons are included in the \enquote{hit}~\cite{Aartsen2017}.
This operation is independent, and as such each DOM does not know what other DOMs are seeing.
However, the PMT hit rate is recorded in millisecond intervals, because an overall increase in hits might indicate a cosmic event, which must be acted upon promptly, and a separate local coincidence wiring---connecting closest DOMs---ensures quick recognition of (near) simultaneous hits~\cite{Aartsen2017}.

The PMT waveforms are digitized by two circuits, ATWD (Analog Transient Waveform Digitizer) and fADC (fast Analog to Digital Converter).
These differ in their sampling time, \SI{427}{\nano\second} for ATWD, \SI{6.4}{\micro\second} for fADC.\@
\todo[inline]{Expound on shortly digitizers; explain the resulting difference in \( \sigma(t) \).}

The DOMs also contain \enquote{flasher boards} outfitted with LEDs, used for calibration purposes such as measuring their positions and the optical properties of the ice.

It should be noted that the DOMs are pretty reliable; by 2016 87 DOMs out of \num{5160} were not operational~\cite{Aartsen2017}; this is rather fortuitous, inasmuch as they are non-repairable (at least they're impossible to retrieve from the ice).

The location of the detector, in the dark ice deep beneath the Antarctic, leads to the majority of noise in the PMTs being \enquote{dark noise}, that is quantum effects such as thermal emission, radioactive decays and the luminescence of the glass; these effects have no outside sources, but have a rate of \SIrange{560}{780}{\hertz}~\cite{Aartsen2017}.
The background muons, for comparison, are at a rate of \SIrange{5}{25}{\hertz}~\cite{Aartsen2017}.

Interestingly, the noise rate decreases after installation, as the DOMs \enquote{freeze in}~\cite{Aartsen2017}.

As can be seen in~\vref{fig:dom_schematic}, the entire southern hemisphere of the DOMs is taken up by the PMT.\@
Accordingly, all DOMs point downwards, a design choice changed in the Upgrade (\vref{sec:icecube_upgrade}).
Communication and power is handled by the cabling---clearly seen in~\vref{fig:dom_pic}---, connecting all DOMs on a string, terminating on the surface in the IceCube Lab.

\begin{figure}
     \centering
     \begin{subfigure}[b]{0.4\textwidth}
         \centering
		    \includegraphics[width=\textwidth]{./images/icecube/ICECUBE_dom_taklampa.jpg}
		    \caption{Image from \url{https://commons.wikimedia.org/wiki/File:ICECUBE_dom_taklampa.jpg}}\label{fig:dom_pic}
     \end{subfigure}
     \hfill
     \begin{subfigure}[b]{0.4\textwidth}
         \centering
		    \includegraphics[width=\textwidth]{./images/icecube/5-DOM-Picture.png}
		    \caption{}\label{fig:dom_schematic}
     \end{subfigure}
        \caption{(a), from~\cite{WikiDOM}, shows a picture of a DOM, (b)~\cite{Abbasi2009} shows a schematic of a DOM.}\label{fig:doms}
\end{figure}

\section{Triggers and on-pole reconstruction}\label{sec:triggers}

As any other particle detector, IceCube employs triggers to maintain the otherwise unmanageable data levels at a reasonable size.
The raw data rate, after triggers have been applied, is \SI{1}{\tera\byte\per\day}, which is whittled down to around \SI{100}{\giga\byte} by event selections performed by the on-Pole servers, appropriate for transfer via satellite to the Northern Hemisphere~\cite{Kelley2014}, where events can be analyzed further.
There are several triggers at work, with different functionalities in mind; one trigger, the \enquote{Single Multiplicity Trigger} (SMT), uses the local coincidence wiring and acts when \( N \) or more locally coincident hits (LC) are detected within a sliding window \( n \) \si{\micro\second}, extended until the local coincidence condition is no longer satisfied.
With \( N = 8 \) and \( n = 5 \) the hit rate of this particular trigger is \SI{2100}{\hertz}~\cite{Kelley2014}.
Another is the string trigger, looking for a certain number of hits along a string, useful for slow-moving, completely up-going, muons.
In general, the triggers rely on the speed of light in ice to figure out whether hits are causally connected, or appear to be noise.
Additionally, because an event may fulfill the conditions set out in several triggers, hit-merging is performed, ensuring that unique hits are not present in several events.
On the other extreme, events can be coincident, needing algorithms to split events from each other.

When triggers have been activated, detector data is sent to IceCube Lab, which uses on-premise servers to perform fast event selection and preliminary reconstruction to determine if a muon is up- or down-going and whether several muons are present in one event.
The reconstruction is important as a filter, because the largest source of noise in the detector comes from background atmospheric muons.
Using earth-filtering is the easiest way of determining whether a muon is atmospheric or caused by a neutrino, as the ground stops all of the former, but allows passage of the latter, but this does present a challenge: if one wants to determine (approximately) whence a muon came, one needs to assemble the disparate hits in the detector into something resembling a line.
As several thousand events are read out each second, the algorithm must be fast, and fast on the available hardware (which tends not to be replaced often, seeing as how it is situated at one of the hardest to reach locations on Earth).
The servers at the IceCube Lab perform first a quick reconstruction, \enquote{linefit}~\cite{Aartsen2014}---a minimizer using a Huber penalty---used as a seed for the \enquote{Single-Photo-Electron-Fit}, a maximum likelihood estimation~\cite{Ahrens2004}.
This reconstruction has a median angular resolution (the arc-distance between reconstruction and the true track) on Monte Carlo (MC,~\vref{sec:monte_carlo_simulation}) data of \( \theta_{\text{med}} = \SI{4.2}{\degree} \)~\cite{Aartsen2014}.

\todo[inline]{From Troels: It would be fitting to list the sources of noise/background and their rates. Noise: from DOMs, background: from particles (cosmic muons).}

\section{Post processing}\label{sec:post_processing}

%\section{Levels}\label{sec:levels}

When data arrives from the Pole, different analysis groups apply different post-processing measures.
In the following we shall focus on the methods applied by the oscillation working group, a suite of processing software termed \enquote{oscNext}, documented in~\cite{Blot2020}.
Six levels are used, each removing different types of noise and muons using several different methods (some of which are machine learning based).
The muon and noise is reduced by a factor \( \mathcal{O}(10^{5}) \) by the post-processing, and at the final level a neutrino rate of \( \mathcal{O}(\SI{1}{\milli\hertz}) \) survives with a few \% background muons and almost no noise.

\todo[inline]{From Troels: What is the neutrino efficiency?!?}

The post-processing levels are shortly touched on in the following.

\subsection{L2}\label{sec:l2}

L2 is run on files received from the Pole.
It first passes finds events passing the SMT3 trigger, meaning the SMT described in~\vref{sec:triggers} with \( N = 3 \).
Next, a Seeded Radius-Time (SRT) cleaning algorithm is run, which relies on the fact that Cherenkov photons will be clustered in space and time, while noise is random.
The SRT algorithm uses this on the LC hits, used as seeds, with non-LC hits outside some pre-set space-time volume discarded, while those inside are kept and used as seeds for further iteration.
Lastly, a DeepCore fiducial volume, and a DeepCore veto volume are defined, and using a center of gravity calculation using hits in the fiducial volume, hits from the veto region are discarded based on their derived velocity, on the basis that such a hit might be related to a crossing muon.

\subsection{L3}\label{sec:l3}

At level 3, a number of cuts are made on several derived variables, described in~\cite{Blot2020}.
The main point of this level is to bring the data into agreement with MC, the discrepancies mainly stemming from processes for which there is no simulation.
\todo[inline]{More about L3; which processes, cuts, derived variables?}

L3 removes around \SI{95}{\percent} of the background muon events, \SI{99}{\percent} of pure noise and keeps \SI{60}{\percent} of atmospheric neutrino events relative to level 2, such that the event rate is below \SI{1}{\hertz}~\cite{Blot2020}.

\subsection{L4}\label{sec:l4}

By level 4 there is good agreement between MC and data, important because at this step machine learning algorithms for classification are introduced.
Two classifiers are used, both Boosted Decision Trees (BDT) LightGBM models: one for pure noise, another for background (atmospheric) muons, both trained on a number of derived variables, and the background sample for the muon classifier being real detector data.
The pure noise rate is reduced at L4 from \SI{36.6}{\milli\hertz} to under \SI{0.3}{\milli\hertz} with \( \approx \SI{96}{\percent} \) of actual neutrino events intact, while \SI{94}{\percent} background muons are removed, keeping \SI{87}{\percent} of neutrinos.

Relative to L3, \SI{99}{\percent} of pure noise and over \SI{94}{\percent} of atmospheric muons events are removed, while \SI{80}{\percent} of neutrinos are kept, resulting in roughly a \( 100 : 10 : 1 \) (muon : neutrino : noise) ratio.

\subsection{L5}\label{sec:l5}

Level 5's main function is removing so-called sneaky muons.
These are particles that \enquote{sneak in} to DeepCore, without hitting any IceCube strings, a possibility caused by the hexagonal pattern of the detector.
This is done by calculating the COG of an event, finding the closest DeepCore string to that, and if a set number of DOMs along some previously defined poorly instrumented corridors, within some volume cylinder and time, are hit, the event is dropped.
Further, SPE is run on the event, and the calculated direction compared to the corridors; if it does not line up with any corridor, the event is kept.
\todo[inline]{Whoops! What's SPE?}
A starting containment cut is also employed, cutting events beginning outside, or close to the edge of, DeepCore.

After L5, the rates are \SI{2.16}{\milli\hertz} for neutrinos, \SI{0.93}{\milli\hertz} for atmospheric muons and \SI{0.07}{\milli\hertz} for pure noise.


\subsection{L6}\label{sec:l6}

L6 splits the remaining data into two sets, verification and high statistics.
One algorithm (or, rather, two: SANTA/LEERA) is run on the verification set, while another is run on the high statistics set (RetroReco).
SANTA is a direction reconstruction algorithm, while LEERA does energy, and they are simpler, although less efficient, than RetroReco, but less impacted by detector uncertainties.
They are fast, but require only unscattered photons, and are intended to be used for checking data/MC agreement.

RetroReco, on the other hand, uses a likelihood function, employing table based lookups which simulate photon propagation from PMTs, under the assumption that scattering and absorption are time symmetric processes.
The tables take up \( \approx \SI{1}{\tera\byte} \), but due to this being extremely impractical RAM-wise, some clever tricks are employed to lower memory consumption (which is still significant when compared to older reconstruction algorithms).

\begin{figure}[ht]
    \centering
    \includegraphics[width=0.5\textwidth]{./images/icecube/retro_pegleg_comp.png}
    \caption{Reconstruction comparisons between PegLeg (the previous reconstruction algorithm) and RetroReco (the current). Reconstructions are measured in seconds per event.}\label{fig:reco_comp}
\end{figure}

Additionally, reconstructions are measured in seconds (see~\vref{fig:reco_comp}), which makes it seem like an obvious candidate for improvements using machine learning; exactly the point of this thesis, using GENIE-MC generated muon neutrinos, with RetroReco reconstruction, numbering \num{7280939} events in total.

%\section{Current reconstruction methods}\label{sec:current_reconstruction_methods}

\section{I3File format}\label{sec:i3_file_format}

All the following information was retrieved from the IceCube documentation server, at \url{https://docs.icecube.aq/icerec/V05-01-04/projects/dataio/portable_binary_archive.html}.
The author does not know of any scientific paper detailing the I3File format.

The IceCube data, be it from the detector or MC generated, is kept in a portable binary archive format called \verb|I3Files|.
A binary file stands in contrast to text files, which include known formats such as XML and JSON.\@
There are pros and cons of this choice: first, binary files tend to be much smaller than text files.
This is simply because storing e.g.\ a 64-bit integer in a text file is done using some form of character encoding (e.g. ASCII) which can take up over 20 bytes, whilst the same number can be stored in a binary format using exactly 8 bytes.
Secondly, binary files tend to be much quicker to read, because the information does not need to be decoded.
The portability is achieved by being built on the Boost C++ library's archive format, widely available on all major platforms, although I3Files have diverged since, and require an old Boost version.

The files basically consist of so-called I3Frames, which may hold either primitives (such as integers, floats, etc.) or objects.
The I3Frames are serialized, and so anything contained within them must be serializable.

An event can be contained within an I3Frame, and will contain information such as pulses and their associated DOMs, indexed by keys, which can then be looked up in a complementary geometry file.
Reading the files can conveniently be done through a Python interface called IceTray, enabling the user to use a modern language for processing and analysis.

The format has stood the test of time, and has been in use since its creation in 2006, but there are a couple of issues with it: first (and most minor) of all, it requires the installation of IceCube software on the target machine.
This is a minor annoyance, as one can build this in relatively short time, but it does not have to be compiled and is not available easily e.g.\ via PyPi (and with good reasons, as it contains many dependencies).
The second issue---which is rather large for machine learning applications---is I3Files inability to seek in frames.
In other words, the file format is sequential, and you have to loop through the frames to find the one you are interested in.

This is unusable for machine learning applications, where it is a requirement that your sets are defined in advance, \textit{and} are easily retrievable on demand, as your training will probably include randomization each epoch.

This was addressed in this project by using SQLite; see~\vref{sec:sqlite}.

\section{DeepCore}\label{sec:deepcore}

\begin{figure}
     \centering
     \begin{subfigure}[b]{0.49\textwidth}
         \centering
            \includegraphics[width=\textwidth]{./images/icecube/4-VetoView2_labeledStringswIcetops.jpg}
            \caption{}\label{fig:deepcore_string_pattern}
     \end{subfigure}
     \hfill
     \begin{subfigure}[b]{0.49\textwidth}
         \centering
            \includegraphics[width=\textwidth]{./images/icecube/Upgrade_Spacing.jpg}
            \caption{}\label{fig:upgrade_string_pattern}
     \end{subfigure}
        \caption{\Vref{fig:deepcore_string_pattern} shows the string pattern of IceCube and DeepCore,~\vref{fig:upgrade_string_pattern}~\cite{Ishihara2019} shows the proposed upgrade.
        Both images are courtesy IceCube,~\vref{fig:deepcore_string_pattern} is from~\cite{IcecubeDOM1}.}\label{fig:string_patterns}
\end{figure}

Before all of IceCube's strings were even deployed---a laborious task only possible during the short austral summer months---it was recognized that the energy range of the detector might not be sufficient for certain analysis purposes.
As such, DeepCore was intended to lower the useful energy limit, doing so by installing a new set of strings, closer together, enclosed by IceCube proper.
Each new string has a DOM spacing of \SI{7}{\meter} vs. IceCube's \SI{17}{\meter}, and many PMTs are of an improved quantum efficiency relative to the \enquote{old} PMTs.

The IceCube DOMs include DOM-to-DOM communication capabilities, which was used in the early years to measure ice properties.
This data was used to determine the optimal placement of DeepCore, around \SI{2100}{\meter}, in the clearest ice measured.
All these measures amount to a new lower limit on the neutrino energy useful for analysis to \SI{10}{\giga\electronvolt}~\cite{Abbasi2012}.
The top-down geometry can be seen in~\vref{fig:deepcore_string_pattern}

DeepCore has made possible precision measurements of neutrino oscillations, and has also been used in the hunt for WIMPs\footnote{Weakly interacting massive particles, a dark matter candidate}.

\todo[inline]{Hmm, need more about DeepCore. Maybe the depth, the number of DOMs, volume, lowered energy reach.}

\section{IceCube Upgrade}\label{sec:icecube_upgrade}

The IceCube Upgrade, a prelude to the humongous \SI{8}{\cubic\kilo\meter} IceCube-Gen2, aims to repeat DeepCore's success, and add an extra trick: new DOM designs.
The Upgrade features two new types of DOMs, the mDOM and the D-Egg; see~\vref{fig:upgrade_doms}.
Whereas IceCube and DeepCore features DOMs whose PMTs only point downward, Upgrade's new DOM types feature instead PMTs that point otherwise.
In the simplest case, D-Egg\footnote{Dual optical sensors in an Ellipsoid Glass for Gen2, a somewhat contrived acronym}, one PMT points up, the other down, while the mDOM\footnote{multi-PMT Digital Optical Module} possesses 24 PMTs arranged in a spherical fashion.
\begin{figure}
	\begin{subfigure}{\linewidth}
		\centering
		\includegraphics[width=0.5\linewidth]{./images/icecube/UpG_OpticalSensors_SensorsOnly_v2.png}
		\caption{}\label{fig:upgrade_all_doms}
	\end{subfigure}\\[1ex]
	\begin{subfigure}{\linewidth}
		\centering
		\includegraphics[width=0.5\linewidth]{./images/icecube/comparison_updated.JPG}
		\caption{}\label{fig:improved_event_upgrade}
	\end{subfigure}
	\caption{(a) shows new DOM types installed with the IceCube upgrade. \Vref{fig:improved_event_upgrade} shows a simulated \SI{3.8}{\giga\electronvolt} track-like event and its Cherenkov cone. More DOMs light up in the Upgrade configuration, giving more information for track reconstruction. Figures are from the IceCube Collaboration~\cite{IcecubeUpgrade1}.}\label{fig:upgrade_doms}
\end{figure}
The seven new Upgrade strings will be deployed in the same clear-ice region as DeepCore, and will feature even closer DOMs that are \SI{3}{\meter} apart, and an inter-string spacing of around \SI{20}{\meter}.
This will enhance the physics potential by enabling an even lower energy threshold, a significantly improved directional precision, more precise energy estimates, and possibly the potential to see double bang events or just statistically identify tau events.
Finally, the installation (centered around IceCube string 36) may serve to improve the overall string calibration of IceCube.

The low energy capabilities of Upgrade are important to \( \Pnu_{\Ptau} \) oscillation studies, because the muon to tau oscillation,
\begin{equation}
    P(\Pnu_{\Pmu} \rightarrow \Pnu_{\Ptau}) \approx \sin^{2}{\left( 2 \theta_{2 3} \right)} \cos^{4}{\left( \theta_{1 3} \right)} \sin^{2}{\left( \frac{\Delta{m_{3 1}^{2}} L}{4 E} \right)},
\end{equation}
has an oscillation maximum for \( \Pnu_{\Ptau} \) at \SI{25}{\giga\electronvolt}, with \( L \) equal to Earth's diameter\cite{Ishihara2019}; the directionality of the new DOMs provide a better pointing resolution, which of course is crucial to determine the zenith angle as a proxy for \( L \).
\begin{figure}[ht]
    \centering
    \includegraphics[width=\textwidth]{./images/icecube/upgrade_neutrinos.png}
    \caption{The fully contained atmospheric \( \Pnu_{\Pmu} \) and \( \Pnu_{\Ptau} \) rates from charged current interactions in a cylindrical volume with a \SI{50}{\meter} radius and a height of \SI{275}{\meter} (\enquote{inner fiducial}; red lines) or a \SI{145}{\meter} radius and same height (\enquote{outer fiducial}; blue lines), both centered on DeepCore, for DeepCore (dotted lines) and Upgrade (solid lines).
    A significant enhancement in rates is seen around the interesting energy range of \( \approx \SI{30}{\giga\electronvolt} \).
    Image from~\cite{Ishihara2019}.}\label{fig:upgrade_rates}
\end{figure}
\Vref{fig:upgrade_rates} shows the Upgrade's improvement in muon- and tau neutrino rates.
Upgrade will aim to test the unitarity of the PMNS matrix, which is rather exciting because non-unitarity always points to new physics, seeing as how it is related to the normalization of probabilities.

Furthermore, the Upgrade will contain various calibration equipment which will measure certain properties of the ice.
These measurement can be applied retroactively, and will hopefully enhance all IceCube/DeepCore datasets.

The Upgrade is to be installed during the austral summer of 2022--23, but as of this writing there is no known reconstruction algorithm working for \( \Pnu_{\Pmu} \), but seeing as the additional information contained in the new DOMs is simply new features for a machine learning algorithm, the work in this thesis is applied to Upgrade MC muon neutrinos.

\end{document}
