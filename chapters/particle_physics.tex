\documentclass[../main.tex]{subfiles}

\begin{document}

Neutrinos are some of the universe's most interesting and mysterious particles, and offer one of the better paths to researching some still unsolved problems in physics.
The Large Hadron Collider nailed down the final missing piece of the model, the Higgs boson, but questions still remain; and some of them concern neutrinos.

The masses of the neutrinos themselves is a mystery, while sterile neutrinos is a dark matter candidate and the possible CP-violating phase in the PMNS matrix can help explain baryon asymmetry.
Needless to say, the more we discover about this little particle, the better we know the universe.

The concept of neutrino oscillations has the ability to constrain the previously mentioned CP-violating phase of the neutrino mixing matrix~\autocite{Abe2019} and so the better we are able to measure this, the better the constraint will be.

This chapter will seek to outline the physics behind the interactions that are captured in the IceCube detector, including the concept of Cherenkov radiation, which is of critical importance to the process.

First, we shortly need to touch on what The Standard Model of particle physics is.

\begin{figure}
     \centering
     \begin{subfigure}[b]{0.49\textwidth}
         \centering
         \includegraphics[width=\textwidth]{./images/particle_physics/Standard_Model_of_Elementary_Particles.pdf}
         \caption{The fermions and bosons of The Standard Model of particle physics. Image from (ref)}\label{fig:standard_model_overview}
     \end{subfigure}
     \hfill
     \begin{subfigure}[b]{0.49\textwidth}
         \centering
         \includegraphics[width=\textwidth]{./images/particle_physics/Elementary_particle_interactions_in_the_Standard_Model.png}
         \caption{The possible interactions in the SM between fermions and bosons. Image from (ref)}\label{fig:standard_model_interactions}
     \end{subfigure}
        \caption{Three simple graphs}
        \label{fig:standard_model}
\end{figure}

\section{The Standard Model}\label{sec:the_standard_model}

The Standard Model (SM) of particle physics is the foundation of modern particle physics research.
It was patched together during the 20th century by scientific collaboration, and tested through inter-country spending on giant particle accelerators.

The theory itself is a gauge quantum field theory, the mathematics of which is not particularly relevant to this thesis, and it describes three out of the four known fundamental interactions.
Of these, the so-called weak one is of importance to the neutrino, while being a critical clue to why some particles acquire mass.

The SM describes two types of particles: fermions and bosons, as seen in~\vref{fig:standard_model_overview}.
The distinguishing feature between the two sets is the statistics they obey; whereas the fermions follow Fermi-Dirac statistics---as a consequence of their half-integer spin---the bosons adhere to Bose-Einstein statistics.
Matter in the universe consists of fermions, while the fundamental forces are mediated by the bosons.

The fermions are divided between the quarks and the leptons, differentiated by their participation in the strong interaction, carried by the massless gluon.
This boson only acts on quarks, and has the peculiar behaviour of creating new particles when quarks are pulled apart, resulting in the fact that it is only possible to observe hadrons (a composite particle made up of two or more quarks); this phenomenon, \enquote{color confinement}, is the progenitor of jets in collider experiments such as the LHC.

Both the charged leptons and the quarks interact electromagnetically and weakly, forces mediated by the photon and the \( \PWpm \) \& \( \PZzero \) bosons respectively, but the leptons include a second sub-species that only interacts weakly: the neutral leptons, known as the neutrinos; 	see~\vref{fig:standard_model_interactions}.
Because they do not participate in anything but the weak force, neutrinos hardly ever interact, and are thus difficult (but not impossible!) to observe.
An \si{\mega\electronvolt} neutrino has a cross section on the order of \SI{1e-44}{\centi\meter\squared}, and a beam of such particles must travel through a light year of lead before half of them is blocked.


The experimental particle physicist uses Feynman diagrams to describe potential interactions between particles,
depending on Fermi's golden rule, the Lorentz-invariant matrix element and for fermions the Dirac equation.

%But what is the Standard Model trying to describe?
%Well, every macroscopic event in our lives (and, indeed, the universe) consists of interactions: the tweeting of a bird, the drizzle falling on cold autumn day, even the keyboard clacking sounds of a graduate student furiously typing up his thesis; all interactions.
%But these are macroscopic, made from relations at a lower level still.
%The bird, spurred on by the invisible and unending power of evolution, forces air through membranes to produce flattering sounds that attract mates.
%Thermodynamics and gravity create conditions in which small water drops fall towards the ground, and the graduate student is compelled by coffee and his supervisor to produce some result or other.
%But this still only explains the phenomenon on a macroscopic level.
%It is possible to keep peeling away abstraction layers, and hopefully arrive at some lowest level that can describe the fundamental forces that conspire to create the universe and life as we know it.
%This is the job of the Standard Model, the most succesful mathematical formulation of fundamentality we have.
%
%The progression from Galilei to Higgs is a remarkable journey.
%It never ceases to amaze the author that Dirac hit upon his equation (\vref{eq:dirac_equation_1}) and discovered that the structure was that of Clifford algebra, discovered 50 years before as a purely mathematical concept.
%The use of group theory, seemingly far removed from everyday life, permeates the Standard Model, and in fact the applicability of abstract mathematics in particle physics is a bewildering thing, contemplated by Wigner~\autocite{Wigner1960}.
%Of course this thesis is not concerned with this philosophical aspect of physics, but the preceding sentences serve as an introduction to the Dirac equation, which will be used as a mathematical introduction to the concepts that need explaining in this chapter.

Paul Dirac and others before him searched for an extension of the Schr\"{o}dinger equation, which incorporated relativity.
%Because relativity treats space and time on the same footing, both need to appear on equal footing in any governing equation, which is not the case for the famous position space, time-dependent, Schr\"{o}dinger equation,
%\begin{equation}\label{eq:schrodinger}
%	i \diffp*{\psi(\bm{r}, t)}{t} = \left[ - \frac{1}{2 m} \nabla^2 + V(\bm{r}, t) \right ] \psi(\bm{r}, t).
%\end{equation}
%This is first order in time, second order in space.
%
%The Klein-Gordon equation, rewriting the Einstein energy-momentum relation as an operator equation,
%\begin{equation}\label{eq:klein_gordon}
%	E^2 = \bm{p}^2 + m^2 \Rightarrow \hat{E}^2 \psi = \left( \hat{\bm{p}}^2 + m^2 \right) \psi \iff \left( \diffp*[2]{}{t} - \nabla^2 + m^2 \right) \psi = 0,
%\end{equation}
%fixes this problem, but instead causes issues leading to nonsense negative probability densities.
%Now, physics has swallowed many unintuitive consequences of equations (bending of space-time, \enquote{spooky action at a distance} to name a few), but negative probability is not something we are quite ready to accept.
%Thus Dirac attempted a fix, trying to bring both space and time to first order,
%\begin{equation}\label{eq:dirac_equation_1}
%	\hat{E} \psi = \left( \bm{\alpha} \cdot \hat{\bm{p}} + \beta m \right) \psi,
%\end{equation}
%where
%\begin{equation}
%	\bm{\alpha} \cdot \hat{\bm{p}} = \begin{bmatrix} \alpha_{x} & \alpha_{y} & \alpha_{z} \end{bmatrix} \begin{bmatrix} \hat{p}_{x} \\ \hat{p}_{y} \\ \hat{p}_{z} \end{bmatrix}
%\end{equation}
%This seemingly simple equation actually leads to some rather surprising results, in no small part owing to the objects \( \bm{\alpha} \) and \( \beta \).
%Constraints on these objects can be placed---because \vref{eq:dirac_equation_1} must satisfy \vref{eq:klein_gordon}, if it is to respect relativity---only allowing them to represent \( 4 \times 4 \) matrices and \( \psi \) thus a four-component wave function.
%
%\todo[inline]{Talk about the consequences of four component wave function, e.g. anti-particles}
%
%\todo[inline]{Talk about spinors}
%
%\Vref{eq:dirac_equation_1} can be rewritten as
%
Requiring adherence to the Einstein energy-momentum relation, Dirac found the matrix equation
\begin{equation}\label{eq:dirac_equation_2}
    \left( i \gamma^{\mu} \partial_{\mu} - m \right) \psi = 0,
\end{equation}
which requires a four-component wave function (not incidentally explaining spin and anti-particles).
This is the governing equation of all fermions.
%with \( \mu \in \left\{ 0, 1, 2, 3 \right\} \) and where \( \gamma^{\mu} \) represents the gamma matrices\footnote{\( \gamma^0 \equiv \beta, \quad \gamma^1 \equiv \beta \alpha_{x}, \quad \gamma^2 \equiv \beta \alpha_{y}, \quad \gamma^3 \equiv \beta \alpha_{z},  \)} and \( \partial \) is the partial differential operator.
%The gamma matrices are defined by their anti-commutation relation
%\begin{equation}\label{eg:gamma_matrices_relation}
%    \left\{ \gamma^{\mu}, \gamma^{\nu} \right\} = \gamma^{\mu} \gamma^{\nu} + \gamma^{\nu} \gamma^{\mu} = 2 \eta^{\mu \nu} I_{4},
%\end{equation}
%where \( \eta \) is the Minkowski metric with signature \( \left( + - - -  \right) \) and \( I_{4} \) is the 4 dimensional identity matrix.
%\Vref{eg:gamma_matrices_relation} generates the so-called Clifford algebra, and is on its own enough to define the system.

%\Vref{eq:dirac_equation_1} came about out of needing to reconcile special relativity and quantum mechanics;
%in short, the wave equation needed to be Lorentz invariant and conform to Einstein's energy-momentum relation, \( E^2 = \left( p c \right)^2 + \left( m c^2 \right)^2 \).

Experiments generally proceed by calculating expected particle rates, and using either accelerators or passive detectors to compare reality to theory using Fermi's golden rule,
\begin{equation}\label{eq:fermi_golden_rule}
	\Gamma_{f i} = 2 \pi \left| T_{f i} \right|^2 \rho(E_{i}),
\end{equation}
describing transition rates from an initial state \( \Ket{i} \) to a final state \( \Ket{f} \).
The equation depends on the transition matrix \( T_{f i} \) from the expansion of the interaction with the perturbation Hamiltonian, with \( \rho \) the energy-dependent density of states.
\Vref{eq:fermi_golden_rule} can be used to calculate the differential cross section
\begin{equation}\label{eq:differential_cross_section}
	\diff{\sigma}{\Omega^{*}} = \frac{1}{64 \pi^2 s} \frac{\mathrm{p}_{f}^{*}}{\mathrm{p}_{i}^{*}} \left| \mathcal{M}_{f i} \right|^{2},
\end{equation}
an expression of quantum mechanical probabilities for an interaction to happen, valid in the centre-of-mass frame (COM), indicated by \( * \).
\( s \) is the squared  COM energy, \( \mathrm{p} \) the particle momenta, and \( \Omega \) the solid angle a particle scatters into.

The SM thus provides observables, as long as one can calculate the Lorentz invariant matrix element \( \mathcal{M} \), which in the fermion case can be done using \vref{eq:dirac_equation_2}.
This process is much simplified by Feynman diagrams, allowing the theorist to draw processes and using Feynman rules stitch together a matrix element describing the interaction.
Examples of Feynman diagrams can be seen in \vref{fig:charged_neutral_current}.
%\begin{equation}\label{eg:fermi_golden_rule}
%	\Gamma_{\text{f} \text{i}} = \frac{\mathrm{p}^{*}}{32 \pi^2 m^{2}} \int \left| \mathcal{M} \right|^2 \, \mathrm{d} \Omega,
%\end{equation}
%where \( \mathrm{p}^{*} \) is the centre-of-mass frame momentum of the final-state particles, \( m \) is the centre-of-mass frame mass of the decaying particle, and \( \mathcal{M} \) is the Lorenz-invariant matrix element.
%If we know \( \mathrm{p}^{*} \) and \( m \), all we need to calculate is the Lorenz-invariant matrix element.
%
%In most cases, cross sections are of most interest to physicists.
%This deals with the quantum mechanical probabilities of whether an interaction happens, and so is related to the geometric definition of cross sections, but in a probability space.
%Consider two types of particles, \( a \) and \( b \), moving with velocities \( \bm{v}_{a} \) and \( \bm{v}_{b} \) in opposite directions.
%The decay rate of Fermi's golden rule, \( \Gamma_{\text{f} \text{i}} \), is then related to the cross section by
%\begin{equation}
%	\sigma = \frac{\Gamma_{\text{f} \text{i}}}{v_{a} + v_{b}},
%\end{equation}
%where \( v_{i} \), \( i \in \left\{ a, b \right\} \) is the speed of particle \( i \).
%
%The cross section for any two-body to two-body interaction can be written as
%\begin{equation}
%	\sigma = \frac{1}{64 \pi^2 s} \frac{\mathrm{p}_{\text{f}}^{*}}{\mathrm{p}_{\text{i}}^{*}} \int \left| \mathcal{M}_{\text{f} \text{i}} \right|^2 \, \mathrm{d} \Omega^{*},
%\end{equation}
%where \( s \) is the squared centre-of-mass energy, \( \mathrm{p}_{\text{i}}^{*} \) is the initial centre-of-mass momentum, \( \mathrm{p}_{\text{f}}^{*} \) is the final centre-of-mass momentum, and the differential \( \mathrm{d} \Omega^{*} \) refers to the centre-of-mass frame as well.
%From this we find the differential cross section,
%\begin{equation}
%	\diff{\sigma}{\Omega}
%\end{equation}

%To actually calculate the expected cross section, all one needs is the problem-specific matrix element, \( \mathcal{M} \).
%The mathematics are embodied in Feynman diagrams (such as~\vref{fig:feynman_e_tau}), in two-body to two-body lowest order cases composed of two vertices and a propagator.
%The Lorenz-invariant matrix element stems from the transition matrix in Fermi's golden rule, and is composed of the interaction strength at each vertex (\( \Braket{\psi_{c} | V | \psi_{a}} \) for the vertex connecting particle \( a \) and particle \( c \), and \( \Braket{\psi_{d} | V | \psi_{b}} \) for the vertex connecting particle \( b \) and particle \( d \); here \( V \) represents a given potential) and a propagator, such that
%\begin{equation}
%	\mathcal{M} = \Braket{\psi_{c} | V | \psi_{a}} \frac{1}{q^2 - m_{X}^2} \Braket{\psi_{d} | V | \psi_{b}},
%\end{equation}
% deriving from the transition matrix of Fermi's golden rule, which is then all one needs to calculate, assuming the initial and final state particle momenta are known.
%and the brilliancy of Feynman diagrams lies in the fact that these can be determined using a set of Feynman rules as soon as the diagram is drawn.
%
%The Dirac equation leads to the form of the matrix elements in the different possible interactions, for example the electromagnetic M{\o}ller scattering of two electrons (seen in~\vref{fig:feynman_e_tau}),
%\begin{equation}\label{eq:qed_interaction}
%    \mathcal{M} = - e^{2} \bar{u}(p_{3}) \gamma^{\mu} u(p_{1}) \frac{g_{\mu \nu}}{q^2} \bar{u}(p_{4}) \gamma^{\nu} u(p_{2}),
%\end{equation}
%where \( u_{\ell}, \ell \in \left\{ e, \tau \right\} \) represents the spinor of a particle, \( \bar{u}_{\ell}, \ell \in \left\{ e, \tau \right\} \) the anti-spinor and the momenta \( \left\{ q, p_{1}, p_{2}, p_{3}, p_{4} \right\} \) correspond to~\vref{fig:feynman_e_tau}.
%
%\begin{figure}
%    \centering
%    \includegraphics[width=0.5\textwidth]{./images/particle_physics/feynman_e_tau.pdf}
%    \caption{Feynman diagram showing \( t \)-channel M{\o}ller scattering \label{fig:feynman_e_tau}}
%\end{figure}
%In this case, two electrons scatter by exchanging a photon, with momentum \( q = p_{1} - p_{3} \), which changes the momentum of both the electrons.
%
%The differential cross section can then laboriously be calculated to, in the ultrarelativistic limit, be
%\begin{equation}
%	\diff{\sigma}{\Omega} = \frac{\alpha^2}{s \sin^4{\theta}} \left( 3 + \cos^2{\theta} \right)^2,
%\end{equation}
%where \( \alpha \) is the fine-structure constant and \( \theta \) the scattering angle.
%Thus we see that we can validate theoretical deductions by observing scattering experiments.

%The Standard Model describes these fundamental interactions of the universe, embodied in Feynman diagrams such as~\vref{fig:feynman_e_tau}, and \vref{eq:qed_interaction} shows the prominence of the gamma matrices in the fundamental interactions, which is of some importance in the weak interaction, to be discussed in~\vref{sec:the_weak_interaction}.
%
%\Vref{fig:standard_model_overview} shows the standard representation of The Standard Model.

\section{The weak interaction}\label{sec:the_weak_interaction}

In the early 20th century the world of nuclear physics was abuzz.
A great many experiments were conducted after Becquerel discovered radioactivity in 1896, some of which explored beta decay, an example of which can be seen in \vref{fig:beta_decay}.
\begin{figure}[ht]
    \centering
    \includegraphics[width=0.8\textwidth]{./images/particle_physics/Beta_spectrum_of_RaE.jpg}
    \caption{Energy spectrum of \ce{^{210}B} beta decay, with \( E_{\text{max}} = \SI{1.16}{\mega\electronvolt} \).
    Image from \url{https://en.wikipedia.org/wiki/File:Beta_spectrum_of_RaE.jpg} }\label{fig:beta_spectrum}
\end{figure}
In this process a neutron turns into a proton, emitting an electron and an anti-neutrino.
This was unexplained without the neutrino, because of two problems:
First, the energy spectrum seemed continuous (see \vref{fig:beta_spectrum}), unexpected as the energy should be discrete if the electron is one escaping from the atom.
Second, the electron is a spin-\sfrac{1}{2} particle, while nitrogen-14, which showed beta decay, is a spin-1 particle; so where goes the spin?

Both issues were explained by the \enquote{neutron} (later renamed neutrino by Fermi) proposed by Pauli in 1930.
The discrete energy spectrum now turns continuous, owing to the neutrino carrying away unseen energy, and the spin can be attributed to it as well, as long as the neutrino is a spin-\sfrac{1}{2} particle.

Fermi explained this with his theory of beta decay, proposing a four-fermion contact interaction, a direct interaction between the four particles.
Although a successful theory, it runs into the same kind of ultraviolet problem that Planck worked on in the late 19th century: because the Fermi interaction cross section goes as \( \sigma \approx G_{\text{F}}^2 E^2 \), where \( G_{\text{F}} \) is the Fermi constant, it grows unbounded, leading to problems at higher energies.
Additionally the weak interaction was observed in the Wu experiment to violate parity, as beta decay of cobalt-60 showed a preferential decay direction for electrons, a phenomenon not explained by Fermi's theory.
This was solved by the electroweak theory, introducing bosons that mediate the interaction over small distances after symmetry breaking via the Higgs mechanism; the \( \PW \) and \( \PZ \) bosons.
The parity violation is explained by the V-A (vector minus axial vector) nature of the weak force, which only permits it to couple with left-handed particles and right-handed antiparticles; this concept is enshrined mathematically by \( \gamma^5 \) of the Dirac matrix family, formed as the product of the gamma matrices in \vref{eq:dirac_equation_2}, which enters in the so-called chiral projection operator.
\begin{figure}[ht]
    \centering
    \includegraphics[width=0.5\textwidth]{./images/particle_physics/Right_left_helicity.jpg}
    \caption{Helicity. Image from \url{https://en.wikipedia.org/wiki/File:Right_left_helicity.jpg}}\label{fig:helicity}
\end{figure}
Describing this operator is made easier by reference to helicity states, which are more easily grasped conceptually.
A helicity state is described in terms of a particle's component of spin along its momentum, \( h \equiv \mathbf{S} \cdot \mathbf{p} / \textup{p} \); see \vref{fig:helicity}.
Chirality, on the other hand, is defined as the eigenstates of \( \gamma^{5} = i \gamma^{0} \gamma^{1} \gamma^{2} \gamma^{3} \), and in the ultra-relativistic regime, where \( E \gg m \), the domain of neutrinos, helicity and chiral eigenstates are the same.
This becomes important in the weak interaction, whose vertex factor from the matrix element is given as
\begin{equation}
	\frac{i g_{\PW}}{\sqrt{2}} \frac{1}{2} \gamma^{\mu} \left( 1 - \gamma^{5} \right),
\end{equation}
where \( \sfrac{1}{2} ( 1 - \gamma^{5} ) \) is known as the left-handed chiral projection operator.
When this operates on a right handed chiral (or helicity, in the high energy limit) state, the result is zero, meaning only left-handed chiral particle states (and right-handed anti-particle states) may participate in the weak current.

An intriguing possibility is presented by this property: as only (anti)neutrinos of one handedness interact weakly, there may be a whole universe of non-interacting neutrinos that only participate gravitationally.
These \enquote{sterile neutrinos} are candidates for dark matter.

The interaction is termed \enquote{weak} as a consequence of its coupling strength, relative to the electromagnetic- and strong forces.
This is due to the massive nature of its mediating bosons, giving them a shorter lifetime than they would possess otherwise.
Furthermore, two of the bosons contain electric charge (the \( \PWpm \) bosons), while the last one is electrically neutral (the \( \PZzero \) boson), leading to the phenomenon of charged- and neutral currents.

\section{Charged- and neutral currents}\label{sec:charged_and_neutral_currents}

\begin{figure}
	\centering
	\begin{subfigure}[b]{0.3\textwidth}
		\centering
		\includegraphics[width=0.5\textwidth]{./images/particle_physics/feynman_nc.pdf}
		\caption{Neutral current}\label{fig:feynman_nc}
	\end{subfigure}
	\hfill
	\begin{subfigure}[b]{0.3\textwidth}
    	\centering
	    \includegraphics[width=\textwidth]{./images/particle_physics/feynman_beta_decay.pdf}
	    \caption{\( \beta^{-} \) decay}\label{fig:beta_decay}
	\end{subfigure}
	\hfill
	\begin{subfigure}[b]{0.3\textwidth}
		\centering
		\includegraphics[width=0.5\textwidth]{./images/particle_physics/feynman_cc_wp.pdf}
		\caption{Charged current}\label{fig:feynman_cc_wp}
	\end{subfigure}
	\caption{Feynman diagrams of neutral- and charged current processes}\label{fig:charged_neutral_current}
\end{figure}

The nature of the weak force allows two fundamental types of interactions: charged- and neutral currents.

The charged current, as the Feynman diagram in \vref{fig:feynman_cc_wp} shows, allows a charged lepton to turn into a neutral one.
By absorbing a \( \PW \) boson, an electron may turn into an electron neutrino, as seen in \vref{fig:feynman_cc_wp}, or---as in the case of beta decay, \vref{fig:beta_decay}---a down quark may convert into an up quark, changing a neurino into a proton, with the resulting \( \PWminus \) boson decaying into an electron and antielectron neutrino, conserving charge and lepton number.
A startling real world example of the charged current in action is seen in \vref{fig:first_neutrino_event}, where a neutrino of sufficient energy collides with a proton, creating both a muon and hadronic decay products.

The neutral current, on the other hand, seen in \vref{fig:feynman_nc}, can deflect two particles, but does not change any of the properties of the interacting particles other than transferring momentum and spin.
 
In this way the weak interaction is special among the fundamental forces; it is the only one that is able to change flavours, it is the only one to couple to neutrinos, and it is---as we shall see---the only one that allows us to define left- and right-handedness unambigously.

\begin{figure}
     \centering
     \begin{subfigure}[b]{0.49\textwidth}
         \centering
         \includegraphics[width=\textwidth]{./images/particle_physics/FirstNeutrinoEvent.png}
         \caption{}\label{fig:first_neutrino_event_annotated}
     \end{subfigure}
%     \hfill
     \begin{subfigure}[b]{0.49\textwidth}
         \centering
         \includegraphics[width=0.9\textwidth]{./images/particle_physics/Neutrino_bubble_chamber_decay_overlay.png}
         \caption{}\label{fig:first_neutrino_event_overlay}
     \end{subfigure}
        \caption{The first observed neutrino event in history, from the Argonne National Laboratory 12-foot bubble chamber, on Friday the 13th of November 1970. It is clearly seen on \vref{fig:first_neutrino_event_annotated} that some unseen particle collides with a proton, and two new particles are created in the process. \Vref{fig:first_neutrino_event_overlay} makes the process clearer using a Feynman diagram: the \PW boson interacts with a quark in the proton. The neutrino has enough energy to create an inelastic scattering, which results in an intermediate state (a Delta baryon), which quickly decays via the strong force into a proton and a pion.}
        \label{fig:first_neutrino_event}
\end{figure}

\section{Neutrino oscillations}\label{sec:neutrino_oscillations}

In the early days, the weak coupling strength of quarks was found to be smaller than that of lepton interactions.
Not only that, but different quark interactions have different coupling strengths.
Nicola Cabibbo developed a framework for explaining this discrepancy, wherein so-called weak eigenstates are distinct from the mass eigenstates that describe the quarks outside of weak interactions.
For example, both down- and strange quarks may decay into an up quark, giving a weak eigenstate
\begin{equation}
	\Pdown^{\prime} = V_{\Pup \Pdown} \Pdown + V_{\Pup \Pstrange} \Pstrange,
\end{equation}
with \( \lvert V_{\Pup \Pdown} \rvert^2 \) the probability of a down quark decaying, and \( \lvert V_{\Pup \Pstrange} \rvert^2 \) the probability of a strange quark decaying.

The weak eigenstates, a superposition of down and strange quarks, is then the entity actually coupling to the up quark, and extending this to include the decay of charm quarks yields the Cabibbo matrix
\begin{equation}
	\begin{bmatrix}
		\Pdown^{\prime} \\
		\Pstrange^{\prime}
	\end{bmatrix}
	=
	\begin{bmatrix}
		V_{\Pup \Pdown} & V_{\Pup \Pstrange} \\
		V_{\Pcharm \Pdown} & V_{\Pcharm \Pstrange}
	\end{bmatrix}
	\begin{bmatrix}
		\Pdown \\
		\Pstrange
	\end{bmatrix}
	=
	\begin{bmatrix}
		\cos{\theta_{\Pcharm}} & \sin{\theta_{\Pcharm}} \\
		-\sin{\theta_{\Pcharm}} & \cos{\theta_{\Pcharm}}
	\end{bmatrix}
	\begin{bmatrix}
		\Pdown \\
		\Pstrange
	\end{bmatrix},
\end{equation}
with \( \theta_{\Pcharm} \) the Cabibbo angle.

The concept was extended to neutrino oscillations.
This Nobel-winning idea was used to explain the apparent failure of the Homestake experiment, also known as the solar neutrino problem, described shortly in the following.

The Sun burns hydrogen via many processes, primarily in the pp cycle, wherein two protons fuse to create deuterium, a positron, an electron neutrino and excess energy:
\begin{equation}
	\ce{{\Pproton} + {\Pproton} -> \Pnu_{\Pe} + \SI{1.442}{\mega\electronvolt}}.
\end{equation}
Other interactions occur, some (such as the \( \beta \)-decay of boron-8) producing neutrinos of energies sufficient to be detected on Earth, but the Standard Solar Model predicts that the Sun only produces electron neutrinos, the flux predicted to be on the order of \( \SI{2e38}{\per\second} \)~\autocite{Thomson2013}.
Testing this model should be somewhat straightforward: fill an underground tank with appropriate liquid, wait for neutrino interactions and count how many atoms have changed.
The tank needs to be underground to shield from cosmic rays other than solar neutrinos, the liquid should be smartly chosen such that it decays into something easily measurable, and the tank needs to be big enough for interactions to happen on a reasonable timescale.
This describes the Homestake experiment: a tank was made out of a closed underground mine, and filled with perchloroethylene, containing plenty of chlorine.
The interaction with a neutrino then proceeds as
\begin{equation}
	\ce{\Pnu_{\Pe} + ^{37}Cl -> ^{37}Ar + \Pelectron}.
\end{equation}
The argon produced by the reaction is radioactive, and can thus be counted.

The experiment, however, saw a deficit of neutrinos.
This, of course, can have many causes: the calculations may be wrong, the experimental setup may be faulty or theories involved may be wrong.
Raymond Davis, the experiment-runner, stood his ground and vouched for the correctness of the experimental setup, and some 30-odd years later the SNO experiment found that the \textit{total} flux of neutrinos from the Sun (that is, the electron, muon and tau variants) was on the order of the predicted electron neutrino flux.
It would thus seem that something had to give: either the Standard Solar Model is wrong, or neutrinos change flavour during propagation from the Sun to the Earth.

Extending the Cabibbo hypothesis to account for this, we have
\begin{equation}\label{eq:pontecorvo}
	\begin{bmatrix}
		\nu_{\Pe} \\
		\nu_{\Pmu}
	\end{bmatrix}
	=
	\begin{bmatrix}
		U_{\Pe 1} & U_{\Pe 2} \\
		U_{\Pmu 1} & U_{\Pmu 2}
	\end{bmatrix}
	\begin{bmatrix}
		\nu_{1} \\
		\nu_{2}
	\end{bmatrix}
	=
	\begin{bmatrix}
		\cos{\theta} & \sin{\theta} \\
		-\sin{\theta} & \cos{\theta}
	\end{bmatrix}
	\begin{bmatrix}
		\nu_{1} \\
		\nu_{2}
	\end{bmatrix}.
\end{equation}
Now the weak eigenstates are described as superpositions of the mass eigenstates \( \nu_{1} \) and \( \nu_{2} \), and a muon neutrino produced at time \( t = 0 \) will have the wave function
\begin{equation}
	\ket{\psi(0)} = U_{\Pmu 1} \ket{\Pnu_{1}} + U_{\Pmu 2} \ket{\Pnu_2} =  \cos{\theta} \ket{\Pnu_{2}} - \sin{\theta} \ket{\Pnu_{1}}.
\end{equation}
Having travelled \( L \) at time \( T \) it will be described as
\begin{equation}\label{eq:neutrino_wavefunction_1}
%	\ket{\psi(L, T)} = \cos{\theta} \ket{\nu_{2}} e^{-i \phi_{2}} - \sin{\theta} \ket{\nu_{1}} e^{-i \phi_{1}},
\end{equation}
with the phases \( \phi_{i} = p_{i} \cdot x = E_{i} T - \textup{p}_{i} L \).
Expressed in terms of the weak eigenstates, \vref{eq:neutrino_wavefunction_1} can be written as
\begin{equation}\label{eq:neutrino_wavefunction_2}
	\ket{\psi(L, T)} = e^{-i \phi_{1}} \left[ \left( e^{i \Delta{\phi_{1 2}}} - 1 \right) \cos{\theta} \sin{\theta} \ket{\Pnu_{\Pe}} - \left( e^{i \Delta{\phi_{1 2}}} \cos^{2}{\theta} + \sin^{2}{\theta} \right) \ket{\Pnu_{\Pmu}} \right],
\end{equation}
where \( \Delta{\phi_{1 2}} = \phi_{1} - \phi_{2} \),
showing that any non-zero value of \( \Delta{\phi_{1 2}} \) will incur mixing of the weak eigenstates in the wave function of the propagating neutrino.
Simplifying \vref{eq:neutrino_wavefunction_2}, we can write it as
\begin{equation}
	\ket{\psi(L, T)} = c_{\Pe} \ket{\Pnu_{\Pe}} + c_{\Pmu} \ket{\Pnu_{\Pmu}},
\end{equation}
and it is seen that the probability of a muon neutrino turning into an electron neutrino becomes
\begin{equation}\label{eq:neutrino_osc_inner_product_1}
	P(\Pnu_{\Pmu} \rightarrow \Pnu_{\Pe}) = \lvert \braket{\Pnu_{\Pe} | \psi(L, T)} \rvert^2 = c_{\Pe} c_{\Pe}^{*}.
\end{equation}
Using \vref{eq:neutrino_wavefunction_2} this becomes
\begin{equation}
	P(\Pnu_{\Pmu} \rightarrow \Pnu_{\Pe}) = \sin^2{\left( 2 \theta \right)} \sin^2{\left( \frac{\Delta{\phi_{1 2}}}{2} \right)},
\end{equation}
which in the ultra-relativistic limit becomes
\begin{equation}
	P(\Pnu_{\Pmu} \rightarrow \Pnu_{\Pe}) = \sin^2{\left( 2 \theta \right)} \sin^2{\left[ \frac{\left( m_{1}^{2} - m_{2}^{2} \right) L}{4 E_{\Pnu}} \right]}.
\end{equation}
In plain words, then, the probability, or oscillation, depends on 1) the mixing angle, 2) the difference of the squared masses of the mass eigenstates, 3) the energy of the neutrino and 4) the length it travels.
An example of oscillation patterns can be seen in \vref{fig:ocsillations}.

\begin{figure}
     \centering
     \begin{subfigure}[b]{0.49\textwidth}
         \centering
		    \includegraphics[width=\textwidth]{./images/particle_physics/Oscillations_electron_long.svg.png}
		    \caption{Image from \url{https://da.m.wikipedia.org/wiki/Fil:Cherenkov_Wavefront.svg}}\label{fig:long_oscillations}
     \end{subfigure}
     \hfill
     \begin{subfigure}[b]{0.4\textwidth}
         \centering
		    \includegraphics[width=\textwidth]{./images/particle_physics/nutauappearonly.png}
		    \caption{Image from \url{https://www.nbi.ku.dk/english/research/experimental-particle-physics/icecube/neutrino-oscillation/}}\label{fig:icecube_oscillations}
     \end{subfigure}
        \caption{\Vref{fig:long_oscillations} shows the theoretical calculation of neutrino oscillations on the long baseline. Here, an electron neutrino is created (black line), and has some probability of turning into a muon neutrino (blue line) and tau neutrino (red line). \Vref{fig:icecube_oscillations} shows the muon neutrino to tau neutrino probability in IceCube, a function of both energy and pathlength (parameterised by the cosine of the zenith angle of the neutrino). Here one can see the effects of neutrinos interacting with matter, the MSW effect, changing the energy levels of the mass eigenstates during propagation on interactions with electrons in the dense core of the Earth. This changes the oscillation profile, seen for upgoing neutrinos at energies below \SI{10}{\giga\electronvolt}.}\label{fig:ocsillations}
\end{figure}

In many cases one uses the survival probability, which is
\begin{equation}\label{eq:survival_probability_1}
	P(\Pnu_{\Pmu} \rightarrow \Pnu_{\Pmu}) = c_{\Pe} c_{\Pe}^{*} = 1 - P(\Pnu_{\Pmu} \rightarrow \Pnu_{\Pe}).
\end{equation}

The two-flavour treatment is sufficient for many purposes, but nature has bestowed upon us three generations of matter (or, at least, three generations of light neutrinos).
To accommodate this fact, it is possible to extend \vref{eq:pontecorvo} to
\begin{equation}\label{eq:pmns}
	\begin{bmatrix}
		\nu_{\Pe} \\
		\nu_{\Pmu} \\
		\nu_{\Ptau}
	\end{bmatrix}
	=
	\begin{bmatrix}
		U_{\Pe 1} & U_{\Pe 2} & U_{\Pe 3} \\
		U_{\Pmu 1} & U_{\Pmu 2} & U_{\Pmu 3} \\
		U_{\Ptau 1} & U_{\Ptau 2} & U_{\Ptau 3}
	\end{bmatrix}
	\begin{bmatrix}
		\nu_{1} \\
		\nu_{2} \\
		\nu_{3}
	\end{bmatrix}.
\end{equation}
This unitary matrix can be parameterised as
\begin{equation}
	\begin{bmatrix}
		U_{\Pe 1} & U_{\Pe 2} & U_{\Pe 3} \\
		U_{\Pmu 1} & U_{\Pmu 2} & U_{\Pmu 3} \\
		U_{\Ptau 1} & U_{\Ptau 2} & U_{\Ptau 3}
	\end{bmatrix}
	=
	\begin{bmatrix}
		1 & 0 & 0 \\
		0 & c_{2 3} & s_{2 3} \\
		0 & -s_{2 3} & c_{2 3}
	\end{bmatrix}
	\begin{bmatrix}
		c_{1 3} & 0 & s_{1 3} e^{-i \delta} \\
		0 & 1 & 0 \\
		-s_{1 3} e^{i \delta} & 0 & c_{1 3}
	\end{bmatrix}
	\begin{bmatrix}
		c_{1 2} & s_{1 2} & 0 \\
		-s_{1 2} & c_{1 2} & 0 \\
		0 & 0 & 1
	\end{bmatrix},
\end{equation}
where \( c_{i j} = \cos{\theta_{i j}} \), \( s_{i j} = \sin{\theta_{i j}} \) and \( \delta \) is a complex phase owing to the unitarity condition of the matrix\footnote{In fact the matrix requires six complex phases to specify all nine independent parameters. However, five of these can be absorbed into the definitions of the lepton phases.}.
Moreover, \vref{eq:survival_probability_1} now becomes
\begin{equation}\label{eq:survival_probability_2}
	P(\Pnu_{\Pmu} \rightarrow \Pnu_{\Pmu}) = 1 - 4 \lvert U_{\Pmu 1} \rvert^2 \lvert U_{\Pmu 2} \rvert^2 \sin^{2}{\Delta_{2 1}} - 4 \lvert U_{\Pmu 1} \rvert^2 \lvert U_{\Pmu 3} \rvert^2 \sin^{2}{\Delta_{3 1}} - 4 \lvert U_{\Pmu 2} \rvert^2 \lvert U_{\Pmu 3} \rvert^2 \sin^{2}{\Delta_{3 2}},
\end{equation}
showing the possibility of experimentally determining the values of the neutrino mixing matrix, provided one can set up an experiment where enough neutrinos are captured, among other things.
In the case of IceCube, this survival probability can be approximated as\autocite{Aartsen2016}
\begin{equation}
P(\Pnu_{\Pmu} \rightarrow \Pnu_{\Pmu}) \approx 1 - \sin^{2}{\left( 2 \theta_{2 3} \right)} \sin^{2}{\left( 1.27 \frac{\Delta{m_{3 2}^{2}} L}{E} \right)}.
\end{equation}

Incidentally, the lepton mixing matrix also provides the possibility of CP violation by the weak force.
In this case, applying a CP transformation on a right-handed anti-neutrino results in a left-handed neutrino, allowed in the matrix element of weak interactions, and the asymmetry in this case is then
\begin{equation}
	P(\Pnu_{\Pe} \rightarrow \Pnu_{\Pmu}) - P(\APnu_{\Pe} \rightarrow \APnu_{\Pmu}) = 16 \Im{\left\{ U_{\Pe 1}^{*} U_{\Pmu 1} U_{\Pe 2} U_{\Pmu 2}^{*} \right\}} \sin{\Delta_{1 2}} \sin{\Delta_{1 3}} \sin{\Delta_{2 3}}.
\end{equation}
Therefore, if the complex phase \( \delta \) of the PMNS matrix is zero, there will be no CP violation in the weak sector.
CP violation is required for explaining the baryon asymmetry, and has been ruled out of both the strong- and the electromagnetic interactions, and so it is hoped that neutrino oscillations will provide the answer.

This of course also means that if we observe neutrino oscillations, it cannot be a massless particle.
However, the observable in oscillation probabilities is the difference of the squared masses, meaning it is the only thing we can experimentally say anything about.
As of this writing, the best fit values from \url{nu-fit.org} is on the order of
\begin{align}
	\Delta{m_{2 1}^{2}} & \approx \SI{7.4e-5}{\electronvolt\squared} \\
	\lvert \Delta{m_{3 2}^{2}} \lvert & \approx \SI{2.5e-3}{\electronvolt\squared}.
\end{align}
The absolute mass scale is constrained by the density of relic neutrinos from The Big Bang to be~\autocite{Thomson2013}
\begin{equation}
	\sum_{i = 1}^{3} m_{\Pnu_{i}} \lesssim \SI{1}{\electronvolt}.
\end{equation}
Because we only know the mass differences, it is not a priori possible to say in which order the mass hierarchy goes.
All we can say currently, is that the difference between \( \Pnu_{1} \) and \( \Pnu_{2} \) is much less than the difference between \( \Pnu_{3} \) and \( \Pnu_{2} \), but we do not know if \( \Pnu_{3} \) is heavier than the other two flavours.
The mass ordering problem can tested using IceCube~\autocite{Leuermann2018}.

This should be cause for some consternation, or---perhaps more---excitement.
First of all, the Standard Model does not automatically describe neutrino mass, and if it is introduced into the Lagrangian it requires that right-handed chiral neutrinos exist.
\begin{figure}[ht]
    \centering
    \includegraphics[width=0.7\textwidth]{./images/particle_physics/masses-600x230.jpg}
    \caption{Particle mass scales. The neutrinos are a million times lighter than the electron, the lightest charged lepton. The top quark also stands out in the hierarchy as a very heavy particle. Image from \url{http://hitoshi.berkeley.edu}}\label{fig:particle_masses}
\end{figure}
Even so, the masses of the neutrinos are on the order of \num{10e6} smaller than the electron (see \vref{fig:particle_masses}, and so something else than the Higgs mechanism could be the cause of their mass---this is exciting!
Suggestions, such as the seesaw mechanism, attempt to explain this, by introducing a heavy neutrino partner to each one of the lighter types.
Additionally, neutrinos and anti-neutrinos may be the same particle, a so-called Majorana particle.

All this to say that the study of neutrinos provide plenty of exciting avenues of physics beyond the Standard Model, and IceCube is in an excellent position to provide answers, and as such the quality of the reconstruction of the incident neutrinos, especially the energy and the zenith angle, is of the highest importance.

\section{Cherenkov radiation}\label{sec:cherenkov_radiation}

Light, in a vacuum, moves at the universally constant speed of \( c \)
When travelling through a medium, the refractive index \( n = c / v \) (where \( v \) is the phase velocity of light in the medium) determines the phase velocity of the photons due to constructive interference with the atoms of the material.
In a dielectric medium, a charged particle may travel faster than light, and will in its path polarise the molecules of the material.
After the passage, the material relaxes into an unpolarised state, sending out photons in the process, and because these travel at the refracted speed of light they conspire to create light at an angle with the particle where constructive interference occurs; see \vref{fig:cherenkov_radiation}.
Here, a particle travelling at \( v_{\text{p}} \approx c \) emits Cherenkov radiation at an angle \( \sin{\alpha} = r / s = v / v_{\text{p}} \), and so the angle depends on the index of refraction and the speed of the charged particle.

Of course the angle produces a cone in three dimensional space.
This is used in particle detectors such as Super-Kamiokande where a tank of liquid is surrounded by an array of photomultiplier tubes.
These will pick up Cherenkov cones as seen in \vref{fig:cherenkov_cone}, and from the shapes particle properties can be inferred which in turn may shed light on properties of the neutrinos that originated the interaction; not unlike IceCube, where the Cherenkov light is what is actually observed by the photomultiplier tubes.

\begin{figure}
     \centering
     \begin{subfigure}[b]{0.49\textwidth}
         \centering
		    \includegraphics[width=\textwidth]{./images/particle_physics/1061px-Cherenkov_Wavefront.svg.png}
		    \caption{Cherenkov radiation. Image from \url{https://da.m.wikipedia.org/wiki/Fil:Cherenkov_Wavefront.svg}}\label{fig:cherenkov_radiation}
     \end{subfigure}
%     \hfill
     \begin{subfigure}[b]{0.49\textwidth}
         \centering
		    \includegraphics[width=\textwidth]{./images/particle_physics/cherenkov.jpg}
		    \caption{Cherenkov cone. Image from \url{http://physicsopenlab.org/2016/04/24/diy-cherenkov-detector/}}\label{fig:cherenkov_cone}
     \end{subfigure}
        \caption{Feynman diagrams of neutral- and charged current processes}
        \label{fig:cherenkov_figure}
\end{figure}

\todo[inline]{Remember to mention the different flavours of quarks at some point...}
\todo[inline]{Must talk about spin}
\todo[inline]{Should expound on Feynman diagrams}
\todo[inline]{Must talk about sterile neutrinos in IceCube}
\todo[inline]{Update figure captions}
\todo[inline]{Mention long/short baselines + oscillation wavelength}

\end{document}
