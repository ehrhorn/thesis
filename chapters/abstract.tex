%!TEX root = ../main.tex
\documentclass[../main.tex]{subfiles}

\begin{document}

\begin{abstract}
	The IceCube Neutrino Observatory, on the Geographic South Pole, finished construction almost exactly 10 years ago, and due to the project's success is due to receive an upgrade in 2022-23.
	This is expected to make it a world-leader in measurements of certain neutrino properties, dependent on the energy \( E \) and path length traveled \( L \) of the detected neutrinos.
	These properties need to be inferred from the observation, and subsequent reconstruction, of Cherenkov light from secondary particles, currently done using classical methods.

	However, the advancement of machine learning (ML) in general---and neural networks in particular---open up the possibility of creating faster, more robust and scalable inference capabilities, as has been seen in areas such as image recognition and machine translation during the past decade.

	This thesis develops a convolutional based neural network and accompanying tooling for low-energy neutrino reconstruction.
	A novel approach to data management in IceCube specifically for ML purposes is introduced, together with a new event viewer, and algorithm comparison dashboard, to aid in the training and comparison of different network performances, generalizable for use by other groups studying ML applications in IceCube.

	A temporal convolutional network is able to outperform the current best classical algorithm Retro Reco at low energies up to around \SI{50}{\giga\electronvolt} for both polar angle (a proxy for \( L \)) and \( E \) reconstruction.
\end{abstract}

\end{document}